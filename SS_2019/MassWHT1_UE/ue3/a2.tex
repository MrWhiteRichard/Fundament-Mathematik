\begin{lemma}
    Sei $\mathfrak{A}$ eine Sigmaalgebra und $\mu$ ein endlicher Inhalt auf $\mathfrak{A}$ sowie $C\subset\Omega$, wobei $C\notin\mathfrak{A}$ sei. Dann kann $\mu$ zu einem Inhalt auf $\mathfrak{A}_\sigma(\mathfrak{A}\cup\{C\})$ fortgesetzt werden.
\end{lemma}

\begin{proof}[Beweis.]
    Aus der ersten Übung wissen wir bereits aus Aufgabe 5, dass $\mathfrak{A}_\sigma(\mathfrak{A}\cup\{C\})=\{(A\cap C)\cup(B\cap C^C)\mid A,B\in\mathfrak{A}\}$. Wählen wir also eine beliebige Menge $A\in\mathfrak{A}_\sigma(\mathfrak{A}\cup\{C\})$, so gibt es $A_1,A_2\in\mathfrak{A}:A=\left(A_1\cap C\right)\cup\left(A_2\cap C^C\right)$. Das bedeutet, falls es einen Inhalt $\tilde{\mu}$ als Fortsetzung von $\mu$ auf $\mathfrak{A}_\sigma(\mathfrak{A}\cup\{C\})$ geben sollte, so erfüllt dieser sicher $\tilde{\mu}(A)=\tilde{\mu}((A_1\cap C)\cup(A_2\cap C^C))=\tilde{\mu}(A_1\cap C)+\tilde{\mu}(A_2\cap C^C)$, da $(A_1\cap C)\cap(A_2\cap C^C)=\emptyset$. Nun reicht es uns für ein beliebiges $U\in\mathfrak{A}:\tilde{\mu}(U\cap C):=\sup\{\mu(B)\mid B\subset A\cap C\land B\in\mathfrak{A}\}$ zu definieren, weil dann $\tilde{\mu}(A_1\cap C)$ bereits direkt definiert ist und $\tilde{\mu}(A_2\cap C^C)=\mu(A_2)-\tilde{\mu}(A_2\cap C)$ sein muss.
    
    Nun weist man  noch nach, dass $\tilde{\mu}$ auf $\mathfrak{A}_\sigma(\mathfrak{A}\cup\{C\})$ tatsächlich ein Inhalt ist und der Beweis ist fertig.
\end{proof}