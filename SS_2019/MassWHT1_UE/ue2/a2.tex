\begin{lemma}
    Jeder endliche Ring $\mathfrak{R}$ wird von einem System von endlich vielen disjunkten Mengen erzeugt.
\end{lemma}
\begin{proof}[Beweis]
    Zuerst wählen wir $x,y\in\mathfrak{R}$ beliebig und definieren
    \begin{align*}
        A_x:=\bigcap_{A\in\mathfrak{R}:x\in A}A
    \end{align*}
    und $\mathfrak{C}:=\{A_x\mid x\in\Omega\}$. Es gilt $\mathfrak{C}\subset\mathfrak{R}$, weil $\mathfrak{R}$ ein endlicher Ring ist und daher beliebige Durchschnitte von Mengen aus $\mathfrak{R}$ endlich sind, also nach \cite[Satz 2.1]{GrillSkript} auch die Durchschnitte wieder in $\mathfrak{R}$ liegen. Nun unterscheiden wir zwei Fälle.

    Der erste Fall ist $A_x\cap A_y=\emptyset$.

    Im zweiten Fall ist $A_x\cap A_y\neq\emptyset$, das bedeutet $\exists z\in A_x\cap A_y$. Betrachtet man ein beliebiges $u\in A_x$, und nimmt an, $u$ wäre nicht in $A_z$, so $\exists B\in \mathfrak{R}:z\in B \land u\notin B$. Es muss allerdings auch $x\in B$ gelten, weil sonst $z\notin A_x$ gelten würde. Also haben wir mit $B$ eine Menge konstruiert, welche $x$ enthält, aber $u$ nicht, woraus wir den Widerspruch $u\notin A_x$ erhalten. Wir haben also $u\in A_x\Rightarrow u\in A_z$ bewiesen. Nun wollen wir noch die Rückrichtung beweisen und wählen dafür ein beliebiges $v\in A_z$. Unter der Annahme, $v$ wäre nicht in $A_x$ gäbe es also eine Menge $C\in \mathfrak{R}$ mit $x\in C\land v\notin C$. Wegen $z\in A_x$ gilt allerdings auch $z\in C$, womit wir auf den Widerspruch $v\notin A_z$ schließen können. Insgesamt haben wir also $u\in A_x\Leftrightarrow u\in A_z$ oder äquivalent dazu $A_x=A_z$ gezeigt. Da $y$ schließlich auch nur ein beliebiges Element aus $\Omega$ ist gilt auch $A_y=A_z$ und damit $A_x=A_y$. 

    Nun wissen wir also, dass alle Mengen aus $\mathfrak{C}$ disjunkt. Jetzt wollen wir noch $\mathfrak{R}(\mathfrak{C})=\mathfrak{R}$ beweisen.

    Offensichtlich gilt $\mathfrak{C}\subset \mathfrak{R}$ und da $\mathfrak{R}$ ein Ring ist natürlich auch $\mathfrak{R}(\mathfrak{C})\subset\mathfrak{R}$. 

    Wählen wir nun ein beliebiges $A\in \mathfrak{R}$, so lässt sich $A$ darstellen als 
    \begin{align*}
        \bigcup_{x\in A}A_x,
    \end{align*}
    weil für $x\in A:A_x\subset A$. Es handelt sich nur um endliche Vereinigungen, da $\mathfrak{C}$ schließlich nur ein endliches Mengensystem ist. Da endliche Vereinigungen von Mengen eines Ringes wieder im Ring liegen, muss $A\in\mathfrak{R}(\mathfrak{C})$ und damit $\mathfrak{R}(\mathfrak{C})\supset\mathfrak{R}$.

\end{proof}

