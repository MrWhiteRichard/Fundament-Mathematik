\documentclass{article}

\usepackage[utf8]{inputenc}
\usepackage{fullpage}
\usepackage{amsmath, amssymb, amsfonts, amsthm}
\usepackage{mathtools}
\usepackage{twoopt}
\usepackage{graphicx, subfig, float}
\usepackage{listings, xcolor}
\usepackage{placeins}
\usepackage{babel, ngerman}
\usepackage{verbatim}
\usepackage{booktabs}
\usepackage{hyperref}

\usepackage{multicol}
\setlength{\columnsep}{0cm}

\def \lastexercisenumber {0}

% special letters:

\newcommand{\N}{\mathbb{N}}
\newcommand{\Z}{\mathbb{Z}}
\newcommand{\Q}{\mathbb{Q}}
\newcommand{\R}{\mathbb{R}}
\newcommand{\C}{\mathbb{C}}
\newcommand{\K}{\mathbb{K}}
\newcommand{\T}{\mathbb{T}}
\newcommand{\E}{\mathbb{E}}
\newcommand{\V}{\mathbb{V}}
\renewcommand{\P}{\mathbb{P}}
\newcommand{\1}{\mathbbm{1}}

\newcommand  {\B}{\mathfrak{B}}
\renewcommand{\S}{\mathfrak{S}}

% quantors:

\newcommand{\Forall}{\forall \,}
\newcommand{\Exists}{\exists \,}
\newcommand{\ExistsOnlyOne}{\exists! \,}
\newcommand{\nExists}{\nexists \,}

% MISC symbols:

\newcommand{\landau}[1]
{
  {\scriptstyle \mathcal{O}}
  \pbraces{#1}
}

\newcommand{\Landau}[1]
{
  \mathcal{O}
  \pbraces{#1}
}

\newcommand{\eps}{\mathrm{eps}}

% graphics in a box:

\newcommandtwoopt
{\includegraphicsboxed}[3][][]
{
  \begin{figure}[!h]
    \begin{boxedin}
      \ifthenelse{\isempty{#2}}
      {
        \begin{center}
          \includegraphics[width = 0.75 \textwidth]{#3}
          \label{fig:#1}
        \end{center}
      }{
        \begin{center}
          \includegraphics[width = 0.75 \textwidth]{#3}
          \caption{#2}
          \label{fig:#1}
        \end{center}
      }
    \end{boxedin}
  \end{figure}
}

% braces:

\newcommand{\pbraces}[1]{{\left  ( #1 \right  )}}
\newcommand{\bbraces}[1]{{\left  [ #1 \right  ]}}
\newcommand{\Bbraces}[1]{{\left \{ #1 \right \}}}
\newcommand{\vbraces}[1]{{\left  | #1 \right  |}}
\newcommand{\Vbraces}[1]{{\left \| #1 \right \|}}
\newcommand{\abraces}[1]{{\left \langle #1 \right \rangle}}
\newcommand{\round}[1]{\bbraces{#1}}

\newcommand
{\floor}[1]
{{\left \lfloor #1 \right \rfloor}}

\newcommand
{\ceil} [1]
{{\left \lceil  #1 \right \rceil }}

% special functions:

\newcommand{\norm}  [2][]{\Vbraces{#2}_{#1}}
\newcommand{\diag}  [1]{\mathrm{diag} \: #1}
\newcommand{\dist}  [1]{\mathrm{dist} \: #1}
\newcommand{\mean}  [1]{\mathrm{mean} \: #1}
\newcommand{\erf}   [1]{\mathrm{erf} \: #1}
\newcommand{\id}    [1]{\mathrm{id} \: #1}
\newcommand{\sgn}   [1]{\mathrm{sgn} \: #1}
\newcommand{\supp}  [1]{\mathrm{supp} \: #1}
\newcommand{\arsinh}[1]{\mathrm{arsinh} \: #1}
\newcommand{\arcosh}[1]{\mathrm{arcosh} \: #1}
\newcommand{\artanh}[1]{\mathrm{artanh} \: #1}
\newcommand{\card}  [1]{\mathrm{card} \: #1}
\newcommand{\Span}  [1]{\mathrm{span} \: #1}
\newcommand{\Aut}   [1]{\mathrm{Aut} \: #1}
\newcommand{\End}   [1]{\mathrm{End} \: #1}
\newcommand{\ggT}   [1]{\mathrm{ggT} \: #1}
\newcommand{\kgV}   [1]{\mathrm{kgV} \: #1}
\newcommand{\ord}   [1]{\mathrm{ord} \: #1}
\newcommand{\grad}  [1]{\mathrm{grad} \: #1}
\newcommand{\ran}   [1]{\mathrm{ran} \: #1}
\newcommand{\graph} [1]{\mathrm{graph} \: #1}
\newcommand{\Inv}   [1]{\mathrm{Inv} \: #1}
\newcommand{\pv}    [1]{\mathrm{pv} \: #1}
\newcommand{\Mod}{\: \mathrm{mod} \:}
\newcommand{\Char}{\mathrm{char}}
\newcommand{\At}{\mathrm{At}}
\newcommand{\Ob}{\mathrm{Ob}}
\newcommand{\Hom}{\mathrm{Hom}}
\newcommand{\orthogonal}[3][]{#2 ~\bot_{#1}~ #3}
\newcommand{\Rang}{\mathrm{Rang}}

\newcommand
{\GL}[2][]
{\mathrm{GL}_{#1} \pbraces{#2}}

% fractions:

\newcommand{\Frac}[2]{\frac{1}{#1} \pbraces{#2}}
\newcommand{\nfrac}[2]{\nicefrac{#1}{#2}}

% derivatives & integrals:

\newcommandtwoopt
{\Int}[4][][]
{\int_{#1}^{#2} #3 ~\mathrm{d} #4}

\newcommandtwoopt
{\derivative}[3][][]
{
  \frac
  {\mathrm{d}^{#1} #2}
  {\mathrm{d} #3^{#1}}
}

\newcommandtwoopt
{\pderivative}[3][][]
{
  \frac
  {\partial^{#1} #2}
  {\partial #3^{#1}}
}

\newcommand
{\primeprime}
{{\prime \prime}}

\newcommand
{\primeprimeprime}
{{\prime \prime \prime}}

% Text:

\newcommand{\Quote}[1]{\glqq #1\grqq{}}
\newcommand{\Text}[1]{{\text{#1}}}
\newcommand{\fastueberall}{\text{f.ü.}}
\newcommand{\fastsicher}{\text{f.s.}}

% -------------------------------- %
% amsthm-stuff:

\theoremstyle{definition}

% numbered theorems
\newtheorem{theorem}    {Satz}   [section]
\newtheorem{lemma}      [theorem]{Lemma}
\newtheorem{corollary}  [theorem]{Korollar}
\newtheorem{proposition}[theorem]{Proposition}
\newtheorem{remark}     [theorem]{Bemerkung}
\newtheorem{definition} [theorem]{Definition}
\newtheorem{example}    [theorem]{Beispiel}

% unnumbered theorems
\newtheorem*{theorem*}    {Satz}
\newtheorem*{lemma*}      {Lemma}
\newtheorem*{corollary*}  {Korollar}
\newtheorem*{proposition*}{Proposition}
\newtheorem*{remark*}     {Bemerkung}
\newtheorem*{definition*} {Definition}
\newtheorem*{example*}    {Beispiel}

% Please define this stuff in project ("main.tex"):

% \def \lastexercisenumber {...}
% This will be 0 by default

% \setcounter{section}{...}
% This will be 0 by default
% and hence, completely ignored

\ifnum \thesection = 0
{
  \newtheorem{exercise}{Aufgabe}
}
\else
{
  \newtheorem{exercise}{Aufgabe}[section]
}
\fi

\ifdef
{\lastexercisenumber}
{\setcounter{exercise}{\lastexercisenumber}}

\newenvironment{solution}
{
  \begin{proof}[Lösung]
}{
  \end{proof}
}

\renewcommand{\proofname}{Beweis}

% -------------------------------- %
% environment zum einkasteln:

% dickere vertical lines
\newcolumntype
{x}
[1]
{
  !{
    \centering
    \arraybackslash
    \vrule
    width #1}
}

% environment selbst (the big cheese)
\newenvironment
{boxedin}
{
  \begin{tabular}
  {
    x{1 pt}
    p{\textwidth}
    x{1 pt}
  }
  \Xhline
  {2 \arrayrulewidth}
}
{
  \\
  \Xhline{2 \arrayrulewidth}
  \end{tabular}
}

% -------------------------------- %
% MISC "Ein-Deutschungen"

\renewcommand{\figurename}{Abbildung}
\renewcommand{\tablename} {Tabelle}

% -------------------------------- %


\parindent 0pt

\title
{
  Maß- und Wahrscheinlichkeitstheorie 1 \\
  Prüfungsbeispiele
}
\author
{
  Richard Weiss
}
\date{}

\begin{document}

\maketitle

In den Lösungen wird öfters auf bereits gelöste Aufgaben verwiesen. Es sei angemerkt, dass an der Stelle Bezug auf die Nummerierung des Originaldokuments genommen wird. Das liegt daran, dass eventuell nicht für alle Aufgaben Lösungen ge-\LaTeX-t wurden.

\setcounter{exercise}{0}

\section{Menensysteme und Maßfunktionen}

--------------------------------------------------------------------------------

\begin{exercise}

Hier könnte Ihre Werbung stehen!

\begin{itemize}
  \item[(a)] Definieren Sie Maßfunktion, endliche Maßfunktion, sigmaendliche Maßfunktion, äußere Maßfunktion.
  \item[(b)] Formulieren und beweisen Sie den Fortsetzungssatz für Maßfunktionen.
  \item[(c)] $\Omega$ sei eine beliebige Menge für $A \subseteq \Omega$
  \begin{align*}
    \mu^\ast(A) =
    \begin{cases}
      0, & \text{wenn} \enspace A = \emptyset, \\
      1, & \text{sonst}.
    \end{cases}
  \end{align*}
  Zeigen Sie, dass $\mu^\ast$ eine äußere Maßfunktion ist und bestimmen Sie das System der $\mu^\ast$-messbaren Mengen.
\end{itemize}

\end{exercise}

\begin{solution}

Sei $\mu: \mathfrak{C} \to \R$ eine Mengenfunktion auf dem Mengensystem $\emptyset \neq \mathfrak{C} \subseteq 2^\Omega$.

\begin{itemize}

  \item $\mu \Text{Maßfunktion} : \Leftrightarrow$
  \begin{itemize}
    \item $\Forall A \in \mathfrak{C}: \mu(A) \geq 0,$
    \item $\mu \enspace \sigma \text{-additiv}, \enspace
    \text{d.h.}
    \Forall (A_n) \in \mathfrak{C}, \text{disj.}, \text{höchst. abz.}:
    A := \sum_{n \in \N} A_n \in \mathfrak{C} \Rightarrow
    \mu(A) = \sum_{n \in \N} \mu(A_n)$
  \end{itemize}

  \item $\mu \Text{endliche Maßfunktion} : \Leftrightarrow$
  \begin{itemize}
    \item $\mu \enspace \text{Maßfunktion},$
    \item $\Forall A \in \mathfrak{C}: \mu(A) < \infty$
  \end{itemize}

  \item $\mu \Text{sigmaendliche Maßfunktion} : \Leftrightarrow$
  \begin{itemize}
    \item $\mu \enspace \text{Maßfunktion},$
    \item $\Forall A \in \mathfrak{C}: \Exists (A_n) \in \mathfrak{C}: A \subseteq \bigcup_{n \in \N} A_n, \Forall n \in \N: \mu(A_n) < \infty$
  \end{itemize}

\end{itemize}

Sei $\mu^\ast: 2^\Omega \to \R$ eine weitere Mengenfunktion.

\begin{itemize}

  \item $\mu^\ast \Text{äußere Maßfunktion} : \Leftrightarrow$
  \begin{itemize}
    \item $\mu^\ast(\emptyset) = 0,$
    \item $\Forall A \in 2^\Omega: \mu^\ast(A) \geq 0,$
    \item $\Forall A, B \in 2^\Omega:
    A \subseteq B \Rightarrow
    \mu^\ast(A) \leq \mu^\ast(B),$
    \item $\Forall A, (B_n) \in 2^\Omega:
    A \subseteq \bigcup_{n \in \N} B_n \Rightarrow
    \mu^\ast(A) \leq \sum_{n \in \N} \mu^\ast(B_n)$
  \end{itemize}

\end{itemize}

Der Beweis des Fortsetzungssatz für Maßfunktionen ist nicht teil des Prüfungsstoffs. Dennoch, er besagt folgendes.

\begin{theorem*}[Fortsetzungssatz für Maßfunktionen]

$\mu$ sei ein Maß auf einem Semiring $\mathfrak{T}$. Dann kann man auf dem von $\mathfrak{T}$ erzeugten Sigmaring ein Maß finden, dass auf $\mathfrak{T}$ mit $\mu$ übereinstimmt. \\
Wenn $\mu$ auf $\mathfrak{T}$ sigmaendlich ist, dann ist die Fortsetzung auf den erzeugten Sigmaring eindeutig bestimmt.

\end{theorem*}

Die ersten drei Eigenschaften sind offensichtlich. Seien zuletzt $A, (B_n) \in 2^\Omega$ mit $A \subseteq \bigcup_{n \in \N} B_n$. Falls $A = \emptyset$, sind wir (wegen der zweiten Eigenschaft) fertig und sonst $\Exists n \in \N: B_n \neq \emptyset$. \\

Das System der Carathéodory-messbaren Mengen lautet wie folgt.

\begin{align*}
  \mathfrak{M}_{\mu^\ast} :=
  \Bbraces
  {
    A \subseteq \Omega:
    \Forall B \subseteq \Omega:
    \mu^\ast(B) = \mu^\ast(B \cap A) + \mu^\ast(B \cap A^\complement)
  }
\end{align*}

Weil dieses System eine $\sigma$-Algebra ist, sind $\emptyset, \Omega$ messbar. $A \subseteq \Omega$ mit $A \neq \emptyset, \Omega$, ist wegen $B := \Omega$ nicht messbar, weil $A^\complement \neq \emptyset$ und damit $\mathfrak{M}_{\mu^\ast} = \Bbraces{\emptyset, \Omega}$.

\end{solution}

--------------------------------------------------------------------------------

\begin{exercise}

Gegeben sei die folgende Funktion auf $2^\R$:

\begin{align*}
  \mu^\ast(A) =
  \begin{cases}
    0 & \text{für} \enspace A = \emptyset, \\
    1 & \text{wenn} \enspace 1 \leq |A| < \infty, \\
    2 & \text{sonst}.
  \end{cases}
\end{align*}

Zeigen Sie, dass $\mu^\ast$ eine äußere Maßfunktion ist und bestimmen Sie das System der $\mu^\ast$-messbaren Mengen.

\end{exercise}

\begin{solution}

Die ersten drei Eigenschaften sind offensichtlich. Seien zuletzt $A, (B_n) \in 2^\Omega$ mit $A \subseteq \bigcup_{n \in \N} B_n$. Falls $A = \emptyset$, sind wir (wegen der zweiten Eigenschaft) fertig. Für $1 \leq |A| < \infty$, muss $\Exists n \in \N: 1 \leq |B_n| < \infty$. Ansonsten, sind die (unendlich vielen) $\omega \in A$ über ganz $(B_n)$ verteilt. Nun gilt aber $\Exists n \in \N: |B_n| = \infty$ oder $\Exists n \neq m \in \N: 1 \leq |B_n|, |B_m| < \infty$. \\

Das System der Carathéodory-messbaren Mengen lautet wie folgt.

\begin{align*}
  \mathfrak{M}_{\mu^\ast} :=
  \Bbraces
  {
    A \subseteq \Omega:
    \Forall B \subseteq \Omega:
    \mu^\ast(B) = \mu^\ast(B \cap A) + \mu^\ast(B \cap A^\complement)
  }
\end{align*}

Weil dieses System eine $\sigma$-Algebra ist, sind $\emptyset, \Omega$ messbar. Sei also $A \subseteq \Omega$ mit $A \neq \emptyset, \Omega$.

\begin{itemize}

  \item[Fall $1$)] $A \enspace \text{endl.} \Rightarrow A^\complement \Text{unendl.}$
  \begin{align*}
    2 = \mu^\ast(\R) =
    \underbrace{\mu^\ast(\R \cap A)}_1 +
    \underbrace{\mu^\ast(\R \cap A^\complement)}_2 = 3
  \end{align*}

  \item[Fall $2$)] $A \enspace \text{unendl.}$
  \begin{itemize}
    \item[Fall a)] $A^\complement \enspace \text{endl.} \Rightarrow$ analog zu Fall $1$
    \item[Fall b)] $A^\complement \enspace \text{unendl.}$
    \begin{align*}
      2 = \mu^\ast(\R) =
      \underbrace{\mu^\ast(\R \cap A)}_2 +
      \underbrace{\mu^\ast(\R \cap A^\complement)}_2 = 4
    \end{align*}
  \end{itemize}

\end{itemize}

Damit, muss $\mathfrak{M}_{\mu^\ast} = \Bbraces{\emptyset, \Omega}$.

\end{solution}

--------------------------------------------------------------------------------

\begin{exercise}

Hier könnte Ihre Werbung stehen!

\begin{itemize}
  \item[(a)] Definieren Sie Ring, Semiring, monotones System, Dynkin-System, Algebra, Sigmaalgebra.
  \item[(b)] Zeigen Sie: Wenn $\mathfrak{R}$ ein Ring ist, dann stimmt das von $\mathfrak{R}$ erzeugte monotone System mit dem erzeugten Sigmaring überein.
\end{itemize}

\end{exercise}

\begin{solution}

Hier könnte Ihre Werbung stehen!

\begin{itemize}

  \item $\emptyset \neq \mathfrak{R} \subseteq 2^\Omega \enspace \text{Ring} : \Leftrightarrow \Forall A, B \in \mathfrak{R}:$
  \begin{itemize}
    \item $A \cup B \in \mathfrak{R}$
    \item $A \setminus B \in \mathfrak{R}$
  \end{itemize}

  \item $\emptyset \neq \mathfrak{T} \subseteq 2^\Omega \enspace \text{Semiring} : \Leftrightarrow \Forall A, B \in \mathfrak{T}:$
  \begin{itemize}
    \item $A \cap B \in \mathfrak{T},$
    \item $A \subseteq B \Rightarrow \Exists C_1, \ldots, C_n \in \mathfrak{T}, \Text{disj.}:
    B \setminus A = \sum_{i=1}^n C_i,$
    \item $\Forall k = 1, \ldots, n:
    A \cup \sum_{i=1}^k C_i \in \mathfrak{T}$
  \end{itemize}

  \item $\mathfrak{M} \subseteq 2^\Omega \enspace \text{monotones System} : \Leftrightarrow \Forall (A_n) \in \mathfrak{M}, \Text{mon.}: \lim_{n \to \infty} A_n \in \mathfrak{M}$

  \item $\emptyset \neq \mathfrak{D} \subseteq 2^\Omega \enspace \text{Dynkin-System} : \Leftrightarrow$
  \begin{itemize}
    \item $\Forall A, B \in \mathfrak{D}:
    A \subseteq B \Rightarrow B \setminus A \in \mathfrak{D}$
    \item $\Forall (A_n) \in \mathfrak{D}, \text{disj.}: \sum_{n \in \N} A_n \in \mathfrak{D}$
    \item $\Omega \in \mathfrak{D}$
  \end{itemize}

  \item $\emptyset \neq \mathfrak{A} \subseteq 2^\Omega \enspace \text{Algebra} : \Leftrightarrow$
  \begin{itemize}
    \item $\mathfrak{A} \enspace \text{Ring},$
    \item $\Omega \in \mathfrak{A}$
  \end{itemize}

  \item $\emptyset \neq \mathfrak{A}_\sigma \subseteq 2^\Omega \enspace \text{Sigmaalgebra} : \Leftrightarrow$
  \begin{itemize}
    \item $\Forall (A_n) \in \mathfrak{A}_\sigma, \text{disj.}: \sum_{n \in \N} A_n \in \mathfrak{A}_\sigma$
    \item $\Forall A, B \in \mathfrak{A}_\sigma: A \setminus B \in \mathfrak{A}_\sigma,$
    \item $\Omega \in \mathfrak{A}_\sigma$
  \end{itemize}

\end{itemize}

Der nächste Teil ist genau das \Quote{Monotone Class Theorem}! Siehe Skript.

\end{solution}

--------------------------------------------------------------------------------

\begin{exercise}

Hier könnte Ihre Werbung stehen!

\begin{itemize}
  \item[(a)] Definieren sie Maßfunktion, endliche Maßfunktion, sigmaendliche Maßfunktion, äußere Maßfunktion, von einem Maß erzeugte äußere Maßfunktion.
  \item[(b)] Zeigen Sie, dass eine äußere Maßfunktion auf dem System der messbaren Mengen ein Maß bildet.
\end{itemize}

\end{exercise}

\begin{solution}

(a) Sei $\inf \emptyset := \infty$. Dann ist die äußere Maßfunktion, für ein Maß $\mu$ auf einem Ring $\mathfrak{R}$, definiert als

\begin{align*}
  \mu^\ast:
  2^\Omega \to \bar \R,
  A \mapsto \inf \Bbraces{\sum_{n \in \N} \mu(E_n): (E_n) \in \mathfrak{R}, A \subseteq \bigcup_{n \in \N} E_n}.
\end{align*}

Für den Rest: siehe Aufgabe 1 (a). \\

(b) Siehe Skript.

\end{solution}

--------------------------------------------------------------------------------

\begin{exercise}

Hier könnte Ihre Werbung stehen!

\begin{itemize}
  \item[(a)] $\mathfrak{S}$ sei eine Sigmaalgebra über $\Omega$, $\mathfrak{C} \subset \Omega$. Zeigen Sie
  \begin{align*}
    \mathfrak{A}_\sigma(\mathfrak{S} \cup \Bbraces{C}) =
    \Bbraces{(A \cap C) \cup (B \cap C^\complement), A, B \in \mathfrak{S}}.
  \end{align*}
  \item[(b)] $\mathfrak{R}$ sei ein Ring über $\Omega$. Zeigen Sie
  \begin{align*}
    \mathfrak{A}(\mathfrak{R}) =
    \mathfrak{R} \cup \Bbraces{A: A^\complement \in \mathfrak{R}}.
  \end{align*}
\end{itemize}

\end{exercise}

\begin{solution}

(a) \Quote{$\supseteq$}: Seien $A, B \in \mathfrak{S}$, dann auch $A, B, C \in \mathfrak{A}_\sigma(\mathfrak{S} \cup \Bbraces{C})$. Als $\sigma$-Algebra ist diese bezüglich \Quote{$\cap$}, \Quote{$\cup$}, \Quote{$^\complement$} abgeschlossen. \\

\Quote{$\subseteq$}: Die linke Seite ist die kleinste $\sigma$-Algebra, die $\mathfrak{S} \cup \Bbraces{C}$ enthält. Sei $\mathfrak{S}^\prime$ die rechte Seite, dann muss

\begin{itemize}
  \item $\mathfrak{S} \subseteq \mathfrak{S}^\prime$, wenn man $A = B$ wählt und
  \item $C \in \mathfrak{S}^\prime$, wenn $A := \Omega$, $B := \emptyset$.
\end{itemize}

Wir weisen also noch nach, dass $\mathfrak{S}^\prime$ eine $\sigma$-Algebra ist.

\begin{itemize}

  \item \Quote{Enthält Grundmenge}: $\Omega \in \mathfrak{S} \subseteq \mathfrak{S}^\prime$

  \item \Quote{Stabil bzgl. abzählbaren Vereinigungen}: Sei $(E_n) \in \mathfrak{S}^\prime$, dann $\Exists (A_n), (B_n) \in \mathfrak{S}:$
  \begin{align*}
    \bigcup_{n \in \N} \pbraces{A_n \cap C} \cup \pbraces{B_n \cap C^\complement} =
    \Bigg ( \underbrace{\bigcup_{n \in \N} A_n}_{\in \mathfrak{S}} \cap C \Bigg ) \cup
    \Bigg ( \underbrace{\bigcup_{n \in \N} B_n}_{\in \mathfrak{S}} \cap C^\complement \Bigg ) \cup
    \in \mathfrak{S}^\prime
  \end{align*}

  \item \Quote{Stabil bzgl. Differenzen}: Diese filetieren wir in
  \begin{itemize}

    \item \Quote{Stabil bzgl. Komplement}: Sei $E \in \mathfrak{S}^\prime$, dann $\Exists A, B \in \mathfrak{S}:$
    \begin{align*}
      E^\complement
      = \pbraces{\pbraces{A \cap C} \cup \pbraces{B \cap C^\complement}}^\complement
      & = \pbraces{A^\complement \cup C^\complement} \cap \pbraces{B^\complement \cup C} \\
      & = \underbrace
          {
            \pbraces{A^\complement \cap B^\complement}
          }_{
            \in \mathfrak{S}
          } \cup
          \underbrace
          {
            \pbraces{A^\complement \cap C} \cup \pbraces{C^\complement \cap B^\complement}
          }_{
            \in \mathfrak{S}^\prime
          } \cup
          \underbrace
          {
            \pbraces{C^\complement \cap C}
          }_\emptyset
          \in \mathfrak{S}^\prime
    \end{align*}

    \item \Quote{Stabil bzgl. (endlichen) Durchschnitten}: Seien $E_1, E_2 \in \mathfrak{S}^\prime$, dann $\Exists A_1, B_1, A_2, B_2 \in \mathfrak{S}:$
    \begin{align*}
      E_1 \cap E_2
      & = \pbraces{\pbraces{A_1 \cap C} \cup \pbraces{B_1 \cap C^\complement}} \cup
        \pbraces{\pbraces{A_2 \cap C} \cup \pbraces{B_2 \cap C^\complement}} \\
      & = \pbraces{\pbraces{A_1 \cap C} \cap \pbraces{A_2 \cap C}} \cup
          \underbrace
          {
            \Big (
            \pbraces{B_1 \cap C^\complement} \cap \pbraces{A_2 \cap C}
            \Big )
          }_\emptyset \\
      & \cup \underbrace
          {
            \Big (
            \pbraces{A_1 \cap C} \cap \pbraces{B_2 \cap C^\complement}
            \Big )
            }_\emptyset
            \pbraces
          {
            \pbraces{B_1 \cap C^\complement} \cap
            \pbraces{B_2 \cap C^\complement}
          } \\
      & = \pbraces{A_1 \cap A_2 \cap C} \cup
          \pbraces{B_1 \cap B_2 \cap C^\complement}
          \in \mathfrak{S}^\prime
    \end{align*}

  \end{itemize}

\end{itemize}

(b) \Quote{$\supseteq$}: Offensichtlich, gilt $\mathfrak{A}(\mathfrak{R}) \supseteq \mathfrak{R}$. Weil Algebren die Grundmenge enthalten und stabil bzgl. Differenzen sind, sind sie es auch bzgl. Komplementen. Damit folgt auch $\mathfrak{A}(\mathfrak{R}) \supseteq \mathfrak{R}^\complement$, wobei $^\complement$ punktweise zu verstehen ist. \\

\Quote{$\subseteq$}: So wie in (a), bemerkt man, dass die rechte Seite $\mathfrak{A} \supseteq \mathfrak{R}$ enhält und weist nach, dass $\mathfrak{A}$ eine Algebra ist.

\end{solution}

--------------------------------------------------------------------------------

\begin{exercise}

Hier könnte Ihre Werbung stehen!

\begin{itemize}
  \item[(a)] Definieren Sie Ring, Semiring, monotones System, Dynkin-System.
  \item[(b)] Gegeben sei der Wahrscheinlichkeitsraum $(\Omega, \mathfrak{S}, \P)$. Für $A \in \mathfrak{S}$ sei
  \begin{align*}
    \mathfrak{U}(A) = \Bbraces{B: \P(A \cap B) = \P(A) \P(B)}
  \end{align*}
  das System aller Mengen, die von $A$ unabhängig sind. Zeigen Sie, dass $\mathfrak{U}(A)$ ein Dynkin-System ist.
\end{itemize}

\end{exercise}

\begin{solution}

(a) Siehe Aufgabe 3 (a) \\

(b)

\begin{itemize}

  \item \Quote{Stabil bzgl. Differenzen von Teilmengen}: Seien $B, C \in \mathfrak{U}(A)$ mit $B \subseteq C$, dann gilt Folgendes.
  \begin{align*}
    \P(A \cap C \setminus B)
    & = \P(A) \P(C \setminus B)
    \Leftrightarrow \\
    \P(A) \P(B) + \P(A \cap C \setminus B)
    & = \P(A) \P(C \setminus B) + \P(A) \P(B)
      = \P(A) \P(C)
    \Leftrightarrow \\
    \P(A \cap C) + \P(A) \P(B)
      =\P(A \cap B) + \P(A) \P(B) + \P(A \cap C \setminus B)
    & = \P(A) \P(C) + \P(A \cap B)
  \end{align*}

  \item \Quote{Stabil bzgl. abzählbaren, disjunkten Vereinigungen}: Sei $(B_n) \in \mathfrak{D}$ disjunkt, dann
  \begin{align*}
    \P \pbraces{A \cap \sum_{n \in \N} B_n}
    =
    \sum_{n \in \N} \P(A \cap B_n)
    =
    \sum_{n \in \N} \P(A) \P(B_n)
    =
    \P(A) \P \pbraces{\sum_{n \in \N} B_n}.
  \end{align*}

  \item \Quote{Enthält Grundmenge}: $\P(A \cap \Omega) = \P(A) = \P(A) \P(\Omega)$

\end{itemize}

\end{solution}

--------------------------------------------------------------------------------

\begin{exercise}

Hier könnte Ihre Werbung stehen!

\begin{itemize}
  \item[(a)] Definieren Sie: äußeres Maß, Messbarkeit nach Caratheodory, von einer Maßfunktion erzeugtes Maß.
  \item[(b)] Zeigen Sie: wenn $\mu^\ast_n$ äußere Maße über derselben Menge sind, dann auch $\sup_n \mu^\ast_n$
\end{itemize}

\end{exercise}

\begin{solution}

(a) Siehe Aufgabe 1 (a) und Aufgabe 4 (a). \\

(b) Die ersten drei Eigenschaften sind offensichtlich. Seien also $A, (B_k) \in 2^\Omega$, mit $A \subseteq \bigcup_{k \in \N} B_k$, dann gilt Folgendes.
\begin{align*}
  \sum_{k \in \N} \sup_n \mu^\ast_n(B_k)
  \geq
  \sup_n \sum_{k \in \N} \mu^\ast_n(B_k)
  \geq
  \sup_n \mu^\ast_n(A)
\end{align*}

\end{solution}

--------------------------------------------------------------------------------

\begin{exercise}

$\mu$ und $\nu$ seinen zwei Maße auf dem Ring $\mathfrak{R}$. Zeigen Sie:

\begin{itemize}
  \item[(a)] $(\mu + \nu)^\ast = \mu^\ast + \nu^\ast$
  \item[(b)] $\mathfrak{M}_{\mu^\ast + \nu^\ast} \supseteq \mathfrak{M}_{\mu^\ast} \cap \mathfrak{M}_{\nu^\ast}$
  \item[(c)] Falls $\mu$ und $\nu$ sigmaendlich sind, dann gilt im vorigen Punkt Gleichheit.
\end{itemize}

\end{exercise}

\begin{solution}

(a) \Quote{$\geq$}: Sei $A \subseteq \Omega$, dann folgt die eine Ungleichung wegen \\

\begin{align*}
  (\mu + \nu)^\ast(A)
  & =
    \inf \Bbraces{\sum_{n \in \N} (\mu + \nu)(E_n): (E_n) \in \mathfrak{R}, A \subseteq \sum_{n \in \N} E_n} \\
  & \geq
    \inf \Bbraces{\sum_{n \in \N} \mu(E_n): (E_n) \in \mathfrak{R}, A \subseteq \sum_{n \in \N} E_n} +
    \inf \Bbraces{\sum_{n \in \N} \nu(E_n): (E_n) \in \mathfrak{R}, A \subseteq \sum_{n \in \N} E_n} \\
  & =
    \mu^\ast(A) + \nu^\ast(A).
\end{align*}

\Quote{$\leq$}: Für die Andere, finden wir für beliebiges $\epsilon > 0$ zwei disjunkte Überdeckungen $(B_n), (C_m) \in \mathfrak{R}$ von $A$, sodass

\begin{align*}
  \mu^\ast(A) \leq \sum_{n \in \N} \mu^\ast(B_n) \leq \mu^\ast(A) + \frac{\epsilon}{2}, \enspace
  \nu^\ast(A) \leq \sum_{m \in \N} \nu^\ast(C_m) \leq \nu^\ast(A) + \frac{\epsilon}{2}.
\end{align*}

Weil $(B_n \cap C_m)$ wieder eine disjunkte Überdeckungen von $A$ ist, folgt

\begin{align*}
  (\mu + \nu)^\ast(A)
  & \leq
    \sum_{n, m \in \N} (\mu + \nu)^\ast(B_n \cap C_m)
    \leq
    \sum_{n \in \N} \sum_{m \in \N} \mu^\ast(B_n \cap C_m) +
    \sum_{m \in \N} \sum_{n \in \N} \nu^\ast(B_n \cap C_m) \\
  & \leq
    \sum_{n \in \N} \mu^\ast(B_n) +
    \sum_{m \in \N} \nu^\ast(C_m)
    \leq
    \mu^\ast(A) + \nu^\ast(A) + \epsilon.
\end{align*}

\end{solution}

--------------------------------------------------------------------------------

\begin{exercise}

Zeigen Sie: Falls $\mu_n$ für jedes $n$ eine Maßfunktion auf $(\Omega, \mathfrak{S})$ ist, so ist auch $\mu = \sum_n \mu_n$ eine Maßfunktion.

\end{exercise}

\begin{solution}

Offensichtlich, ist $\Forall A \in \mathfrak{S}: \mu(A) \geq 0$. Es fehlt also nur noch die $\sigma$-Additivität von $\mu$. $\emptyset \in \mathfrak{S}$, also wählen wir $(A_k) \in \mathfrak{S}$ disjunkt.

\begin{align*}
  \mu(\sum_{k \in \N} A_k)
  =
  \sum_{n \in \N} \mu_n(\sum_{k \in \N} A_k)
  =
  \sum_{n \in \N} \sum_{k \in \N} \mu_n(A_k)
  =
  \sum_{k \in \N} \sum_{n \in \N} \mu_n(A_k)
  =
  \sum_{k \in \N} \mu(A_k)
\end{align*}

Dabei darf man die beiden Summen vertauschen, weil die Reihen entweder $\infty$ sind, oder absolut (also unbedingt) konvergieren. Der Rest ist die $\sigma$-Additivität von $\mu_n$ und pure Definition.

\end{solution}

--------------------------------------------------------------------------------

\begin{exercise}

Zeigen Sie: Falls $\mu_1$ und $\mu_2$ zwei äußere Maßfunktionen sind, so ist auch $\mu = \max(\mu_1, \mu_2)$ eine äußere Maßfunktion.

\end{exercise}

\begin{solution}

Die ersten drei Punkte sind offensichtlich. Für die Subadditivität sei $A \subseteq \bigcap_{n \in \N} B_n$, dann

\begin{align*}
  \mu(A)
  =
  \max(\mu_1(A), \mu_2(A))
  \leq
  \max(\mu_1(\sum_{n \in \N} B_n), \mu_2(\sum_{n \in \N} B_n))
  =
  \mu(\sum_{n \in \N} B_n).
\end{align*}

\end{solution}

--------------------------------------------------------------------------------

\begin{exercise}

Es sei

\begin{align*}
  \mathfrak{T} = \Bbraces{A \subset \R: |A| \leq 2}.
\end{align*}

\begin{itemize}
  \item[(a)] Zeigen Sie: $\mathfrak{T}$ ist ein Semiring.
  \item[(b)] Bestimmmen Sie den von $\mathfrak{T}$ erzeugten Ring bzw. Sigmaring.
  \item[(c)] Was muss eine Mengenfunktion $\mu$ auf $\mathfrak{T}$ erfüllen, damit sie ein Maß ist.
\end{itemize}

\end{exercise}

\begin{solution}

(a)

\begin{itemize}

  \item \Quote{Stabil bzgl. (endlichen) Durchschnitten}: Seien $A, B \in \mathfrak{T}$, so auch $A \cap B \in \mathfrak{T}$, weil
  \begin{align*}
    |A \cap B| \leq |A|, |B| \leq 2.
  \end{align*}

  \item \Quote{Enthält Leiterpflöcke} Seien $A, B \in \mathfrak{T}$, mit $A \subseteq B$. Nachdem $\Forall x \in \R: \Bbraces{x} \in \mathfrak{T}$, findet man disjunkte $C_1, \leq, C_n: B \setminus A = \sum_{i=1}^n C_i$.

  \item \Quote{Enthält Unterleitern} Offensichtlich gilt auch $\Forall k = 1, \ldots, n: A \cup \sum_{i=1}^k \in \mathfrak{T}$.

\end{itemize}

(b) Dazu gibt es zwei wunderbare Resultate:

\begin{itemize}

  \item $\mathfrak{R}(\mathfrak{T})
  = \Bbraces{\bigcup_{i=1}^n A_i: (A_i)_{i=1}^n \in \mathfrak{T}^n}
  = \Bbraces{\sum   _{i=1}^n A_i: (A_i)_{i=1}^n \in \mathfrak{T}^n}
  = \Bbraces{A \subset \R: |A| < \infty}$.

  \item Laut dem \Quote{Monotone Class Theorem}, gilt
  $\mathfrak{R}_\sigma(\mathfrak{T})
  = \mathfrak{R}_\sigma(\mathfrak{R}(\mathfrak{T}))
  = \mathfrak{M}(\mathfrak{R}(\mathfrak{T}))$.
  Dieses Mengenssytem ist stabil bzgl. $\lim_{n \to \infty} A_n = \bigcup_{n \in \N} A_n$, also genau $\Bbraces{A \subset \R: |A| \leq \aleph_0}$.

\end{itemize}

(c) Siehe Aufgabe 1 (a).

\end{solution}

--------------------------------------------------------------------------------

\begin{exercise}

Es sei $\Omega = (0, 1] \times (0, 1]$ und

\begin{align*}
  \mathfrak{T}
  =
  \Bbraces{(a, b] \times (0, 1]: 0 \leq a \leq b \leq 1}
  \cup
  \Bbraces{(0, 1] \times (a, b]: 0 \leq a \leq b \leq 1}
\end{align*}

Ferner sei

\begin{align*}
  \mu((a, b] \times (0, 1]) = \mu((0, 1] \times (a, b]) = b - a
\end{align*}

\begin{itemize}
  \item[(a)] Zeigen Sie: $\mu$ ist ein Inhalt auf $\mathfrak{T}$.
  \item[(b)] Zeigen Sie $\mathfrak{T}$ ist kein Semiring.
  \item[(c)] Zeigen Sie: Die Fortsetzung von $\mu$ zu einem Inhalt auf dem erzeugten Ring ist nicht eindeutig bestimmt.
\end{itemize}

\end{exercise}

\begin{solution}

(a) Offensichtlich gilt $\Forall A \in \mathfrak{T}: \mu(A) \geq 0$. Zu zeigen, bleibt also nur noch die Additivität. Seien dazu $A_1, \ldots, A_n \in \mathfrak{T}$ disjunkt und $\sum_{i=1}^n A_i \in \mathfrak{T}$. $A_1, \ldots, A_n$ wollen wir aber darstellen, mit $\Forall i = 1, \ldots, n: \Exists a_i, b_i \in (0, 1):$

\begin{align*}
  A_i = (a_i, b_i] \times (0, 1]
  \enspace \text{oder} \enspace
  A_i = (0, 1] \times (a_i, b_i].
\end{align*}

Weil $A_1, \ldots, A_n$ ja disjunkt sind, müssen sie aber alle die selbe, einer der oberen, Darstellungen haben. Weil $\sum_{i=1}^n A_i \in \mathfrak{T}$, müssen $A_1, \ldots, A_n$ auch direkt an einender angrenzen. Das führt zu

\begin{align*}
  \sum_{i=1}^n \mu(A_i)
  =
  \sum_{i=1}^n b_i - a_i
  =
  \mu(\sum_{i=1}^n A_i).
\end{align*}

(b) $\mathfrak{T}$ ist nicht stabil bzgl. (endlichen) Durchschnitten, weil $\Forall a, b, c, d \in (0, 1): a \leq b$, $c \leq d \Rightarrow$

\begin{align*}
  (a, b] \times (0, 1]
  \cap
  (0, 1] \times (c, d]
  =
  (a, b] \times (c, d]
  \notin \mathfrak{T}
\end{align*}

(c)

\end{solution}

--------------------------------------------------------------------------------

\begin{exercise}

Bestimmen Sie die Wahrscheinlichkeit, dass eine zufällig (gleichverteilt) gewählte Permuation von $n$ Elementen keinen Fixpunkt hat.

\end{exercise}

\begin{solution}

Click \href{https://de.wikipedia.org/wiki/Fixpunktfreie_Permutation}{here}!

\end{solution}

--------------------------------------------------------------------------------

\begin{exercise}

Ein Würfel wird einmal geworfen, $X$ sei die erzielte Augenzahl. Anschließend werden in eine Urne $X$ weiße und eine schwarze Kugel gelegt, und eine Kugel wird gezogen. $A$ sei das Ereignis, dass die schwarze Kugel gezogen wurde. Bestimmen Sie

\begin{itemize}
  \item[(a)] $\mathbf{P}(A)$
  \item[(b)] $\mathbf{P}(X = 3 | A)$
\end{itemize}

\end{exercise}

\begin{solution}

(a) Die Urne enthält also $X + 1$ Kugeln und damit ist $\mathbf{P}(A) = \frac{1}{X + 1}$. \\

(b) Sei $H_i := [X = i]$, für $i = 1, \ldots, 6$, so ist $(H_i)_{i=1}^6$ ein vollständiges Ereignissystem. Offensichtlich, gilt $\Forall i = 1, \ldots, 6: \mathbf{P}(H_i) = \frac{1}{6}$. Nachdem $\mathbf{P}(A) > 0$, können wir das \Quote{Bayes'sche Theorem} anwenden und erhalten

\begin{align*}
  \mathbf{P}(H_3 | A)
  =
  \frac
  {\mathbf{P}(H_3) \mathbf{P}(A | H_3)}
  {\sum_{i=1}^6 \mathbf{P}(H_i) \mathbf{P}(A | H_i)}
  =
  \frac
  {\frac{1}{3+1}}
  {\sum_{i=1}^6 \frac{1}{i+1}}
  = \frac{35}{223}.
\end{align*}

\end{solution}

--------------------------------------------------------------------------------

\begin{exercise}

Stellen Sie fest, welche der folgenden Mengenfunktionen auf $2^\R$ äußere Maße sind. Bestimmen Sie für die äußeren Maßfunktionen jeweils das System der $\mu^\ast$-messbaren Mengen.

\begin{itemize}
  \item[(a)] $\mu^\ast(A) = |A|$
  \item[(b)]
  \begin{align*}
    \mu^\ast(A) =
    \begin{cases}
      0 & \text{wenn} \enspace |A| < \infty, \\
      1 & \text{sonst},
    \end{cases}
  \end{align*}
  \item[(c)]
  \begin{align*}
    \mu^\ast(A) =
    \begin{cases}
      0 & \text{wenn} \enspace \card(A) \leq \aleph_0, \\
      1 & \text{sonst},
    \end{cases}
  \end{align*}
  \item[(d)]
  \begin{align*}
    \mu^\ast(A) =
    \begin{cases}
      0 & \text{wenn} \enspace A = \emptyset, \\
      2 & \text{wenn} \enspace A = \R, \\
      1 & \text{sonst},
    \end{cases}
  \end{align*}
\end{itemize}

\end{exercise}

\begin{solution}

(a) Offensichtlich, ist $\mu^\ast$ ein äußeres Maß und $\mathfrak{M}_{\mu^\ast} = 2^\R$. \\

(b) Seien $A := \N$ und $B_n := \Bbraces{n}$, $n \in \N$. Dann gilt aber $\mu^\ast(A) = 1$ und $\Forall n \in \N: \mu^\ast(B_n) = 0$ und somit $\mu^\ast(A) \nleq \sum_{n \in \N} \mu^\ast(B_n)$. \\

(c) Die ersten drei Eigenschaften sind offensichtlich. Seien also $A, (B_n) \in 2^\R$, mit $A \subseteq \bigcup_{n \in \N} B_n$. Wenn $\card(A) \leq \aleph_0$, sind wir fertig. Ansonsten, $\Exists n \in \N: \card(B_n) > \aleph_0$ weil die obere Vereinigung (nur) abzählbar ist. \\

(d)

\end{solution}

--------------------------------------------------------------------------------

\begin{exercise}

Hier könnte Ihre Werbung stehen!

\begin{itemize}
  \item[(a)] Definieren Sie: Ring, Semiring, Sigmaring, Algebra, Sigmaalgebra, Dynkin-System, monotones System.
  \item[(b)] Von welchem Typ sind folgende Mengensysteme über $\Omega = \R$?
  \begin{align*}
    \mathfrak{C}_1 & = \Bbraces{A \subseteq \R: |A| < \infty}, \\
    \mathfrak{C}_2 & = \Bbraces{A \subseteq \R: |A| < 5}, \\
    \mathfrak{C}_3 & = \Bbraces{A \subseteq \R: |A| < \infty, \enspace \text{gerade}}, \\
    \mathfrak{C}_4 & = \Bbraces{A \subseteq \R: \card(A) \leq \aleph_0}, \\
  \end{align*}
\end{itemize}

\end{exercise}

\begin{solution}

(a) $\emptyset \neq \mathfrak{R}_\sigma \subseteq 2^\Omega \Text{Sigmaring} : \Leftrightarrow$
\begin{itemize}
  \item $\Forall (A_n) \in \mathfrak{R}_\sigma, \text{disj.}: \sum_{n \in \N} A_n \in \mathfrak{R}_\sigma$
  \item $\Forall A, B \in \mathfrak{R}_\sigma: A \setminus B \in \mathfrak{R}_\sigma,$
\end{itemize}

Für den Rest: Siehe Aufgabe 3 (a). \\

(b)

\begin{itemize}

  \item $\mathfrak{C}_1$ ist
  \begin{itemize}
    \item ein Ring, weil mit $|A|, |B| < \infty$, auch $|A \cup B|, |A \setminus B| < \infty$,
    \item ein Semiring, weil Ring,
    \item kein Sigmaring, weil $\Forall n \in \N: \Bbraces{n} \in \mathfrak{C}_1$, aber $\N \notin \mathfrak{C}_1$,
    \item keine Algebra, weil $\R \notin \mathfrak{C}_1$,
    \item keine Sigmaalgebra, weil kein Algebra,
    \item kein Dynkin-System, weil kein monotones System,
    \item und kein monotones System, weil $\Forall n \in \N: \Bbraces{1, \leq, n} \in \mathfrak{C}_1$, aber $\N \notin \mathfrak{C}_1$.
  \end{itemize}

  \item$\mathfrak{C}_2$ ist
  \begin{itemize}
    \item kein Ring, weil $|A|, |B| < 5$, im Allgemeinen nicht $|A \cup B| < 5$ folgt,
    \item ein Semiring, weil $\mathfrak{C}_2$ stabil bzgl. (endlicher) Durchschnitte ist, und $\Forall x \in \R: \Bbraces{x} \in \mathfrak{C}_2$, also \Quote{Leitern} gebaut werden können, bzgl. denen $\mathfrak{C}_2$ stabil ist,
    \item kein Sigmaring, weil kein Ring,
    \item keine Algebra, weil kein Ring,
    \item keine Sigmaalgebra, weil kein Algebra,
    \item kein Dynkin-System, weil $\Forall n \in \N: \Bbraces{n} \in \mathfrak{C}_1$, aber $\N \notin \mathfrak{C}_1$,
    \item ein monotones System, weil jede monotone Mengenfolge, nur höchstens $5$ verschiedene Folgenglieder haben kann, also fast überall konstant ist.
  \end{itemize}

  \item $\mathfrak{C}_3$ ist
  \begin{itemize}
    \item kein Ring, weil kein Semiring,
    \item kein Semiring, weil $\Bbraces{-1, 0}, \Bbraces{0, 1} \in \mathfrak{C}_3$, aber $\Bbraces{-1, 0} \cap \Bbraces{0, 1} = \Bbraces{0} \notin \mathfrak{C}_3$,
    \item kein Sigmaring, weil kein Semiring,
    \item keine Algebra, weil kein Semiring,
    \item keine Sigmaalgebra, weil kein Semiring,
    \item kein Dynkin-System, weil kein monotones System,
    \item kein monotones System, weil $\Forall n \in \N: \Bbraces{1, \ldots, 2n} \in \mathfrak{C}_3$, aber $\N \notin \mathfrak{C}_3$.
  \end{itemize}

  \item $\mathfrak{C}_4$ ist
  \begin{itemize}
    \item ein Ring, weil Sigmaring,
    \item ein Semiring, weil Sigmaring,
    \item ein Sigmaring, weil mit $\card(A_n), \card(A), \card(B) \leq \aleph_0$, $n \in \N$, auch $\card(\bigcup_{n \in \N} A_n), \card(A \setminus B) \leq \aleph_0$,
    \item keine Algebra, weil $\card(\R) > \aleph_0$,
    \item keine Sigmaalgebra, weil keine Algebra,
    \item kein Dynkin-System, weil $\card(\R) > \aleph_0$,
    \item ein monotones System, weil Sigmaring.
  \end{itemize}

\end{itemize}

\end{solution}

--------------------------------------------------------------------------------

\begin{exercise}

Hier könnte Ihre Werbung stehen!

\begin{itemize}
  \item[(a)] Definieren Sie: Maßfunktion, endliche Maßfunktion, sigmaendliche Maßfunktion, äußere Maßfunktion, von einem Maß auf einem Ring erzeugte äußere Maßfunktion, messbare Menge.
  \item[(b)] Zeigen Sie, dass das System $\mathfrak{M}$ der $\mu^\ast$-messbaren Mengen eine Sigmaalgebra ist, und dass die Einschränkung von $\mu^\ast$ auf $\mathfrak{M}$ ein Maß ist.
\end{itemize}

\end{exercise}

\begin{solution}

(a) Siehe Aufgabe 4. \\

(b) Siehe Skript.

\end{solution}

--------------------------------------------------------------------------------

\begin{exercise}

Hier könnte Ihre Werbung stehen!

\begin{itemize}
  \item[(a)] Definieren Sie: Maßfunktion, endliche Maßfunktion, sigmaendliche Maßfunktion, äußere Maßfunktion, von einem Maß auf einem Ring erzeugte äußere Maßfunktion.
  \item[(b)] $\mu$ und $\nu$ seien zwei Maßfunktionen auf dem Ring $\mathfrak{R}$. Zeigen Sie:
  \begin{align*}
    (\mu + \nu)^\ast = \mu^\ast + \nu^\ast
  \end{align*}
  und
  \begin{align*}
    \mathfrak{M}_{\mu^\ast} \cap \mathfrak{M}_{\nu^\ast}
    \subseteq
    \mathfrak{M}_{\mu^\ast + \nu^\ast}.
  \end{align*}
\end{itemize}

\end{exercise}

\begin{solution}

(a) Siehe Aufgabe 4. \\

(b) Siehe Aufgabe 9.

\end{solution}

--------------------------------------------------------------------------------

\begin{exercise}

\begin{itemize}
  \item[(a)] Definieren Sie: Ring, Semiring, Sigmaring, Algebra, Sigmaalgebra, Dynkin-System.
  \item[(b)] $\mu$ und $\nu$ seien zwei Wahrscheinlichkeitsmaße auf dem Messraum $(\Omega, \mathfrak{S})$. Zeigen Sie, dass
  \begin{align*}
    \mathfrak{D} = \Bbraces{A \in \mathfrak{S} : \mu(A) = \nu(A)}
  \end{align*}
  ein Dynkin-System ist.
\end{itemize}

\end{exercise}

\begin{solution}

(a) Siehe Aufgabe (a).

(b)

\begin{itemize}

  \item \Quote{Stabil bzgl. Differenzen von Teilmengen}: Seien $A, B \in \mathfrak{D}$ mit $A \subseteq B$, dann gilt Folgendes.
  \begin{align*}
    \mu(B \setminus A)
    & =
    \nu(B \setminus A)
    \Leftrightarrow \\
    \nu(A) + \mu(B \setminus A)
    & =
    \nu(B \setminus A) + \nu(A)
    =
    \nu(B)
    \Leftrightarrow \\
    \nu(A) + \mu(B)
    =
    \mu(A) + \nu(A) + \mu(B \setminus A)
    & =
    \nu(B) + \mu(A)
  \end{align*}

  \item \Quote{Stabil bzgl. abzählbaren, disjunkten Vereinigungen}: Sei $(A_n) \in \mathfrak{D}$ disjunkt, dann
  \begin{align*}
    \mu(\sum_{n \in \N} A_n)
    =
    \sum_{n \in \N} \mu(A_n)
    =
    \sum_{n \in \N} \nu(A_n)
    =
    \nu(\sum_{n \in \N} A_n).
  \end{align*}

  \item \Quote{Enthält Grundmenge}: $\mu(\Omega) = 1 = \nu(\Omega)$

\end{itemize}

\end{solution}

\setcounter{exercise}{0}

\section{Lebesgue-Stieltjes Maße}

--------------------------------------------------------------------------------

\begin{exercise}

Zeigen Sie:

\begin{align*}
  F(x, y) =
  \begin{cases}
    x y^2 & \text{falls} \enspace y > 0. \\
    0     & \text{sonst}
  \end{cases}
\end{align*}

ist eine zweidimensionale Verteilungsfunktion. Bestimmen Sie das Maß des Einheitskreises.

\end{exercise}

\begin{solution}

Hier könnte Ihre Werbung stehen!

\begin{itemize}

  \item \Quote{Rechtsstetigkeit}: Tatsächlich gilt sogar $\Forall x \in \R: F(x, 0 + 0) = F(x, 0) = 0$.

  \item \Quote{nichtnegativer Differenzenoperator}: Seien $a, b \in \R^2$, mit $a \leq b$.
  \begin{align*}
    \Delta^{(a, b)} F(x)
    & =
    \Delta_1^{(a_1, b_1)}
    \Delta_2^{(a_2, b_2)}
    F(x) \\
    & =
    \Delta_1^{(a_1, b_1)}
    \pbraces{F(x_1, b_2) - F(x_1, a_2)} \\
    & =
    \pbraces{\Delta_1^{(a_1, b_1)} F(x_1, b_2)} -
    \pbraces{\Delta_1^{(a_1, b_1)} F(x_1, a_2)} \\
    & =
    \pbraces{F(b_1, b_2) - F(a_1, b_2)} -
    \pbraces{F(b_1, a_2) - F(a_1, a_2)} \\
    & =
    \pbraces{b_1 b_2^2 - a_1 b_2^2} -
    \pbraces{b_1 a_2^2 - a_1 a_2^2} \\
    & =
    b_2^2 (b_1 - a_1) - a_2^2 (b_1 - a_1) \\
    & =
    (b_1 - a_1) (b_2^2 - a_2^2)
    \geq
    0
  \end{align*}

\end{itemize}

Die untere Hälfte des Einheitskreises, hat offensichtlich Maß Null. Die Obere, werden wir mit Untersummen approximieren. Dazu betrachten wir vorerst die rechte Hälfte.

\begin{align*}
  \sum_{i=0}^{n-1}
  \mu_F
  \pbraces
  {
    \Bigg ]
      \pbraces{\frac{i}{n}, 0},
      \pbraces{\frac{i+1}{n}, \sqrt{1 - \pbraces{\frac{i+1}{n}}^2}}
    \Bigg ]
  }
  =
  \sum_{i=0}^{n-1}
  \pbraces{\frac{i+1}{n} - \frac{i}{n}}
  \pbraces{\sqrt{1 - \pbraces{\frac{i+1}{n}}^2}^2 - 0^2} \\
  \xrightarrow[n \to \infty]{}
  \Int[0][1]{1 - x^2}{x}
  =
  1 - \frac{x^3}{3} \Bigg |_0^1
  =
  \frac{2}{3}
\end{align*}

Nachdem dieser Ausdruck symmetrisch um die $y$-Achse ist (Quadrat), gilt dieser auch für die linke Hälfte und damit $\mu_F(B(0, 1)) = \frac{4}{3}$.

\end{solution}

--------------------------------------------------------------------------------

\begin{exercise}

Hier könnte Ihre Werbung stehen!

\begin{itemize}
  \item[(a)] Definieren Sie Maßfunktion, Lebesgue-Stieltjes Maßfunktion, Verteilungsfunktion.
  \item[(b)] Zeigen Sie, dass jede nichtfallende und rechtstetige Funktion Verteilungsfunktion eines Lebesgue-Stieltjes Maßes auf $\R$ ist.
  \item[(c)] Zeigen Sie, dass $F(x) = \min(x_1, x_2)$ zweidimensionale Verteilungsfunktion ist.
\end{itemize}

\end{exercise}

\begin{solution}

(a) Siehe Aufgabe 1 (a), Kapitel 1 und Aufgabe 11 (a) Kapitel 2. \\

(b) Siehe Skript! \\

(c) Siehe Aufgabe 6.

\end{solution}

--------------------------------------------------------------------------------

\begin{exercise}

Zeigen Sie:

\begin{align*}
  F(x, y) = \min(x, y)
\end{align*}

ist eine zweidimensionale Verteilungsfunktion, und bestimmen Sie $\mu_F(]0, 1] \times ]0, 1])$ und $\mu_F(\Bbraces{(x, x) : 0 < x \leq 1})$.

\end{exercise}

\begin{solution}

Hier könnte Ihre Werbung stehen!

\begin{itemize}

  \item \Quote{Rechtsstetigkeit}: $\min(x, y) = \Frac{2}{x + y - |x - y|}$

  \item \Quote{nichtnegativer Differenzenoperator}: Seien $a, b \in \R^2$, mit $a \leq b$.
  \begin{align*}
    \Delta^{(a, b)} \min
    =
    \min(b_1, b_2) - \min(a_1, b_2) - \min(b_1, a_2) + \min(a_1, a_2)
  \end{align*}

\end{itemize}

Es könnte jetzt eine langweilige Fallunterscheidung folgen. Wir führen diese aber nicht zur Gänze aus.

\begin{itemize}
  \item[Fall 1:] $b_1 \leq b_2 \Rightarrow a_1 \leq b_2$
  \begin{itemize}
    \item[Fall a:] $a_1 \leq a_2 \Rightarrow a_1 \leq b_2$
    \begin{itemize}
      \item[Fall i:] $a_2 \leq b_1 \Rightarrow
      \Delta^{(a, b)} \min = b_1 - a_1 - a_2 + a_1 \geq 0$
      \item[Fall ii:] $b_1 \leq a_2 \Rightarrow
      \Delta^{(a, b)} \min = b_1 - a_1 - b_1 + a_1 \geq 0$
    \end{itemize}
  \end{itemize}
\end{itemize}

Wir berechnen nun die Maße der Mengen

\begin{align*}
  A := ]0, 1]^2, \\
  B := \Bbraces{(x, x): 0 < x \leq 1}.
\end{align*}

\begin{itemize}

  \item $\mu(A) =
  \Delta^{(0, 0), (1, 1)} \min =
  \min(1, 1) - \min(0, 1) - \min(1, 0) + \min(0, 0) = 1$

  \item Wir betrachten die Überdeckungen von
  \begin{align*}
    B \supseteq B_n
    :=
    \sum_{i=1}^n
    \left ] \frac{i-1}{n}, \frac{i}{n} \right ]^2
    \xrightarrow{n \in \N^2} B.
  \end{align*}
  Weil $(B_{n_k})$ mit $n_k := k^2$ monoton gegen $B$ fällt, kann man die Stetigkeit von oben ausnützen.
  \begin{align*}
    \mu(B)
    =
    \mu \pbraces{\lim_{n \in \N^2} B_n}
    =
    \lim_{n \in \N^2} \mu(B_n)
    =
    \lim_{n \in \N^2} \sum_{i=1}^n \mu \pbraces
    {\left ] \frac{i-1}{n}, \frac{i}{n} \right ]^2}
    = \\
    \lim_{n \in \N^2} \sum_{i=1}^n
    \min \pbraces{\frac{i-1}{n}, \frac{i}{n}} -
    \min \pbraces{\frac{i-1}{n}, \frac{i}{n}} -
    \min \pbraces{\frac{i-1}{n}, \frac{i}{n}} +
    \min \pbraces{\frac{i-1}{n}, \frac{i}{n}}
    = 0
  \end{align*}

\end{itemize}

\end{solution}

--------------------------------------------------------------------------------

\begin{exercise}

Zeigen Sie:

\begin{align*}
  F(x, y) =
  \begin{cases}
    x y^2 & \text{falls} \enspace y > 0. \\
    0     & \text{sonst}
  \end{cases}
\end{align*}

ist eine zweidimensionale Verteilungsfunktion. Bestimmen Sie das Maß des Einheitskreises.

\end{exercise}

\begin{solution}

Siehe Aufgabe 1.

\end{solution}

--------------------------------------------------------------------------------

\begin{exercise}

Gegeben ist die Funktion $F: \R \to \R:$

\begin{align*}
  F(x) =
  \begin{cases}
    x       & \text{für} \enspace x < 0 \\
    x^2 + 1 & \text{für} \enspace 0 \leq x < 1, \\
    2x      & \text{für} \enspace 1 \leq x < 3, \\
    8       & \text{für} \enspace x \geq 3.
  \end{cases}
\end{align*}

Weisen Sie nach, dass $F$ eine Verteilungsfunktion ist, und bestimmen Sie $\mu_F(]0, 1])$, $\mu_F([0, 1])$, $\mu_F(]0, 1[)$ und $\mu_F(\Q)$.

\end{exercise}

\begin{solution}

Nachweisen:

\begin{itemize}

  \item \Quote{Rechtsstetigkeit}: $F$ ist stückweise stetig und $\Forall x = 0, 1, 3: F \text{ist rechtsstetig bei} \enspace x$.

  \item \Quote{Steigende Monotonie}: $F$ ist stückweise monoton steigend und $\Forall x = 0, 1, 3: F(x - 0) \leq F(x)$.

\end{itemize}

Bestimmen:

\begin{itemize}

  \item $\mu_F(]0, 1]) =$
  \begin{align*}
    F(1) - F(0) = 2 - 1 = 1
  \end{align*}

  \item $\mu_F([0, 1]) =$
  \begin{align*}
    \mu_F \pbraces{\bigcap_{n \in \N} \left ] 0 - \frac{1}{n}, 1 \right]}
    =
    \lim_{n \in \N} \mu_F \pbraces{ \left ] 0 - \frac{1}{n}, 1 \right]}
    =
    \lim_{n \in \N} F(1) - F \pbraces{0 - \frac{1}{n}}
    =
    \lim_{n \in \N} 2 + \frac{1}{n} = 2
  \end{align*}

  \item $\mu_F(]0, 1[) =$
  \begin{align*}
    \mu_F \pbraces{\bigcup_{n \in \N} \left ] 0, 1 - \frac{1}{n} \right ]}
    =
    \lim_{n \in \N} \mu_F \pbraces{\left ] 0, 1 - \frac{1}{n} \right ]}
    =
    \lim_{n \in \N} F \pbraces{1 - \frac{1}{n}} - F(0)
    =
    \lim_{n \in \N} \pbraces{1 - \frac{1}{n}}^2 - 1 = 0
  \end{align*}

  \item $\mu_F(\Q) =$
  \begin{align*}
    \mu_F \pbraces{\sum_{q \in \Q} \Bbraces{q}}
    & =
    \sum_{q \in \Q} \mu_F \pbraces
    {\bigcap_{n \in \N} \left ] q - \frac{1}{n}, q \right ]} \\
    & =
    \sum_{q \in \Q} \lim_{n \in \N} \mu_F \pbraces
    {\left ] q - \frac{1}{n}, q \right ]} \\
    & =
    \sum_{q \in \Q} F(q - 0) - F(q) \\
    & =
    \sum_{x = 0, 1, 3} (F(x) - F(x - 0)) \\
    & =
    (1 - 0) + (2 - 2) + (8 - 6) = 3
  \end{align*}

\end{itemize}

\end{solution}

--------------------------------------------------------------------------------

\begin{exercise}

Gegeben ist die Funktion $F: \R \to \R:$

\begin{align*}
  F(x) =
  \begin{cases}
    0   & \text{wenn} \enspace x < 0, \\
    1   & \text{wenn} \enspace 0 \leq x < 1, \\
    x^2 & \text{wenn} \enspace 1 \leq x < 2, \\
    5   & \text{wenn} \enspace x \geq 2.
  \end{cases}
\end{align*}

\begin{itemize}
  \item[(a)] Zeigen Sie, dass $F$ eine Verteilungsfunktion ist.
  \item[(b)] Bestimmen Sie $\mu_F(]0, 1[)$, $\mu_F([0, 2])$, $\mu_F(\Q)$.
  \item[(c)] Bestimmen Sie $\Int{e^x}{\mu_F(x)}$.
\end{itemize}

\end{exercise}

\begin{solution}

(a)

\begin{itemize}

  \item \Quote{Rechtsstetigkeit}: $F$ ist stückweise stetig und $\Forall x = 0, 1, 2: F \text{ist rechtsstetig bei} \enspace x$.

  \item \Quote{Steigende Monotonie}: $F$ ist stückweise monoton steigend und $\Forall x = 0, 1, 2: F(x - 0) \leq F(x)$.

\end{itemize}

(b)

\begin{itemize}

  \item $\mu_F(]0, 1[) =$
  \begin{align*}
    \mu_F \pbraces{\bigcup_{n \in \N} \left ] 0, 1 - \frac{1}{n} \right ]}
    =
    \lim_{n \in \N} \mu_F \pbraces{\left ] 0, 1 - \frac{1}{n} \right ]}
    =
    \lim_{n \in \N} F \pbraces{1 - \frac{1}{n}} - F(0)
    = 1 - 1 = 0
  \end{align*}

  \item $\mu_F([0, 2]) =$
  \begin{align*}
    \mu_F \pbraces{\bigcap_{n \in \N} \left ] 0 - \frac{1}{n}, 2 \right ]}
    =
    \lim_{n \in \N} \mu_F \pbraces{\left ] 0 - \frac{1}{n}, 2 \right ]}
    =
    \lim_{n \in \N} F(2) - F \pbraces{0 - \frac{1}{n}}
    =
    5 - 0 = 5
  \end{align*}

  \item $\mu_F(\Q) =$
  \begin{align*}
    \mu_F \pbraces{\sum_{q \in \Q} \Bbraces{q}}
    & =
    \sum_{q \in \Q} \mu_F \pbraces
    {\bigcap_{n \in \N} \left ] q - \frac{1}{n}, q \right ]} \\
    & =
    \sum_{q \in \Q} \lim_{n \in \N} \mu_F \pbraces
    {\left ] q - \frac{1}{n}, q \right ]} \\
    & =
    \sum_{q \in \Q} F(q - 0) - F(q) \\
    & =
    \sum_{x = 0, 1, 2} (F(x) - F(x - 0)) \\
    & =
    (1 - 0) + (1 - 1) + (5 - 4) = 2
  \end{align*}

\end{itemize}

(c) Seien $f = \exp$ und $a_1, \ldots, a_n$ die Sprünge von $F$, sowie $a_0 = - \infty$ und $a_{n+1} = \infty$.

\begin{align*}
  \Int{f}{\mu_F}
  =
  \sum_{i=1}^{n+1} \Int[a_{i-1}][a_i]{f(x) F^\prime(x)}{x} +
  \sum_{i=1}^n f(a_i) (F(a_i) - F(a_i - 0))
\end{align*}

Also ...

\begin{align*}
  \Int{e^x}{\mu_F(x)}
  & =
  \underbrace{\Int[-\infty][0]{e^x 0}{x}}_0
  +
  \underbrace{\Int[0][1]{e^x 0}{x}}_0
  +
  \Int[1][2]{e^x 2x}{x}
  +
  \underbrace{\Int[2][\infty]{e^x 0}{x}}_0 \\
  & +
  e^0 \underbrace{(F(0) - F(0 - 0))}_{= 1-0 = 1}
  +
  e^1 \underbrace{(F(1) - F(1 - 0))}_{= 1-1 = 0}
  +
  e^2 \underbrace{(F(2) - F(2 - 0))}_{= 5-4 = 1} \\
  & =
  e^x 2x |_1^2 - 2 \Int[1][2]{e^x}{x} + 1 + e^2 \\
  & =
  (4e^2 - 2e) - 2 (e^2 - e) + 1 + e^2
  =
  1 + 3e^2
\end{align*}

\end{solution}

--------------------------------------------------------------------------------

\begin{exercise}

Hier könnte Ihre Werbung stehen!

\begin{itemize}
  \item[(a)] Definieren Sie Lebesgue-Stieltjes Maß, Verteilungsfunktion.
  \item[(b)] $F_1$ und $F_2$ seien zwei eindimensionale Verteilungsfunktionen. Zeigen Sie, dass
  \begin{align*}
    F(x_1, x_2) = F_1(x_1) F_2(x_2)
  \end{align*}
  eine zweidimensionale Verteilungsfunktion ist.
\end{itemize}

\end{exercise}

\begin{solution}

(a)

\begin{itemize}

  \item $\mu \enspace \text{Lebesuge-Stiletjes Maß} : \Leftrightarrow$
  \begin{itemize}
    \item $\mu: (\R^k, \mathfrak{B}_k) \to \R, \text{Maß}$
    \item $\Forall A \in \mathfrak{B}_k, \text{beschr.}: \mu(A) < \infty$
  \end{itemize}

  \item $F: \R \to \R, \text{Verteilungsfunktion von} \enspace \mu : \Leftrightarrow$
  \begin{itemize}
    \item $\mu \enspace \text{Lebesgue-Stieltjes Maß}$
    \item $\Forall a \leq b \in \R: \mu((a, b]) = F(b) - F(a)$
  \end{itemize}

\end{itemize}

(b)

\begin{itemize}

  \item \Quote{Rechtsstetigkeit}: $F$ ist als Produkt der rechtsstetigen $F_1, F_2$ wieder rechtsstetig.

  \item \Quote{nichtnegativer Differenzenoperator}: Seien $a, b \in \R^2$, mit $a \leq b$.
  \begin{align*}
    \Delta^{(a, b)} F
    & =
    F(b_1, b_2) - F(a_1, b_2) - F(b_1, a_2) + F(a_1, a_2) \\
    & =
    F_1(b_1) F_2(b_2) -
    F_1(a_1) F_2(b_2) -
    F_1(b_1) F_2(a_2) +
    F_1(a_1) F_2(a_2) \\
    & =
    F_1(b_1) (F_2(b_2) - F_2(a_2)) -
    F_1(a_1) (F_2(b_2) - F_2(a_2)) \\
    & =
    \underbrace{(F_1(b_1) - F_1(a_1))}_{\geq 0}
    \underbrace{(F_2(b_2) - F_2(a_2))}_{\geq 0}
    \geq 0
  \end{align*}

\end{itemize}

\end{solution}

\setcounter{exercise}{0}

\section{Messbare Funktionen}

--------------------------------------------------------------------------------

\begin{exercise}

Es sei $\Omega = \N$, $\mathfrak{S} = 2^\N$, $\mu(A) = \sum_{x \in A} 2^{-x}$. Wann konvergiert die Funktionenfolge $f_n$ im Maßraum $(\Omega, \mathfrak{S}, \mu)$

\begin{itemize}
  \item[(a)] fast überall
  \item[(b)] fast gleichmäßig
  \item[(c)] im Maß?
\end{itemize}

\end{exercise}

\begin{solution}

(a) $f_n \xrightarrow{\text{f.ü.}} f : \Leftrightarrow$

\begin{align*}
  \Exists N \in \mathfrak{S}:
  \mu(N) = 0,
  f_n|_{N^\complement} \xrightarrow{\text{punktw.}} f|_{N^\complement}
\end{align*}

Aber ...

\begin{align*}
  \mu(N) = 0
  \Leftrightarrow
  \sum_{x \in N} 2^{-x} = 0
  \Leftrightarrow
  N = \emptyset
\end{align*}

Also ...

\begin{align*}
  f_n \xrightarrow{\text{f.ü.}} f
  \Leftrightarrow
  f_n \xrightarrow{\text{punktw.}} f
\end{align*}

\end{solution}

(b) $f_n \xrightarrow{\mu \text{-fast glm.}} f : \Leftrightarrow$

\begin{align*}
  \Forall \epsilon > 0:
  \Exists A_\epsilon \in \mathfrak{S}:
  \mu(A_\epsilon) < \epsilon,
  f_n|_{A_\epsilon^\complement} \xrightarrow{\text{glm.}} f|_{A_\epsilon^\complement}
\end{align*}

$\mu$ ist ein endliches Maß, weil

\begin{align*}
  \mu(\Omega) = \mu(\N) = \sum_{n \in \N} \frac{1}{2^n} = 2 < \infty.
\end{align*}

Laut \Quote{Egorov}, gilt also

\begin{align*}
  f_n \xrightarrow{\mu \text{-fast glm.}} f
  \Leftrightarrow
  f_n \xrightarrow{\text{f.ü.}} f.
\end{align*}

(c) $f_n \xrightarrow{\text{im Maß}} f : \Leftrightarrow$

\begin{align*}
  \Forall \epsilon > 0:
  \lim_{n \to \infty} \mu(|f_n - f| > \epsilon) = 0
\end{align*}

zz: $f_n \xrightarrow{\text{im Maß}} f \Leftrightarrow f_n \xrightarrow{\text{punktw.}} f$ \\

Nun gilt

\begin{align*}
  \mu(|f_n - f| > \epsilon)
  =
  \sum_{x \in [|f_n - f| > \epsilon]} 2^{-x}
\end{align*}

\begin{itemize}

  \item[\Quote{$\Rightarrow$}:] Angenommen,
  \begin{align*}
    \Exists x \in \Omega:
    \Exists \epsilon > 0:
    \Forall N \in \N:
    \Exists n \geq N:
    |f_n(x) - f(x)| > \epsilon,
  \end{align*}
  dann muss $[|f_n - f| > \epsilon] \neq \emptyset$, also $\mu(|f_n - f| > \epsilon) \neq 0$ und somit $\lim_{n \to \infty} \mu(|f_n - f| > \epsilon) \neq 0$.

  \item[\Quote{$\Leftarrow$}:] Angenommen,
  \begin{align*}
    \Exists \epsilon > 0:
    \Forall N \in \N:
    \Exists n \leq N:
    \mu(|f_n - f| > \epsilon) \neq 0,
  \end{align*}
  dann muss $[|f_n - f| > \epsilon] \neq 0$, also $\Exists x \in \Omega: |f_n - f| > \epsilon$.

\end{itemize}

--------------------------------------------------------------------------------

\begin{exercise}

Es sei $\Omega = \N$, $\mathfrak{S} = 2^\N$, $\mu(A) = |A|$. Wann konvergiert die Funktionenfolge $f_n$ im Maßraum $(\Omega, \mathfrak{S}, \mu)$

\begin{itemize}
  \item[(a)] fast überall
  \item[(b)] fast gleichmäßig
  \item[(c)] im Maß?
\end{itemize}

\end{exercise}

\begin{solution}

Man beachte, dass $\Forall A \in 2^\N: |A| = 0 \Leftarrow A = \emptyset$, d.h. $\emptyset$ ist die einzige Nullmenge. Somit gilt eine Aussage genau dann fast überall, wenn sie auf $\emptyset^\complement = \N$ gilt, also überall.

(a)

\begin{align*}
  f_n \xrightarrow[n \to \infty]{\text{f.ü.}} f
  \Leftarrow
  f_n \xrightarrow[n \to \infty]{\text{punktw.}} f
\end{align*}

(b)

\begin{align*}
  f_n \xrightarrow[n \to \infty]{\text{fast glm.}} f
  \Leftarrow
  f_n \xrightarrow[n \to \infty]{\text{glm.}} f
\end{align*}

(c)

\end{solution}

--------------------------------------------------------------------------------

\begin{exercise}

$f: \R \to \R$ sei überall differenzierbar. Zeigen Sie, dass $f^\prime$ Borel-messbar ist.

\end{exercise}

\begin{solution}

\begin{align*}
  \Forall n \in \N:
  f_n: x \mapsto \frac{f(x + 1/n) - f(x)}{1/n},
  \enspace \text{messb.}
  \Rightarrow
  f^\prime = \lim_{n \to \infty} f_n
  \enspace \text{messb.}
\end{align*}

\end{solution}

--------------------------------------------------------------------------------

\begin{exercise}

\begin{itemize}
  \item[(a)] Definieren Sie: messbare Funktion, Treppenfunktion, Konvergenz im Maß, Konvergenz fast überall, Konvergenz fast gleichmäßig.
  \item[(b)] Formulieren und beweisen Sie den Approximationssatz für reellwertige messbare Funktionen.
\end{itemize}

\end{exercise}

\begin{solution}

(a) $f: (\Omega_1, \mathfrak{S}_1) \to (\Omega_2, \mathfrak{S}_2) \enspace \text{Treppenfunktion}
: \Leftrightarrow
\Exists a_1, \ldots, a_n \in \Omega_2,
\Exists A_1, \ldots, A_n \in \mathfrak{S}_1:$

\begin{align*}
  \sum_{i=1}^n A_i = \Omega_1, \enspace
  \sum_{i=1}^n a_i A_i = f
\end{align*}

Rest siehe Aufgabe 1 und 12 (a).

(b) Siehe Skript.

\end{solution}

--------------------------------------------------------------------------------

\begin{exercise}

Hier könnte Ihre Werbung stehen!

\begin{itemize}
  \item Definieren Sie: messbare Funktion, Treppenfunktion, Konvergenz im Maß, Konvergenz fast überall, Konvergenz fast gleichmäßig.
  \item In welchem Sinn (fast überall gleichmäßig/fast gleichmäßig/fast überall/im Maß) konvergieren die folgenden Folgen in $(\R, \mathfrak{B}, \lambda)$?
  \begin{itemize}
    \item[i.] $f_n(x) = \sin(x)/n$
    \item[ii.] $f_n(x) = e^{-n |x|}$
    \item[iii.] $f_n(x) = x/n$
    \item[iv.] $f_n(x) = f_n(x) =
    \begin{cases}
      1 & \text{wenn} \enspace \sqrt{n} - \floor{\sqrt{n}} \leq x \leq \sqrt{n+1} - \floor{\sqrt{n}} \\
      0 & \text{sonst}.
    \end{cases}$
  \end{itemize}
\end{itemize}

\end{exercise}

\begin{solution}

(a) Siehe Aufgabe 7 (a) \\

(b) Hier könnte Ihre Werbung stehen!

\begin{itemize}

  \item[i.]
  \begin{align*}
     \norm[\infty]{f_n}
     =
     \sup_{x \in \R} \vbraces{\sin(x)/n}
     =
     1/n
     \xrightarrow[n \to \infty]{} 0 \\
     \Rightarrow
     f_n \xrightarrow[n \to \infty]{\text{glm.}} 0
     \Rightarrow
     f_n \xrightarrow[n \to \infty]{\text{fast glm.}} 0
     \Rightarrow
     f_n \xrightarrow[n \to \infty]{\text{f.ü.}} 0,
     \enspace
     f_n \xrightarrow[n \to \infty]{\text{im Maß}} 0
   \end{align*}

  \item[ii.]
  \begin{align*}
    \Forall A \subseteq \R \setminus \Bbraces{0}:
    \norm[\infty]{f_n|_A}
    =
    \sup_{x \in A} \vbraces{e^{-n |x|}}
    =
    e^{-n \inf_{x \in A} |x|}
    \xrightarrow[n \to \infty]{} 0 \\
    \Rightarrow
    f_n \xrightarrow[n \to \infty]{\text{fast glm.}} 0
    \Rightarrow
    f_n \xrightarrow[n \to \infty]{\text{f.ü.}} 0,
    \enspace
    f_n \xrightarrow[n \to \infty]{\text{im Maß}} 0
  \end{align*}
  Sei $N \in \mathfrak{B}$, mit $\lambda(N) = 0$, dann gilt trotzdem noch $\Forall \epsilon > 0: \vbraces{B(0, \epsilon) \cap N^\complement} = \infty$, also konvergiert $(f_n)$ nicht fast überall gleichmäßig.

  \item[iii.]
  \begin{align*}
    f_n \xrightarrow[n \to \infty]{\text{punktw.}} 0
    \Rightarrow
    f_n \xrightarrow[n \to \infty]{\text{f.ü.}} 0
  \end{align*}
  $\Forall \epsilon > 0, \Forall n \in \N:$
  \begin{align*}
    \lambda(x/n > \epsilon)
    =
    \lambda(x > \epsilon n)
    =
    \lambda(]\epsilon n, \infty])
    =
    \infty
  \end{align*}
  Also konvergiert $(f_n)$ nicht im Maß. \\
  Wenn $(f_n)$ fast (überall) gleichmäßig konvergiert, dann offensichtlich gegen $0$. Aber $\Forall \epsilon > 0, \Forall N \in \mathfrak{B}:$
  \begin{align*}
    \lambda(N) < \epsilon
    \Rightarrow
    \Forall n \in \N:
    \norm[\infty]{f_n|_{N^\complement}}
    =
    \sup_{x \in N^\complement} |x|/n
    =
    \infty
  \end{align*}
  Also konvergiert $(f_n)$ weder fast gleichmäßig, noch fast überall gleichmäßig. \\

  \item[iv.] $\Forall \epsilon > 0:$
  \begin{align*}
    \lambda(|f_n| > \epsilon)
    =
    \lambda(f_n = 1)
    =
    \lambda
    ([\sqrt{n} - \floor{\sqrt{n}}, \sqrt{n+1} - \floor{\sqrt{n}}])
    =
    \sqrt{n+1} - \sqrt{n}
    \xrightarrow[n \to \infty]{} 0 \\
    \Rightarrow
    f_n \xrightarrow[n \to \infty]{\text{im Maß}} 0
  \end{align*}
  $\Forall q \in \N^2:$
  \begin{align*}
    ]0, 1]
    =
    \sum_{i = q}^{(\sqrt{q}+1)^2-1}
    \left ]
    \sqrt{i} - \floor{\sqrt{i}}, \sqrt{i+1} - \floor{\sqrt{i}}
    \right ]
  \end{align*}
  Also, muss $\Forall x \in \: ]0, 1], \Forall N \in \N: \Exists n, m \geq N:$
  \begin{align*}
    |f_n(x) - f_m(x)| = 1,
  \end{align*}
  und es konvergiert $(f_n)$ nicht fast überall und somit auch weder fast gleichmäßig, noch fast überall gleichmäßig.

\end{itemize}

\end{solution}

--------------------------------------------------------------------------------

\begin{exercise}

Hier könnte Ihre Werbung stehen!

\begin{itemize}
  \item Definieren Sie: messbare Funktion, Treppenfunktion, Konvergenz im Maß, Konvergenz fast überall, Konvergenz fast gleichmäßig.
  \item In welchem Sinn (fast überall gleichmäßig/fast gleichmäßig/fast überall/im Maß) konvergieren die folgenden Folgen in $(\R, \mathfrak{B}, \lambda)$?
  \begin{itemize}
    \item[i.] $f_n(x) = \sin(x)/n$
    \item[ii.] $f_n(x) = e^{-n |x|}$
    \item[iii.] $f_n(x) = x/n$
    \item[iv.] $f_n(x) = f_n(x) =
    \begin{cases}
      1 & \text{wenn} \enspace \sqrt{n} - \floor{\sqrt{n}} \leq x \leq \sqrt{n+1} - \floor{\sqrt{n}} \\
      0 & \text{sonst}.
    \end{cases}$
  \end{itemize}
\end{itemize}

\end{exercise}

\begin{solution}

Siehe Aufgabe 8.

\end{solution}

--------------------------------------------------------------------------------

\begin{exercise}

Hier könnte Ihre Werbung stehen!

\begin{itemize}
  \item[(a)] Definieren Sie: messbare Funktion, Treppenfunktion, Konvergenz im Maß, Konvergenz fast überall, Konvergenz fast gleichmäßig.
  \item[(b)] $f_n$, $n \in \N$ und $f$ seien reellwertige messbare Funktionen auf dem Maßraum $(\Omega, \mathfrak{S}, \mu)$. Zeigen Sie:
  \begin{itemize}
    \item[i.] Wenn $f_n \to f$ fast überall und $g: \R \to \R$ stetig ist, dann $g \circ f_n \to g \circ f$ fast überall.
    \item[ii.] Wenn $f_n \to f$ im Maß und $g: \R \to \R$ gleichmäßig stetig ist, dann $g \circ f_n \to g \circ f$ im Maß.
    \item[iii.] Geben Sie ein Beispiel einer Folge $f_n$ und einer stetigen Funktion $g$, sodass $f_n$ im Maß konvergiert, aber nicht $g \circ f_n$.
  \end{itemize}
\end{itemize}

\end{exercise}

\begin{solution}

(a) Siehe Aufgabe 1 (a) und Kapitel 4 Aufgabe 7 (a). \\

(b) i. Wegen der Konvergenz fast überall von $(f_n)$, gilt

\begin{align*}
  \Exists N \in \mathfrak{S}:
  \mu(N) = 0,
  \Forall \omega \in N^\complement:
  \Forall \delta > 0:
  \Exists n_0 \in \N:
  \Forall n \geq n_0:
  |f_n(\omega) - f(\omega)| < \delta.
\end{align*}

Wegen der Stetigkeit von $g$, gilt

\begin{align*}
  \Forall x \in \R:
  \Exists \delta > 0:
  \Forall y \in \R:
  |x - y| < \delta
  \Rightarrow
  |g(x) - g(y)| < \epsilon.
\end{align*}

Damit folgt die Konvergenz fast überall von $(g \circ f_n)$ ...

\begin{align*}
  \Forall \omega \in N^\complement:
  \Forall \epsilon > 0:
  \Exists n_0^\prime \in \N:
  \Forall n \geq n_0^\prime:
  |g(f_n(\omega)) - g(f(\omega))| < \epsilon
\end{align*}

ii. Wegen der Konvergenz im Maß von $(f_n)$, gilt

\begin{align*}
  \Forall \delta > 0:
  \mu(|f_n - f| > \delta)
  \xrightarrow[n \to \infty]{} 0.
\end{align*}

Wegen der gleichmäßigen Stetigkeit von $g$, gilt

\begin{align*}
  \Forall \epsilon > 0:
  \Exists \delta > 0:
  \Forall x, y \in \R:
  |x - y| \leq \delta
  \Rightarrow
  |g(x) - g(y)| \leq \epsilon
\end{align*}

Damit folgt die Konvergenz im Maß von $(g \circ f_n)$ ...

\begin{align*}
  \Forall \epsilon > 0:
  \mu(|g \circ f_n - g \circ f| > \epsilon)
  \leq
  \mu(|f_n - f| > \delta)
  \xrightarrow[n \to \infty]{} 0
\end{align*}

\end{solution}

--------------------------------------------------------------------------------

\begin{exercise}

Hier könnte Ihre Werbung stehen!

\begin{itemize}
  \item[(a)] Definieren Sie: Konvergenz fast überall, fast überall gleichmäßig, fast gleichmäßig, im Maß.
  \item[(b)]  Gegeben ist der Maßraum $(\N, 2^\N, \mu)$ mit $\mu(\Bbraces{x}) = 2^{-x}$, $x \in \N$. Zeigen Sie, dass in diesem Maßraum die Konvergenzen fast überall, fast gleichmäßig und im Maß äquivalent sind.
\end{itemize}

\end{exercise}

\begin{solution}

(a) Siehe Aufgabe 1. \\

(b) Hier könnte Ihre Werbung stehen!

\begin{itemize}

  \item \Quote{f.ü. $\Rightarrow$ fast glm.}: $\mu$ ist ein endliches Maß, weil
  \begin{align*}
    \mu(\N)
    =
    \sum_{x \in \N} \mu(\Bbraces{x})
    =
    \sum_{x \in \N} 2^{-x}
    =
    2 < \infty.
  \end{align*}
  Also gilt, laut \Quote{Egorov},
  \begin{align*}
    f_n \xrightarrow[n \to \infty]{\text{f.ü.}} f
    \Rightarrow
    f_n \xrightarrow[n \to \infty]{\text{fast glm.}} f.
  \end{align*}

  \item \Quote{fast glm. $\Rightarrow$ im Maß}: Zudem, gilt immer
  \begin{align*}
    f_n \xrightarrow[n \to \infty]{\text{fast glm.}} f
    \Rightarrow
    f_n \xrightarrow[n \to \infty]{\text{im Maß}} f.
  \end{align*}

  \item \Quote{im Maß $\Rightarrow$ f.ü.}:
  \begin{align*}
    f_n \xrightarrow[n \to \infty]{\text{im Maß}} f
  \end{align*}
  heißt, dass $\Forall \epsilon > 0:$
  \begin{align*}
    \sum_{x \in [|f_n - f| > \epsilon]} 2^{-x}
    =
    \mu(|f_n - f| > \epsilon)
    \xrightarrow[n \to \infty]{} 0,
  \end{align*}
  also gilt $\Forall x \in \N: \Exists N \in \N: \Forall n \geq N:$
  \begin{align*}
    x \notin [|f_n - f| > \epsilon]
    \Leftrightarrow
    |f_n(x) - f(x)| \leq \epsilon,
  \end{align*}
  und somit schließlich
  \begin{align*}
    f_n \xrightarrow[n \to \infty]{\text{f.ü.}} f.
  \end{align*}

\end{itemize}

\end{solution}

--------------------------------------------------------------------------------

\begin{exercise}

Hier könnte Ihre Werbung stehen!

\begin{itemize}
  \item[(a)] Definieren Sie messbare Funktion, maßtreue Funktion.
  \item[(b)] Auf $\Omega = \N$ ist die Funktion
  \begin{align*}
    f(x) = x^2 - 3x
  \end{align*}
  gegeben. Bestimmen Sie die kleinste Sigmaalgebra $\mathfrak{S}$ über $\Omega$, für die $f$ $\mathfrak{S}$-messbar ist.
\end{itemize}

\end{exercise}

\begin{solution}

(a) Hier könnte Ihre Werbung stehen!

\begin{itemize}

  \item $f: (\Omega_1, \mathfrak{S}_1) \to (\Omega_2, \mathfrak{S}_2) : \Leftrightarrow f^{-1}(\mathfrak{S}_2) \subseteq \mathfrak{S}_1$

  \item $f: (\Omega_1, \mathfrak{S}_1, \mu_1) \to (\Omega_2, \mathfrak{S}_2, \mu_2) \enspace \text{maßtreu} : \Leftrightarrow \mu_1 \circ f^{-1} = \mu_2$

\end{itemize}

(b) Das wäre dann die initiale Sigmaalgebra

\begin{align*}
  \mathfrak{S}
  =
  f^{-1}(\mathfrak{B})
  =
  \Bbraces{\Bbraces{n \in \N: n^2 - 3n \in A}: A \in \mathfrak{B}}.
\end{align*}

zz: $\mathfrak{S} = \mathfrak{S}^\prime := \Bbraces{B \in 2^\N: 1 \in B \Leftrightarrow 2 \in B}$

\begin{itemize}

  \item[\Quote{$\subseteq$}:] $f$ ist bis auf $1, 2$ injektiv.

  \item[\Quote{$\supseteq$}:] $\Forall x \in \R: \Bbraces{x} \in \mathfrak{B}$

\end{itemize}

\end{solution}

--------------------------------------------------------------------------------

\begin{exercise}

Hier könnte Ihre Werbung stehen!

\begin{itemize}
  \item[(a)] Definieren Sie Konvergenz im Maß, fast überall, fast gleichmäßig, fast überall gleichmäßig.
  \item[(b)] $(X_n)$ sei eine Folge von unabhängigen Zufallsvariablen. Zeigen Sie, dass genau dann fast sicher
  \begin{align*}
    \lim_{n \to \infty} X_n = 0
  \end{align*}
  gilt, wenn für jedes $\epsilon > 0$
  \begin{align*}
    \sum_{n \in \N} \P(|X_n| > \epsilon) < \infty.
  \end{align*}
\end{itemize}

\end{exercise}

\begin{solution}

(a) Siehe Aufgabe 1. \\

(b) Hier könnte Ihre Werbung stehen!

\begin{itemize}

  \item[\Quote{$\Rightarrow$}:] Angenommen, $\Exists \epsilon > 0:$
  \begin{align*}
    \sum_{n \in \N} \P(|X_n| > \epsilon) = \infty,
  \end{align*}
  dann gilt laut dem \Quote{zweiten Lemma von Borel-Cantelli}, dass
  \begin{align*}
    \P(\limsup_{n \in \N} [|X_n| > \epsilon]) = 1.
  \end{align*}
  $\limsup_{n \in \N} [|X_n| > \epsilon]$ ist dabei die Menge aller Punkte, die in unendlich vielen $[|X_n| > \epsilon]$ enthalten ist.

  \item[\Quote{$\Leftarrow$}:]
  $1 - \P(|X_n| \leq \epsilon)
  =
  \P(|X_n| > \epsilon)
  \xrightarrow[n \to \infty]{} 0$

\end{itemize}

\end{solution}

--------------------------------------------------------------------------------

\setcounter{exercise}{0}

\section{Das Integral}

--------------------------------------------------------------------------------

\begin{algebraUE}{141}
In dieser Übungsaufgabe interessieren wir uns für Unteralgebren und Kongruenzrelationen
auf $\mathbb{N}$ bezüglich additiver und/oder multiplikativer Struktur. Versuchen
Sie jeweils alle Objekte der angegebenen Art zu beschreiben. Wenn Ihnen das zu
schwierig erscheint (was in der Mehrzahl der Fälle wahrscheinlich ist), ermitteln
Sie, wieviele es davon gibt. Unterscheiden Sie dabei verschiedene unendliche
Kardinalitäten, insbesondere $|\mathbb{N}|$ und $|\mathbb{R}|$.
\begin{itemize}
  \item [1.] Unteralgebren von $(\mathbb{N},+,0)$
  \item [2.] Kongruenzrelationen von $(\mathbb{N},+,0)$
  \item [3.] Unteralgebren von $(\mathbb{N},\cdot,1)$
  \item [4.] Kongruenzrelationen von $(\mathbb{N},\cdot,1)$
  \item [5.] Unteralgebren von $(\mathbb{N},+,0,\cdot,1)$
  \item [6.] Kongruenzrelationen von $(\mathbb{N},+,0,\cdot,1)$
\end{itemize}
\end{algebraUE}
\begin{solution}
\leavevmode \\
\begin{itemize}
  \item [1.] Bezeichne mit $\mathcal{U}$ die Menge aller Unteralgebren
  von $(\mathbb{N},+,0)$. \\
  Einige Beispiele an Unteralgebren lauten
  \begin{itemize}
    \item Die beiden trivialen Unteralgebren: $\{0\}$ und $\mathbb{N}$.
    \item Für jede natürliche Zahl $n \geq 2: n\cdot\mathbb{N}$.
    \item Für jede natürliche Zahl $n \geq 1: \mathbb{N}~\backslash\{1,\dots,n\}$.
  \end{itemize}
Im allgemeinen können wir zu jeder beliebiger Teilmenge $X \subset \mathbb{N}$
die erzeugte Unteralgebra mit $\{\sum_{j=1}^nx_jn_j: n, n_j \in \mathbb{N}, x_j \in X \}$
bestimmen. Also haben wir eine Abbildung
\begin{align*}
  \varphi: \begin{cases}
    2^\N \rightarrow \mathcal{U} \\
    X \mapsto \{\sum_{j=1}^nx_jn_j: n, n_j \in \mathbb{N}, x_j \in X \}
  \end{cases}
\end{align*}
gefunden, welche $2^\N$ surjektiv auf $\mathcal{U}$ abbildet.
Damit haben wir ein System um alle Unteralgebren vollständig zu beschreiben.
Natürlich werden wir damit öfters die gleichen Unteralgebren erzeugen,
allerdings haben wir mit $|2^{\mathbb{N}}| = |\mathbb{R}|$ zumindest eine obere
Schranke für die Kardinalität der Unteralgebren gefunden. Bleibt noch die Frage,
ob wir die Schranke noch auf abzählbar unendlich herunterschrauben können. \\
Dafür zeigen wir zunächst, dass jede Unteralgebra endlich erzeugt ist.
Sei $U \in \mathcal{U}$ beliebig und sei angenommen,
dass $U$ nicht endlich erzeugt ist.
Dazu definieren wir uns die Folge $(a_n)_{n \in \mathbb{N}}$ induktiv wie folgt
\begin{align*}
  a_0 := \min U \\
a_{n+1} := \min U \backslash \langle a_0,\dots,a_n\rangle
\end{align*}
Betrachte nun die endliche Menge $(a_0,\dots,a_m)$ mit einem $m > a_0$.
Nun folgt mit dem Schubfachprinzip $\exists k,n \leq m: a_k \equiv a_n (\mod a_0)$.
Wir nehmen o.B.d.A. an $a_k < a_n$. Dann gilt also
\begin{align*}
  \exists c \in \mathbb{N}: a_n = a_k + ca_0 \in \langle a_0,\dots,a_m \rangle
\end{align*}
Dies ist ein Widerspruch zu unserer Konstruktion und $U$ muss somit endlich erzeugt sein.
Also können wir $\mathcal{U}$ als Bild einer nicht notwendigerweise injektiven
Abbildung von der Menge $\mathcal{E}(\N)$ aller endlichen Teilmengen von $\N$ darstellen.
Die Menge $\mathcal{E}(\N)$ können wir als
abzählbare Vereinigung endlicher Mengensysteme anschreiben
\begin{align*}
  \mathcal{E}(\N) = \bigcup_{n \in \N} \{A \subset 2^\N: |A| = n \}
\end{align*}
und ist damit ebenfalls abzählbar. Also gilt $|\mathcal{U}| \leq |\N|$.
\item [2.] Wieder haben wir zwei triviale Kongruenzrelationen am Anfang:
Die Allrelation $\mathbb{N} \times \mathbb{N}$ und die Identitätsrelation
$\{(n,n): n \in \mathbb{N}\}$. \\

Sei nun $\sim$ eine beliebige Kongruenzrelation ungleich der Identitätsrelation.
Dann gibt es ein kleinstes $a_{0}$, das mit mindestens einem anderen Element in Relation steht. In der Menge aller $b$, die mit $a_{0}$ in
Relation stehen, gibt es wieder ein kleinstes Element, das wir $b_{0}$ nennen. Da $\sim$ eine Kongruenzrelation bezüglich $+$ ist und $1
\sim 1$, gilt (mit $m := b_{0}-a_{0} > 0$)
\begin{align*}
  a_{0} \sim b_{0} \Rightarrow a_{0}+1 \sim b_{0}+1 \Rightarrow \dots \Rightarrow a_{0}+(m-1) \sim b_{0}+(m-1) \Rightarrow b_{0} \sim
  b_{0}+m \Rightarrow \dots
\end{align*}
Zusammengefasst gilt also
\begin{align*}
  \forall n \geq a_0: n \sim n + m
\end{align*}
Also ist $\sim$ eine Kongruenzrelation mit $a_{0}$ einelementigen Äquivalenzklassen und maximal $m$ weiteren Quasi-Modulo-Äquivalenzklassen
(beginnend ab $a_{0}$).
Für jedes $n \in \N,a_0 \leq n < b_0$ ist also
\begin{align*}
  [n]_{\sim} \supseteq \{n + km: k \in \N\}.
\end{align*}
Angenomenn es gäbe zusätzlich $a_1 \sim a_1 + m^{\prime}$ mit $m^{\prime} \neq m \in \N$. \\
\begin{itemize}
\item Fall 1: $a_1 < a_0$:
Widerspruch zur Annahme, dass $a_0$ das kleinste Element ist, welches nicht nur
mit sich selbst in Relation steht.
  \item Fall 2: $a_1 \geq a_0, m^{\prime} < m$:
  Es existiert ein kleinstes $k \in \N$, sodass $a_0 + km \geq a_1$.
  Es folgt $a_0 + km \sim a_0 + km + m^{\prime} \sim a_0 + m^{\prime}$
  im Widerspruch dazu, dass $b_0$ das kleinste Element ungleich $a_0$ ist, welches mit
  $a_0$ in Relation steht.
  \item Fall 3: $a_1 \geq a_0, m^{\prime} > m$:
  Es gibt ein kleinstes $k \in \N: mk < m^{\prime}$.
  Es folgt $a_1 + mk \sim a_1 \sim a_1 + m^{\prime}$ und wir erhalten einen
  Widerpruch mittels Fall 1 angewandt auf das Paar $(a_1 + mk, a_1 + m^{\prime})$.
\end{itemize}
Insgesamt gilt also, dass jede Kongruenzrelation auf $(\mathbb{N},+,0)$ einer Partition
der Form
\begin{align*}
  P_{(a_0,m)} := \{\{1\},\dots,\{a_0 - 1\},\{a_0 + km: k \in \N\},\dots,\{a_0 + m - 1 + km: k \in \N\}\}
\end{align*}
mit $a_0,m \in N$ entspricht. Damit haben wir eine bijektive Abbildung
\begin{align*}
  \varphi: \begin{cases}
    \N^2 \rightarrow \mathcal{K} \\
    (a_0,m) \mapsto P_{(a_0,m)}
  \end{cases}
\end{align*}
gefunden und es gilt $|K| = |\N|$. \\
*************************** (andere Argumentation) \\
Dass Gleichheit gilt, erkennt man im Folgenden. Sei angenommen, es gibt zwei Zahlen $n_1 < n_2$ zwischen $a_0$ und $b_0$ (also $n_1 = a_0 + d_1$ und $n_2 = a_0 + d_2$ mit $d_1,d_2 \in {0,...,m-1}$), die in Relation stehen.

Dann folgt (wiederum durch induktives Anwenden der Kongruenz bzgl. $+$ und $1 \sim 1$), dass
\begin{align*}
n_1 + (m-d_2) = \underbrace{a_0 + m}_{\sim a_0} \sim a_0 + \underbrace{(m + d_1 - d_2)}_{<m} = n_2 + (m-d_2),
\end{align*}
was ein Widerspruch zur Minimalität von $b$ wäre.\\
********************* \\
Umgekehrt kann man zeigen, dass für alle $a_0 \in \N$ und $b_0 > a_0$ diese Konstruktion eine Kongruenzrelation ergibt. Die Äquivalenzklassen sind wie oben festgelegt, die Verträglichkeit mit der Addition zeigt man durch Fallunterscheidung.
Wir wollen zeigen, dass für alle $x_1,x_2,y_1,y_2 \in \N$ gilt:
\begin{align*}
  x_1 \sim y_1,x_2 \sim y_2 \Rightarrow x_1 + x_2 \sim y_1 + y_2
\end{align*}

\begin{enumerate}[label = \textit{\arabic*.}]
\item Fall ($x_1 < a_0, x_2 < a_0$):
\begin{align*}
  &\Rightarrow y_1 = x_1, y_2 = x_2 \\
  &\Rightarrow x_1 + x_2 = y_1 + y_2
\end{align*}
\item Fall ($x_1 < a_0, x_2 \geq a_0$):
\begin{align*}
  &\Rightarrow y_1 = x_1, y_2 = x_2 + nm \\
  &\Rightarrow  y_1 + y_2 = x_1 + x_2 + nm \sim x_1 + x_2
\end{align*}
\item Fall ($x_1 \geq a_0, x_2 \geq a_0$):
\begin{align*}
  &\Rightarrow y_1 = x_1 + n_1 m, y_2 = x_2 + n_2 m \\
  &\Rightarrow  y_1 + y_2 = x_1 + x_2 + (n_1+n_2) m \sim x_1 + x_2
\end{align*}
\end{enumerate}

Die Kongruenzrelationen von $(\mathbb{N},+,0)$ sind also genau festgelegt durch ein beliebiges $a_0 \in \N$ und $b_0 > a_0$. Insgesamt gibt es also abzählbar unendlich viele.

Insbesondere haben wir (für $a_0 = 0$) die Modulo-Kongruenzrelationen. Für jede natürliche Zahl
$n \geq 1$ ist $a \sim b: \iff a \equiv b \mod n$ eine Kongruenzrelation, welche
die Partition in die jeweiligen Restklassen induziert.
\item [3.] Jede Teilmenge $A$ der Potenzmenge der Primzahlen induziert eine
Unteralgebra von $(\mathbb{N},\cdot,1)$ mit
\begin{align*}
  U_A := \left\{\prod_{j=1}^na_j^{n_j}: n, n_j \in \mathbb{N}, a_j \in A\right\}.
\end{align*}
Diese Unteralgebren sind aufgrund der Eindeutigkeit
der Primfaktorzerlegung allesamt paarweise verschieden.
Damit können wir zwar noch nicht alle möglichen Unteralgebren
vollständig beschreiben, allerdings ist aufgrund $|2^\mathbb{P}| = |2^\mathbb{N}|
= |\mathbb{R}|$ die Kardinalität der Unteralgebren auf jeden
Fall überabzählbar. \\
Sei $X \subset \mathbb{N}$ beliebig.
Dann ist die Menge
\begin{align*}
  U_X := \left\{\prod_{j=1}^nx_j^{n_j}: n, n_j \in \mathbb{N}, x_j \in X\right\}.
\end{align*}
eine Unteralgebra und offensichtlicherweise können damit auch alle möglichen
Unteralgebren beschrieben werden, allerdings ist diese Darstellung im Allgemeinen
nicht mehr eindeutig.
Also haben wir sowohl eine injektive, als auch eine surjektive Abbildung von
einer Teilmenge der Potenzmenge der natürlichen Zahlen in die Menge aller
Unteralgebren von $(\mathbb{N},\cdot,1)$ gefunden. \\
Ich denke, dass es nicht möglich ist, eine elementare bijektive Abbildung zu finden.
\item [4.] Wieder gibt es die beiden trivialen Relationen: Allrelation + Identitätsrelation.
Weiters ist auch die Modulo 2 Relation eine Kongruenzrelation.
Wiederum eine weitere Kongruenzrelation ist durch
\begin{align*}
  a = \prod_{p \in \mathbb{P}}p^{a_p} \sim b = \prod_{p \in \mathbb{P}}p^{b_p}: \iff \sum_{p \in \mathbb{P}} a_p = \sum_{p \in \mathbb{P}} b_p
\end{align*}
definiert.
\item [5.] Jede Unteralgebra muss auf jeden Fall $\{0,1\}$ enthalten. Aufgrund
der Abgeschlossenheit unter der Addition und dem Induktionsprinzip
muss jede Unteralgebra bereits ganz $\mathbb{N}$ enthalten.
\item [6.] Wieder gibt es zwei triviale Relationen mit der Allrelation und der
Identitätsrelation.

Es kommen nur die in Punkt 2. beschriebenen Kongruenzrelationen in Frage. Tatsächlich sind sogar alle davon auch mit der Multiplikation verträglich, wie sich wiederum durch Unterscheidung in die 3 Fälle zeigt:

Wir wollen zeigen, dass für alle $x_1,x_2,y_1,y_2 \in \N$ gilt:
\begin{align*}
  x_1 \sim y_1,x_2 \sim y_2 \Rightarrow x_1 \cdot x_2 \sim y_1 \cdot y_2
\end{align*}

\begin{enumerate}[label = \textit{\arabic*.}]
\item Fall ($x_1 < a_0, x_2 < a_0$):
\begin{align*}
  &\Rightarrow y_1 = x_1, y_2 = x_2 \\
  &\Rightarrow x_1 \cdot x_2 = y_1 \cdot y_2
\end{align*}
\item Fall ($x_1 < a_0, x_2 \geq a_0$):
\begin{align*}
  &\Rightarrow y_1 = x_1, y_2 = x_2 + nm \\
  &\Rightarrow  y_1 \cdot y_2 = x_1 \cdot x_2 + (nx_1)m \sim x_1 \cdot x_2
\end{align*}
\item Fall ($x_1 \geq a_0, x_2 \geq a_0$):
\begin{align*}
  &\Rightarrow y_1 = x_1 + n_1 m, y_2 = x_2 + n_2 m \\
  &\Rightarrow  y_1 \cdot y_2 = x_1 \cdot x_2 + (n_1 x_2 +n_2 x_1 + n_1 n_2 m) m \sim x_1 \cdot x_2
\end{align*}
\end{enumerate}

\end{itemize}
\end{solution}


--------------------------------------------------------------------------------

% --------------------------------------------------------------------------------

\begin{exercise}

Zeigen Sie, dass
\begin{align*}
  F = \frac{1}{\sigma_n}\frac{x}{|x|^n}
\end{align*}
eine Fundamentallösung des Differentialoperators $L(u) = \mathrm{div}(u)$ auf $\R^n$ ist,
wobei $\sigma_n$ die Oberfläche der Einheitskugel im $\R^n$ ist. Achtung:
Obwohl $F$ eigentlich eine vektorwertige Distribution in $L_{\mathrm{loc}}^1(\R^n)^n$ ist,
wird das nicht gebraucht um die Behauptung
\begin{align*}
  \langle \mathrm{div} F, \varphi \rangle = \varphi(0)
\end{align*}
zu zeigen, da
$\mathrm{div} F = \frac{1}{\sigma_n}\sum_{i}\partial_i\left(\frac{x_i}{|x|^n}\right) \in \mathcal{D}^{\prime}(\R)$
ist.
\end{exercise}

% --------------------------------------------------------------------------------

\begin{solution}
Eine Fundamentallösung von $L(u) = \mathrm{div}(u)$ mit Pol in $\xi$ ist definitionsgemäß
eine distributionelle Lösung von $L(U_{\xi}) = \delta_{\xi}$. Wir berechnen also
für $\phi \in \mathcal{D}(\R^n)$ beliebig und $A$ eine offene, beschränkte Obermenge
von $\supp(\phi)$ mit $C^1$-Rand:
\begin{align*}
  \langle L F, \phi \rangle
  &= \left\langle \frac{1}{\sigma_n}\sum_{i = 1}^n\partial_i\left(\frac{x_i}{|x|^n}\right), \phi \right\rangle
  = -\frac{1}{\sigma_n}\sum_{i = 1}^n\left\langle \left(\frac{x_i}{|x|^n}\right), \partial_i\phi \right\rangle \\
  &= -\frac{1}{\sigma_n}\sum_{i = 1}^n\int_{\R^n}\left(\frac{x_i}{|x|^n}\right)\partial_i\phi(x)dx
  = \lim_{\epsilon \to 0^+}-\frac{1}{\sigma_n}\int_{A\backslash\overline{B_{\epsilon}(0)}}
  \left(\frac{x}{|x|^n}\right)\nabla \phi(x)dx \\
  &= \lim_{\epsilon \to 0^+}\frac{1}{\sigma_n}\left(\int_{A\backslash\overline{B_{\epsilon}(0)}}\mathrm{div}\left(\frac{x}{|x|^n}\right) \phi(x)dx
  - \int_{\partial A}\frac{x}{|x|^n}\underbrace{\phi(x)}_{=0}\nu ds
  - \int_{\partial B_{\epsilon}(0)}\frac{x}{|x|^n}\phi(x)\nu ds\right) \\
  &= \lim_{\epsilon \to 0^+}\frac{1}{\sigma_n}\left(\int_{A\backslash\overline{B_{\epsilon}(0)}}\mathrm{div}\left(\frac{x}{|x|^n}\right) \phi(x)dx
  + \int_{\partial B_{\epsilon}(0)}\frac{\epsilon^2}{\epsilon^{n+1}}\phi(x) ds\right) \\
  &\stackrel{MWS}{=} \lim_{\epsilon \to 0^+}\frac{1}{\sigma_n}\left(\int_{A\backslash\overline{B_{\epsilon}(0)}}\mathrm{div}\left(\frac{x}{|x|^n}\right) \phi(x)dx
  + \epsilon^{1-n}\epsilon^{n-1}\sigma_n\phi(x_{\epsilon})\right) \\
  &= \lim_{\epsilon \to 0^+}\left(\phi(x_{\epsilon})\right) = \phi(0)\\
\end{align*}
Für $|x| \neq 0$ gilt
\begin{align*}
  \mathrm{div}\left(\frac{x}{|x|^n}\right)
  =\frac{1}{\sigma_n}\sum_{i=1}^n\partial_i\left(\frac{x_i}{|x|^n}\right)
  &= \frac{1}{\sigma_n}\sum_{i=1}^n\frac{\left(\sum_{i=1}^nx_i^2\right)^{n/2} - x_i2x_i\frac{n}{2}\left(\sum_{i=1}^nx_i^2\right)^{n/2 - 1}}{\left(\sum_{i=1}^nx_i^2\right)^n} \\
  &= \frac{1}{\sigma_n}\sum_{i=1}^n\frac{1 - \frac{nx_i^2}{\sum_{i=1}^nx_i^2}}{\left(\sum_{i=1}^nx_i^2\right)^{n/2}} \\
  &= \frac{1}{\sigma_n}\frac{n - n\frac{\sum_{i=1}^nx_i^2}{\sum_{i=1}^nx_i^2}}{\left(\sum_{i=1}^nx_i^2\right)^{n/2}} = 0.
\end{align*}

\end{solution}

% --------------------------------------------------------------------------------


--------------------------------------------------------------------------------

% --------------------------------------------------------------------------------

\begin{exercise}[Central Limit Theorem]

Let $\bar X_1$ and $\bar X_2$ be the means of two independent samples of size $n$ from the same population with variance $\sigma^2$.
Use the Central limit theorem to find a value for $n$ so that

\begin{align*}
    P(|\bar X_1 - \bar X_2| < \frac{\sigma}{50}) \approx 0.99.
\end{align*}

Justify your calculations.

\end{exercise}

% --------------------------------------------------------------------------------

\begin{solution}

ToDo!

\end{solution}

% --------------------------------------------------------------------------------


--------------------------------------------------------------------------------

% --------------------------------------------------------------------------------

\begin{exercise}

$(X_n)$ sei eine Folge von unabhängigen, auf $[0, 1]$ gleichverteilten Zufallsvariablen. Gegen welchen Wert konvergiert

\begin{align*}
  (\prod_{i=1}^n)^{1/n}?
\end{align*}

\end{exercise}

% --------------------------------------------------------------------------------

\begin{solution}

$\ln$ ist messbar, also auf $(\ln X_n)$ unabhängig.

\begin{align*}
  \lim_{n \to \infty}
  \pbraces{\prod_{i=1}^n X_i}^\frac{1}{n}
  & =
  \lim_{n \to \infty}
  \exp \ln \pbraces{\pbraces{\prod_{i=1}^n X_i}^\frac{1}{n}}
  =
  \lim_{n \to \infty}
  \exp \pbraces{\frac{1}{n} \sum_{i=1}^n \ln X_i} \\
  & =
  \exp \pbraces
  {\lim_{n \to \infty} \frac{1}{n} \sum_{i=1}^n \ln X_i}
  =
  \exp \E(\ln X_1)
  =
  \exp \underbrace{\Int[0][1]{\ln(x) \frac{1}{1-0}}{x}}_{-\infty} = 0
\end{align*}

\end{solution}

% --------------------------------------------------------------------------------


--------------------------------------------------------------------------------

% --------------------------------------------------------------------------------

\begin{exercise}

Für ein beschränktes Lipschitz-Gebiet $\Omega \subset \R^2$ definieren wir den Hilbertraum

\begin{align*}
  H(\curl, \Omega)
  :=
  \Bbraces{\xi \in [L^2(\Omega)]^2 | \curl \xi \in L^2(\Omega)}, 
  \quad
  (\xi, \zeta)_{H(\curl)}
  :=
  (\xi, \zeta)_{L^2(\Omega)} + (\curl \xi, \curl \zeta)_{L^2(\Omega)}, 
\end{align*}

mit $\curl \xi := \frac{\partial \xi_2}{\partial x} - \frac{\partial \xi_1}{\partial y}$ für $\xi(x, y) = (\xi_1(x, y), \xi_2(x, y))^\top$.
Weiter sei $X := H^1_0(\Omega) \times H(\curl, \Omega)$ ein Hilbert-Raum wie in Aufgabe 6.

Für ein $c \geq 0$ und $f \in L^2(\Omega)$ sei das folgende Variationsproblem gegeben: Finden Sie $(u, \xi) \in X$ sodass für alle $(v, \zeta) \in X$

\begin{align}
  \Int[\Omega]{(\nabla u - \xi) \cdot (\nabla v - \zeta)}{x}
  +
  c \Int[\Omega]{\xi \cdot \zeta}{x}
  +
  \Int[\Omega]{\curl \xi \curl \zeta}{x}
  =
  \Int[\Omega]{f v}{x}
\end{align}

\begin{enumerate}[label = \textbf{\alph*)}]

  \item Zeigen Sie mit Hilfe des Lemmas von Lax-Milgram, dass für $c > 0$ das Problem eine eindeutige Lösung hat.
  Verwenden Sie dazu am besten die Young Ungleichung $-ab \geq - \frac{\varepsilon}{2} a^2 - \frac{1}{2\varepsilon}b^2$ für geeignete $a, b \in \R$ und $\varepsilon > 0$.

  \item Es sei nun $c = 0$.
  Zeigen Sie durch geschicktes Wählen von $(u, \xi) \in X$, dass das Problem nicht koerziv ist.
  \textit{Hinweis}:
  Was gilt für $\curl \nabla u$?

  \item Begründen Sie mit den Funktionen $\xi_\varepsilon \in [H^1(\Omega)]^2$ definiert durch $\xi_\varepsilon(x, y) := (\sin(\frac{1}{\varepsilon}x), 0)^\top$ mit $\varepsilon > 0$, dass das Problem auf dem Produktraum $\hat{X} := H^1_0(\Omega) \times [H^1(\Omega)]^2$ mit $c > 0$ nicht koerziv und damit nicht wohlgestellt ist.

\end{enumerate}

\end{exercise}

% --------------------------------------------------------------------------------

\begin{solution}

\phantom{}

\begin{align*}
  \pbraces
  {
    \begin{pmatrix}
      u \\ \xi
    \end{pmatrix},
    \begin{pmatrix}
      v \\ \zeta
    \end{pmatrix}
  }_X
  & =
  (u, v)_{H^1(\Omega)}
  +
  (\xi, \zeta)_{H(\curl, \Omega)} \\
  & =
  (u, v)_{L^2(\Omega)}
  +
  (\nabla u, \nabla v)_{L^2(\Omega)}
  +
  (\xi, \zeta)_{L^2(\Omega)}
  +
  (\curl \xi, \curl \zeta)_{L^2(\Omega)}
\end{align*}

\begin{align*}
  \implies
  \norm[X]
  {
    \begin{pmatrix}
      u \\ \xi
    \end{pmatrix}
  }
  & =
  \pbraces
  {
    \begin{pmatrix}
      u \\ \xi
    \end{pmatrix},
    \begin{pmatrix}
      u \\ \xi
    \end{pmatrix}
  }_X^{1/2} \\
  & =
  \pbraces
  {
    (u, u)_{L^2(\Omega)}
    +
    (\nabla u, \nabla u)_{L^2(\Omega)}
    +
    (\xi, \xi)_{L^2(\Omega)}
    +
    (\curl \xi, \curl \xi)_{L^2(\Omega)}
  }^{1/2} \\
  & =
  \pbraces
  {
    \norm[L^2(\Omega)]{u}^2
    +
    \norm[L^2(\Omega)]{\nabla u}^2
    +
    \norm[L^2(\Omega)]{\xi}^2
    +
    \norm[L^2(\Omega)]{\curl \xi}^2
  }^{1/2}
\end{align*}

\begin{enumerate}[label = \textbf{\alph*)}]

  \item \phantom{}

  \begin{center}
    \includegraphics[width = 0.75 \textwidth]{NumPDEs/NumPDEs - Exercise 7.1 (Lemma of Lax-Milgram).png} \\
    \includegraphics[width = 0.75 \textwidth]{NumPDEs/NumPDEs - Exercise 7.2 (Lemma of Lax-Milgram).png}
  \end{center}

  \begin{align*}
    a
    \pbraces
    {
      \begin{pmatrix}
        u \\ \xi
      \end{pmatrix},
      \begin{pmatrix}
        v \\ \zeta
      \end{pmatrix}
    }
    & :=
    \Int[\Omega]{(\nabla u - \xi) \cdot (\nabla v - \zeta)}{x}
    +
    c \Int[\Omega]{\xi \cdot \zeta}{x}
    +
    \Int[\Omega]{\curl \xi \curl \zeta}{x} \\
    F
    \begin{pmatrix}
      v \\ \xi
    \end{pmatrix}
    & :=
    \Int[\Omega]{f v}{x}
  \end{align*}

  \begin{enumerate}[label = \arabic*.]

    \item Stetigkeit von $a$:
    
    Auf dem $\R^3$ sind die Normen $\norm[1]{\cdot}$ und $\norm[2]{\cdot}$ äquivalent.
    Wir erhalten also eine Konstante $C > 0$, sodass $\norm[1]{\cdot} \leq C \norm[2]{\cdot}$.

    \begin{align*}
      \norm[1]{\cdot}
      \sim
      \norm[2]{\cdot}
      ~\text{auf $\R^3$}~
      \implies
      \Exists C > 0:
      \norm[1]{\cdot}
      \leq
      C
      \norm[2]{\cdot}
    \end{align*}

    \begin{align*}
      a
      \pbraces
      {
        \begin{pmatrix}
          u \\ \xi
        \end{pmatrix},
        \begin{pmatrix}
          v \\ \zeta
        \end{pmatrix}
      }
      & =
      \Int[\Omega]{(\nabla u - \xi) \cdot (\nabla v - \zeta)}{x}
      +
      c \Int[\Omega]{\xi \cdot \zeta}{x}
      +
      \Int[\Omega]{\curl \xi \curl \zeta}{x} \\
      & \stackrel
      {
        \mathrm{CSB}
      }{\leq}
      \norm[L^2(\Omega)]{\nabla u - \xi}
      \norm[L^2(\Omega)]{\nabla v - \zeta} \\
      & \quad
      +
      c
      \norm[L^2(\Omega)]{\xi}
      \norm[L^2(\Omega)]{\zeta}
      +
      \norm[L^2(\Omega)]{\curl \xi}
      \norm[L^2(\Omega)]{\curl \zeta} \\
      & \leq
      \pbraces
      {
        \norm[L^2(\Omega)]{\nabla u}
        +
        \norm[L^2(\Omega)]{\xi}
      }
      \pbraces
      {
        \norm[L^2(\Omega)]{\nabla u}
        +
        \norm[L^2(\Omega)]{\zeta}
      } \\
      & \quad
      +
      c
      \norm[L^2(\Omega)]{\xi}
      \norm[L^2(\Omega)]{\zeta}
      +
      \norm[L^2(\Omega)]{\curl \xi}
      \norm[L^2(\Omega)]{\curl \zeta} \\
      & \leq
      2 \max \Bbraces{1, c} \\
      & \quad
      \pbraces
      {
        \norm[H^1(\Omega)]{u}
        +
        \norm[L^2(\Omega)]{\xi}
        +
        \norm[L^2(\Omega)]{\curl \xi}
      } \\
      & \quad
      \pbraces
      {
        \norm[H^1(\Omega)]{v}
        +
        \norm[L^2(\Omega)]{\zeta}
        +
        \norm[L^2(\Omega)]{\curl \zeta}
      } \\
      & =
      2 \max \Bbraces{1, c} \\
      & \quad
      \norm[1]
      {
        \pbraces
        {
          \norm[H^1(\Omega)]{u},
          \norm[L^2(\Omega)]{\xi},
          \norm[L^2(\Omega)]{\curl \xi}
        }^\top
      } \\
      & \quad
      \norm[1]
      {
        \pbraces
        {
          \norm[H^1(\Omega)]{v},
          \norm[L^2(\Omega)]{\zeta},
          \norm[L^2(\Omega)]{\curl \zeta}
        }^\top
      } \\
      & \leq
      2 \max \Bbraces{1, c} C^2 \\
      & \quad
      \norm[2]
      {
        \pbraces
        {
          \norm[H^1(\Omega)]{u},
          \norm[L^2(\Omega)]{\xi},
          \norm[L^2(\Omega)]{\curl \xi}
        }^\top
      } \\
      & \quad
      \norm[2]
      {
        \pbraces
        {
          \norm[H^1(\Omega)]{v},
          \norm[L^2(\Omega)]{\zeta},
          \norm[L^2(\Omega)]{\curl \zeta}
        }^\top
      } \\
      & =
      2 \max \Bbraces{1, c} C^2 \\
      & \quad
      \pbraces
      {
        \norm[H^1(\Omega)]{u}^2
        +
        \norm[L^2(\Omega)]{\xi}^2
        +
        \norm[L^2(\Omega)]{\curl \xi}^2
      }^{1/2} \\
      & \quad
      \pbraces
      {
        \norm[H^1(\Omega)]{v}^2
        +
        \norm[L^2(\Omega)]{\zeta}^2
        +
        \norm[L^2(\Omega)]{\curl \zeta}^2
      }^{1/2} \\
      & =
      2 \max \Bbraces{1, c} C^2
      \norm[X]
      {
        \begin{pmatrix}
          u \\ \xi
        \end{pmatrix}
      }
      \norm[X]
      {
        \begin{pmatrix}
          v \\ \zeta
        \end{pmatrix}
      }
    \end{align*}

    \item Elliptizität von $a$:
    
    \includegraphicsunboxed{PDEs/PDEs_-_Satz_5-11_(Poincare-Ungleichung).png}

    Die Poincaré-Ungleichung von \cite{PDEs} liefert uns ein $C_p > 0:$

    \begin{align*}
      \norm[H^1(\Omega)]{u}^2
      =
      \norm[L^2(\Omega)]{u}^2
      +
      \norm[L^2(\Omega)]{\nabla u}^2
      \leq
      (C_p + 1)^2 \norm[L^2(\Omega)]{\nabla u}^2
      \implies
      \norm[L^2(\Omega)]{\nabla u}
      \geq
      \frac{1}{C_p + 1}
      \norm[H^1(\Omega)]{u}^2.
    \end{align*}

    In der folgenden Abschätzung verwenden wir die Young Ungleichung mit $\varepsilon \in \pbraces{\frac{1}{c + 1}, 1}$.
    
    \begin{align*}
      a
      \pbraces
      {
        \begin{pmatrix}
          u \\ \xi
        \end{pmatrix},
        \begin{pmatrix}
          u \\ \xi
        \end{pmatrix}
      }
      & =
      \Int[\Omega]{(\nabla u - \xi) \cdot (\nabla u - \xi)}{x}
      +
      c \Int[\Omega]{\xi \cdot \xi}{x}
      +
      \Int[\Omega]{\curl \xi \curl \xi}{x} \\
      & =
      \norm[L^2(\Omega)]{\nabla u - \xi}^2
      +
      c \norm[L^2(\Omega)]{\xi}^2
      +
      \norm[L^2(\Omega)]{\curl \xi}^2 \\
      & \geq
      \pbraces
      {
        \norm[L^2(\Omega)]{\nabla u}
        -
        \norm[L^2(\Omega)]{\xi}
      }^2
      +
      c \norm[L^2(\Omega)]{\xi}^2
      +
      \norm[L^2(\Omega)]{\curl \xi}^2 \\
      & =
      \norm[L^2(\Omega)]{\nabla u}^2
      -
      2
      \norm[L^2(\Omega)]{\nabla u}
      \norm[L^2(\Omega)]{\xi}
      +
      \norm[L^2(\Omega)]{\xi}^2
      +
      c
      \norm[L^2(\Omega)]{\xi}^2
      +
      \norm[L^2(\Omega)]{\curl \xi}^2 \\
      & \stackrel
      {
        \mathrm{Y}
      }{\geq}
      \norm[L^2(\Omega)]{\nabla u}^2
      -
      \varepsilon
      \norm[L^2(\Omega)]{\nabla u}^2
      -
      \frac{1}{\varepsilon}
      \norm[L^2(\Omega)]{\xi}^2
      +
      \norm[L^2(\Omega)]{\xi}^2
      +
      c
      \norm[L^2(\Omega)]{\xi}^2
      +
      \norm[L^2(\Omega)]{\curl \xi}^2 \\
      & =
      (1 - \varepsilon)
      \norm[L^2(\Omega)]{\nabla u}^2
      +
      \pbraces{1 + c - \frac{1}{\varepsilon}}
      \norm[L^2(\Omega)]{\xi}^2
      +
      \norm[L^2(\Omega)]{\curl \xi}^2 \\
      & \geq
      \frac{1 - \varepsilon}{(C_p + 1)^2}
      \norm[H^1(\Omega)]{\nabla u}^2
      +
      \pbraces{1 + c - \frac{1}{\varepsilon}}
      \norm[L^2(\Omega)]{\xi}^2
      +
      \norm[L^2(\Omega)]{\curl \xi}^2 \\
      & \geq
      \min
      \Bbraces
      {
        \frac{1 - \varepsilon}{(C_p + 1)^2},
        1 + c - \frac{1}{\varepsilon},
        1
      }
      \pbraces
      {
        \norm[H^1(\Omega)]{\nabla u}^2
        +
        \norm[L^2(\Omega)]{\xi}^2
        +
        \norm[L^2(\Omega)]{\curl \xi}^2
      } \\
      & =
      \min
      \Bbraces
      {
        \frac{1 - \varepsilon}{(C_p + 1)^2},
        1 + c - \frac{1}{\varepsilon},
        1
      }
      \norm[X]
      {
        \begin{pmatrix}
          u \\ \xi
        \end{pmatrix}
      }^2
    \end{align*}

    Wir sind fertig, weil $c > 0$.

    \item Stetigkeit von $F$:
    
    \begin{align*}
      F
    \begin{pmatrix}
      v \\ \xi
    \end{pmatrix}
    =
    \Int[\Omega]{f v}{x}
    \stackrel
    {
      \mathrm{CSB}
    }{\leq}
    \norm[L^2(\Omega)]{f}
    \norm[L^2(\Omega)]{v}
    \leq
    \norm[L^2(\Omega)]{f}
    \norm[X]
    {
      \begin{pmatrix}
        v \\ \xi
      \end{pmatrix}  
    }
    \end{align*}

  \end{enumerate}

\end{enumerate}

\end{solution}

% --------------------------------------------------------------------------------


--------------------------------------------------------------------------------


\end{document}
