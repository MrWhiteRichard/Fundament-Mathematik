\setcounter{exercise}{0}

\section{Messbare Funktionen}

--------------------------------------------------------------------------------

\begin{exercise}

Es sei $\Omega = \N$, $\mathfrak{S} = 2^\N$, $\mu(A) = \sum_{x \in A} 2^{-x}$. Wann konvergiert die Funktionenfolge $f_n$ im Maßraum $(\Omega, \mathfrak{S}, \mu)$

\begin{itemize}
  \item[(a)] fast überall
  \item[(b)] fast gleichmäßig
  \item[(c)] im Maß?
\end{itemize}

\end{exercise}

\begin{solution}

(a) $f_n \xrightarrow{\text{f.ü.}} f : \Leftrightarrow$

\begin{align*}
  \Exists N \in \mathfrak{S}:
  \mu(N) = 0,
  f_n|_{N^\complement} \xrightarrow{\text{punktw.}} f|_{N^\complement}
\end{align*}

Aber ...

\begin{align*}
  \mu(N) = 0
  \Leftrightarrow
  \sum_{x \in N} 2^{-x} = 0
  \Leftrightarrow
  N = \emptyset
\end{align*}

Also ...

\begin{align*}
  f_n \xrightarrow{\text{f.ü.}} f
  \Leftrightarrow
  f_n \xrightarrow{\text{punktw.}} f
\end{align*}

\end{solution}

(b) $f_n \xrightarrow{\mu \text{-fast glm.}} f : \Leftrightarrow$

\begin{align*}
  \Forall \epsilon > 0:
  \Exists A_\epsilon \in \mathfrak{S}:
  \mu(A_\epsilon) < \epsilon,
  f_n|_{A_\epsilon^\complement} \xrightarrow{\text{glm.}} f|_{A_\epsilon^\complement}
\end{align*}

$\mu$ ist ein endliches Maß, weil

\begin{align*}
  \mu(\Omega) = \mu(\N) = \sum_{n \in \N} \frac{1}{2^n} = 2 < \infty.
\end{align*}

Laut \Quote{Egorov}, gilt also

\begin{align*}
  f_n \xrightarrow{\mu \text{-fast glm.}} f
  \Leftrightarrow
  f_n \xrightarrow{\text{f.ü.}} f.
\end{align*}

(c) $f_n \xrightarrow{\text{im Maß}} f : \Leftrightarrow$

\begin{align*}
  \Forall \epsilon > 0:
  \lim_{n \to \infty} \mu(|f_n - f| > \epsilon) = 0
\end{align*}

zz: $f_n \xrightarrow{\text{im Maß}} f \Leftrightarrow f_n \xrightarrow{\text{punktw.}} f$ \\

Nun gilt

\begin{align*}
  \mu(|f_n - f| > \epsilon)
  =
  \sum_{x \in [|f_n - f| > \epsilon]} 2^{-x}
\end{align*}

\begin{itemize}

  \item[\Quote{$\Rightarrow$}:] Angenommen,
  \begin{align*}
    \Exists x \in \Omega:
    \Exists \epsilon > 0:
    \Forall N \in \N:
    \Exists n \geq N:
    |f_n(x) - f(x)| > \epsilon,
  \end{align*}
  dann muss $[|f_n - f| > \epsilon] \neq \emptyset$, also $\mu(|f_n - f| > \epsilon) \neq 0$ und somit $\lim_{n \to \infty} \mu(|f_n - f| > \epsilon) \neq 0$.

  \item[\Quote{$\Leftarrow$}:] Angenommen,
  \begin{align*}
    \Exists \epsilon > 0:
    \Forall N \in \N:
    \Exists n \leq N:
    \mu(|f_n - f| > \epsilon) \neq 0,
  \end{align*}
  dann muss $[|f_n - f| > \epsilon] \neq 0$, also $\Exists x \in \Omega: |f_n - f| > \epsilon$.

\end{itemize}

--------------------------------------------------------------------------------

\begin{exercise}

Es sei $\Omega = \N$, $\mathfrak{S} = 2^\N$, $\mu(A) = |A|$. Wann konvergiert die Funktionenfolge $f_n$ im Maßraum $(\Omega, \mathfrak{S}, \mu)$

\begin{itemize}
  \item[(a)] fast überall
  \item[(b)] fast gleichmäßig
  \item[(c)] im Maß?
\end{itemize}

\end{exercise}

\begin{solution}

Man beachte, dass $\Forall A \in 2^\N: |A| = 0 \Leftarrow A = \emptyset$, d.h. $\emptyset$ ist die einzige Nullmenge. Somit gilt eine Aussage genau dann fast überall, wenn sie auf $\emptyset^\complement = \N$ gilt, also überall.

(a)

\begin{align*}
  f_n \xrightarrow[n \to \infty]{\text{f.ü.}} f
  \Leftarrow
  f_n \xrightarrow[n \to \infty]{\text{punktw.}} f
\end{align*}

(b)

\begin{align*}
  f_n \xrightarrow[n \to \infty]{\text{fast glm.}} f
  \Leftarrow
  f_n \xrightarrow[n \to \infty]{\text{glm.}} f
\end{align*}

(c)

\end{solution}

--------------------------------------------------------------------------------

\begin{exercise}

$f: \R \to \R$ sei überall differenzierbar. Zeigen Sie, dass $f^\prime$ Borel-messbar ist.

\end{exercise}

\begin{solution}

\begin{align*}
  \Forall n \in \N:
  f_n: x \mapsto \frac{f(x + 1/n) - f(x)}{1/n},
  \enspace \text{messb.}
  \Rightarrow
  f^\prime = \lim_{n \to \infty} f_n
  \enspace \text{messb.}
\end{align*}

\end{solution}

--------------------------------------------------------------------------------

\begin{exercise}

\begin{itemize}
  \item[(a)] Definieren Sie: messbare Funktion, Treppenfunktion, Konvergenz im Maß, Konvergenz fast überall, Konvergenz fast gleichmäßig.
  \item[(b)] Formulieren und beweisen Sie den Approximationssatz für reellwertige messbare Funktionen.
\end{itemize}

\end{exercise}

\begin{solution}

(a) $f: (\Omega_1, \mathfrak{S}_1) \to (\Omega_2, \mathfrak{S}_2) \enspace \text{Treppenfunktion}
: \Leftrightarrow
\Exists a_1, \ldots, a_n \in \Omega_2,
\Exists A_1, \ldots, A_n \in \mathfrak{S}_1:$

\begin{align*}
  \sum_{i=1}^n A_i = \Omega_1, \enspace
  \sum_{i=1}^n a_i A_i = f
\end{align*}

Rest siehe Aufgabe 1 und 12 (a).

(b) Siehe Skript.

\end{solution}

--------------------------------------------------------------------------------

\begin{exercise}

Hier könnte Ihre Werbung stehen!

\begin{itemize}
  \item Definieren Sie: messbare Funktion, Treppenfunktion, Konvergenz im Maß, Konvergenz fast überall, Konvergenz fast gleichmäßig.
  \item In welchem Sinn (fast überall gleichmäßig/fast gleichmäßig/fast überall/im Maß) konvergieren die folgenden Folgen in $(\R, \mathfrak{B}, \lambda)$?
  \begin{itemize}
    \item[i.] $f_n(x) = \sin(x)/n$
    \item[ii.] $f_n(x) = e^{-n |x|}$
    \item[iii.] $f_n(x) = x/n$
    \item[iv.] $f_n(x) = f_n(x) =
    \begin{cases}
      1 & \text{wenn} \enspace \sqrt{n} - \floor{\sqrt{n}} \leq x \leq \sqrt{n+1} - \floor{\sqrt{n}} \\
      0 & \text{sonst}.
    \end{cases}$
  \end{itemize}
\end{itemize}

\end{exercise}

\begin{solution}

(a) Siehe Aufgabe 7 (a) \\

(b) Hier könnte Ihre Werbung stehen!

\begin{itemize}

  \item[i.]
  \begin{align*}
     \norm[\infty]{f_n}
     =
     \sup_{x \in \R} \vbraces{\sin(x)/n}
     =
     1/n
     \xrightarrow[n \to \infty]{} 0 \\
     \Rightarrow
     f_n \xrightarrow[n \to \infty]{\text{glm.}} 0
     \Rightarrow
     f_n \xrightarrow[n \to \infty]{\text{fast glm.}} 0
     \Rightarrow
     f_n \xrightarrow[n \to \infty]{\text{f.ü.}} 0,
     \enspace
     f_n \xrightarrow[n \to \infty]{\text{im Maß}} 0
   \end{align*}

  \item[ii.]
  \begin{align*}
    \Forall A \subseteq \R \setminus \Bbraces{0}:
    \norm[\infty]{f_n|_A}
    =
    \sup_{x \in A} \vbraces{e^{-n |x|}}
    =
    e^{-n \inf_{x \in A} |x|}
    \xrightarrow[n \to \infty]{} 0 \\
    \Rightarrow
    f_n \xrightarrow[n \to \infty]{\text{fast glm.}} 0
    \Rightarrow
    f_n \xrightarrow[n \to \infty]{\text{f.ü.}} 0,
    \enspace
    f_n \xrightarrow[n \to \infty]{\text{im Maß}} 0
  \end{align*}
  Sei $N \in \mathfrak{B}$, mit $\lambda(N) = 0$, dann gilt trotzdem noch $\Forall \epsilon > 0: \vbraces{B(0, \epsilon) \cap N^\complement} = \infty$, also konvergiert $(f_n)$ nicht fast überall gleichmäßig.

  \item[iii.]
  \begin{align*}
    f_n \xrightarrow[n \to \infty]{\text{punktw.}} 0
    \Rightarrow
    f_n \xrightarrow[n \to \infty]{\text{f.ü.}} 0
  \end{align*}
  $\Forall \epsilon > 0, \Forall n \in \N:$
  \begin{align*}
    \lambda(x/n > \epsilon)
    =
    \lambda(x > \epsilon n)
    =
    \lambda(]\epsilon n, \infty])
    =
    \infty
  \end{align*}
  Also konvergiert $(f_n)$ nicht im Maß. \\
  Wenn $(f_n)$ fast (überall) gleichmäßig konvergiert, dann offensichtlich gegen $0$. Aber $\Forall \epsilon > 0, \Forall N \in \mathfrak{B}:$
  \begin{align*}
    \lambda(N) < \epsilon
    \Rightarrow
    \Forall n \in \N:
    \norm[\infty]{f_n|_{N^\complement}}
    =
    \sup_{x \in N^\complement} |x|/n
    =
    \infty
  \end{align*}
  Also konvergiert $(f_n)$ weder fast gleichmäßig, noch fast überall gleichmäßig. \\

  \item[iv.] $\Forall \epsilon > 0:$
  \begin{align*}
    \lambda(|f_n| > \epsilon)
    =
    \lambda(f_n = 1)
    =
    \lambda
    ([\sqrt{n} - \floor{\sqrt{n}}, \sqrt{n+1} - \floor{\sqrt{n}}])
    =
    \sqrt{n+1} - \sqrt{n}
    \xrightarrow[n \to \infty]{} 0 \\
    \Rightarrow
    f_n \xrightarrow[n \to \infty]{\text{im Maß}} 0
  \end{align*}
  $\Forall q \in \N^2:$
  \begin{align*}
    ]0, 1]
    =
    \sum_{i = q}^{(\sqrt{q}+1)^2-1}
    \left ]
    \sqrt{i} - \floor{\sqrt{i}}, \sqrt{i+1} - \floor{\sqrt{i}}
    \right ]
  \end{align*}
  Also, muss $\Forall x \in \: ]0, 1], \Forall N \in \N: \Exists n, m \geq N:$
  \begin{align*}
    |f_n(x) - f_m(x)| = 1,
  \end{align*}
  und es konvergiert $(f_n)$ nicht fast überall und somit auch weder fast gleichmäßig, noch fast überall gleichmäßig.

\end{itemize}

\end{solution}

--------------------------------------------------------------------------------

\begin{exercise}

Hier könnte Ihre Werbung stehen!

\begin{itemize}
  \item Definieren Sie: messbare Funktion, Treppenfunktion, Konvergenz im Maß, Konvergenz fast überall, Konvergenz fast gleichmäßig.
  \item In welchem Sinn (fast überall gleichmäßig/fast gleichmäßig/fast überall/im Maß) konvergieren die folgenden Folgen in $(\R, \mathfrak{B}, \lambda)$?
  \begin{itemize}
    \item[i.] $f_n(x) = \sin(x)/n$
    \item[ii.] $f_n(x) = e^{-n |x|}$
    \item[iii.] $f_n(x) = x/n$
    \item[iv.] $f_n(x) = f_n(x) =
    \begin{cases}
      1 & \text{wenn} \enspace \sqrt{n} - \floor{\sqrt{n}} \leq x \leq \sqrt{n+1} - \floor{\sqrt{n}} \\
      0 & \text{sonst}.
    \end{cases}$
  \end{itemize}
\end{itemize}

\end{exercise}

\begin{solution}

Siehe Aufgabe 8.

\end{solution}

--------------------------------------------------------------------------------

\begin{exercise}

Hier könnte Ihre Werbung stehen!

\begin{itemize}
  \item[(a)] Definieren Sie: messbare Funktion, Treppenfunktion, Konvergenz im Maß, Konvergenz fast überall, Konvergenz fast gleichmäßig.
  \item[(b)] $f_n$, $n \in \N$ und $f$ seien reellwertige messbare Funktionen auf dem Maßraum $(\Omega, \mathfrak{S}, \mu)$. Zeigen Sie:
  \begin{itemize}
    \item[i.] Wenn $f_n \to f$ fast überall und $g: \R \to \R$ stetig ist, dann $g \circ f_n \to g \circ f$ fast überall.
    \item[ii.] Wenn $f_n \to f$ im Maß und $g: \R \to \R$ gleichmäßig stetig ist, dann $g \circ f_n \to g \circ f$ im Maß.
    \item[iii.] Geben Sie ein Beispiel einer Folge $f_n$ und einer stetigen Funktion $g$, sodass $f_n$ im Maß konvergiert, aber nicht $g \circ f_n$.
  \end{itemize}
\end{itemize}

\end{exercise}

\begin{solution}

(a) Siehe Aufgabe 1 (a) und Kapitel 4 Aufgabe 7 (a). \\

(b) i. Wegen der Konvergenz fast überall von $(f_n)$, gilt

\begin{align*}
  \Exists N \in \mathfrak{S}:
  \mu(N) = 0,
  \Forall \omega \in N^\complement:
  \Forall \delta > 0:
  \Exists n_0 \in \N:
  \Forall n \geq n_0:
  |f_n(\omega) - f(\omega)| < \delta.
\end{align*}

Wegen der Stetigkeit von $g$, gilt

\begin{align*}
  \Forall x \in \R:
  \Exists \delta > 0:
  \Forall y \in \R:
  |x - y| < \delta
  \Rightarrow
  |g(x) - g(y)| < \epsilon.
\end{align*}

Damit folgt die Konvergenz fast überall von $(g \circ f_n)$ ...

\begin{align*}
  \Forall \omega \in N^\complement:
  \Forall \epsilon > 0:
  \Exists n_0^\prime \in \N:
  \Forall n \geq n_0^\prime:
  |g(f_n(\omega)) - g(f(\omega))| < \epsilon
\end{align*}

ii. Wegen der Konvergenz im Maß von $(f_n)$, gilt

\begin{align*}
  \Forall \delta > 0:
  \mu(|f_n - f| > \delta)
  \xrightarrow[n \to \infty]{} 0.
\end{align*}

Wegen der gleichmäßigen Stetigkeit von $g$, gilt

\begin{align*}
  \Forall \epsilon > 0:
  \Exists \delta > 0:
  \Forall x, y \in \R:
  |x - y| \leq \delta
  \Rightarrow
  |g(x) - g(y)| \leq \epsilon
\end{align*}

Damit folgt die Konvergenz im Maß von $(g \circ f_n)$ ...

\begin{align*}
  \Forall \epsilon > 0:
  \mu(|g \circ f_n - g \circ f| > \epsilon)
  \leq
  \mu(|f_n - f| > \delta)
  \xrightarrow[n \to \infty]{} 0
\end{align*}

\end{solution}

--------------------------------------------------------------------------------

\begin{exercise}

Hier könnte Ihre Werbung stehen!

\begin{itemize}
  \item[(a)] Definieren Sie: Konvergenz fast überall, fast überall gleichmäßig, fast gleichmäßig, im Maß.
  \item[(b)]  Gegeben ist der Maßraum $(\N, 2^\N, \mu)$ mit $\mu(\Bbraces{x}) = 2^{-x}$, $x \in \N$. Zeigen Sie, dass in diesem Maßraum die Konvergenzen fast überall, fast gleichmäßig und im Maß äquivalent sind.
\end{itemize}

\end{exercise}

\begin{solution}

(a) Siehe Aufgabe 1. \\

(b) Hier könnte Ihre Werbung stehen!

\begin{itemize}

  \item \Quote{f.ü. $\Rightarrow$ fast glm.}: $\mu$ ist ein endliches Maß, weil
  \begin{align*}
    \mu(\N)
    =
    \sum_{x \in \N} \mu(\Bbraces{x})
    =
    \sum_{x \in \N} 2^{-x}
    =
    2 < \infty.
  \end{align*}
  Also gilt, laut \Quote{Egorov},
  \begin{align*}
    f_n \xrightarrow[n \to \infty]{\text{f.ü.}} f
    \Rightarrow
    f_n \xrightarrow[n \to \infty]{\text{fast glm.}} f.
  \end{align*}

  \item \Quote{fast glm. $\Rightarrow$ im Maß}: Zudem, gilt immer
  \begin{align*}
    f_n \xrightarrow[n \to \infty]{\text{fast glm.}} f
    \Rightarrow
    f_n \xrightarrow[n \to \infty]{\text{im Maß}} f.
  \end{align*}

  \item \Quote{im Maß $\Rightarrow$ f.ü.}:
  \begin{align*}
    f_n \xrightarrow[n \to \infty]{\text{im Maß}} f
  \end{align*}
  heißt, dass $\Forall \epsilon > 0:$
  \begin{align*}
    \sum_{x \in [|f_n - f| > \epsilon]} 2^{-x}
    =
    \mu(|f_n - f| > \epsilon)
    \xrightarrow[n \to \infty]{} 0,
  \end{align*}
  also gilt $\Forall x \in \N: \Exists N \in \N: \Forall n \geq N:$
  \begin{align*}
    x \notin [|f_n - f| > \epsilon]
    \Leftrightarrow
    |f_n(x) - f(x)| \leq \epsilon,
  \end{align*}
  und somit schließlich
  \begin{align*}
    f_n \xrightarrow[n \to \infty]{\text{f.ü.}} f.
  \end{align*}

\end{itemize}

\end{solution}

--------------------------------------------------------------------------------

\begin{exercise}

Hier könnte Ihre Werbung stehen!

\begin{itemize}
  \item[(a)] Definieren Sie messbare Funktion, maßtreue Funktion.
  \item[(b)] Auf $\Omega = \N$ ist die Funktion
  \begin{align*}
    f(x) = x^2 - 3x
  \end{align*}
  gegeben. Bestimmen Sie die kleinste Sigmaalgebra $\mathfrak{S}$ über $\Omega$, für die $f$ $\mathfrak{S}$-messbar ist.
\end{itemize}

\end{exercise}

\begin{solution}

(a) Hier könnte Ihre Werbung stehen!

\begin{itemize}

  \item $f: (\Omega_1, \mathfrak{S}_1) \to (\Omega_2, \mathfrak{S}_2) : \Leftrightarrow f^{-1}(\mathfrak{S}_2) \subseteq \mathfrak{S}_1$

  \item $f: (\Omega_1, \mathfrak{S}_1, \mu_1) \to (\Omega_2, \mathfrak{S}_2, \mu_2) \enspace \text{maßtreu} : \Leftrightarrow \mu_1 \circ f^{-1} = \mu_2$

\end{itemize}

(b) Das wäre dann die initiale Sigmaalgebra

\begin{align*}
  \mathfrak{S}
  =
  f^{-1}(\mathfrak{B})
  =
  \Bbraces{\Bbraces{n \in \N: n^2 - 3n \in A}: A \in \mathfrak{B}}.
\end{align*}

zz: $\mathfrak{S} = \mathfrak{S}^\prime := \Bbraces{B \in 2^\N: 1 \in B \Leftrightarrow 2 \in B}$

\begin{itemize}

  \item[\Quote{$\subseteq$}:] $f$ ist bis auf $1, 2$ injektiv.

  \item[\Quote{$\supseteq$}:] $\Forall x \in \R: \Bbraces{x} \in \mathfrak{B}$

\end{itemize}

\end{solution}

--------------------------------------------------------------------------------

\begin{exercise}

Hier könnte Ihre Werbung stehen!

\begin{itemize}
  \item[(a)] Definieren Sie Konvergenz im Maß, fast überall, fast gleichmäßig, fast überall gleichmäßig.
  \item[(b)] $(X_n)$ sei eine Folge von unabhängigen Zufallsvariablen. Zeigen Sie, dass genau dann fast sicher
  \begin{align*}
    \lim_{n \to \infty} X_n = 0
  \end{align*}
  gilt, wenn für jedes $\epsilon > 0$
  \begin{align*}
    \sum_{n \in \N} \P(|X_n| > \epsilon) < \infty.
  \end{align*}
\end{itemize}

\end{exercise}

\begin{solution}

(a) Siehe Aufgabe 1. \\

(b) Hier könnte Ihre Werbung stehen!

\begin{itemize}

  \item[\Quote{$\Rightarrow$}:] Angenommen, $\Exists \epsilon > 0:$
  \begin{align*}
    \sum_{n \in \N} \P(|X_n| > \epsilon) = \infty,
  \end{align*}
  dann gilt laut dem \Quote{zweiten Lemma von Borel-Cantelli}, dass
  \begin{align*}
    \P(\limsup_{n \in \N} [|X_n| > \epsilon]) = 1.
  \end{align*}
  $\limsup_{n \in \N} [|X_n| > \epsilon]$ ist dabei die Menge aller Punkte, die in unendlich vielen $[|X_n| > \epsilon]$ enthalten ist.

  \item[\Quote{$\Leftarrow$}:]
  $1 - \P(|X_n| \leq \epsilon)
  =
  \P(|X_n| > \epsilon)
  \xrightarrow[n \to \infty]{} 0$

\end{itemize}

\end{solution}

--------------------------------------------------------------------------------
