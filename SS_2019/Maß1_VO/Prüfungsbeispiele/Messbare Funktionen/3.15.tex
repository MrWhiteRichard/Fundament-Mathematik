\begin{exercise}

Hier könnte Ihre Werbung stehen!

\begin{itemize}
  \item[(a)] Definieren Sie Konvergenz im Maß, fast überall, fast gleichmäßig, fast überall gleichmäßig.
  \item[(b)] $(X_n)$ sei eine Folge von unabhängigen Zufallsvariablen. Zeigen Sie, dass genau dann fast sicher
  \begin{align*}
    \lim_{n \to \infty} X_n = 0
  \end{align*}
  gilt, wenn für jedes $\epsilon > 0$
  \begin{align*}
    \sum_{n \in \N} \P(|X_n| > \epsilon) < \infty.
  \end{align*}
\end{itemize}

\end{exercise}

% --------------------------------------------------------------------------------

\begin{solution}

(a) Siehe Aufgabe 1. \\

(b) Hier könnte Ihre Werbung stehen!

\begin{itemize}

  \item[\Quote{$\Rightarrow$}:] Angenommen, $\Exists \epsilon > 0:$
  \begin{align*}
    \sum_{n \in \N} \P(|X_n| > \epsilon) = \infty,
  \end{align*}
  dann gilt laut dem \Quote{zweiten Lemma von Borel-Cantelli}, dass
  \begin{align*}
    \P(\limsup_{n \in \N} [|X_n| > \epsilon]) = 1.
  \end{align*}
  $\limsup_{n \in \N} [|X_n| > \epsilon]$ ist dabei die Menge aller Punkte, die in unendlich vielen $[|X_n| > \epsilon]$ enthalten ist.

  \item[\Quote{$\Leftarrow$}:]
  $1 - \P(|X_n| \leq \epsilon)
  =
  \P(|X_n| > \epsilon)
  \xrightarrow[n \to \infty]{} 0$

\end{itemize}

\end{solution}
