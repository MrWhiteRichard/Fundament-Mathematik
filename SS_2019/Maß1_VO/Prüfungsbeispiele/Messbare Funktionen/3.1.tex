\begin{exercise}

Es sei $\Omega = \N$, $\mathfrak{S} = 2^\N$, $\mu(A) = \sum_{x \in A} 2^{-x}$. Wann konvergiert die Funktionenfolge $f_n$ im Maßraum $(\Omega, \mathfrak{S}, \mu)$

\begin{itemize}
  \item[(a)] fast überall
  \item[(b)] fast gleichmäßig
  \item[(c)] im Maß?
\end{itemize}

\end{exercise}

% --------------------------------------------------------------------------------

\begin{solution}

(a) $f_n \xrightarrow{\text{f.ü.}} f : \Leftrightarrow$

\begin{align*}
  \Exists N \in \mathfrak{S}:
  \mu(N) = 0,
  f_n|_{N^\complement} \xrightarrow{\text{punktw.}} f|_{N^\complement}
\end{align*}

Aber ...

\begin{align*}
  \mu(N) = 0
  \Leftrightarrow
  \sum_{x \in N} 2^{-x} = 0
  \Leftrightarrow
  N = \emptyset
\end{align*}

Also ...

\begin{align*}
  f_n \xrightarrow{\text{f.ü.}} f
  \Leftrightarrow
  f_n \xrightarrow{\text{punktw.}} f
\end{align*}

\end{solution}

(b) $f_n \xrightarrow{\mu \text{-fast glm.}} f : \Leftrightarrow$

\begin{align*}
  \Forall \epsilon > 0:
  \Exists A_\epsilon \in \mathfrak{S}:
  \mu(A_\epsilon) < \epsilon,
  f_n|_{A_\epsilon^\complement} \xrightarrow{\text{glm.}} f|_{A_\epsilon^\complement}
\end{align*}

$\mu$ ist ein endliches Maß, weil

\begin{align*}
  \mu(\Omega) = \mu(\N) = \sum_{n \in \N} \frac{1}{2^n} = 2 < \infty.
\end{align*}

Laut \Quote{Egorov}, gilt also

\begin{align*}
  f_n \xrightarrow{\mu \text{-fast glm.}} f
  \Leftrightarrow
  f_n \xrightarrow{\text{f.ü.}} f.
\end{align*}

(c) $f_n \xrightarrow{\text{im Maß}} f : \Leftrightarrow$

\begin{align*}
  \Forall \epsilon > 0:
  \lim_{n \to \infty} \mu(|f_n - f| > \epsilon) = 0
\end{align*}

zz: $f_n \xrightarrow{\text{im Maß}} f \Leftrightarrow f_n \xrightarrow{\text{punktw.}} f$ \\

Nun gilt

\begin{align*}
  \mu(|f_n - f| > \epsilon)
  =
  \sum_{x \in [|f_n - f| > \epsilon]} 2^{-x}
\end{align*}

\begin{itemize}

  \item[\Quote{$\Rightarrow$}:] Angenommen,
  \begin{align*}
    \Exists x \in \Omega:
    \Exists \epsilon > 0:
    \Forall N \in \N:
    \Exists n \geq N:
    |f_n(x) - f(x)| > \epsilon,
  \end{align*}
  dann muss $[|f_n - f| > \epsilon] \neq \emptyset$, also $\mu(|f_n - f| > \epsilon) \neq 0$ und somit $\lim_{n \to \infty} \mu(|f_n - f| > \epsilon) \neq 0$.

  \item[\Quote{$\Leftarrow$}:] Angenommen,
  \begin{align*}
    \Exists \epsilon > 0:
    \Forall N \in \N:
    \Exists n \leq N:
    \mu(|f_n - f| > \epsilon) \neq 0,
  \end{align*}
  dann muss $[|f_n - f| > \epsilon] \neq 0$, also $\Exists x \in \Omega: |f_n - f| > \epsilon$.

\end{itemize}
