\setcounter{exercise}{0}

\section{Lebesgue-Stieltjes Maße}

--------------------------------------------------------------------------------

\begin{exercise}

Zeigen Sie:

\begin{align*}
  F(x, y) =
  \begin{cases}
    x y^2 & \text{falls} \enspace y > 0. \\
    0     & \text{sonst}
  \end{cases}
\end{align*}

ist eine zweidimensionale Verteilungsfunktion. Bestimmen Sie das Maß des Einheitskreises.

\end{exercise}

\begin{solution}

Hier könnte Ihre Werbung stehen!

\begin{itemize}

  \item \Quote{Rechtsstetigkeit}: Tatsächlich gilt sogar $\Forall x \in \R: F(x, 0 + 0) = F(x, 0) = 0$.

  \item \Quote{nichtnegativer Differenzenoperator}: Seien $a, b \in \R^2$, mit $a \leq b$.
  \begin{align*}
    \Delta^{(a, b)} F(x)
    & =
    \Delta_1^{(a_1, b_1)}
    \Delta_2^{(a_2, b_2)}
    F(x) \\
    & =
    \Delta_1^{(a_1, b_1)}
    \pbraces{F(x_1, b_2) - F(x_1, a_2)} \\
    & =
    \pbraces{\Delta_1^{(a_1, b_1)} F(x_1, b_2)} -
    \pbraces{\Delta_1^{(a_1, b_1)} F(x_1, a_2)} \\
    & =
    \pbraces{F(b_1, b_2) - F(a_1, b_2)} -
    \pbraces{F(b_1, a_2) - F(a_1, a_2)} \\
    & =
    \pbraces{b_1 b_2^2 - a_1 b_2^2} -
    \pbraces{b_1 a_2^2 - a_1 a_2^2} \\
    & =
    b_2^2 (b_1 - a_1) - a_2^2 (b_1 - a_1) \\
    & =
    (b_1 - a_1) (b_2^2 - a_2^2)
    \geq
    0
  \end{align*}

\end{itemize}

Die untere Hälfte des Einheitskreises, hat offensichtlich Maß Null. Die Obere, werden wir mit Untersummen approximieren. Dazu betrachten wir vorerst die rechte Hälfte.

\begin{align*}
  \sum_{i=0}^{n-1}
  \mu_F
  \pbraces
  {
    \Bigg ]
      \pbraces{\frac{i}{n}, 0},
      \pbraces{\frac{i+1}{n}, \sqrt{1 - \pbraces{\frac{i+1}{n}}^2}}
    \Bigg ]
  }
  =
  \sum_{i=0}^{n-1}
  \pbraces{\frac{i+1}{n} - \frac{i}{n}}
  \pbraces{\sqrt{1 - \pbraces{\frac{i+1}{n}}^2}^2 - 0^2} \\
  \xrightarrow[n \to \infty]{}
  \Int[0][1]{1 - x^2}{x}
  =
  1 - \frac{x^3}{3} \Bigg |_0^1
  =
  \frac{2}{3}
\end{align*}

Nachdem dieser Ausdruck symmetrisch um die $y$-Achse ist (Quadrat), gilt dieser auch für die linke Hälfte und damit $\mu_F(B(0, 1)) = \frac{4}{3}$.

\end{solution}

--------------------------------------------------------------------------------

\begin{exercise}

Hier könnte Ihre Werbung stehen!

\begin{itemize}
  \item[(a)] Definieren Sie Maßfunktion, Lebesgue-Stieltjes Maßfunktion, Verteilungsfunktion.
  \item[(b)] Zeigen Sie, dass jede nichtfallende und rechtstetige Funktion Verteilungsfunktion eines Lebesgue-Stieltjes Maßes auf $\R$ ist.
  \item[(c)] Zeigen Sie, dass $F(x) = \min(x_1, x_2)$ zweidimensionale Verteilungsfunktion ist.
\end{itemize}

\end{exercise}

\begin{solution}

(a) Siehe Aufgabe 1 (a), Kapitel 1 und Aufgabe 11 (a) Kapitel 2. \\

(b) Siehe Skript! \\

(c) Siehe Aufgabe 6.

\end{solution}

--------------------------------------------------------------------------------

\begin{exercise}

Zeigen Sie:

\begin{align*}
  F(x, y) = \min(x, y)
\end{align*}

ist eine zweidimensionale Verteilungsfunktion, und bestimmen Sie $\mu_F(]0, 1] \times ]0, 1])$ und $\mu_F(\Bbraces{(x, x) : 0 < x \leq 1})$.

\end{exercise}

\begin{solution}

Hier könnte Ihre Werbung stehen!

\begin{itemize}

  \item \Quote{Rechtsstetigkeit}: $\min(x, y) = \Frac{2}{x + y - |x - y|}$

  \item \Quote{nichtnegativer Differenzenoperator}: Seien $a, b \in \R^2$, mit $a \leq b$.
  \begin{align*}
    \Delta^{(a, b)} \min
    =
    \min(b_1, b_2) - \min(a_1, b_2) - \min(b_1, a_2) + \min(a_1, a_2)
  \end{align*}

\end{itemize}

Es könnte jetzt eine langweilige Fallunterscheidung folgen. Wir führen diese aber nicht zur Gänze aus.

\begin{itemize}
  \item[Fall 1:] $b_1 \leq b_2 \Rightarrow a_1 \leq b_2$
  \begin{itemize}
    \item[Fall a:] $a_1 \leq a_2 \Rightarrow a_1 \leq b_2$
    \begin{itemize}
      \item[Fall i:] $a_2 \leq b_1 \Rightarrow
      \Delta^{(a, b)} \min = b_1 - a_1 - a_2 + a_1 \geq 0$
      \item[Fall ii:] $b_1 \leq a_2 \Rightarrow
      \Delta^{(a, b)} \min = b_1 - a_1 - b_1 + a_1 \geq 0$
    \end{itemize}
  \end{itemize}
\end{itemize}

Wir berechnen nun die Maße der Mengen

\begin{align*}
  A := ]0, 1]^2, \\
  B := \Bbraces{(x, x): 0 < x \leq 1}.
\end{align*}

\begin{itemize}

  \item $\mu(A) =
  \Delta^{(0, 0), (1, 1)} \min =
  \min(1, 1) - \min(0, 1) - \min(1, 0) + \min(0, 0) = 1$

  \item Wir betrachten die Überdeckungen von
  \begin{align*}
    B \supseteq B_n
    :=
    \sum_{i=1}^n
    \left ] \frac{i-1}{n}, \frac{i}{n} \right ]^2
    \xrightarrow{n \in \N^2} B.
  \end{align*}
  Weil $(B_{n_k})$ mit $n_k := k^2$ monoton gegen $B$ fällt, kann man die Stetigkeit von oben ausnützen.
  \begin{align*}
    \mu(B)
    =
    \mu \pbraces{\lim_{n \in \N^2} B_n}
    =
    \lim_{n \in \N^2} \mu(B_n)
    =
    \lim_{n \in \N^2} \sum_{i=1}^n \mu \pbraces
    {\left ] \frac{i-1}{n}, \frac{i}{n} \right ]^2}
    = \\
    \lim_{n \in \N^2} \sum_{i=1}^n
    \min \pbraces{\frac{i-1}{n}, \frac{i}{n}} -
    \min \pbraces{\frac{i-1}{n}, \frac{i}{n}} -
    \min \pbraces{\frac{i-1}{n}, \frac{i}{n}} +
    \min \pbraces{\frac{i-1}{n}, \frac{i}{n}}
    = 0
  \end{align*}

\end{itemize}

\end{solution}

--------------------------------------------------------------------------------

\begin{exercise}

Zeigen Sie:

\begin{align*}
  F(x, y) =
  \begin{cases}
    x y^2 & \text{falls} \enspace y > 0. \\
    0     & \text{sonst}
  \end{cases}
\end{align*}

ist eine zweidimensionale Verteilungsfunktion. Bestimmen Sie das Maß des Einheitskreises.

\end{exercise}

\begin{solution}

Siehe Aufgabe 1.

\end{solution}

--------------------------------------------------------------------------------

\begin{exercise}

Gegeben ist die Funktion $F: \R \to \R:$

\begin{align*}
  F(x) =
  \begin{cases}
    x       & \text{für} \enspace x < 0 \\
    x^2 + 1 & \text{für} \enspace 0 \leq x < 1, \\
    2x      & \text{für} \enspace 1 \leq x < 3, \\
    8       & \text{für} \enspace x \geq 3.
  \end{cases}
\end{align*}

Weisen Sie nach, dass $F$ eine Verteilungsfunktion ist, und bestimmen Sie $\mu_F(]0, 1])$, $\mu_F([0, 1])$, $\mu_F(]0, 1[)$ und $\mu_F(\Q)$.

\end{exercise}

\begin{solution}

Nachweisen:

\begin{itemize}

  \item \Quote{Rechtsstetigkeit}: $F$ ist stückweise stetig und $\Forall x = 0, 1, 3: F \text{ist rechtsstetig bei} \enspace x$.

  \item \Quote{Steigende Monotonie}: $F$ ist stückweise monoton steigend und $\Forall x = 0, 1, 3: F(x - 0) \leq F(x)$.

\end{itemize}

Bestimmen:

\begin{itemize}

  \item $\mu_F(]0, 1]) =$
  \begin{align*}
    F(1) - F(0) = 2 - 1 = 1
  \end{align*}

  \item $\mu_F([0, 1]) =$
  \begin{align*}
    \mu_F \pbraces{\bigcap_{n \in \N} \left ] 0 - \frac{1}{n}, 1 \right]}
    =
    \lim_{n \in \N} \mu_F \pbraces{ \left ] 0 - \frac{1}{n}, 1 \right]}
    =
    \lim_{n \in \N} F(1) - F \pbraces{0 - \frac{1}{n}}
    =
    \lim_{n \in \N} 2 + \frac{1}{n} = 2
  \end{align*}

  \item $\mu_F(]0, 1[) =$
  \begin{align*}
    \mu_F \pbraces{\bigcup_{n \in \N} \left ] 0, 1 - \frac{1}{n} \right ]}
    =
    \lim_{n \in \N} \mu_F \pbraces{\left ] 0, 1 - \frac{1}{n} \right ]}
    =
    \lim_{n \in \N} F \pbraces{1 - \frac{1}{n}} - F(0)
    =
    \lim_{n \in \N} \pbraces{1 - \frac{1}{n}}^2 - 1 = 0
  \end{align*}

  \item $\mu_F(\Q) =$
  \begin{align*}
    \mu_F \pbraces{\sum_{q \in \Q} \Bbraces{q}}
    & =
    \sum_{q \in \Q} \mu_F \pbraces
    {\bigcap_{n \in \N} \left ] q - \frac{1}{n}, q \right ]} \\
    & =
    \sum_{q \in \Q} \lim_{n \in \N} \mu_F \pbraces
    {\left ] q - \frac{1}{n}, q \right ]} \\
    & =
    \sum_{q \in \Q} F(q - 0) - F(q) \\
    & =
    \sum_{x = 0, 1, 3} (F(x) - F(x - 0)) \\
    & =
    (1 - 0) + (2 - 2) + (8 - 6) = 3
  \end{align*}

\end{itemize}

\end{solution}

--------------------------------------------------------------------------------

\begin{exercise}

Gegeben ist die Funktion $F: \R \to \R:$

\begin{align*}
  F(x) =
  \begin{cases}
    0   & \text{wenn} \enspace x < 0, \\
    1   & \text{wenn} \enspace 0 \leq x < 1, \\
    x^2 & \text{wenn} \enspace 1 \leq x < 2, \\
    5   & \text{wenn} \enspace x \geq 2.
  \end{cases}
\end{align*}

\begin{itemize}
  \item[(a)] Zeigen Sie, dass $F$ eine Verteilungsfunktion ist.
  \item[(b)] Bestimmen Sie $\mu_F(]0, 1[)$, $\mu_F([0, 2])$, $\mu_F(\Q)$.
  \item[(c)] Bestimmen Sie $\Int{e^x}{\mu_F(x)}$.
\end{itemize}

\end{exercise}

\begin{solution}

(a)

\begin{itemize}

  \item \Quote{Rechtsstetigkeit}: $F$ ist stückweise stetig und $\Forall x = 0, 1, 2: F \text{ist rechtsstetig bei} \enspace x$.

  \item \Quote{Steigende Monotonie}: $F$ ist stückweise monoton steigend und $\Forall x = 0, 1, 2: F(x - 0) \leq F(x)$.

\end{itemize}

(b)

\begin{itemize}

  \item $\mu_F(]0, 1[) =$
  \begin{align*}
    \mu_F \pbraces{\bigcup_{n \in \N} \left ] 0, 1 - \frac{1}{n} \right ]}
    =
    \lim_{n \in \N} \mu_F \pbraces{\left ] 0, 1 - \frac{1}{n} \right ]}
    =
    \lim_{n \in \N} F \pbraces{1 - \frac{1}{n}} - F(0)
    = 1 - 1 = 0
  \end{align*}

  \item $\mu_F([0, 2]) =$
  \begin{align*}
    \mu_F \pbraces{\bigcap_{n \in \N} \left ] 0 - \frac{1}{n}, 2 \right ]}
    =
    \lim_{n \in \N} \mu_F \pbraces{\left ] 0 - \frac{1}{n}, 2 \right ]}
    =
    \lim_{n \in \N} F(2) - F \pbraces{0 - \frac{1}{n}}
    =
    5 - 0 = 5
  \end{align*}

  \item $\mu_F(\Q) =$
  \begin{align*}
    \mu_F \pbraces{\sum_{q \in \Q} \Bbraces{q}}
    & =
    \sum_{q \in \Q} \mu_F \pbraces
    {\bigcap_{n \in \N} \left ] q - \frac{1}{n}, q \right ]} \\
    & =
    \sum_{q \in \Q} \lim_{n \in \N} \mu_F \pbraces
    {\left ] q - \frac{1}{n}, q \right ]} \\
    & =
    \sum_{q \in \Q} F(q - 0) - F(q) \\
    & =
    \sum_{x = 0, 1, 2} (F(x) - F(x - 0)) \\
    & =
    (1 - 0) + (1 - 1) + (5 - 4) = 2
  \end{align*}

\end{itemize}

(c) Seien $f = \exp$ und $a_1, \ldots, a_n$ die Sprünge von $F$, sowie $a_0 = - \infty$ und $a_{n+1} = \infty$.

\begin{align*}
  \Int{f}{\mu_F}
  =
  \sum_{i=1}^{n+1} \Int[a_{i-1}][a_i]{f(x) F^\prime(x)}{x} +
  \sum_{i=1}^n f(a_i) (F(a_i) - F(a_i - 0))
\end{align*}

Also ...

\begin{align*}
  \Int{e^x}{\mu_F(x)}
  & =
  \underbrace{\Int[-\infty][0]{e^x 0}{x}}_0
  +
  \underbrace{\Int[0][1]{e^x 0}{x}}_0
  +
  \Int[1][2]{e^x 2x}{x}
  +
  \underbrace{\Int[2][\infty]{e^x 0}{x}}_0 \\
  & +
  e^0 \underbrace{(F(0) - F(0 - 0))}_{= 1-0 = 1}
  +
  e^1 \underbrace{(F(1) - F(1 - 0))}_{= 1-1 = 0}
  +
  e^2 \underbrace{(F(2) - F(2 - 0))}_{= 5-4 = 1} \\
  & =
  e^x 2x |_1^2 - 2 \Int[1][2]{e^x}{x} + 1 + e^2 \\
  & =
  (4e^2 - 2e) - 2 (e^2 - e) + 1 + e^2
  =
  1 + 3e^2
\end{align*}

\end{solution}

--------------------------------------------------------------------------------

\begin{exercise}

Hier könnte Ihre Werbung stehen!

\begin{itemize}
  \item[(a)] Definieren Sie Lebesgue-Stieltjes Maß, Verteilungsfunktion.
  \item[(b)] $F_1$ und $F_2$ seien zwei eindimensionale Verteilungsfunktionen. Zeigen Sie, dass
  \begin{align*}
    F(x_1, x_2) = F_1(x_1) F_2(x_2)
  \end{align*}
  eine zweidimensionale Verteilungsfunktion ist.
\end{itemize}

\end{exercise}

\begin{solution}

(a)

\begin{itemize}

  \item $\mu \enspace \text{Lebesuge-Stiletjes Maß} : \Leftrightarrow$
  \begin{itemize}
    \item $\mu: (\R^k, \mathfrak{B}_k) \to \R, \text{Maß}$
    \item $\Forall A \in \mathfrak{B}_k, \text{beschr.}: \mu(A) < \infty$
  \end{itemize}

  \item $F: \R \to \R, \text{Verteilungsfunktion von} \enspace \mu : \Leftrightarrow$
  \begin{itemize}
    \item $\mu \enspace \text{Lebesgue-Stieltjes Maß}$
    \item $\Forall a \leq b \in \R: \mu((a, b]) = F(b) - F(a)$
  \end{itemize}

\end{itemize}

(b)

\begin{itemize}

  \item \Quote{Rechtsstetigkeit}: $F$ ist als Produkt der rechtsstetigen $F_1, F_2$ wieder rechtsstetig.

  \item \Quote{nichtnegativer Differenzenoperator}: Seien $a, b \in \R^2$, mit $a \leq b$.
  \begin{align*}
    \Delta^{(a, b)} F
    & =
    F(b_1, b_2) - F(a_1, b_2) - F(b_1, a_2) + F(a_1, a_2) \\
    & =
    F_1(b_1) F_2(b_2) -
    F_1(a_1) F_2(b_2) -
    F_1(b_1) F_2(a_2) +
    F_1(a_1) F_2(a_2) \\
    & =
    F_1(b_1) (F_2(b_2) - F_2(a_2)) -
    F_1(a_1) (F_2(b_2) - F_2(a_2)) \\
    & =
    \underbrace{(F_1(b_1) - F_1(a_1))}_{\geq 0}
    \underbrace{(F_2(b_2) - F_2(a_2))}_{\geq 0}
    \geq 0
  \end{align*}

\end{itemize}

\end{solution}
