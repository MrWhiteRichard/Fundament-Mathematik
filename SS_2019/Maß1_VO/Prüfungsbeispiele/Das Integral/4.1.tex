\begin{exercise}

Es sei

\begin{align*}
  F(x) =
  \begin{cases}
    0         & \text{für} \enspace x < 0 \\
    x         & \text{für} \enspace 0 \leq x < 1 \\
    (x + 1)^2 & \text{für} \enspace 1 \leq x \leq 2 \\
    0         & \text{für} \enspace x \geq 2
  \end{cases}
\end{align*}

\begin{itemize}
  \item[(a)] Zeigen Sie: $F$ ist eine Verteilungsfunktion
  \item[(b)] Bestimmen Sie $\Int{f}{\mu_F}$ für $f(x) = x^2$.
\end{itemize}

\end{exercise}

% --------------------------------------------------------------------------------

\begin{solution}

(a) Hier könnte Ihre Werbung stehen!

\begin{itemize}

  \item \Quote{Rechtsstetigkeit}: $F$ ist stückweise stetig und $\Forall x = 0, 1, 2: F \text{ist rechtsstetig bei} \enspace x$.

  \item \Quote{Steigende Monotonie}: $F$ ist stückweise monoton steigend und $\Forall x = 0, 1, 2: F(x - 0) \leq F(x)$.

\end{itemize}

(b)

\begin{align*}
  \Int{f}{\mu_F}
  & =
  \Int[-\infty][0]{f(x) \underbrace{F^\prime(x)}_0}
  +
  \Int[0][1]{f(x) \underbrace{F^\prime(x)}_1}
  +
  \Int[1][2]{f(x) \underbrace{F^\prime(x)}_{2 (x+1)}}
  +
  \Int[2][\infty]{f(x) \underbrace{F^\prime(x)}_0} \\
  & +
  f(0) \underbrace{(F(0) - F(0 - 0))}_0
  +
  \underbrace{f(1)}_1 \underbrace{(F(1) - F(1 - 0))}_{= 4-1 = 3}
  +
  \underbrace{f(2)}_4 \underbrace{(F(2) - F(2 - 0))}_{= 9-9 = 0} \\
  & =
  \frac{1}{3}
  +
  2 \underbrace
  {
    \Int[1][2]{x^2 (x + 1)}{x}
  }_{
    = \frac{x^4}{4} |_1^2 + \frac{x^3}{3} |_1^2
    = 4 - \frac{1}{4} + \frac{8}{3} - \frac{1}{3}
  }
  =
  \frac{25}{2}
\end{align*}

\end{solution}
