\setcounter{exercise}{0}

\section{Das Integral}

--------------------------------------------------------------------------------

\begin{exercise}

Es sei

\begin{align*}
  F(x) =
  \begin{cases}
    0         & \text{für} \enspace x < 0 \\
    x         & \text{für} \enspace 0 \leq x < 1 \\
    (x + 1)^2 & \text{für} \enspace 1 \leq x \leq 2 \\
    0         & \text{für} \enspace x \geq 2
  \end{cases}
\end{align*}

\begin{itemize}
  \item[(a)] Zeigen Sie: $F$ ist eine Verteilungsfunktion
  \item[(b)] Bestimmen Sie $\Int{f}{\mu_F}$ für $f(x) = x^2$.
\end{itemize}

\end{exercise}

\begin{solution}

(a) Hier könnte Ihre Werbung stehen!

\begin{itemize}

  \item \Quote{Rechtsstetigkeit}: $F$ ist stückweise stetig und $\Forall x = 0, 1, 2: F \text{ist rechtsstetig bei} \enspace x$.

  \item \Quote{Steigende Monotonie}: $F$ ist stückweise monoton steigend und $\Forall x = 0, 1, 2: F(x - 0) \leq F(x)$.

\end{itemize}

(b)

\begin{align*}
  \Int{f}{\mu_F}
  & =
  \Int[-\infty][0]{f(x) \underbrace{F^\prime(x)}_0}
  +
  \Int[0][1]{f(x) \underbrace{F^\prime(x)}_1}
  +
  \Int[1][2]{f(x) \underbrace{F^\prime(x)}_{2 (x+1)}}
  +
  \Int[2][\infty]{f(x) \underbrace{F^\prime(x)}_0} \\
  & +
  f(0) \underbrace{(F(0) - F(0 - 0))}_0
  +
  \underbrace{f(1)}_1 \underbrace{(F(1) - F(1 - 0))}_{= 4-1 = 3}
  +
  \underbrace{f(2)}_4 \underbrace{(F(2) - F(2 - 0))}_{= 9-9 = 0} \\
  & =
  \frac{1}{3}
  +
  2 \underbrace
  {
    \Int[1][2]{x^2 (x + 1)}{x}
  }_{
    = \frac{x^4}{4} |_1^2 + \frac{x^3}{3} |_1^2
    = 4 - \frac{1}{4} + \frac{8}{3} - \frac{1}{3}
  }
  =
  \frac{25}{2}
\end{align*}

\end{solution}

--------------------------------------------------------------------------------

\begin{exercise}

Es sei

\begin{align*}
  F(x) =
  \begin{cases}
                            & \text{wenn} \enspace x < 0, \\
    1 - \frac{1}{2} e^{-x}  & \text{wenn} \enspace x \geq 0.
  \end{cases}
\end{align*}

\begin{itemize}
  \item[(a)] Zeigen Sie: $F$ ist eine Verteilungsfunktion.
  \item[(b)] Bestimmen Sie $\Int{f(x)}{\mu_F(x)}$ für $f(x) = e^{-x}$.
\end{itemize}

\end{exercise}

\begin{solution}

(a) Hier könnte Ihre Werbung stehen!

\begin{itemize}

  \item \Quote{Rechtsstetigkeit}: $F$ ist stückweise stetig und $\Forall x = 0: F \text{ist rechtsstetig bei} \enspace x$.

  \item \Quote{Steigende Monotonie}: $F$ ist stückweise monoton steigend und $\Forall x = 0: F(x - 0) \leq F(x)$.

\end{itemize}

(b)

\begin{align*}
  \Int{f}{\mu_F}
  & =
  \Int[-\infty][0]{f(x) \underbrace{F^\prime(x)}_0}
  +
  \Int[0][\infty]
  {
    \underbrace{f(x)}_{e^{-x}}
    \underbrace{F^\prime(x)}_{\frac{1}{2} e^{-x}}
  }
  +
  \underbrace{f(0)}_1 \underbrace{(F(0) - F(0 - 0))}_\frac{1}{2}
  & =
  \frac{1}{2} \underbrace
  {
    \Int[0][\infty]{e^{-2x}}{x}
  }_{
    = -\frac{1}{2} e^{-2x} |_0^\infty
    = -\frac{1}{2}
  }
  +
  \frac{1}{2}
  =
  -\frac{1}{4} + \frac{1}{2} = \frac{1}{4}
\end{align*}

\end{solution}

--------------------------------------------------------------------------------

\begin{exercise}

Hier könnte Ihre Werbung stehen!

\begin{itemize}
  \item[(a)] Definieren Sie das Integral einer nichtnegativen Treppenfunktion, einer nichtnegativen messbaren Funktion, einer messbaren und einer fast uberall messbaren Funktion.
  \item[(b)] Es sei
  \begin{align*}
    F(x) =
    \begin{cases}
      0     & \text{wenn} \enspace  x < 0, \\
      x + 1 & \text{wenn} \enspace 1 \leq x < 2, \\
      2x^2  & \text{wenn} \enspace 2 \leq x < 3, \\
      20    & \text{wenn} \enspace 3 \leq x.
    \end{cases}
  \end{align*}
  Überzeugen Sie sich, dass $F$ eine Verteilungsfunktion ist und bestimmen Sie $\Int{f}{\mu_F}$ für $f(x) = e^x$.
  \item[(c)] Formulieren und beweisen Sie den Satz von der Konvergenz durch Majorisierung.
\end{itemize}

\end{exercise}

\begin{solution}

(a) Sei $t = \sum_{i=1}^n a_i A_i$ eine nichtnegative, messbare Treppenfunktion auf $(\Omega, \mathfrak{S}, \mu)$, dann

\begin{align*}
  \Int{t}{\mu} := \sum_{i=1}^n a_i \mu(A_i).
\end{align*}

Sei $f$ eine nichtnegative, messbare Funktion, dann

\begin{align*}
  \Int{f}{\mu} := \sup \Bbraces{\Int{t}{\mu}: f \geq t \geq 0, t \text{Treppenf.}}.
\end{align*}

Sei $f$ eine messbare Funktion, dann

\begin{align*}
  \Int{f}{\mu} := \Int{f^+}{\mu} - \Int{f^-}{\mu}.
\end{align*}

Sei $f$ eine fast überall messbare Funktion, d.h. $\Exists N \in \mathfrak{S}: \mu(N) = 0, fN^\complement \enspace \text{messb.}$, dann

\begin{align*}
  \Int{f}{\mu} := \Int[N^\complement]{f}{\mu}.
\end{align*}

\end{solution}

--------------------------------------------------------------------------------

\begin{exercise}

$(X_n)$ sei eine Folge von unabhängigen, auf $[0, 1]$ gleichverteilten Zufallsvariablen. Gegen welchen Wert konvergiert

\begin{align*}
  (\prod_{i=1}^n)^\frac{1}{n}?
\end{align*}

\end{exercise}

\begin{solution}

$\ln$ ist messbar, also auf $(\ln X_n)$ unabhängig.

\begin{align*}
  \lim_{n \to \infty}
  \pbraces{\prod_{i=1}^n X_i}^\frac{1}{n}
  & =
  \lim_{n \to \infty}
  \exp \ln \pbraces{\pbraces{\prod_{i=1}^n X_i}^\frac{1}{n}}
  =
  \lim_{n \to \infty}
  \exp \pbraces{\frac{1}{n} \sum_{i=1}^n \ln X_i} \\
  & =
  \exp \pbraces
  {\lim_{n \to \infty} \frac{1}{n} \sum_{i=1}^n \ln X_i}
  =
  \exp \E(\ln X_1)
  =
  \exp \underbrace{\Int[0][1]{\ln(x) \frac{1}{1-0}}{x}}_{-\infty} = 0
\end{align*}

\end{solution}

--------------------------------------------------------------------------------

\begin{exercise}

$(X_n)$ sei eine Folge von unabhängigen Zufallsvariablen mit $\P(X_n = 1) = \P(X_n = -1) = 1/2$, $(a_n)$ eine Folge von reellen Zahlen. Zeigen Sie, dass die Reihe

\begin{align*}
  \sum_{n \in \N} a_n X_n
\end{align*}

genau dann fast sicher konvergiert, wenn

\begin{align*}
  \sum_{n \in \N} a_n^2 < \infty.
\end{align*}

\end{exercise}

\begin{solution}

Wir benützen den \Quote{Dreireihensatz von Kolmogorov}: Wenn $(X_n)$ unabhängige Zufallsvariablen sind, dann

\begin{align*}
  \sum_{n \in \N} X_n
  \enspace \text{konv. f.s.}
  \Leftrightarrow
  \Forall \epsilon > 0:
  \sum_{n \in \N} \P(|X_n| > \epsilon),
  \sum_{n \in \N} \E(X_n [|X_n| \leq \epsilon]),
  \sum_{n \in \N} \V(X_n [|X_n| \leq \epsilon])
  < \infty.
\end{align*}

Zuerst bemerken wir, dass

\begin{align*}
  1
  =
  \P(X_n = 1) + \P(X_n = -1)
  =
  \P(X_n = \pm 1)
  =
  \P(|X_n| = 1)
  =
  \P(X_n^2 = 1)
  \Rightarrow
  \P(X_n \neq \pm 1) = 0.
\end{align*}

Damit berechnet man die Erwartungswerte

\begin{itemize}

  \item $\E(X_n) = 1 \cdot \P(X_n = 1) + (-1) \cdot \P(X_n = -1) = 0$,

  \item $\E(X_n [|X_n| \leq \epsilon]) \leq \E(X_n) = 0$,

  \item $\E(X_n^2) = 1 \cdot \P(X_n^2 = 1) = 1$,

  \item $\E(X_n^2 [|X_n| \leq \epsilon]) = \E(X_n^2 [X_n^2 \leq \epsilon^2]) =
  \begin{cases}
    \E(X_n^2) = 1 & \text{wenn} \enspace \epsilon^2 \geq 1 \\
    0             & \text{sonst}
  \end{cases}$,

\end{itemize}

und Varianzen (mit dem \Quote{Verschiebungssatz von Steiner})

\begin{itemize}

  \item $\V(X_n) = \E(X_n^2) - \E(X_n)^2 = 1^2 - 0 = 1$

  \item $\V(X_n [X_n \leq \epsilon]) = \E(X_n^2 [X_n \leq \epsilon]) - \E(X_n [X_n \leq \epsilon])^2 =
  \begin{cases}
    \E(X_n^2) = 1 & \text{wenn} \enspace \epsilon^2 \geq 1 \\
    0             & \text{sonst}
  \end{cases}$

\end{itemize}

\begin{itemize}

  \item[\Quote{$\Rightarrow$}] $\Exists \epsilon > 0$
  \begin{align*}
    \infty
    >
    \sum_{n \in \N} \V(a_n X_n [|a_n X_n| \leq \epsilon])
    =
    \sum_{n \in \N} a_n^2
  \end{align*}

  \item[\Quote{$\Leftarrow$}]
  \begin{itemize}

    \item $a_n^2 \to 0
    \Rightarrow
    |a_n| \to 0
    \Rightarrow
    |a_n X_n| \to 0$,
    d.h. $\Forall \epsilon > 0: \Exists N \in \N: \Forall n \geq N: |a_n X_n| < \epsilon$ und
    \begin{align*}
      \sum_{n \in \N} \P(X_n > \epsilon) < \infty.
    \end{align*}

    \item
    \begin{align*}
      \sum_{n \in \N} \E(a_n X_n [|a_n X_n| \leq \epsilon])
      \leq
      \sum_{n \in \N} a_n \E(X_n) = 0
    \end{align*}

    \item
    \begin{align*}
      \sum_{n \in \N} \E(a_n X_n [|a_n X_n| \leq \epsilon])
      \leq
      \sum_{n \in \N} a_n^2 \V(X_n)
      =
      \sum_{n \in \N} a_n^2
    \end{align*}

  \end{itemize}

\end{itemize}

\end{solution}

--------------------------------------------------------------------------------
