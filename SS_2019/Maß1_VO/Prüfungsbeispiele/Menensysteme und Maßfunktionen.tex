\setcounter{exercise}{0}

\section{Menensysteme und Maßfunktionen}

--------------------------------------------------------------------------------

\begin{exercise}

Hier könnte Ihre Werbung stehen!

\begin{itemize}
  \item[(a)] Definieren Sie Maßfunktion, endliche Maßfunktion, sigmaendliche Maßfunktion, äußere Maßfunktion.
  \item[(b)] Formulieren und beweisen Sie den Fortsetzungssatz für Maßfunktionen.
  \item[(c)] $\Omega$ sei eine beliebige Menge für $A \subseteq \Omega$
  \begin{align*}
    \mu^\ast(A) =
    \begin{cases}
      0, & \text{wenn} \enspace A = \emptyset, \\
      1, & \text{sonst}.
    \end{cases}
  \end{align*}
  Zeigen Sie, dass $\mu^\ast$ eine äußere Maßfunktion ist und bestimmen Sie das System der $\mu^\ast$-messbaren Mengen.
\end{itemize}

\end{exercise}

\begin{solution}

Sei $\mu: \mathfrak{C} \to \R$ eine Mengenfunktion auf dem Mengensystem $\emptyset \neq \mathfrak{C} \subseteq 2^\Omega$.

\begin{itemize}

  \item $\mu \Text{Maßfunktion} : \Leftrightarrow$
  \begin{itemize}
    \item $\Forall A \in \mathfrak{C}: \mu(A) \geq 0,$
    \item $\mu \enspace \sigma \text{-additiv}, \enspace
    \text{d.h.}
    \Forall (A_n) \in \mathfrak{C}, \text{disj.}, \text{höchst. abz.}:
    A := \sum_{n \in \N} A_n \in \mathfrak{C} \Rightarrow
    \mu(A) = \sum_{n \in \N} \mu(A_n)$
  \end{itemize}

  \item $\mu \Text{endliche Maßfunktion} : \Leftrightarrow$
  \begin{itemize}
    \item $\mu \enspace \text{Maßfunktion},$
    \item $\Forall A \in \mathfrak{C}: \mu(A) < \infty$
  \end{itemize}

  \item $\mu \Text{sigmaendliche Maßfunktion} : \Leftrightarrow$
  \begin{itemize}
    \item $\mu \enspace \text{Maßfunktion},$
    \item $\Forall A \in \mathfrak{C}: \Exists (A_n) \in \mathfrak{C}: A \subseteq \bigcup_{n \in \N} A_n, \Forall n \in \N: \mu(A_n) < \infty$
  \end{itemize}

\end{itemize}

Sei $\mu^\ast: 2^\Omega \to \R$ eine weitere Mengenfunktion.

\begin{itemize}

  \item $\mu^\ast \Text{äußere Maßfunktion} : \Leftrightarrow$
  \begin{itemize}
    \item $\mu^\ast(\emptyset) = 0,$
    \item $\Forall A \in 2^\Omega: \mu^\ast(A) \geq 0,$
    \item $\Forall A, B \in 2^\Omega:
    A \subseteq B \Rightarrow
    \mu^\ast(A) \leq \mu^\ast(B),$
    \item $\Forall A, (B_n) \in 2^\Omega:
    A \subseteq \bigcup_{n \in \N} B_n \Rightarrow
    \mu^\ast(A) \leq \sum_{n \in \N} \mu^\ast(B_n)$
  \end{itemize}

\end{itemize}

Der Beweis des Fortsetzungssatz für Maßfunktionen ist nicht teil des Prüfungsstoffs. Dennoch, er besagt folgendes.

\begin{theorem*}[Fortsetzungssatz für Maßfunktionen]

$\mu$ sei ein Maß auf einem Semiring $\mathfrak{T}$. Dann kann man auf dem von $\mathfrak{T}$ erzeugten Sigmaring ein Maß finden, dass auf $\mathfrak{T}$ mit $\mu$ übereinstimmt. \\
Wenn $\mu$ auf $\mathfrak{T}$ sigmaendlich ist, dann ist die Fortsetzung auf den erzeugten Sigmaring eindeutig bestimmt.

\end{theorem*}

Die ersten drei Eigenschaften sind offensichtlich. Seien zuletzt $A, (B_n) \in 2^\Omega$ mit $A \subseteq \bigcup_{n \in \N} B_n$. Falls $A = \emptyset$, sind wir (wegen der zweiten Eigenschaft) fertig und sonst $\Exists n \in \N: B_n \neq \emptyset$. \\

Das System der Carathéodory-messbaren Mengen lautet wie folgt.

\begin{align*}
  \mathfrak{M}_{\mu^\ast} :=
  \Bbraces
  {
    A \subseteq \Omega:
    \Forall B \subseteq \Omega:
    \mu^\ast(B) = \mu^\ast(B \cap A) + \mu^\ast(B \cap A^\complement)
  }
\end{align*}

Weil dieses System eine $\sigma$-Algebra ist, sind $\emptyset, \Omega$ messbar. $A \subseteq \Omega$ mit $A \neq \emptyset, \Omega$, ist wegen $B := \Omega$ nicht messbar, weil $A^\complement \neq \emptyset$ und damit $\mathfrak{M}_{\mu^\ast} = \Bbraces{\emptyset, \Omega}$.

\end{solution}

--------------------------------------------------------------------------------

\begin{exercise}

Gegeben sei die folgende Funktion auf $2^\R$:

\begin{align*}
  \mu^\ast(A) =
  \begin{cases}
    0 & \text{für} \enspace A = \emptyset, \\
    1 & \text{wenn} \enspace 1 \leq |A| < \infty, \\
    2 & \text{sonst}.
  \end{cases}
\end{align*}

Zeigen Sie, dass $\mu^\ast$ eine äußere Maßfunktion ist und bestimmen Sie das System der $\mu^\ast$-messbaren Mengen.

\end{exercise}

\begin{solution}

Die ersten drei Eigenschaften sind offensichtlich. Seien zuletzt $A, (B_n) \in 2^\Omega$ mit $A \subseteq \bigcup_{n \in \N} B_n$. Falls $A = \emptyset$, sind wir (wegen der zweiten Eigenschaft) fertig. Für $1 \leq |A| < \infty$, muss $\Exists n \in \N: 1 \leq |B_n| < \infty$. Ansonsten, sind die (unendlich vielen) $\omega \in A$ über ganz $(B_n)$ verteilt. Nun gilt aber $\Exists n \in \N: |B_n| = \infty$ oder $\Exists n \neq m \in \N: 1 \leq |B_n|, |B_m| < \infty$. \\

Das System der Carathéodory-messbaren Mengen lautet wie folgt.

\begin{align*}
  \mathfrak{M}_{\mu^\ast} :=
  \Bbraces
  {
    A \subseteq \Omega:
    \Forall B \subseteq \Omega:
    \mu^\ast(B) = \mu^\ast(B \cap A) + \mu^\ast(B \cap A^\complement)
  }
\end{align*}

Weil dieses System eine $\sigma$-Algebra ist, sind $\emptyset, \Omega$ messbar. Sei also $A \subseteq \Omega$ mit $A \neq \emptyset, \Omega$.

\begin{itemize}

  \item[Fall $1$)] $A \enspace \text{endl.} \Rightarrow A^\complement \Text{unendl.}$
  \begin{align*}
    2 = \mu^\ast(\R) =
    \underbrace{\mu^\ast(\R \cap A)}_1 +
    \underbrace{\mu^\ast(\R \cap A^\complement)}_2 = 3
  \end{align*}

  \item[Fall $2$)] $A \enspace \text{unendl.}$
  \begin{itemize}
    \item[Fall a)] $A^\complement \enspace \text{endl.} \Rightarrow$ analog zu Fall $1$
    \item[Fall b)] $A^\complement \enspace \text{unendl.}$
    \begin{align*}
      2 = \mu^\ast(\R) =
      \underbrace{\mu^\ast(\R \cap A)}_2 +
      \underbrace{\mu^\ast(\R \cap A^\complement)}_2 = 4
    \end{align*}
  \end{itemize}

\end{itemize}

Damit, muss $\mathfrak{M}_{\mu^\ast} = \Bbraces{\emptyset, \Omega}$.

\end{solution}

--------------------------------------------------------------------------------

\begin{exercise}

Hier könnte Ihre Werbung stehen!

\begin{itemize}
  \item[(a)] Definieren Sie Ring, Semiring, monotones System, Dynkin-System, Algebra, Sigmaalgebra.
  \item[(b)] Zeigen Sie: Wenn $\mathfrak{R}$ ein Ring ist, dann stimmt das von $\mathfrak{R}$ erzeugte monotone System mit dem erzeugten Sigmaring überein.
\end{itemize}

\end{exercise}

\begin{solution}

Hier könnte Ihre Werbung stehen!

\begin{itemize}

  \item $\emptyset \neq \mathfrak{R} \subseteq 2^\Omega \enspace \text{Ring} : \Leftrightarrow \Forall A, B \in \mathfrak{R}:$
  \begin{itemize}
    \item $A \cup B \in \mathfrak{R}$
    \item $A \setminus B \in \mathfrak{R}$
  \end{itemize}

  \item $\emptyset \neq \mathfrak{T} \subseteq 2^\Omega \enspace \text{Semiring} : \Leftrightarrow \Forall A, B \in \mathfrak{T}:$
  \begin{itemize}
    \item $A \cap B \in \mathfrak{T},$
    \item $A \subseteq B \Rightarrow \Exists C_1, \ldots, C_n \in \mathfrak{T}, \Text{disj.}:
    B \setminus A = \sum_{i=1}^n C_i,$
    \item $\Forall k = 1, \ldots, n:
    A \cup \sum_{i=1}^k C_i \in \mathfrak{T}$
  \end{itemize}

  \item $\mathfrak{M} \subseteq 2^\Omega \enspace \text{monotones System} : \Leftrightarrow \Forall (A_n) \in \mathfrak{M}, \Text{mon.}: \lim_{n \to \infty} A_n \in \mathfrak{M}$

  \item $\emptyset \neq \mathfrak{D} \subseteq 2^\Omega \enspace \text{Dynkin-System} : \Leftrightarrow$
  \begin{itemize}
    \item $\Forall A, B \in \mathfrak{D}:
    A \subseteq B \Rightarrow B \setminus A \in \mathfrak{D}$
    \item $\Forall (A_n) \in \mathfrak{D}, \text{disj.}: \sum_{n \in \N} A_n \in \mathfrak{D}$
    \item $\Omega \in \mathfrak{D}$
  \end{itemize}

  \item $\emptyset \neq \mathfrak{A} \subseteq 2^\Omega \enspace \text{Algebra} : \Leftrightarrow$
  \begin{itemize}
    \item $\mathfrak{A} \enspace \text{Ring},$
    \item $\Omega \in \mathfrak{A}$
  \end{itemize}

  \item $\emptyset \neq \mathfrak{A}_\sigma \subseteq 2^\Omega \enspace \text{Sigmaalgebra} : \Leftrightarrow$
  \begin{itemize}
    \item $\Forall (A_n) \in \mathfrak{A}_\sigma, \text{disj.}: \sum_{n \in \N} A_n \in \mathfrak{A}_\sigma$
    \item $\Forall A, B \in \mathfrak{A}_\sigma: A \setminus B \in \mathfrak{A}_\sigma,$
    \item $\Omega \in \mathfrak{A}_\sigma$
  \end{itemize}

\end{itemize}

Der nächste Teil ist genau das \Quote{Monotone Class Theorem}! Siehe Skript.

\end{solution}

--------------------------------------------------------------------------------

\begin{exercise}

Hier könnte Ihre Werbung stehen!

\begin{itemize}
  \item[(a)] Definieren sie Maßfunktion, endliche Maßfunktion, sigmaendliche Maßfunktion, äußere Maßfunktion, von einem Maß erzeugte äußere Maßfunktion.
  \item[(b)] Zeigen Sie, dass eine äußere Maßfunktion auf dem System der messbaren Mengen ein Maß bildet.
\end{itemize}

\end{exercise}

\begin{solution}

(a) Sei $\inf \emptyset := \infty$. Dann ist die äußere Maßfunktion, für ein Maß $\mu$ auf einem Ring $\mathfrak{R}$, definiert als

\begin{align*}
  \mu^\ast:
  2^\Omega \to \bar \R,
  A \mapsto \inf \Bbraces{\sum_{n \in \N} \mu(E_n): (E_n) \in \mathfrak{R}, A \subseteq \bigcup_{n \in \N} E_n}.
\end{align*}

Für den Rest: siehe Aufgabe 1 (a). \\

(b) Siehe Skript.

\end{solution}

--------------------------------------------------------------------------------

\begin{exercise}

Hier könnte Ihre Werbung stehen!

\begin{itemize}
  \item[(a)] $\mathfrak{S}$ sei eine Sigmaalgebra über $\Omega$, $\mathfrak{C} \subset \Omega$. Zeigen Sie
  \begin{align*}
    \mathfrak{A}_\sigma(\mathfrak{S} \cup \Bbraces{C}) =
    \Bbraces{(A \cap C) \cup (B \cap C^\complement), A, B \in \mathfrak{S}}.
  \end{align*}
  \item[(b)] $\mathfrak{R}$ sei ein Ring über $\Omega$. Zeigen Sie
  \begin{align*}
    \mathfrak{A}(\mathfrak{R}) =
    \mathfrak{R} \cup \Bbraces{A: A^\complement \in \mathfrak{R}}.
  \end{align*}
\end{itemize}

\end{exercise}

\begin{solution}

(a) \Quote{$\supseteq$}: Seien $A, B \in \mathfrak{S}$, dann auch $A, B, C \in \mathfrak{A}_\sigma(\mathfrak{S} \cup \Bbraces{C})$. Als $\sigma$-Algebra ist diese bezüglich \Quote{$\cap$}, \Quote{$\cup$}, \Quote{$^\complement$} abgeschlossen. \\

\Quote{$\subseteq$}: Die linke Seite ist die kleinste $\sigma$-Algebra, die $\mathfrak{S} \cup \Bbraces{C}$ enthält. Sei $\mathfrak{S}^\prime$ die rechte Seite, dann muss

\begin{itemize}
  \item $\mathfrak{S} \subseteq \mathfrak{S}^\prime$, wenn man $A = B$ wählt und
  \item $C \in \mathfrak{S}^\prime$, wenn $A := \Omega$, $B := \emptyset$.
\end{itemize}

Wir weisen also noch nach, dass $\mathfrak{S}^\prime$ eine $\sigma$-Algebra ist.

\begin{itemize}

  \item \Quote{Enthält Grundmenge}: $\Omega \in \mathfrak{S} \subseteq \mathfrak{S}^\prime$

  \item \Quote{Stabil bzgl. abzählbaren Vereinigungen}: Sei $(E_n) \in \mathfrak{S}^\prime$, dann $\Exists (A_n), (B_n) \in \mathfrak{S}:$
  \begin{align*}
    \bigcup_{n \in \N} \pbraces{A_n \cap C} \cup \pbraces{B_n \cap C^\complement} =
    \Bigg ( \underbrace{\bigcup_{n \in \N} A_n}_{\in \mathfrak{S}} \cap C \Bigg ) \cup
    \Bigg ( \underbrace{\bigcup_{n \in \N} B_n}_{\in \mathfrak{S}} \cap C^\complement \Bigg ) \cup
    \in \mathfrak{S}^\prime
  \end{align*}

  \item \Quote{Stabil bzgl. Differenzen}: Diese filetieren wir in
  \begin{itemize}

    \item \Quote{Stabil bzgl. Komplement}: Sei $E \in \mathfrak{S}^\prime$, dann $\Exists A, B \in \mathfrak{S}:$
    \begin{align*}
      E^\complement
      = \pbraces{\pbraces{A \cap C} \cup \pbraces{B \cap C^\complement}}^\complement
      & = \pbraces{A^\complement \cup C^\complement} \cap \pbraces{B^\complement \cup C} \\
      & = \underbrace
          {
            \pbraces{A^\complement \cap B^\complement}
          }_{
            \in \mathfrak{S}
          } \cup
          \underbrace
          {
            \pbraces{A^\complement \cap C} \cup \pbraces{C^\complement \cap B^\complement}
          }_{
            \in \mathfrak{S}^\prime
          } \cup
          \underbrace
          {
            \pbraces{C^\complement \cap C}
          }_\emptyset
          \in \mathfrak{S}^\prime
    \end{align*}

    \item \Quote{Stabil bzgl. (endlichen) Durchschnitten}: Seien $E_1, E_2 \in \mathfrak{S}^\prime$, dann $\Exists A_1, B_1, A_2, B_2 \in \mathfrak{S}:$
    \begin{align*}
      E_1 \cap E_2
      & = \pbraces{\pbraces{A_1 \cap C} \cup \pbraces{B_1 \cap C^\complement}} \cup
        \pbraces{\pbraces{A_2 \cap C} \cup \pbraces{B_2 \cap C^\complement}} \\
      & = \pbraces{\pbraces{A_1 \cap C} \cap \pbraces{A_2 \cap C}} \cup
          \underbrace
          {
            \Big (
            \pbraces{B_1 \cap C^\complement} \cap \pbraces{A_2 \cap C}
            \Big )
          }_\emptyset \\
      & \cup \underbrace
          {
            \Big (
            \pbraces{A_1 \cap C} \cap \pbraces{B_2 \cap C^\complement}
            \Big )
            }_\emptyset
            \pbraces
          {
            \pbraces{B_1 \cap C^\complement} \cap
            \pbraces{B_2 \cap C^\complement}
          } \\
      & = \pbraces{A_1 \cap A_2 \cap C} \cup
          \pbraces{B_1 \cap B_2 \cap C^\complement}
          \in \mathfrak{S}^\prime
    \end{align*}

  \end{itemize}

\end{itemize}

(b) \Quote{$\supseteq$}: Offensichtlich, gilt $\mathfrak{A}(\mathfrak{R}) \supseteq \mathfrak{R}$. Weil Algebren die Grundmenge enthalten und stabil bzgl. Differenzen sind, sind sie es auch bzgl. Komplementen. Damit folgt auch $\mathfrak{A}(\mathfrak{R}) \supseteq \mathfrak{R}^\complement$, wobei $^\complement$ punktweise zu verstehen ist. \\

\Quote{$\subseteq$}: So wie in (a), bemerkt man, dass die rechte Seite $\mathfrak{A} \supseteq \mathfrak{R}$ enhält und weist nach, dass $\mathfrak{A}$ eine Algebra ist.

\end{solution}

--------------------------------------------------------------------------------

\begin{exercise}

Hier könnte Ihre Werbung stehen!

\begin{itemize}
  \item[(a)] Definieren Sie Ring, Semiring, monotones System, Dynkin-System.
  \item[(b)] Gegeben sei der Wahrscheinlichkeitsraum $(\Omega, \mathfrak{S}, \P)$. Für $A \in \mathfrak{S}$ sei
  \begin{align*}
    \mathfrak{U}(A) = \Bbraces{B: \P(A \cap B) = \P(A) \P(B)}
  \end{align*}
  das System aller Mengen, die von $A$ unabhängig sind. Zeigen Sie, dass $\mathfrak{U}(A)$ ein Dynkin-System ist.
\end{itemize}

\end{exercise}

\begin{solution}

(a) Siehe Aufgabe 3 (a) \\

(b)

\begin{itemize}

  \item \Quote{Stabil bzgl. Differenzen von Teilmengen}: Seien $B, C \in \mathfrak{U}(A)$ mit $B \subseteq C$, dann gilt Folgendes.
  \begin{align*}
    \P(A \cap C \setminus B)
    & = \P(A) \P(C \setminus B)
    \Leftrightarrow \\
    \P(A) \P(B) + \P(A \cap C \setminus B)
    & = \P(A) \P(C \setminus B) + \P(A) \P(B)
      = \P(A) \P(C)
    \Leftrightarrow \\
    \P(A \cap C) + \P(A) \P(B)
      =\P(A \cap B) + \P(A) \P(B) + \P(A \cap C \setminus B)
    & = \P(A) \P(C) + \P(A \cap B)
  \end{align*}

  \item \Quote{Stabil bzgl. abzählbaren, disjunkten Vereinigungen}: Sei $(B_n) \in \mathfrak{D}$ disjunkt, dann
  \begin{align*}
    \P \pbraces{A \cap \sum_{n \in \N} B_n}
    =
    \sum_{n \in \N} \P(A \cap B_n)
    =
    \sum_{n \in \N} \P(A) \P(B_n)
    =
    \P(A) \P \pbraces{\sum_{n \in \N} B_n}.
  \end{align*}

  \item \Quote{Enthält Grundmenge}: $\P(A \cap \Omega) = \P(A) = \P(A) \P(\Omega)$

\end{itemize}

\end{solution}

--------------------------------------------------------------------------------

\begin{exercise}

Hier könnte Ihre Werbung stehen!

\begin{itemize}
  \item[(a)] Definieren Sie: äußeres Maß, Messbarkeit nach Caratheodory, von einer Maßfunktion erzeugtes Maß.
  \item[(b)] Zeigen Sie: wenn $\mu^\ast_n$ äußere Maße über derselben Menge sind, dann auch $\sup_n \mu^\ast_n$
\end{itemize}

\end{exercise}

\begin{solution}

(a) Siehe Aufgabe 1 (a) und Aufgabe 4 (a). \\

(b) Die ersten drei Eigenschaften sind offensichtlich. Seien also $A, (B_k) \in 2^\Omega$, mit $A \subseteq \bigcup_{k \in \N} B_k$, dann gilt Folgendes.
\begin{align*}
  \sum_{k \in \N} \sup_n \mu^\ast_n(B_k)
  \geq
  \sup_n \sum_{k \in \N} \mu^\ast_n(B_k)
  \geq
  \sup_n \mu^\ast_n(A)
\end{align*}

\end{solution}

--------------------------------------------------------------------------------

\begin{exercise}

$\mu$ und $\nu$ seinen zwei Maße auf dem Ring $\mathfrak{R}$. Zeigen Sie:

\begin{itemize}
  \item[(a)] $(\mu + \nu)^\ast = \mu^\ast + \nu^\ast$
  \item[(b)] $\mathfrak{M}_{\mu^\ast + \nu^\ast} \supseteq \mathfrak{M}_{\mu^\ast} \cap \mathfrak{M}_{\nu^\ast}$
  \item[(c)] Falls $\mu$ und $\nu$ sigmaendlich sind, dann gilt im vorigen Punkt Gleichheit.
\end{itemize}

\end{exercise}

\begin{solution}

(a) \Quote{$\geq$}: Sei $A \subseteq \Omega$, dann folgt die eine Ungleichung wegen \\

\begin{align*}
  (\mu + \nu)^\ast(A)
  & =
    \inf \Bbraces{\sum_{n \in \N} (\mu + \nu)(E_n): (E_n) \in \mathfrak{R}, A \subseteq \sum_{n \in \N} E_n} \\
  & \geq
    \inf \Bbraces{\sum_{n \in \N} \mu(E_n): (E_n) \in \mathfrak{R}, A \subseteq \sum_{n \in \N} E_n} +
    \inf \Bbraces{\sum_{n \in \N} \nu(E_n): (E_n) \in \mathfrak{R}, A \subseteq \sum_{n \in \N} E_n} \\
  & =
    \mu^\ast(A) + \nu^\ast(A).
\end{align*}

\Quote{$\leq$}: Für die Andere, finden wir für beliebiges $\epsilon > 0$ zwei disjunkte Überdeckungen $(B_n), (C_m) \in \mathfrak{R}$ von $A$, sodass

\begin{align*}
  \mu^\ast(A) \leq \sum_{n \in \N} \mu^\ast(B_n) \leq \mu^\ast(A) + \frac{\epsilon}{2}, \enspace
  \nu^\ast(A) \leq \sum_{m \in \N} \nu^\ast(C_m) \leq \nu^\ast(A) + \frac{\epsilon}{2}.
\end{align*}

Weil $(B_n \cap C_m)$ wieder eine disjunkte Überdeckungen von $A$ ist, folgt

\begin{align*}
  (\mu + \nu)^\ast(A)
  & \leq
    \sum_{n, m \in \N} (\mu + \nu)^\ast(B_n \cap C_m)
    \leq
    \sum_{n \in \N} \sum_{m \in \N} \mu^\ast(B_n \cap C_m) +
    \sum_{m \in \N} \sum_{n \in \N} \nu^\ast(B_n \cap C_m) \\
  & \leq
    \sum_{n \in \N} \mu^\ast(B_n) +
    \sum_{m \in \N} \nu^\ast(C_m)
    \leq
    \mu^\ast(A) + \nu^\ast(A) + \epsilon.
\end{align*}

\end{solution}

--------------------------------------------------------------------------------

\begin{exercise}

Zeigen Sie: Falls $\mu_n$ für jedes $n$ eine Maßfunktion auf $(\Omega, \mathfrak{S})$ ist, so ist auch $\mu = \sum_n \mu_n$ eine Maßfunktion.

\end{exercise}

\begin{solution}

Offensichtlich, ist $\Forall A \in \mathfrak{S}: \mu(A) \geq 0$. Es fehlt also nur noch die $\sigma$-Additivität von $\mu$. $\emptyset \in \mathfrak{S}$, also wählen wir $(A_k) \in \mathfrak{S}$ disjunkt.

\begin{align*}
  \mu(\sum_{k \in \N} A_k)
  =
  \sum_{n \in \N} \mu_n(\sum_{k \in \N} A_k)
  =
  \sum_{n \in \N} \sum_{k \in \N} \mu_n(A_k)
  =
  \sum_{k \in \N} \sum_{n \in \N} \mu_n(A_k)
  =
  \sum_{k \in \N} \mu(A_k)
\end{align*}

Dabei darf man die beiden Summen vertauschen, weil die Reihen entweder $\infty$ sind, oder absolut (also unbedingt) konvergieren. Der Rest ist die $\sigma$-Additivität von $\mu_n$ und pure Definition.

\end{solution}

--------------------------------------------------------------------------------

\begin{exercise}

Zeigen Sie: Falls $\mu_1$ und $\mu_2$ zwei äußere Maßfunktionen sind, so ist auch $\mu = \max(\mu_1, \mu_2)$ eine äußere Maßfunktion.

\end{exercise}

\begin{solution}

Die ersten drei Punkte sind offensichtlich. Für die Subadditivität sei $A \subseteq \bigcap_{n \in \N} B_n$, dann

\begin{align*}
  \mu(A)
  =
  \max(\mu_1(A), \mu_2(A))
  \leq
  \max(\mu_1(\sum_{n \in \N} B_n), \mu_2(\sum_{n \in \N} B_n))
  =
  \mu(\sum_{n \in \N} B_n).
\end{align*}

\end{solution}

--------------------------------------------------------------------------------

\begin{exercise}

Es sei

\begin{align*}
  \mathfrak{T} = \Bbraces{A \subset \R: |A| \leq 2}.
\end{align*}

\begin{itemize}
  \item[(a)] Zeigen Sie: $\mathfrak{T}$ ist ein Semiring.
  \item[(b)] Bestimmmen Sie den von $\mathfrak{T}$ erzeugten Ring bzw. Sigmaring.
  \item[(c)] Was muss eine Mengenfunktion $\mu$ auf $\mathfrak{T}$ erfüllen, damit sie ein Maß ist.
\end{itemize}

\end{exercise}

\begin{solution}

(a)

\begin{itemize}

  \item \Quote{Stabil bzgl. (endlichen) Durchschnitten}: Seien $A, B \in \mathfrak{T}$, so auch $A \cap B \in \mathfrak{T}$, weil
  \begin{align*}
    |A \cap B| \leq |A|, |B| \leq 2.
  \end{align*}

  \item \Quote{Enthält Leiterpflöcke} Seien $A, B \in \mathfrak{T}$, mit $A \subseteq B$. Nachdem $\Forall x \in \R: \Bbraces{x} \in \mathfrak{T}$, findet man disjunkte $C_1, \leq, C_n: B \setminus A = \sum_{i=1}^n C_i$.

  \item \Quote{Enthält Unterleitern} Offensichtlich gilt auch $\Forall k = 1, \ldots, n: A \cup \sum_{i=1}^k \in \mathfrak{T}$.

\end{itemize}

(b) Dazu gibt es zwei wunderbare Resultate:

\begin{itemize}

  \item $\mathfrak{R}(\mathfrak{T})
  = \Bbraces{\bigcup_{i=1}^n A_i: (A_i)_{i=1}^n \in \mathfrak{T}^n}
  = \Bbraces{\sum   _{i=1}^n A_i: (A_i)_{i=1}^n \in \mathfrak{T}^n}
  = \Bbraces{A \subset \R: |A| < \infty}$.

  \item Laut dem \Quote{Monotone Class Theorem}, gilt
  $\mathfrak{R}_\sigma(\mathfrak{T})
  = \mathfrak{R}_\sigma(\mathfrak{R}(\mathfrak{T}))
  = \mathfrak{M}(\mathfrak{R}(\mathfrak{T}))$.
  Dieses Mengenssytem ist stabil bzgl. $\lim_{n \to \infty} A_n = \bigcup_{n \in \N} A_n$, also genau $\Bbraces{A \subset \R: |A| \leq \aleph_0}$.

\end{itemize}

(c) Siehe Aufgabe 1 (a).

\end{solution}

--------------------------------------------------------------------------------

\begin{exercise}

Es sei $\Omega = (0, 1] \times (0, 1]$ und

\begin{align*}
  \mathfrak{T}
  =
  \Bbraces{(a, b] \times (0, 1]: 0 \leq a \leq b \leq 1}
  \cup
  \Bbraces{(0, 1] \times (a, b]: 0 \leq a \leq b \leq 1}
\end{align*}

Ferner sei

\begin{align*}
  \mu((a, b] \times (0, 1]) = \mu((0, 1] \times (a, b]) = b - a
\end{align*}

\begin{itemize}
  \item[(a)] Zeigen Sie: $\mu$ ist ein Inhalt auf $\mathfrak{T}$.
  \item[(b)] Zeigen Sie $\mathfrak{T}$ ist kein Semiring.
  \item[(c)] Zeigen Sie: Die Fortsetzung von $\mu$ zu einem Inhalt auf dem erzeugten Ring ist nicht eindeutig bestimmt.
\end{itemize}

\end{exercise}

\begin{solution}

(a) Offensichtlich gilt $\Forall A \in \mathfrak{T}: \mu(A) \geq 0$. Zu zeigen, bleibt also nur noch die Additivität. Seien dazu $A_1, \ldots, A_n \in \mathfrak{T}$ disjunkt und $\sum_{i=1}^n A_i \in \mathfrak{T}$. $A_1, \ldots, A_n$ wollen wir aber darstellen, mit $\Forall i = 1, \ldots, n: \Exists a_i, b_i \in (0, 1):$

\begin{align*}
  A_i = (a_i, b_i] \times (0, 1]
  \enspace \text{oder} \enspace
  A_i = (0, 1] \times (a_i, b_i].
\end{align*}

Weil $A_1, \ldots, A_n$ ja disjunkt sind, müssen sie aber alle die selbe, einer der oberen, Darstellungen haben. Weil $\sum_{i=1}^n A_i \in \mathfrak{T}$, müssen $A_1, \ldots, A_n$ auch direkt an einender angrenzen. Das führt zu

\begin{align*}
  \sum_{i=1}^n \mu(A_i)
  =
  \sum_{i=1}^n b_i - a_i
  =
  \mu(\sum_{i=1}^n A_i).
\end{align*}

(b) $\mathfrak{T}$ ist nicht stabil bzgl. (endlichen) Durchschnitten, weil $\Forall a, b, c, d \in (0, 1): a \leq b$, $c \leq d \Rightarrow$

\begin{align*}
  (a, b] \times (0, 1]
  \cap
  (0, 1] \times (c, d]
  =
  (a, b] \times (c, d]
  \notin \mathfrak{T}
\end{align*}

(c)

\end{solution}

--------------------------------------------------------------------------------

\begin{exercise}

Bestimmen Sie die Wahrscheinlichkeit, dass eine zufällig (gleichverteilt) gewählte Permuation von $n$ Elementen keinen Fixpunkt hat.

\end{exercise}

\begin{solution}

Click \href{https://de.wikipedia.org/wiki/Fixpunktfreie_Permutation}{here}!

\end{solution}

--------------------------------------------------------------------------------

\begin{exercise}

Ein Würfel wird einmal geworfen, $X$ sei die erzielte Augenzahl. Anschließend werden in eine Urne $X$ weiße und eine schwarze Kugel gelegt, und eine Kugel wird gezogen. $A$ sei das Ereignis, dass die schwarze Kugel gezogen wurde. Bestimmen Sie

\begin{itemize}
  \item[(a)] $\mathbf{P}(A)$
  \item[(b)] $\mathbf{P}(X = 3 | A)$
\end{itemize}

\end{exercise}

\begin{solution}

(a) Die Urne enthält also $X + 1$ Kugeln und damit ist $\mathbf{P}(A) = \frac{1}{X + 1}$. \\

(b) Sei $H_i := [X = i]$, für $i = 1, \ldots, 6$, so ist $(H_i)_{i=1}^6$ ein vollständiges Ereignissystem. Offensichtlich, gilt $\Forall i = 1, \ldots, 6: \mathbf{P}(H_i) = \frac{1}{6}$. Nachdem $\mathbf{P}(A) > 0$, können wir das \Quote{Bayes'sche Theorem} anwenden und erhalten

\begin{align*}
  \mathbf{P}(H_3 | A)
  =
  \frac
  {\mathbf{P}(H_3) \mathbf{P}(A | H_3)}
  {\sum_{i=1}^6 \mathbf{P}(H_i) \mathbf{P}(A | H_i)}
  =
  \frac
  {\frac{1}{3+1}}
  {\sum_{i=1}^6 \frac{1}{i+1}}
  = \frac{35}{223}.
\end{align*}

\end{solution}

--------------------------------------------------------------------------------

\begin{exercise}

Stellen Sie fest, welche der folgenden Mengenfunktionen auf $2^\R$ äußere Maße sind. Bestimmen Sie für die äußeren Maßfunktionen jeweils das System der $\mu^\ast$-messbaren Mengen.

\begin{itemize}
  \item[(a)] $\mu^\ast(A) = |A|$
  \item[(b)]
  \begin{align*}
    \mu^\ast(A) =
    \begin{cases}
      0 & \text{wenn} \enspace |A| < \infty, \\
      1 & \text{sonst},
    \end{cases}
  \end{align*}
  \item[(c)]
  \begin{align*}
    \mu^\ast(A) =
    \begin{cases}
      0 & \text{wenn} \enspace \card(A) \leq \aleph_0, \\
      1 & \text{sonst},
    \end{cases}
  \end{align*}
  \item[(d)]
  \begin{align*}
    \mu^\ast(A) =
    \begin{cases}
      0 & \text{wenn} \enspace A = \emptyset, \\
      2 & \text{wenn} \enspace A = \R, \\
      1 & \text{sonst},
    \end{cases}
  \end{align*}
\end{itemize}

\end{exercise}

\begin{solution}

(a) Offensichtlich, ist $\mu^\ast$ ein äußeres Maß und $\mathfrak{M}_{\mu^\ast} = 2^\R$. \\

(b) Seien $A := \N$ und $B_n := \Bbraces{n}$, $n \in \N$. Dann gilt aber $\mu^\ast(A) = 1$ und $\Forall n \in \N: \mu^\ast(B_n) = 0$ und somit $\mu^\ast(A) \nleq \sum_{n \in \N} \mu^\ast(B_n)$. \\

(c) Die ersten drei Eigenschaften sind offensichtlich. Seien also $A, (B_n) \in 2^\R$, mit $A \subseteq \bigcup_{n \in \N} B_n$. Wenn $\card(A) \leq \aleph_0$, sind wir fertig. Ansonsten, $\Exists n \in \N: \card(B_n) > \aleph_0$ weil die obere Vereinigung (nur) abzählbar ist. \\

(d)

\end{solution}

--------------------------------------------------------------------------------

\begin{exercise}

Hier könnte Ihre Werbung stehen!

\begin{itemize}
  \item[(a)] Definieren Sie: Ring, Semiring, Sigmaring, Algebra, Sigmaalgebra, Dynkin-System, monotones System.
  \item[(b)] Von welchem Typ sind folgende Mengensysteme über $\Omega = \R$?
  \begin{align*}
    \mathfrak{C}_1 & = \Bbraces{A \subseteq \R: |A| < \infty}, \\
    \mathfrak{C}_2 & = \Bbraces{A \subseteq \R: |A| < 5}, \\
    \mathfrak{C}_3 & = \Bbraces{A \subseteq \R: |A| < \infty, \enspace \text{gerade}}, \\
    \mathfrak{C}_4 & = \Bbraces{A \subseteq \R: \card(A) \leq \aleph_0}, \\
  \end{align*}
\end{itemize}

\end{exercise}

\begin{solution}

(a) $\emptyset \neq \mathfrak{R}_\sigma \subseteq 2^\Omega \Text{Sigmaring} : \Leftrightarrow$
\begin{itemize}
  \item $\Forall (A_n) \in \mathfrak{R}_\sigma, \text{disj.}: \sum_{n \in \N} A_n \in \mathfrak{R}_\sigma$
  \item $\Forall A, B \in \mathfrak{R}_\sigma: A \setminus B \in \mathfrak{R}_\sigma,$
\end{itemize}

Für den Rest: Siehe Aufgabe 3 (a). \\

(b)

\begin{itemize}

  \item $\mathfrak{C}_1$ ist
  \begin{itemize}
    \item ein Ring, weil mit $|A|, |B| < \infty$, auch $|A \cup B|, |A \setminus B| < \infty$,
    \item ein Semiring, weil Ring,
    \item kein Sigmaring, weil $\Forall n \in \N: \Bbraces{n} \in \mathfrak{C}_1$, aber $\N \notin \mathfrak{C}_1$,
    \item keine Algebra, weil $\R \notin \mathfrak{C}_1$,
    \item keine Sigmaalgebra, weil kein Algebra,
    \item kein Dynkin-System, weil kein monotones System,
    \item und kein monotones System, weil $\Forall n \in \N: \Bbraces{1, \leq, n} \in \mathfrak{C}_1$, aber $\N \notin \mathfrak{C}_1$.
  \end{itemize}

  \item$\mathfrak{C}_2$ ist
  \begin{itemize}
    \item kein Ring, weil $|A|, |B| < 5$, im Allgemeinen nicht $|A \cup B| < 5$ folgt,
    \item ein Semiring, weil $\mathfrak{C}_2$ stabil bzgl. (endlicher) Durchschnitte ist, und $\Forall x \in \R: \Bbraces{x} \in \mathfrak{C}_2$, also \Quote{Leitern} gebaut werden können, bzgl. denen $\mathfrak{C}_2$ stabil ist,
    \item kein Sigmaring, weil kein Ring,
    \item keine Algebra, weil kein Ring,
    \item keine Sigmaalgebra, weil kein Algebra,
    \item kein Dynkin-System, weil $\Forall n \in \N: \Bbraces{n} \in \mathfrak{C}_1$, aber $\N \notin \mathfrak{C}_1$,
    \item ein monotones System, weil jede monotone Mengenfolge, nur höchstens $5$ verschiedene Folgenglieder haben kann, also fast überall konstant ist.
  \end{itemize}

  \item $\mathfrak{C}_3$ ist
  \begin{itemize}
    \item kein Ring, weil kein Semiring,
    \item kein Semiring, weil $\Bbraces{-1, 0}, \Bbraces{0, 1} \in \mathfrak{C}_3$, aber $\Bbraces{-1, 0} \cap \Bbraces{0, 1} = \Bbraces{0} \notin \mathfrak{C}_3$,
    \item kein Sigmaring, weil kein Semiring,
    \item keine Algebra, weil kein Semiring,
    \item keine Sigmaalgebra, weil kein Semiring,
    \item kein Dynkin-System, weil kein monotones System,
    \item kein monotones System, weil $\Forall n \in \N: \Bbraces{1, \ldots, 2n} \in \mathfrak{C}_3$, aber $\N \notin \mathfrak{C}_3$.
  \end{itemize}

  \item $\mathfrak{C}_4$ ist
  \begin{itemize}
    \item ein Ring, weil Sigmaring,
    \item ein Semiring, weil Sigmaring,
    \item ein Sigmaring, weil mit $\card(A_n), \card(A), \card(B) \leq \aleph_0$, $n \in \N$, auch $\card(\bigcup_{n \in \N} A_n), \card(A \setminus B) \leq \aleph_0$,
    \item keine Algebra, weil $\card(\R) > \aleph_0$,
    \item keine Sigmaalgebra, weil keine Algebra,
    \item kein Dynkin-System, weil $\card(\R) > \aleph_0$,
    \item ein monotones System, weil Sigmaring.
  \end{itemize}

\end{itemize}

\end{solution}

--------------------------------------------------------------------------------

\begin{exercise}

Hier könnte Ihre Werbung stehen!

\begin{itemize}
  \item[(a)] Definieren Sie: Maßfunktion, endliche Maßfunktion, sigmaendliche Maßfunktion, äußere Maßfunktion, von einem Maß auf einem Ring erzeugte äußere Maßfunktion, messbare Menge.
  \item[(b)] Zeigen Sie, dass das System $\mathfrak{M}$ der $\mu^\ast$-messbaren Mengen eine Sigmaalgebra ist, und dass die Einschränkung von $\mu^\ast$ auf $\mathfrak{M}$ ein Maß ist.
\end{itemize}

\end{exercise}

\begin{solution}

(a) Siehe Aufgabe 4. \\

(b) Siehe Skript.

\end{solution}

--------------------------------------------------------------------------------

\begin{exercise}

Hier könnte Ihre Werbung stehen!

\begin{itemize}
  \item[(a)] Definieren Sie: Maßfunktion, endliche Maßfunktion, sigmaendliche Maßfunktion, äußere Maßfunktion, von einem Maß auf einem Ring erzeugte äußere Maßfunktion.
  \item[(b)] $\mu$ und $\nu$ seien zwei Maßfunktionen auf dem Ring $\mathfrak{R}$. Zeigen Sie:
  \begin{align*}
    (\mu + \nu)^\ast = \mu^\ast + \nu^\ast
  \end{align*}
  und
  \begin{align*}
    \mathfrak{M}_{\mu^\ast} \cap \mathfrak{M}_{\nu^\ast}
    \subseteq
    \mathfrak{M}_{\mu^\ast + \nu^\ast}.
  \end{align*}
\end{itemize}

\end{exercise}

\begin{solution}

(a) Siehe Aufgabe 4. \\

(b) Siehe Aufgabe 9.

\end{solution}

--------------------------------------------------------------------------------

\begin{exercise}

\begin{itemize}
  \item[(a)] Definieren Sie: Ring, Semiring, Sigmaring, Algebra, Sigmaalgebra, Dynkin-System.
  \item[(b)] $\mu$ und $\nu$ seien zwei Wahrscheinlichkeitsmaße auf dem Messraum $(\Omega, \mathfrak{S})$. Zeigen Sie, dass
  \begin{align*}
    \mathfrak{D} = \Bbraces{A \in \mathfrak{S} : \mu(A) = \nu(A)}
  \end{align*}
  ein Dynkin-System ist.
\end{itemize}

\end{exercise}

\begin{solution}

(a) Siehe Aufgabe (a).

(b)

\begin{itemize}

  \item \Quote{Stabil bzgl. Differenzen von Teilmengen}: Seien $A, B \in \mathfrak{D}$ mit $A \subseteq B$, dann gilt Folgendes.
  \begin{align*}
    \mu(B \setminus A)
    & =
    \nu(B \setminus A)
    \Leftrightarrow \\
    \nu(A) + \mu(B \setminus A)
    & =
    \nu(B \setminus A) + \nu(A)
    =
    \nu(B)
    \Leftrightarrow \\
    \nu(A) + \mu(B)
    =
    \mu(A) + \nu(A) + \mu(B \setminus A)
    & =
    \nu(B) + \mu(A)
  \end{align*}

  \item \Quote{Stabil bzgl. abzählbaren, disjunkten Vereinigungen}: Sei $(A_n) \in \mathfrak{D}$ disjunkt, dann
  \begin{align*}
    \mu(\sum_{n \in \N} A_n)
    =
    \sum_{n \in \N} \mu(A_n)
    =
    \sum_{n \in \N} \nu(A_n)
    =
    \nu(\sum_{n \in \N} A_n).
  \end{align*}

  \item \Quote{Enthält Grundmenge}: $\mu(\Omega) = 1 = \nu(\Omega)$

\end{itemize}

\end{solution}
