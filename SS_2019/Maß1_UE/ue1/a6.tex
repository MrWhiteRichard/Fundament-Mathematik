\begin{lemma}
    Es gelten folgende Aussagen:
    \begin{enumerate}[(a)]
        \item Wenn $\mathfrak{R}$ ein Ring ist, dann ist $\mathfrak{A}(\mathfrak{R})=\{A\subset\Omega\mid A\in\mathfrak{R}\lor A^C\in\mathfrak{R}\}$.
        \item Wenn $\mathfrak{R}$ ein $\sigma$-Ring ist, dann ist $\mathfrak{A}(\mathfrak{R})$ eine $\sigma$-Algerba.
    \end{enumerate}  
\end{lemma}
 \begin{proof}[Beweis.]
     Um (a) zu beweisen weist man nach, dass $M:=\{A\subset\Omega\mid A\in\mathfrak{R}\lor A^C\in\mathfrak{R}\}$ eine Algebra ist.\newline
     Für den Beweis von (b) weist man dann nach, dass $M$ eine $\sigma$-Algebra ist.
 \end{proof}