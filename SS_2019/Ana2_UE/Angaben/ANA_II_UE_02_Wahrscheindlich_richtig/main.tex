\documentclass{article}
\usepackage[utf8]{inputenc}
\usepackage{fullpage}
\usepackage{relsize}
\usepackage{amsmath, amssymb}

\title{
    \textbf{Übungen zu Analysis 2, 2. Übung 19. 3. 2019} \\
    \smaller ENTWÜRZT
}
\date{}

\begin{document}

\maketitle

\begin{enumerate}
\setcounter{enumi}{10}

\item Beweisen Sie die Ungleichung von Young: Für $A, B > 0$, $p, q > 1$, $\frac{1}{p} + \frac{1}{q} = 1$ gilt

\[
    A^{\frac{1}{p}} B^{\frac{1}{q}} \leq
    \frac{A}{p} + \frac{B}{q}
\]

Hinw.: Verwenden Sie die Konvexität der Exponentialfunktion.

\item Beweisen Sie die Ungleichung von Hölder: Für $p, q > 1$, $\frac{1}{p} + \frac{1}{q} = 1$, gilt für $(x_1, ..., x_n), (y_1, ..., y_n) \in \mathbb{R}^n
$

\[
    \sum_{k=1}^n |x_k y_k| \leq
    \Bigg( \sum_{k=1}^n |x_k|^p \Bigg)^{1/p}
    \Bigg( \sum_{k=1}^n |y_k|^p \Bigg)^{1/q}
    \text{.}
\]

\item Zeigen Sie , dass für $1 \leq p < \infty$, $( \sum\nolimits_{k=1}^n |x_k|^p )^{1/p}$ eine Norm auf $\mathbb{R}^n$ ist. \\
Hinw.: Für die Dreiecksungleichung zeige man

\[
    \sum_{k=1}^n |x_k + y_k|^p \leq
    \sum_{k=1}^n |x_k| |x_k + y_k|^{p-1} +
    \sum_{k=1}^n |y_k| |x_k + y_k|^{p-1}
\]

und wende die Hölder-Ungleichung an.

\item Man zeige, dass alle Normen $|| \cdot ||_p, || \cdot ||_q$, für $1 \leq p, q \leq \infty$ auf $\mathbb{R}^n$ äquivalent sind. Man zeige insbesondere, dass für $1 \leq p < q < \infty$, $|| \cdot ||_p \geq || \cdot ||_q \geq || \cdot ||_{\infty}$. \\
Hinw.: Man verwende, dass aus $0 < p < q$ und $\lambda \in [0, 1]$, $\lambda^q \leq \lambda^p$ folgt.

\item Bestimmen Sie die Operatornormen der Identität als Abbildung

\[
    (\mathbb{R}^n, || \cdot ||_p) \rightarrow (\mathbb{R}^n, || \cdot ||_q)
    \text{ für }
    1 \leq p < q \leq \infty
\]

und von

\[
    (\mathbb{R}^n, || \cdot ||_2) \rightarrow (\mathbb{R}^n, || \cdot ||_1)
    \text{,}
\]

sowie von $A_x: (\mathbb{R}^n, || \cdot ||_2) \rightarrow \mathbb{R}$, $A_x y = (x, y)$ ($( \cdot, \cdot )$ Skalarprodukt). \\
(Scheinbar nicht obligatorischer) Hinw.: Mit dem Skalarprodukt, ist tatsächlich jenes, aus der Schulmathematik gemeint. d.h. also, wir identifizieren $A_x$ mit $x^T$ und behandeln $A_x y$ als "Matrix-Matrix" Produkt.

\item Zeigen Sie, dass die Gruppe $SO_n(\mathbb{R})$, der orthogonalen $n \times n$-Matrizen mit Determinante $1$, abgeschlossen (wahrscheindlich im topologischen Sinne) in $GL_n(\mathbb{R})$ ist, wenn man $GL_n(\mathbb{R})$ als Teilmenge des $\mathbb{R}^{n \times n}$ auffasst. \\
Hinw.: Verwenden Sie, dass eine $n \times n$-Matrix genau dann orthogonal ist, wenn alle ihre Spaltenvektoren Euklidische Norm $1$ haben und ihre Determinante $\pm 1$ ist. Verwenden Sie dann Prop. 6.1.12.

\item Zeigen Sie, dass für $A: (\mathbb{R}^n, || \cdot ||_p) \rightarrow (\mathbb{R}^n, || \cdot ||_p)$ und $||A|| < \ln{2}$ (!! siehe Beispiel 9.2.10 (iii) !!), die Abbildung $x \mapsto e^A x$ bijektiv ist.

\item Zeigen Sie, dass für $r, s \in \mathbb{R}$ und $A$ eine $n \times n$-Matrix, $e^{rA} e^{sA} = e^{(r+s)A}$ gilt. \\
Hinw.: Berechnen Sie $\frac{d}{dr} (e^{rA} e^{(u-r)A})$.

\item Für

\[
    F: \mathbb{R} \rightarrow \mathbb{R}^3: t \mapsto
    \begin{pmatrix}
        t \cosh{t} \\
        t e^{t^2} \\
        1
    \end{pmatrix}
    \text{,}
\]

berechne man $\int_{0}^{1} F(t) dt$ und $F'$.

\item Berechnen Sie die Abbildungsnorm der lin. Abbildung $T$:

\[
    Tf(x) =
    \begin{pmatrix}
        \int_{0}^{x} F(t) dt \\
        \int_{x}^{1} F(s) ds
    \end{pmatrix}
\]

als Abbildung

\[
    ( C[0, 1], || \cdot ||_{\infty} ) \rightarrow (
    ( C[0, 1], || \cdot ||_{\infty} ) \times
    ( C[0, 1], || \cdot ||_{\infty} ), || \cdot ||_1 )
    \text{,}
\]

bzw. nach

\[
    ( C[0, 1], || \cdot ||_{\infty} ) \rightarrow (
    ( C[0, 1], || \cdot ||_{\infty} ) \times
    ( C[0, 1], || \cdot ||_{\infty} ), || \cdot ||_{\infty} )
    \text{.}
\]

\end{enumerate}

\end{document}
