\begin{lemma}
    Es gelten folgende Aussagen:
    \begin{itemize}
        \item[(a)] Wenn $\mu$ und $\nu$ zwei endliche Maße auf dem Messraum $(\Omega,\mathfrak{S})$ sind, dann ist $\mathfrak{D}:=\{A\in\mathfrak{S}\mid\mu(A)=\nu(A)\}$ ein Dynkin-System im weiteren Sinn.
        \item[(b)] Sind $\mu$ und $\nu$ sogar Wahrscheinlichkeitsmaße, dann ist $\mathfrak{D}$ ein Dynkin-System im engeren Sinn. 
    \end{itemize}
\end{lemma} 
\begin{proof}[Beweis.]
    Ausständig.
\end{proof}