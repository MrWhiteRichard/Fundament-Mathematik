\begin{lemma}
    Wenn $(\mathfrak{R}_n)_{n\in\mathbb{N}}$ eine nichtfallende Folge von Ringen über derselben Grundmenge $\Omega$ ist, dann ist auch $\mathfrak{R}:=\bigcup_{n\in\mathbb{N}}\mathfrak{R}_n$ ein Ring. Die analoge Aussage für $\sigma$-Ringe gilt im Allgemeinen nicht.
\end{lemma}
\begin{proof}[Beweis.]
    Um den ersten Teil zu beweisen muss man einfach nachrechnen, dass $\mathfrak{R}$ ein Ring ist.\newline
    Für den zweiten Teil definieren wir eine Folge von $\sigma$-Ringen $\mathfrak{R}_n:=2^{\{1,\dots,n\}}$, wobei $\mathfrak{R}:=\bigcup_{n\in\mathbb{N}}\mathfrak{R}_n$ sein soll. Nun ist $\forall n\in\mathbb{N}:\{n\}\in\mathfrak{R}$, aber $\bigcup_{n\in\mathbb{N}}\{n\}\notin\mathfrak{R}$, womit $\mathfrak{R}$ kein $\sigma$-Ring sein kann.
\end{proof}