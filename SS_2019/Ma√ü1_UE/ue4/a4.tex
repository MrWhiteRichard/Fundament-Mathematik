\begin{lemma}
    Wenn $\mathfrak{C}$ ein Mengensystem über $\Omega$ mit $\emptyset\in\mathfrak{C}$ ist und $f:\mathfrak{C}\rightarrow[0,\infty]$ mit $f(\emptyset)=0$, dann ist 
    \begin{align*}
        \mu^*:2^\Omega\rightarrow\overline{\mathbb{R}}:A\mapsto\inf\left\{\sum_{n\in\mathbb{N}}f(B_n)\mid\forall n\in\mathbb{N}:B_n\in\mathfrak{C}\land A\subset\bigcup_{n\in\mathbb{N}}B_n\right\}
    \end{align*}
    eine äußere Maßfunktion.
\end{lemma}
\begin{proof}[Beweis.]
    Die erste Eigenschaft einer äußeren Maßfunktion $\mu^*(\emptyset)=0$ ist wegen $f(\emptyset)=0$ trivialerweise erfüllt.

    Aufgrund des Bildbereichs von $f$ ist auch die zweite Eigenschaft $\forall A\subset\Omega:\mu^*(A)\geq0$ leicht einzusehen.

    Um die Monotonie nachzuweisen wählen wir beliebige $A\subset B\subset\Omega$ und ein beliebiges $\epsilon>0$. Nun gibt es eine Folge $(C_n)_{n\in\mathbb{N}}$ aus $\mathfrak{C}$ mit $B\subset\bigcup_{n\in\mathbb{N}}C_n$ für die
    \begin{align*}
        \mu^*(B)+\epsilon\geq\sum_{n\in\mathbb{N}}f(C_n)\geq\mu^*(A)
    \end{align*}
    gilt. Die Monotonie ist gezeigt, weil $\epsilon$ beliebig war.

    Zuletzt ist noch die Sigmasubadditivität nachzuweisen. Dafür wählen wir $A\subset\Omega$ beliebig sowie eine beliebige Folge $(B_n)_{n\in\mathbb{N}}$ aus $2^\Omega$ mit $A\subset\bigcup_{n\in\mathbb{N}}B_n$. Für ein beliebiges $\epsilon>0$ und $n\in\mathbb{N}$ gibt es eine Folge $(C_{nk})_{k\in\mathbb{N}}$ mit $B_n\subset\bigcup_{k\in\mathbb{N}}C_{nk}$ so, dass
    \begin{align*}
        \sum_{n\in\mathbb{N}}\mu^*(B_n)+\epsilon\geq\sum_{(n,k)\in\mathbb{N}\times\mathbb{N}}f(C_{nk})\geq\mu^*(A)
    \end{align*}
    gilt. Da $\epsilon$ beliebig war haben wir die Sigmasubadditivität gezeigt.
\end{proof}