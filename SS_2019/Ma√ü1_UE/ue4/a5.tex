\begin{lemma}
    Wenn $(\mu_i^*)_{i\in I}$ eine Familie von äußeren Maßen auf $\Omega$ ist, so ist auch $\mu^*:2^\Omega\rightarrow\overline{\mathbb{R}}:A\mapsto\sup\{\mu_i^*(A)\mid i\in I\}$ ein äußeres Maß.
\end{lemma}
\begin{proof}[Beweis.]
    Die erste Eigenschaft eines äußeren Maßes $\mu^*(\emptyset)=0$ ist offensichtlich erfüllt.

    Auch die zweite Eigenschaft $\forall A\subset\Omega:\mu^*(A)\geq 0$ überträgt sich direkt auf das Supremum.

    Die Monotonie ist ebenfalls leicht nachzuweisen.

    Um die Sigmasubadditivität nachzuweisen wählen wir $A\subset\Omega$ und eine beliebige Folge $(B_n)_{n\in\mathbb{N}}$ aus $2^\Omega$, welche $A\subset\bigcup_{n\in\mathbb{N}}B_n$ erfüllt. Für ein beliebiges $\epsilon>0$ gibt es ein $j\in I$ das
    \begin{align*}
        \mu^*(A)-\epsilon\leq\mu_j^*(A)\leq\sum_{n\in\mathbb{N}}\mu_j^*(B_n)\leq\sum_{n\in\mathbb{N}}\mu^*(B_n)
    \end{align*}
    erfüllt.
\end{proof}