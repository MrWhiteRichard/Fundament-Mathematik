\begin{exercise}

Es sei $\Omega = \N$, $\mathfrak{S} = 2^\N$, $\mu(A) = |A|$. Wann konvergiert die Funktionenfolge $f_n$ im Maßraum $(\Omega, \mathfrak{S}, \mu)$

\begin{itemize}
  \item[(a)] fast überall
  \item[(b)] fast gleichmäßig
  \item[(c)] im Maß?
\end{itemize}

\end{exercise}

% --------------------------------------------------------------------------------

\begin{solution}

Man beachte, dass $\Forall A \in 2^\N: |A| = 0 \Leftarrow A = \emptyset$, d.h. $\emptyset$ ist die einzige Nullmenge. Somit gilt eine Aussage genau dann fast überall, wenn sie auf $\emptyset^\complement = \N$ gilt, also überall.

(a)

\begin{align*}
  f_n \xrightarrow[n \to \infty]{\text{f.ü.}} f
  \Leftarrow
  f_n \xrightarrow[n \to \infty]{\text{punktw.}} f
\end{align*}

(b)

\begin{align*}
  f_n \xrightarrow[n \to \infty]{\text{fast glm.}} f
  \Leftarrow
  f_n \xrightarrow[n \to \infty]{\text{glm.}} f
\end{align*}

(c)

\end{solution}
