% --------------------------------------------------------------------------------

\begin{exercise}

\phantom{}

\begin{enumerate}[label = (\alph*)]

  \item
  Definieren Sie: messbare Funktion, Treppenfunktion, Konvergenz im Maß, Konvergenz fast überall, Konvergenz fast gleichmäßig.
  
  \item
  In welchem Sinn (fast überall gleichmäßig/fast gleichmäßig/fast überall/im Maß) konvergieren die folgenden Folgen in $(\R, \mathfrak{B}, \lambda)$?
  
  \begin{enumerate}[label = \roman*.]
  
    \item
    $f_n(x) = \sin(x)/n$

    \item
    $f_n(x) = e^{-n |x|}$
    
    \item
    $f_n(x) = x/n$
    
    \item
    $f_n(x) = f_n(x) =
    \begin{cases}
      1 & \text{wenn} \enspace \sqrt{n} - \floor{\sqrt{n}} \leq x \leq \sqrt{n+1} - \floor{\sqrt{n}} \\
      0 & \text{sonst}.
    \end{cases}$

  \end{enumerate}

\end{enumerate}

\end{exercise}

% --------------------------------------------------------------------------------

\begin{solution}

(a) Siehe Aufgabe 7 (a) \\

(b) \phantom{}

\begin{itemize}

  \item[i.]
  \begin{align*}
     \norm[\infty]{f_n}
     =
     \sup_{x \in \R} \vbraces{\sin(x)/n}
     =
     1/n
     \xrightarrow[n \to \infty]{} 0 \\
     \Rightarrow
     f_n \xrightarrow[n \to \infty]{\text{glm.}} 0
     \Rightarrow
     f_n \xrightarrow[n \to \infty]{\text{fast glm.}} 0
     \Rightarrow
     f_n \xrightarrow[n \to \infty]{\text{f.ü.}} 0,
     \enspace
     f_n \xrightarrow[n \to \infty]{\text{im Maß}} 0
   \end{align*}

  \item[ii.]
  \begin{align*}
    \Forall A \subseteq \R \setminus \Bbraces{0}:
    \norm[\infty]{f_n|_A}
    =
    \sup_{x \in A} \vbraces{e^{-n |x|}}
    =
    e^{-n \inf_{x \in A} |x|}
    \xrightarrow[n \to \infty]{} 0 \\
    \Rightarrow
    f_n \xrightarrow[n \to \infty]{\text{fast glm.}} 0
    \Rightarrow
    f_n \xrightarrow[n \to \infty]{\text{f.ü.}} 0,
    \enspace
    f_n \xrightarrow[n \to \infty]{\text{im Maß}} 0
  \end{align*}
  Sei $N \in \mathfrak{B}$, mit $\lambda(N) = 0$, dann gilt trotzdem noch $\Forall \epsilon > 0: \vbraces{B(0, \epsilon) \cap N^\complement} = \infty$, also konvergiert $(f_n)$ nicht fast überall gleichmäßig.

  \item[iii.]
  \begin{align*}
    f_n \xrightarrow[n \to \infty]{\text{punktw.}} 0
    \Rightarrow
    f_n \xrightarrow[n \to \infty]{\text{f.ü.}} 0
  \end{align*}
  $\Forall \epsilon > 0, \Forall n \in \N:$
  \begin{align*}
    \lambda(x/n > \epsilon)
    =
    \lambda(x > \epsilon n)
    =
    \lambda(]\epsilon n, \infty])
    =
    \infty
  \end{align*}
  Also konvergiert $(f_n)$ nicht im Maß. \\
  Wenn $(f_n)$ fast (überall) gleichmäßig konvergiert, dann offensichtlich gegen $0$. Aber $\Forall \epsilon > 0, \Forall N \in \mathfrak{B}:$
  \begin{align*}
    \lambda(N) < \epsilon
    \Rightarrow
    \Forall n \in \N:
    \norm[\infty]{f_n|_{N^\complement}}
    =
    \sup_{x \in N^\complement} |x|/n
    =
    \infty
  \end{align*}
  Also konvergiert $(f_n)$ weder fast gleichmäßig, noch fast überall gleichmäßig. \\

  \item[iv.] $\Forall \epsilon > 0:$
  \begin{align*}
    \lambda(|f_n| > \epsilon)
    =
    \lambda(f_n = 1)
    =
    \lambda
    ([\sqrt{n} - \floor{\sqrt{n}}, \sqrt{n+1} - \floor{\sqrt{n}}])
    =
    \sqrt{n+1} - \sqrt{n}
    \xrightarrow[n \to \infty]{} 0 \\
    \Rightarrow
    f_n \xrightarrow[n \to \infty]{\text{im Maß}} 0
  \end{align*}
  $\Forall q \in \N^2:$
  \begin{align*}
    ]0, 1]
    =
    \sum_{i = q}^{(\sqrt{q}+1)^2-1}
    \left ]
    \sqrt{i} - \floor{\sqrt{i}}, \sqrt{i+1} - \floor{\sqrt{i}}
    \right ]
  \end{align*}
  Also, muss $\Forall x \in \: ]0, 1], \Forall N \in \N: \Exists n, m \geq N:$
  \begin{align*}
    |f_n(x) - f_m(x)| = 1,
  \end{align*}
  und es konvergiert $(f_n)$ nicht fast überall und somit auch weder fast gleichmäßig, noch fast überall gleichmäßig.

\end{itemize}

\end{solution}

% --------------------------------------------------------------------------------
