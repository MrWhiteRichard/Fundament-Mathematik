% --------------------------------------------------------------------------------

\begin{exercise}

\phantom{}

\begin{enumerate}[label = (\alph*)]

  \item
  Definieren Sie das Integral einer nichtnegativen Treppenfunktion, einer nichtnegativen messbaren Funktion, einer messbaren und einer fast uberall messbaren Funktion.
  
  \item
  Es sei
  
  \begin{align*}
    F(x) =
    \begin{cases}
      0     & \text{wenn} \enspace  x < 0, \\
      x + 1 & \text{wenn} \enspace 1 \leq x < 2, \\
      2x^2  & \text{wenn} \enspace 2 \leq x < 3, \\
      20    & \text{wenn} \enspace 3 \leq x.
    \end{cases}
  \end{align*}

  Überzeugen Sie sich, dass $F$ eine Verteilungsfunktion ist und bestimmen Sie $\Int{f}{\mu_F}$ für $f(x) = e^x$.

  \item
  Formulieren und beweisen Sie den Satz von der Konvergenz durch Majorisierung.

\end{enumerate}

\end{exercise}

% --------------------------------------------------------------------------------

\begin{solution}

(a) Sei $t = \sum_{i=1}^n a_i A_i$ eine nichtnegative, messbare Treppenfunktion auf $(\Omega, \mathfrak{S}, \mu)$, dann

\begin{align*}
  \Int{t}{\mu} := \sum_{i=1}^n a_i \mu(A_i).
\end{align*}

Sei $f$ eine nichtnegative, messbare Funktion, dann

\begin{align*}
  \Int{f}{\mu} := \sup \Bbraces{\Int{t}{\mu}: f \geq t \geq 0, t \text{Treppenf.}}.
\end{align*}

Sei $f$ eine messbare Funktion, dann

\begin{align*}
  \Int{f}{\mu} := \Int{f^+}{\mu} - \Int{f^-}{\mu}.
\end{align*}

Sei $f$ eine fast überall messbare Funktion, d.h. $\Exists N \in \mathfrak{S}: \mu(N) = 0, fN^\complement \enspace \text{messb.}$, dann

\begin{align*}
  \Int{f}{\mu} := \Int[N^\complement]{f}{\mu}.
\end{align*}

\end{solution}

% --------------------------------------------------------------------------------
