% --------------------------------------------------------------------------------

\begin{exercise}

Für welche $n$ ist $(x_1 + \ldots + x_n)^2$ eine $n$-dimensionale Verteilungsfunktion?

\end{exercise}

% --------------------------------------------------------------------------------

\begin{solution}

Behauptung: $\Forall n \in \N: \norm[1]{\cdot}^2: \R^n \to \R, \text{Verteilungsfunktion}$ \\

Beweis: \\

\blockquote{Rechtsstetigkeit}: Normen sind, aufgrund der Dreiecksungleichung nach unten, stetig. \\

\blockquote{nichtnegativer Differenzenoperator}: zz: $\Forall a, b \in \R^n: a \leq b \Rightarrow$

\begin{align*}
  \Delta^{(a, b)} F
  =
  \mu(]a, b])
  =
  \sum_{e \in \Bbraces{0, 1}^n}
  (-1)^{\norm[1]{e}}
  F(a \cdot e + b \cdot (1 - e)) \geq 0
\end{align*}

Wir partitionieren $\Bbraces{0, 1}^n = A_n + B_n$, wobei

\begin{align*}
  A_n & := \Bbraces{e \in \Bbraces{0, 1}^n: \norm[1]{e} \in 2 \N_0}, \\
  B_n & := \Bbraces{e \in \Bbraces{0, 1}^n: \norm[1]{e} \in 2 \N_0 + 1}.
\end{align*}

Durch Induktion, erhält man $|A_n| = |B_n|$.

\begin{align*}
  S_{e, n}
  :=
  F(a \cdot e + b \cdot (1 - e))
  =
  \sum_{\substack{i = 1 \\ e_i = 1}}^n a_i +
  \sum_{\substack{i = 1 \\ e_i = 0}}^n b_i,
  \quad
  \text{lhs} := \sum_{e \in A_n} S_{e, n}^2,
  \quad
  \text{rhs} := \sum_{e \in B_n} S_{e, n}^2.
\end{align*}

Dann ist das obere \blockquote{zz} äquivalent zur IB: $\Forall a, b \in \R^n: a \leq b \Rightarrow \text{lhs} \geq \text{rhs}$. \\

IA ($n = 1$): $A_1 = \Bbraces{(0)}$, $B_1 = \Bbraces{(1)}$, also

\begin{align*}
  \text{lhs} = b_1^2 \geq \text{rhs} = a_1^2.
\end{align*}

IS ($n \mapsto n+1$): Wir partitionieren weiters $A_{n+1} = A_n^0 + B_n^1$, $B_{n+1} = A_n^1 + B_n^0$, wobei

\begin{align*}
  A_n^0 & := \Bbraces{e \in A_{n+1}: e_{n+1} = 0}, &
  A_n^1 & := \Bbraces{e \in A_{n+1}: e_{n+1} = 1}, \\
  B_n^0 & := \Bbraces{e \in B_{n+1}: e_{n+1} = 0}, &
  B_n^1 & := \Bbraces{e \in B_{n+1}: e_{n+1} = 1}.
\end{align*}

Wir formen die linke Seite um.

\begin{align*}
  \text{lhs}
  =
  \sum_{e \in A_{n+1}} S_{e, n+1}^2
  =
  \sum_{e \in A_n^0} S_{e, n+1}^2 +
  \sum_{e \in B_n^1} S_{e, n+1}^2
  =
  \sum_{e \in A_n} (S_{e, n} + b_{n+1})^2 +
  \sum_{e \in B_n} (S_{e, n} + a_{n+1})^2
\end{align*}

Wir formen die rechte Seite um.

\begin{align*}
  \text{rhs}
  =
  \sum_{e \in B_{n+1}} S_{e, n+1}^2
  =
  \sum_{e \in A_n^1} S_{e, n+1}^2 +
  \sum_{e \in B_n^0} S_{e, n+1}^2
  =
  \sum_{e \in A_n} (S_{e, n} + b_{n+1})^2 +
  \sum_{e \in B_n} (S_{e, n} + a_{n+1})^2
\end{align*}

Man beachte, dass folgendes gilt ...

\begin{align*}
  (S_{e, n} + b_{n+1})^2 - (S_{e, n} + a_{n+1})^2
  & =
  S_{e, n}^2 + 2 S_{e, n} b_{n+1} + b_{n+1}^2 -
  S_{e, n}^2 - 2 S_{e, n} a_{n+1} - a_{n+1}^2 \\
  & =
  2 S_{e, n} (b_{n+1} - a_{n+1}) + (b_{n+1}^2 - a_{n+1}^2)
\end{align*}

... und erhält damit ...

\begin{align*}
  \text{lhs}
  & \geq
  \text{rhs}
  \Leftrightarrow \\
  \sum_{e \in A_n} (S_{e, n} + b_{n+1})^2 - (S_{e, n} + a_{n+1})^2
  & \geq
  \sum_{e \in B_n} (S_{e, n} + b_{n+1})^2 - (S_{e, n} + a_{n+1})^2
  \Leftrightarrow \\
  2 \pbraces{\sum_{e \in A_n} S_{e, n}}(b_{n+1} - a_{n+1}) + |A_n| (b_{n+1}^2 - a_{n+1}^2)
  & \geq
  2 \pbraces{\sum_{e \in B_n} S_{e, n}}(b_{n+1} - a_{n+1}) + |B_n| (b_{n+1}^2 - a_{n+1}^2)
  \Leftarrow \\
  \sum_{e \in A_n} S_{e, n}^2
  & \geq
  \sum_{e \in B_n} S_{e, n}^2
\end{align*}

... was laut IV gilt.

\end{solution}

% --------------------------------------------------------------------------------
