% --------------------------------------------------------------------------------

\begin{exercise}

Zeigen Sie:

\begin{align*}
  F(x, y) =
  \begin{cases}
    x y^2 & \text{falls} \enspace y > 0. \\
    0     & \text{sonst}
  \end{cases}
\end{align*}

ist eine zweidimensionale Verteilungsfunktion. Bestimmen Sie das Maß des Einheitskreises.

\end{exercise}

% --------------------------------------------------------------------------------

\begin{solution}

\phantom{}

\begin{itemize}

  \item \enquote{Rechtsstetigkeit}: Tatsächlich gilt sogar $\Forall x \in \R: F(x, 0 + 0) = F(x, 0) = 0$.

  \item \enquote{nichtnegativer Differenzenoperator}: Seien $a, b \in \R^2$, mit $a \leq b$.
  \begin{align*}
    \Delta^{(a, b)} F(x)
    & =
    \Delta_1^{(a_1, b_1)}
    \Delta_2^{(a_2, b_2)}
    F(x) \\
    & =
    \Delta_1^{(a_1, b_1)}
    \pbraces{F(x_1, b_2) - F(x_1, a_2)} \\
    & =
    \pbraces{\Delta_1^{(a_1, b_1)} F(x_1, b_2)} -
    \pbraces{\Delta_1^{(a_1, b_1)} F(x_1, a_2)} \\
    & =
    \pbraces{F(b_1, b_2) - F(a_1, b_2)} -
    \pbraces{F(b_1, a_2) - F(a_1, a_2)} \\
    & =
    \pbraces{b_1 b_2^2 - a_1 b_2^2} -
    \pbraces{b_1 a_2^2 - a_1 a_2^2} \\
    & =
    b_2^2 (b_1 - a_1) - a_2^2 (b_1 - a_1) \\
    & =
    (b_1 - a_1) (b_2^2 - a_2^2)
    \geq
    0
  \end{align*}

\end{itemize}

Die untere Hälfte des Einheitskreises, hat offensichtlich Maß Null. Die Obere, werden wir mit Untersummen approximieren. Dazu betrachten wir vorerst die rechte Hälfte.

\begin{align*}
  \sum_{i=0}^{n-1}
  \mu_F
  \pbraces
  {
    \Bigg ]
      \pbraces{\frac{i}{n}, 0},
      \pbraces{\frac{i+1}{n}, \sqrt{1 - \pbraces{\frac{i+1}{n}}^2}}
    \Bigg ]
  }
  =
  \sum_{i=0}^{n-1}
  \pbraces{\frac{i+1}{n} - \frac{i}{n}}
  \pbraces{\sqrt{1 - \pbraces{\frac{i+1}{n}}^2}^2 - 0^2} \\
  \xrightarrow[n \to \infty]{}
  \Int[0][1]{1 - x^2}{x}
  =
  1 - \frac{x^3}{3} \Bigg |_0^1
  =
  \frac{2}{3}
\end{align*}

Nachdem dieser Ausdruck symmetrisch um die $y$-Achse ist (Quadrat), gilt dieser auch für die linke Hälfte und damit $\mu_F(B(0, 1)) = \frac{4}{3}$.

\end{solution}

% --------------------------------------------------------------------------------
