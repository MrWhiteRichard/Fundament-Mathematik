% --------------------------------------------------------------------------------

\begin{exercise}

Gegeben sei die folgende Funktion auf $2^\R$:

\begin{align*}
  \mu^\ast(A) =
  \begin{cases}
    0 & \text{für} \enspace A = \emptyset, \\
    1 & \text{wenn} \enspace 1 \leq |A| < \infty, \\
    2 & \text{sonst}.
  \end{cases}
\end{align*}

Zeigen Sie, dass $\mu^\ast$ eine äußere Maßfunktion ist und bestimmen Sie das System der $\mu^\ast$-messbaren Mengen.

\end{exercise}

% --------------------------------------------------------------------------------

\begin{solution}

Die ersten drei Eigenschaften sind offensichtlich. Seien zuletzt $A, (B_n) \in 2^\Omega$ mit $A \subseteq \bigcup_{n \in \N} B_n$. Falls $A = \emptyset$, sind wir (wegen der zweiten Eigenschaft) fertig. Für $1 \leq |A| < \infty$, muss $\Exists n \in \N: 1 \leq |B_n| < \infty$. Ansonsten, sind die (unendlich vielen) $\omega \in A$ über ganz $(B_n)$ verteilt. Nun gilt aber $\Exists n \in \N: |B_n| = \infty$ oder $\Exists n \neq m \in \N: 1 \leq |B_n|, |B_m| < \infty$. \\

Das System der Carathéodory-messbaren Mengen lautet wie folgt.

\begin{align*}
  \mathfrak{M}_{\mu^\ast} :=
  \Bbraces
  {
    A \subseteq \Omega:
    \Forall B \subseteq \Omega:
    \mu^\ast(B) = \mu^\ast(B \cap A) + \mu^\ast(B \cap A^\complement)
  }
\end{align*}

Weil dieses System eine $\sigma$-Algebra ist, sind $\emptyset, \Omega$ messbar. Sei also $A \subseteq \Omega$ mit $A \neq \emptyset, \Omega$.

\begin{itemize}

  \item[Fall $1$)] $A \enspace \text{endl.} \Rightarrow A^\complement \Text{unendl.}$
  \begin{align*}
    2 = \mu^\ast(\R) =
    \underbrace{\mu^\ast(\R \cap A)}_1 +
    \underbrace{\mu^\ast(\R \cap A^\complement)}_2 = 3
  \end{align*}

  \item[Fall $2$)] $A \enspace \text{unendl.}$
  \begin{itemize}
    \item[Fall a)] $A^\complement \enspace \text{endl.} \Rightarrow$ analog zu Fall $1$
    \item[Fall b)] $A^\complement \enspace \text{unendl.}$
    \begin{align*}
      2 = \mu^\ast(\R) =
      \underbrace{\mu^\ast(\R \cap A)}_2 +
      \underbrace{\mu^\ast(\R \cap A^\complement)}_2 = 4
    \end{align*}
  \end{itemize}

\end{itemize}

Damit, muss $\mathfrak{M}_{\mu^\ast} = \Bbraces{\emptyset, \Omega}$.

\end{solution}

% --------------------------------------------------------------------------------
