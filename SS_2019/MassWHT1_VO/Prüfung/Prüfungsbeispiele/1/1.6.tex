% --------------------------------------------------------------------------------

\begin{exercise}

\phantom{}

\begin{enumerate}[label = (\alph*)]

  \item
  Definieren Sie Ring, Semiring, monotones System, Dynkin-System.
  
  \item
  Gegeben sei der Wahrscheinlichkeitsraum $(\Omega, \mathfrak{S}, \P)$. Für $A \in \mathfrak{S}$ sei
  
  \begin{align*}
    \mathfrak{U}(A) = \Bbraces{B: \P(A \cap B) = \P(A) \P(B)}
  \end{align*}
  
  das System aller Mengen, die von $A$ unabhängig sind. Zeigen Sie, dass $\mathfrak{U}(A)$ ein Dynkin-System ist.

\end{enumerate}

\end{exercise}

% --------------------------------------------------------------------------------

\begin{solution}

(a) Siehe Aufgabe 3 (a) \\

(b)

\begin{itemize}

  \item \enquote{Stabil bzgl. Differenzen von Teilmengen}: Seien $B, C \in \mathfrak{U}(A)$ mit $B \subseteq C$, dann gilt Folgendes.
  \begin{align*}
    \P(A \cap C \setminus B)
    & = \P(A) \P(C \setminus B)
    \Leftrightarrow \\
    \P(A) \P(B) + \P(A \cap C \setminus B)
    & = \P(A) \P(C \setminus B) + \P(A) \P(B)
      = \P(A) \P(C)
    \Leftrightarrow \\
    \P(A \cap C) + \P(A) \P(B)
      =\P(A \cap B) + \P(A) \P(B) + \P(A \cap C \setminus B)
    & = \P(A) \P(C) + \P(A \cap B)
  \end{align*}

  \item \enquote{Stabil bzgl. abzählbaren, disjunkten Vereinigungen}: Sei $(B_n) \in \mathfrak{D}$ disjunkt, dann
  \begin{align*}
    \P \pbraces{A \cap \sum_{n \in \N} B_n}
    =
    \sum_{n \in \N} \P(A \cap B_n)
    =
    \sum_{n \in \N} \P(A) \P(B_n)
    =
    \P(A) \P \pbraces{\sum_{n \in \N} B_n}.
  \end{align*}

  \item \enquote{Enthält Grundmenge}: $\P(A \cap \Omega) = \P(A) = \P(A) \P(\Omega)$

\end{itemize}

\end{solution}

% --------------------------------------------------------------------------------
