% -------------------------------------------------------------------------------- %

\begin{exercise}

\phantom{}

\begin{enumerate}[label = (\alph*)]

  \item
  Definieren Sie Maßfunktion, endliche Maßfunktion, sigmaendliche Maßfunktion, äußere Maßfunktion.

  \item
  Formulieren und beweisen Sie den Fortsetzungssatz für Maßfunktionen.
  
  \item
  $\Omega$ sei eine beliebige Menge für $A \subseteq \Omega$
  
  \begin{align*}
    \mu^\ast(A) =
    \begin{cases}
      0, & \text{wenn} \enspace A = \emptyset, \\
      1, & \text{sonst}.
    \end{cases}
  \end{align*}
  
  Zeigen Sie, dass $\mu^\ast$ eine äußere Maßfunktion ist und bestimmen Sie das System der $\mu^\ast$-messbaren Mengen.

\end{enumerate}

\end{exercise}

% -------------------------------------------------------------------------------- %

\begin{solution}

Sei $\mu: \mathfrak{C} \to \R$ eine Mengenfunktion auf dem Mengensystem $\emptyset \neq \mathfrak{C} \subseteq 2^\Omega$.

\begin{itemize}

  \item $\mu \enspace \text{Maßfunktion} : \Leftrightarrow$
  \begin{itemize}
    \item $\Forall A \in \mathfrak{C}: \mu(A) \geq 0,$
    \item $\mu \enspace \sigma \text{-additiv}, \enspace
    \text{d.h.}
    \Forall (A_n) \in \mathfrak{C}, \text{disj.}, \text{höchst. abz.}:
    A := \sum_{n \in \N} A_n \in \mathfrak{C} \Rightarrow
    \mu(A) = \sum_{n \in \N} \mu(A_n)$
  \end{itemize}

  \item $\mu \enspace \text{endliche Maßfunktion} : \Leftrightarrow$
  \begin{itemize}
    \item $\mu \enspace \text{Maßfunktion},$
    \item $\Forall A \in \mathfrak{C}: \mu(A) < \infty$
  \end{itemize}

  \item $\mu \enspace \text{sigmaendliche Maßfunktion} : \Leftrightarrow$
  \begin{itemize}
    \item $\mu \enspace \text{Maßfunktion},$
    \item $\Forall A \in \mathfrak{C}: \Exists (A_n) \in \mathfrak{C}: A \subseteq \bigcup_{n \in \N} A_n, \Forall n \in \N: \mu(A_n) < \infty$
  \end{itemize}

\end{itemize}

Sei $\mu^\ast: 2^\Omega \to \R$ eine weitere Mengenfunktion.

\begin{itemize}

  \item $\mu^\ast \enspace \text{äußere Maßfunktion} : \Leftrightarrow$
  \begin{itemize}
    \item $\mu^\ast(\emptyset) = 0,$
    \item $\Forall A \in 2^\Omega: \mu^\ast(A) \geq 0,$
    \item $\Forall A, B \in 2^\Omega:
    A \subseteq B \Rightarrow
    \mu^\ast(A) \leq \mu^\ast(B),$
    \item $\Forall A, (B_n) \in 2^\Omega:
    A \subseteq \bigcup_{n \in \N} B_n \Rightarrow
    \mu^\ast(A) \leq \sum_{n \in \N} \mu^\ast(B_n)$
  \end{itemize}

\end{itemize}

Der Beweis des Fortsetzungssatz für Maßfunktionen ist nicht teil des Prüfungsstoffs. Dennoch, er besagt folgendes.

\begin{theorem*}[Fortsetzungssatz für Maßfunktionen]

$\mu$ sei ein Maß auf einem Semiring $\mathfrak{T}$. Dann kann man auf dem von $\mathfrak{T}$ erzeugten Sigmaring ein Maß finden, dass auf $\mathfrak{T}$ mit $\mu$ übereinstimmt. \\
Wenn $\mu$ auf $\mathfrak{T}$ sigmaendlich ist, dann ist die Fortsetzung auf den erzeugten Sigmaring eindeutig bestimmt.

\end{theorem*}

Die ersten drei Eigenschaften sind offensichtlich. Seien zuletzt $A, (B_n) \in 2^\Omega$ mit $A \subseteq \bigcup_{n \in \N} B_n$. Falls $A = \emptyset$, sind wir (wegen der zweiten Eigenschaft) fertig und sonst $\Exists n \in \N: B_n \neq \emptyset$. \\

Das System der Carathéodory-messbaren Mengen lautet wie folgt.

\begin{align*}
  \mathfrak{M}_{\mu^\ast} :=
  \Bbraces
  {
    A \subseteq \Omega:
    \Forall B \subseteq \Omega:
    \mu^\ast(B) = \mu^\ast(B \cap A) + \mu^\ast(B \cap A^\complement)
  }
\end{align*}

Weil dieses System eine $\sigma$-Algebra ist, sind $\emptyset, \Omega$ messbar. $A \subseteq \Omega$ mit $A \neq \emptyset, \Omega$, ist wegen $B := \Omega$ nicht messbar, weil $A^\complement \neq \emptyset$ und damit $\mathfrak{M}_{\mu^\ast} = \Bbraces{\emptyset, \Omega}$.

\end{solution}

% -------------------------------------------------------------------------------- %
