% -------------------------------------------------------------------------------- %

\begin{exercise}

\phantom{}

\begin{enumerate}[label = (\alph*)]

  \item
  Definieren Sie: Ring, Semiring, Sigmaring, Algebra, Sigmaalgebra, Dynkin-System, monotones System.
  
  \item
  Von welchem Typ sind folgende Mengensysteme über $\Omega = \R$?
  
  \begin{align*}
    \mathfrak{C}_1 & = \Bbraces{A \subseteq \R: |A| < \infty}, \\
    \mathfrak{C}_2 & = \Bbraces{A \subseteq \R: |A| < 5}, \\
    \mathfrak{C}_3 & = \Bbraces{A \subseteq \R: |A| < \infty, \enspace \text{gerade}}, \\
    \mathfrak{C}_4 & = \Bbraces{A \subseteq \R: \card(A) \leq \aleph_0}, \\
  \end{align*}

\end{enumerate}

\end{exercise}

\begin{solution}

(a) $\emptyset \neq \mathfrak{R}_\sigma \subseteq 2^\Omega \Text{Sigmaring} : \Leftrightarrow$
\begin{itemize}
  \item $\Forall (A_n) \in \mathfrak{R}_\sigma, \text{disj.}: \sum_{n \in \N} A_n \in \mathfrak{R}_\sigma$
  \item $\Forall A, B \in \mathfrak{R}_\sigma: A \setminus B \in \mathfrak{R}_\sigma,$
\end{itemize}

Für den Rest: Siehe Aufgabe 3 (a). \\

(b)

\begin{itemize}

  \item $\mathfrak{C}_1$ ist
  \begin{itemize}
    \item ein Ring, weil mit $|A|, |B| < \infty$, auch $|A \cup B|, |A \setminus B| < \infty$,
    \item ein Semiring, weil Ring,
    \item kein Sigmaring, weil $\Forall n \in \N: \Bbraces{n} \in \mathfrak{C}_1$, aber $\N \notin \mathfrak{C}_1$,
    \item keine Algebra, weil $\R \notin \mathfrak{C}_1$,
    \item keine Sigmaalgebra, weil kein Algebra,
    \item kein Dynkin-System, weil kein monotones System,
    \item und kein monotones System, weil $\Forall n \in \N: \Bbraces{1, \leq, n} \in \mathfrak{C}_1$, aber $\N \notin \mathfrak{C}_1$.
  \end{itemize}

  \item$\mathfrak{C}_2$ ist
  \begin{itemize}
    \item kein Ring, weil $|A|, |B| < 5$, im Allgemeinen nicht $|A \cup B| < 5$ folgt,
    \item ein Semiring, weil $\mathfrak{C}_2$ stabil bzgl. (endlicher) Durchschnitte ist, und $\Forall x \in \R: \Bbraces{x} \in \mathfrak{C}_2$, also \enquote{Leitern} gebaut werden können, bzgl. denen $\mathfrak{C}_2$ stabil ist,
    \item kein Sigmaring, weil kein Ring,
    \item keine Algebra, weil kein Ring,
    \item keine Sigmaalgebra, weil kein Algebra,
    \item kein Dynkin-System, weil $\Forall n \in \N: \Bbraces{n} \in \mathfrak{C}_1$, aber $\N \notin \mathfrak{C}_1$,
    \item ein monotones System, weil jede monotone Mengenfolge, nur höchstens $5$ verschiedene Folgenglieder haben kann, also fast überall konstant ist.
  \end{itemize}

  \item $\mathfrak{C}_3$ ist
  \begin{itemize}
    \item kein Ring, weil kein Semiring,
    \item kein Semiring, weil $\Bbraces{-1, 0}, \Bbraces{0, 1} \in \mathfrak{C}_3$, aber $\Bbraces{-1, 0} \cap \Bbraces{0, 1} = \Bbraces{0} \notin \mathfrak{C}_3$,
    \item kein Sigmaring, weil kein Semiring,
    \item keine Algebra, weil kein Semiring,
    \item keine Sigmaalgebra, weil kein Semiring,
    \item kein Dynkin-System, weil kein monotones System,
    \item kein monotones System, weil $\Forall n \in \N: \Bbraces{1, \ldots, 2n} \in \mathfrak{C}_3$, aber $\N \notin \mathfrak{C}_3$.
  \end{itemize}

  \item $\mathfrak{C}_4$ ist
  \begin{itemize}
    \item ein Ring, weil Sigmaring,
    \item ein Semiring, weil Sigmaring,
    \item ein Sigmaring, weil mit $\card(A_n), \card(A), \card(B) \leq \aleph_0$, $n \in \N$, auch $\card(\bigcup_{n \in \N} A_n), \card(A \setminus B) \leq \aleph_0$,
    \item keine Algebra, weil $\card(\R) > \aleph_0$,
    \item keine Sigmaalgebra, weil keine Algebra,
    \item kein Dynkin-System, weil $\card(\R) > \aleph_0$,
    \item ein monotones System, weil Sigmaring.
  \end{itemize}

\end{itemize}

\end{solution}

% -------------------------------------------------------------------------------- %
