% -------------------------------------------------------------------------------- %

\begin{exercise}

$\mu$ und $\nu$ seinen zwei Maße auf dem Ring $\mathfrak{R}$. Zeigen Sie:

\begin{enumerate}[label = (\alph*)]

  \item
  $(\mu + \nu)^\ast = \mu^\ast + \nu^\ast$
  
  \item
  $\mathfrak{M}_{\mu^\ast + \nu^\ast} \supseteq \mathfrak{M}_{\mu^\ast} \cap \mathfrak{M}_{\nu^\ast}$
  
  \item
  Falls $\mu$ und $\nu$ sigmaendlich sind, dann gilt im vorigen Punkt Gleichheit.

\end{enumerate}

\end{exercise}

% -------------------------------------------------------------------------------- %

\begin{solution}

(a) \enquote{$\geq$}: Sei $A \subseteq \Omega$, dann folgt die eine Ungleichung wegen \\

\begin{align*}
  (\mu + \nu)^\ast(A)
  & =
    \inf \Bbraces{\sum_{n \in \N} (\mu + \nu)(E_n): (E_n) \in \mathfrak{R}, A \subseteq \sum_{n \in \N} E_n} \\
  & \geq
    \inf \Bbraces{\sum_{n \in \N} \mu(E_n): (E_n) \in \mathfrak{R}, A \subseteq \sum_{n \in \N} E_n} +
    \inf \Bbraces{\sum_{n \in \N} \nu(E_n): (E_n) \in \mathfrak{R}, A \subseteq \sum_{n \in \N} E_n} \\
  & =
    \mu^\ast(A) + \nu^\ast(A).
\end{align*}

\enquote{$\leq$}: Für die Andere, finden wir für beliebiges $\epsilon > 0$ zwei disjunkte Überdeckungen $(B_n), (C_m) \in \mathfrak{R}$ von $A$, sodass

\begin{align*}
  \mu^\ast(A) \leq \sum_{n \in \N} \mu^\ast(B_n) \leq \mu^\ast(A) + \frac{\epsilon}{2}, \enspace
  \nu^\ast(A) \leq \sum_{m \in \N} \nu^\ast(C_m) \leq \nu^\ast(A) + \frac{\epsilon}{2}.
\end{align*}

Weil $(B_n \cap C_m)$ wieder eine disjunkte Überdeckungen von $A$ ist, folgt

\begin{align*}
  (\mu + \nu)^\ast(A)
  & \leq
    \sum_{n, m \in \N} (\mu + \nu)^\ast(B_n \cap C_m)
    \leq
    \sum_{n \in \N} \sum_{m \in \N} \mu^\ast(B_n \cap C_m) +
    \sum_{m \in \N} \sum_{n \in \N} \nu^\ast(B_n \cap C_m) \\
  & \leq
    \sum_{n \in \N} \mu^\ast(B_n) +
    \sum_{m \in \N} \nu^\ast(C_m)
    \leq
    \mu^\ast(A) + \nu^\ast(A) + \epsilon.
\end{align*}

\end{solution}

% -------------------------------------------------------------------------------- %
