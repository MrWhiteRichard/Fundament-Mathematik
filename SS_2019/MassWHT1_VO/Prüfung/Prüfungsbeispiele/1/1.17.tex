% -------------------------------------------------------------------------------- %

\begin{exercise}

Ein Würfel wird einmal geworfen, $X$ sei die erzielte Augenzahl. Anschließend werden in eine Urne $X$ weiße und eine schwarze Kugel gelegt, und eine Kugel wird gezogen. $A$ sei das Ereignis, dass die schwarze Kugel gezogen wurde. Bestimmen Sie

\begin{enumerate}[label = (\alph*)]

  \item
  $\mathbf{P}(A)$
  
  \item
  $\mathbf{P}(X = 3 | A)$

\end{enumerate}

\end{exercise}

% -------------------------------------------------------------------------------- %

\begin{solution}

(a) Die Urne enthält also $X + 1$ Kugeln und damit ist $\mathbf{P}(A) = \frac{1}{X + 1}$. \\

(b) Sei $H_i := [X = i]$, für $i = 1, \ldots, 6$, so ist $(H_i)_{i=1}^6$ ein vollständiges Ereignissystem. Offensichtlich, gilt $\Forall i = 1, \ldots, 6: \mathbf{P}(H_i) = \frac{1}{6}$. Nachdem $\mathbf{P}(A) > 0$, können wir das \enquote{Bayes'sche Theorem} anwenden und erhalten

\begin{align*}
  \mathbf{P}(H_3 | A)
  =
  \frac
  {\mathbf{P}(H_3) \mathbf{P}(A | H_3)}
  {\sum_{i=1}^6 \mathbf{P}(H_i) \mathbf{P}(A | H_i)}
  =
  \frac
  {\frac{1}{3+1}}
  {\sum_{i=1}^6 \frac{1}{i+1}}
  = \frac{35}{223}.
\end{align*}

\end{solution}

% -------------------------------------------------------------------------------- %
