% --------------------------------------------------------------------------------

\begin{exercise}

\phantom{}

\begin{enumerate}[label = (\alph*)]

  \item
  $\mathfrak{S}$ sei eine Sigmaalgebra über $\Omega$, $\mathfrak{C} \subset \Omega$. Zeigen Sie

  \begin{align*}
    \mathfrak{A}_\sigma(\mathfrak{S} \cup \Bbraces{C}) =
    \Bbraces{(A \cap C) \cup (B \cap C^\complement), A, B \in \mathfrak{S}}.
  \end{align*}
  
  \item
  $\mathfrak{R}$ sei ein Ring über $\Omega$. Zeigen Sie
  
  \begin{align*}
    \mathfrak{A}(\mathfrak{R}) =
    \mathfrak{R} \cup \Bbraces{A: A^\complement \in \mathfrak{R}}.
  \end{align*}

\end{enumerate}

\end{exercise}

% --------------------------------------------------------------------------------

\begin{solution}

(a) \enquote{$\supseteq$}: Seien $A, B \in \mathfrak{S}$, dann auch $A, B, C \in \mathfrak{A}_\sigma(\mathfrak{S} \cup \Bbraces{C})$. Als $\sigma$-Algebra ist diese bezüglich \enquote{$\cap$}, \enquote{$\cup$}, \enquote{$^\complement$} abgeschlossen. \\

\enquote{$\subseteq$}: Die linke Seite ist die kleinste $\sigma$-Algebra, die $\mathfrak{S} \cup \Bbraces{C}$ enthält. Sei $\mathfrak{S}^\prime$ die rechte Seite, dann muss

\begin{itemize}
  \item $\mathfrak{S} \subseteq \mathfrak{S}^\prime$, wenn man $A = B$ wählt und
  \item $C \in \mathfrak{S}^\prime$, wenn $A := \Omega$, $B := \emptyset$.
\end{itemize}

Wir weisen also noch nach, dass $\mathfrak{S}^\prime$ eine $\sigma$-Algebra ist.

\begin{itemize}

  \item \enquote{Enthält Grundmenge}: $\Omega \in \mathfrak{S} \subseteq \mathfrak{S}^\prime$

  \item \enquote{Stabil bzgl. abzählbaren Vereinigungen}: Sei $(E_n) \in \mathfrak{S}^\prime$, dann $\Exists (A_n), (B_n) \in \mathfrak{S}:$
  \begin{align*}
    \bigcup_{n \in \N} \pbraces{A_n \cap C} \cup \pbraces{B_n \cap C^\complement} =
    \Bigg ( \underbrace{\bigcup_{n \in \N} A_n}_{\in \mathfrak{S}} \cap C \Bigg ) \cup
    \Bigg ( \underbrace{\bigcup_{n \in \N} B_n}_{\in \mathfrak{S}} \cap C^\complement \Bigg ) \cup
    \in \mathfrak{S}^\prime
  \end{align*}

  \item \enquote{Stabil bzgl. Differenzen}: Diese filetieren wir in
  \begin{itemize}

    \item \enquote{Stabil bzgl. Komplement}: Sei $E \in \mathfrak{S}^\prime$, dann $\Exists A, B \in \mathfrak{S}:$
    \begin{align*}
      E^\complement
      = \pbraces{\pbraces{A \cap C} \cup \pbraces{B \cap C^\complement}}^\complement
      & = \pbraces{A^\complement \cup C^\complement} \cap \pbraces{B^\complement \cup C} \\
      & = \underbrace
          {
            \pbraces{A^\complement \cap B^\complement}
          }_{
            \in \mathfrak{S}
          } \cup
          \underbrace
          {
            \pbraces{A^\complement \cap C} \cup \pbraces{C^\complement \cap B^\complement}
          }_{
            \in \mathfrak{S}^\prime
          } \cup
          \underbrace
          {
            \pbraces{C^\complement \cap C}
          }_\emptyset
          \in \mathfrak{S}^\prime
    \end{align*}

    \item \enquote{Stabil bzgl. (endlichen) Durchschnitten}: Seien $E_1, E_2 \in \mathfrak{S}^\prime$, dann $\Exists A_1, B_1, A_2, B_2 \in \mathfrak{S}:$
    \begin{align*}
      E_1 \cap E_2
      & = \pbraces{\pbraces{A_1 \cap C} \cup \pbraces{B_1 \cap C^\complement}} \cup
        \pbraces{\pbraces{A_2 \cap C} \cup \pbraces{B_2 \cap C^\complement}} \\
      & = \pbraces{\pbraces{A_1 \cap C} \cap \pbraces{A_2 \cap C}} \cup
          \underbrace
          {
            \Big (
            \pbraces{B_1 \cap C^\complement} \cap \pbraces{A_2 \cap C}
            \Big )
          }_\emptyset \\
      & \cup \underbrace
          {
            \Big (
            \pbraces{A_1 \cap C} \cap \pbraces{B_2 \cap C^\complement}
            \Big )
            }_\emptyset
            \pbraces
          {
            \pbraces{B_1 \cap C^\complement} \cap
            \pbraces{B_2 \cap C^\complement}
          } \\
      & = \pbraces{A_1 \cap A_2 \cap C} \cup
          \pbraces{B_1 \cap B_2 \cap C^\complement}
          \in \mathfrak{S}^\prime
    \end{align*}

  \end{itemize}

\end{itemize}

(b) \enquote{$\supseteq$}: Offensichtlich, gilt $\mathfrak{A}(\mathfrak{R}) \supseteq \mathfrak{R}$. Weil Algebren die Grundmenge enthalten und stabil bzgl. Differenzen sind, sind sie es auch bzgl. Komplementen. Damit folgt auch $\mathfrak{A}(\mathfrak{R}) \supseteq \mathfrak{R}^\complement$, wobei $^\complement$ punktweise zu verstehen ist. \\

\enquote{$\subseteq$}: So wie in (a), bemerkt man, dass die rechte Seite $\mathfrak{A} \supseteq \mathfrak{R}$ enhält und weist nach, dass $\mathfrak{A}$ eine Algebra ist.

\end{solution}

% --------------------------------------------------------------------------------
