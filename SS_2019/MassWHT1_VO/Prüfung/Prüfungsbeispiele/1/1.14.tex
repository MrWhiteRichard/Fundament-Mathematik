% -------------------------------------------------------------------------------- %

\begin{exercise}

Es sei

\begin{align*}
  \mathfrak{T} = \Bbraces{A \subset \R: |A| \leq 2}.
\end{align*}

\begin{enumerate}[label = (\alph*)]

  \item
  Zeigen Sie: $\mathfrak{T}$ ist ein Semiring.
  
  \item
  Bestimmmen Sie den von $\mathfrak{T}$ erzeugten Ring bzw. Sigmaring.
  
  \item
  Was muss eine Mengenfunktion $\mu$ auf $\mathfrak{T}$ erfüllen, damit sie ein Maß ist.

\end{enumerate}

\end{exercise}

% -------------------------------------------------------------------------------- %

\begin{solution}

(a) \phantom{}

\begin{itemize}

  \item \enquote{Stabil bzgl. (endlichen) Durchschnitten}: Seien $A, B \in \mathfrak{T}$, so auch $A \cap B \in \mathfrak{T}$, weil
  \begin{align*}
    |A \cap B| \leq |A|, |B| \leq 2.
  \end{align*}

  \item \enquote{Enthält Leiterpflöcke} Seien $A, B \in \mathfrak{T}$, mit $A \subseteq B$. Nachdem $\Forall x \in \R: \Bbraces{x} \in \mathfrak{T}$, findet man disjunkte $C_1, \leq, C_n: B \setminus A = \sum_{i=1}^n C_i$.

  \item \enquote{Enthält Unterleitern} Offensichtlich gilt auch $\Forall k = 1, \ldots, n: A \cup \sum_{i=1}^k \in \mathfrak{T}$.

\end{itemize}

(b) Dazu gibt es zwei wunderbare Resultate:

\begin{itemize}

  \item $\mathfrak{R}(\mathfrak{T})
  = \Bbraces{\bigcup_{i=1}^n A_i: (A_i)_{i=1}^n \in \mathfrak{T}^n}
  = \Bbraces{\sum   _{i=1}^n A_i: (A_i)_{i=1}^n \in \mathfrak{T}^n}
  = \Bbraces{A \subset \R: |A| < \infty}$.

  \item Laut dem \enquote{Monotone Class Theorem}, gilt
  $\mathfrak{R}_\sigma(\mathfrak{T})
  = \mathfrak{R}_\sigma(\mathfrak{R}(\mathfrak{T}))
  = \mathfrak{M}(\mathfrak{R}(\mathfrak{T}))$.
  Dieses Mengenssytem ist stabil bzgl. $\lim_{n \to \infty} A_n = \bigcup_{n \in \N} A_n$, also genau $\Bbraces{A \subset \R: |A| \leq \aleph_0}$.

\end{itemize}

(c) Siehe Aufgabe 1 (a).

\end{solution}

% -------------------------------------------------------------------------------- %
