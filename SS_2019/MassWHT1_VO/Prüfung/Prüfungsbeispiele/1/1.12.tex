% -------------------------------------------------------------------------------- %

\begin{exercise}

Zeigen Sie: Falls $\mu_n$ für jedes $n$ eine Maßfunktion auf $(\Omega, \mathfrak{S})$ ist, so ist auch $\mu = \sum_n \mu_n$ eine Maßfunktion.

\end{exercise}

% -------------------------------------------------------------------------------- %

\begin{solution}

Offensichtlich, ist $\Forall A \in \mathfrak{S}: \mu(A) \geq 0$. Es fehlt also nur noch die $\sigma$-Additivität von $\mu$. $\emptyset \in \mathfrak{S}$, also wählen wir $(A_k) \in \mathfrak{S}$ disjunkt.

\begin{align*}
  \mu(\sum_{k \in \N} A_k)
  =
  \sum_{n \in \N} \mu_n(\sum_{k \in \N} A_k)
  =
  \sum_{n \in \N} \sum_{k \in \N} \mu_n(A_k)
  =
  \sum_{k \in \N} \sum_{n \in \N} \mu_n(A_k)
  =
  \sum_{k \in \N} \mu(A_k)
\end{align*}

Dabei darf man die beiden Summen vertauschen, weil die Reihen entweder $\infty$ sind, oder absolut (also unbedingt) konvergieren. Der Rest ist die $\sigma$-Additivität von $\mu_n$ und pure Definition.

\end{solution}

% -------------------------------------------------------------------------------- %
