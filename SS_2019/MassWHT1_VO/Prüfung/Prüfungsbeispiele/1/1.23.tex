% -------------------------------------------------------------------------------- %

\begin{exercise}

\phantom{}

\begin{enumerate}[label = (\alph*)]

  \item
  Definieren Sie: Ring, Semiring, Sigmaring, Algebra, Sigmaalgebra, Dynkin-System.
  
  \item
  $\mu$ und $\nu$ seien zwei Wahrscheinlichkeitsmaße auf dem Messraum $(\Omega, \mathfrak{S})$. Zeigen Sie, dass
  
  \begin{align*}
    \mathfrak{D} = \Bbraces{A \in \mathfrak{S} : \mu(A) = \nu(A)}
  \end{align*}
  
  ein Dynkin-System ist.

\end{enumerate}

\end{exercise}

% -------------------------------------------------------------------------------- %

\begin{solution}

(a) Siehe Aufgabe (a).

(b)

\begin{itemize}

  \item \enquote{Stabil bzgl. Differenzen von Teilmengen}: Seien $A, B \in \mathfrak{D}$ mit $A \subseteq B$, dann gilt Folgendes.
  \begin{align*}
    \mu(B \setminus A)
    & =
    \nu(B \setminus A)
    \Leftrightarrow \\
    \nu(A) + \mu(B \setminus A)
    & =
    \nu(B \setminus A) + \nu(A)
    =
    \nu(B)
    \Leftrightarrow \\
    \nu(A) + \mu(B)
    =
    \mu(A) + \nu(A) + \mu(B \setminus A)
    & =
    \nu(B) + \mu(A)
  \end{align*}

  \item \enquote{Stabil bzgl. abzählbaren, disjunkten Vereinigungen}: Sei $(A_n) \in \mathfrak{D}$ disjunkt, dann
  \begin{align*}
    \mu(\sum_{n \in \N} A_n)
    =
    \sum_{n \in \N} \mu(A_n)
    =
    \sum_{n \in \N} \nu(A_n)
    =
    \nu(\sum_{n \in \N} A_n).
  \end{align*}

  \item \enquote{Enthält Grundmenge}: $\mu(\Omega) = 1 = \nu(\Omega)$

\end{itemize}

\end{solution}

% -------------------------------------------------------------------------------- %
