% -------------------------------------------------------------------------------- %

\begin{exercise}

\begin{enumerate}[label = (\alph*)]

  \item
  Definieren Sie: messbare Funktion, Treppenfunktion, Konvergenz im Maß, Konvergenz fast überall, Konvergenz fast gleichmäßig.
  
  \item
  Formulieren und beweisen Sie den Approximationssatz für reellwertige messbare Funktionen.

\end{enumerate}

\end{exercise}

% -------------------------------------------------------------------------------- %

\begin{solution}

\phantom{}

(a) $f: (\Omega_1, \mathfrak{S}_1) \to (\Omega_2, \mathfrak{S}_2) \enspace \text{Treppenfunktion}
: \Leftrightarrow
\Exists a_1, \ldots, a_n \in \Omega_2,
\Exists A_1, \ldots, A_n \in \mathfrak{S}_1:$

\begin{align*}
  \sum_{i=1}^n A_i = \Omega_1, \enspace
  \sum_{i=1}^n a_i A_i = f
\end{align*}

Rest siehe Aufgabe 1 und 12 (a). \\

(b) Siehe Skript.

\end{solution}

% -------------------------------------------------------------------------------- %
