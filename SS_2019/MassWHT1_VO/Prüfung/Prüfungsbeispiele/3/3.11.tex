% --------------------------------------------------------------------------------

\begin{exercise}

\phantom{}

\begin{enumerate}[label = (\alph*)]

  \item
  Definieren Sie: Konvergenz fast überall, fast überall gleichmäßig, fast gleichmäßig, im Maß.
  
  \item
  Gegeben ist der Maßraum $(\N, 2^\N, \mu)$ mit $\mu(\Bbraces{x}) = 2^{-x}$, $x \in \N$. Zeigen Sie, dass in diesem Maßraum die Konvergenzen fast überall, fast gleichmäßig und im Maß äquivalent sind.

\end{enumerate}

\end{exercise}

% --------------------------------------------------------------------------------

\begin{solution}

(a) Siehe Aufgabe 1. \\

(b) \phantom{}

\begin{itemize}

  \item \enquote{f.ü. $\Rightarrow$ fast glm.}: $\mu$ ist ein endliches Maß, weil
  \begin{align*}
    \mu(\N)
    =
    \sum_{x \in \N} \mu(\Bbraces{x})
    =
    \sum_{x \in \N} 2^{-x}
    =
    2 < \infty.
  \end{align*}
  Also gilt, laut \enquote{Egorov},
  \begin{align*}
    f_n \xrightarrow[n \to \infty]{\text{f.ü.}} f
    \Rightarrow
    f_n \xrightarrow[n \to \infty]{\text{fast glm.}} f.
  \end{align*}

  \item \enquote{fast glm. $\Rightarrow$ im Maß}: Zudem, gilt immer
  \begin{align*}
    f_n \xrightarrow[n \to \infty]{\text{fast glm.}} f
    \Rightarrow
    f_n \xrightarrow[n \to \infty]{\text{im Maß}} f.
  \end{align*}

  \item \enquote{im Maß $\Rightarrow$ f.ü.}:
  \begin{align*}
    f_n \xrightarrow[n \to \infty]{\text{im Maß}} f
  \end{align*}
  heißt, dass $\Forall \epsilon > 0:$
  \begin{align*}
    \sum_{x \in [|f_n - f| > \epsilon]} 2^{-x}
    =
    \mu(|f_n - f| > \epsilon)
    \xrightarrow[n \to \infty]{} 0,
  \end{align*}
  also gilt $\Forall x \in \N: \Exists N \in \N: \Forall n \geq N:$
  \begin{align*}
    x \notin [|f_n - f| > \epsilon]
    \Leftrightarrow
    |f_n(x) - f(x)| \leq \epsilon,
  \end{align*}
  und somit schließlich
  \begin{align*}
    f_n \xrightarrow[n \to \infty]{\text{f.ü.}} f.
  \end{align*}

\end{itemize}

\end{solution}

% --------------------------------------------------------------------------------
