% -------------------------------------------------------------------------------- %

\begin{exercise}

\phantom{}

\begin{enumerate}[label = (\alph*)]

  \item
  Definieren Sie: messbare Funktion, Treppenfunktion, Konvergenz im Maß, Konvergenz fast überall, Konvergenz fast gleichmäßig.
  
  \item
  In welchem Sinn (fast überall gleichmäßig/fast gleichmäßig/fast überall/im Maß) konvergieren die folgenden Folgen in $(\R, \mathfrak{B}, \lambda)$?
  
  \begin{enumerate}[label = \roman*.]

    \item
    $f_n(x) = \sin(x)/n$

    \item
    $f_n(x) = e^{-n |x|}$
    
    \item
    $f_n(x) = x/n$
    
    \item
    $f_n(x) = f_n(x) =
    \begin{cases}
      1 & \text{wenn} \enspace \sqrt{n} - \floor{\sqrt{n}} \leq x \leq \sqrt{n+1} - \floor{\sqrt{n}} \\
      0 & \text{sonst}.
    \end{cases}$

  \end{enumerate}

\end{enumerate}

\end{exercise}

% -------------------------------------------------------------------------------- %

\begin{solution}

Siehe Aufgabe 8.

\end{solution}

% -------------------------------------------------------------------------------- %
