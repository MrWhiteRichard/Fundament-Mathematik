\begin{exercise}

Es sei $\Omega = (0, 1] \times (0, 1]$ und

\begin{align*}
  \mathfrak{T}
  =
  \Bbraces{(a, b] \times (0, 1]: 0 \leq a \leq b \leq 1}
  \cup
  \Bbraces{(0, 1] \times (a, b]: 0 \leq a \leq b \leq 1}
\end{align*}

Ferner sei

\begin{align*}
  \mu((a, b] \times (0, 1]) = \mu((0, 1] \times (a, b]) = b - a
\end{align*}

\begin{itemize}
  \item[(a)] Zeigen Sie: $\mu$ ist ein Inhalt auf $\mathfrak{T}$.
  \item[(b)] Zeigen Sie $\mathfrak{T}$ ist kein Semiring.
  \item[(c)] Zeigen Sie: Die Fortsetzung von $\mu$ zu einem Inhalt auf dem erzeugten Ring ist nicht eindeutig bestimmt.
\end{itemize}

\end{exercise}

% --------------------------------------------------------------------------------

\begin{solution}

(a) Offensichtlich gilt $\Forall A \in \mathfrak{T}: \mu(A) \geq 0$. Zu zeigen, bleibt also nur noch die Additivität. Seien dazu $A_1, \ldots, A_n \in \mathfrak{T}$ disjunkt und $\sum_{i=1}^n A_i \in \mathfrak{T}$. $A_1, \ldots, A_n$ wollen wir aber darstellen, mit $\Forall i = 1, \ldots, n: \Exists a_i, b_i \in (0, 1):$

\begin{align*}
  A_i = (a_i, b_i] \times (0, 1]
  \enspace \text{oder} \enspace
  A_i = (0, 1] \times (a_i, b_i].
\end{align*}

Weil $A_1, \ldots, A_n$ ja disjunkt sind, müssen sie aber alle die selbe, einer der oberen, Darstellungen haben. Weil $\sum_{i=1}^n A_i \in \mathfrak{T}$, müssen $A_1, \ldots, A_n$ auch direkt an einender angrenzen. Das führt zu

\begin{align*}
  \sum_{i=1}^n \mu(A_i)
  =
  \sum_{i=1}^n b_i - a_i
  =
  \mu(\sum_{i=1}^n A_i).
\end{align*}

(b) $\mathfrak{T}$ ist nicht stabil bzgl. (endlichen) Durchschnitten, weil $\Forall a, b, c, d \in (0, 1): a \leq b$, $c \leq d \Rightarrow$

\begin{align*}
  (a, b] \times (0, 1]
  \cap
  (0, 1] \times (c, d]
  =
  (a, b] \times (c, d]
  \notin \mathfrak{T}
\end{align*}

(c)

\end{solution}
