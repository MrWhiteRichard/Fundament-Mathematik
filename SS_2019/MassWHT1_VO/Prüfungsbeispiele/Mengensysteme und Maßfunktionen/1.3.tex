\begin{exercise}

Hier könnte Ihre Werbung stehen!

\begin{itemize}
  \item[(a)] Definieren Sie Ring, Semiring, monotones System, Dynkin-System, Algebra, Sigmaalgebra.
  \item[(b)] Zeigen Sie: Wenn $\mathfrak{R}$ ein Ring ist, dann stimmt das von $\mathfrak{R}$ erzeugte monotone System mit dem erzeugten Sigmaring überein.
\end{itemize}

\end{exercise}

% --------------------------------------------------------------------------------

\begin{solution}

Hier könnte Ihre Werbung stehen!

\begin{itemize}

  \item $\emptyset \neq \mathfrak{R} \subseteq 2^\Omega \enspace \text{Ring} : \Leftrightarrow \Forall A, B \in \mathfrak{R}:$
  \begin{itemize}
    \item $A \cup B \in \mathfrak{R}$
    \item $A \setminus B \in \mathfrak{R}$
  \end{itemize}

  \item $\emptyset \neq \mathfrak{T} \subseteq 2^\Omega \enspace \text{Semiring} : \Leftrightarrow \Forall A, B \in \mathfrak{T}:$
  \begin{itemize}
    \item $A \cap B \in \mathfrak{T},$
    \item $A \subseteq B \Rightarrow \Exists C_1, \ldots, C_n \in \mathfrak{T}, \Text{disj.}:
    B \setminus A = \sum_{i=1}^n C_i,$
    \item $\Forall k = 1, \ldots, n:
    A \cup \sum_{i=1}^k C_i \in \mathfrak{T}$
  \end{itemize}

  \item $\mathfrak{M} \subseteq 2^\Omega \enspace \text{monotones System} : \Leftrightarrow \Forall (A_n) \in \mathfrak{M}, \Text{mon.}: \lim_{n \to \infty} A_n \in \mathfrak{M}$

  \item $\emptyset \neq \mathfrak{D} \subseteq 2^\Omega \enspace \text{Dynkin-System} : \Leftrightarrow$
  \begin{itemize}
    \item $\Forall A, B \in \mathfrak{D}:
    A \subseteq B \Rightarrow B \setminus A \in \mathfrak{D}$
    \item $\Forall (A_n) \in \mathfrak{D}, \text{disj.}: \sum_{n \in \N} A_n \in \mathfrak{D}$
    \item $\Omega \in \mathfrak{D}$
  \end{itemize}

  \item $\emptyset \neq \mathfrak{A} \subseteq 2^\Omega \enspace \text{Algebra} : \Leftrightarrow$
  \begin{itemize}
    \item $\mathfrak{A} \enspace \text{Ring},$
    \item $\Omega \in \mathfrak{A}$
  \end{itemize}

  \item $\emptyset \neq \mathfrak{A}_\sigma \subseteq 2^\Omega \enspace \text{Sigmaalgebra} : \Leftrightarrow$
  \begin{itemize}
    \item $\Forall (A_n) \in \mathfrak{A}_\sigma, \text{disj.}: \sum_{n \in \N} A_n \in \mathfrak{A}_\sigma$
    \item $\Forall A, B \in \mathfrak{A}_\sigma: A \setminus B \in \mathfrak{A}_\sigma,$
    \item $\Omega \in \mathfrak{A}_\sigma$
  \end{itemize}

\end{itemize}

Der nächste Teil ist genau das \Quote{Monotone Class Theorem}! Siehe Skript.

\end{solution}
