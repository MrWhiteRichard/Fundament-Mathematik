\begin{exercise}

Hier könnte Ihre Werbung stehen!

\begin{itemize}
  \item[(a)] Definieren sie Maßfunktion, endliche Maßfunktion, sigmaendliche Maßfunktion, äußere Maßfunktion, von einem Maß erzeugte äußere Maßfunktion.
  \item[(b)] Zeigen Sie, dass eine äußere Maßfunktion auf dem System der messbaren Mengen ein Maß bildet.
\end{itemize}

\end{exercise}

% --------------------------------------------------------------------------------

\begin{solution}

(a) Sei $\inf \emptyset := \infty$. Dann ist die äußere Maßfunktion, für ein Maß $\mu$ auf einem Ring $\mathfrak{R}$, definiert als

\begin{align*}
  \mu^\ast:
  2^\Omega \to \bar \R,
  A \mapsto \inf \Bbraces{\sum_{n \in \N} \mu(E_n): (E_n) \in \mathfrak{R}, A \subseteq \bigcup_{n \in \N} E_n}.
\end{align*}

Für den Rest: siehe Aufgabe 1 (a). \\

(b) Siehe Skript.

\end{solution}
