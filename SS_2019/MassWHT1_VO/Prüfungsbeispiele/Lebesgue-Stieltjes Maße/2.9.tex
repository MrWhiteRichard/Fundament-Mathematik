\begin{exercise}

Gegeben ist die Funktion $F: \R \to \R:$

\begin{align*}
  F(x) =
  \begin{cases}
    x       & \text{für} \enspace x < 0 \\
    x^2 + 1 & \text{für} \enspace 0 \leq x < 1, \\
    2x      & \text{für} \enspace 1 \leq x < 3, \\
    8       & \text{für} \enspace x \geq 3.
  \end{cases}
\end{align*}

Weisen Sie nach, dass $F$ eine Verteilungsfunktion ist, und bestimmen Sie $\mu_F(]0, 1])$, $\mu_F([0, 1])$, $\mu_F(]0, 1[)$ und $\mu_F(\Q)$.

\end{exercise}

% --------------------------------------------------------------------------------

\begin{solution}

Nachweisen:

\begin{itemize}

  \item \Quote{Rechtsstetigkeit}: $F$ ist stückweise stetig und $\Forall x = 0, 1, 3: F \text{ist rechtsstetig bei} \enspace x$.

  \item \Quote{Steigende Monotonie}: $F$ ist stückweise monoton steigend und $\Forall x = 0, 1, 3: F(x - 0) \leq F(x)$.

\end{itemize}

Bestimmen:

\begin{itemize}

  \item $\mu_F(]0, 1]) =$
  \begin{align*}
    F(1) - F(0) = 2 - 1 = 1
  \end{align*}

  \item $\mu_F([0, 1]) =$
  \begin{align*}
    \mu_F \pbraces{\bigcap_{n \in \N} \left ] 0 - \frac{1}{n}, 1 \right]}
    =
    \lim_{n \in \N} \mu_F \pbraces{ \left ] 0 - \frac{1}{n}, 1 \right]}
    =
    \lim_{n \in \N} F(1) - F \pbraces{0 - \frac{1}{n}}
    =
    \lim_{n \in \N} 2 + \frac{1}{n} = 2
  \end{align*}

  \item $\mu_F(]0, 1[) =$
  \begin{align*}
    \mu_F \pbraces{\bigcup_{n \in \N} \left ] 0, 1 - \frac{1}{n} \right ]}
    =
    \lim_{n \in \N} \mu_F \pbraces{\left ] 0, 1 - \frac{1}{n} \right ]}
    =
    \lim_{n \in \N} F \pbraces{1 - \frac{1}{n}} - F(0)
    =
    \lim_{n \in \N} \pbraces{1 - \frac{1}{n}}^2 - 1 = 0
  \end{align*}

  \item $\mu_F(\Q) =$
  \begin{align*}
    \mu_F \pbraces{\sum_{q \in \Q} \Bbraces{q}}
    & =
    \sum_{q \in \Q} \mu_F \pbraces
    {\bigcap_{n \in \N} \left ] q - \frac{1}{n}, q \right ]} \\
    & =
    \sum_{q \in \Q} \lim_{n \in \N} \mu_F \pbraces
    {\left ] q - \frac{1}{n}, q \right ]} \\
    & =
    \sum_{q \in \Q} F(q - 0) - F(q) \\
    & =
    \sum_{x = 0, 1, 3} (F(x) - F(x - 0)) \\
    & =
    (1 - 0) + (2 - 2) + (8 - 6) = 3
  \end{align*}

\end{itemize}

\end{solution}
