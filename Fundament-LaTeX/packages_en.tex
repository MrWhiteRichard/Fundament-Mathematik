% ---------------------------------------------------------------- %

% Accept different input encodings
\usepackage[utf8]{inputenc}

% Standard package for selecting font encodings
\usepackage[T1]{fontenc}

% ---------------------------------------------------------------- %

% Multilingual support for Plain TEX or LATEX
\usepackage[english]{babel}

% ---------------------------------------------------------------- %

% Set all page margins to 1.5cm
\usepackage{fullpage}

% Margin adjustment and detection of odd/even pages
\usepackage{changepage}

% Flexible and complete interface to document dimensions
\usepackage{geometry}

% ---------------------------------------------------------------- %
% mathematics

\usepackage{amsmath}  % AMS mathematical facilities for LATEX
\usepackage{amssymb}
\usepackage{amsfonts} % TEX fonts from the American Mathematical Society
\usepackage{amsthm}   % Typesetting theorems (AMS style)

% Mathematical tools to use with amsmath
\usepackage{mathtools}

% Support for using RSFS fonts in maths
\usepackage{mathrsfs}

% Commands to produce dots in math that respect font size
\usepackage{mathdots}

% "Blackboard-style" cm fonts
\usepackage{bbm}

% Typeset in-line fractions in a "nice" way
\usepackage{nicefrac}

% Typeset quotient structures with LATEX
\usepackage{faktor}

% Vector arrows
\usepackage{esvect}

% St Mary Road symbols for theoretical computer science
\usepackage{stmaryrd}

% ---------------------------------------------------------------- %
% algorithms

% Package for typesetting pseudocode
\usepackage{algpseudocode}

% Typeset source code listings using LATEX
\usepackage{listings}

% Reimplementation of and extensions to LATEX verbatim
\usepackage{verbatim}

% If necessary, please use the following 2 packages locally, but never both.
% This is because the algorithm environment gets defined in both packages, which leads to name conflicts.
% \usepackage{algorithm2e}
% \usepackage{algorithm}

% ---------------------------------------------------------------- %
% utilities

% Definitions with two optional arguments
\usepackage{twoopt}

% Extended conditional commands
\usepackage{xifthen}

% e-TEX tools for LATEX
\usepackage{etoolbox}

% Define commands with suffixes
\usepackage{suffix}

% Extensive support for hypertext in LATEX
\usepackage{hyperref}

% Driver-independent color extensions for LATEX and pdfLATEX
\usepackage{xcolor}

% ---------------------------------------------------------------- %
% graphics

\usepackage{tikz}
\usetikzlibrary{patterns}
\usetikzlibrary{decorations.markings}
\usetikzlibrary{automata}
\usetikzlibrary{positioning}
\usetikzlibrary{arrows}

% Draw tree structures
\usepackage[noeepic]{qtree}

% Enhanced support for graphics
\usepackage{graphicx}

% Figures broken into subfigures
\usepackage{subfig}

% Improved interface for floating objects
\usepackage{float}

% Control float placement
\usepackage{placeins}

% include .pdf-files
\usepackage{pdfpages}

% ---------------------------------------------------------------- %

% Control layout of itemize, enumerate, description
\usepackage{enumitem}

% Intermix single and multiple columns
\usepackage{multicol}
\setlength{\columnsep}{1cm}

% Coloured boxes, for LATEX examples and theorems, etc
\usepackage{tcolorbox}

% ---------------------------------------------------------------- %
% tables

% Tabulars with adjustable-width columns
\usepackage{tabularx}

% Tabular column heads and multilined cells
\usepackage{makecell}

% Publication quality tables in LATEX
\usepackage{booktabs}

% ---------------------------------------------------------------- %
% bibliography and quoting

% Sophisticated Bibliographies in LATEX
\usepackage[backend = biber, style = alphabetic]{biblatex}

% Context sensitive quotation facilities
\usepackage{csquotes}

% ---------------------------------------------------------------- %
