% ---------------------------------------------------------------- %

% special letters:

\newcommand{\N}{\mathbb{N}}
\newcommand{\Z}{\mathbb{Z}}
\newcommand{\Q}{\mathbb{Q}}
\newcommand{\R}{\mathbb{R}}
\newcommand{\C}{\mathbb{C}}
\newcommand{\K}{\mathbb{K}}
\newcommand{\T}{\mathbb{T}}
\newcommand{\E}{\mathbb{E}}
\newcommand{\V}{\mathbb{V}}
\renewcommand{\S}{\mathbb{S}}
\renewcommand{\P}{\mathbb{P}}
\newcommand{\1}{\mathbbm{1}}
\newcommand{\G}{\mathbb{G}}

% ---------------------------------------------------------------- %

% quantors:

\newcommand{\Forall}{\forall \,}
\newcommand{\Exists}{\exists \,}
\newcommand{\nExists}{\nexists \,}
\newcommand{\ExistsOnlyOne}{\exists! \,}
\newcommand{\nExistsOnlyOne}{\nexists! \,}
\newcommand{\ForAlmostAll}{\forall^\infty \,}

% ---------------------------------------------------------------- %

% graphics:

\newcommand
{\includegraphicsboxed}
[2][0.75]
{
  \begin{center}
    \begin{tcolorbox}[standard jigsaw, opacityback = 0]

      \centering
      \includegraphics[width = #1 \textwidth]{#2}

    \end{tcolorbox}
  \end{center}
}

\newcommand
{\includegraphicsunboxed}
[2][0.75]
{
  \begin{center}
    \includegraphics[width = #1 \textwidth]{#2}
  \end{center}
}

\newcommandtwoopt
{\includegraphicsgraphicsboxed}
[4][0.75][0.25]
{
  \begin{center}
    \begin{tcolorbox}[standard jigsaw, opacityback = 0]

      \centering
      \includegraphics[width = #1 \textwidth]{#3} \\
      \vspace{#2 cm}
      \includegraphics[width = #1 \textwidth]{#4}

    \end{tcolorbox}
  \end{center}
}

\newcommandtwoopt
{\includegraphicsgraphicsunboxed}
[4][0.75][0.25]
{
  \begin{center}

    \centering
    \includegraphics[width = #1 \textwidth]{#3} \\
    \vspace{#2 cm}
    \includegraphics[width = #1 \textwidth]{#4}

  \end{center}
}

% ---------------------------------------------------------------- %

% braces:

\newcommand{\pbraces}[1]{{\left  ( #1 \right  )}}
\newcommand{\bbraces}[1]{{\left  [ #1 \right  ]}}
\newcommand{\Bbraces}[1]{{\left \{ #1 \right \}}}
\newcommand{\vbraces}[1]{{\left  | #1 \right  |}}
\newcommand{\Vbraces}[1]{{\left \| #1 \right \|}}

\newcommand{\abraces}[1]{{\left \langle #1 \right \rangle}}

\newcommand{\floorbraces}[1]{{\left \lfloor #1 \right \rfloor}}
\newcommand{\ceilbraces} [1]{{\left \lceil  #1 \right \rceil }}

\newcommand{\dbbraces}    [1]{{\llbracket     #1 \rrbracket}}
\newcommand{\dpbraces}    [1]{{\llparenthesis #1 \rrparenthesis}}
\newcommand{\dfloorbraces}[1]{{\llfloor       #1 \rrfloor}}
\newcommand{\dceilbraces} [1]{{\llceil        #1 \rrceil}}

\newcommand{\dabraces}[1]{{\left \langle \left \langle #1 \right \rangle \right \rangle}}

\newcommand{\abs}[1]{\vbraces{#1}}
\newcommand{\round}[1]{\bbraces{#1}}
\newcommand{\floor}[1]{\floorbraces{#1}}
\newcommand{\ceil}[1]{\ceilbraces{#1}}

% ---------------------------------------------------------------- %

% MISC

% metric spaces
\newcommand{\norm}[2][]{\Vbraces{#2}_{#1}}
\DeclareMathOperator{\metric}{d}
\DeclareMathOperator{\dist}{dist}
\DeclareMathOperator{\diam}{diam}

% orthogonality
\newcommand{\orthogonal}[3][]{#2 ~\bot_{#1}~ #3}

% exclusive 'or'
\newcommand{\lxor}{\dot \lor}

% O-notation
\newcommand{\landau}{{\scriptstyle \mathcal{O}}}
\newcommand{\Landau}{\mathcal{O}}

% ---------------------------------------------------------------- %

% math operators

% hyperbolic trigonometric functions inverses
\DeclareMathOperator{\areasinh}{areasinh}
\DeclareMathOperator{\areacosh}{areacosh}
\DeclareMathOperator{\areatanh}{areatanh}

% special functions
\DeclareMathOperator{\id}{id}
\DeclareMathOperator{\sgn}{sgn}
\DeclareMathOperator{\Inv}{Inv}
\DeclareMathOperator{\erf}{erf}
\DeclareMathOperator{\pv}{pv}

% exponential function as power
\WithSuffix \newcommand \exp* [1]{\mathrm{e}^{#1}}

% operations on sets
\DeclareMathOperator{\meas}{meas}
\DeclareMathOperator{\card}{card}
\DeclareMathOperator{\Span}{span}
\DeclareMathOperator{\conv}{conv}
\DeclareMathOperator{\cof}{cof}
\DeclareMathOperator{\mean}{mean}

% number theory stuff
\DeclareMathOperator{\ggT}{ggT}
\DeclareMathOperator{\kgV}{kgV}
\DeclareMathOperator{\modulo}{mod}

% polynomial stuff
\DeclareMathOperator{\ord}{ord}
\DeclareMathOperator{\grad}{grad}

% function properties
\DeclareMathOperator{\Ker}{ker}
\DeclareMathOperator{\ran}{ran}
\DeclareMathOperator{\supp}{supp}
\DeclareMathOperator{\graph}{graph}
\DeclareMathOperator{\dom}{dom}
\DeclareMathOperator{\Def}{def}
\DeclareMathOperator{\rg}{rg}

% matrix stuff
\DeclareMathOperator{\GL}{GL}
\DeclareMathOperator{\SL}{SL}
\DeclareMathOperator{\U}{U}
\DeclareMathOperator{\SU}{SU}
\DeclareMathOperator{\PSU}{PSU}
% \DeclareMathOperator{\O}{O}
% \DeclareMathOperator{\PO}{PO}
% \DeclareMathOperator{\PSO}{PSO}
\DeclareMathOperator{\diag}{diag}

% algebra stuff
\DeclareMathOperator{\At}{At}
\DeclareMathOperator{\Ob}{Ob}
\DeclareMathOperator{\Hom}{Hom}
\DeclareMathOperator{\End}{End}
\DeclareMathOperator{\Aut}{Aut}
\DeclareMathOperator{\Lin}{L}

% other function classes
\DeclareMathOperator{\Lip}{Lip}
\DeclareMathOperator{\Mod}{Mod}
\DeclareMathOperator{\Dil}{Dil}

% constants
\DeclareMathOperator{\NIL}{NIL}
\DeclareMathOperator{\eps}{eps}

% ---------------------------------------------------------------- %

% extra fractions:

\newcommand{\Frac}[2]{\frac{1}{#2} \pbraces{#1}}
\newcommand{\nfrac}[2]{\nicefrac{#1}{#2}}

% ---------------------------------------------------------------- %

% derivatives & integrals:

\newcommandtwoopt
{\Int}[4][][]
{\int_{#1}^{#2} #3 ~\mathrm{d} #4}

\newcommandtwoopt
{\Iint}[5][][]
{\iint_{#1}^{#2} #3 ~\mathrm{d} #4 ~\mathrm{d} #5}

\newcommandtwoopt
{\Iiint}[6][][]
{\iiint_{#1}^{#2} #3 ~\mathrm{d} #4 ~\mathrm{d} #5 \dots ~\mathrm{d} #6}

\newcommandtwoopt
{\Iiiint}[7][][]
{\iiiint_{#1}^{#2} #3 ~\mathrm{d} #4 ~\mathrm{d} #5 \dots ~\mathrm{d} #6 \dots ~\mathrm{d} #7}

\newcommandtwoopt
{\Idotsint}[5][][]
{\idotsint_{#1}^{#2} #3 ~\mathrm{d} #4 \dots ~\mathrm{d} #5}

\newcommandtwoopt
{\Oint}[4][][]
{\oint_{#1}^{#2} #3 ~\mathrm{d} #4}

\newcommandtwoopt
{\derivative}[3][][]
{
  \frac
  {\mathrm{d}^{#1} #2}
  {\mathrm{d} #3^{#1}}
}

\newcommandtwoopt
{\pderivative}[3][][]
{
  \frac
  {\partial^{#1} #2}
  {\partial #3^{#1}}
}

\newcommand
{\primeprime}
{{\prime \prime}}

\newcommand
{\primeprimeprime}
{{\prime \prime \prime}}

\DeclareMathOperator{\Div}{div}
\DeclareMathOperator{\rot}{rot}
\DeclareMathOperator{\curl}{curl}
\DeclareMathOperator{\Hess}{Hess}

% ---------------------------------------------------------------- %

% Text:

\newcommand{\Quote}[1]{\glqq #1\grqq{}}

% ---------------------------------------------------------------- %
