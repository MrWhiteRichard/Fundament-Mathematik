% -------------------------------- %
% amsthm-stuff:

\theoremstyle{definition}

% numbered theorems
\newtheorem{theorem}    {Satz}   [section]
\newtheorem{lemma}      [theorem]{Lemma}
\newtheorem{corollary}  [theorem]{Korollar}
\newtheorem{proposition}[theorem]{Proposition}
\newtheorem{remark}     [theorem]{Bemerkung}
\newtheorem{definition} [theorem]{Definition}
\newtheorem{example}    [theorem]{Beispiel}

% unnumbered theorems
\newtheorem*{theorem*}    {Satz}
\newtheorem*{lemma*}      {Lemma}
\newtheorem*{corollary*}  {Korollar}
\newtheorem*{proposition*}{Proposition}
\newtheorem*{remark*}     {Bemerkung}
\newtheorem*{definition*} {Definition}
\newtheorem*{example*}    {Beispiel}

% Please define this stuff in project ("main.tex"):

% \def \lastexercisenumber {...}
% This will be 0 by default

% \setcounter{section}{...}
% This will be 0 by default
% and hence, completely ignored

\ifnum \thesection = 0
{
  \newtheorem{exercise}{Aufgabe}
}
\else
{
  \newtheorem{exercise}{Aufgabe}[section]
}
\fi

\ifdef
{\lastexercisenumber}
{\setcounter{exercise}{\lastexercisenumber}}

\newenvironment{solution}
{
  \begin{proof}[Lösung]
}{
  \end{proof}
}

\renewcommand{\proofname}{Beweis}

% -------------------------------- %
% environment zum einkasteln:

% dickere vertical lines
\newcolumntype
{x}
[1]
{
  !{
    \centering
    \arraybackslash
    \vrule
    width #1}
}

% environment selbst (the big cheese)
\newenvironment
{boxedin}
{
  \begin{tabular}
  {
    x{1 pt}
    p{\textwidth}
    x{1 pt}
  }
  \Xhline
  {2 \arrayrulewidth}
}
{
  \\
  \Xhline{2 \arrayrulewidth}
  \end{tabular}
}

% -------------------------------- %
% MISC "Ein-Deutschungen"

\renewcommand{\figurename}{Abbildung}
\renewcommand{\tablename} {Tabelle}

% -------------------------------- %
