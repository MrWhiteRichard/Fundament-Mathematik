% -------------------------------- %
% amsthm-stuff:

\theoremstyle{definition}

% numbered theorems
\newtheorem{theorem}{Satz}
\newtheorem{lemma}{Lemma}
\newtheorem{corollary}{Korollar}
\newtheorem{proposition}{Proposition}
\newtheorem{remark}{Bemerkung}
\newtheorem{definition}{Definition}
\newtheorem{example}{Beispiel}

% unnumbered theorems
\newtheorem*{theorem*}{Satz}
\newtheorem*{lemma*}{Lemma}
\newtheorem*{corollary*}{Korollar}
\newtheorem*{proposition*}{Proposition}
\newtheorem*{remark*}{Bemerkung}
\newtheorem*{definition*}{Definition}
\newtheorem*{example*}{Beispiel}

% Please define this stuff in project ("main.tex"):

% \def \lastexercisenumber {...}
% This will be 0 by default

% \setcounter{section}{...}
% This will be 0 by default
% and hence, completely ignored

\ifnum \thesection = 0
{\newtheorem{exercise}{Aufgabe}}
\else
{\newtheorem{exercise}{Aufgabe}[section]}
\fi

\ifdef
{\lastexercisenumber}
{\setcounter{exercise}{\lastexercisenumber}}

% The following code is obsolete.
% Please use "exercise"-environment instead

\newtheorem{inneralgebraUE}{UE}
\newenvironment{algebraUE}[1]
{
  \renewcommand \theinneralgebraUE{#1} \inneralgebraUE
}
{\endinneralgebraUE}

\newcommand{\solution}
{
    \renewcommand{\proofname}{Lösung}
    \renewcommand{\qedsymbol}{}
    \proof
}

\renewcommand{\proofname}{Beweis}

% -------------------------------- %
% environment zum einkasteln:

% dickere vertical lines
\newcolumntype
{x}
[1]
{!{\centering\arraybackslash\vrule width #1}}

% environment selbst (the big cheese)
\newenvironment
{boxedin}
{
  \begin{tabular}
  {
    x{1 pt}
    p{\textwidth}
    x{1 pt}
  }
  \Xhline
  {2 \arrayrulewidth}
}
{
  \\
  \Xhline{2 \arrayrulewidth}
  \end{tabular}
}

% -------------------------------- %
% MISC "Ein-Deutschungen"

\renewcommand
{\figurename}
{Abbildung}

\renewcommand
{\tablename}
{Tabelle}

% -------------------------------- %
