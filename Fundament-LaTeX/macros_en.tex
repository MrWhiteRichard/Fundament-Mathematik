% ---------------------------------------------------------------- %
% special letters

\newcommand{\N}{\mathbb N}
\newcommand{\Z}{\mathbb Z}
\newcommand{\Q}{\mathbb Q}
\newcommand{\R}{\mathbb R}
\newcommand{\C}{\mathbb C}
\newcommand{\K}{\mathbb K}
\newcommand{\T}{\mathbb T}
\newcommand{\E}{\mathbb E}
\newcommand{\V}{\mathbb V}
\renewcommand{\S}{\mathbb S}
\renewcommand{\P}{\mathbb P}
\newcommand{\1}{\mathbbm 1}
\newcommand{\G}{\mathbb G}

\newcommand{\iu}{\mathrm i}

% ---------------------------------------------------------------- %
% quantors

\newcommand{\Forall}        {\forall \,}
\newcommand{\Exists}        {\exists \,}
\newcommand{\nExists}       {\nexists \,}
\newcommand{\ExistsOnlyOne} {\exists! \,}
\newcommand{\nExistsOnlyOne}{\nexists! \,}
\newcommand{\ForAlmostAll}  {\forall^\infty \,}

% ---------------------------------------------------------------- %
% graphics boxed

\newcommand
{\includegraphicsboxed}
[2][0.75]
{
  \begin{center}
    \begin{tcolorbox}[standard jigsaw, opacityback = 0]

      \centering
      \includegraphics[width = #1 \textwidth]{#2}

    \end{tcolorbox}
  \end{center}
}

\newcommand
{\includegraphicsunboxed}
[2][0.75]
{
  \begin{center}
    \includegraphics[width = #1 \textwidth]{#2}
  \end{center}
}

\NewDocumentCommand
{\includegraphicsgraphicsboxed}
{ O{0.75} O{0.25} m m}
{
  \begin{center}
    \begin{tcolorbox}[standard jigsaw, opacityback = 0]

      \centering
      \includegraphics[width = #1 \textwidth]{#3} \\
      \vspace{#2 cm}
      \includegraphics[width = #1 \textwidth]{#4}

    \end{tcolorbox}
  \end{center}
}

\NewDocumentCommand
{\includegraphicsgraphicsunboxed}
{ O{0.75} O{0.25} m m}
{
  \begin{center}

    \centering
    \includegraphics[width = #1 \textwidth]{#3} \\
    \vspace{#2 cm}
    \includegraphics[width = #1 \textwidth]{#4}

  \end{center}
}

% ---------------------------------------------------------------- %
% braces

\newcommand{\pbraces}[1]{{\left  ( #1 \right  )}}
\newcommand{\bbraces}[1]{{\left  [ #1 \right  ]}}
\newcommand{\Bbraces}[1]{{\left \{ #1 \right \}}}
\newcommand{\vbraces}[1]{{\left  | #1 \right  |}}
\newcommand{\Vbraces}[1]{{\left \| #1 \right \|}}

\newcommand{\abraces}[1]{{\left \langle #1 \right \rangle}}

\newcommand{\floorbraces}[1]{{\left \lfloor #1 \right \rfloor}}
\newcommand{\ceilbraces} [1]{{\left \lceil  #1 \right \rceil }}

\newcommand{\dbbraces}    [1]{{\llbracket     #1 \rrbracket}}
\newcommand{\dpbraces}    [1]{{\llparenthesis #1 \rrparenthesis}}
\newcommand{\dfloorbraces}[1]{{\llfloor       #1 \rrfloor}}
\newcommand{\dceilbraces} [1]{{\llceil        #1 \rrceil}}

\newcommand{\dabraces}[1]{{\left \langle \left \langle #1 \right \rangle \right \rangle}}

\newcommand{\abs}  [1]{\vbraces{#1}}
\newcommand{\round}[1]{\bbraces{#1}}
\newcommand{\floor}[1]{\floorbraces{#1}}
\newcommand{\ceil} [1]{\ceilbraces{#1}}

% ---------------------------------------------------------------- %

% MISC

% metric spaces
\newcommand{\norm}[2][]{\Vbraces{#2}_{#1}}
\DeclareMathOperator{\metric}{d}
\DeclareMathOperator{\dist}  {dist}
\DeclareMathOperator{\diam}  {diam}

% O-notation
\newcommand{\landau}{{\scriptstyle \mathcal{O}}}
\newcommand{\Landau}{\mathcal{O}}

% ---------------------------------------------------------------- %

% math operators

% hyperbolic trigonometric function inverses
\DeclareMathOperator{\areasinh}{areasinh}
\DeclareMathOperator{\areacosh}{areacosh}
\DeclareMathOperator{\areatanh}{areatanh}

% special functions
\DeclareMathOperator{\id} {id}
\DeclareMathOperator{\sgn}{sgn}
\DeclareMathOperator{\Inv}{Inv}
\DeclareMathOperator{\erf}{erf}
\DeclareMathOperator{\pv} {pv}

% exponential function as power
\WithSuffix \newcommand \exp* [1]{\mathrm{e}^{#1}}

% operations on sets
\DeclareMathOperator{\meas}{meas}
\DeclareMathOperator{\card}{card}
\DeclareMathOperator{\Span}{span}
\DeclareMathOperator{\conv}{conv}
\DeclareMathOperator{\cof}{cof}
\DeclareMathOperator{\mean}{mean}
\DeclareMathOperator{\avg}{avg}
\DeclareMathOperator*{\argmax}{argmax}
\DeclareMathOperator*{\argsmax}{argsmax}

% number theory stuff
\DeclareMathOperator{\kgV}{kgV}
\DeclareMathOperator{\modulo}{mod}

% polynomial stuff
\DeclareMathOperator{\ord}{ord}

% function properties
\DeclareMathOperator{\ran}{ran}
\DeclareMathOperator{\supp}{supp}
\DeclareMathOperator{\graph}{graph}
\DeclareMathOperator{\dom}{dom}
\DeclareMathOperator{\Def}{def}
\DeclareMathOperator{\rg}{rg}

% matrix stuff
\DeclareMathOperator{\GL}{GL}
\DeclareMathOperator{\SL}{SL}
\DeclareMathOperator{\U}{U}
\DeclareMathOperator{\SU}{SU}
\DeclareMathOperator{\PSU}{PSU}
% \DeclareMathOperator{\O}{O}
% \DeclareMathOperator{\PO}{PO}
% \DeclareMathOperator{\PSO}{PSO}
\DeclareMathOperator{\diag}{diag}

% algebra stuff
\DeclareMathOperator{\At}{At}
\DeclareMathOperator{\Ob}{Ob}
\DeclareMathOperator{\Hom}{Hom}
\DeclareMathOperator{\End}{End}
\DeclareMathOperator{\Aut}{Aut}
\DeclareMathOperator{\Lin}{L}

% other function classes
\DeclareMathOperator{\Lip}{Lip}
\DeclareMathOperator{\Mod}{Mod}
\DeclareMathOperator{\Dil}{Dil}

% constants
\DeclareMathOperator{\NIL}{NIL}
\DeclareMathOperator{\eps}{eps}

% ---------------------------------------------------------------- %
% doubble & tripple powers

\newcommand
{\primeprime}
{{\prime \prime}}

\newcommand
{\primeprimeprime}
{{\prime \prime \prime}}

\newcommand
{\astast}
{{\ast \ast}}

\newcommand
{\astastast}
{{\ast \ast \ast}}

% ---------------------------------------------------------------- %
% derivatives

\NewDocumentCommand
{\derivative}
{ O{} O{} m m}
{
  \frac
  {\mathrm d^{#2} {#1}}
  {\mathrm d {#3}^{#2}}
}

\NewDocumentCommand
{\pderivative}
{ O{} O{} m m}
{
  \frac
  {\partial^{#2} {#1}}
  {\partial {#3}^{#2}}
}

\DeclareMathOperator{\Div}{div}
\DeclareMathOperator{\curl}{curl}

% ---------------------------------------------------------------- %
% integrals

\NewDocumentCommand
{\Int}
{ O{} O{} m m}
{\int_{#1}^{#2} #3 ~\mathrm d #4}

\NewDocumentCommand
{\Iint}
{ O{} O{} m m m}
{\iint_{#1}^{#2} #3 ~\mathrm d #4 ~\mathrm d #5}

\NewDocumentCommand
{\Iiint}
{ O{} O{} m m m m}
{\iiint_{#1}^{#2} #3 ~\mathrm d #4 ~\mathrm d #5 ~\mathrm d #6}

\NewDocumentCommand
{\Iiiint}
{ O{} O{} m m m m m}
{\iiiint_{#1}^{#2} #3 ~\mathrm d #4 ~\mathrm d #5 ~\mathrm d #6 ~\mathrm d #7}

\NewDocumentCommand
{\Iint}
{ O{} O{} m m m}
{\idotsint_{#1}^{#2} #3 ~\mathrm d #4 \dots ~\mathrm d #5}

\NewDocumentCommand
{\Oint}
{ O{} O{} m m}
{\oint_{#1}^{#2} #3 ~\mathrm d #4}

% ---------------------------------------------------------------- %
