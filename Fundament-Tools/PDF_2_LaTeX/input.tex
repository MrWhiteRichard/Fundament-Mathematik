Man spricht von einem autonomen Anfangswertproblem, wenn die rechte Seite der Differentialgleichung nicht explizit von der Zeit abh�angt, d.h. wenn Y L�osung des Anfangswertproblems
Y
0
(t) = F(Y (t)), t ? J, Y (t0) = Y0 (2)
ist. Jedes ,,normale� Anfangswertproblem
y = f(t, y(t)), t ? J, y(t0) = y0 (3)
1
kann �aquivalent in ein autonomes Anfangswertproblem mit Y (t) := (t, y(t))>, Y0 := (t0, y0)
>
und F(x) := (1, f(x))> umgeformt werden. Ein Einschrittverfahren heisst invariant gegenuber �
Autonomisierung, wenn es angewendet auf (2) fur beliebiges � f exakt die gleichen Approximationen
erzeugt wie bei Anwendung auf (3).
Zeigen Sie: Ein explizites, s-stufiges Runge-Kutte-Verfahren ist genau dann invariant gegenuber �
Autonomisierung, wenn gilt
cj =
X
j-1
i=1
aji, j = 1, . . . , s. (4)