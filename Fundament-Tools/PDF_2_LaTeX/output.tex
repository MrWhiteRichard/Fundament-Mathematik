Man spricht von einem autonomen Anfangswertproblem, wenn die rechte Seite der Differentialgleichung nicht explizit von der Zeit abh�ngt, d.h. wenn $Y$ L�sung des Anfangswertproblems
Y^\prime(t) = F(Y(t)), t \in J, Y(t_0) = Y_0 (2)
ist.
Jedes \Quote{normale} Anfangswertproblem
y = f(t, y(t)), t \in J, y(t_0) = y_0 (3)
kann �quivalent in ein autonomes Anfangswertproblem mit Y(t) := (t, y(t))^T, Y_0 := (t_0, y_0)^T und F(x) := (1, f(x))^T umgeformt werden.
Ein Einschrittverfahren heisst invariant gegen�ber Autonomisierung, wenn es angewendet auf (2) f�r beliebiges $f$ exakt die gleichen Approximationen
erzeugt wie bei Anwendung auf (3). \\

Zeigen Sie:
Ein explizites, $s$-stufiges Runge-Kutte-Verfahren ist genau dann invariant gegen�ber Autonomisierung, wenn gilt
c_j =
X
j-1
i=1
a_{ji}, j = 1, \ldots, s.
(4)