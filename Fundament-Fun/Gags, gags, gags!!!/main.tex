\documentclass{article}

\input{../../Fundament-LaTeX/packages_de.tex}
\input{../../Fundament-LaTeX/macros_de.tex}
% ---------------------------------------------------------------- %
% amsthm-environments:

\theoremstyle{definition}

% numbered theorems
\newtheorem{theorem}             {Satz}[section]
\newtheorem{lemma}      [theorem]{Lemma}
\newtheorem{corollary}  [theorem]{Korollar}
\newtheorem{proposition}[theorem]{Proposition}
\newtheorem{remark}     [theorem]{Bemerkung}
\newtheorem{definition} [theorem]{Definition}
\newtheorem{example}    [theorem]{Beispiel}
\newtheorem{heuristics} [theorem]{Heuristik}

% unnumbered theorems
\newtheorem*{theorem*}    {Satz}
\newtheorem*{lemma*}      {Lemma}
\newtheorem*{corollary*}  {Korollar}
\newtheorem*{proposition*}{Proposition}
\newtheorem*{remark*}     {Bemerkung}
\newtheorem*{definition*} {Definition}
\newtheorem*{example*}    {Beispiel}
\newtheorem*{heuristics*} {Heuristik}

% ---------------------------------------------------------------- %
% exercise- and solution-environments:

\newtheorem{exercise}{Aufgabe}

% if the exercise counter should start at a given exercise number please include the following in the main.tex document
% \setcounter{exercise}{<last exercise number>}

\newenvironment{solution}
{
  \begin{proof}[Lösung]
}{
  \end{proof}
}

% ---------------------------------------------------------------- %
% a tcolorbox-preset designed to mimic the text boxes typically used by Prof. Stefan Hetzl

% starting template:
% https://tex.stackexchange.com/a/527829

% provide box title as optional argument
\newtcolorbox[auto counter]{hetzlbox}[1][]{%
    colback = white,
    coltitle = black,
    fonttitle = \bfseries,
    sharp corners,
    detach title,
    width = 12cm,
    #1,
    code = {\ifdefempty{\tcbtitletext}{}{\tcbset{before upper = {{\centering \tcbtitle \par} \medskip}}}},
    boxrule = 0.5pt
}

% ---------------------------------------------------------------- %
% MISC translations for environment-names

\renewcommand{\proofname} {Beweis}
\renewcommand{\figurename}{Abbildung}
\renewcommand{\tablename} {Tabelle}

% ---------------------------------------------------------------- %


\newtheorem{gag}{Scherz}[section]

\parskip 0pt
\parindent 0pt

\title
{
    Gags, gags, gags!!! \\
    \vspace{4pt}
    \normalsize
    \textit{Hierhin kommen alle Späße die uns einfallen!}
}
\author{}
\date{}

\begin{document}

\maketitle

\section{Apples}

\begin{gag}

    \href{https://knowyourphrase.com/how-do-you-like-them-apples}{Zitat} aus \enquote{Good Will Hunting}: \\

    Hunting:
    \enquote{Do you like apples?}

    Rival:
    \enquote{Yeah.}

    Hunting:
    \enquote{Well, I got her number. How do you like them apples?}

\end{gag}

\begin{gag}

    Ein \textit{Spindeltorus} entsteht durch Rotation des Kreissegmentes

    \begin{align*}
        x & = (R + r \cos{\theta}) \cos{\varphi} \\
        y & = (R + r \cos{\theta}) \sin{\varphi} \\
        z & = r \sin{\theta}
    \end{align*}

    um die $z$-Achse für $0 < R < r$, $R + r \cos{\theta} \geq 0$.

    \begin{figure}[H]
        \centering
        \subfloat[Paul Winkler]{
            \includegraphics[width = 0.35 \textwidth]{images/apfel}
        }
        \subfloat[Michael Kaltenbäck]{
            \includegraphics[width = 0.35 \textwidth]{images/apple}
        }
        \hspace{0mm}
        \caption{Der Spindeltorus als Apfel}
        \label{fig:apfel_apple}
    \end{figure}

\end{gag}

\section{xkcd's}

\begin{figure}[H]
    \centering
    \subfloat[GIT]
    {
        \includegraphics[width = 0.3 \textwidth]{images/git_2x.png}
    }
    \subfloat[Depth-, Breadth-, \dots]
    {
        \includegraphics[width = 0.3 \textwidth]{images/depth_and_breadth2.png}
    }
    \subfloat[Frequentists vs Bayesians]
    {
        \includegraphics[width = 0.3 \textwidth]{images/frequentists_vs_bayesians.png}
    }
    \hspace{0mm}
    \caption{xkcd's}
    \label{fig:xkcd}
\end{figure}

\section{MISC}

\begin{theorem}[von Goldstern]

    \begin{align*}
        B(0, 1)
        =
        \text{Einheits-Kreis}
        \approx
        \text{Einheits-Greis}
        \approx
        \text{Einheiz-Greis}
    \end{align*}

\end{theorem}

\begin{proof}
    Bekanntes Resultat aus LinAG.
\end{proof}

\begin{theorem}

    \enquote{A Kuh is a Vieh vom Mühl-Viertl}, d.h.

    \begin{align*}
        a q = a \phi \pbraces{\frac{\mu}{4}}.
    \end{align*}

\end{theorem}

\begin{proof}
    Triviale Übungsaufgabe!
\end{proof}

\end{document}
