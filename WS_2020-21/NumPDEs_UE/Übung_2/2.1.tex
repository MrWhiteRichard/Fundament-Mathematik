% --------------------------------------------------------------------------------

\begin{exercise}
Sei $H$ ein Hilbertraum und $a: H \times H \rightarrow \R$ eine stetige,
ellipitische und symmetrische Bilinearform, dh.h. es existieren Konstanten
$\alpha, \beta > 0$ sodass

\begin{align*}
  a(u;v)
  \leq
  \beta \norm{u}\norm{v},
  \quad
  a(u,u) \geq \alpha \norm{u}^2,
  \quad
  u,v \in H.
\end{align*}

Wir definieren auf dem Hilbertraum $H \times H$ mit Skalarprodukt

\begin{align*}
  ((u_1;u_2);(v_1;v_2))_{H \times H} := (u_1;v_1)_H + (u_2;v_2)_H
\end{align*}

die Bilinearform

\begin{align*}
  b((u_1;u_2);(v_1;v_2)) := a(u_1;v_1) + Ca(u_1;v_2) + a(u_2;v_2)
\end{align*}

mit einem beliebigen $C \in \R$. Zeigen Sie, dass für $|C| < 2$ die Bilinearform $b$
elliptisch ist.
\end{exercise}

% --------------------------------------------------------------------------------

\begin{solution}

\end{solution}

% --------------------------------------------------------------------------------
