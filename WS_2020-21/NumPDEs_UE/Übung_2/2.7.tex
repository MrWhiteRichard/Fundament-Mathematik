% --------------------------------------------------------------------------------

\begin{exercise}

\phantom{}

\begin{enumerate}[label = \textbf{\alph*)}]

  \item Sei $\Omega = (0,1)$ das offene Einheitsintervall.
  Beweisen Sie, dass der Raum $H^1(\Omega)$ kompakt in den Raum $L^2(\Omega)$ eingebettet ist.

  \item Sei $\Omega = (0,1)^2$ das offene Einheitsquadrat.
  Beweisen Sie, dass der Raum $H^1_0(\Omega)$ kompakt in den Raum $L^2(\Omega)$ eingebettet ist.

\end{enumerate}

\end{exercise}

% --------------------------------------------------------------------------------

\begin{solution}

\phantom{}

\includegraphicsboxed{Definition - Sobolevräume.png}
\includegraphicsboxed{Definition - Sobolevräume ++.png}

\includegraphicsboxed
[\cite{Ana3}]
[S2.6.6BA3]
{Satz 2.6.6 - Blümlinger - Analysis 3.png}

Wir zeigen, dass der lineare Einbettungsoperator $T: H^1(\Omega) \to L^2(\Omega)$ Grenzwert von kompakten Operatoren $(T_N)_{N \in \N}$ (mit endlich-dimensionalem Bild), bezüglich der Operatornorm, ist.
Für deren Konstruktion benutzen die ONBs vom $L^2$ von \cite[Satz 2.6.6]{Ana3}.

\begin{enumerate}[label = \textbf{\alph*)}]

  \item Unsere ONB vom $L^2(\Omega)$ lautet wie folgt.
  
  \begin{align*}
    \widetilde{\phi}_n(x) = \exp{(2 \pi i n x)},
    \quad
    n \in \Z,
    \quad
    x \in \Omega
  \end{align*}

  Diese gilt es, zu einer ONB vom $H^1(\Omega)$ zu machen.
  Dazu, werden wir die obere ONB bezüglich $\norm[H^1(\Omega)]{\cdot}$ normieren, d.h.

  \begin{align*}
    \phi_n
    :=
    \widetilde{\phi}_n
    /
    \norm[H^1(\Omega)]{\widetilde{\phi}_n}
    n \in \Z.
  \end{align*}

  \begin{align*}
    \implies
    \norm[H^1(\Omega)]{\widetilde{\phi}_n}^2
    =
    \norm[L^2(\Omega)]{\widetilde{\phi}_n}^2
    +
    \norm[L^2(\Omega)]{\widetilde{\phi}_n^\prime}^2
    1
    +
    (2 \pi i n)^2
    \underbrace
    {
      \Int[0][1]{|\exp{(2 \pi i n x)}|^2}{x}
    }_{
      =
      \norm[L^2(\Omega)]{\widetilde{\phi}_n}^2
      =
      1
    }
  \end{align*}

  Die Orthogonalitätseigenschaft von $(\phi_n)_{n \in \Z}$ bzgl. $\norm[H^1(\Omega)]{\cdot}$ folgt aus derer von $(\widetilde{\phi}_n)_{n \in \Z}$ bzgl. $\norm[L^2(\Omega)]{\cdot}$.
  $\Forall n, m \in \Z, n \neq m:$

  \begin{align*}
    (\phi_n, \phi_m)_{H^1(\Omega)}
    =
    (\phi_n, \phi_m)_{L^2(\Omega)}
    +
    (\phi_n^\prime, \phi_m^\prime)_{L^2(\Omega)}
    =
    0
  \end{align*}

  Wir betrachten nun also die folgenden beiden Operatoren.

  \begin{align*}
    T:
    \begin{cases}
      H^1(\Omega) \to L^2(\Omega) \\
      f
      \mapsto
      \sum_{n = -\infty}^\infty
      (f; \phi_n)_{H^1(\Omega)} \phi_n
    \end{cases},
    \quad
    T_N:
    \begin{cases}
      H^1(\Omega)
      \to
      \Span
      \Bbraces{\phi_n}_{n = 0}^{\pm N}
      \subseteq
      L^2(\Omega) \\
      f
      \mapsto
      \sum_{n = -\infty}^\infty
      (f; \phi_n)_{H^1(\Omega)} \phi_n
    \end{cases},
    \quad
    N \in \N
  \end{align*}

  Um Operatornorm-Konvergenz zu zeigen, wählen wir $f \in L^2(\Omega)$ beliebig und schätzen (mit der Dreiecksungleichung) wie folgt ab.

  \begin{multline*}
    \implies
    \norm[L^2(\Omega)]{T f - T_n f}
    =
    \norm[L^2(\Omega)]
    {
      \sum_{n = -\infty}^\infty
      (f, \phi_n)_{H^1(\Omega)} \phi_n
      -
      \sum_{n = -N}^N
      (f, \phi_n)_{H^1(\Omega)} \phi_n
    } \\
    \leq
    \sum_{n = -\infty}^\infty
    \norm[L^2(\Omega)]
    {
      (f, \phi_n)_{H^1(\Omega)}
      \phi_n
    }
    +
    \sum_{n = -N}^N
    \norm[L^2(\Omega)]
    {
      (f, \phi_n)_{H^1(\Omega)}
      \phi_n
    }
  \end{multline*}

  Wir schätzen o.B.d.A. nur den ersten Summanden weiter ab.
  Dazu verwenden wir die Cauchy-Schwarz-Bunjakovski und Bessel'sche Ungleichung.

  \includegraphicsboxed
  [\cite{Ana3}]
  {(2.13) - Blümlinger - Analysis 3.png}

  \begin{align*}
    \implies
    \sum_{n = -\infty}^\infty
    |(f, \phi_n)_{H^1(\Omega)}|
    \norm[L^2(\Omega)]{\phi_n}
    \stackrel
    {
      \text{CSB}
    }
    {\leq}
    \underbrace
    {
      \pbraces
      {
        \sum_{n=N+1}^\infty
        |(f, \phi_n)_{H^1(\Omega)}|^2
      }^{1/2}
    }_{
      \stackrel
      {
        \text{BESS}
      }{\leq}
      \norm[H^1(\Omega)]{\phi_n}
    }
    \underbrace
    {
      \pbraces
      {
        \sum_{n=N+1}^\infty
        \norm[L^2(\Omega)]{\phi_n}^2
      }^{1/2}
    }_{
      \xrightarrow{N \to \infty} 0
    }
  \end{align*}

  \item Unsere ONB vom $L^2(\Omega)$ lautet wie folgt.
  
  \begin{align*}
    \widetilde{\phi}_n
    =
    \exp{(2 \pi i n \cdot x)}
    =
    \exp{(2 \pi i (n_1 x_1 + n_2 x_1))},
    \quad
    n \in \Z^2,
    \quad
    x \in \Omega
  \end{align*}

  Der Rest ist analog zu \textbf{a)}.

\end{enumerate}

\end{solution}

% --------------------------------------------------------------------------------
