% --------------------------------------------------------------------------------

\begin{exercise}

Sei $u \in H_0^1(\Omega)$ für ein beschränktes Lipschitz-Gebiet $\Omega \subset \R^2$
die Lösung des Poisson-Problems
\begin{align}
  \forall v \in H_0^1(\Omega): (\nabla u;\nabla v)_{L^2(\Omega)} = (f;v)_{L^2(\Omega)}
\end{align}
und $u_h \in \mathcal{S}_0^1(\mathcal{T})$ die zugehörige diskrete Finite-Elemente Lösung.
Weiter sei $\mathcal{K}$ die Knotenmenge der Triangulierung $\mathcal{T}$,
$\zeta_z \in \mathcal{S}_0^1(\mathcal{T})$ für alle $z \in \mathcal{K}$ die Hutfunktionen,
$\tilde{\Omega}_z := \{T \in \mathcal{T}: z \in T\}$ der Knotenpatch aus Abschnitt 4.2 und
\begin{align}
  \tilde{v}_h := \sum_{z \in \mathcal{K}}\left(\frac{1}{|\tilde{\Omega}_z|}
  \sum_{T \in \tilde{\Omega}_z}\nabla u_h|_T\right)\zeta_z
\end{align}
der gemittelte Gradient von $u_h$.
\begin{enumerate}[label = \textbf{\alph*)}]
  \item Beweisen Sie, dass
  \begin{align}
    \eta_{ZZ} := \|\nabla u_h - \tilde{v}_h\|_{L^2(\Omega)}
  \end{align}
  ein zuverlässiger und effizienter Fehlerschätzer ist, wenn der gemittelte
  Gradient eine bessere Approximation an den echten Gradienten ist, als der
  Gradient der FE-Lösung, das heißt, wenn eine Konstante $\alpha < 1$ existiert,
  sodass
  \begin{align}
    \|\nabla u - \tilde{v}_h\|_{L^2(\Omega)} \leq \alpha\|\nabla u - \nabla u_h\|_{L^2(\Omega)}.
    \label{ineq_grad}
  \end{align}
  \item Implementieren Sie diesen Fehlerschätzer in NGSolve und testen Sie numerisch
  sowohl die Güte des Fehlerschätzers als auch die Ungleichung \eqref{ineq_grad}.
  Konstruieren Sie sich dazu analog zu Beispiel 10 geeignete Referenzlösungen. \\

  \textit{Hinweis:} In NGSolve ist $\tilde{v}_h$ sehr elegant mit den Code-Zeilen
  \begin{itemize}
    \item $\mathrm{fe\_ag = VectorH1(mesh, order = 1)}$
    \item $\mathrm{vtilde = GridFunction(fe\_ag)}$
    \item $\mathrm{flux = grad(gfu)}$
    \item $\mathrm{vtilde.Set(flux)}$
  \end{itemize}
  zu berechnen.
\end{enumerate}
\end{exercise}

% --------------------------------------------------------------------------------

\begin{solution}

\begin{enumerate}[label = \textbf{\alph*)}]
  \item Erinnern wir uns daran, was es heißt zuverlässig und effizient zu sein. Es gilt zu zeigen

  \begin{align*}
    \norm[H^1(\Omega)]{u - u_h} &\leq C_{\text{rel}}~ \eta_{ZZ} \\
    C_{\text{eff}}~ \eta_{ZZ} &\leq \norm[H^1(\Omega)]{u - u_h}
  \end{align*}

  Um die Effizienz zu zeigen können wir \ref{ineq_grad} direkt verwenden:

  \begin{align*}
    \eta_{ZZ}
    &=
    \norm[L^2(\Omega)]{\nabla u_h - \tilde{v}_h}
    \leq
    \norm[L^2(\Omega)]{\nabla u - \nabla u_h} + \norm[L^2(\Omega)]{\nabla u - \tilde{v}_h}
    \stackrel{\ref{ineq_grad}}{\leq}
    (1 + \alpha) \norm[L^2(\Omega)]{\nabla u - \nabla u_h} \\
    &\leq
    (1 + \alpha) \norm[H^1(\Omega)]{u - u_h}
  \end{align*}

  Für die Zuverlässigkeit multiplizieren wir \ref{ineq_grad} mit $-1$ und sehen dadurch

  \begin{align*}
    \norm[L^2(\Omega)]{\nabla u - \nabla u_h} - \norm[L^2(\Omega)]{\nabla u - \tilde{v}_h}
    \geq
    (1 - \alpha) \norm[L^2(\Omega)]{\nabla u - \nabla u_h}
  \end{align*}

  Damit, sowie mit der Poincare-Ungleichung schätzen wir nun ab:
  \includegraphicsboxed{PDEs/PDEs_-_Satz_5-11_(Poincare-Ungleichung)}
  \begin{align*}
    \norm[H^1{\Omega}]{u - u_h}
    &=
    \sqrt{\norm[L^2(\Omega)]{u - u_h}^2 + \norm[L^2(\Omega)]{\nabla u - \nabla u_h}^2}
    \leq
    \underbrace{\sqrt{(1+C_p^2)}}_{:=C} \norm[L^2(\Omega)]{\nabla u - \nabla u_h} \\
    &\leq
    \frac{C}{1-\alpha} \Big(
      \norm[L^2(\Omega)]{\nabla u - \nabla u_h} - \norm[L^2(\Omega)]{\nabla u - \tilde{v}_h}
    \Big) \\
    &\leq
    \frac{C}{1-\alpha} \Big(
      \norm[L^2(\Omega)]{\nabla u - \tilde{v}_h} + \norm[L^2(\Omega)]{\nabla u_h - \tilde{v}_h} - \norm[L^2(\Omega)]{\nabla u - \tilde{v}_h}
      \Big)
    =
    \frac{C}{1-\alpha} \eta_{ZZ}
  \end{align*}

  \item Siehe ipynb-File.
\end{enumerate}

\end{solution}

% --------------------------------------------------------------------------------
