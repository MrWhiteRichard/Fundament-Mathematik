% --------------------------------------------------------------------------------

\begin{exercise}

Beweisen Sie die Aussagen aus Ex. 29 und Ex. 30 des Vorlesungsskriptes.
\includegraphicsunboxed{NumPDEs/NumPDEs - Exercise 29}
\includegraphicsunboxed{NumPDEs/NumPDEs - Exercise 30}

\end{exercise}

% --------------------------------------------------------------------------------

\begin{solution}

\textit{Aufgabe 29.}

Dem Hinweis nach definieren wir

\begin{align*}
  \mathcal{X}_\infty
  :=
  \overline{\bigcup_{\ell=0}^\infty \mathcal{X}_\ell}
  \subset H
\end{align*}

Um die Aussage zu zeigen werden wir nun Proposition $1.7$ benutzen.

\includegraphicsunboxed{NumPDEs/NumPDEs - Proposition 1.7}

Da $\bigcup_{\ell \in \N} \mathcal{X}_\ell$ nach Konstruktion dicht in $\mathcal{X}_\infty$ liegen gilt es also zu zeigen:

\begin{align*}
  \forall v \in \bigcup_{\ell \in \N} \mathcal{X}_\ell: ~
  \lim_{\ell \to \infty} \min_{v_\ell \in \mathcal{X}_\ell}
  \norm[H]{v - v_\ell} = 0
\end{align*}

Sei also $v \in \bigcup_{\ell \in \N} \mathcal{X}_\ell$, dann gilt sicher

\begin{align*}
  \exists k \in \N: v \in \mathcal{X}_k,
\end{align*}

Weil diese Unterräume bezüglich $\subset$ aufsteigend angeordnet sind sogar

\begin{align*}
  \forall \ell \geq k: v \in \mathcal{X}_\ell
\end{align*}

Damit schließen wir, da das Minimum auf einem kleineren Raum stets größer ist

\begin{align*}
  \forall \ell \geq k: \quad
  \min_{v_\ell \in \mathcal{X}_\ell} \norm[H]{v - v_\ell}
  \leq
  \min_{v_k \in \mathcal{X}_k} \norm[H]{v - v_k}
  \leq
  \norm[H]{v-v}
  =
  0
\end{align*}

Da $\mathcal{X}_\infty \supseteq \mathcal{X}_l$ gilt
\begin{align*}
  \forall v \in \mathcal{X}_l: \langle \langle U_\infty; v \rangle \rangle =
  \langle \langle u; v \rangle \rangle
  = \langle \langle \mathbb{G}_l u; v \rangle \rangle
\end{align*}
und es folgt $\mathbb{G}_l(u) = \mathbb{G}_l(U_\infty)$ und es gilt nach Proposition $1.7$:

\begin{align*}
  \lim_{\ell \to \infty} ||U_\infty - \underbrace{\mathbb{G}_\ell U_\infty}_{U_\ell}||_H
  =
  0
\end{align*}

\textit{Aufgabe 30.}

Definieren wir $\mathcal{X}_\infty$ so wie im vorherigen Aufgabenteil. Dann gilt es zu zeigen

\begin{align*}
  \forall u \in H^1_D(\Omega):
  \exists (u_k)_{k \in \N} \in \Big(\bigcup_{\ell \in \N} \mathcal{S}^1_D(\mathcal{T}_\ell)\Big)^\N:
  \quad
  \lim_{k \to \infty} \norm[H^1(\Omega)]{u - u_k} = 0
\end{align*}

Da wir gleichmäßig verfeinern gilt, wenn wir mit $h_0 := \max_{T \in \mathcal{T}_0} \diam T$ bezeichnen

\begin{align*}
  \forall \ell \in \N:~
  h_\ell
  =
  \frac{h_0}{2^\ell}
\end{align*}

Es gilt ebenfalls

\begin{align*}
  \forall \ell \in \N:~
  \sigma(\mathcal{T}_\ell) = \sigma(\mathcal{T}_0)
\end{align*}

Für beliebiges $u \in H^1_D(\Omega) = \overline{C^\infty_D(\overline{\Omega})}^{||\cdot||_{H^1(\Omega)}}$ wollen wir nun wieder Proposition $1.7$ verwenden. Unseren dichten Teilraum für den wir die Approximationseigenschaft zeigen wollen haben wir nun schon bestimmt. Zusätzlich benötigen wir noch Korollar $3.6$ (wobei hier $C^\infty_D(\overline{\Omega}) \subset H^2(\Omega) \cap H^1_D(\Omega)$ mit eingeht).

\includegraphicsunboxed{NumPDEs/NumPDEs - Corollary 3.6}

Sei nun also $v \in C^\infty_D(\overline{\Omega})$, dann erhalten wir

\begin{align*}
  \min_{v_\ell \in \mathcal{S}^1_D(\mathcal{T}_\ell)} \norm[H^1(\Omega)]{v - v_\ell}
  \stackrel{3.6}{\leq}
  C \sigma(\mathcal{T}_\ell)\norm[L^2(\Omega)]{h_\ell D^2v}
  =
  C \sigma(\mathcal{T}_0) \frac{h_0}{2^\ell} \norm[L^2(\Omega)]{D^2v}
  \stackrel{\ell \to \infty}{\longrightarrow}
  0
\end{align*}

Damit gilt nach Proposition $1.7$ nun auch

\begin{align*}
  \lim_{k \to \infty} \|u - \underbrace{\mathbb{G}_ku}_{=:u_k}\|_{H^1(\Omega)} = 0
\end{align*}
\end{solution}

% --------------------------------------------------------------------------------
