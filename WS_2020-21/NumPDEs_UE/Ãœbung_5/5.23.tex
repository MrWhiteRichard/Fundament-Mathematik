% --------------------------------------------------------------------------------

\begin{exercise}

Beweisen Sie die Aussagen aus Ex. 29 und Ex. 30 des Vorlesungsskriptes.
\includegraphicsunboxed{NumPDEs/NumPDEs - Exercise 29}
\includegraphicsunboxed{NumPDEs/NumPDEs - Exercise 30}

\end{exercise}

% --------------------------------------------------------------------------------

\begin{solution}

\textit{Aufgabe 29.}

Dem Hinweis nach definieren wir

\begin{align*}
  \mathcal{X}_\infty
  :=
  \overline{\bigcup_{l=0}^\infty \mathcal{X}_l}
  \subset H
\end{align*}

Um die Aussage zu zeigen werden wir nun Proposition $1.7$ benutzen.

\includegraphicsunboxed{NumPDEs/NumPDEs - Proposition 1.7}

Da $\bigcup_{l \in \N} \mathcal{X}_l$ nach Konstruktion dicht in $\mathcal{X}_\infty$ liegen gilt es also zu zeigen:

\begin{align*}
  \forall v \in \bigcup_{l \in \N} \mathcal{X}_l: ~
  \lim_{l \to \infty} \min_{v_l \in \mathcal{X}_l}
  \norm[H]{v - v_l} = 0
\end{align*}

Sei also $v \in \bigcup_{l \in \N} \mathcal{X}_l$, dann gilt sicher

\begin{align*}
  \exists k \in \N: v \in \mathcal{X}_k,
\end{align*}

Weil diese Unterräume bezüglich $\subset$ aufsteigend Angeordnet sind sogar

\begin{align*}
  \forall l \geq k: v \in \mathcal{X}_l
\end{align*}

Damit schließen wir, da das Minimum auf einem kleineren Raum stets größer ist

\begin{align*}
  \forall l \geq k: \quad
  \min_{v_l \in \mathcal{X}_l} \norm[H]{v - v_l}
  \leq
  \min_{v_k \in \mathcal{X}_k} \norm[H]{v - v_k}
  \leq
  \norm[H]{v-v}
  =
  0
\end{align*}

Also gilt nach Proposition $1.7$:

\begin{align*}
  \lim_{l \to \infty} \norm[H]{U_\infty - \underbrace{\mathbb{G}_l U_\infty}_{U_l}}
  =
  0
\end{align*}

\textit{Aufgabe 30.}
\end{solution}

% --------------------------------------------------------------------------------
