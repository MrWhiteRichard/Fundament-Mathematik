% -------------------------------------------------------------------------------- %

\begin{exercise}

Verwenden Sie die Dörfler Markierung (Equation (4.54)) zusammen mit dem
Fehlerschätzer aus Aufgabe 24, um das Beispiel aus Aufgabe 15.b) adaptiv in
NGSolve zu lösen. Hilfreich können dabei die Python-Codeschnipsel
\begin{itemize}
  \item Integrate(\dots, mesh, VOL, element$\_$wise = True),
  \item autoupdate=True als Argument von einer Gridfunction und eines FE-Raumes,
  \item die Markierung von Elementen eines ngsolve-Gitters zur Verfeinerung über \\
  for el in mesh.Elements(): mesh.SetRefinementFlag(e1,\dots)
  \item und die NGSolve-Verfeinerung eines so markierten Gitters über mesh.Refine()
\end{itemize}
sein. Ein NGSolve Vector kann via .NumPy() zu einem NumPy-Array konvertiert werden.
\end{exercise}

% -------------------------------------------------------------------------------- %

\begin{solution}

\phantom{}

\end{solution}

% -------------------------------------------------------------------------------- %
