% --------------------------------------------------------------------------------

\begin{exercise}

  Sei $\hat{T}$ das Referenzdreieck mit den Eckpunkten $V_1 := (0,0), V_2 := (1,0)$ und $V_3 := (0,1)$ und den Randkanten $E_1 :=\overline{V_1V_2}, E_2 :=\overline{V_2V_3}$ und $E_3 :=\overline{V_3V_1}$.
  Für $p \in \N$ definieren wir die Punkte

  \begin{align}
    z_{jk}
    =
    \pbraces{\frac{j}{p}, \frac{k}{p}} \in \hat{T},
    \quad
    j = 0, \dots, p - k,
    \quad
    k = 0, \dots, p,
  \end{align}

  die dazugehörigen Lagrange Basisfunktionen $L_{jk} \in P_p$ mit $L_{jk}(z_{j^\prime k^\prime}) = \delta_{jj^\prime}\delta_{kk^\prime}$ und für Funktionen $u \in C(\hat{T})$ den Interpolationsoperator $\hat I_p$ durch

  \begin{align}
    \hat I_pu
    :=
    \sum_{k=0}^p
    \sum_{j=0}^{p-k}
    u(z_{jk})L_{jk}.
  \end{align}

  \begin{enumerate}[label = \textbf{\alph*)}]

    \item Skizzieren Sie für $p = 1, \cdot, 4$ die Punkte $z_{jk}$ und beweisen Sie, dass $\hat I_p u|_{V_j}$ und $\hat I_p u|_{E_j}$ für $j = 1, 2, 3$ und beliebige $p \in \N$ nur von den Werten $u(V_j)$ bzw. $u|_{E_j}$ abhängt.

    \item Konstruieren Sie aus $\hat I_p$ einen stetigen Interpolationsoperator

    \begin{align*}
      I_{h, p}:
      H^2(\Omega) \to \S_0^p(\mathcal{T})
      :=
      \Bbraces
      {
        v_h \in C(\Omega):
        \Forall T \in \mathcal(\T) v_h|_T \in P_p
      }.
    \end{align*}

    Verwenden Sie dazu die Transformationen aus Lemma 3.9 des Vorlesungsskriptes.

    \item Formulieren und beweisen Sie Theorem 3.5 des Vorlesungsskriptes für $I_{h,p}$.

  \end{enumerate}

  \end{exercise}

  % --------------------------------------------------------------------------------

  \begin{solution}

  \begin{align*}
    Z_p
    :=
    \Bbraces{z_{jk}: j = 0, \dots, p - k, k = 0, \dots, p},
    \quad
    L_{z_{jk}}
    :=
    L_{jk},
    \quad
    j = 0, \dots, p - k, k = 0, \dots, p
  \end{align*}

  \begin{enumerate}[label = \textbf{\alph*)}]

    \item

    \begin{enumerate}[label = (\roman*)]

      \item Für die \Quote{Skizze} siehe 4.16.ipynb.

      \item Seien $u \in H^{p+1}(\hat T)$ und $j \in \Bbraces{1, 2, 3}$.
      Dann ist $V_j \in Z_p$.

      \begin{align*}
        u \in H^{p+1}(\hat T),
        \quad
        j \in \Bbraces{1, 2, 3}
        \implies
        V_j \in Z_p
      \end{align*}

      $\hat I_p u|_{V_j}$ kann nur an dem Punkt $V_j$ ausgewertet werden.

      \begin{align*}
        \implies
        \hat I_p u|_{V_j}(V_j)
        =
        \sum_{z \in Z_p}
        u(z) L_z(V_j)
        =
        u(V_j)
      \end{align*}

      \item Folgende (eindimensionale) Polynome sind vom Grad $\leq p$.

      \begin{align*}
        \hat I_p(u)|_{E_j} \circ \gamma_j \in \Pi_p,
        \quad
        j = 1, 2, 3,
        \quad
        \gamma_1: \xi \mapsto (\xi, 0),
        \quad
        \gamma_2: \xi \mapsto (\xi, 1 - \xi),
        \quad
        \gamma_3: \xi \mapsto (0, \xi)
      \end{align*}

      Sie sind durch die $p + 1$ Punkte aus $\gamma_j(Z_p \cap E_j)$, $j = 1, 2, 3$ eindeutig bestimmet.

    \end{enumerate}

    \item Wir konstruieren den Operator durch Fallunterscheidung, d.h. auf jedem Element $T \in \mathcal{T}$ separat.

    \begin{align*}
      I_{h, p} u|_T := \hat I_p(u \circ \Phi_T) \circ \Phi_T^{-1},
      \quad
      T \in \mathcal{T}
    \end{align*}

    Dabei bezeichne $\Phi_T: \hat T \to T$, $T \in \mathcal{T}$ die Transformation aus Lemma 3.9, und $B_T$ die zugehörige Matrix.

    \includegraphicsunboxed{NumPDEs/NumPDEs - Lemma 3.9.png}

    $I_{h, p}: H^{p+1}(\Omega) \to \S_0^p(\mathcal{T})$ ist wegen \textbf{a)} wohldefiniert.
    Für die Stetigkeit, versehen wir den endlich-dimensionalen $\S_0^p(\mathcal{T})$ mit $\norm[H^2(\Omega)]{\cdot}$.

    Laut Theorem 3.4 (Sobolev), gibt es eine stetige Einbettung $H^{p+1}(\Omega) \to C(\overline{\Omega})$.

    \includegraphicsunboxed{NumPDEs/NumPDEs - Theorem 3.4 (Sobolev).png}

    Wir bezeichnen die Stetigkeits-Konstante auf dem Element $T \in \mathcal{T}$ mit $C_\mathrm{Sobolev}(T)$ und $C_\mathrm{Sobolev} := \max_{T \in \mathcal{T}} C_\mathrm{Sobolev}(T)$.
    Da im endlich-dimensionalen alle Normen äquivalent sind, erhalten wir somit

    \begin{align*}
      \implies
      \norm[L^2(\Omega)]{I_{h, p} u}^2
      & =
      \sum_{T \in \mathcal{T}}
      \norm[L^2(T)]{I_{h, p} u|_T}^2
      =
      \sum_{T \in \mathcal{T}}
      \Int[T]{\abs{\hat{I_p}(u \circ \Phi_T) \circ \Phi_T^{-1}}^2}{x} \\
      & \stackrel
      {
        \mathrm{TRAFO}
      }{=}
      \sum_{T \in \mathcal{T}}
      \Int[\Phi_T^{-1}(T)]{\abs{\hat{I_p}(u \circ \Phi_T)}^2 \abs{\det B_T^{-1}}^{-1}}{x} \\
      & =
      \sum_{T \in \mathcal{T}}
      \Int[\hat T]{\abs{\hat{I_p}(u \circ \Phi_T)}^2 \abs{\det B_T}}{x} \\
      & \stackrel
      {
        \mathrm{3.9}
      }{=}
      \sum_{T \in \mathcal{T}}
      2 |T| \norm[L^2(\hat T)]{\hat{I_p}(u \circ \Phi_T)}^2 \\
      & \leq
      \sum_{T \in \mathcal{T}}
      2 |T| C_\mathrm{norm}^2
      \pbraces
      {
        \max_{z \in Z_p}
        \abs{\hat{I_p}(u \circ \Phi_T)(z)}
      }^2 \\
      & \leq
      C_\mathrm{norm}^2
      \sum_{T \in \mathcal{T}}
      2 |T| \norm[C(T)]{u}^2
      \leq
      2 |\Omega| C_\mathrm{norm}^2 C_\mathrm{sobolev}^2
      \sum_{T \in \mathcal{T}}
      \norm[H^{p+1}(T)]{u}^2 \\
      & \leq
      2 |\Omega| C_\mathrm{norm}^2 C_\mathrm{sobolev}^2 \norm[H^{p+1}(\Omega)]{u}^2.
    \end{align*}

    \item \phantom{}

    \includegraphicsunboxed{NumPDEs/NumPDEs - Theorem 3.5 (Approximation Theorem).png}

    \begin{enumerate}[label = \arabic*.]

      \item Schritt (Abschätzung auf dem Referenz-Element $\hat T$):

      Wir betrachten den Operator

      \begin{align*}
        A := 1 - \hat I_p:
        H^{p+1}(\hat T) \to H^k(\hat T),
        \quad
        k = 0, 1, \dots, p
      \end{align*}

      und bemerken, dass $P_p(\hat T) \subseteq \ker A$.
      Jedes $v \in P_p(\hat T)$ wird nämlich von $\hat I_p$ exakt interpoliert.

      \begin{align*}
        \Forall v \in P_p(\hat T):
        A v = v - \hat I_p v = v - v = 0
      \end{align*}

      Um zu sehen, dass $A$ stetig ist, schätzen wir ab.

      \begin{align*}
        \norm[H^k(\hat T)]{A v}
        =
        \norm[H^k(\hat T)]{v - \hat I_p v}
        \leq
        \norm[H^k(\hat T)]{v}
        +
        \norm[H^k(\hat T)]{\hat I_p v}
      \end{align*}

      Wir nutzen Theorem 3.4 (Sobolev), mit $C_\mathrm{Sobolev} := C_\mathrm{Sobolev}(\hat T)$, um zu sehen, dass

      \begin{multline*}
        \norm[H^k(\hat T)]{\hat I_p v}
        =
        \norm[H^k(\hat T)]
        {
          \sum_{z \in Z_p}
          v(z) L_z
        }
        \leq
        \sum_{z \in Z_p}
        \abs{v(z)} \norm[H^k(\hat T)]{L_z} \\
        \leq
        \underbrace
        {
          \sum_{z \in Z_p}
          \norm[H^k(\hat T)]{L_z}
        }_{
          =: C_\mathrm{Norm}
        }
        \norm[C(\hat T)]{v}
        \stackrel
        {
          \mathrm{S}
        }{\leq}
        C_\mathrm{Norm}
        C_\mathrm{Sobolev}
        \norm[H^p(\hat T)]{v}.
      \end{multline*}

      Zusammengefasst, bekommen wir

      \begin{align*}
        \norm[H^k(\hat T)]{A v}
        \leq
        \norm[H^p(\hat T)]{v}
        +
        C_\mathrm{Norm}
        C_\mathrm{Sobolev}
        \norm[H^p(\hat T)]{v}
        =
        (1 + C_\mathrm{Norm} C_\mathrm{Sobolev})
        \norm[H^p(\hat T)]{v},
      \end{align*}

      also die Stetigkeit des Operators $A$.

      \includegraphicsunboxed{NumPDEs/NumPDEs - Lemma 3.7 (Bramble-Hilbert).png}

      Daher liefert Lemma 3.7 (Bramble-Hilbert) eine Konstante $C_\mathrm{ref} > 0$, die nur von $\hat T$ abhängt mit

      \begin{align*}
        \Forall v \in H^{p+1}(\hat T):
        \norm[H^k(\hat T)]{v - \hat I_p v}
        =
        \norm[H^k(\hat T)]{A v}
        \leq
        C_\mathrm{ref}
        \norm[L^2(\hat T)]{D^{p+1} v},
        \quad
        k = 0, 1, \dots, p.
      \end{align*}

      \item Schritt (Skalierungs-Argument für Abschätzung auf jedem Element $T \in \mathcal{T}$):

      Sei $\Phi := \Phi_T$ der affine Diffeomorphismus aus Lemma 3.9.
      Setze $v := u \circ \Phi$ und observiere, dass

      \begin{align*}
        & \implies
        \hat I_p v
        =
        \hat I_p (u \circ \Phi)
        =
        \hat I_p (u \circ \Phi) \circ \Phi^{-1} \circ \Phi
        =
        I_{h, p} u \circ \Phi \\
        & \implies
        (u - I_{h, p} u) \circ \Phi
        =
        u \circ \Phi - I_{h, p} u \circ \Phi
        =
        v - \hat I_p v
        =
        (1 - \hat I_p) v
        =
        A v.
      \end{align*}

      Zuerst, wenden wir Lemma 3.8 (Transformation Formula) auf $\Phi^{-1}$ an.

      \includegraphicsunboxed{NumPDEs/NumPDEs - Lemma 3.8 (Transformation Formula).png}

      \begin{align*}
        \implies
        \norm[L^2(T)]{D^k (u - I_{h, p} u)}
        & =
        \norm[L^2(T)]{D^k ((v - \hat I_p v) \circ \Phi^{-1})} \\
        & \stackrel
        {
          \mathrm{3.8}
        }{\leq}
        \abs{\det B^{-1}}^{-1/2} \norm[F]{B^{-1}}^k \norm[L^2(\hat T)]{D^k (v - \hat I_p v)} \\
        & \leq
        \abs{\det B^{-1}}^{-1/2} \norm[F]{B^{-1}}^k \norm[H^k(\hat T)]{v - \hat I_p v} \\
        & \stackrel
        {
          \mathrm{1.}
        }{\leq}
        C_\mathrm{ref} \abs{\det B}^{1/2} \norm[F]{B^{-1}}^k \norm[L^2(\hat T)]{D^{p+1} v},
        \quad
        k = 0, 1, \cdots, p.
      \end{align*}

      Also nächstes, setzen wir $v = u \circ \Phi$ ein und wenden Lemma 3.8 (Transformation Formula) auf $\Phi$ an.

      \begin{align*}
        \implies
        \norm[L^2(\hat T)]{D^{p+1} v}
        =
        \norm[L^2(\hat T)]{D^{p+1} (u \circ \Phi)}
        \stackrel
        {
          \mathrm{3.8}
        }{\leq}
        \abs{\det B}^{-1/2} \norm[F]{B}^{p+1} \norm[L^2(T)]{D^{p+1} u}
      \end{align*}

      Die Kombination der letzten beiden Abschätzungen zeigt, dass

      \begin{align*}
        \implies
        \norm[L^2(T)]{D^k (u - I_{h, p} u)}
        \leq
        C_\mathrm{ref} \norm[F]{B^{-1}}^k \norm[F]{B}^{p+1} \norm[L^2(T)]{D^{p+1} u}
        \stackrel
        {
          \mathrm{3.9}
        }{\leq}
        C_\mathrm{ref} 2^\frac{k+p+1}{2} h_T^{p+1} \rho_T^{-k} \norm[L^2(T)]{D^{p+1} u},
      \end{align*}

      wobei wir die geometrische Interpretation von $\norm[F]{B}$ und $\norm[F]{B^{-1}}$ verwendet haben.
      Das zeigt, dass

      \begin{align*}
        \norm[L^2(T)]{u - I_{h, p} u}
        \leq
        2^{\frac{p+1}{2}} C_\mathrm{ref} \norm[L^2(T)]{h^{p+1} D^{p+1} u}
        \quad
        \text{und}
        \quad
        \norm[L^2(T)]{\nabla (u - I_{h, p} v)}
        \leq 2^{\frac{p+3}{2}} C_\mathrm{ref} \sigma(\mathcal{T}) \norm[L^2(T)]{h^p D^{p+1} u}
      \end{align*}

      und schließt daher den Beweis.

    \end{enumerate}

  \end{enumerate}

  \end{solution}

  % --------------------------------------------------------------------------------
