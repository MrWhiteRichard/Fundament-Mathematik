% --------------------------------------------------------------------------------

\begin{exercise}

Sei $\mathcal{T}$ eine reguläre Triangulierung des beschränkten Lipschitz-Gebietes $\Omega \subset \R^2$ und $h \in L^\infty(\Omega)$ definiert durch $h|_T = h_T$ für alle $T \in \mathcal{T}$.

\begin{enumerate}[label = \textbf{\alph*)}]
  \item Zu einer Funktion $v \in H^1(\Omega)$ definieren wir die Funktion $v_\mathcal{T} \in \S_{-1}^0(\mathcal{T}) := \Bbraces{v_h \in L^2(\Omega): \forall T \in \mathcal{T}~~ v_h|_T \in P_0}$ durch $v_\mathcal{T}|_T := |T|^{-1} \Int[T]{v}{x}$ für alle $T \in \mathcal{T}$. Zeigen Sie, dass

  \begin{align*}
    \norm[L^2(\Omega)]{v - v_\mathcal{T}}
    \leq
    C \norm[L^2(\Omega)]{h \nabla v}
  \end{align*}

  gilt. Die Konstante $C > 0$ hängt dabei weder von $\Omega$ noch von $v$ oder $\mathcal{T}$ ab.

  \item Zeigen Sie für $\norm[L^\infty(\Omega)]{h} < 1$ und $p \in \N$

  \begin{align}
    \norm[L^2(\Omega)]{h D^2v_h}
    \leq
    C \norm[H^1(\Omega)]{v_h}, \quad
    v_h \in \S_0^p(\mathcal{T}).
  \end{align}

  Die Konstante $C > 0$ hängt dabei nur von $\sigma(\mathcal{T})$ ab.
\end{enumerate}

\end{exercise}

% --------------------------------------------------------------------------------

\begin{solution}

ToDo!

\end{solution}

% --------------------------------------------------------------------------------
