% --------------------------------------------------------------------------------

\begin{exercise}

\begin{enumerate}[label = \textbf{\alph*)}]
  \item Es sei $\Omega = (0,1), t >0, f \in L^2(\Omega)$, und $X := H^1_D(\Omega) \times H^1_D(\Omega)$ mit $H^1_D(\Omega) := \Bbraces{u \in H^1(\Omega) | u(0) = 0}$.
  Das Problem des Timoshenko Balkens lautet: Gesucht ist $(w, \beta) \in X$ sodass für alle $(v, \delta) \in X$

  \begin{align}
    \Int[\Omega]{\beta^\prime \delta^\prime}{x}
    +
    \frac{1}{t^2} \Int[\Omega]{(w^\prime - \beta)(v^\prime - \delta)}{x}
    =
    \Int[\Omega]{fv}{x}.
  \end{align}

  Zeigen Sie, dass das Problem eindeutig lösbar ist. Wie verhält sich die Konstante in Cea's Lemma wenn $t \rightarrow 0$? \textit{Hinweis}: Verwenden Sie wie in Aufgabe $19$ die Young Ungleichung für den gemischten Term sowie die Friedrich Ungleichung.

  \item Sei nun $\Omega \subset \R^2$ und $X := H^1_D(\Omega) \times [H^1_D(\Omega)]^2$. Betrachten Sie die Reissner-Mindlin Platte als zweidimiensionale Erweiterung des Timoshenko Balkens beschrieben durch das Problem:
  Gesucht ist $(w, \beta) \in X$ sodass für alle $(v, \delta) \in X$

  \begin{align}
    \Int[\Omega]{\epsilon(\beta):\epsilon(\delta)}{x}
    +
    \frac{1}{t^2} \Int[\Omega]{(\nabla w - \beta) \cdot (\nabla v - \delta)}{x}
    =
    \Int[\Omega]{fv}{x},
  \end{align}

  wobei $\epsilon(\beta) := 0.5(\nabla \beta + (\nabla \beta)^\top)$ der symmetrische Gradient ist. Untersuchen Sie mit dem zur Verfügung gestellten Jupyter-File das Konvergenzverhalten für lineare Elemente bei verschiedenen Dickenparametern, $t \in \Bbraces{1, 0.1, 0.01, 0.001}$. Was beobachten Sie? Wie ändert sich das Verhalten für quadratisch finite Elemente?
\end{enumerate}

\end{exercise}

% --------------------------------------------------------------------------------

\begin{solution}

ToDo!

\end{solution}

% --------------------------------------------------------------------------------
