% -------------------------------------------------------------------------------- %

\begin{exercise}

\phantom{}

\begin{enumerate}[label = \textbf{\alph*)}]

  \item Beweisen Sie die Aussage aus Ex. 43 des Vorlesungsskriptes.

  \includegraphicsboxed{NumPDEs/NumPDEs - Exercise 43 (Céa's Lemma for Petrov-Galerkin Schemes).png}

  \item Beweisen Sie die Aussage aus Ex. 45 des Vorlesungsskriptes.

  \includegraphicsboxed{NumPDEs/NumPDEs - Exercise 45.png}

\end{enumerate}

\end{exercise}

% -------------------------------------------------------------------------------- %

\begin{solution}

\phantom{}

\begin{enumerate}[label = \textbf{\alph*)}]

  \item According to Proposition 6.4 (iii) the discrete inf-sup condition is equivalent to the discrete non-degeneracy condition provided that $\dim(X_h) = \dim(Y_h)$.
  
  \includegraphicsboxed{NumPDEs/NumPDEs - Proposition 6.4.png}
  
  With Corollary 6.3 we obtain the unique solvability of the discrete problem

  \begin{align*}
    a(x_h, \cdot) = y^\ast \in Y_h^\ast.
  \end{align*}

  \includegraphicsboxed{NumPDEs/NumPDEs - Corollary 6.3.png}

  Now let $x \in X$ be a solution to (6.5), the weak form

  \begin{align*}
    a(x, \cdot) = y^\ast \in Y^\ast.
  \end{align*}

  We note the Galerkin orthogonality

  \begin{align*}
    a(x - x_h, \cdot) = 0 \in Y_h^\ast.
  \end{align*}

  According to the definition of $\alpha_h$ we obtain, with arbitrary $v_h \in X_h$, that

  \begin{align*}
    \norm[X]{x - x_h} \leq \norm[X]{x- v_h} + \norm[X]{v_h - x_h}
    \leq
    \norm[X]{x - v_h} + \frac{1}{\alpha_h} \sup_{y_h \in Y_h \setminus \Bbraces{0}} \frac{a(v_h - x_h, y_h)}{\norm[Y]{y_h}}.
  \end{align*}

  The Galerkin orthogonality yields

  \begin{align*}
    a(v_h - x_h, y_h) = a(v_h - x_h, y_h) + \underbrace{a(x_h - x, y_h)}_0 = a(v_h - x, y_h).
  \end{align*}

  Therefore,

  \begin{multline*}
    \norm[X]{x - x_h} \leq \norm[X]{x - v_h} + \frac{1}{\alpha_h} \sup_{y_h \in Y_h\setminus \Bbraces{0}} \frac{a(v_h-x, y_h)}{\norm[Y]{y_h}} \\
    \leq
    \norm[X]{x- v_h} + \frac{1}{\alpha_h} \sup_{y_h \in Y_h\setminus \Bbraces{0}} \frac{\norm{a}
    \norm[X]{v_h - x} \norm[Y]{y_h}}{\norm[Y]{y_h}}
    =
    \pbraces{1 + \frac{\norm{a}}{\alpha_h}} \norm[X]{x- v_h}.
  \end{multline*}

  Since $v_h \in X_h$ was chosen arbitrarily we finally obtain

  \begin{align*}
    \norm[X]{x - x_h}
    \leq
    \pbraces{1 + \frac{\norm{a}}{\alpha_h}} \inf_{v_h \in X_h} \norm[X]{x - v_h}.
  \end{align*}

  All that is left to show now is that the infimum is in fact attained.
  To that end, choose an infimizing sequence $(v_k)_{k \in \N}$, such that

  \begin{align*}
    \lim_{k \to \infty} \norm[X]{x - v_k}
    =
    \inf_{v_h \in X_h} \norm[X]{x - v_h}.
  \end{align*}

  According to the triangle inequality, there holds
  
  \begin{align*}
    \norm[X]{v_k}
    \leq
    \norm[X]{x} + \norm[X]{x - v_k}.
  \end{align*}

  Therefore the sequence $(v_k)_{k \in \N}$ is a bounded sequence in the finite dimensional space $X_h$.
  The Bolzano-Weierstrass theorem yields the existence of a convergent subsequence $(v_{k_l})_{l \in \N}$ with limit $v_0 \in X_h$.
  By $\norm[X]{\cdot}$-continuity of $\norm[X]{\cdot}$, we conclude

  \begin{align*}
    \inf_{v_h \in X_h} \norm[X]{x - v_h}
    =
    \lim_{l \to \infty} \norm[X]{x - v_{k_l}}
    =
    \norm[X]{x - v_0}.
  \end{align*}

  \item According to Theorem 6.2 the \textbf{LBB condition for the first argument} is equivalent to the existence of an isomorphism

  \begin{align*}
    T \in L(X, Y^\ast),
    \quad
    T(x)(y) := a(x, y),
    \quad
    x \in X,
    \quad
    y \in Y.
  \end{align*}

  \includegraphicsboxed{NumPDEs/NumPDEs - Theorem 6.2.png}

  Consider the, also isomorphic, adjoint operator

  \begin{align*}
    T^\ast \in L(Y^\astast, X^\ast),
    \quad
    T^\ast(y^\astast)(x^\ast) := y^\astast(T x^\ast),
    \quad
    x^\ast \in X,
    \quad
    y^\astast \in Y.
  \end{align*}

  Due to the reflexivity of $Y$, $I_Y \in L(X, X^\astast)$ is an isometric isomorphism.
  Then $T^\prime := T^\ast \circ I_Y \in L(Y, X^\ast)$ fulfills

  \begin{align*}
    T^\prime(y)(x) := T^\ast(I_Y(y))(x) = I_Y(y)(T x) = T(x)(y) = a(x, y).
  \end{align*}

  As a composition of isomorphisms $T^\prime$ is an isomorphism as well.
  According to Theorem 6.2, this is equivalent to the \textbf{LBB condition for the second argument}.
  The converse follows analogously.

  It remains to show, that $\alpha_1 = \alpha_2$.
  $\alpha_1 \leq \alpha_2$ is easily shown, by introducing $\sup$ and then $\inf$ to the inequality

  \begin{align*}
    \alpha_1
    =
    \inf_{x_1 \in X \setminus \Bbraces{0}}
      \sup_{y_1 \in Y \setminus \Bbraces{0}}
        \frac{a(x_1, y_1)}{\norm[X]{x_1} \norm[Y]{y_1}}
    \leq
    \frac{a(x_2, y_2)}{\norm[X]{x_2} \norm[Y]{y_2}},
    \quad
    x_2 \in X \setminus \Bbraces{0},
    \quad
    y_2 \in Y \setminus \Bbraces{0}.
  \end{align*}

  The converse follows analogously.

\end{enumerate}

\end{solution}

% -------------------------------------------------------------------------------- %
