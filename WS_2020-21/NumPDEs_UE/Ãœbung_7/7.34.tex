% --------------------------------------------------------------------------------

\begin{exercise}

\phantom{}

\begin{enumerate}[label = \textbf{\alph*)}]
  \item Überprüfen Sie durch ein Dimensionsargument ob das Stokes Element mit
  nicht-konformen Crouzeix-Raviart und konstanten unstetigen Druck
  $[V_h^{CR}]^2 \times P^0(\mathcal{T})$ stabil sein könnte. \\
  \textit{Hinweis:} Theorem 6.14 und
  $3\#\mathcal{K} - \#(\mathcal{K}\cap \partial\Omega) = 3 + \# E$ mit $\# E$
  Anzahl der Kanten.
  \item Der Operator $\Pi: H^1(\Omega) \to V_h^{CR}$ in den Crouzeix-Raviart Raum
  sei definiert durch die Eigenschaft

  \begin{align}
    \int_E \Pi(u) ds = \int_E u ds \quad \text{für alle Kanten } E.
  \end{align}

  Damit hat der Operator die eindeutige Form

  \begin{align}
    \Pi(u) = \sum_{i=1}^{N_E} \frac{1}{|E_i|}\int_{E_i} u\, ds\, \lambda_i,
  \end{align}

  wobei $N_E$ die Anzahl der Kanten und $\lambda_i$ die baryzentrischen Koordinaten
  an den Kantenmittelpunkten bezeichnen. Beweisen Sie, dass für alle $v \in H^1(\Omega)$ gilt

  \begin{align}
    \|\Pi(v)\|_{V_h^{CR}} \leq c\|v\|_{H^1}.
  \end{align}

    \textit{Hinweis:} Abbildung auf das Referenzelement.

    \item Zeigen Sie für alle $(u,q_h) \in [V_h^{CR}]^2 \times P^0(\mathcal{T})$

    \begin{align}
      b(\tilde{\Pi}(u),q_h) = b(u,q_h),
    \end{align}

    wobei $\tilde{\Pi}(u) := (\Pi(u_x),\Pi(u_y))^\top$ der komponentenweise
    Operator und $b(u,q) := \int_\Omega \Div(u)\, q\, dx$ ist.

    \item Zeigen Sie, dass das Stokes Element $[V_h^{CR}]^2 \times P^0(\mathcal{T})$ stabil ist.

\end{enumerate}

\end{exercise}

% --------------------------------------------------------------------------------

\begin{solution}

\phantom{}


\end{solution}

% --------------------------------------------------------------------------------
