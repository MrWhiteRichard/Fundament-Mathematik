% --------------------------------------------------------------------------------

\begin{exercise}

\phantom{}

\begin{enumerate}[label = \textbf{\alph*)}]
  \item Überprüfen Sie durch ein Dimensionsargument ob das Stokes Element mit
  nicht-konformen Crouzeix-Raviart und konstanten unstetigen Druck
  $[V_h^{CR}]^2 \times P^0(\mathcal{T})$ stabil sein könnte. \\
  \textit{Hinweis:} Theorem 6.14 und
  $3\#\mathcal{K} - \#(\mathcal{K}\cap \partial\Omega) = 3 + \# E$ mit $\# E$
  Anzahl der Kanten.
  \item Der Operator $\Pi: H^1(\Omega) \to V_h^{CR}$ in den Crouzeix-Raviart Raum
  sei definiert durch die Eigenschaft

  \begin{align}
    \int_E \Pi(u) ds = \int_E u ds \quad \text{für alle Kanten } E.
  \end{align}

  Damit hat der Operator die eindeutige Form

  \begin{align}
    \Pi(u) = \sum_{i=1}^{N_E} \frac{1}{|E_i|}\int_{E_i} u\, ds\, \lambda_i,
  \end{align}

  wobei $N_E$ die Anzahl der Kanten und $\lambda_i$ die baryzentrischen Koordinaten
  an den Kantenmittelpunkten bezeichnen. Beweisen Sie, dass für alle $v \in H^1(\Omega)$ gilt

  \begin{align}
    \|\Pi(v)\|_{V_h^{CR}} \leq c\|v\|_{H^1}.
  \end{align}

    \textit{Hinweis:} Abbildung auf das Referenzelement.

    \item Zeigen Sie für alle $(u,q_h) \in [H^1(\Omega)]^2 \times P^0(\mathcal{T})$

    \begin{align}
      b(\tilde{\Pi}(u),q_h) = b(u,q_h),
    \end{align}

    wobei $\tilde{\Pi}(u) := (\Pi(u_x),\Pi(u_y))^\top$ der komponentenweise
    Operator und $b(u,q) := \int_\Omega \Div(u)\, q\, dx$ ist.

    \item Zeigen Sie, dass das Stokes Element $[V_h^{CR}]^2 \times P^0(\mathcal{T})$ stabil ist.

\end{enumerate}

\end{exercise}

% --------------------------------------------------------------------------------

\begin{solution}

\phantom{}

\begin{enumerate}[label = \textbf{\alph*)}]
  \item
  \begin{align*}
    X_h &= [V_h^{CR}]^2, \quad Y_h = P^0(\mathcal{T}) \cap L^2_*(\Omega)\\
    V_h^{CR} &:= \{v \in L^2(\Omega) \mid \forall T \in \mathcal{T}: v|_T \in P^1(\mathcal{T}),
    v \text{ stetig auf Kantenmitten }\} \\
  \end{align*}
  With Theorem 6.14 we obtain
  \begin{align*}
  \dim(Y_h) = \# T - 1 \stackrel{6.14}{=} 2\#(K\cap \Omega) + \#(K \cap \partial\Omega) - 3.
  \end{align*}

  With $3\#(\mathcal{K} \cap \partial\Omega) > 3$ we obtain
  \begin{align*}
    \dim(X_h) &= 2\dim(V_h^{CR}) = 2\# E = 2(3\#\mathcal{K} - \#(\mathcal{K}\cap \partial\Omega) - 3)
    = 2(2\#\mathcal{K} + \#(\mathcal{K} \cap \Omega) - 3) \\
    &\geq 2(2\#(\mathcal{K} \cap \partial\Omega) + \#(\mathcal{K} \cap \Omega) - 3)
    \geq 2\#(\mathcal{K} \cap \Omega) + \#(\mathcal{K} \cap \partial\Omega) - 3 = \dim(Y_h).
  \end{align*}

  \item

  \begin{align*}
    \|\Pi(u)\|_{V_h^{CR}}^2 &= \sum_{T \in \mathcal{T}}\|\Pi(u)\|_{H^1(T)}^2
    = \sum_{T \in \mathcal{T}} \left\|\sum_{E \in \mathcal{E}_T}\frac{1}{|E|}\int_E u ds
     \lambda_E\right\|_{H^1(T)}^2
  \end{align*}

  \begin{align*}
  &\sum_{T \in \mathcal{T}} \left\| D^m\left(\sum_{E \in \mathcal{E}_T}\frac{1}{|E|}\int_E u ds
   \lambda_E\right)\right\|_{L^2(T)}^2 \leq
   \sum_{T \in \mathcal{T}}|\det(B_T^{-1})|^{-1}\|B_T^{-1}\|_F^{2m}
   \left\|D^m\left(\sum_{E \in \mathcal{E}_T}\frac{1}{|E|}\int_E u ds
    \lambda_{i_E}\right)\right\|_{L^2(T_{ref})}^2 \\
    &\leq \sum_{T \in \mathcal{T}}|\det(B_T)|\|B_T^{-1}\|_F^{2m}
    \sum_{E \in \mathcal{E}_T}
    \frac{1}{|E|^2}\|u\|_{L^2(E)}^2|E|
    \left\|D^m\left(\sum_{E \in \mathcal{E}_T}
     \lambda_{i_E}\right)\right\|_{L^2(T_{ref})}^2 \\
     &\leq \sum_{T \in \mathcal{T}}|\det(B_T)|\|B_T^{-1}\|_F^{2m}
     \sum_{E \in \mathcal{E}_T}
     \frac{1}{|E|}\|u\|_{L^2(E)}^2|T_{ref}|^2 \\
     &\leq \frac{1}{4}\sum_{T \in \mathcal{T}}2|T|\left(\frac{2}{\rho_T^2}\right)^m
     \sum_{E \in \mathcal{E}_T}
     \frac{1}{|E|}\frac{|E|}{|T|}\left(1 + 2\frac{h_T}{d}\right)\|u\|_{H^1(T)}^2\\
     &= 3\cdot2^{m - 1}\sum_{T \in \mathcal{T}}\rho_T^{-2m}
     \left(1 + 2\frac{h_T}{d}\right)\|u\|_{H^1(T)}^2\\
  \end{align*}

  \item

  \begin{align*}
    b(\tilde{\Pi}(u) - u, q_h) = \int_\Omega \Div(\tilde{\Pi}(u) - u)q_h dx
    = \sum_{T \in \mathcal{T}}\int_T \Div(\tilde{\Pi}(u) - u)q_h dx
    = \sum_{T \in \mathcal{T}}\int_{\partial T} (\tilde{\Pi}(u) - u)q_h dx = 0.
  \end{align*}
\end{enumerate}


\end{solution}

% --------------------------------------------------------------------------------
