% --------------------------------------------------------------------------------

\begin{exercise}

\phantom{}

\begin{enumerate}[label = \textbf{\alph*)}]
  \item Überprüfen Sie durch ein Dimensionsargument ob das Stokes Element mit
  nicht-konformen Crouzeix-Raviart und konstanten unstetigen Druck
  $[V_h^{CR}]^2 \times P^0(\mathcal{T})$ stabil sein könnte. \\
  \textit{Hinweis:} Theorem 6.14 und
  $3\#\mathcal{K} - \#(\mathcal{K}\cap \partial\Omega) = 3 + \# E$ mit $\# E$
  Anzahl der Kanten.
  \item Der Operator $\Pi: H^1(\Omega) \to V_h^{CR}$ in den Crouzeix-Raviart Raum
  sei definiert durch die Eigenschaft

  \begin{align}
    \int_E \Pi(u) ds = \int_E u ds \quad \text{für alle Kanten } E.
  \end{align}

  Damit hat der Operator die eindeutige Form

  \begin{align}
    \Pi(u) = \sum_{i=1}^{N_E} \frac{1}{|E_i|}\int_{E_i} u\, ds\, \lambda_{E_i},
  \end{align}

  wobei $N_E$ die Anzahl der Kanten und $\lambda_i$ die baryzentrischen Koordinaten
  an den Kantenmittelpunkten bezeichnen. Beweisen Sie, dass für alle $v \in H^1(\Omega)$ gilt

  \begin{align}
    \|\Pi(v)\|_{V_h^{CR}} \leq c\|v\|_{H^1}.
  \end{align}

    \textit{Hinweis:} Abbildung auf das Referenzelement.

    \item Zeigen Sie für alle $(u,q_h) \in [H^1(\Omega)]^2 \times P^0(\mathcal{T})$

    \begin{align}
      b(\tilde{\Pi}(u),q_h) = b(u,q_h),
    \end{align}

    wobei $\tilde{\Pi}(u) := (\Pi(u_x),\Pi(u_y))^\top$ der komponentenweise
    Operator und $b(u,q) := \int_\Omega \Div(u)\, q\, dx$ ist.

    \item Zeigen Sie, dass das Stokes Element $[V_h^{CR}]^2 \times P^0(\mathcal{T})$ stabil ist.

\end{enumerate}

\end{exercise}

% --------------------------------------------------------------------------------

\begin{solution}

\phantom{}

\begin{enumerate}[label = \textbf{\alph*)}]
  \item
  \begin{align*}
    X_h &= [V_h^{CR}]^2, \quad Y_h = P^0(\mathcal{T}) \cap L^2_*(\Omega)\\
    V_h^{CR} &:= \{v \in L^2(\Omega) \mid \forall T \in \mathcal{T}: v|_T \in P^1(\mathcal{T}),
    v \text{ stetig auf Kantenmitten }\} \\
  \end{align*}
  With Theorem 6.14 we obtain
  \begin{align*}
  \dim(Y_h) = \# T - 1 \stackrel{6.14}{=} 2\#(K\cap \Omega) + \#(K \cap \partial\Omega) - 3.
  \end{align*}

  With $3\#(\mathcal{K} \cap \partial\Omega) > 3$ we obtain
  \begin{align*}
    \dim(X_h) &= 2\dim(V_h^{CR}) = 2\# E = 2(3\#\mathcal{K} - \#(\mathcal{K}\cap \partial\Omega) - 3)
    = 2(2\#\mathcal{K} + \#(\mathcal{K} \cap \Omega) - 3) \\
    &\geq 2(2\#(\mathcal{K} \cap \partial\Omega) + \#(\mathcal{K} \cap \Omega) - 3)
    \geq 2\#(\mathcal{K} \cap \Omega) + \#(\mathcal{K} \cap \partial\Omega) - 3 = \dim(Y_h).
  \end{align*}

  \item

  Definiere den Operator $\Pi_{ref}$ auf dem Referenzelement durch

  \begin{align*}
    \Pi_{ref}(u \circ \Phi_T) \stackrel{!}{=} \Pi(u) \circ \Phi_T.
  \end{align*}

  Transformation auf Referenzelement: Definiere $v := u \circ \Phi_T$

  \begin{align*}
    \|D^m(\Pi(u))\|_{L^2(T)}^2 &= \|D^m(\Pi_{ref}(v) \circ \Phi_T^{-1})\|_{L^2(T)}^2
    \leq |\det(B)^{-1}|^{-1}\|B^{-1}\|_F^{2m} \|D^m(\Pi_{ref}(v))\|_{L^2(T_{ref})}^2
  \end{align*}

  Abschätzung auf dem Referenzelement:

  \begin{align*}
    \|\Pi_{ref}(v)\|_{L^2(T_{ref})}^2 &=
    \left\|\sum_{i=1}^{3} \frac{1}{|E_{T_{ref},i}|}\int_{E_{T_{ref},i}} v\, ds\,
    \tilde{\lambda}_{E_{T_{ref},i}} \right\|_{L^2(T_{ref})}^2 \\
    &\leq C \sum_{i=1}^{3}\|v\|_{L^2(E_{T_{ref},i})}^2
  \end{align*}

  Nachbesserung:

  \begin{align*}
    \partial_j (\Pi_{ref}(v)) &= \frac{1}{|T_{ref}|}\int_{T_{ref}} \partial_j (\Pi_{ref}(v)) dx
    = \frac{1}{|T_{ref}|}\int_{\partial T_{ref}} (\Pi_{ref}(v)) \nu_j ds
    = \frac{1}{|T_{ref}|}\sum_{i=1}^3\int_{E_{T_{ref},i}} (\Pi_{ref}(v)) \nu_j ds \\
    &= \frac{1}{|T_{ref}|}\sum_{i=1}^3\int_{E_{T_{ref},i}} v\, ds\, \nu_j
    = \frac{1}{|T_{ref}|}\int_{T_{ref}} \partial_j v\, dx
  \end{align*}

  Damit erhalten wir

  \begin{align*}
    \|\partial_j (\Pi_{ref}(v))\|_{L^2(T_{ref})}^2 &= |T_{ref}| (\partial_j (\Pi_{ref}(v)))^2
    \leq \frac{1}{|T_{ref}|^2} \|\partial_j u\|_{L^2(T_{ref})}^2
  \end{align*}

  und

  \begin{align*}
    \|\nabla(\Pi_{ref}(u))\|_{L^2(T_{ref})}^2 \leq \frac{1}{4} \|\nabla u\|_{L^2(T_{ref})}^2
  \end{align*}

Trace inequality:

\begin{align*}
  \sum_{i=1}^{3}\|v\|_{L^2(E_{T_{ref},i})}^2 \leq \sum_{i=1}^{3}\frac{|E_{T_{ref},i}|}{|T_{ref}|}
  \left(1 + \sqrt{2}\right)\|v\|_{H^1(T_{ref})}^2.
\end{align*}

Rücktransformation:

\begin{align*}
  \|D^m(v)\|_{L^2(T_{ref})}^2 \leq |\det(B)|^{-1}\|B\|_F^{2m}\|D^m(u)\|_{L^2(T)}^2
\end{align*}

Zusammengesetzt:

\begin{align*}
  \|\Pi(u)\|_{L^2(T)}^2 &\leq C|\det(B)^{-1}|^{-1}\|B^{-1}\|_F^{2m} \sum_{i=1}^{3}\|v\|_{L^2(E_{T_{ref},i})}^2  \\
  &\leq C|\det(B)^{-1}|^{-1} \sum_{i=1}^{3}\frac{|E_{T_{ref},i}|}{|T_{ref}|}
  \left(1 + \sqrt{2}\right)\|v\|_{H^1(T_{ref})}^2 \\
  &\leq \tilde{C}|\det(B)|
  \sum_{k=0}^1 |\det(B)|^{-1}\|B\|_F^{2k}\|D^k(u)\|_{L^2(T)}^2 \\
  &\leq \tilde{C}(1  + 2h_T^2)\|u\|_{H^1(T)}^2
\end{align*}
Und nochmal:
\begin{align*}
  \|\nabla(\Pi(u))\|_{L^2(T_{ref})}^2 &\leq \frac{1}{4} |\det(B)^{-1}|^{-1}
  \|B^{-1}\|_F^{2}\|\nabla u\|_{L^2(T_{ref})}^2 \\
  &\leq \frac{1}{4}4\frac{h_T^2}{\rho_T^2} \|\nabla(u)\|_{L^2(T)}^2
  = \sigma(\mathcal{T})^2 \|\nabla(u)\|_{L^2(T)}^2 \leq
  \sigma(\mathcal{T})^2 \|u\|_{H^1(T)}^2
\end{align*}

Aufsummiert:

\begin{align*}
  \|\Pi(u)\|_{V_h^{CR}}^2 = \sum_{T \in \mathcal{T}}\|\Pi(u)\|_{H_1(T)}^2
  \leq \sum_{T \in \mathcal{T}}C'\|u\|_{H_1(T)}^2
  \leq C'\|u\|_{H^1(\Omega)}^2
\end{align*}

  \item

  \begin{align*}
    b(\tilde{\Pi}(u) - u, q_h) = \int_\Omega \Div(\tilde{\Pi}(u) - u)q_h dx
    = \sum_{T \in \mathcal{T}}q_h\int_T \Div(\tilde{\Pi}(u) - u) dx
    = \sum_{T \in \mathcal{T}}q_h\int_{\partial T} (\tilde{\Pi}(u) - u) dx = 0.
  \end{align*}

  \item Mit Lemma 6.11 erhalten wir die diskrete inf-sup-Bedingung:
  \begin{align*}
    \inf_{0 \neq \lambda \in Y_h}\sup_{0 \neq u \in X_h}\frac{b(u,\lambda)}{\|u\|_{X_h}\|\lambda\|_Y}
    \geq \frac{\gamma}{C'} > 0.
  \end{align*}
  Da unsere Bilinearform $a$ auf ganz $X \times X$ koerziv ist, haben wir damit
  alle Voraussezungen für das Céa Lemma für Sattelpunktprobleme
  erfüllt und sind somit stabil.
\end{enumerate}


\end{solution}

% --------------------------------------------------------------------------------
