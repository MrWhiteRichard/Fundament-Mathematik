% --------------------------------------------------------------------------------

\begin{exercise}

Beweisen Sie die Aussage aus Ex. 46 des Vorlesungsskriptes.
\includegraphicsboxed{ex46.png}

\end{exercise}

% --------------------------------------------------------------------------------

\begin{solution}

We first show (ii) $\implies$ (i): \\
In the proof of Theorem 6.6 the ellipticity of $a(\cdot,\cdot)$ on $X_0$ is only used
to provide a unique $x_1 \in X_0$ with $a(x_1,\cdot) = x^* - a(x_2,\cdot) \in X_0^*$ in Step 5.
Replacing the assumption for $a$ in Theorem 6.6 with our assumptions, we first
obtain that $A_1: X_0 \to X_0^*$ is an isomorphism.
With that we can conclude again, that there is an unique $x_1 \in X_0$ with $a(x_1,\cdot) = x^* - a(x_2,\cdot) \in X_0^*$. \\
Now for the converse (i) $\implies$ (ii): \\
The operator $C \in L(X^* \times Y^*, X \times Y)$ which maps each $(x^*,y^*)$
to the respective unique solution $(x,y) \in X \times Y$ is bijective.
With the orthogonal decomposition $X = X_0 \oplus X_0^{\bot}$ and $x = x_1 + x_2$
with $x_1 \in X_0$ and $x_2 \in X_0^{\bot}$ we can equivalently reformulate the
saddle point problem to

\begin{align}
  b(x_2,\cdot) &= y^* \in Y^*, \label{eq1}\\
  a(x_1,\cdot) &= x^* - a(x_2,\cdot) \in X_0^*, \label{eq2}\\
  b(\cdot,y) &= x^* - a(x_1 + x_2,\cdot) \in X^*. \label{eq3}
\end{align}
From equation \eqref{eq2} we see, if we choose $y^* = 0$ that the operator $A \in L(X_0,X_0^*)$ defined by
\begin{align*}
  Ax := a(x,\cdot) \in A_0^*
\end{align*}
is bijective (since with Hahn-Banach every $x^* \in X_0^*$ can be extended to a functional in $X^*$), therefore with the open mapping theorem an isomorphism,
which implies the LBB-condition for $a$. \\
It remains to show, that the operator $B_1 \in L(Y,X^*)$ defined by
\begin{align*}
  B_1y := b(\cdot,y)
\end{align*}
is injective. Suppose $0 \neq \tilde{y} \in \ker(B_1)$. Then for all solutions
$(x,y)$ to the saddle point problem $(x,y+\tilde{y})$ is also a solution, which
contradicts the unique solvability. Therefore Lemma 6.1 yields $\beta > 0$.


\end{solution}

% --------------------------------------------------------------------------------
