% -------------------------------------------------------------------------------- %

\begin{exercise}

Beweisen Sie die Aussage aus Ex. 46 des Vorlesungsskriptes.

\includegraphicsboxed{NumPDEs/NumPDEs - Exercise 46.png}

\end{exercise}

% -------------------------------------------------------------------------------- %

\begin{solution}

\phantom{}

\begin{itemize}
  
  \item \enquote{$\impliedby$}:
  
  Going through the proof of Theorem 6.6, one realizes that ellipticity of $a(\cdot, \cdot)$ on $X_0$ is only used to provide a unique $x_1 \in X_0$ with $a(x_1, \cdot) = x_0^\ast \in X_0^\ast$ in step 5.
  To prove unique existence of $x_1$ it is, however sufficient to assume that the operator $A_1: X_0 \to X_0^\ast$ defined by $A_1 x := a(x, \cdot)$ is an isomorphism.

  \includegraphicsgraphicsboxed[0.75][0]
  {NumPDEs/NumPDEs - Theorem 6.6.1 (Brezzi).png}
  {NumPDEs/NumPDEs - Theorem 6.6.2 (Brezzi).png}

  \begin{comment}

    In the proof of Theorem 6.6 the ellipticity of $a(\cdot, \cdot)$ on $X_0$ is only used to provide a unique $x_1 \in X_0$ with $a(x_1, \cdot) = x^\ast - a(x_2, \cdot) \in X_0^\ast$ in Step 5.
    Replacing the assumption for $a$ in Theorem 6.6 with our assumptions, we first obtain that $A_1: X_0 \to X_0^\ast$ is an isomorphism.
    With that we can conclude again, that there is an unique $x_1 \in X_0$ with $a(x_1, \cdot) = x^\ast - a(x_2, \cdot) \in X_0^\ast$.

  \end{comment}

  \item \enquote{$\implies$}:

  The operator $C \in L(X^\ast \times Y^\ast, X \times Y)$ which maps each $(x^\ast, y^\ast)$ to the respective unique solution $(x, y) \in X \times Y$ is bijective.

  Let $(x, y) \in X \times Y$.
  With the orthogonal decomposition $X = X_0 \oplus X_0^\bot$, we write $x = x_1 + x_2$ with $x_1 \in X_0$ and $x_2 \in X_0^\bot$.
  Note that (6.11) is equivalent to the following three identities:

  \begin{itemize}
    \item $b(x_2, \cdot) = y^\ast \in Y^\ast$,
    \item $a(x_1, \cdot) = x^\ast - a(x_2, \cdot) \in X_0^\ast$,
    \item $b(\cdot, y)   = x^\ast - a(x_1 + x_2, \cdot) \in X^\ast$.
  \end{itemize}

  Consider the operator $B_1$, where

  \begin{align*}
    A_1 \in L(X_0, X_0^\ast),
    \quad
    A_1 x := a(x, \cdot) \in X_0^\ast,
    \quad
    x \in X_0.
  \end{align*}

  Because of the second identity, $A_1$ is bijective.
  With Hahn-Banach, every $x^\ast \in X_0^\ast$ can be extended to a functional in $X^\ast$.
  According the open mapping theorem, $A_1$ is an isomorphism.
  Due to Theorem 6.2, this implies the $X_0$-LBB-condition, i.e. first two bullet points.

  It remains to show, that $\beta > 0$.
  Consider the operator $B_1$, where

  \begin{align*}
    B_1 \in L(Y, X^\ast),
    \quad
    B_1 y := b(\cdot, y),
    \quad
    y \in Y.
  \end{align*}

  Suppose $0 \neq \tilde y \in \ker(B_1)$.
  Then for all solutions $(x, y)$ to the saddle point problem $(x, y + \tilde y)$ is also a solution.
  That would contradict the unique solvability.

  Therefore, $B_1$ is injective.
  According to Lemma 6.1, this is equivalent to $\beta > 0$.

  \includegraphicsboxed{NumPDEs/NumPDEs - Lemma 6.1.png}

\end{itemize}

\end{solution}

% -------------------------------------------------------------------------------- %
