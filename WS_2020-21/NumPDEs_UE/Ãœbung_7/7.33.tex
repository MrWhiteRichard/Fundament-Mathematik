% --------------------------------------------------------------------------------

\begin{exercise}

\phantom{}

\begin{enumerate}[label = \textbf{\alph*)}]
  \item Beweisen Sie die Aussage aus Ex. 49 des Vorlesungsskriptes.
  \includegraphicsboxed{ex49.png}
  \item Beweisen Sie die Aussage aus Ex. 50 des Vorlesungsskriptes
  durch direktes Arbeiten mit der Matrix ohne Verwendung von Brezzi
  (oder darauf aufbauenden Resultaten).
  \includegraphicsboxed{ex50.png}
\end{enumerate}

\end{exercise}

% --------------------------------------------------------------------------------

\begin{solution}

\phantom{}

\begin{enumerate}[label = \textbf{\alph*)}]
  \item Let $\{\phi_1,\dots,\phi_n\}$ be a basis of $X_h$ and
  $\{\psi_1,\dots,\psi_m\}$ be a basis of $Y_h$. Define
  \begin{align*}
    A_{jk} &:= a(\phi_k,\phi_j) \\
    B_{jk} &:= b(\phi_k,\psi_j) \\
    b_j &:= x^*(\phi_j) \\
    c_j &:= y^*(\psi_j)
  \end{align*}
  Since the computation of the discrete solution boils down to
  \begin{align*}
    x^*(\phi_j) &= \sum_{i=1}^n x_ia(\phi_i,\phi_j) + \sum_{i=1}^m y_i b(\phi_j,\psi_i), \quad j = 1,\dots,n \\
    y^*(\psi_j) &= \sum_{i=1}^m y_i b(\phi_i,\psi_j), \quad j = 1,\dots,m
  \end{align*}
  we can reformulate the problem to: Find $x_h = \sum_{i=1}^n x_i \phi_i$
  and $y_h = \sum_{i=1}^m y_i \psi_i$ such that
  \begin{align*}
    \begin{pmatrix}
      A & B^{\top} \\
      B & 0
    \end{pmatrix}
    \begin{pmatrix}
      x \\ y
    \end{pmatrix}
    =
    \begin{pmatrix}
      c \\ d
    \end{pmatrix}.
  \end{align*}
  \item Let $B$ be regular, then the $m$ lower rows of $M$ must be linearly independent,
  which implies $\ran(B) = \R^m$. \\
  Conversely, suppose $M\cdot x = 0$ for $x \neq 0$. This implies $(x_1,\dots,x_n) \in \ker(B)$.
  Since $A$ is positive definite on $\ker$
\end{enumerate}

\end{solution}

% --------------------------------------------------------------------------------
