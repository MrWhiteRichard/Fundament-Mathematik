% --------------------------------------------------------------------------------

\begin{exercise}

Sei $H$ ein Hilbert-Raum mit Skalarprodukt $(\cdot;\cdot)_H$, $a : H \times H \rightarrow \R$
eine stetige, elliptische und symmetrische Bilinearform, $u \in H$ und $V \subset H$ ein abgeschlossener,
nicht-trivialer linearer Unterraum. Zeigen Sie
\begin{itemize}
  \item[\textbf{a)}]
    Es existiert ein Minimierer $u_0 \in V$ von $a(u- \cdot; u- \cdot)$ auf $V$.
  \item[\textbf{b)}]
    Für diesen Minimierer gilt $a(u-u_0,v) = 0$ für alle $v \in V$.
  \item[\textbf{c)}]
    Dieser Minimierer ist eindeutig.
  \item[\textbf{d)}]
    Die Abbildung $P: u \mapsto Pu := u_0$ ist ein Projektor auf $V$ mit
    \begin{align}
      \sup_{v \in H \setminus \Bbraces{0}}
      \frac{a(Pv;Pv)}{a(v;v)} = 1
    \end{align}
\end{itemize}
\textit{Hinweis zu a:} Wählen Sie eine Minimierungsfolge $(v_n)_{n \in \N}$ von
$d := \inf_{v \in V} a(u-v;u-v)$ und zeigen Sie, dass diese Folge eine Cauchy-Folge ist.
Hilfreich kann dabei folgende Identität sein:

\begin{align}
  a(v_n - v_m ; v_n - v_m )
  =
  2a(v_n - u; v_n - u) + 2a(v_m - u; v_m - u) -
  4a\Bigg(\frac{1}{2}(v_n + v_m ) - u;\frac{1}{2}(v_n + v_m )- u\Bigg)
\end{align}
\end{exercise}

% --------------------------------------------------------------------------------

\begin{solution}

\begin{itemize}
  \item[\textbf{a)}]
    Sei $(v_n)_{n\in\N}$ eine Minimierungsfolge aus dem Hinweis.
    \begin{align*}
    M\|v_n - v_m\|_H^2 &\leq
    a(v_n - v_m ; v_n - v_m ) \\
    &=
    2a(v_n - u; v_n - u) + 2a(v_m - u; v_m - u) -
    4a\Bigg(\frac{1}{2}(v_n + v_m ) - u;\frac{1}{2}(v_n + v_m )- u\Bigg) \\
    &\leq \underbrace{2a(v_n - u; v_n - u)}_{\to 2d} + \underbrace{2a(v_m - u; v_m - u)}_{\to 2d} - 4d
    \end{align*}
    Also ist $(v_n)_{n \in \N}$ eine Cauchy-Folge, welche aufgrund der Abgeschlossenheit
    von $V$ gegen einen Grenzwert $u_0 \in V$ konvergiert.
    Die Funktion $f(v) := a(u-v,u-v)$ ist nun als Verknüpfung von stetigen Abbildungen wieder stetig.
    Damit ist $u_0$ tatsächlich ein gesuchter Minimierer.
  \item[\textbf{b)}]
  \begin{align*}
    \forall v \in V: a(u - u_0, u - u_0) &\leq a(u - u_0 + v, u - u_0 + v) \\
    &= a(u - u_0, u - u_0) + a(v,v) + 2a(u - u_0, v) \\
    &\implies 0 \leq a(v,v) + 2a(u - u_0, v)
  \end{align*}
  Angenommen $a(u - u_0, v) \neq 0$. Dann betrachte die quadratische Gleichung
  \begin{align*}
    \lambda^2 a(v,v) + \lambda a(u-u_0,v) = 0
  \end{align*}
  mit den Nullstellen $-\frac{a(u-u_0,v)}{2} \pm \sqrt{\frac{a^2(u-u_0,v)^2}{4}}$.
  Die Gleichung hat zwei unterschiedliche Nullstellen, also existiert ein $\lambda \in \R$, sodass
  \begin{align*}
    a(\lambda v, \lambda v) + a(u-u_0,\lambda v) = \lambda^2 a(v,v) + \lambda a(u-u_0,v) < 0.
  \end{align*}
  Widerspruch!
  \item[\textbf{c)}]
    Sei $u_1$ ein weiterer Minimierer. Da $u_1 - u_0 \in V$ folgt
    \begin{align*}
      M\|u_1-u_0\|_H^2 &\leq a(u_1 - u_0, u_1 - u_0) = a(u_1 - u_0, u_1 - u_0) - a(u - u_0, u_1 - u_0) \\
      &= a(u_1 -u, u_1 - u_0) = 0.
    \end{align*}
  \item[\textbf{d)}] $P$ ist auf $V$ klarerweise die Identität auf $V$, da $a(u-u,u-u) = 0$.
  Seien $u,v \in H$ beliebig.
  \begin{align*}
     \forall w \in V&:~a(u + \lambda v - (P(u) + \lambda P(v)),u + \lambda v - (P(u) + \lambda P(v))) \\
    &= a(u - P(u), u - P(u)) + \lambda^2a(v - P(v), v - P(v)) +
    2\lambda a(u - P(u), v - P(v))\\
    &\leq a(u - w, u - w) + \lambda^2a(v - w/\lambda, v - w/\lambda)
    + 2\lambda a(u - P(u), v - w/\lambda) \\
    &= a(u - w, u - w) + a(\lambda v - w, \lambda v - w)
    + 2 a(u - P(u),\lambda v - w)
  \end{align*}

\end{itemize}

\end{solution}

% --------------------------------------------------------------------------------
