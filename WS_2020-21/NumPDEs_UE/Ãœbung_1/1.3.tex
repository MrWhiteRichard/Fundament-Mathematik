% --------------------------------------------------------------------------------

\begin{exercise}

Sei $H$ ein Hilbert-Raum mit Skalarprodukt $(\cdot; \cdot)_H$, $a: H \times H \to \R$ eine stetige, elliptische und symmetrische Bilinearform, $u \in H$ und $V \subset H$ ein abgeschlossener, nicht-trivialer linearer Unterraum.
Zeigen Sie

\begin{enumerate}[label = \textbf{\alph*)}]

  \item Es existiert ein Minimierer $u_0 \in V$ von $a(u- \cdot; u- \cdot)$ auf $V$.

  \item Für diesen Minimierer gilt $a(u-u_0,v) = 0$ für alle $v \in V$.

  \item Dieser Minimierer ist eindeutig.

  \item Die Abbildung $P: u \mapsto Pu := u_0$ ist ein Projektor auf $V$ mit

  \begin{align}
    \sup_{v \in H \setminus \Bbraces{0}}
    \frac{a(Pv; Pv)}{a(v; v)} = 1
  \end{align}

\end{enumerate}

\textit{Hinweis zu a:}
Wählen Sie eine Minimierungsfolge $(v_n)_{n \in \N}$ von $d := \inf_{v \in V} a(u - v; u - v)$ und zeigen Sie, dass diese Folge eine Cauchy-Folge ist.
Hilfreich kann dabei folgende Identität sein:

\begin{align}
  a(v_n - v_m ; v_n - v_m)
  =
  2 a(v_n - u; v_n - u)
  +
  2 a(v_m - u; v_m - u) -
  4 a\pbraces
  {
    \frac{1}{2}(v_n + v_m) - u;
    \frac{1}{2}(v_n + v_m) - u
  }
\end{align}

\end{exercise}

% --------------------------------------------------------------------------------

\begin{solution}

Zunächst sei bemerkt, dass $V$ selber, aufgrund der Abgeschlossenheit, wieder ein Hilbertraum ist.

\begin{enumerate}[label = \textbf{\alph*)}]

  \item Wähle eine Minimierungsfolge $(v_n)_{n \in \N}$ von $d := \inf_{v \in V} a(u - v; u - v)$.
  Wir zeigen, dass diese Folge eine Cauchy-Folge ist. \\
  (Die Identität aus der Angabe erhält man durch Anwendung der Parallelogrammregel
  auf die Elemente $v_n -u$ und $v_m - u$.)

  \begin{align*}
    M \norm[H]{v_n - v_m}^2
    & \leq
    a(v_n - v_m; v_n - v_m ) \\
    &=
    2 a(v_n - u; v_n - u)
    +
    2 a(v_m - u; v_m - u)
    -
    4 a\pbraces
  {
    \frac{1}{2}(v_n + v_m) - u;
    \frac{1}{2}(v_n + v_m) - u
  } \\
  & \leq
  2 \underbrace
  {
    a(v_n - u; v_n - u)
  }_{
    \xrightarrow{n \to \infty} d
  }
  +
  2 \underbrace{
    a(v_m - u; v_m - u)
  }_{
    \xrightarrow{m \to \infty} d
  }
  - 4d
  \xrightarrow{n, m \to \infty} 0
  \end{align*}

  Weil $V$ ja ein Hilbertraum also vollständig ist, konvergiert $(v_n)_{n \in \N}$ als Cauchy-Folge gegen einen Grenzwert $u_0 \in V$.
  Die Funktion $v \mapsto a(u - v, u - v)$ ist als Verknüpfung von stetigen Abbildungen wieder stetig.
  Damit ist $u_0$ tatsächlich ein gesuchter Minimierer.

  \item Sei $v \in V$.

  \begin{align*}
    & \implies
    a(u - u_0, u - u_0)
    \leq
    a(u - u_0 + v, u - u_0 + v)
    =
    a(u - u_0, u - u_0) + a(v, v) + 2 a(u - u_0, v) \\
    & \implies
    0 \leq a(v, v) + 2 a(u - u_0, v)
  \end{align*}

  Dazu leiten wir einen Widerspruch her.
  Angenommen $a(u - u_0, v) \neq 0$.
  Dann betrachte die folgende quadratische Gleichung.

  \begin{align*}
    \lambda^2 a(v, v) + \lambda a(u - u_0, v) = 0
    \implies
    \lambda_1 = 0
    \implies
    \lambda a(v, v) + a(u - u_0, v) = 0
    \implies
    \lambda_2 \neq 0
  \end{align*}

  Die Gleichung hat zwei unterschiedliche Nullstellen.

  \begin{align*}
    \implies
    \Exists \lambda \in \R:
    a(\lambda v, \lambda v) + a(u - u_0, \lambda v)
    =
    \lambda^2 a(v,v) + \lambda a(u - u_0, v) < 0.
  \end{align*}

  Widerspruch!

  \item Sei $u_1 \in V$ ein weiterer Minimierer.
  Weil $V$ ein linearer Unterraum ist, ist $u_1 - u_0 \in V$.
  Die Aussage aus b) gilt also für $u_0$ und $u_1$.

  \begin{multline*}
    \implies
    M \norm[H]{u_1 - u_0}^2
    \leq
    a(u_1 - u_0, u_1 - u_0) \\
    =
    a(u_1 - u, u_1 - u_0)
    -
    \underbrace{a(u - u_0, u_1 - u_0)}_0
    =
    a(u_1 - u, u_1 - u_0) = 0.
  \end{multline*}

  \item Klarerweise ist $P|_V = \id_V$, weil $a(u - u, u - u) = 0$.
  Also ist $P$ idempotent. \\
  Nun zur Linearität.

  \begin{itemize}

    \item Additivität:

    Seien $u, v \in H$.
    Wegen \textbf{b)} gilt $\Forall w \in V:$

      \begin{align*}
        a(u + v - P(u + v); w) & = 0, \\
        a(u - Pu; w) & = 0, \\
        a(v - Pv; w) & = 0
      \end{align*}

    Das gilt also insbesondere für $w := P(u + v) - (Pu + Pv) \in V$.
    Wir sind fertig, wenn wir zeigen können, dass $w = 0$.
    Weil $a$ elliptisch ist, $\Exists M > 0:$

    \begin{align*}
      0 =
      a(u - Pu; w)
      +
      a(v - Pv; w)
      -
      a(u + v - P(u + v); w)
      =
      a(w; w)
      \geq
      M \norm[H]{w}^2
      \geq 0.
    \end{align*}

    \item Homogenität:

    Sei $\lambda \in \R$.

    \begin{enumerate}[label = \arabic*.]

      \item Fall ($\lambda = 0$):
      
      Weil $a$ elliptisch mit $M$ ist, gilt $\Forall w \in V:$

      \begin{align*}
        a(\lambda u - w; \lambda u - w)
        =
        a(w; w)
        \geq
        M \norm[H]{w}
        \geq
        0 = a(\lambda u - \lambda Pu; \lambda u - \lambda Pu).
      \end{align*}

      \item Fall ($\lambda \neq 0$):

      \begin{multline*}
        \implies
        \Forall w \in \lambda^{-1} V = V:
        a(\lambda u - \lambda Pu;\lambda u - \lambda Pu)
        =
        \lambda^2 a(u - Pu;u - Pu) \\
        \leq
        \lambda^2 a(u - w; u - w)
        =
        a(\lambda u - \lambda w; \lambda u - \lambda w)
      \end{multline*}
    
    \end{enumerate}

\end{itemize}

  Nun zur \enquote{Operatornorm}.

  \begin{itemize}

    \item
    [\enquote{$\geq$}:]

    Weil wir ja bereits wissen, dass $P|_V = \id_V$, müssen wir dazu bloß ein $u \in V$ betrachten.

    \item
    [\enquote{$\leq$}:]

    Sei $u \in H \setminus \Bbraces{0}$.
    Wegen \textbf{b)} und weil $a$ elliptische ist, $\Exists M > 0:$

    \begin{align*}
      0 \leq M \norm[H]{u - Pu}^2
      \leq
      a(u - Pu; u - Pu)
      =
      a(u - Pu; u)
      -
      \underbrace{a(u - Pu; Pu)}_0.
    \end{align*}

    $a$ ist außerdem symmetrisch.

    \begin{multline*}
      a(Pu; Pu)
      =
      a(Pu; Pu)
      +
      \underbrace{a(u - Pu; Pu)}_0
      =
      a(Pu + u - Pu; Pu) \\
      \leq
      \underbrace{a(u; Pu)}_{a(Pu; u)}
      +
      \underbrace{a(u - Pu; u)}_{\geq 0}
      =
      a(Pu + u - Pu; u)
      =
      a(u, u)
    \end{multline*}

  \end{itemize}

\end{enumerate}

\end{solution}

% --------------------------------------------------------------------------------
