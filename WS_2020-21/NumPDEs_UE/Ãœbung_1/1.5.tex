% --------------------------------------------------------------------------------

\begin{exercise}

Programmieren Sie eine Gauss-Quadratur mit $m$ Stützstellen in einem Intervall $(a,b)$.
Verwenden Sie dazu die Berechnung der Stützstellen und Gewichte über das Eigenwertproblem der
Tridiagonalmatrix

\begin{align}
  \begin{pmatrix}
  0 & \beta_1 & & & \\
  \beta_1 & 0 & \ddots & & \\
  & \ddots & \ddots & \beta_{n-2} & \\
  & & \beta_{n-2} & 0 & \beta_{n-1} \\
  & & & \beta_{n-1} & 0
  \end{pmatrix}
  \in \R^{n \times n}, \quad
  \beta_n = \frac{n}{\sqrt{4n^2 -1}}
\end{align}

Für letzeres können Sie einen fertigen Eigenwertlöser verwenden. Testen Sie ihr Programm an
unterschiedlichen Funktionen. Bestimmen Sie dafür den Quadraturfehler numerisch und vergleichen
Sie ihn mit theoretischen Fehlerschranken.ToDo!
\end{exercise}

% --------------------------------------------------------------------------------

\begin{solution}

Theoretische Fehlerschranke für Gauß-Quadraturen mit $n + 1$ Quadraturknoten:
\begin{align*}
  Q(f) - Q_n(f) = \frac{f^{(2n+2)}(\xi)}{(2n+2)!}\int_a^bw(x)\prod_{j=0}^n(x - x_j)^2dx,
  \quad \text{für ein } \xi \in (a,b).
\end{align*}
Für $w(x) \equiv 0$ folgt also
\begin{align*}
  |Q(f) - Q_n(f)| \leq 2\frac{\|f^{(2n+2)}\|_{\infty,[a,b]}}{(2n+2)!}\max_{j=0}^{N-1} (x_{j+1} - x_j)^p
\end{align*}
\end{solution}

% --------------------------------------------------------------------------------
