% --------------------------------------------------------------------------------

\begin{exercise}

Sei $H$ ein Hilbert-Raum mit Skalarprodukt $(\cdot;\cdot)_H$, $a : H \times H \rightarrow \R$
eine stetige, elliptische und symmetrische Bilinearform und $f \in H^*$ ein stetiges,
lineares Funktional. Weiter sei

\begin{align}
  J(v) := \frac{1}{2} a(v;v) - f(v), \quad v \in H.
\end{align}

\begin{itemize}
  \item[\textbf{a)}]
    Zeigen Sie, dass $u \in H$ genau dann eine Lösung von $a(u;\cdot) = f$ in $H$ ist,
    wenn es das Funktional $J$ auf $H$ minimiert.
  \item[\textbf{b)}]
    Sei nun $H_0 \subset H$ ein linearer Teilraum, $g \in H$ und
    $H_g := \Bbraces{v \in H: v - g \in H_0}$. Zeigen Sie, dass $u \in H_g$ genau
    dann eine Lösung von $a(u; \cdot) = f$ in $H_0$ ist, wenn es das Funktional $J$
    auf $H_g$ minimiert.
\end{itemize}
\end{exercise}

% --------------------------------------------------------------------------------

\begin{solution}
\phantom{}
\begin{itemize}
  \item[\textbf{a)}]
  Dass $a$ elliptisch ist bedeutet, es existiert $C > 0$ sodass

  \begin{align*}
    a(u;u) \geq C \norm[H]{u}^2
  \end{align*}

  Sei also $u \in H$ die eindeutig bestimmte Lösung von $a(u;\cdot) = f$ in $H$, also
  \begin{align*}
    \forall v \in H: a(u;v) = f(v).
  \end{align*}
  Dann gilt für $v \in H$ beliebig:

  \begin{align*}
    2(J(v) - J(u)) &=
     a(v;v) - 2f(v) - a(u;u) + 2f(u) \\
     &=
     a(v;v) -2a(u;v) + a(u;u) \\
     &=
     a(v-u;v) - a(u;v) + a(u;u) \\
     &=
     a(v-u;v) - a(v;u) + a(u;u) \\
     &=
     a(v-u;v) + a(u-v;u) \\
     &=
     a(v-u;v-u)
     \geq
     C \norm[H]{v-u}^2
  \end{align*}
  Also ist $u$ ein Minimum von $J$. \\
  Mit der selben Rechnung folgt ebenso, dass Minima von $J$ eindeutig sind.
  Damit ist bereits jedes Minimum von $J$ auch eine Lösung von $a(u;\cdot) = f$ in $H$.

  \item[\textbf{b)}]
  Nach Punkt a) ist das Minimum $u_0$ von $J$ in $H_0$ genau die eindeutige Lösung von
  $a(u;\cdot) = f$ auf $H_0$. Sei $g \in H$ beliebig.
  Definiere $f_g(v_0) := f(v_0)  - a(g;v_0)$ und wähle $u_g \in H_0$, sodass
  \begin{align*}
    \forall v_0 \in H_0: a(u_g;v_0) = f_g(v_0)
  \end{align*}
  Damit folgt für $u := u_g + g$
  \begin{align*}
    a(u;v_0) = a(u_g;v_0) + a(g;v_0) = f_g(v_0) + a(g;v_0) = f(v_0).
  \end{align*}
  Weiters gilt für $v_0 + g \in H_g$ beliebig
  \begin{align*}
    2(J(v_0 + g) - J(u_g + g)) &= a(v_0 + g; v_0 + g) - a(u_g + g; u_g + g) + 2f(u_g - v_0) \\
    &= a(v_0 ; v_0) + a(g;g) + 2a(g; v_0) - a(u_g; u_g) - a(g;g) - 2a(g; u_g) + 2f(u_g - v_0) \\
    &= a(v_0 ; v_0)  - a(u_g; u_g) + 2(f(u_g - v_0) - a(g;u_g - v_0)) \\
    &= a(v_0 ; v_0)  - a(u_g; u_g) + 2f_g(u_g - v_0) \\
    &= a(v_0 ; v_0)  - a(u_g; u_g) + 2a(u_g;u_g - v_0) \\
    &= a(v_0 ; v_0)  + a(u_g; u_g) - 2a(u_g;v_0) \\
    &= a(v_0 - u_g; v_0 - u_g) \geq C\|v_0 - u_g\|_H^2
  \end{align*}
  Also ist $u = u_g + g$ ein Minimum von $J$ auf $H_g$, welches nach obiger Rechnung
  ebenfalls eindeutig ist.
\end{itemize}
\end{solution}

% --------------------------------------------------------------------------------
