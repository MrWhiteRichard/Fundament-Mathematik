% --------------------------------------------------------------------------------

\begin{exercise}

Sei $u \in H_0^1(\Omega)$ die Lösung des Poisson-Problems
\begin{align}
  \forall v \in H_0^1(\Omega): a(u;v) = (f,v)_{L^2(\Omega)}
  \quad \text{mit } a(u,v) := (\nabla u; \nabla v)_{L^2(\Omega)}
\end{align}
und $u_h \in V_h$ die zugehörige diskrete Lösung mit
$V_h := \mathcal{S}_0^1(\mathcal{T})$. Weiter sei $u \in H^2(\Omega)$ für
beliebige Funktionen $f \in L^2(\Omega)$. Zeigen Sie, dass dann eine Konstante
$C > 0$ unabhängig von $u$ und $h$ existiert, sodass
\begin{align}
  \|u - v_h\|_{L^2(\Omega)} \leq Ch^2\|u\|_{H^2(\Omega)}.
\end{align}
\textit{Hinweis:} Verwenden Sie die Lösung $w$ des Problems
$a(v,w) = (u - u_h,v)_{L^2(\Omega)}, v \in H_0^1(\Omega)$, bei gegebenem $u - u_h$.
Nutzen Sie die Galerkin-Orthogonalität und das Approximationstheorem 3.5 für $w$.
\end{exercise}

% --------------------------------------------------------------------------------

\begin{solution}

\phantom{}

\end{solution}

% --------------------------------------------------------------------------------
