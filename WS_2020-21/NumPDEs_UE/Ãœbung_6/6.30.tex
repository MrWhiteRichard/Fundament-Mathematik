% --------------------------------------------------------------------------------

\begin{exercise}

Implementieren Sie einen \textit{Goal Driven} Fehlerschätzer für lineare
Funktionale. Verwenden Sie dazu den ZZ-Fehlerschätzer von Bsp. 24/25 und die
Zusatzmaterialien \textit{goal\_driven\_error\_estimator.pdf} und
\textit{GoalDriven\_estimator.ipynb}. \\
Sei $\Omega := [0,1]^2$. Gesucht ist ein $u \in H_0^1(\Omega)$, sodass für alle
$v \in H_0^1(\Omega)$
\begin{align}
  \int_\Omega \nabla u \nabla v\, \mathrm{d}x = \int_{[0.2,0.3]\times [0.45,0.55]}100\, v\, \mathrm{d}x.
\end{align}
Testen Sie die Funktionale
\begin{itemize}
  \item $b_1(u) = 100\int_{[0.7,0.8]\times[0.45,0.55]}u\, \mathrm{d}x$ \quad
  (Referenzwert $= 0.042556207995730$),
  \item $b_2(u) = 10\int_{\{0.75\}\times[0.45,0.55]}u\, \mathrm{d}s$ \quad
  (Referenzwert $= 0.042349426604237$),
  \item $b_3(u) = u(0.75,0.5)$ \quad
  (Referenzwert $= 0.042557119266960$).
\end{itemize}
Erstellen Sie Konvergenzplots für den Fehler im Zielfunktional mit verschiedenen
Polynomordnungen und vergleichen Sie mit der Verfeinerungsstrategie aus Bsp. 24/25.
Wie unterscheiden sich die generierten Meshes? \\
\textit{Bemerkung:} Falls Probleme beim Meshen der Geometrie auftreten, können Sie
das mesh direkt mit \texttt{mesh = Mesh("{}mesh\_bsp30.vol"{})} laden. \\
\textit{Hinweis:} Mit \texttt{..*ds("{}bcname"{})} kann über einen spezifischen Rand
integriert werden und mit \texttt{b += v(0.75,0.5)} eine Punktkraft als rechte
Seite angegeben werden. Auswertung eines linearen Funktionals via
\texttt{InnerProduct(b.vec, u.vec)}.
\end{exercise}

% --------------------------------------------------------------------------------

\begin{solution}

\phantom{}

\end{solution}

% --------------------------------------------------------------------------------
