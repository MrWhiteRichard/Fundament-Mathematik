% -------------------------------------------------------------------------------- %

\begin{exercise}

Sei $\Omega \subset \R^2$ ein Lipschitz-Gebiet.
Formulieren und beweisen Sie die wesentlichen Aussagen aus der a priori Analysis in Chapter 3, wenn $\mathcal{T}$ keine Triangulierung aus Dreiecken, sondern aus Rechtecken ist.
Der diskrete Raum $\mathcal{Q}^{1, 1}(\mathcal{T})$ sei dabei definiert durch

\begin{align}
  \mathcal{Q}^{1, 1}(\mathcal{T})
  :=
  \Bbraces
  {
    v \in C(\Omega)
    \mid
    \Forall Q \in \mathcal{T}:
    v|_Q \in \mathcal{P}^1 \otimes \mathcal{P}^1
  },
\end{align}

wobei $\mathcal{P}^1$ der Raum der eindimensionalen, affin linearen Polynome ist.
Welche Vor- bzw. Nachteile haben Rechteckselemente im Vergleich zu Dreieckselementen?

\end{exercise}

% -------------------------------------------------------------------------------- %

\begin{solution}

Wir erinnern kurz an die Definition einer Triangulierung, und bemerken, dass sich bloß der $1$-te Punkt, gemäß Angabe auf nicht-degenerierte Quader, d.h. $Q$ mit $|Q| \neq 0$, ändern muss.

\includegraphicsboxed{NumPDEs/NumPDEs - Definition (triangulation, ...).png}

Wir fordern zudem, dass $\mathcal{T}$ regulär ist, so ähnlichen wie bei Triangulierungen aus Dreiecken;
d.h., dass der Schnitt zweier verschiedener Quader entweder leer, ein Knoten, oder eine Kante ist.

In Chapter 3 gibt es folgende wesentlichen Aussagen.

\begin{enumerate}[label = \arabic*]
  \item Theorem 3.5 (Approximation Theorem)
  \includegraphicsboxed{NumPDEs/NumPDEs - Theorem 3.5 (Approximation Theorem).png}
  \item Corollary 3.6 \dots \enquote{Approximation Theorem für $\mathcal{S}_D^1$ und $\mathcal{S}_\ast^1$}
  \item Lemma 3.7 (Bramble-Hilbert)
  \item Lemma 3.8 (Transformation Formula)
  \item Lemma 3.9 \dots \enquote{Referenz-Dreieck-Transformation und deren geometrische Interpretation}
\end{enumerate}

Die $1$-te Aussage für Rechtecke (in einer geeigneten Formulierung), folgt aus dem selben Beweis wie für Dreiecke.
Die $2$-te Aussage für Rechtecke, folgt dann aus dem selben Beweis wie für Dreiecke, aus der $1$-ten für Rechtecke.
Die $3$-te und $4$-te gelten sogar für allgemeinere Lipschitz-Gebiete, also auch für Rechtecke.
Die $5$-te Aussage werden wir nun zeigen.

\begin{tcolorbox}[standard jigsaw, opacityback = 0]

  \textbf{Lemma 3.9 (für Rechtecke).}
  Für $\hat Q = Q_\mathrm{ref} = [0, 1]^2$ das Referenz-Element und $T = [a, b] \times [c, d] \subset \R^2$ ein nicht-degeneriertes Rechteck, definieren wir

  \begin{align*}
    \Phi_Q:
    Q_\mathrm{ref} \to Q,
    \begin{pmatrix}
      x \\ y
    \end{pmatrix}
    \mapsto
    B
    \begin{pmatrix}
      x \\ y
    \end{pmatrix}
    +
    \begin{pmatrix}
      a \\ c
    \end{pmatrix},
    \quad
    ~\text{wobei}~
    B := \diag(b - a, d - c) \in \R^{2 \times 2}.
  \end{align*}

  Dann gilt, dass $\abs{\det B} = |Q|$ und

  \begin{align*}
    \norm[F]{B} \leq \sqrt 2 h_Q
    \quad
    \text{als auch}
    \quad
    \norm[F]{B^{-1}} \leq \sqrt 2 \rho_Q^{-1},
  \end{align*}

  wobei

  \begin{align*}
    h_Q = \max \Bbraces{b - a, d - c},
    \quad
    \rho_Q = \min \Bbraces{b - a, d - c}.
  \end{align*}

\end{tcolorbox}

\textit{Beweis.}

\begin{align*}
  |Q| & = (b - a) (d - c) = |\det B| \\
  \norm[F]{B} & = \sqrt{(b - a)^2 + (d - c)^2} \leq \sqrt{h_Q^2 + h_Q^2} = \sqrt 2 h_Q \\
  \norm[F]{B^{-1}} & = \sqrt{\pbraces{\frac{1}{b - a}}^2 + \pbraces{\frac{1}{d - c}}^2} \leq \sqrt{\frac{1}{\rho_Q^2} + \frac{1}{\rho_Q^2}} = \sqrt 2 \rho_Q^{-1}
\end{align*}

\hfill
$\square$

Offensichtlich ist $\phi_Q$, als $(b - a)$-Skalierung entlang der $x$-Achse und $(d - c)$-Skalierung entlang der $y$-Achse, gefolgt von einer Translation um $(a, c)^\top$, ein Diffeomorphismus.
Für das Theorem 3.5 (Approximation Theorem) wollen wir nun $\zeta_z \in \mathcal{Q}^{1, 1}(\mathcal{T})$ verwenden, die $\zeta_z(z^\prime) = \delta_{z z^\prime}$ erfüllen.
Diese kann man durch Verkettung folgender Polynome mit geeigneten Transformationen $\phi$ aus unserem neuen Lemma 3.9 stückweise konstruieren.

\begin{align*}
  p_{(0, 0)}(x, y) := (1 - x) (1 - y),
  \quad
  p_{(1, 0)}(x, y) := x (1 - y),
  \quad
  p_{(0, 1)}(x, y) := (1 - x) y,
  \quad
  p_{(1, 1)}(x, y) := x y
\end{align*}

\begin{center}

  \small
  
  \begin{tabular}{l|l|l|l|l}
              & Gebiete                        & reguläre Verfeinerung & Adaptivität  & nicht konforme FE \\ \hline
    Dreiecke  & Polygonzug-Rand                & lokal                 & oft möglich  & geht so           \\
    Rechtecke & Polygonzug-Rand achsenparallel & nur global            & kaum möglich & ur schwierig      \\
  \end{tabular}

\end{center}

\end{solution}

% -------------------------------------------------------------------------------- %
