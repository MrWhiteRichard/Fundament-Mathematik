% --------------------------------------------------------------------------------

\begin{exercise}

Sei $\Omega \subset \R^2$ ein Lipschitz-Gebiet. Formulieren und beweisen Sie die
wesentlichen Aussagen aus der a priori Analysis in Chapter 3, wenn $\mathcal{T}$
keine Triangulierung aus Dreiecken, sondern aus Rechtecken ist. Der diskrete
Raum $\mathcal{Q}^{1,1}(\mathcal{T})$ sei dabei definiert durch
\begin{align}
  \mathcal{Q}^{1,1}(\mathcal{T}) := \{c \in C(\Omega) \mid
  \forall Q \in \mathcal{T}: c|_Q = \mathcal{P}^1 \otimes \mathcal{P}^1\},
\end{align}
wobei $\mathcal{P}^1$ der Raum der eindimensionalen, affin linearen Polynome ist.
Welche Vor- bzw. Nachteile haben Rechteckselemente im Vergleich zu Dreieckselementen?
\end{exercise}

% --------------------------------------------------------------------------------

\begin{solution}

\phantom{}

\end{solution}

% --------------------------------------------------------------------------------
