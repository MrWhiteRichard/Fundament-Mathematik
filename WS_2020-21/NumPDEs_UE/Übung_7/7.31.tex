% --------------------------------------------------------------------------------

\begin{exercise}

\phantom{}

\begin{enumerate}[label = \textbf{\alph*)}]
  \item Beweisen Sie die Aussage aus Ex. 43 des Vorlesungsskriptes.
    \includegraphicsboxed{ex43.png}
  \item Beweisen Sie die Aussage aus Ex. 45 des Vorlesungsskriptes.
    \includegraphicsboxed{ex45.png}
\end{enumerate}

\end{exercise}

% --------------------------------------------------------------------------------

\begin{solution}

\phantom{}

\begin{enumerate}[label = \textbf{\alph*)}]
  \item According to Proposition 6.4 the discrete inf-sup condition is equivalent to
  the discrete non-degeneracy condition provided that $\dim(X_h) = \dim(Y_h)$, and
  with Corollary 6.3 we obtain the unique solvability of the discrete problem
  \begin{align*}
    a(x_h,\cdot) = y^* \in Y_h^*
  \end{align*}
  Now let $x \in X$ be a solution to the weak form
  \begin{align*}
    a(x,\cdot) = y^* \in Y^*.
  \end{align*}
  We note the Galerkin orthogonality
  \begin{align*}
    a(x-x_h,\cdot) = 0 \in Y_h^*
  \end{align*}
  and according to the definition of $\alpha_h$ we obtain with $v_h \in X_h$ arbitrary
  \begin{align*}
    \|x-x_h\|_X \leq \|x- v_h\|_X + \|v_h-x_h\|_X
    \leq \|x- v_h\|_X + \frac{1}{\alpha_h}\sup_{y_h \in Y_h\setminus \{0\}}\frac{a(v_h-x_h,y_h)}{\|y_h\|_Y}.
  \end{align*}
  The Galerkin orthogonality yields
  \begin{align*}
    a(v_h-x_h,y_h) = a(v_h-x_h,y_h) + a(x_h - x,y_h) = a(v_h - x,y_h)
  \end{align*}
  and therefore
  \begin{align*}
    \|x-x_h\|_X \leq \|x- v_h\|_X + \frac{1}{\alpha_h}\sup_{y_h \in Y_h\setminus \{0\}}\frac{a(v_h-x,y_h)}{\|y_h\|_Y}
    \leq  \|x- v_h\|_X + \frac{1}{\alpha_h}\sup_{y_h \in Y_h\setminus \{0\}}\frac{\|a\|
    \|v_h-x\|_X\|y_h\|_Y}{\|y_h\|_Y} \\
    = \left(1 + \frac{\|a\|}{\alpha_h}\right)\|x- v_h\|_X.
  \end{align*}
  Since $v_h \in X_h$ was chosen arbitrarily we finally obtain
  \begin{align*}
    \|x-x_h\|_X \leq \left(1 + \frac{\|a\|}{\alpha_h}\right)\inf_{v_h \in X_h}\|x - v_h\|_X.
  \end{align*}
  All that's left to show now is that the infimum is in fact attained.
  To that end, choose an infimizing sequence $(v_k)$, such that
  \begin{align*}
    \lim_{k \to \infty}\|x - v_k\|_X = \inf_{v_h \in X_h}\|x- v_h\|_X.
  \end{align*}
  According to the triangle inequality, there holds $\|v_k\|_X \leq \|x\|_X + \|x - v_k\|_X$.
  Therefore the sequence $(v_k)$ is a bounded sequence in the finite dimensional space $X_h$
  and the Bolzano-Weierstrass theorem yields the existenxe of a convergent subsequence $(v_{k_l})$
  with limit $v_0 \in X_h$. By continuity, we conclude
  \begin{align*}
    \inf_{v_h \in X_h} \|x - v_h\|_X = \lim_{l \to \infty} \|x - v_{k_l}\|_X = \|x - v_0\|_X.
  \end{align*}
  \item The first LBB condition reads
  \begin{align*}
    \alpha_1 := \inf_{x \in X\setminus \{0\}} \sup_{y \in Y \setminus \{0\}}
    \frac{a(x,y)}{\|x\|_X\|y\|_Y} > 0, \\
    \forall y \in Y \setminus \{0\}\, \exists x \in X: a(x,y) \neq 0.
  \end{align*}
  which is according to Theorem 6.2 equivalent to the existence of an isomorphism
  $T \in L(X,Y^*)$ which satisfies

  \begin{align*}
    T(x)(y) := a(x,y) \in Y^*
  \end{align*}
  Consider the adjoint operator $T \in L(Y^{**}, X^*)$ defined by

  \begin{align*}
    T^*(y^{**})(x) := y^{**}(Tx)
  \end{align*}
  Then $T' := T^* \circ I_Y \in L(Y,X^*)$ fulfills
  \begin{align*}
    T'(y)(x) := T^*(I_Y(y))(x) = I_Y(y)(Tx) = Tx(y) = a(x,y)
  \end{align*}
  As a composition of isomorphisms $T'$ is a isomorphism as well, which is again by Theorem 6.2.
  equivalent to the second LBB condition.
  The converse follows analogously.

\end{enumerate}

\end{solution}

% --------------------------------------------------------------------------------
