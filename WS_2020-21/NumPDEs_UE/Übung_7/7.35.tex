% --------------------------------------------------------------------------------

\begin{exercise}

\phantom{}

\begin{enumerate}[label = \textbf{\alph*)}]
  \item Sei $(u,p) \in [H_0^1(\Omega)]^2 \times L^2(\Omega)$ eine Lösung der
  schwachen Stokes Formulierung. Zeigen Sie, dass dann auch $(u, p + c)$ mit
  beliebigen $c \in \R$ eine schwache Lösung ist.
  \item Sei nun $0 < |\Gamma_D| < |\partial\Omega|$ und
  $(u,p) \in [H_{\Gamma_D}^1(\Omega)]^2 \times L^2(\Omega)$. Zeigen Sie, dass
  jetzt $(u, p + c)$ für $c \neq 0$ im Allgemeinen keine weitere Lösung ist.
  Was ist der richtige Raum für den Druck $p$?
  \item Lösen Sie mit den zur Verfügung gestellten Jupyter-Notebooks ein
  Stokes Problem. Testen Sie dazu
  \begin{itemize}
    \item das Crouzeix-Raviart-$P^0(\mathcal{T})$ Stokes Element aus Aufgabe 34.
    \item das Taylor-Hood-type Element $[\mathcal{S}^2(\mathcal{T})]^2 \times P^0(\mathcal{T})$.
  \end{itemize}
  Welche Konvergenzraten erwarten Sie? \\
  \textit{Bemerkung:} Nicht-homogene Dirichletdaten werden zum Beispiel in
  Unit 1.3 auf der NGSolve Dokumentation(ngsolve.org/docu/latest/) behandelt.
\end{enumerate}

\end{exercise}

% --------------------------------------------------------------------------------

\begin{solution}

\phantom{}

\begin{enumerate}[label = \textbf{\alph*)}]

\item

\begin{align*}
  \int_\Omega (p + c)\Div(v) = \int_\Omega p\Div(v) + c\int_\Omega \Div(v) =
  \int_\Omega p\Div(v) + c\int_{\partial\Omega} v \nu ds = \int_\Omega p\Div(v).
\end{align*}

\item 
Wir vermuten die Randbedingung $\frac{\partial u}{\partial \nu} = g$ und stellen
unsere angepasste schwache Formulierung auf:
\begin{align*}
  \forall v \in [H_D^1(\Omega)]^2&: \int_\Omega \nabla u : \nabla v dx - \int_\Omega p \Div(v) dx + \int_{\Gamma_N} (g - p\nu)v ds
  = \int_\Omega fv dx \\
  \forall q \in L^2(\Omega)&:  - \int_\Omega q\Div(u) = 0
\end{align*}

\end{enumerate}


\end{solution}

% --------------------------------------------------------------------------------
