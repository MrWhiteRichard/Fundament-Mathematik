% -------------------------------------------------------------------------------- %

\begin{exercise}

Zeigen Sie, dass ein stetiger und wohldefinierter Spuroperator $\gamma: H^1(\Omega) \rightarrow L^2(\partial \Omega)$ in den folgenden Fällen existiert.

\begin{enumerate}[label = \textbf{\alph*)}]

  \item $\Omega = Q := (0, 1)^2$

  \item Sei $\Omega$ ein beschränktes, stückweise $C^1$-glattes Gebiet.
  Genauer soll $\partial \Omega$ aus $M \in \N$ Stücken $\Gamma_i$ mit $i = 1, \dots, M$ bestehen, sodass invertierbare Abbildungen $s_i \in C^1(Q,\Omega)$ existieren für die gilt

  \begin{enumerate}[label = (\alph*)]

    \item $s_i((0, 1) \times \Bbraces{0}) = \Gamma_i$

    \item $\det s_i^\prime (\tilde{x}) > 0$ für alle $\tilde{x} \in \Omega$ und

    \item es existiert eine Konstante $C > 0$ mit $\sup_{\tilde{x} \in Q}\norm[2]{s_i^\prime (\tilde{x})} < C$ und $\sup_{\tilde{x} \in Q}\norm[2]{(s_i^\prime (\tilde{x}))^{-1}} < C$.

  \end{enumerate}

\end{enumerate}

\end{exercise}

% -------------------------------------------------------------------------------- %

\begin{solution}

Zu zeigen ist allgemein Folgendes.

\begin{align*}
  \ExistsOnlyOne \gamma \in C
  (
    H^1(\Omega);
    L^2(\partial \Omega)
  ):
  \Forall u \in C^1(\overline{\Omega}):
  \gamma u = u|_{\partial \Omega}
\end{align*}

\begin{enumerate}[label = \textbf{\alph*)}]

  \item

  \begin{enumerate}[label = \arabic*.]

    \item Schritt (\enquote{$C^1$-Stetigkeit}):

    Wir betrachten $\gamma$ zunächst nur auf $C^1(\overline{Q})$.
    Dort zeigen wir Stetigkeit.
    $\Forall u \in C^1(\overline{Q}):$

    \begin{multline*}
      \norm[L^2(\partial Q)]{\gamma u}^2
      =
      \norm[L^2(\partial Q)]{u|_{\partial Q}}^2
      =
      \Int[\partial Q]{|u(x, y)|_{\partial Q}|^2}{(x, y)} \\
      =
      \Int[0][1]{|u(0, y)|^2}{y}
      +
      \Int[0][1]{|u(1, y)|^2}{y}
      +
      \Int[0][1]{|u(x, 0)|^2}{x}
      +
      \Int[0][1]{|u(x, 1)|^2}{x}
    \end{multline*}

    Wir betrachten zunächst das erste Integral auf der rechten Seite.
    Wir erinnern wir uns an die vorletzte Vorlesung.

    \includegraphicsboxed
    [NumPDEs Vorlesung 21.10.2020 28:40]
    {2.8 Vorwissen.png}

    Darin wurde gezeigt, dass $\Exists C > 0: \Forall y \in (0, 1):$

    \begin{align*}
      |u(0, y)|^2
      \leq
      C\Int[0][1]
      {
        \pbraces
        {
          |u(x, y)|^2
          +
          \vbraces
          {
            \pderivative[][u]{x}(x, y)
          }^2
        }
      }{x}.
    \end{align*}

    Wir integrieren die Ungleichung bezüglich $y$ und erhalten eine Abschätzung unseres ersten Integrals.

    \begin{multline*}
      \implies
      \Int[0][1]{|u(0, y)|^2}{y}
      \leq
      C
      \Int[0][1]
      {
        \Int[0][1]
        {
          \pbraces
          {
            |u(x, y)|^2
            +
            \vbraces
            {
              \pderivative[][u]{y}(x, y)
            }^2
          }
        }{x}
      }{y} \\
      \leq
      C
      \pbraces
      {
        \Int[Q]{|u|^2}{(x, y)}
        +
        \Int[Q]{\norm[2]{\nabla u}^2}{(x, y)}
      }
      =
      c \norm[H^1(Q)]{u}
    \end{multline*}

    Die anderen $3$ Integrale funktionieren analog.
    Wir erhalten also insgesamt Stetigkeit von $\gamma$ auf $C^1(\overline{Q})$.

    \item Schritt (\enquote{Existenz} bzw. \enquote{Fortestzung}):

    Wir wollen $\gamma$ nun von $C^1(\overline{Q})$ auf seine $\norm[H^1(Q)]{\cdot}$-Vervollständigung $H^1(Q)$ fortsetzen.
    Sei dazu $u \in H^1(Q)$.

    \begin{align*}
      \implies
      \Exists (u_n)_{n \in \N} \in C^1(\overline{Q})^\N:
      \lim_{n \to \infty}
      \norm[H^1(Q)]{u - u_n}
      =
      0
    \end{align*}

    Offenbar ist $(u_n)_{n \in \N}$ dabei eine Cauchy-Folge in $C^1(\overline{Q})$.
    Wegen Schritt 1, ist daher auch $(\gamma u_n)_{n \in \N}$ eine Cauchy-Folge.
    So wie es bei Vervollständigungen meistens gehandhabt wird, wählen wir als $\gamma u$ den
    (in der Tat existenten) Grenzwert in $L^2(\partial \Omega)$ letzterer Cauchy-Folge. \\

    Diese Wahl von $\gamma u$ ist in der Tat wohldefiniert.
    Sei nämlich $(\widetilde{u}_n)_{n \in \N} \in C^1(\overline{Q})^\N$ eine weitere Folge, mit

    \begin{align*}
      \lim_{n \to \infty}
      \norm[H^1(Q)]{u - \widetilde{u}_n}
      =
      0.
    \end{align*}

    Wir nutzen die Dreiecksungleichung.

    \begin{align*}
      \implies
      \norm[L^1(Q)]{\gamma u - \gamma \widetilde{u}_n}
      \leq
      \norm[L^1(Q)]{\gamma u - \gamma u_n}
      +
      \norm[L^1(Q)]{\gamma u_n - \gamma \widetilde{u}_n}
      \xrightarrow[n \to \infty]{!}
      0
    \end{align*}

    Der erste Summand verschwindet laut der Definition von $\gamma u$ und der zweite wegen der Stetigkeit (bzw. Beschränktheit) von $\gamma$ auf $C^1(\overline{Q})$.
    Genauer ...

    \begin{multline*}
      \norm[L^1(Q)]{\gamma u_n - \gamma \widetilde{u}_n}
      =
      \norm[L^1(Q)]{\gamma (u_n - \widetilde{u}_n)}
      \leq
      \norm{\gamma|_{C^1(\overline{Q})}}
      \norm[H^1(Q)]{u_n - \widetilde{u}_n} \\
      \leq
      \norm{\gamma|_{C^1(\overline{Q})}}
      \pbraces
      {
        \norm[H^1(Q)]{u_n - u}
        +
        \norm[H^1(Q)]{u - \widetilde{u}_n}
      }
      \xrightarrow[]{n \to \infty}
      0
    \end{multline*}

    \item Schritt ($H^1$-Stetigkeit):

    \begin{align*}
      \norm[L^2(\partial Q)]{\gamma u}
      =
      \lim_{n \to \infty}
      \norm[L^2(Q)]{u_n|_{\partial Q}}
      \leq
      \lim_{n \to \infty}
      \norm{\gamma|_{C^1(\overline{Q})}}
      \norm[H^1(Q)]{u_n}
      =
      \norm{\gamma|_{C^1(\overline{Q})}}
      \norm[H^1(Q)]{u}
    \end{align*}

  \end{enumerate}

  \item

  \begin{enumerate}[label = \arabic*.]

    \item Schritt (\enquote{$C^1$-Stetigkeit}):

    Das ist der einzige Schritt, den wir noch ausführen.
    Der Rest funktioniert analog zu \textbf{b)}.
    Sei dazu $u \in C^1(\overline{\Omega})$.

    Zunächst aber einige Abschätzungen für die Funktionaldeterminante.
    Sei dazu $i = 1, \dots, M$.
    Die Normen $\norm[\text{max}]{\cdot}$ und $\norm[2]{\cdot}$ sind äquivalent, vermöge der Konstanten $\widetilde{C} > 0$.

    \begin{align}
      \det \partial s_i(x, y)
      &
      \notag
      =
      \det
      \begin{pmatrix*}
        \pderivative[][(s_i)_x(x, y)]{x} & \pderivative[][(s_i)_x(x, y)]{y} \\
        \pderivative[][(s_i)_y(x, y)]{x} & \pderivative[][(s_i)_y(x, y)]{y}
      \end{pmatrix*} \\
      &
      \label{eq:det1}
      =
      \pderivative[][(s_i)_x(x, y)]{x}
      \pderivative[][(s_i)_y(x, y)]{y}
      -
      \pderivative[][(s_i)_y(x, y)]{x}
      \pderivative[][(s_i)_x(x, y)]{y} \\
      &
      \notag
      \leq
      4 \norm[\max]{\partial s_i(x, y)}
      \leq
      4 \widetilde{C} \norm[2]{\partial s_i(x, y)}
      <
      4 \widetilde{C} C
      =:
      C^\prime
    \end{align}

    Analoges gilt für die Inverse.

    \begin{align}
      \label{eq:det2}
      \det (\partial s_i(x, y))^{-1}
      =
      \cdots
      \leq
      4 \widetilde{C} \norm[2]{(\partial s_i(x, y))^{-1}}
      <
      4 \widetilde{C} C
      =
      C^\prime
    \end{align}

    Dies führt zu einer weiteren Abschätzung, diesmal nach unten.

    \begin{align}
      &
      \notag
      \implies
      \det \partial s_i(x, y)
      =
      (\det (\partial s_i(x, y))^{-1})^{-1}
      >
      (C^\prime)^{-1} \\
      &
      \label{eq:det3}
      \implies
      1
      <
      \det (\partial s_i(x, y))
      C^\prime
    \end{align}

    Laut der mehrdimensionalen Kettenregel und Definition der Abbildungsnormen als $\sup$, gilt folgende Abschätzung.

    \begin{multline}
      \label{eq:Kettenregel-Ungleichung}
      \implies
      \norm[2]{\partial (u \circ s_i)(x, y)}^2
      =
      \norm[2]
      {
        (
          (\partial u \circ s_i)(x, y)
        )(
          \partial s_i(x, y)
        )}^2 \\
      \leq
      \norm[2]{(\partial u \circ s_i)(x, y)}^2
      \norm[2]{\partial s_i(x, y)}^2
      <
      \norm[2]{(\partial u \circ s_i)(x, y)}^2
      C^2
    \end{multline}

    Wir werden im Folgenden die Transformationsformel verwenden.

    \includegraphicsboxed
    [\cite{Ana3}]
    {Satz 4.3.1 (Transformationsformel) - Blümlinger - Analysis 3.png}

    \begin{enumerate}[label = \arabic*.]

      \item Teil-Abschätzung:

      \begin{multline}
        \label{eq:part1}
        \norm[L^2(Q)]{u \circ s_i}^2
        =
        \Int[Q]
        {
          |(u \circ s_i)(x, y)|^2
          |
            \underbrace
            {
              \det (\partial s_i(x, y))
              \det (\partial s_i(x, y))^{-1}
            }_1
          |
        }{(x, y)} \\
        \stackrel
        {
          \text{TRAFO}
        }{=}
        \Int[s_i(Q)]
        {
          |u|^2
          |\det (\partial s_i(x, y))^{-1}|
        }{(x, y)}
        \stackrel{\eqref{eq:det2}}{\leq}
        C^\prime
        \Int[s_i(Q)]{|u(x, y)|^2}{(x, y)}
      \end{multline}

      \item Teil-Abschätzung:

      \begin{multline}
        \label{eq:part2}
        \norm[L^2(Q)]{\nabla (u \circ s_i)}^2
        =
        \Int[Q]{\norm[2]{\partial (u \circ s_i)(x, y)}^2}{(x, y)}
        \stackrel
        {
          \eqref{eq:Kettenregel-Ungleichung}
        }{\leq}
        C^2 \Int[Q]{\norm[2]{(\partial u \circ s_i)(x, y)}^2}{(x, y)} \\
        \stackrel
        {
          \eqref{eq:det3}
        }{\leq}
        C^2 C^\prime \Int[Q]{\norm[2]{(\partial u \circ s_i)(x, y)}^2 |\det \partial s_i|}{(x, y)}
        \stackrel
        {
          \text{TRAFO}
        }{=}
        C^2 C^\prime \Int[s_i(Q)]{\norm[2]{\partial u(x, y)}^2}{(x, y)}
      \end{multline}

    \end{enumerate}

    Wir fügen jetzt endlich Alles zusammen!
    Sei dabei $\overline{C} > 0$ die Operatornorm des Trace-Operators aus \textbf{a)}.

    \begin{align*}
      \norm[L^2(\partial \Omega)]{\gamma u}^2
      & =
      \sum_{i=1}^M
      \Int[\Gamma_i]{|u(x, y)|^2}{(x, y)}
      \stackrel
      {
        \text{TRAFO}
      }{=}
      \sum_{i=1}^M
      \Int[0][1]{|(u \circ s_i)(x, 0)|^2 |\det \partial s_i(x, 0)|}{x} \\
      & \stackrel{\eqref{eq:det1}}{\leq}
      C^\prime
      \sum_{i=1}^M
      \norm[L^2(\partial Q)]{u \circ s_i \mid_{\partial Q}}^2 \\
      & \stackrel
      {
        \textbf{a)}
      }{\leq}
      C^\prime
      \overline{C}
      \sum_{i=1}^M
      \norm[H^1(Q)]{u \circ s_i}^2
      =
      C^\prime
      \overline{C}
      \sum_{i=1}^M
      \pbraces
      {
        \norm[L^2(\Omega)]{u \circ s_i}^2
        +
        \norm[L^2(\Omega)]{\nabla (u \circ s_i)}^2
      } \\
      & \stackrel
      {
        \eqref{eq:part1},
        \eqref{eq:part2}
      }{\leq}
      C^\prime
      \overline{C}
      \sum_{i=1}^M
      \pbraces
      {
        C^\prime
        \Int[s_i(Q)]{|u(x, y)|^2}{(x, y)}
        +
        C^2 C^\prime
        \Int[s_i(Q)]{\norm[2]{\partial u(x, y)}^2}{(x, y)}
      } \\
      & \leq
      (
        C^\prime
        \max \Bbraces{1, C}
      )^2
      \overline{C}
      \Bigg (
        \underbrace
        {
          \sum_{i=1}^M
          \Int[s_i(Q)]{|u(x, y)|^2}{(x, y)}
        }_{
          \norm[L^2(\Omega)]{u}^2
        }
        +
        \underbrace
        {
          \sum_{i=1}^M
          \Int[s_i(Q)]{\norm[2]{\partial u(x, y)}^2}{(x, y)}
        }_{
          \norm[L^2(\Omega)]{\nabla u}^2
        }
      \Bigg ) \\
      & =
      (
        C^\prime
        \max \Bbraces{1, C}
      )^2
      \overline{C}
      \norm[H^1(\Omega)]{u}^2
    \end{align*}

  \end{enumerate}

  \end{enumerate}

\end{solution}

% -------------------------------------------------------------------------------- %
