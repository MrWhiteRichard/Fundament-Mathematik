% --------------------------------------------------------------------------------

\begin{exercise}

Zeigen Sie, dass ein stetiger und wohldefinierter Spuroperator $\gamma: H^1(\Omega) \rightarrow L^2(\partial \Omega)$ in den folgenden Fällen existiert.

\begin{enumerate}[label = \textbf{\alph*)}]

  \item $\Omega = Q := (0, 1)^2$

  \item Sei $\Omega$ ein beschränktes, stückweise $C^1$-glattes Gebiet.
  Genauer soll $\partial \Omega$ aus $M \in \N$ Stücken $\Gamma_i$ mit $i = 1, \dots, M$ bestehen, sodass invertierbare Abbildungen $s_i \in C^1(Q,\Omega)$ existieren für die gilt

  \begin{enumerate}[label = (\alph*)]

    \item $s_i((0, 1) \times \Bbraces{0}) = \Gamma_i$

    \item $\det s_i^\prime (\tilde{x}) > 0$ für alle $\tilde{x} \in \Omega$ und

    \item es existiert eine Konstante $C > 0$ mit $\sup_{\tilde{x} \in Q}\norm[2]{s_i^\prime (\tilde{x})} < C$ und $\sup_{\tilde{x} \in Q}\norm[2]{(s_i^\prime (\tilde{x}))^{-1}} < C$.

  \end{enumerate}

\end{enumerate}

\end{exercise}

% --------------------------------------------------------------------------------

\begin{solution}

Zu zeigen ist allgemein Folgendes.

\begin{align*}
  \ExistsOnlyOne \gamma \in C
  (
    H^1(\Omega);
    L^2(\partial \Omega)
  ):
  \Forall u \in C^1(\overline{\Omega}):
  \gamma u = u|_{\partial \Omega}
\end{align*}

\begin{enumerate}[label = \textbf{\alph*)}]

  \item

  \begin{enumerate}[label = \arabic*.]

    \item Schritt (\Quote{$C^1$-Stetigkeit}):

    Wir betrachten $\gamma$ zunächst nur auf $C^1(\overline{\Omega})$.
    Dort zeigen wir Stetigkeit.
    $\Forall u \in C^1(\overline{\Omega}):$

    \begin{multline*}
      \norm[L^2(\partial \Omega)]{\gamma u}^2
      =
      \norm[L^2(\partial \Omega)]{u|_{\partial \Omega}}^2
      =
      \Int[\partial \Omega]{|u|_{\partial \Omega}|^2}{s} \\
      =
      \Int[0][1]{|u(0, t)|^2}{t}
      +
      \Int[0][1]{|u(1, t)|^2}{t}
      +
      \Int[0][1]{|u(t, 0)|^2}{t}
      +
      \Int[0][1]{|u(t, 1)|^2}{t}
    \end{multline*}

    Wir betrachten zunächst das erste Integral auf der rechten Seite.
    Wir erinnern wir uns an die vorletzte Vorlesung.

    \includegraphicsboxed
    [NumPDEs Vorlesung 21.10.2020 28:40]
    {2.8 Vorwissen.png}

    Darin wurde gezeigt, dass $\Exists c > 0: \Forall t \in (0, 1):$

    \begin{align*}
      |u(0, t)|^2
      \leq
      c
      \Int[0][1]
      {
        \pbraces
        {
          |u(s, t)|^2
          +
          \vbraces
          {
            \pderivative[][u]{s}(s, t)
          }^2
        }
      }{s}.
    \end{align*}

    Wir integrieren die Ungleichung bezüglich $t$ und erhalten eine Abschätzung unseres ersten Integrals.

    \begin{multline*}
      \implies
      \Int[0][1]{|v(0, t)|^2}{t}
      \leq
      c
      \Int[0][1]
      {
        \Int[0][1]
        {
          \pbraces
          {
            |u(s, t)|^2
            +
            \vbraces
            {
              \pderivative[][u]{s}(s, t)
            }^2
          }
        }{s}
      }{t} \\
      \leq
      c
      \pbraces
      {
        \Int[\Omega]{|u|^2}{x}
        +
        \Int[\Omega]{|\nabla u|^2}{x}
      }
      =
      c \norm[H^1(\Omega)]{u}
    \end{multline*}

    Die anderen $3$ Integrale funktionieren analog.
    Wir erhalten also insgesamt Stetigkeit von $\gamma$ in $u$.

    \item Schritt (\Quote{Existenz} bzw. \Quote{Fortestzung}):

    Wir wollen $\gamma$ nun von $C^1(\overline{\Omega})$ auf seine $\norm[H^1(\Omega)]{\cdot}$-Vervollständigung $H^1(\Omega)$ fortsetzen.
    Sei dazu $u \in H^1(\Omega)$.

    \begin{align*}
      \implies
      \Exists (u_n)_{n \in \N} \in C^1(\overline{\Omega})^\N:
      \lim_{n \to \infty}
      \norm[H^1(\Omega)]{u - u_n}
      =
      0
    \end{align*}

    Offenbar ist $(u_n)_{n \in \N}$ dabei eine Cauchy-Folge.
    Wegen Schritt 1, ist daher auch $(\gamma u_n)_{n \in \N}$ eine Cauchy-Folge.
    So wie es bei Vervollständigungen meistens gehandhabt wird, wählen wir als $\gamma u$ den (in der Tat existentn) Grenzwert letzterer Cauchy-Folge. \\

    Diese Wahl von $\gamma u$ ist in der Tat wohldefiniert.
    Sei nämlich $(\widetilde{u}_n)_{n \in \N} \in C^1(\overline{\Omega})^\N$ eine weitere Folge, mit

    \begin{align*}
      \lim_{n \to \infty}
      \norm[H^1(\Omega)]{u - \widetilde{u}_n}
      =
      0.
    \end{align*}

    Wir nutzen die Dreiecksungleichung.

    \begin{align*}
      \implies
      \norm[L^1(\Omega)]{\gamma u - \gamma \widetilde{u}_n}
      \leq
      \norm[L^1(\Omega)]{\gamma u - \gamma u_n}
      +
      \norm[L^1(\Omega)]{\gamma u_n - \gamma \widetilde{u}_n}
      \xrightarrow[n \to \infty]{!}
      0
    \end{align*}

    Der erste Summand verschwindet laut der Definition von $\gamma u$;
    der Zweite wegen der Stetigkeit (bzw. Beschränktheit) von $\gamma$ auf $C^1(\overline{\Omega})$.

    \begin{multline*}
      \norm[L^1(\Omega)]{\gamma u_n - \gamma \widetilde{u}_n}
      =
      \norm[L^1(\Omega)]{\gamma (u_n - \widetilde{u}_n)}
      \leq
      \norm{\gamma|_{C^1(\overline{\Omega})}}
      \norm[H^1(\Omega)]{u_n - \widetilde{u}_n} \\
      \leq
      \norm{\gamma|_{C^1(\overline{\Omega})}}
      \pbraces
      {
        \norm[H^1(\Omega)]{u_n - u}
        +
        \norm[H^1(\Omega)]{u - \widetilde{u}_n}
      }
      \xrightarrow[]{n \to \infty}
      0
    \end{multline*}

    \item Schritt ($H^1$-Stetigkeit):
    
    \begin{align*}
      \norm[L^2(\partial \Omega)]{\gamma u}
      =
      \lim_{n \to \infty}
      \norm[L^2(\Omega)]{u_n|_{\partial \Omega}}
      \leq
      \lim_{n \to \infty}
      \norm{\gamma|_{C^1(\overline{\Omega})}}
      \norm[H^1(\Omega)]{u_n}
      =
      \norm{\gamma|_{C^1(\overline{\Omega})}}
      \norm[H^1(\Omega)]{u}
    \end{align*}

  \end{enumerate}

  \item ToDo!

\end{enumerate}

\end{solution}

% --------------------------------------------------------------------------------
