% --------------------------------------------------------------------------------

\begin{exercise}

Sei $H$ ein Hilbertraum und $a: H \times H \rightarrow \R$ eine stetige, ellipitische und symmetrische Bilinearform, d.h. es existieren Konstanten $\alpha, \beta > 0$ sodass

\begin{align*}
  a(u; v)
  \leq
  \beta \norm{u}\norm{v},
  \quad
  a(u; u) \geq \alpha \norm{u}^2,
  \quad
  u, v \in H.
\end{align*}

Wir definieren auf dem Hilbertraum $H \times H$ mit Skalarprodukt

\begin{align*}
  (
    (u_1; u_2);
    (v_1; v_2)
  )_{
    H \times H
  }
  :=
  (u_1; v_1)_H
  +
  (u_2; v_2)_H
\end{align*}

die Bilinearform

\begin{align*}
  b
  (
    (u_1; u_2);
    (v_1; v_2)
  )
  :=
  a(u_1; v_1)
  +
  C a(u_1; v_2)
  +
  a(u_2;v_2)
\end{align*}

mit einem beliebigen $C \in \R$. Zeigen Sie, dass für $|C| < 2$ die Bilinearform $b$ elliptisch ist.

\end{exercise}

% --------------------------------------------------------------------------------

\begin{solution}

Wir betrachten auf $H \times H$ das durch $a$ induzierte Skalarprodukt $(\cdot; \cdot)_\ast$.

\begin{align*}
  (
    (u_1; u_2);
    (v_1; v_2)
  )_\ast
  :=
  a(u_1; v_1) + a(u_2; v_2)
\end{align*}

Dadurch wird eine zu $\norm[H \times H]{\cdot}$ äquivalente Norm $\norm[\ast]{\cdot}$ induziert.
Das folgt aus der Elliptizität bzw. Stetigkeit von $a$.

\begin{align*}
  \norm[\ast]{(u_1; u_2)}^2
  =
  a(u_1; u_1) + a(u_2; u_2)
  \:
  \Bigg \{
  \begin{matrix}
    \geq
    \alpha
    (
      \norm[H]{u_1}^2
      +
      \norm[H]{u_2}^2
    )
    =
    \alpha
    \norm[H \times H]{(u_1; u_2)}^2 \\  
    \leq
    \beta
    (
      \norm[H]{u_1}^2
      +
      \norm[H]{u_2}^2
    )
    =
    \beta
    \norm[H \times H]{(u_1; u_2)}^2
  \end{matrix}
\end{align*}

Jetzt zeigen wir die Elliptizität für $\norm[\ast]{\cdot}$ für $|C| < 2$.
Dann benutzen wir deren Äquivalenz zu $\norm[H \times H]{\cdot}$, um auch Elliptizität für diese zu garantieren.
Dazu verwenden wir die Cauchy-Schwarz-Bunjakowski Ungleichung und eine passende quadratische Ergänzung.

\begin{align*}
  b((u_1; u_2),(u_1; u_2))
  & =
  a(u_1; u_1) + C a(u_1; u_2) + a(u_2; u_2) \\
  & \geq
  a(u_1; u_1) + a(u_2; u_2) - |C| |a(u_1; u_2)| \\
  & \stackrel
  {
    \text{CSB}
  }{\geq}
  a(u_1; u_1)
  +
  a(u_2; u_2)
  -
  2 \frac{|C|}{2}
  \sqrt{a(u_1; u_1)}
  \sqrt{a(u_2; u_2)} \\
  & =
  \frac{|C|}{2}
  \pbraces
  {
    \sqrt{a(u_1; u_1)}
    -
    \sqrt{a(u_2; u_2)}
  }^2
  +
  \pbraces
  {
    1 - \frac{|C|}{2}
  }
  (
    a(u_1; u_1)
    +
    a(u_2; u_2)
  ) \\
  & \geq
  \pbraces
  {
    1 - \frac{|C|}{2}
  }
  (
    a(u_1; u_1)
    +
    a(u_2; u_2)
  )
  =
  \pbraces
  {
    1 - \frac{|C|}{2}
  }
  \norm[\ast]{(u_1; u_2)}^2 \\
  & \geq
  \alpha
  \pbraces
  {
    1 - \frac{|C|}{2}
  }
  \norm[H \times H]{(u_1; u_2)}^2
\end{align*}

\end{solution}

% --------------------------------------------------------------------------------
