% -------------------------------------------------------------------------------- %

\begin{exercise}

\phantom{}

\begin{enumerate}[label = \textbf{\alph*)}]

  \item Sei $\Omega = (0,1)$ das offene Einheitsintervall.
  Beweisen Sie, dass der Raum $H^1(\Omega)$ kompakt in den Raum $L^2(\Omega)$ eingebettet ist.

  \item Sei $\Omega = (0,1)^2$ das offene Einheitsquadrat.
  Beweisen Sie, dass der Raum $H^1_0(\Omega)$ kompakt in den Raum $L^2(\Omega)$ eingebettet ist.

\end{enumerate}

\end{exercise}

% -------------------------------------------------------------------------------- %

\begin{solution}

\phantom{}

\includegraphicsboxed{Definition - Sobolevräume.png}
\includegraphicsboxed{Definition - Sobolevräume ++.png}

\includegraphicsboxed
[\cite{Ana3}]
[S2.6.6BA3]
{Satz 2.6.6 - Blümlinger - Analysis 3.png}

Wir zeigen, dass der lineare Einbettungsoperator $T: H^1(\Omega) \to L^2(\Omega)$ Grenzwert von kompakten Operatoren $(T_N)_{N \in \N}$ (mit endlich-dimensionalem Bild), bezüglich der Operatornorm, ist.
Für deren Konstruktion benutzen die ONBs vom $L^2$ von \cite[Satz 2.6.6]{Ana3}.

\begin{enumerate}[label = \textbf{\alph*)}]

  \item Unsere ONB vom $L^2(\Omega)$ lautet wie folgt.

  \begin{align*}
    \widetilde{\phi}_n(x) = \exp{(2 \pi i n x)},
    \quad
    n \in \Z,
    \quad
    x \in \Omega
  \end{align*}

  Zur Überprüfung der Orthonormalität berechnen wir
  \begin{align*}
    (\widetilde{\phi}_n, \widetilde{\phi}_m)
    &= \int_\Omega \widetilde{\phi}_n(x) \overline{\widetilde{\phi}_m(x)} dx
    = \int_0^1 \exp(2\pi inx) \exp(-2\pi imx) dx
    = \int_0^1 \exp(2\pi ix(n -m)) dx \\
    &= \int_0^1 \cos(2\pi x(n-m)) + i\sin(2\pi x(n-m)) dx \\
    &=  \begin{cases}\left[\frac{\sin(2\pi x(n-m))}{2\pi(n-m)}\right]_{x=0}^1
    - i\left[\frac{\cos(2\pi x(n-m))}{2\pi(n-m)}\right]_{x=0}^1 = 0 &, n \neq m \\
      1 &, n = m
    \end{cases}.
  \end{align*}

  Diese gilt es, zu einer ONB vom $H^1(\Omega)$ zu machen.
  Dazu, werden wir die obere ONB bezüglich $\norm[H^1(\Omega)]{\cdot}$ normieren, d.h.

  \begin{align*}
    \phi_n
    :=
    \widetilde{\phi}_n
    /
    \norm[H^1(\Omega)]{\widetilde{\phi}_n}
    n \in \Z.
  \end{align*}

  \begin{align*}
    \implies
    \norm[H^1(\Omega)]{\widetilde{\phi}_n}^2
    =
    \norm[L^2(\Omega)]{\widetilde{\phi}_n}^2
    +
    \norm[L^2(\Omega)]{\widetilde{\phi}_n^\prime}^2
    = 1 + (2 \pi n)^2
    \underbrace
    {
      \Int[0][1]{|\exp{(2 \pi i n x)}|^2}{x}
    }_{
      =
      \norm[L^2(\Omega)]{\widetilde{\phi}_n}^2
      =
      1
    } = 1 + 4n^2\pi^2
  \end{align*}

  Die Orthogonalitätseigenschaft von $(\phi_n)_{n \in \Z}$ bzgl. $\norm[H^1(\Omega)]{\cdot}$ folgt aus derer von $(\widetilde{\phi}_n)_{n \in \Z}$ bzgl. $\norm[L^2(\Omega)]{\cdot}$.
  $\Forall n, m \in \Z, n \neq m:$

  \begin{align*}
    (\phi_n, \phi_m)_{H^1(\Omega)}
    =
    (\phi_n, \phi_m)_{L^2(\Omega)}
    +
    (\phi_n^\prime, \phi_m^\prime)_{L^2(\Omega)}
    =
    0
  \end{align*}

  Wir betrachten nun also die folgenden beiden Operatoren.

  \begin{align*}
    T:
    \begin{cases}
      H^1(\Omega) \to L^2(\Omega) \\
      f
      \mapsto
      \sum_{n \in \Z}
      (f; \phi_n)_{H^1(\Omega)} \phi_n
    \end{cases},
    \quad
    T_N:
    \begin{cases}
      H^1(\Omega)
      \to
      \Span
      \Bbraces{\phi_n}_{n = 0}^{\pm N}
      \subseteq
      L^2(\Omega) \\
      f
      \mapsto
      \sum_{n = -N}^N
      (f; \phi_n)_{H^1(\Omega)} \phi_n
    \end{cases},
    \quad
    N \in \N
  \end{align*}

  Um Operatornorm-Konvergenz zu zeigen, wählen wir $f \in L^2(\Omega)$ beliebig und schätzen (mit der Dreiecksungleichung) wie folgt ab.

  \begin{align*}
    \norm[L^2(\Omega)]{T f - T_n f}
    &=
    \norm[L^2(\Omega)]
    {
      \sum_{n \in \Z}
      (f, \phi_n)_{H^1(\Omega)} \phi_n
      -
      \sum_{n = -N}^N
      (f, \phi_n)_{H^1(\Omega)} \phi_n
    } \\
    &= \left\|\sum_{|n| > N}(f, \phi_n)_{H^1(\Omega)} \phi_n\right\|_{L^2(\Omega)}
    \leq \sum_{|n| > N}\left|(f, \phi_n)_{H^1(\Omega)}| \|\phi_n\right\|_{L^2(\Omega)} \\
    &\leq \sqrt{\sum_{|n| > N}|(f, \phi_n)_{H^1(\Omega)}|^2 \sum_{|n| > N}\|\phi_n\|_{L^2(\Omega)}^2}
  \end{align*}
  Weiters gilt mit
  der Bessel'schen Ungleichung:

  \includegraphicsboxed
  [\cite{Ana3}]
  {(2.13) - Blümlinger - Analysis 3.png}

  \begin{align*}
  \sqrt{\sum_{|n| > N}|(f, \phi_n)_{H^1(\Omega)}|^2 \sum_{|n| > N}\|\phi_n\|_{L^2(\Omega)}^2}
  \leq \|f\|_{H^1(\Omega)}\underbrace{\sqrt{\sum_{|n| > N}\frac{1}{1 + 4n^2\pi^2}}}_{\xrightarrow{N \to \infty} 0}
  \end{align*}
  Also haben wir die Operatorkonvergenz von $T_N$ gegen $T$ gezeigt.
  \item Unsere ONB vom $L^2(\Omega)$ lautet wie folgt.

  \begin{align*}
    \widetilde{\phi}_n
    =
    \exp{(2 \pi i n \cdot x)}
    =
    \exp{(2 \pi i (n_1 x_1 + n_2 x_1))},
    \quad
    n \in \Z^2,
    \quad
    x \in \Omega
  \end{align*}
  Wir überprüfen nur noch die Orthonormalität und zeigen, dass
  $\sum_{n \in \Z^2}\|\widetilde{\phi}_n\|_{H^1(\Omega)}^{-1} < \infty$.
  Der Rest ist analog zu \textbf{a)}.

  \begin{align*}
    (\widetilde{\phi}_n, \widetilde{\phi}_m)_{L^2(\Omega)}
    &= \int_0^1\int_0^1\exp(2\pi(n_1x_1 + n_2x_2))\exp(-2\pi(m_1x_1 + m_2x_2)) dx_1 dx_2 \\
    &= \int_0^1\exp(2\pi(n_1 - m_1)x_1)dx_1 \int_0^1\exp(2\pi(n_2 - m_2)x_2)dx_2 \\
    &= \begin{cases}
      1 &, (n_1,n_2) = (m_1,m_2) \\
      0 &, \text{ sonst}
    \end{cases}.
  \end{align*}
  Nun berechne noch $\|\widetilde{\phi}_n\|_{H^1(\Omega)}$:
  \begin{align*}
    \|\widetilde{\phi}_n\|_{H^1(\Omega)}^2
    &= \|\widetilde{\phi}_n\|_{L^2(\Omega)}^2  + \|\nabla \widetilde{\phi}_n\|_{L^2(\Omega)}^2
    = 1 + \int_0^1\int_0^1 \|(2\pi in_1, 2\pi in_2)\widetilde{\phi}_n(x_1,x_2)\|^2 dx_1 dx_2 \\
    &= 1 + 4\pi^2(n_1^2 + n_2^2)
  \end{align*}

  Großes Problem! Die Reihe divergiert laut Wolfram-Alpha, also
  lässt sich die Argumentation aus a) doch nicht so schön übertragen.
\end{enumerate}

\end{solution}

% -------------------------------------------------------------------------------- %
