% --------------------------------------------------------------------------------

\begin{exercise}

Sei $P_p:= \mathcal{L}\{x^i y^j: i,j \geq 0 \land i+j \leq p\}$ für $p \in \N$ der Raum
der Polynome vom maximalen Grad $p$.

\begin{enumerate}[label = \textbf{\alph*)}]
  \item Geben Sie eine Basis von $P_0$ an.
  \item Zeigen Sie, dass die Funktionen

  \begin{align}
    \lambda_1(x,y) := 1-x-y, \quad \lambda_2(x,y) := x \quad \lambda_3(x,y) := y
  \end{align}

  eine Basis des $P_1$ bilden.
  \item Zeigen Sie, dass die Funktionen $\lambda_1, \lambda_2, \lambda_3, \lambda_1\lambda_2, \lambda_1\lambda_3, \lambda_2\lambda_3$
  eine Basis des $P_2$ bilden.
  \item Zeigen Sie, dass eine Basis des $P_p$ für $p \geq 3$ aus folgenden Funktionen gebildet werden kann:
  \begin{enumerate}[label = (\roman*)]
    \item $(x,y) \mapsto \lambda_j(x,y)$ mit $1 \leq j \leq 3$
    \item $(x,y) \mapsto p_{jk}(x,y)\lambda_j(x,y)\lambda_k(x,y)$ mit $1 \leq k < j \leq 3$ und
    \item $(x,y) \mapsto p_{123}(x,y) \prod_{j=1}^3 \lambda_j(x,y)$ und $p_{123} \in P_{p-3}$
  \end{enumerate}
  Die folgenden Einschränkungen der Polynome $p_{jk}$ sind dabei jeweils eindimensionale Polynome vom maximalen
  Grad $p-2: \xi \mapsto p_{12}(\xi,0), \xi \mapsto p_{13}(0,\xi)$ und $\xi \mapsto p_{23}(\xi, 1-\xi)$.
  \item Erklären Sie andhand des Referenzdreiecks mit den Eckpunkten $(0,0), (1,0)$ und $(0,1)$
  die Bedeutung dieser Aufgabe auf eine Erweiterung von Proposition $3.1$ auf Polynomräume höheren Grades.
\end{enumerate}

\end{exercise}

% --------------------------------------------------------------------------------

\begin{solution}

ToDo!

\end{solution}

% --------------------------------------------------------------------------------
