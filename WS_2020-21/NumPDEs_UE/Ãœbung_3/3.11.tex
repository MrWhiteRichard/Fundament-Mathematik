% --------------------------------------------------------------------------------

\begin{exercise}

\phantom{}

\begin{enumerate}[label = \textbf{\alph*)}]

  \item Sei $(T_j)_{j \in \N} \subset \R^2$ eine Folge von nicht-degenerierten Dreiecken.
  Zeigen Sie, dass die shape regularity Konstanten $\sigma(T_j) = h_{T_j} / \rho_{T_j}$ genau dann gegen unendlich divergieren wenn der kleinste Winkel in $T_j$ gegen Null geht.

  \item Eine alternative Definition der shape regularity Konstante ist gegeben durch $\tilde{\sigma}(T) := h_T / r_t$, wobei

  \begin{align*}
    r_T := \max
    \Bbraces
    {
      \text{diam}(B):
      B ~\text{ein Kreis enthalten in}~ T
    }.
  \end{align*}

  Wie hängen $\sigma(T)$ und $\tilde{\sigma}(T)$ zusammen?

\end{enumerate}

\end{exercise}

  % --------------------------------------------------------------------------------

  \begin{solution}

  \phantom{}

  \begin{figure}[h!]
    \centering
    \begin{tikzpicture}[scale=1.25]

      \coordinate [label = left:        $A$] (A) at (0, 0);
      \coordinate [label = below right: $B$] (B) at (8, 0);
      \coordinate [label = above:       $C$] (C) at (2, 4);
      \draw
        (A) -- node [below]       {$c$}
        (B) -- node [above right] {$a$}
        (C) -- node [above left]  {$b$}
        cycle;

        \coordinate (M) at (2, 0);
        \draw (C) -- node [right] {$h$} (M);

        \draw (1, 0) node [above] {$c_A$};
        \draw (5, 0) node [above] {$c_B$};

        \draw (0.5, 0) arc (0:64:0.5);
        \draw (0.25, 0.15) node {$\alpha$};

        \draw (7, 0) arc (180:146:1);
        \draw (7.25, 0.2) node {$\beta$};

        \draw (2, 3) arc (270:243.5:1);
        \draw (2, 3) arc (270:326:1);
        \draw (2.35, 3.4) node {$\gamma$};

        \draw (2.25, 0) arc (0:90:0.25);
        \draw (2.1, 0.1) node {$\cdot$};

      \end{tikzpicture}
      \caption{Dreieck $T$}
      \label{fig:dreieck}
  \end{figure}

  \begin{enumerate}[label = \textbf{\alph*)}]

    \item Sei $T$ ein Dreieck.
    Man betrachte die Skizze aus Abbildung \ref{fig:dreieck}.
    Der $\sin$-Satz besagt Folgendes.

    \begin{align*}
      \frac{a}{\sin \alpha}
      =
      \frac{b}{\sin \beta}
      =
      \frac{c}{\sin \gamma}
    \end{align*}

    $\arcsin$ steigt monoton und $c \geq a, b$.

    \begin{align*}
      \implies
      \gamma
      =
      \arcsin \frac{c \sin \gamma}{c}
      \geq
      \begin{cases}
        \arcsin \frac{a \sin \gamma}{c} = \alpha \\
        \arcsin \frac{b \sin \gamma}{c} = \beta
      \end{cases}
    \end{align*}

    Also ist $\gamma$ der größte Winkel ($\to$ uninteressant).

    \begin{align*}
      \implies
      \sigma(T)
      =
      \frac{c}{h}
      =
      \frac{c_A + c_B}{h}
      =
      \frac{c_A}{h} + \frac{c_B}{h}
      =
      \frac{1}{\tan \alpha} + \frac{1}{\tan \beta}
    \end{align*}

    Daraus folgt die Behauptung.

    \item Sei $r$ der Inkreis-Radius und $d = 2r$ der Inkreis-Durchmesser.
    Wikipedia kennt folgende Formel.

    \begin{align*}
      r = \frac{2 |T|}{a + b + c} = \frac{ch}{a + b + c}
    \end{align*}

    \begin{align*}
      \implies
      \sigma(T)
      =
      \frac{c}{h}
      \leq
      \frac{c}{d}
      =
      \tilde{\sigma}(T)
      =
      \frac{c}{d}
      =
      \frac{c}
      {
        2 \frac{ch}{a + b + c}
      }
      =
      \frac{a + b + c}{2h}
      \leq
      \frac{3c}{2h}
      =
      \frac{3}{2} \sigma(T)
    \end{align*}

  \end{enumerate}

  \end{solution}

  % --------------------------------------------------------------------------------
