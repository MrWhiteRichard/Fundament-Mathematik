% --------------------------------------------------------------------------------

\begin{exercise}

\begin{enumerate}[label = \textbf{\alph*)}]
  \item Sei $(T_j)_{j \in \N} \subset \R^2$ eine Folge von nicht-degenerierten Dreiecken.
  Zeigen Sie, dass die shape regularity Konstanten $\sigma(T_j) = h_{T_j} / \rho_{T_j}$ genau dann gegen unendlich divergieren
  wenn der kleinste Winkel in $T_j$ gegen Null geht.
  \item Eine alternative Definition der shape regularity Konstante ist gegeben durch
  $\tilde{\sigma}(T) := h_T / r_t$, wobei

  \begin{align*}
    r_T := \max\{\text{diam}(B): B~ \text{ein Kreis enthalten in}~ T\}.
  \end{align*}

  Wie hängen $\sigma(T)$ und $\tilde{\sigma}(T)$ zusammen?
\end{enumerate}

\end{exercise}

% --------------------------------------------------------------------------------

\begin{solution}

ToDo!

\end{solution}

% --------------------------------------------------------------------------------
