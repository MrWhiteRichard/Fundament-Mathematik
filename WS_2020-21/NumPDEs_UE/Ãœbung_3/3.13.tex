% -------------------------------------------------------------------------------- %

\begin{exercise}

  Sei $\hat Q := (0, 1) \times (0, 1)$ und $\hat T$ das offene Dreieck mit den Eckpunkten $(0, 0), (1, 0), (0, 1)$.
  Sei weiters

  \begin{align*}
    \Psi:
    \begin{cases}
      \R^2   & \to \R^2 \\
      (x, y) & \mapsto (x, (1 - x)y)
    \end{cases}
  \end{align*}

  \begin{enumerate}[label = \textbf{\alph*)}]

    \item Zeigen Sie, dass die Abbildung $\Psi$ ein Diffeomorphismus zwischen $\hat Q$ und $\hat T$ ist.

    \item Seien für $N, M \in \N$ zwei Quadraturformeln $Q_N, Q_M$ der Ordnung $N$ bzw. $M$ auf dem Einheitsintervall gegeben.
    Konstruieren Sie daraus eine Quadraturformel $Q_{\hat Q}$ auf $\hat Q$.
    Welche Funktionen werden durch $Q_{\hat Q}$ exakt integriert?

    \item Verwenden Sie die Abbildung $\Psi$ und die Quadratur aus b) um eine Quadratur $Q_{\hat T}$ auf $\hat T$ zu konstruieren.
    Welche Funktionen werden durch $Q_{\hat T}$ exakt integriert?

  \end{enumerate}

  \end{exercise}

  % -------------------------------------------------------------------------------- %

  \begin{solution}

  \begin{align*}
    \hat Q
    =
    (0, 1)^2
    =
    \Bbraces{(x, y) \in \R^2: 0 < x, y < 1},
    \quad
    \hat T
    =
    \Bbraces{(x, y) \in (\R^+)^2: x + y < 1}
  \end{align*}

  \begin{enumerate}[label = \textbf{\alph*)}]

    \item

    \begin{enumerate}[label = \arabic*.]

      \item Schritt:

      Wir zeigen, dass $\Psi: \hat Q \to \hat T$ tatsächlich von $\hat Q$ nach $\hat T$ geht.
      Sei dazu $(x, y) \in \hat Q$.

      \begin{align*}
        \implies
        \Exists \varepsilon_x, \varepsilon_y \in (0, 1):
        x = 1 - \varepsilon_x,
        y = 1 - \varepsilon_y
      \end{align*}

      \begin{align*}
        \implies
        x + (1 - x) y
        =
        x + y - xy
        =
        1 - \varepsilon_x + 1 - \varepsilon_y - (1 - \varepsilon_x)(1 - \varepsilon_y)
        =
        1 - \varepsilon_x \varepsilon_y
        <
        1
      \end{align*}

      \item Schritt:

      \begin{align*}
        \Psi^{-1}:
        \begin{cases}
          \hat T & \to \hat Q \\
          (x, y) & \mapsto (x, \frac{y}{1 - x})
        \end{cases}
      \end{align*}

      Wir zeigen, dass $\Psi^{-1}: \hat T \to \hat Q$ tatsächlich von $\hat T$ nach $\hat Q$ geht.
      Sei dazu $(x, y) \in \hat T$.

      \begin{align*}
        \implies
        x < 1, \quad x + y < 1
        \implies
        x < 1, \quad y < 1 - x
        \implies
        x < 1, \quad \frac{y}{1-x} < 1
      \end{align*}

      Diese Funktion ist wirklich die Inverse.
      Das rechnen wir einfach nach.

      \begin{align*}
        (\Psi^{-1} \circ \Psi)(x, y)
        =
        \Psi^{-1}(x, (1 - x) y)
        =
        \pbraces{x, \frac{(1 - x) y}{1 - x}}
        =
        (x, y) \\
        (\Psi \circ \Psi^{-1})(x, y)
        =
        \Psi \pbraces{x, \frac{y}{1 - x}}
        =
        \pbraces{x, (1 - x) \frac{y}{1 - x}}
        =
        (x, y)
      \end{align*}

      \item Schritt:

      $\Psi$ und $\Psi^{-1}$ sind differenzierbar und deren Jakobimatrizen stetig und invertierbar.
      Also sind $\Psi$ und $\Psi^{-1}$ stetig differenzierbar und $\Psi$ ein Diffeomorphismus.

      \begin{align*}
        D \Psi(x, y)
        =
        \begin{pmatrix}
          1  & 0 \\
          -y & 1 - x
        \end{pmatrix},
        & \quad
        \det D \Psi(x, y) = 1 - x > 0 \\
        D \Psi^{-1}(x, y)
        =
        \begin{pmatrix}
          1                   & 0 \\
          \frac{y}{(1 - x)^2} & \frac{1}{1 - x}
        \end{pmatrix},
        & \quad
        \det D \Psi^{-1}(x, y) = \frac{1}{1 - x} > 0
      \end{align*}

    \end{enumerate}


    \item Die gegebenen Quadraturformeln $Q_N$ und $Q_M$ integrieren Polynome vom maximalen Grad $N - 1$ bzw. $M - 1$ exakt, d.h.

    \begin{align*}
      \Forall p \in \Pi_{N - 1}:
      Q_N(p)
      =
      \sum_{i = 1}^n \alpha_i p(x_i)
      =
      \Int[0][1]{p(x)}{x}
      =
      Q(p), \\
      \Forall q \in \Pi_{M - 1}:
      Q_M(q)
      =
      \sum_{j = 1}^m \beta_j q(y_j)
      =
      \Int[0][1]{q(x)}{x}
      =
      Q(q).
    \end{align*}

    Für $i = 1, \dots, n$ und $j = 1, \dots, m$ sind $\alpha_i$, $\beta_j$ die Gewichte und $x_i, y_j$ die Stützstellen der Quadraturformeln $Q_N$ bzw. $Q_M$.
    Wir definieren unsere Quadraturformel auf $\hat Q$ wie folgt.

    \begin{align*}
      Q_{\hat Q}(f)
      :=
      \sum_{i=1}^n
      \sum_{j=1}^m
      \alpha_i \beta_j
      f(x_i, y_j)
    \end{align*}

    Diese integriert Polynome vom maximalen Grad $N - 1$ in der ersten Variable und maximalen Grad $M - 1$ in der zweiten Variable exakt, d.h.

    \begin{align*}
      \Forall p \in \Pi_{N - 1, M - 1}
      :=
      \mathcal{L}
      \Bbraces{x^i y^j: i, j \in \N_0, i < N, j < M}:
    \end{align*}

    \begin{multline*}
      Q_{\hat Q}(r)
      =
      \sum_{i=1}^n
      \sum_{j=1}^m
      \alpha_i \beta_j
      p(x_i, y_j)
      =
      \sum_{i=1}^n
      \alpha_i
      \sum_{j=1}^m
      \beta_j
      p(x_i, y_j) \\
      =
      \sum_{i=1}^n
      \alpha_i
      \Int[0][1]{p(x_i, y)}{y}
      =
      \Int[0][1]
      {
        \sum_{i=1}^n
        \alpha_i
        p(x_i, y)
      }{y}
      =
      \Int[0][1]
      {
        \Int[0][1]{p(x, y)}{x}
      }{y}.
    \end{multline*}

    \item Um nun eine Quadraturformel auf $\hat{T}$ zu konstruieren verwenden wir die vorhergehenden Aufgabenteile
  und die Transformationsfomel

  \begin{align*}
    Q_{\hat{T}}(f)
    :=
    Q_{\hat{Q}}((f\circ \Psi) |\det D\Psi|)
  \end{align*}

  Durch diese Quadraturformel werden genau die Polynome der Form

  \begin{align*}
    p: (p \circ \Psi)(x,y) |\det D\Psi(x,y)| = p(x,(1-x)y)(1-x)\in \Pi_{N-1,M-1}
  \end{align*}

  exakt integriert. Die Bedingung wird mit Sicherheit von allen $p \in \Pi_{N-2,\min\{N-2,M-1\}}$
  erfüllt. Das folgt aus

  \begin{align*}
  Q_{\hat{T}}(p)
  &:=
  Q_{\hat{Q}}((p\circ \Psi) |\det D\Psi|)
  \stackrel{\textbf{b)}}{=}
  \Int[0][1]{
      \Int[0][1]{(p \circ \Psi)(x,y) |\det D\Psi|}{x}
            }{y} \\
  &=
  \Int[\hat{Q}]{(p \circ \Psi)|\det D\Psi|}{\lambda^2}
  \stackrel{\mathrm{TRAFO}}{=}
  \Int[\hat{T}]{p}{\lambda^2}
  \end{align*}
\end{enumerate}

  \end{solution}

  % -------------------------------------------------------------------------------- %
