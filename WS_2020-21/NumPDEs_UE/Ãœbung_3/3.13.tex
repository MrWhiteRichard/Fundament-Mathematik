% --------------------------------------------------------------------------------

\begin{exercise}

Sei $\hat{Q}:= (0,1) \times (0,1)$ und $\hat{T}$ das offene Dreieck mit den Eckpunkten
$(0,0),(1,0),(0,1)$. Sei weiters

\begin{align*}
  \Psi:
  \begin{cases}
    \R^2 &\rightarrow \R^2 \\
    (x,y) & \mapsto (x,(1-x)y)
  \end{cases}
\end{align*}

\begin{enumerate}[label = \textbf{\alph*)}]
  \item Zeigen Sie, dass die Abbildung $\Psi$ ein Diffeomorphismus zwischen $\hat{Q}$ und $\hat{T}$ ist.
  \item Seien für $N, M \in \N$ zwei Quadraturformeln $Q_N, Q_M$ der Ordnung $N$ bzw. $M$ auf dem
  Einheitsintervall gegeben. Konstruieren Sie daraus eine Quadraturformel $Q_{\hat{Q}}$ auf $\hat{Q}$.
  Welche Funktionen werden durch $Q_{\hat{Q}}$ exakt integriert?
  \item Verwenden Sie die Abbildung $\Psi$ und die Quadratur aus b) um eine Quadratur $Q_{\hat{T}}$ auf $\hat{T}$
  zu konstruieren. Welche Funktionen werden durch $Q_{\hat{T}}$ exakt integriert?
  \end{enumerate}
\end{exercise}

% --------------------------------------------------------------------------------

\begin{solution}

Wir bemerken zuerst $\hat{T} = \{(x,y) \in \R_+^2: x + y < 1\}$.

\begin{enumerate}[label = \textbf{\alph*)}]
  \item Zeigen wir zuerst die Wohldefiniertheit von $\Psi: \hat{Q} \rightarrow \hat{T}$.
  Sei dazu $(x,y) \in \hat{Q}$. Da $\hat{Q}$ ja genau das offene Einheitsquadrat ist,
  gibt es also $\varepsilon_x, \varepsilon_y \in (0,1): x = 1 - \varepsilon_x, y = 1 -\varepsilon_y$. Dann gilt also

  \begin{align*}
    x + (1-x)y
    =
    x + y - xy
    =
    1 - \varepsilon_x + 1 - \varepsilon_y - (1 - \varepsilon_x)(1 - \varepsilon_y)
    =
    1 - \varepsilon_x \varepsilon_y
    <
    1
  \end{align*}

  Also bildet $\Psi$ nach $\hat{T}$ ab. Die Inverse ist gegeben durch

  \begin{align*}
    \Psi^{-1}: \begin{cases}
    \hat{T} \rightarrow \hat{Q} \\
    (x,y) \mapsto (x, \frac{y}{1-x})
    \end{cases}
  \end{align*}
 $\Psi^{-1}$ ist als Abbildung von $\hat{T} \to \hat{Q}$ wohldefiniert, das für $(x,y) \in T$
 \begin{align*}
   x < 1, \quad x+y < 1 \iff y < 1 - x \iff \frac{y}{1-x} < 1.
 \end{align*}
  Das diese Funktion wirklich die Inverse ist rechnen wir einfach einfach nach

  \begin{align*}
    (\Psi^{-1} \circ \Psi)(x,y)
    =
    \Psi^{-1}(x,(1-x)y)
    =
    \left(x, \frac{(1-x)y}{1-x}\right)
    =
    (x,y) \\
    (\Psi \circ \Psi^{-1})(x,y)
    =
    \Psi\left(x, \frac{y}{1-x}\right)
    =
    \left(x,(1-x)\frac{y}{1-x}\right)
    =
    (x,y)
  \end{align*}

  Um zu zeigen, dass $\Psi$ ein Diffeomorphismus ist, sehen wir uns die Ableitungsmatrizen an

  \begin{align*}
    D\Psi(x,y)
    =
    \begin{pmatrix}
      1 & 0 \\
      -y & 1-x
    \end{pmatrix}, \quad \mathrm{Det} D\Psi(x,y) = 1 - x > 0 \\
    D\Psi^{-1}(x,y)
    =
    \begin{pmatrix}
      1 & 0 \\
      \frac{y}{(1-x)^2} & \frac{1}{1-x}
    \end{pmatrix}, \quad \mathrm{Det} D\Psi^{-1}(x,y) = \frac{1}{1-x} > 0
  \end{align*}

  \item Die gegebenen Quadraturformeln sind für Polynome vom Grad $N-1$ bzw Grad $M-1$ exakt

  \begin{align*}
    \forall p \in \Pi_{N-1}:
    Q_N(p) = Q(p)
    \quad \text{d.h.}~
    \sum_{i = 1}^n \alpha_i p(x_i)
    =
    \Int[0][1]{p(x)}{x} \\
    \forall q \in \Pi_{M-1}:
    Q_M(q) = Q(q)
    \quad \text{d.h.}~
    \sum_{j = 1}^m \beta_j q(y_j)
    =
    \Int[0][1]{q(x)}{x}
  \end{align*}

  Wobei die $\alpha_i, \beta_j$ die Gewichte und $x_i, y_j$ die Stützstellen sind.
  Wir definieren unsere Quadraturformel auf $\hat{Q}$ wie folgt:

  \begin{align*}
    Q_{\hat{Q}}(f)
    :=
    \sum_{i=1}^n \sum_{j=1}^m \alpha_i \beta_j f(x_i, y_j)
  \end{align*}

  Wir behaupten, dass diese exakt ist für

  \begin{align*}
    r \in \Pi_{N-1,M-1} := \mathcal{L}\{x^i y^j: i,j \in \N_0, i < N, j < M\}
  \end{align*}

  Für solche Polynome gilt

  \begin{align*}
  Q_{\hat{Q}}(r)
  =
  \sum_{i=1}^n \sum_{j=1}^m \alpha_i \beta_j r(x_i, y_j)
  =
  \sum_{i=1}^n \alpha_i \sum_{j=1}^m \beta_j r(x_i, y_j)
  =
  \sum_{i=1}^n \alpha_i \Int[0][1]{r(x_i,y)}{y}
  =
  \Int[0][1]{\sum_{i=1}^n \alpha_i r(x_i,y)}{y} \\
  =
  \Int[0][1]{
      \Int[0][1]{r(x,y)}{x}
            }{y}
  \end{align*}

  \item Um nun eine Quadraturformel auf $\hat{T}$ zu konstruieren verwenden wir die vorhergehenden Aufgabenteile
  und die Transformationsfomel

  \begin{align*}
    Q_{\hat{T}}(f)
    :=
    Q_{\hat{Q}}((f\circ \Psi) |\det D\Psi|)
  \end{align*}

  Durch diese Quadraturformel werden genau die Polynome der Form

  \begin{align*}
    p: (p \circ \Psi)(x,y) |\det D\Psi(x,y)| = p(x,(1-x)y)(1-x)\in \Pi_{N-1,M-1}
  \end{align*}

  exakt integriert. Die Bedingung wird mit Sicherheit von allen $p \in \Pi_{N-2,\min\{N-2,M-1\}}$
  erfüllt. Das folgt aus

  \begin{align*}
  Q_{\hat{T}}(p)
  &:=
  Q_{\hat{Q}}((p\circ \Psi) |\det D\Psi|)
  \stackrel{\textbf{b)}}{=}
  \Int[0][1]{
      \Int[0][1]{(p \circ \Psi)(x,y) |\det D\Psi|}{x}
            }{y} \\
  &=
  \Int[\hat{Q}]{(p \circ \Psi)|\det D\Psi|}{\lambda^2}
  \stackrel{\mathrm{TRAFO}}{=}
  \Int[\hat{T}]{p}{\lambda^2}
  \end{align*}
\end{enumerate}

\end{solution}

% --------------------------------------------------------------------------------
