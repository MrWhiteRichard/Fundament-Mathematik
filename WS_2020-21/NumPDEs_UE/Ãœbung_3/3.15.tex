% --------------------------------------------------------------------------------

\begin{exercise}

Sei $\Omega_\beta := \{r(\cos \phi, \sin \phi)^T \in \R^2: r \in (0,1), \phi \in (0, \pi / \beta)\}$
für $\beta \in (1/2,1)$ ein nicht-konvexer Kreissektor.
\begin{enumerate}[label = \textbf{(\alph*)}]
  \item Verwenden Sie den Ansatz $u(r,\phi) = (1-r^2)r^\beta \sin(\beta \phi)$ zur Konstruktion
  einer Lösung $u \in H^1_0(\Omega_\beta)$ mit $u \notin H^2(\Omega_\beta)$ der Poisson Gleichung
  $\Delta u = f$ mit passendem $f \in L^2(\Omega_\beta)$.
  \item Überprüfen Sie numersich mit Hilfe des zur Verfügung gestellten Jupyter-Notebooks für
  $\beta = 2/3$, welche Konvergenzrate $h^s$ mit $s > 0$ sich für diese Lösung bei unterschiedlichen
  Polynomordnungen bei Verfeinerungen von $h$ ergibt.
  \item Testen Sie die Konvergenzrate einer geeigneten Lösung auf dem konvexen Gebiet $\Omega_\beta$ mit $\beta = 2$.
\end{enumerate}

\end{exercise}

% --------------------------------------------------------------------------------

\begin{solution}

Siehe Jupyter-Notebook.

\end{solution}

% --------------------------------------------------------------------------------
