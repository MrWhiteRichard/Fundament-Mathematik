% --------------------------------------------------------------------------------

\begin{exercise}

Sei $\mathcal{T}$ eine reguläre Triangulierung des beschränkten Lipschitz-Gebietes $\Omega \subset \R^2$ und $h \in L^\infty(\Omega)$ definiert durch $h|_T = h_T$ für alle $T \in \mathcal{T}$.
\begin{enumerate}[label = \textbf{\alph*)}]
  \item Zu einer Funktion $v \in H^1(\Omega)$ definieren wir die Funktion
  \begin{align*}
    v_\mathcal{T} \in \S_{-1}^0(\mathcal{T})
    :=
    \Bbraces
    {
      v_h \in L^2(\Omega):
      \Forall T \in \mathcal{T}
      \quad
      v_h|_T \in P_0
    }
  \end{align*}
  durch $v_\mathcal{T}|_T := |T|^{-1} \Int[T]{v}{x}$ für alle $T \in \mathcal{T}$.
  Zeigen Sie, dass
  \begin{align*}
    \norm[L^2(\Omega)]{v - v_\mathcal{T}}
    \leq
    C \norm[L^2(\Omega)]{h \nabla v}
  \end{align*}
  gilt. Die Konstante $C > 0$ hängt dabei weder von $\Omega$ noch von $v$ oder $\mathcal{T}$ ab.
  \item Zeigen Sie für $\norm[L^\infty(\Omega)]{h} < 1$ und $p \in \N$
  \begin{align}
    \norm[L^2(\Omega)]{h D^2 v_h}
    \leq
    C \norm[H^1(\Omega)]{v_h},
    \quad
    v_h \in \S_0^p(\mathcal{T}).
  \end{align}
  Die Konstante $C > 0$ hängt dabei nur von $\sigma(\mathcal{T})$ ab.
\end{enumerate}

\end{exercise}

% --------------------------------------------------------------------------------

\begin{solution}
\phantom{}
\begin{enumerate}[label = \textbf{\alph*)}]
  \item
  \begin{enumerate}[label = \arabic*.]
    \item Schritt (Abschätzung auf dem Referenz-Element $T_\mathrm{ref}$):
    Definiere dafür das Integralmittel
    \begin{align*}
      v_{T_\mathrm{ref}}
      :=
      |T_\mathrm{ref}|^{-1}
      \Int[T_\mathrm{ref}]{v}{x}.
    \end{align*}
    \includegraphicsunboxed{NumPDEs/NumPDEs - Corollary 2.8 (Poincaré Inequality).png}
    Laut Corollary 2.8 (Poincaré Inequality) $\Exists C_P > 0: v \in H^1(T_\mathrm{ref}):$
    \begin{align*}
      \norm[L^2(T_\mathrm{ref})]{v - v_{T_\mathrm{ref}}}
      \leq
      C_P \norm[L^2(T_\mathrm{ref})]{\nabla (v- v_{T_\mathrm{ref}})}
      =
      C_P \norm[L^2(T_\mathrm{ref})]{\nabla v}.
    \end{align*}
    \item Schritt (Skalierungs-Argument für Abschätzung auf jedem Element $T \in \mathcal{T}$):
    Auch hier, definiere dafür das Integralmittel
    \begin{align*}
      v_T
      :=
      |T|^{-1}
      \Int[T]{v}{x}.
    \end{align*}
    Sei $\Phi_T$ der affine Diffeomorphismus aus Lemma 3.9.
    \includegraphicsunboxed{NumPDEs/NumPDEs - Lemma 3.9.png}
    Das bilden des Integralmittels ist damit in gewisser Weise verträglich.
    \begin{align*}
      (v \circ \Phi_T)_{T_\mathrm{ref}}
      &=
      |T_\mathrm{ref}|^{-1}
      \Int[T_\mathrm{ref}]{v \circ \Phi_T}{x} \\
      \stackrel
      {
        \mathrm{TRAFO}
      }{=}
      |T_\mathrm{ref}|^{-1}
      |\det(B)|^{-1}
      \Int[\Phi_T(T_\mathrm{ref})]{v}{x}
      &=
      2 \frac{1}{2 |T|}
      \Int[T]{v}{x}
      =
      v_\mathcal{T}|_T \circ \Phi_T
    \end{align*}
    \includegraphicsunboxed{NumPDEs/NumPDEs - Lemma 3.8 (Transformation Formula).png}
    Definiere nun $u := v \circ \Phi_T$.
    Wende Lemma 3.8 (Transformation Formula) für $m = 0$ auf $\Phi_T^{-1}$ an.
    \begin{multline*}
      \norm[L^2(T)]{v - v_\mathcal{T}|_T}
      =
      \norm[L^2(T)]{(u - u_{T_\mathrm{ref}}) \circ \Phi_T^{-1}} \\
      \stackrel
      {
        \mathrm{3.8}
      }{\leq}
      |\det(B)^{-1}|^{-1/2}
      \norm[L^2(T_\mathrm{ref})]{u - u_{T_\mathrm{ref}}}
      \leq
      |\det(B)|^{1/2}
      C_P
      \norm[L^2(T_\mathrm{ref})]{\nabla u}
    \end{multline*}
    Mit Lemma 3.8 (Transformation Formula) für $m = 1$ auf $\Phi_T$, sowie der
    geometrischen Interpretation von $\norm[F]{B}$ aus Lemma 3.9 erhalten wir
    \begin{multline*}
      \norm[L^2(T_\mathrm{ref})]{\nabla u}
      =
      \norm[L^2(T_\mathrm{ref})]{\nabla(v \circ \Phi_T)} \\
      \leq
      |\det(B)|^{-1/2}
      \norm[F]{B}^1
      \norm[L^2(T)]{\nabla v}
      \leq
      |\det(B)|^{-1/2}
      \sqrt{2}
      h_T
      \norm[L^2(T)]{\nabla v}.
    \end{multline*}
    Die Kombination der letzten beiden Abschätzungen zeigt, dass
    \begin{align*}
      \norm[L^2(T)]{v - v_\mathcal{T}|_T}
      \leq
      |\det(B)|^{1/2} C_P |\det(B)|^{-1/2} \sqrt{2} h_T \norm[L^2(T)]{\nabla v}
      =
      \sqrt{2} C_P h_T \norm[L^2(T)]{\nabla v}.
    \end{align*}
    \item Schritt (Aufsummieren zum gesamten Gebiet $\Omega$):
    \begin{align*}
      \norm[L^2(\Omega)]{v - v_\mathcal{T}}^2
      =
      \sum_{T \in \mathcal{T}}
      \norm[L^2(T)]{v - v_\mathcal{T}|_T}^2
      \leq
      \sum_{T \in \mathcal{T}}
      2 C_P^2 \norm[L^2(T)]{h_T \nabla v}^2
      =
      2 C_P^2 \norm[L^2(\Omega)]{h \nabla v}^2
    \end{align*}
    Die Konstante $C_P$ hängt nur vom Referenzdreieck $T_\mathrm{ref}$ ab.
  \end{enumerate}
  \item
  \begin{enumerate}[label = \arabic*.]
    \item Schritt (Abschätzung auf jedem Element $T \in \mathcal{T}$):
    \includegraphicsunboxed{NumPDEs/NumPDEs - Theorem 4.12 (inverse estimate).png}
    Wir verwenden Theorem 4.12 (inverse estimate) mit $k = 2$, $r = 1$:
    Also gilt $\Forall v_h \in \mathcal{P}^p(\mathcal{T}) \supset \S_0^p(\mathcal{T})$:
    \begin{align*}
      \norm[L^2(T)]{D^2 v_h}
      & \leq
      C \sigma(\mathcal{T})^2 h_T^{1-2} \norm[L^2(T)]{\nabla v_h} \\
      \implies
      \norm[L^2(T)]{h_T D^2 v_h}
      & \leq
      C \sigma(\mathcal{T})^2 \norm[L^2(T)]{\nabla v_h}
      \leq
      C \sigma(\mathcal{T})^2 \norm[H^1(T)]{v_h}
    \end{align*}
    Die Konstante $C$ hängt nur vom Referenzdreieck $T_\mathrm{ref}$ ab.
    \item Schritt (Aufsummieren zum gesamten Gebiet $\Omega$):
    \begin{align*}
      \norm[L^2(\Omega)]{h D^2 v_h}^2
      =
      \sum_{T \in \mathcal{T}}
      \norm[L^2(T)]{h_T D^2 v_h}^2
      \leq
      \sum_{T \in \mathcal{T}}
      C^2 \sigma(\mathcal{T})^4 \norm[H^1(T)]{v_h}^2
      =
      C^2 \sigma(\mathcal{T})^4 \norm[H^1(\Omega)]{v_h}^2
    \end{align*}
  \end{enumerate}
\end{enumerate}
\end{solution}

% --------------------------------------------------------------------------------
