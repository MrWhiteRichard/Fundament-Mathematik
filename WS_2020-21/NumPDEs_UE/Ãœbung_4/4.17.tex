% --------------------------------------------------------------------------------

\begin{exercise}

Sei $\mathcal{T}$ eine reguläre Triangulierung des beschränkten Lipschitz-Gebietes $\Omega \subset \R^2$ und $h \in L^\infty(\Omega)$ definiert durch $h|_T = h_T$ für alle $T \in \mathcal{T}$.

\begin{enumerate}[label = \textbf{\alph*)}]
  \item Zu einer Funktion $v \in H^1(\Omega)$ definieren wir die Funktion $v_\mathcal{T} \in \S_{-1}^0(\mathcal{T}) := \Bbraces{v_h \in L^2(\Omega): \forall T \in \mathcal{T}~~ v_h|_T \in P_0}$ durch $v_\mathcal{T}|_T := |T|^{-1} \Int[T]{v}{x}$ für alle $T \in \mathcal{T}$. Zeigen Sie, dass

  \begin{align*}
    \norm[L^2(\Omega)]{v - v_\mathcal{T}}
    \leq
    C \norm[L^2(\Omega)]{h \nabla v}
  \end{align*}

  gilt. Die Konstante $C > 0$ hängt dabei weder von $\Omega$ noch von $v$ oder $\mathcal{T}$ ab.

  \item Zeigen Sie für $\norm[L^\infty(\Omega)]{h} < 1$ und $p \in \N$

  \begin{align}
    \norm[L^2(\Omega)]{h D^2v_h}
    \leq
    C \norm[H^1(\Omega)]{v_h}, \quad
    v_h \in \S_0^p(\mathcal{T}).
  \end{align}

  Die Konstante $C > 0$ hängt dabei nur von $\sigma(\mathcal{T})$ ab.
\end{enumerate}

\end{exercise}

% --------------------------------------------------------------------------------

\begin{solution}
\phantom{}
\begin{enumerate}[label = \textbf{\alph*)}]
  \item Wir beginnen mit einer Abschätzung auf dem Referenzdreieck $T_{\mathrm{ref}}$: \\
  Definiere dafür $v_{T_{\mathrm{ref}}} := |T_{\mathrm{ref}}|^{-1}\int_{T_{\mathrm{ref}}} v dx$.
  Mit der Poincaréschen Ungleichung für den $H_*^1$ erhalten wir für $v \in H^1(T_{\mathrm{ref}})$
  \begin{align*}
    \|v - v_{T_{\mathrm{ref}}}\|_{L^2(T_{\mathrm{ref}})} \leq
    C_P\|\nabla(v- v_{T_{\mathrm{ref}}})\|_{L^2(T_{\mathrm{ref}})}
    = C_P\|\nabla(v)\|_{L^2(T_{\mathrm{ref}})}.
  \end{align*}
  Nebenrechnung:
  \begin{align*}
    (v\circ\Phi_T)_{T_{\mathrm{ref}}} = |T_{\mathrm{ref}}|^{-1}\int_{T_{\mathrm{ref}}} v \circ \Phi_T dx
    = |T_{\mathrm{ref}}|^{-1}|\det(B)|^{-1}\int_{\Phi_T(T_{\mathrm{ref}})} v dx
    = 2 \frac{1}{2|T|}\int_{T} v dx = v_{\mathcal{T}}|_T\circ \Phi_T
  \end{align*}
  Definiere nun $u := v \circ \Phi_T$ und wende die Transformationsformel auf $m = 0$ und $\Phi_T^{-1}$ an:
  \begin{align*}
    \|v - v_{\mathcal{T}}|_T\|_{L^2(T)} &= \|(u - u_{T_{\mathrm{ref}}}) \circ \Phi_T^{-1}\|_{L^2(T)}
    \leq |\det(B)^{-1}|^{\nicefrac{-1}{2}}\|u - u_{T_{\mathrm{ref}}}\|_{L^2(T_{\mathrm{ref}})} \\
    &\leq |\det(B)|^{\nicefrac{1}{2}}C_P\|\nabla(u)\|_{L^2(T_{\mathrm{ref}})}
  \end{align*}
  Mit der Transformationsformel rückwärts für $m = 1$, sowie der geometrischen Interpretation
  von $\|B\|_F$ erhalten wir
  \begin{align*}
    \|\nabla(u)\|_{L^2(T_{\mathrm{ref}})} = \|\nabla(v \circ \Phi_T)\|_{L^2(T_{\mathrm{ref}})}
    \leq |\det(B)|^{\nicefrac{-1}{2}}\|B\|_F^1\|\nabla(v)\|_{L^2(T)}
    \leq |\det(B)|^{\nicefrac{-1}{2}}\sqrt{2}h_T\|\nabla(v)\|_{L^2(T)}.
  \end{align*}
  Insgesamt ergibt das
  \begin{align*}
    \|v - v_{\mathcal{T}}|_T\|_{L^2(T)} \leq
    |\det(B)|^{\nicefrac{1}{2}}C_P|\det(B)|^{\nicefrac{-1}{2}}\sqrt{2}h_T\|\nabla(v)\|_{L^2(T)}
    = \sqrt{2}C_Ph_T\|\nabla(v)\|_{L^2(T)}.
  \end{align*}
  Nun müssen wir nur noch aufsummieren:
  \begin{align*}
    \|v - v_{\mathcal{T}}\|_{L^2(\Omega)}^2 = \sum_{T \in \mathcal{T}}\|v-v_{\mathcal{T}}|_T\|_{L^2(T)}^2
    \leq \sum_{T \in \mathcal{T}}2C_P^2\|h_T \nabla(v)\|_{L^2(T)}^2 = 2C_P^2\|h\nabla(v)\|_{L^2(\Omega)}^2
  \end{align*}
  Die Konstante $C_P$ hängt hierbei nur vom Referenzdreieck $T_{\mathrm{ref}}$ ab.
  \item Wir verwenden die Inversionabschätzung für $k = 2, r = 1$: Also gilt für alle
 $v_h \in \mathcal{P}^p(\mathcal{T}) \supset \S_0^p(\mathcal{T})$:
  \begin{align*}
    \|D^2v_h\|_{L^2(T)} &\leq C\sigma(\mathcal{T})h_T^{1-2}\|\nabla(v_h)\|_{L^2(T)} \\
    \iff \|h_TD^2v_h\|_{L^2(T)} &\leq C\sigma(\mathcal{T})\|\nabla(v_h)\|_{L^2(T)}
    \leq C\sigma(\mathcal{T})\|v_h\|_{H^1(T)}
  \end{align*}
  Die Konstante $C$ hängt dabei nur vom Referenzdreieck ab. Jetzt müssen wir nur
  noch aufsummieren:
  \begin{align*}
    \|hD^2v_h\|_{L^2(\Omega)}^2 = \sum_{T \in \mathcal{T}}\|h_TD^2v_h\|_{L^2(T)}^2
    \leq \sum_{T \in \mathcal{T}}C^2\sigma(\mathcal{T})^2\|v_h\|_{H^1(T)}^2
    = C^2\sigma(\mathcal{T})^2\|v_h\|_{H^1(\Omega)}^2
  \end{align*}

\end{enumerate}



\end{solution}

% --------------------------------------------------------------------------------
