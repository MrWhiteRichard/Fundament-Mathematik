% --------------------------------------------------------------------------------

\begin{exercise}

Verallgemeinern Sie Theorem 3.5 auf die Triangulierung eines beschränkten Lipschitz-Gebietes $\Omega \in \R^3$ mit Tetraedern.
Beweisen Sie dazu die Abschätzungen (3.20) und (3.21) für ein nicht-entartetes Tetraeder $T$.
Die Konstante $\sigma(T)$ wird dabei analog zu Aufgabe 11b definiert durch den Quotienten aus dem Durchmesser $h_T := \max \Bbraces{|x - y|: x, y \in T}$ von $T$ und dem Radius $\rho_T$ der größten Kugel, welche noch ganz in $T$ liegt, d.h.

\begin{align}
  \rho_T
  :=
  \sup \Bbraces{\rho > 0: \Exists x \in T \quad B(x, \rho) \subset T}.
\end{align}

\end{exercise}

% --------------------------------------------------------------------------------

\begin{solution}

\phantom{}

\includegraphicsunboxed{NumPDEs/NumPDEs - Theorem 3.5 (Approximation Theorem).png}

Die einzigen Teile im Beweis von Theorem 3.5 (Approximation Theorem), die nicht für den $\R^3$ funktionieren, sind die, wo Lemma 3.9 verwendet wird.

\includegraphicsunboxed{NumPDEs/NumPDEs - Lemma 3.9.png}

\begin{tcolorbox}[standard jigsaw, opacityback = 0]

\textbf{Lemma 3.9}
($\R^3$)
Für

\begin{align*}
  \hat T
  =
  T_\mathrm{ref}
  =
  \conv
  \Bbraces
  {
    \begin{pmatrix}
      0 \\ 0 \\ 0
    \end{pmatrix},
    \begin{pmatrix}
      1 \\ 0 \\ 0
    \end{pmatrix},
    \begin{pmatrix}
      0 \\ 1 \\ 0
    \end{pmatrix},
    \begin{pmatrix}
      0 \\ 0 \\ 1
    \end{pmatrix}
  }
\end{align*}

das Referenz-Element und $T = \conv \Bbraces{z_1, z_2, z_3, z_4} \subset \R^2$ ein nicht-degeneriertes Dreieck, definieren wir

\begin{align*}
  \Phi_T:
  T_\mathrm{ref} \to T,
  x \mapsto z_1 + B x,
  \quad
  ~\text{wobei}~
  B := (z_2 - z_1, z_3 - z_1, z_4 - z_1) \in \R^{3 \times 3}.
\end{align*}

Dann gilt, dass $\abs{\det B} = 6 |T|$ und

\begin{align*}
  h_T / \sqrt{2} \leq \norm[F]{B} \leq \sqrt{3} h_T
  \quad
  \text{als auch}
  \quad
  \norm[F]{B^{-1}} \leq 3 C \rho_T^{-1},
\end{align*}

wobei $C > 0$ unabhängig von $T$ ist.

\end{tcolorbox}

\textbf{\textit{Beweis.}}
Es gilt, dass

\begin{align*}
  \norm[F]{B}^2
  =
  |z_2 - z_1|^2 + |z_3 - z_1|^2 + |z_4 - z_1|^2
  \leq
  3 h_T^2.
\end{align*}

Des weiteren

\begin{align*}
  \begin{rcases}
    |z_3 - z_2| \leq |z_3 - z_1| + |z_1 - z_2| \leq \sqrt{2} \pbraces{|z_3 - z_1|^2 + |z_1 - z_2|^2}^{1/2} \\
    |z_4 - z_2| \leq |z_4 - z_1| + |z_1 - z_2| \leq \sqrt{2} \pbraces{|z_4 - z_1|^2 + |z_1 - z_2|^2}^{1/2} \\
    |z_4 - z_3| \leq |z_4 - z_1| + |z_1 - z_3| \leq \sqrt{2} \pbraces{|z_4 - z_1|^2 + |z_1 - z_3|^2}^{1/2}
  \end{rcases}
  \leq
  \sqrt{2} \norm[F]{B}.
\end{align*}

Speziell, $h_T = \max \Bbraces{|z_2 - z_1|, z_3 - z_1|, |z_3 - z_2|} \leq \sqrt{2} \norm[F]{B}$.

\includegraphicsboxed{Ana3/Ana3 - Satz 4.4.1 (Koflächenformel).png}

Wir benützen die Koflächenformel und berechnen

\begin{align*}
  |T_\mathrm{ref}|
  =
  \Int[T_\mathrm{ref}]{}{\lambda}
  =
  \Int[0][1]{(1 - h)^2 / 2}{h}
  =
  \frac{1}{2} \Int[0][1]{1 - 2 h + h^2}{h}
  =
  \frac{1}{2} \pbraces{1 - 1 + \frac{1}{3}}
  =
  \frac{1}{6}.
\end{align*}

\includegraphicsboxed{Ana3/Ana3 - Satz 4.3.1 (Transformationsformel).png}

Die Transformationsformel gibt uns

\begin{align*}
  \frac{1}{6} \abs{\det B}
  =
  |T_\mathrm{ref}| \abs{\det B}
  =
  \Int[T_\mathrm{ref}]{\abs{\det D \Phi_T}}{x}
  =
  \Int[T]{}{x}
  =
  |T|
  >
  0.
\end{align*}

Also, $\abs{\det B} = 6 |T|$.
Insbesondere, ist $B^{-1}$ wohldefiniert.
Es gilt, dass

\begin{align*}
  B^{-1}
  =
  \frac{1}{\det B} \cof(B)^\top,
\end{align*}

wobei ja

\begin{align*}
  B
  =
  \begin{pmatrix}
    (z_2 - z_1)_1 & (z_3 - z_1)_1 & (z_4 - z_1)_1 \\
    (z_2 - z_1)_2 & (z_3 - z_1)_2 & (z_4 - z_1)_2 \\
    (z_2 - z_1)_3 & (z_3 - z_1)_3 & (z_4 - z_1)_3
  \end{pmatrix},
\end{align*}

und daher

\begin{multline*}
  \cof(B)^\top \\
  =
  \begin{pmatrix}
    +
    \det
    \begin{pmatrix}
      (z_3 - z_1)_2 & (z_4 - z_1)_2 \\
      (z_3 - z_1)_3 & (z_4 - z_1)_3
    \end{pmatrix}
    &
    -
    \det
    \begin{pmatrix}
      (z_2 - z_1)_2 & (z_4 - z_1)_2 \\
      (z_2 - z_1)_3 & (z_4 - z_1)_3
    \end{pmatrix}
    &
    +
    \det
    \begin{pmatrix}
      (z_2 - z_1)_2 & (z_3 - z_1)_2 \\
      (z_2 - z_1)_3 & (z_3 - z_1)_3
    \end{pmatrix}
    \\
    -
    \det
    \begin{pmatrix}
      (z_3 - z_1)_1 & (z_4 - z_1)_1 \\
      (z_3 - z_1)_3 & (z_4 - z_1)_3
    \end{pmatrix}
    &
    +
    \det
    \begin{pmatrix}
      (z_2 - z_1)_1 & (z_4 - z_1)_1 \\
      (z_2 - z_1)_3 & (z_4 - z_1)_3
    \end{pmatrix}
    &
    -
    \det
    \begin{pmatrix}
      (z_2 - z_1)_1 & (z_3 - z_1)_1 \\
      (z_2 - z_1)_3 & (z_3 - z_1)_3
    \end{pmatrix}
    \\
    +
    \det
    \begin{pmatrix}
      (z_3 - z_1)_1 & (z_4 - z_1)_1 \\
      (z_3 - z_1)_2 & (z_4 - z_1)_2
    \end{pmatrix}
    &
    -
    \det
    \begin{pmatrix}
      (z_2 - z_1)_1 & (z_4 - z_1)_1 \\
      (z_2 - z_1)_2 & (z_4 - z_1)_2
    \end{pmatrix}
    &
    +
    \det
    \begin{pmatrix}
      (z_2 - z_1)_1 & (z_3 - z_1)_1 \\
      (z_2 - z_1)_2 & (z_3 - z_1)_2
    \end{pmatrix}
  \end{pmatrix}.
\end{multline*}

Nachem $\dim \GL_3(\R) = 9 < \infty$, können wir statt der Frobeinius-Norm die äquivalente Norm $\norm{\cdot}$ verwenden.
Diese ist also nur um eine gleichmäßige Konstante $C > 0$ größer.

\begin{align*}
  \norm{B}
  :=
  \sum_{i,j=1}^3 |b_{ij}|,
  \quad
  \norm[F]{\cdot}
  \sim
  \norm{\cdot}
  \implies
  \Exists C > 0:
  \norm[F]{\cdot}
  \leq
  C \norm{\cdot}.
\end{align*}

Wir definieren die Seiten-Flächen des Tetraeders $T$.

\begin{align*}
  T_1 & := \conv \Bbraces{z_1, z_2, z_3} \\
  T_2 & := \conv \Bbraces{z_2, z_3, z_4} \\
  T_3 & := \conv \Bbraces{z_3, z_4, z_1} \\
  T_4 & := \conv \Bbraces{z_4, z_1, z_2}
\end{align*}

Das Flächenmaß $|A|$ einer Fläche $A$ ist immer kleiner gleich $|f(A)|$, dem Flächenmaß der orthogonal-projezierten Fläche $f(A)$.
Das basiert auf den folgenden beiden Resultaten aus \cite{Ana3} bzw. \cite{FAna1}.

\includegraphicsboxed{Ana3/Ana3 - Proposition 4.1.2.png}
\includegraphicsboxed{FAna1/FAna1 - orthogonale Projektion 4.png}

Damit erhalten wir folgende Abschätzungen, wenn $\pi$ eine orthogonale Projektion ist.

\begin{align*}
  |T_1| & \geq |\pi(T_1)| \geq \frac{1}{2} \det(\pi(z_3 - z_1, \pi z_4 - z_1)) \\
  |T_3| & \geq |\pi(T_3)| \geq \frac{1}{2} \det(\pi(z_2 - z_1, \pi z_4 - z_1)) \\
  |T_4| & \geq |\pi(T_4)| \geq \frac{1}{2} \det(\pi(z_2 - z_1, \pi z_3 - z_1))
\end{align*}

Wir erhalten also folgende Abschätzung.

\begin{align*}
  & \norm{\cof(B)^\top} \\
  =
  & \vbraces
  {
    \det
    \pbraces
    {
      \pi_{2, 3}(z_3 - z_1),
      \pi_{2, 3}(z_4 - z_1)
    }
  }
  +
  \vbraces
  {
    \det
    \pbraces
    {
      \pi_{2, 3}(z_2 - z_1),
      \pi_{2, 3}(z_4 - z_1)
    }
  }
  +
  \vbraces
  {
    \det
    \pbraces
    {
      \pi_{2, 3}(z_2 - z_1),
      \pi_{2, 3}(z_3 - z_1)
    }
  } \\
  +
  & \vbraces
  {
    \det
    \pbraces
    {
      \pi_{1, 3}(z_3 - z_1),
      \pi_{1, 3}(z_4 - z_1)
    }
  }
  +
  \vbraces
  {
    \det
    \pbraces
    {
      \pi_{1, 3}(z_2 - z_1),
      \pi_{1, 3}(z_4 - z_1)
    }
  }
  +
  \vbraces
  {
    \det
    \pbraces
    {
      \pi_{1, 3}(z_2 - z_1),
      \pi_{1, 3}(z_3 - z_1)
    }
  } \\
  +
  & \vbraces
  {
    \det
    \pbraces
    {
      \pi_{1, 2}(z_3 - z_1),
      \pi_{1, 2}(z_4 - z_1)
    }
  }
  +
  \vbraces
  {
    \det
    \pbraces
    {
      \pi_{1, 2}(z_2 - z_1),
      \pi_{1, 2}(z_4 - z_1)
    }
  }
  +
  \vbraces
  {
    \det
    \pbraces
    {
      \pi_{1, 2}(z_2 - z_1),
      \pi_{1, 2}(z_3 - z_1)
    }
  } \\
  & \leq
  2 |T_1| + 2 |T_3| + 2 |T_4| +
  2 |T_1| + 2 |T_3| + 2 |T_4| +
  2 |T_1| + 2 |T_3| + 2 |T_4| \\
  & \leq
  6 (|T_1| + |T_2| + |T_3| + |T_4|)
\end{align*}

Laut TUWEL-Forum, gilt folgende Formel.

\begin{align*}
  |T|
  =
  \frac{1}{3} \rho_T (|T_1| + |T_2| + |T_3| + |T_4|).
\end{align*}

Jetzt setzen wir alles zusammen.

\begin{multline*}
  \norm[F]{B^{-1}}
  \leq
  C \norm{B^{-1}}
  =
  C
  \norm
  {
    \frac{1}{\det B}
    \cof(B)^\top
  }
  =
  C
  \frac{1}{\abs{\det B}}
  \norm{\cof(B)^\top} \\
  =
  C
  \frac{1}{6 |T|}
  \norm{\cof(B)^\top}
  \leq
  C
  \frac
  {
    6 (|T_1| + |T_2| + |T_3| + |T_4|)
  }{
    6 \frac{1}{3} \rho_T (|T_1| + |T_2| + |T_3| + |T_4|)
  }
  =
  3 C \rho_T^{-1}
\end{multline*}

\end{solution}

% --------------------------------------------------------------------------------
