% --------------------------------------------------------------------------------

\begin{exercise}

Verallgemeinern Sie Theorem $3.5$ auf die Triangulierung eines beschränkten Lipschitz-Gebietes $\Omega \in \R^3$ mit Tetraedern. Beweisen Sie dazu die Abschätzungen $(3.20)$ und $(3.21)$ für ein nicht-entartetes Tetraeder $T$. Die Konstante $\sigma(T)$ wird dabei analog zu Aufgabe $11$b definiert durch den Quotienten aus dem Durchmesser $h_T := \max \Bbraces{|x-y|:x,y \in T}$ von $T$ und dem Radius $\rho_T$ der größten Kugel, welche noch ganz in $T$ liegt, d.h.

\begin{align}
  \rho_T
  :=
  \sup\Bbraces{\rho > 0: \exists x \in T~~ B(x,\rho) \subset T}.
\end{align}

\end{exercise}

% --------------------------------------------------------------------------------

\begin{solution}

ToDo!

\end{solution}

% --------------------------------------------------------------------------------
