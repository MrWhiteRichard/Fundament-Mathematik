% --------------------------------------------------------------------------------

\begin{exercise}

Sei $\hat{T}$ das Referenzdreieck mit den Eckpunkten $V_1 := (0,0), V_2 := (1,0)$ und $V_3 := (0,1)$ und den Randkanten $E_1 :=\overline{V_1V_2}, E_2 :=\overline{V_2V_3}$ und
$E_3 :=\overline{V_3V_1}$. Für $p \in \N$ definieren wir die Punkte

\begin{align}
  z_{jk}
  =
  \Big(\frac{j}{p}, \frac{k}{p} \Big) \in \hat{T},
  \quad
  j = 0, \dots, p-k,
  \quad
  k = 0, \dots, p,
\end{align}

die dazugehörigen Lagrange Basisfunktionen $L_{jk} \in P_p$ mit $L_{jk}(z_{j^\prime k^\prime}) = \delta_{jj^\prime}\delta_{kk^\prime}$ und für Funktionen $u \in C(\hat{T})$ den Interpolationsoperator $\hat{I}_p$ durch

\begin{align}
  \hat{I}_pu
  :=
  \sum_{k=0}^p
    \sum_{j=0}^{p-k} u(z_{jk})L_{jk}.
\end{align}

\begin{enumerate}[label = \textbf{\alph*)}]
  \item Skizzieren Sie für $p = 1, \cdot, 4$ die Punkte $z_{jk}$ und beweisen Sie, dass $\hat{I}_p u|_{V_j}$ und $\hat{I}_p u|_{E_j}$ für $j = 1,2,3$ und beliebige $p \in \N$ nur von den Werten $u(V_j)$ bzw. $u|_{E_j}$ abhängt.

  \item Konstruieren Sie aus $\hat{I}_p$ einen stetigen Interpolationsoperator $I_{h,p}: H^2(\Omega) \rightarrow \S_0^p(\mathcal{T}) :=$

  $\Bbraces{v_h \in C(\Omega): \forall T \in \mathcal(\T) v_h|_T \in P_p}$.
  Verwenden Sie dazu die Transformationen aus Lemma $3.9$ des Vorlesungsskriptes.

  \item Formulieren und beweisen Sie Theorem $3.5$ des Vorlesungsskriptes für $I_{h,p}$.
\end{enumerate}
\end{exercise}

% --------------------------------------------------------------------------------

\begin{solution}

ToDo!

\end{solution}

% --------------------------------------------------------------------------------
