% --------------------------------------------------------------------------------

\begin{exercise}

Sei $\hat{Q}:= (0,1) \times (0,1)$ und $\hat{T}$ das offene Dreieck mit den Eckpunkten
$(0,0),(1,0),(0,1)$. Sei weiters

\begin{align*}
  \Psi:
  \begin{cases}
    \R^2 &\rightarrow \R^2 \\
    (x,y) & \mapsto (x,(1-x)y)
  \end{cases}
\end{align*}

\begin{enumerate}[label = \textbf{\alph*)}]
  \item Zeigen Sie, dass die Abbildung $\Psi$ ein Diffeomorphismus zwischen $\hat{Q}$ und $\hat{T}$ ist.
  \item Seien für $N, M \in \N$ zwei Quadraturformeln $Q_N, Q_M$ der Ordnung $N$ bzw. $M$ auf dem
  Einheitsintervall gegeben. Konstruieren Sie daraus eine Quadraturformel $Q_{\hat{Q}}$ auf $\hat{Q}$.
  Welche Funktionen werden durch $Q_{\hat{Q}}$ exakt integriert?
  \item Verwenden Sie die Abbildung $\Psi$ und die Quadratur aus b) um eine Quadratur $Q_{\hat{T}}$ auf $\hat{T}$
  zu konstruieren. Welche Funktionen werden durch $Q_{\hat{T}}$ exakt integriert?
  \end{enumerate}
\end{exercise}

% --------------------------------------------------------------------------------

\begin{solution}

Siehe Numerik Übungsbsp.

\end{solution}

% --------------------------------------------------------------------------------
