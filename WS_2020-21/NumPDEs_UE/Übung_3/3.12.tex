% --------------------------------------------------------------------------------

\begin{exercise}

Sei $P_p:= \mathcal{L} \Bbraces{x^i y^j: i, j \geq 0 \land i+j \leq p}$ für $p \in \N$ der Raum der Polynome vom maximalen Grad $p$.

\begin{enumerate}[label = \textbf{\alph*)}]

  \item Geben Sie eine Basis von $P_0$ an.

  \item Zeigen Sie, dass die Funktionen

  \begin{align}
    \lambda_1(x, y) := 1-x-y,
    \quad
    \lambda_2(x, y) := x,
    \quad
    \lambda_3(x, y) := y
  \end{align}

  eine Basis des $P_1$ bilden.

  \item Zeigen Sie, dass die Funktionen $\lambda_1, \lambda_2, \lambda_3, \lambda_1 \lambda_2, \lambda_1 \lambda_3, \lambda_2\lambda_3$ eine Basis des $P_2$ bilden.

  \item Zeigen Sie, dass eine Basis des $P_p$ für $p \geq 3$ aus folgenden Funktionen gebildet werden kann:

  \begin{enumerate}[label = (\roman*)]
    \item $(x, y) \mapsto \lambda_j(x, y)$ mit $1 \leq j \leq 3$
    \item $(x, y) \mapsto p_{jk}(x, y)\lambda_j(x, y)\lambda_k(x, y)$ mit $1 \leq k < j \leq 3$ und
    \item $(x, y) \mapsto p_{123}(x, y) \prod_{j=1}^3 \lambda_j(x, y)$ und $p_{123} \in P_{p-3}$
  \end{enumerate}

  Die folgenden Einschränkungen der Polynome $p_{jk}$ sind dabei jeweils eindimensionale Polynome vom maximalen Grad $p-2: \xi \mapsto p_{12}(\xi, 0), \xi \mapsto p_{13}(0, \xi)$ und $\xi \mapsto p_{23}(\xi, 1-\xi)$.

  \item Erklären Sie anhand des Referenzdreiecks mit den Eckpunkten $(0, 0)$, $(1, 0)$ und $(0, 1)$ die Bedeutung dieser Aufgabe auf eine Erweiterung von Proposition $3.1$ auf Polynomräume höheren Grades.

\end{enumerate}

\end{exercise}

% --------------------------------------------------------------------------------

\begin{solution}

\begin{align*}
  \lambda_1(x, y) & = 1 - x - y \\
  \lambda_2(x, y) & = x \\
  \lambda_3(x, y) & = y \\
  (\lambda_1 \lambda_2)(x, y) & = x - x^2 - xy \\
  (\lambda_2 \lambda_3)(x, y) & = xy \\
  (\lambda_3 \lambda_1)(x, y) & = y - xy - y^2 \\
  (\lambda_1 \lambda_2 \lambda_3)(x, y) & = xy - x^2 y - x y^2
\end{align*}
Wir bezeichnen mit $B_p$ die kanonische Basis des $P_p$ und zeigen, dass
sich die gewünschten Basen durch eine Matrixtransformation der kanonischen Basis
darstellen lassen.
\begin{align*}
  B_p(x, y) := (1, x, y, x^2, xy, y^2, \dots, x^p, \dots, y^p)
\end{align*}

\begin{enumerate}[label = \textbf{\alph*)}]

  \item $\Lambda_0(x,y) := (1) = B_0(x,y)$
  \item

  \begin{align*}
    \Lambda_1
    :=
    \begin{pmatrix}
      \lambda_1 \\
      \lambda_2 \\
      \lambda_3
    \end{pmatrix}
  \end{align*}

  \begin{align*}
    \implies
    \underbrace
    {
      \begin{pmatrix}
        1 & -1 & -1 \\
          &  1 &    \\
          &    &  1
      \end{pmatrix}
    }_{
      =: T_1 \in \GL(3, \R)
    }
    B_1(x, y)
    =
    \Lambda_1(x, y)
  \end{align*}

  \item

  \begin{align*}
    \Lambda_2
    :=
    \begin{pmatrix}
      \Lambda_1 \\
      \lambda_1 \lambda_2 \\
      \lambda_2 \lambda_3 \\
      \lambda_3 \lambda_1
    \end{pmatrix}
  \end{align*}

  \begin{align*}
    \implies
    \underbrace
    {
      \pbraces
      {
        \begin{array}{ccc|ccc}
        1  &  -1   & -1   &    & & \\
          & 1 &    &    & & \\
          &     & 1   &    & & \\
          \hline
          & 1   &    & -1 & -1 & \\
          &     &    &    &  1 & \\
          &     &  1 & -1 &    & -1
        \end{array}
      }
    }_{
      =: T_2 \in \GL(6, \R)
    }
    B_2(x, y)
    =
    \Lambda_2(x, y)
  \end{align*}

  \item Wir definieren unsere Basis rekursiv.

  RA($p = 3$):

  \begin{align*}
    \Lambda_3(x, y)
    :=
    \begin{pmatrix}
      \Lambda_2(x, y) \\
      x \lambda_1(x, y) \lambda_2(x, y) \\
      x \lambda_2(x, y) \lambda_3(x, y) \\
      y \lambda_3(x, y) \lambda_1(x, y) \\
      x \lambda_3(x, y) \lambda_1(x, y)
    \end{pmatrix}
    =
    \begin{pmatrix}
      \Lambda_2 \\
      p_{12} \lambda_1 \lambda_2 \\
      p_{23} \lambda_2 \lambda_3 \\
      p_{31} \lambda_3 \lambda_1 \\
      \lambda_1 \lambda_2 \lambda_3
    \end{pmatrix}
    (x, y)
  \end{align*}

  $\Lambda_3$ ist linear unabhängig, weil $\Lambda_2$ es ist und jede Funktion $f$, ist zu $\id \cdot f$.
  Wir müssen die Transformationsmatrix also nicht einmal explizit hinschreiben. \\

  RS($p \mapsto p+1$):

  (i)-(iii) bedeuten hier, dass man $p_{12}$ mit $x$ und $p_{21}$ mit $y$ sowie $p_{23}, p_{123}$ entweder mit $x$ oder $y$ multiplizieren darf.
  Die Polynome werden dann (i)-(iii) im $p+1$ setting erfüllen.

  Seien $\Lambda_{p-1}$ und $\Lambda_p$ Basen von $P_p$ bzw. $P_{p-1} \subset P_p \subset P_{p+1}$.
  Wir ergänzen $\Lambda_p$ zu einer Basis $\Lambda_{p+1}$ von $P_{p+1}$, die (i)-(iii) für $p+1$ erfüllt.

  Die Elemente aus $\Lambda_p \setminus \Lambda_{p-1}$ (und $\Lambda_{p-1}$) haben folgende Form.

  \begin{align*}
    p_{12} \lambda_1 \lambda_2,
    \quad
    p_{23} \lambda_2 \lambda_3,
    \quad
    p_{31} \lambda_3 \lambda_1,
    \quad
    p_{123} \lambda_1 \lambda_2 \lambda_3,
    \quad
    (\text{und}~ \lambda_1, \lambda_2, \lambda_3)
  \end{align*}

  Polynome 1. und 3. Form multiplizieren wir mit $x$ bzw $y$;
  bei den anderen dürfen wir es uns jeweils aussuche, z.B. immer $x$.
  Wir erhalten $|\Lambda_p \setminus \Lambda_{p-1}| = p$ neue Polynome.
  Für das $(p+1)$-te Polynom nehmen wir die 2. oder 4. Form mal $y$.

  \begin{align*}
    \begin{Bmatrix}
      p_{12} \lambda_1 \lambda_2 \\
      p_{23} \lambda_2 \lambda_3 \\
      p_{31} \lambda_3 \lambda_1 \\
      p_{123} \lambda_1 \lambda_2 \lambda_3
    \end{Bmatrix}
    \mapsto
    \begin{Bmatrix}
      x \, p_{12} \lambda_1 \lambda_2 \\
      x \, p_{23} \lambda_2 \lambda_3 \\
      y \, p_{31} \lambda_3 \lambda_1 \\
      x \, p_{123} \lambda_1 \lambda_2 \lambda_3 \\
      y \, p_{123} \lambda_1 \lambda_2 \lambda_3
    \end{Bmatrix}
    \mapsto
    \begin{Bmatrix}
      x x \, p_{12} \lambda_1 \lambda_2 \\
      x x \, p_{23} \lambda_2 \lambda_3 \\
      y y \, p_{31} \lambda_3 \lambda_1 \\
      x x \, p_{123} \lambda_1 \lambda_2 \lambda_3 \\
      x y \, p_{123} \lambda_1 \lambda_2 \lambda_3 \\
      y y \, p_{123} \lambda_1 \lambda_2 \lambda_3
    \end{Bmatrix}
    \mapsto
    \begin{Bmatrix}
      x x x \, p_{12} \lambda_1 \lambda_2 \\
      x x x \, p_{23} \lambda_2 \lambda_3 \\
      y y y \, p_{31} \lambda_3 \lambda_1 \\
      x x x \, p_{123} \lambda_1 \lambda_2 \lambda_3 \\
      x x y \, p_{123} \lambda_1 \lambda_2 \lambda_3 \\
      x y y \, p_{123} \lambda_1 \lambda_2 \lambda_3 \\
      y y y \, p_{123} \lambda_1 \lambda_2 \lambda_3
    \end{Bmatrix}
    \mapsto \dots
  \end{align*}

  \item

  \begin{figure}[h!]
    \centering
    \includegraphics
    [width = 0.75 \textwidth]
    {NumPDEs/NumPDEs - Proposition 3.1.png}
    \caption{\cite{NumPDEs}}
  \end{figure}

  Die Basisvektoren aus $\Lambda_p$ sind nodal, d.h.

  \begin{align*}
    \Forall \lambda \in \Lambda_p,
    \Forall E \in \mathcal{E}_{T_\mathrm{ref}}:
    \lambda(E) \in \Bbraces{0, 1}.
  \end{align*}

  Das sieht man leicht durch Einsetzten der Refernez-Eckpunkte.
  Es sei dabei bemerkt, dass wir nur mit $x$ bzw. $y$ multipliziert haben.

\end{enumerate}

\end{solution}

% --------------------------------------------------------------------------------
