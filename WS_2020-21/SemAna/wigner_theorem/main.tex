\documentclass{article}

% ---------------------------------------------------------------- %
% short package descriptions are copied from
% https://ctan.org/

% ---------------------------------------------------------------- %

% Accept different input encodings
\usepackage[utf8]{inputenc}

% Standard package for selecting font encodings
\usepackage[T1]{fontenc}

% ---------------------------------------------------------------- %

% Multilingual support for Plain TEX or LATEX
\usepackage[english]{babel}

% ---------------------------------------------------------------- %

% Set all page margins to 1.5cm
\usepackage{fullpage}

% Margin adjustment and detection of odd/even pages
\usepackage{changepage}

% Flexible and complete interface to document dimensions
\usepackage{geometry}

% ---------------------------------------------------------------- %
% mathematics

\usepackage{amsmath}  % AMS mathematical facilities for LATEX
\usepackage{amssymb}
\usepackage{amsfonts} % TEX fonts from the American Mathematical Society
\usepackage{amsthm}   % Typesetting theorems (AMS style)

% Mathematical tools to use with amsmath
\usepackage{mathtools}

% Support for using RSFS fonts in maths
\usepackage{mathrsfs}

% Commands to produce dots in math that respect font size
\usepackage{mathdots}

% "Blackboard-style" cm fonts
\usepackage{bbm}

% Typeset in-line fractions in a "nice" way
\usepackage{nicefrac}

% Typeset quotient structures with LATEX
\usepackage{faktor}

% Vector arrows
\usepackage{esvect}

% St Mary Road symbols for theoretical computer science
\usepackage{stmaryrd}

% Three series of mathematical symbols
\usepackage{mathabx}

% ---------------------------------------------------------------- %
% algorithms

% Package for typesetting pseudocode
\usepackage{algpseudocode}

% Typeset source code listings using LATEX
\usepackage{listings}

% Reimplementation of and extensions to LATEX verbatim
\usepackage{verbatim}

% If necessary, please use the following 2 packages locally, but never both.
% This is because the algorithm environment gets defined in both packages, which leads to name conflicts.
% \usepackage{algorithm2e}
% \usepackage{algorithm}

% ---------------------------------------------------------------- %
% utilities

% A generic document command parser
\usepackage{xparse}

% Extended conditional commands
\usepackage{xifthen}

% e-TEX tools for LATEX
\usepackage{etoolbox}

% Define commands with suffixes
\usepackage{suffix}

% Extensive support for hypertext in LATEX
\usepackage{hyperref}

% Driver-independent color extensions for LATEX and pdfLATEX
\usepackage{xcolor}

% ---------------------------------------------------------------- %
% graphics

% -------------------------------- %

\usepackage{tikz}

% MISC
\usetikzlibrary{patterns}
\usetikzlibrary{decorations.markings}
\usetikzlibrary{positioning}
\usetikzlibrary{arrows}
\usetikzlibrary{arrows.meta}
\usetikzlibrary{overlay-beamer-styles}

% finite state machines
\usetikzlibrary{automata}

% turing machines
\usetikzlibrary{calc}
\usetikzlibrary{chains}
\usetikzlibrary{decorations.pathmorphing}

% -------------------------------- %

% Draw tree structures
\usepackage[noeepic]{qtree}

% Enhanced support for graphics
\usepackage{graphicx}

% Figures broken into subfigures
\usepackage{subfig}

% Improved interface for floating objects
\usepackage{float}

% Control float placement
\usepackage{placeins}

% Include PDF documents in LATEX
\usepackage{pdfpages}

% ---------------------------------------------------------------- %

% Control layout of itemize, enumerate, description
\usepackage[inline]{enumitem}

% Intermix single and multiple columns
\usepackage{multicol}
\setlength{\columnsep}{1cm}

% Coloured boxes, for LATEX examples and theorems, etc
\usepackage{tcolorbox}

% ---------------------------------------------------------------- %
% tables

% Tabulars with adjustable-width columns
\usepackage{tabularx}

% Tabular column heads and multilined cells
\usepackage{makecell}

% Publication quality tables in LATEX
\usepackage{booktabs}

% ---------------------------------------------------------------- %
% bibliography and quoting

% Sophisticated Bibliographies in LATEX
\usepackage[backend = biber, style = alphabetic]{biblatex}

% Context sensitive quotation facilities
\usepackage{csquotes}

% ---------------------------------------------------------------- %

% special letters:

\newcommand{\N}{\mathbb{N}}
\newcommand{\Z}{\mathbb{Z}}
\newcommand{\Q}{\mathbb{Q}}
\newcommand{\R}{\mathbb{R}}
\newcommand{\C}{\mathbb{C}}
\newcommand{\K}{\mathbb{K}}
\newcommand{\T}{\mathbb{T}}
\newcommand{\E}{\mathbb{E}}
\newcommand{\V}{\mathbb{V}}
\renewcommand{\P}{\mathbb{P}}
\newcommand{\1}{\mathbbm{1}}

\newcommand  {\B}{\mathfrak{B}}
\renewcommand{\S}{\mathfrak{S}}

% quantors:

\newcommand{\Forall}{\forall \,}
\newcommand{\Exists}{\exists \,}
\newcommand{\ExistsOnlyOne}{\exists! \,}
\newcommand{\nExists}{\nexists \,}

% MISC symbols:

\newcommand{\landau}[1]
{
  {\scriptstyle \mathcal{O}}
  \pbraces{#1}
}

\newcommand{\Landau}[1]
{
  \mathcal{O}
  \pbraces{#1}
}

\newcommand{\eps}{\mathrm{eps}}

% graphics in a box:

\newcommandtwoopt
{\includegraphicsboxed}[3][][]
{
  \begin{figure}[!h]
    \begin{boxedin}
      \ifthenelse{\isempty{#2}}
      {
        \begin{center}
          \includegraphics[width = 0.75 \textwidth]{#3}
          \label{fig:#1}
        \end{center}
      }{
        \begin{center}
          \includegraphics[width = 0.75 \textwidth]{#3}
          \caption{#2}
          \label{fig:#1}
        \end{center}
      }
    \end{boxedin}
  \end{figure}
}

% braces:

\newcommand{\pbraces}[1]{{\left  ( #1 \right  )}}
\newcommand{\bbraces}[1]{{\left  [ #1 \right  ]}}
\newcommand{\Bbraces}[1]{{\left \{ #1 \right \}}}
\newcommand{\vbraces}[1]{{\left  | #1 \right  |}}
\newcommand{\Vbraces}[1]{{\left \| #1 \right \|}}
\newcommand{\abraces}[1]{{\left \langle #1 \right \rangle}}
\newcommand{\round}[1]{\bbraces{#1}}

\newcommand
{\floor}[1]
{{\left \lfloor #1 \right \rfloor}}

\newcommand
{\ceil} [1]
{{\left \lceil  #1 \right \rceil }}

% special functions:

\newcommand{\norm}  [2][]{\Vbraces{#2}_{#1}}
\newcommand{\diag}  [1]{\mathrm{diag} \: #1}
\newcommand{\dist}  [1]{\mathrm{dist} \: #1}
\newcommand{\mean}  [1]{\mathrm{mean} \: #1}
\newcommand{\erf}   [1]{\mathrm{erf} \: #1}
\newcommand{\id}    [1]{\mathrm{id} \: #1}
\newcommand{\sgn}   [1]{\mathrm{sgn} \: #1}
\newcommand{\supp}  [1]{\mathrm{supp} \: #1}
\newcommand{\arsinh}[1]{\mathrm{arsinh} \: #1}
\newcommand{\arcosh}[1]{\mathrm{arcosh} \: #1}
\newcommand{\artanh}[1]{\mathrm{artanh} \: #1}
\newcommand{\card}  [1]{\mathrm{card} \: #1}
\newcommand{\Span}  [1]{\mathrm{span} \: #1}
\newcommand{\Aut}   [1]{\mathrm{Aut} \: #1}
\newcommand{\End}   [1]{\mathrm{End} \: #1}
\newcommand{\ggT}   [1]{\mathrm{ggT} \: #1}
\newcommand{\kgV}   [1]{\mathrm{kgV} \: #1}
\newcommand{\ord}   [1]{\mathrm{ord} \: #1}
\newcommand{\grad}  [1]{\mathrm{grad} \: #1}
\newcommand{\ran}   [1]{\mathrm{ran} \: #1}
\newcommand{\graph} [1]{\mathrm{graph} \: #1}
\newcommand{\Inv}   [1]{\mathrm{Inv} \: #1}
\newcommand{\pv}    [1]{\mathrm{pv} \: #1}
\newcommand{\Mod}{\: \mathrm{mod} \:}
\newcommand{\Char}{\mathrm{char}}
\newcommand{\At}{\mathrm{At}}
\newcommand{\Ob}{\mathrm{Ob}}
\newcommand{\Hom}{\mathrm{Hom}}
\newcommand{\orthogonal}[3][]{#2 ~\bot_{#1}~ #3}
\newcommand{\Rang}{\mathrm{Rang}}

\newcommand
{\GL}[2][]
{\mathrm{GL}_{#1} \pbraces{#2}}

% fractions:

\newcommand{\Frac}[2]{\frac{1}{#1} \pbraces{#2}}
\newcommand{\nfrac}[2]{\nicefrac{#1}{#2}}

% derivatives & integrals:

\newcommandtwoopt
{\Int}[4][][]
{\int_{#1}^{#2} #3 ~\mathrm{d} #4}

\newcommandtwoopt
{\derivative}[3][][]
{
  \frac
  {\mathrm{d}^{#1} #2}
  {\mathrm{d} #3^{#1}}
}

\newcommandtwoopt
{\pderivative}[3][][]
{
  \frac
  {\partial^{#1} #2}
  {\partial #3^{#1}}
}

\newcommand
{\primeprime}
{{\prime \prime}}

\newcommand
{\primeprimeprime}
{{\prime \prime \prime}}

% Text:

\newcommand{\Quote}[1]{\glqq #1\grqq{}}
\newcommand{\Text}[1]{{\text{#1}}}
\newcommand{\fastueberall}{\text{f.ü.}}
\newcommand{\fastsicher}{\text{f.s.}}

% ---------------------------------------------------------------- %
% amsthm-environments:

\theoremstyle{definition}

% numbered theorems
\newtheorem{theorem}             {Theorem}[section]
\newtheorem{lemma}      [theorem]{Lemma}
\newtheorem{corollary}  [theorem]{Corollary}
\newtheorem{proposition}[theorem]{Proposition}
\newtheorem{remark}     [theorem]{Remark}
\newtheorem{definition} [theorem]{Definition}
\newtheorem{example}    [theorem]{Example}
\newtheorem{heuristics} [theorem]{Heuristics}

% unnumbered theorems
\newtheorem*{theorem*}    {Theorem}
\newtheorem*{lemma*}      {Lemma}
\newtheorem*{corollary*}  {Corollary}
\newtheorem*{proposition*}{Proposition}
\newtheorem*{remark*}     {Remark}
\newtheorem*{definition*} {Definition}
\newtheorem*{example*}    {Example}
\newtheorem*{heuristics*} {Heuristics}

% ---------------------------------------------------------------- %
% exercise- and solution-environments:

% Please define this stuff in project ("main.tex"):
% \def \lastexercisenumber {...}

\newtheorem{exercise}{Exercise}
\setcounter{exercise}{\lastexercisenumber}

\newenvironment{solution}
{
  \begin{proof}[Solution]
}{
  \end{proof}
}

% ---------------------------------------------------------------- %
% MISC translations for environment-names

\renewcommand{\proofname} {Proof}
\renewcommand{\figurename}{Figure}
\renewcommand{\tablename} {Table}

% ---------------------------------------------------------------- %


\addbibresource{../../../Fundament-LaTeX/references.bib}
\addbibresource{wigner_bibliogarphy.bib}

\parskip 0pt
\parindent 0pt


\title
{
  Wigner's Theorem
}
\author
{
  Fabian Zehetgruber \\ [1cm]{\small Advisor: Prof. Michael Kaltenbäck}
}
\date{17.02.2021}

\begin{document}
	
\maketitle

\begin{abstract}
	This paper presents a detailed proof of Wigner's Theorem. The proof here was given by Daniel D. Spiegel in 2018. As this is a seminar-paper it comprises a lot of detailed calculations that are needed for the proof.
\end{abstract}

\section{Introduction}

Wigner's theorem has its motivation in physics. It plays a role in the mathematical formulation of quantum mechanics. In this paper we proof a rather general form of Wigner's theorem. As already mentioned in the abstract most of the ideas in this paper were taken from \cite{spiegel2018constructive}. The present paper is a seminar paper. Therefore, it was written with the intention of practicing the writing process and not with the intention to present new results. Nevertheless the paper might be interesting, especially for less experienced mathematicians, because everything is presented in great detail. Furthermore the paper comprises some additional ideas from \cite{Geh_r_2014} or \cite{Bargmann_1964}. 






\section{Complex numbers $\C$}

As we will have to work a lot with the complex numbers $\C$ we want to start with some of their properties.

\begin{definition}
	Let $K$ be a field and $\zeta: K \to K$ a bijective function. We call $\zeta$ an \textit{automorphism} on $K$ if for all $\lambda, \mu \in K$ the equalities
	\begin{align*}
		\zeta(\lambda + \mu) = \zeta(\lambda) + \zeta(\mu) \quad \text{and} \quad \zeta(\lambda \mu) = \zeta(\lambda) \zeta(\mu)
	\end{align*}
	hold true.
\end{definition}


\begin{definition}
	Throughout this paper $\overline{\cdot}: \C \to \C: \lambda_1 + i \lambda_2 \mapsto \lambda_1 - i \lambda_2$ will be the \textit{complex conjugation}.
\end{definition}


\begin{lemma} \label{lemma:continuous_auto}
	There exist only two continuous automorphisms on $\C$, namely the identity function and the complex conjugation. These two functions both conincide with their own inverse and act as the identity function on the real line.
\end{lemma}

\begin{proof}
	The identity function and the complex conjugation are both isometries. Thus, they are continuous and they are clearly automorphisms on $\C$. They both coincide with their own inverse and act as the identity function on the real line.
	
	For an automorphism $\zeta$ on $\C$ we have $\zeta(0) = 0$ and $\zeta(1) = 1$. Assume $\zeta(\alpha) = \alpha$ for some $\alpha \in \N$. We conclude $\zeta(\alpha + 1) = \zeta(\alpha) + \zeta(1) = \alpha + 1$. Hence, by induction we showed that for all $\alpha \in \N$ the equality $\zeta(\alpha) = \alpha$ holds true. For an arbitrary $\lambda \in \C \setminus \{0\}$ we have
	\begin{align*}
		0 = \zeta(0) = \zeta(\lambda - \lambda) = \zeta(\lambda) + \zeta(-\lambda) \quad \text{and} \quad 1 = \zeta(1) = \zeta\pbraces{\frac{\lambda}{\lambda}} = \zeta(\lambda) \zeta(\lambda^{-1}).
	\end{align*}
	Thus, $\zeta(- \lambda) = -\zeta(\lambda)$ and $\zeta\pbraces{\lambda^{-1}}= \zeta(\lambda)^{-1}$. First we conclude $\zeta(\beta) = \beta$ for all $\beta \in \Z$ and once we have this we obtain $\zeta(\gamma) = \gamma$ for all $\gamma \in \Q$. For any $\delta \in \R$ there exists a sequence of rational numbers $\pbraces{\gamma_n}_{n \in \N}$ that converges to $\delta$. Due to continuity of $\zeta$ we have 
	\begin{align*}
		\zeta(\delta) = \zeta\pbraces{\lim_{n \to \infty} \gamma_n} = \lim_{n \to \infty} \zeta(\gamma_n) = \lim_{n \to \infty} \gamma_n = \delta.
	\end{align*}
	From
	\begin{align*}
		-1 = \zeta(-1) = \zeta\pbraces{i^2} = \zeta(i)^2
	\end{align*}
	we conclude that $\zeta(i) \in \Bbraces{i, -i}$. For a complex number $\mu = \mu_1 + i \mu_2$, where $\mu_1, \mu_2 \in \R$, we obtain
	\begin{align*}
		\zeta(\mu) = \zeta\pbraces{\mu_1 + i \mu_2} = \zeta\pbraces{\mu_1} + \zeta(i) \zeta\pbraces{\mu_2} = \mu_1 + \zeta(i) \mu_2.
	\end{align*}
	Thus, $\zeta$ is either the identity function or the complex conjugation.
\end{proof}


\begin{figure}[!h]
	\centering
	
	\begin{tikzpicture}[scale=3]
	\draw[->] (-1.5cm,0cm) -- (1.5cm,0cm) node[right,fill=white] {$\Re$};
	\draw[->] (0cm,-1.5cm) -- (0cm,1.5cm) node[above,fill=white] {$\Im$};
	
	\draw[thick] (0cm,0cm) circle(1cm);
	
	\draw (0cm, 0cm) node[anchor=north west] {$0$};
	\draw (1cm, 0cm) node[anchor=north west] {$1$};
	\draw (0cm, 1cm) node[anchor=south east] {$i$};
	\draw (-1cm, 0cm) node[anchor=north west] {$-1$};
	\draw (0cm, -1cm) node[anchor=south east] {$-i$};
	
	\draw (1cm, 0cm) -- node[anchor=north east] {$\sqrt{2}$} (0cm, 1cm) ;
	\draw (1cm, 0cm) -- node[anchor=south east] {$\sqrt{2}$} (0cm, -1cm) ;
	
	\draw[line width=0.5mm, blue, dashed] (1cm, 0cm) -- node[anchor=south east] {$2$} (-1cm, 0cm);
	\end{tikzpicture}
	\caption{Geometric interpretaion of lemma \ref{lemma:complex_geom}}
\end{figure}


\begin{lemma} \label{lemma:complex_geom}
	For $\lambda \in \C$ with $|\lambda| = 1$ the following statements are true.
	\begin{enumerate}
		\item If $-\lambda$ has euclidean distance $2$ from the complex number $1$ then $\lambda = 1$. Written as a formula we have
			\begin{align*}
				|1 + \lambda| = 2 \Rightarrow \lambda = 1.
			\end{align*}
		\item If $-\lambda$ has euclidean distance $\sqrt{2}$ from the complex number $1$ then either $\lambda = i$ or $\lambda = -i$ holds true. Written as a formula we have
			\begin{align*}
				|1 + \lambda| = \sqrt{2} \Rightarrow \lambda = i \lor \lambda = -i.
			\end{align*}
	\end{enumerate}
\end{lemma}

\begin{proof}
	We will proof the two statements separately. We have
	\begin{align*}
		|1 + \lambda|^2 \lambda = (1 + \lambda) \pbraces{1 + \overline{\lambda}} \lambda = (1 + \lambda)^2 = 1 + 2 \lambda + \lambda^2.
	\end{align*}
	\begin{enumerate}
		\item If $|1 + \lambda|^2 = 4$ then we obtain
			\begin{align*}
				0 = 1 - 2 \lambda + \lambda^2 = (1 - \lambda)^2.
			\end{align*} 
		Thus, $\lambda = 1$. 
		\item If $|1 + \lambda|^2 = 2$ then we obtain 
		\begin{align*}
			0 = 1 + \lambda^2.
		\end{align*}
		Thus, $\lambda \in \Bbraces{i, -i}$.  
	\end{enumerate}
\end{proof}

\begin{lemma} \label{lemma:phase_adjustment_complex}
	Let $\mu \in \C \setminus \{0\}$. If a function $\zeta:\C \to \C$ satisfies $\vbraces{\zeta(\mu)} = \vbraces{\mu}$, then there exists a unique $\lambda \in \C$ with $|\lambda| = 1$ such that $|\mu| = \lambda \zeta(\mu)$. 
\end{lemma}

\begin{proof}
	Let $\mu \in \C \setminus \{0\}$ be an arbitrary number. Defining
	\begin{align*}
		\lambda := \frac{|\mu|}{\zeta(\mu)} \quad \text{we have} \quad |\lambda| = \frac{|\mu|}{|\zeta(\mu)|} = \frac{\vbraces{\mu}}{\vbraces{\mu}} = 1
	\end{align*}
	and
	\begin{align*}
		|\mu| = \frac{|\mu|}{\zeta(\mu)} \zeta(\mu) = \lambda \zeta(\mu).
	\end{align*}
	For another $\nu \in \C$ with $|\nu| = 1$ and $|\mu| = \nu \zeta(\mu)$ we obtain
	\begin{align*}
		\nu = \frac{|\mu|}{\zeta(\mu)} = \lambda.
	\end{align*}
\end{proof}

\begin{lemma} \label{lemma:complex_alg}
	Let $\lambda, \mu, \nu \in \C$ where $\lambda \neq 0$ and $|\mu| = |\nu|$. Then the following implication is  true. 
	\begin{align*}
		|\lambda + \nu| = |\lambda + \mu| \land |\lambda - i\nu| = |\lambda - i\mu| \Rightarrow \nu = \mu.
	\end{align*}
\end{lemma}

\begin{proof}
	We have
	\begin{align*}
		|\lambda|^2 + \lambda \overline{\mu} + \overline{\lambda} \mu + \vbraces{\mu}^2 = \pbraces{\lambda + \mu}\pbraces{\overline{\lambda} + \overline{\mu}} = \vbraces{\lambda + \mu}^2 = \vbraces{\lambda + \nu}^2 = \pbraces{\lambda + \nu}\pbraces{\overline{\lambda} + \overline{\nu}} = |\lambda|^2 + \lambda \overline{\nu} + \overline{\lambda} \nu + \vbraces{\nu}^2
	\end{align*}
	and because of $\vbraces{\mu} = \vbraces{\nu}$ we conclude $\lambda\pbraces{\overline{\nu} - \overline{\mu}} = -\overline{\lambda} \pbraces{\nu - \mu}$. Furthermore, we have
	\begin{align*}
		|\lambda|^2 + i\lambda \overline{\mu} - i\overline{\lambda} \mu + \vbraces{\mu}^2 = \pbraces{\lambda - i\mu}\pbraces{\overline{\lambda} + i\overline{\mu}} = \vbraces{\lambda - i\mu}^2 = \vbraces{\lambda - i\nu}^2 = \pbraces{\lambda - i\nu}\pbraces{\overline{\lambda} + i\overline{\nu}} = |\lambda|^2 + i\lambda \overline{\nu} - i\overline{\lambda} \nu + \vbraces{\nu}^2.
	\end{align*}
	Again because of the assumption $\vbraces{\mu} = \vbraces{\nu}$ we obtain
	\begin{align*}
		\overline{\lambda} \pbraces{\nu - \mu} = \lambda\pbraces{\overline{\nu} - \overline{\mu}} = -\overline{\lambda} \pbraces{\nu - \mu}
	\end{align*}
	and $\overline{\lambda} \neq 0$ implies $\nu = \mu$.
\end{proof}

\section{Hilbert spaces}

\begin{definition}
	Let $H$ be a vector space over $\C$. A function $(\cdot, \cdot): H \times H \to \C$ is called \textit{inner product} if 
	\begin{enumerate}
		\item $(x,x) > 0$ for all $x \in H \setminus \{0\}$.
		\item $(x,y) = \overline{(y,x)}$ for all $x,y \in H$.
		\item $(x + y, z) = (x,z) + (y,z)$ for all $x,y,z \in H$, and $(\lambda x, y) = \lambda (x,y)$ for all $\lambda \in \C$, $x,y \in H$. 
	\end{enumerate}
\end{definition}


\begin{remark}
	We know from \cite[p.41]{FAna1} that an inner product induces a norm $\norm{x} = \sqrt{\pbraces{x, x}}$. Throughout this paper a vector space $H$ provided with an inner product will always be normed with this norm.
\end{remark}


\begin{remark} \label{remark:csb}
	Let $V$ be a vector space and $(\cdot, \cdot)$ an inner product on $V$. Then for all $x,y \in V$ the inequality $\vbraces{(x,y)} \leq \norm{x} \norm{y}$ holds true. Equality holds if and only if $x$ and $y$ are linearly dependent. This inequality is called \textit{Cauchy-Schwarz inequality}. The proof can be found in \cite[p. 41]{FAna1}.
\end{remark}


\begin{remark} \label{remark:inner_product_continuity}
	For a vector space with inner product $(\cdot, \cdot): V \times V \to \C$ the inner product is continuous when $V$ is endowed with the topology induced by the norm and $V \times V$ is endowed with the product topology. Furthermore for every $y \in V$ the linear functional $f_y:V \to \C: x \mapsto (x,y)$ is continuous. The proof of these facts can be found in \cite[p.43]{FAna1} 
\end{remark}

\begin{definition}
	A vector space $H$ over $\C$ with a scalar product that is complete as a normed space endowed with the norm induced by the scalar product is called \textit{Hilbert space}.
\end{definition}

In this paper a Hilbert space is by definition a vector space over the field $\C$ and not over $\R$. 

\begin{definition}
	Let $V$ be a vector space with an inner product $(\cdot, \cdot)$. We call two subsets $M,N \subseteq V$ \textit{orthogonal}, denoted by $M \perp N$, if for all $x \in M$ and all $y \in N$ we have $(x,y) = 0$. Two vectors $v,w \in V$ are called \textit{orthogonal} if $(v,w) = 0$. 
\end{definition}

\begin{definition}
	Let $H$ be a Hilbert space. A subset $M \subseteq H$ is called an \textit{orthonormal system} if for all $u,v \in M$
	\begin{align*}
		(u,v) = 
		\begin{cases}
			1 &, \text{if } u = v, \\
			0 &, \text{if } u \neq v.
		\end{cases}
	\end{align*}
	If $M$ is an orthonormal system and every orthonormal system $\tilde{M}$  with $\tilde{M} \supseteq M$ satisfies $\tilde{M} = M$ then $M$ is called an \textit{orthonormal basis} of $H$.
\end{definition}


\begin{remark}
	Whenever we write an orthonormal system $M$ as an indexed set $M = \{e_j \mid j \in J\}$ in this paper, we require that $e_j \neq e_k$ for $j,k \in J$ with $j \neq k$.
\end{remark}


\begin{lemma}\label{lemma:onb}
	Let $H$ be a Hilbert space and $M$ an orthonormal system. Then there exists an orthonormal basis $\tilde{M} \supseteq M$. In particular, there exists an orthonormal basis of $H$. 
\end{lemma}

The proof can be found in \cite[p.52]{FAna1}.


\begin{theorem}
	Let $H$ be a Hilbert space and $M = \{e_j \mid j \in J\}$ an orthonromal system. Then the following statements are equivalent.
	\begin{enumerate}
		\item $M$ is an orthonormal basis.
		
		\item For every $x \in H$
		\begin{align}\label{eq:parzeval}
		\sum_{j \in J} \vbraces{(x, e_j)}^2 = \norm[]{x}^2.
		\end{align}
		
		\item For all $x,y \in H$ the equality
		\begin{align*}
			\sum_{j \in J} (x, e_j) \overline{(y,e_j)} = (x,y)
		\end{align*}
		holds true.
		
		\item For every $x \in H$  
		\begin{align} \label{eq:fourierseries}
		x = \sum_{j \in J} (x, e_j) e_j
		\end{align}
	\end{enumerate} 
\end{theorem}

The proof can be found in \cite[p. 54]{FAna1}.


\begin{definition}
	For a Hilbert space $H$, an orthonormal basis $M = \{e_j \mid j \in J\}$ of $H$ and $x \in H$, equality \eqref{eq:parzeval} is called \textit{Parseval's equality}. The series in \eqref{eq:fourierseries} is called \textit{Fourier series} of $x$ with respect to the orthonormal basis $M$. 
\end{definition}


\begin{lemma}
	Let $H$ be a Hilbert space and $M := \{e_j \mid j \in J\}$ be a non-empty orthonormal system. Then for every $x \in H$ we have
	\begin{align}
		\norm[]{x}^2 = \sum_{j \in J} \vbraces{(x, e_j)}^2 \Leftrightarrow x = \sum_{j \in J} (x,e_j) e_j. \label{eq:parceval_to_fourier}
	\end{align}
\end{lemma}

\begin{proof}
	We proof the two implications separately. For both directions we consider an orthonormal basis $\{f_k \mid k \in K\} \supseteq M$ that exists according to Lemma \ref{lemma:onb}. 
	\begin{enumerate}
		\item[\Quote{$\Rightarrow$}]  Using Parzeval's equality \eqref{eq:parzeval} we obtain
		\begin{align*}
			\sum_{j \in J} |(x, e_j)|^2 = \norm[]{x}^2 = \sum_{k \in K} |(x,f_k)|^2.
		\end{align*}
		Hence,  for all $k \in K$ with $f_k \notin M$ the equality $(x, f_k) = 0$ must hold true. Finally, using the representation as a Fourier series \eqref{eq:fourierseries} we obtain
		\begin{align*}
			x = \sum_{k \in K} (x,f_k) f_k = \sum_{j \in J} (x,e_j) e_j.
		\end{align*}
		
		\item[\Quote{$\Leftarrow$}] We observe that for all $k \in K$ with $f_k \notin M$ we have 
		\begin{align*}
			\pbraces{x, f_k} = \pbraces{\sum_{j \in J} \pbraces{x, e_j} e_j, f_k} = \sum_{j \in J} \pbraces{x, e_j} \pbraces{e_j, f_k} = 0.
		\end{align*}
		Hence, with Parseval's equality we obtain
		\begin{align*}
			\sum_{j \in J} \vbraces{\pbraces{x,e_j}}^2 = \sum_{k \in K} \vbraces{\pbraces{x,f_k}}^2 = \norm[]{x}^2.
		\end{align*}
	\end{enumerate}
	
\end{proof}



\begin{definition}
	Let $V$ and $W$ be two vector spaces over the same field $K$ and $\zeta$ be an automorphism on $K$. A function $f: V \to W$ is called \textit{semilinear} with respect to $\zeta$ or $\zeta$\textit{-linear}, if for all $x,y \in V$ and all $\lambda \in K$ the equations
	\begin{align*}
		f(x + y) = f(x) + f(y) \quad \text{and} \quad f(\lambda x) = \zeta(\lambda) f(x)
	\end{align*}
	are satisfied. If $K = \C$ and $\zeta$ is the complex conjugation, then $f$ is called an \textit{antilinear function}.
\end{definition}


\begin{remark}
	If $f$ is a $\zeta$-linear function and $\zeta = id_K$ then $f$ is simply a \textit{linear function}. The properties of $\zeta$-linear functions are very similar to the ones we know from linear function. See \cite[p. 138]{LinAG1&2} for these results. We will use the property that a $\zeta$-linear function $f$ is injective if $\ker f = \{0\}$. Furthermore, a scalar product in this paper is linear in the first and antilinear in the second argument, as can be found in \cite[p. 41]{FAna1}.
\end{remark}


It is not necessary to precisely define a topological vector space here. We only need to know that every normed space is a topological vector space. This result can be found in \cite[p. 18]{FAna1}


\begin{definition}
	Let $(X,\mathcal{T}_X)$ and $(Y,\mathcal{T}_Y)$ be topological vector spaces. We denote the set of all $\zeta$-linear and continuous functions from $X$ to $Y$ with $\zeta$-$L_b(X,Y)$. In the case $(X, \mathcal{T}_X) = (Y, \mathcal{T}_Y)$ we write $\zeta$-$L_b(X) = \zeta$-$L_b(X,Y)$. If $\zeta$ is the identity function then we write $L_b(X,Y)$ and $L_b(X)$.
\end{definition}


\begin{definition}
	If $(X, \mathcal{T})$ is a topological vector space over $\C$, then we denote by $(X, \mathcal{T})^\prime$ the set of all linear and continuous functions from $X$ into the field $\C$. We call this set the \textit{continuous dual space} of $(X, \mathcal{T})$.
\end{definition}

\begin{remark}
	Let $X$ be a normed space. Then $X^\prime$ provided with the operator norm 
	\begin{align*}
		\norm{f} = \sup\Bbraces{\vbraces{f(x)} : x \in X \land \norm[X]{x} \leq 1}, \quad f \in X^\prime,
	\end{align*}
	is a Banach space. See \cite[p. 25]{FAna1} for this result.
\end{remark}


\begin{proposition} \label{prop:riesz}
	Let $H$ be a Hilbert space. Then the function
	\begin{align*}
		\Phi: 
		\begin{cases}
			H \to H^\prime \\
			y \mapsto f_y
		\end{cases}
	\end{align*}
	where $f_y: H \to \C$ defined by $x \mapsto (x,y)_H$ is an isometric and antilinear bijection from $H$ onto $H^\prime$. 
\end{proposition}

The proof can be found in \cite[p. 50]{FAna1}


\begin{definition}
	Let $A$ be an algebra with an identity element $e$. This is a vector space additionally provided with a bilinear and associative multiplication $\cdot: A \times A \to A$, where $e \in A$ satisfies $ea = ae = a$ for all $a \in  A$. See \cite[p.121-122]{FAna1} for this definition and some properties of an algebra. An element $a \in A$ is called \textit{inveritble}, if there exists $b \in A$ with $ab = ba = e$. We define
	\begin{align*}
		\Inv(A) := \{a \in A \mid a \text{ is invertible}\}
	\end{align*}
	and based on this the \textit{spectrum} of an element $a \in A$ as
	\begin{align*}
		\sigma(a) = \{\lambda \in \C \mid (a - \lambda e) \notin \Inv(A)\}.
	\end{align*}
	Furthermore, we define the \textit{spectral radius} of an element $a \in A$ by
	\begin{align*}
		r(a) := \sup\{|\lambda| : \lambda \in \sigma(a)\},
	\end{align*}
	where $\sup \emptyset := 0$.
\end{definition}


\begin{remark}
	For a Banach space $X$ the space $L_b(X)$ is a Banach algebra with the identity mapping as the identity element, see \cite[p.121-122]{FAna1} for this result. 
\end{remark}


\begin{definition}
	Let $X$ be a Banach space and $T \in L_b(X)$. Then $\lambda \in \C$ is called \textit{eigenvalue} of $T$ if $\ker(T - \lambda I) \neq \{0\}$. 
\end{definition}


\begin{definition}
	Let $X, Y$ be Banach spaces. A linear function $T: X \to Y$ is called compact, if $T\pbraces{\Bbraces{x \in X: \vbraces{x} \leq 1}}$ is relatively compact in $Y$. 
\end{definition}

\begin{remark} \label{remark:compact}
	Let $X, Y$ be Banach spaces and $T \in L_b(X,Y)$ with $\dim \ran T < \infty$. Then $T$ is compact. This result can be found in \cite[p. 133]{FAna1}.
\end{remark}

\begin{remark} \label{remark:compact_spectrum}
	Let $X$ be a Banach space and $T: X \to X$ compact. Then every $\lambda \in \sigma(T) \setminus\{0\}$, where $\sigma(T)$ is the spectrum of $T$, is an eigenvalue of $T$. This result can be found in \cite[p.138]{FAna1}.
\end{remark}

\begin{lemma}
	Let $H_1$ and $H_2$ be Hilbert spaces and $\zeta$ an automorphism of $\C$ with continuous inverse $\zeta^{-1}$. If $T \in \zeta\text{-}L_b(H_1, H_2)$, then there exists a unique function $T_\zeta^\ast: H_2 \to H_1$ such that for all $x \in H_1$ and $y \in H_2$ the equation 
	\begin{align*}
		(Tx, y)_{H_2} = \zeta\pbraces{(x, T_\zeta^\ast y)_{H_1}}
	\end{align*}
	holds. The function $T_\zeta^\ast$ is $\zeta$-linear.
\end{lemma}

\begin{proof}
	For an arbitrary $y \in H_2$ we define $f_y: H_1 \to \C$ by $f_y (x) := \zeta^{-1}\pbraces{(Tx,y)_{H_2}}$. For $u,v \in H_1$ and $\lambda, \mu \in \C$ we obtain
	\begin{align*}
		f_y(\mu u + \lambda v) &= \zeta^{-1}\pbraces{(T(\mu u + \lambda v), y)_{H_2}} = \zeta^{-1}\pbraces{\zeta(\mu) (Tu, y)_{H_2} + \zeta(\lambda) (Tv, y)_{H_2}} \\
		&= \mu \zeta^{-1}\pbraces{(Tu,y)_{H_2}} + \lambda \zeta^{-1}\pbraces{(Tv,y)_{H_2}} = \mu f_y(u) + \lambda f_y(v).
	\end{align*}
	Hence, $f_y$ is a linear function. Furthermore, by Remark \ref{remark:inner_product_continuity} the function $(\cdot, \cdot)_{H_2}: H_2 \times H_2 \to \C$ is continuous. By assumption, $\zeta^{-1}$ is continuous and we conclude continuity of $f_y$. Using Proposition \ref{prop:riesz} there exists a unique $z_y \in H_1$ which fulfills $f_y(x) = (x,z_y)_{H_1}$ for all $x \in H_1$. This allows us to uniquely define a function
	\begin{align*}
		T_\zeta^\ast: H_2 \to H_1, \quad y \mapsto z_y
	\end{align*}
	that satisfies
	\begin{align*}
		(Tx,y)_{H_2} = \zeta\pbraces{\zeta^{-1}\pbraces{(Tx,y)_{H_2}}} = \zeta\pbraces{f_y(x)} = \zeta \pbraces{(x, T_\zeta^\ast y)_{H_1}}.
	\end{align*}
	for all $x \in H_1$ and all $y \in H_2$.
	
	Consider arbitrary $y,z \in H_2$ and $\lambda, \mu \in \C$. For every $x \in H_1$ we have
	\begin{align*}
		\pbraces{x, T_\zeta^\ast \pbraces{\mu y + \lambda z}}_{H_1} &= \zeta^{-1}\pbraces{\pbraces{Tx, \mu y + \lambda z}_{H_2}} = \zeta^{-1}\pbraces{\overline{\mu}} \zeta^{-1}\pbraces{\pbraces{Tx, y}_{H_2}} + \zeta^{-1}\pbraces{\overline{\lambda}} \zeta^{-1}\pbraces{\pbraces{Tx, z}_{H_2}} \\
		&= \zeta^{-1}\pbraces{\overline{\mu}} \pbraces{x, T_\zeta^\ast y}_{H_1} + \zeta^{-1}\pbraces{\overline{\lambda}} \pbraces{x, T_\zeta^\ast z}_{H_1} = \pbraces{x, \overline{\zeta^{-1}\pbraces{\overline{\mu}}}T_\zeta^\ast y + \overline{\zeta^{-1}\pbraces{\overline{\lambda}}}T_\zeta^\ast z}_{H_1}.
	\end{align*}
	We conclude $T_\zeta^\ast \pbraces{\mu y + \lambda z} = \overline{\zeta^{-1}\pbraces{\overline{\mu}}}T_\zeta^\ast y + \overline{\zeta^{-1}\pbraces{\overline{\lambda}}}T_\zeta^\ast z$. The function $\zeta^{-1}$ is a continuous automorphism on $\C$. Thus, by Lemma \ref{lemma:continuous_auto}, it is either the identity mapping or the complex conjugation. In both cases we see that $T_\zeta^\ast$ is a $\zeta$-linear function.
\end{proof}


\begin{definition}
	Let $H$ be a Hilbert space and $T\in L_b(H)$. Then $T$ is called \textit{normal} if $TT^\ast = T^\ast T$. 
\end{definition}


\begin{remark} \label{remark:spectral_radius}
	If $H$ is a Hilbert space and $N: H \to H$ is normal, then $r(N) = \norm{N}$. The proof of this statement can be found in \cite[p.142]{FAna1}.
\end{remark}


\begin{definition}
	Let $H_1$ and $H_2$ be Hilbert spaces, $\zeta$ an automorphism on $\C$ with continuous inverse and $U \in \zeta$-$L_b(H_1, H_2)$. If $\zeta$ is the identity mapping then $U$ is called \textit{unitary} and if $\zeta$ is the complex conjugation then $U$ is called \textit{antiunitary}. Note that, according to Lemma \ref{lemma:continuous_auto}, there exist only these two automorphisms on $\C$ with continuous inverse.
\end{definition}


\begin{remark}\label{remark:operator_equivalence_hilbert}
		If $H$ is a Hilbert space and if $T\in L_b(H)$ satisfies $(Tx,x)_H = 0$ for all $x \in H$, then $T = 0$. The proof of this can be found in \cite[p.142]{FAna1}.
\end{remark}


\begin{proposition} \label{prop:unitary}
	Let $H_1$ and $H_2$ be Hilbert spaces $U \in \zeta$-$L_b\pbraces{H_1, H_2}$, where $\zeta$ is an automorphism of $\C$ with continuous inverse $\zeta^{-1}$. Then the following statements are equivalent.
	\begin{enumerate}[label = (\roman*)]
		\item $U$ is $\zeta$-unitary. 
		\item $\ran U = H_2$ and $(Ux , Uy)_{H_2} = \zeta\pbraces{(x,y)_{H_1}}$ for all $x,y \in H_1$.
		\item $\ran U = H_2$ and $\norm[H_2]{Ux} = \norm[H_1]{x}$ for all $x \in H_1$. 
	\end{enumerate}
\end{proposition}

\begin{proof}
	\begin{enumerate}
		\phantom{}
		\item[]\Quote{$(\mathrm{i}) \Rightarrow \ (\mathrm{ii})$}. Due to the fact that $U U_\zeta^\ast = I_{H_2}$ we have $\ran U = H_2$. Because of the assumption $U_\zeta^\ast U = I_{H_1}$ we obtain for $x,y \in H_1$ 
		\begin{align*}
			(Ux, Uy)_{H_2} = \zeta\pbraces{(x, U_\zeta^\ast U y)_{H_1}} = \zeta \pbraces{(x,y)_{H_1}}.
		\end{align*}
		
		\item[]\Quote{$(\mathrm{ii}) \Rightarrow \ (\mathrm{iii})$}. By Lemma \ref{lemma:continuous_auto} the function $\zeta$ is either the identity function or the complex conjugation. Thus, $\zeta(\alpha) = \alpha$ for all $\alpha \in \R$. Given $x \in H_1$ we have
		\begin{align*}
			\norm[H_2]{Ux}^2 = \pbraces{Ux, Ux}_{H_2} = \zeta\pbraces{\pbraces{x, x}_{H_2}} = \zeta\pbraces{\norm[H_1]{x}^2} = \norm[H_1]{x}^2.
		\end{align*} 
		
		\item[]\Quote{$(\mathrm{iii}) \Rightarrow \ (\mathrm{i})$}. By Lemma \ref{lemma:continuous_auto} the function $\zeta^{-1}$ is either the identity function or the complex conjugation. For every $x \in H_1$ we have 
		\begin{align*}
			(x, U_\zeta^\ast U x)_{H_1} = \zeta^{-1}\pbraces{(Ux, Ux)_{H_2}} = \zeta^{-1} \pbraces{\norm[H_2]{Ux}^2} = \norm[H_2]{Ux}^2 = \norm[H_1]{x}^2 = (x,x)_{H_1}.
		\end{align*}
		Using Remark \ref{remark:operator_equivalence_hilbert} we obtain $U_\zeta^\ast U = I_{H_1}$. For $x \in H_1$ with $Ux = 0$ we derive from 
		\begin{align*}
			0 = (Ux, Ux)_{H_2} = \zeta((x,x)_{H_1})
		\end{align*}
		the equality $x = 0$. Hence, $U$ is bijective. Finally, it is a consequence of $U_\zeta^\ast U = I_{H_1}$ that
		\begin{align*}
			U U_\zeta^\ast = U U_\zeta^\ast UU^{-1} = UI_{H_1}U^{-1} = UU^{-1} = I_{H_2}.
		\end{align*}
	\end{enumerate}
\end{proof}


\begin{definition}
	Let $V$ be a vector space with an inner product $(\cdot, \cdot)$. We call a linear function $P: V \to V$ an \textit{orthogonal projection}, if $P = P^2$ and $\ran P \perp \ker P$.
\end{definition}


\begin{remark}
	In a vector space $V$ with an inner product $(\cdot, \cdot)$ a linear function $P: V \to V$ with $P^2 = P$ is an orthogonal projection if and only if for all $x,y \in V$
	\begin{align*}
		(Px, y) = (x,Py).
	\end{align*}
	This result can be found in \cite[p. 47]{FAna1}.
\end{remark}


\begin{remark} \label{remark:orth_proj_uniqueness}
	Let $H$ be a Hilbert space. If $M \subseteq H$ is a closed subspace, then there exists a unique orthogonal projection $P$ with $\ran P = M$. The proof of this statement can be found in \cite[p. 48]{FAna1}.
\end{remark}

\section{Projective Hilbert spaces}

\begin{definition}
	Let $V$ be a vector space over the field $K$. The set $\mathcal{P}(V) = \{Kx \mid x \in V \setminus \{0\}\}$ consisting of all onedimensional subspaces of $V$ is called the \textit{projective space} of $V$. If $V$ is a Hilbert space then $\mathcal{P}(V)$ is called \textit{projective Hilbert space}. We call the elements of a projective Hilbert space \textit{rays}.
\end{definition}


\begin{lemma} \label{lemma:ray_prod}
	Let $R_1$ and $R_2$ be rays of the projective Hilbert space $\mathcal{P}(H)$. Then there exists a unique $\rho \in [0, 1]$ such that for all $x_1 \in R_1 \setminus \{0\}$ and $x_2 \in R_2 \setminus \{0\}$ the equation
	\begin{align*}
		\frac{\vbraces{(x_1, x_2)_H}}{\norm[H]{x_1} \norm[H]{x_2}} = \rho
	\end{align*}
	holds.
\end{lemma}

\begin{proof}
	Let $x_1, y_1 \in R_1 \setminus \{0\}$ and $x_2, y_2 \in R_2 \setminus \{0\}$. We know that $y_1 = \lambda_1 x_1$ and $y_2 = \lambda_2 x_2$ for some $\lambda_1, \lambda_2 \in \C \setminus \{0\}$. Now we just start calculating and obtain
	\begin{align*}
		\rho := \frac{\vbraces{(y_1, y_2)_H}}{\norm[H]{y_1} \norm[H]{y_2}} = \frac{\vbraces{(\lambda_1 x_1, \lambda_2 x_2)_H}}{\norm[H]{\lambda_1 x_1} \norm[H]{\lambda_2 x_2}} = \frac{\vbraces{\lambda_1 \lambda_2}\vbraces{(x_1, x_2)_H}}{\vbraces{\lambda_1 \lambda_2}\norm[H]{x_1} \norm[H]{x_2}} = \frac{\vbraces{(x_1, x_2)_H}}{\norm[H]{x_1} \norm[H]{x_2}}.
	\end{align*}
	Because of the Cauchy-Schwarz inequality it is clear that $\rho \in [0,1]$. 
\end{proof}


\begin{definition}
	The previous lemma \ref{lemma:ray_prod} allows us to define a \textit{ray-product} on a projective Hilbert space $\mathcal{P}(H)$.
	\begin{align*}
		(\cdot, \cdot)_{\mathcal{P}(H)}: \mathcal{P}(H) \times \mathcal{P}(H) \to [0,1] : (\C x, \C y) \mapsto \frac{\vbraces{(x, y)_H}}{\norm[H]{x} \norm[H]{y}}
	\end{align*}
\end{definition}


\begin{lemma} \label{lemma:projective_metric}
	Let $\mathcal{P}(H)$ be a projective Hilbert space and
	\begin{align*}
		f: \mathcal{P}(H) \to L_b(H): R \mapsto 
		\begin{cases}
			H \to H \\
			x \mapsto (x,v_R)_H v_R
		\end{cases},
	\end{align*} 
	where $v_R \in R$ is a normalized vector. Then for all $R \in \mathcal{P}(H)$ we find that $f(R)$ is the orthogonal projection with $\ran f(R) = R$ and $d: \mathcal{P}(H) \times \mathcal{P}(H) \to [0, \infty): (R,S) \mapsto \norm{f(R) - f(S)}$ is a metric. 
\end{lemma}

\begin{proof}
	First we consider some $R \in \mathcal{P}(H)$ and define $P := f(R): H \to H: x \mapsto (x,v_R)_H v_R$. We find that for every $x \in H$
	\begin{align*}
		P^2x = P(x, v_R)_H v_R = \pbraces{(x, v_R)_H v_R, v_R}_H v_R = (x,v_R)_H v_R = Px
	\end{align*}
	and for all $x,y \in H$
	\begin{align*}
		(Px, y)_H = \pbraces{(x, v_R)_H v_R, y}_H = (x, v_R)_H (v_R, y)_H = \pbraces{x, (y, v_R)_H v_R}_H = (x, Py)_H
	\end{align*}
	and we observe that $P = f(R)$ is a linear function and hence an orthogonal projection. Due to the uniqueness of the orthogonal projection that we know from remark \ref{remark:orth_proj_uniqueness} it is clear that $d$ is a metric.
\end{proof}


\begin{remark}
	Throughout this paper a projective Hilbert space will be endowed with the metric from lemma \ref{lemma:projective_metric}. 
\end{remark}


\begin{lemma} \label{lemma:metric_representation}
	In a projective Hilbert space $\mathcal{P}(H)$ for all rays $R$ and $S$ the equality 
	\begin{align*}
		d(R,S) = \sqrt{1 - (R,S)_{\mathcal{P}(H)}^2}
	\end{align*}
	holds.
\end{lemma}

\begin{proof}
	Let $R,S \in\mathcal{P}(H)$ be arbitrary rays and $P: H \to R: x \mapsto (x, u)_H u$ as well as $Q: H \to S: x \mapsto (x, v)_H v$, where $u \in R$ and $v \in S$ are normalized vectors. If $R = S$ then, using remark \ref{remark:csb}, we easily observe that the equation holds, thus from now on we assume $R \neq S$. Now we are going to have a look at the spectrum of $T: H \to H: x  \mapsto Px - Qx$. According to lemma \ref{lemma:projective_metric} $P$ and $Q$ are orthogonal projections and we observe
	\begin{align*}
		T^\ast = P^\ast - Q^\ast = P - Q = T
	\end{align*}
	hence $T$ is normal. We also observe that $\ran T \subseteq \mathrm{span}\{u, v\} =: W$ thus $\dim \ran T < \infty$ and according to remark \ref{remark:compact}¸ this implies that $T$ is compact.
	
	Let us now assume $\lambda \in \C \setminus \{0\}$ is in the spectrum of $T$. Due to the fact that $T$ is compact we know from remark \ref{remark:compact_spectrum} that $\lambda$ is eigenvalue of $T$ which gives us $T x = \lambda x$ for some $x \in H \setminus \{0\}$. This equation lets us conclude that $x \in \mathrm{span 
	}\ T$ which means there exist $\mu, \nu \in \C$ with $x = \mu u + \nu v$. As we know that $x \neq 0$ we can also conclude that $\mu \neq 0$ or $\nu \neq 0$ and without loss of generality we assume $\mu \neq 0$. Now we start calculating.
	\begin{align*}
		\lambda \mu u + \lambda \nu v = \lambda x = Tx = Px - Qx = \mu Pu + \nu Pv - \mu Qu - \nu Qv = \mu u + \nu Pv - \mu Qu - \nu v .
	\end{align*}
	As $R \neq S$ we know that $u$ and $v$ are linearly independent and $Qu = (u,v)_H v$ as well as $Pv = (v,u)_H u$ hence the two equations
	\begin{align}
		\lambda \mu  = \mu  + \nu (v,u)_H \label{eq:metric1}\\
		 \lambda \nu  = - \nu  - \mu (u,v)_H \label{eq:metric2}
	\end{align}
	must be fulfilled. 
	
	If $(v,u)_H = 0$ then from \eqref{eq:metric1} we conclude that $\lambda \mu = \mu$ and hence $\lambda = 1$. In this case we also observe that $(R,S)_{\mathcal{P}(H)} = |(u,v)_H| = 0$ and hence according to remark \ref{remark:spectral_radius}
	\begin{align*}
		d(R,S) = \norm{P - Q} = r(P - Q) = 1 = \sqrt{1 - (R,S)_{\mathcal{P}(H)}}
	\end{align*}
	which is what we had to show.
	
	Now assuming $(v,u)_H \neq 0$ we can do further calculations. First we use \eqref{eq:metric1} and obtain
	\begin{align*}
		\lambda = \frac{\mu + \nu (v,u)_H}{\mu} = 1 + \frac{\nu}{\mu} (v,u)_H 
	\end{align*}
	which lets us conclude that 
	\begin{align}
		\frac{\nu}{\mu} = \frac{\lambda - 1}{(v,u)_H}. \label{eq:metric3}
	\end{align}
	Now, using \eqref{eq:metric2},we obtain
	\begin{align*}
		 (\lambda + 1) \frac{\nu}{\mu} =  -(u,v)_H 
	\end{align*}
	and plugging in \eqref{eq:metric3} we conclude that
	\begin{align*}
			\frac{\lambda^2 - 1}{(v,u)_H} = (\lambda + 1) \frac{\lambda - 1}{(v,u)_H} =  - (u,v)_H.
	\end{align*}
	With a simple transformation we get
	\begin{align*}
		\lambda = \pm \sqrt{1 - |(u,v)|^2} = \pm \sqrt{1 - (R,S)_{\mathcal{P}(H)}^2}.
	\end{align*}
	Although we do not know for sure whether $0$ is in the spectrum of $T$ we now know the spectral radius of $T$ which finishes the proof because using remark \ref{remark:spectral_radius} we obtain
	\begin{align*}
		d(R,S) = \norm{P - Q} = r(P - Q) = \sqrt{1 - (R,S)_{\mathcal{P}(H)}^2}.
	\end{align*}
\end{proof}


\begin{lemma}
	Let $\mathcal{P}(H_1)$ and $\mathcal{P}(H_2)$ be two projective Hilbert spaces and $g: \mathcal{P}(H_1) \to \mathcal{P}(H_2)$ an isometry. Let furthermore $M := \{e_i \mid i \in I\}$ be an orthonormal basis of $H_1$ and $x,y \in H_1 $. Let $\tilde{x} \in g(\C x)$ and $\tilde{y} \in g(\C y)$  be vectors with $\norm[H_1]{x} = \norm[H_2]{\tilde{x}}$ and $\norm[H_1]{y} = \norm[H_2]{\tilde{y}}$. Lastly, for every $i \in I$ let $\tilde{e}_i \in g(\C e_i)$ be a normalized vector. Then the  following statements are true.
	
	\begin{enumerate}
		\item The equality
		\begin{align} 
			\vbraces{(\tilde{x}, \tilde{y})_{H_1}} = \vbraces{(x, y)_{H_2}} \label{eq:vector_isometry}
		\end{align}
		holds.		
		
		\item The set $L:=\{e_i \mid i \in I\}$ is an orthonormal system of $H_2$. \label{bullet:ran_ons}
		
		\item The equality
		\begin{align}
			\tilde{x} = \sum_{i \in I} (\tilde{x}, \tilde{e}_i)_{H_2} \tilde{e}_i. \label{eq:ran_fourier}
		\end{align}
		holds.
		
	\end{enumerate} 
\end{lemma}

\begin{proof}
	We will proof the statements separately.
	\begin{enumerate}
		\item In case $x = 0$ or $y = 0$ the statement is clearly true. From now on we assume $x,y \in H_1 \setminus \{0\}$. Using lemma \ref{lemma:metric_representation} we obtain
		\begin{align*}
			\sqrt{1 - \dfrac{|(x,y)_{H_1}|^2}{\norm[H_1]{x}^2 \norm[H_1]{y}^2}} &= \sqrt{1 - (\C x,\C y)_{\mathcal{P}(H_1)}^2} = d(\C x,\C y) \\
			&= d\pbraces{g(\C x), g(\C y)} = \sqrt{1 - (g(\C x), g(\C y))_{\mathcal{P}(H_2)}^2} = \sqrt{1 - \dfrac{|(\tilde{x}, \tilde{y})_{H_2}|^2}{\norm[H_2]{\tilde{x}}^2 \norm[H_2]{\tilde{y}}^2}}
		\end{align*}
		and we immediately observe that $|(x,y)_{H_1}| = |(\tilde{x},\tilde{y})_{H_2}|$. 
		
		\item Using what we just proofed \eqref{eq:vector_isometry} and the fact that $M$ is an orthonormal basis of $H_1$ we pbtain for every $i,j \in I$ the equality
		\begin{align*}
			|(\tilde{e}_i, \tilde{e}_j)_{H_2}| = |(e_i, e_j)_{H_1}| = 
			\begin{cases}
				0 &, \text{if } i \neq j \\
				1 &, \text{if } i = j
			\end{cases}.
		\end{align*}
		Hence $L$ is an orthonormal system.
		
		\item In case $\tilde{x} = 0$ the statement is clearly true. From now on we assume $\tilde{x} \neq 0$. Using \eqref{eq:vector_isometry} and Parzeval's equality \eqref{eq:parzeval} we obtain
		\begin{align*}
			\norm[H_2]{\tilde{x}}^2 = \norm[H_1]{x}^2 = \sum_{i \in I} |(x, e_i)_{H_1}|^2 = \sum_{i \in I} |(\tilde{x}, \tilde{e}_i)_{H_2}|^2.
		\end{align*}
		Because of \ref{bullet:ran_ons} we can now use \eqref{eq:parceval_to_fourier} and immediately obtain \eqref{eq:ran_fourier}, which is just what we wanted to show.
	\end{enumerate}
\end{proof}

\section{Statement and proof of Wigner's Theorem}

\begin{example} \label{example:zerodim}
	Let $H_1$ and $H_2$ be Hilbert spaces where $\dim H_1 = 0$ and $g: \mathcal{P}(H_1) \to \mathcal{P}(H_2)$ be an isometry. As $H_1 = \{0\}$ we obtain $\mathcal{P}(H_1) = \emptyset$. Now by defining $U: H_1 \to H_2: 0 \mapsto 0$ we observe that $U$ is linear as well as antilinear and also both unitary and antiunitary. Furthermore for every $R \in \mathcal{P}(H_1)$ and every $x \in H_1$ the implication 
	\begin{align*}
		x \in R \Rightarrow Ux \in g(R)
	\end{align*} 
	is true simply because $\mathcal{P}(H_1) = \emptyset$.
\end{example}

\begin{example} \label{example:onedim}
	Let $H_1$ and $H_2$ be Hilbert sapces where $\dim H_1 = 1$ and $g: \mathcal{P}(H_1) \to \mathcal{P}(H_2)$ be an isometry. Now we have $\mathcal{P}(H_1) = \{H_1\}$ which means there exists only one ray in $\mathcal{P}(H_1)$. Now we take a normalized $x \in H_1$ and a normalized $y \in g(H_1)$ and define $U: H_1 \to H_2: \lambda x \mapsto \lambda y$ and $T: H_1 \to H_2: \lambda x \mapsto \overline{\lambda} y$. For any $\lambda, \mu \in \C$
	\begin{align*}
		(U \lambda x, U \mu x)_{H_2} = (\lambda y, \mu y)_{H_2} = \lambda \overline{\mu} \norm[H_2]{y} = \lambda \overline{\mu} = \lambda \overline{\mu} \norm[H_1]{x} = (\lambda x, \mu x)_{H_2}
	\end{align*}
	and 
	\begin{align*}
		(T\lambda x, T \mu x)_{H_2} = (\overline{\lambda} y, \overline{\mu} y)_{H_2} = \overline{\lambda} \mu \norm[H_2]{y} = \overline{\lambda} \mu = \overline{\lambda } \mu \norm[H_1]{x} = \overline{(\lambda x, \mu x)_{H_2}}.
	\end{align*}
	Hence $U$ is unitary and $T$ is antiunitary. Furthermore by definition for any ray $R \in \mathcal{P}(H)$ and any $z \in R$ we have $Uz, Tz \in g(R)$. 
\end{example}

\begin{lemma} \label{lemma:phase_adjustment_ray}
	Let $H_1$ and $H_2$ be Hilbert spaces and $g: \mathcal{P}(H_1) \to \mathcal{P}(H_2)$ be an isometry. For two vectors $x,y \in H_1$ where $(x,y)_{H_1} \neq 0$ and a normalized vector $\tilde{x} \in g(\C x)$ with $\norm[H_2]{\tilde{x}} = \norm[H_1]{x}$ there exists a unique $\tilde{y} \in g(\C y)$ that fulfills $\norm[H_2]{\tilde{y}} = \norm[H_1]{y}$ and $(v,u)_{H_2} = |(v,u)_{H_2}| $.
\end{lemma}

\begin{proof}
	We take some arbitrary vector $\tilde{w} \in g(\C y)$ with $\norm[H_2]{\tilde{w}} = \norm[H_1]{\tilde{w}}$ and define $\mu := (\tilde{w}, \tilde{x})_{H_2}$. From \eqref{eq:vector_isometry} we know that $|\mu| = |(\tilde{w},\tilde{x})_{H_2}| = |(y,x)_{H_1}| \neq 0$ and hence we know that $\mu \in \C \setminus \{0\}$. Now we know from \ref{lemma:phase_adjustment_complex} that there exists a unique $\lambda \in \C$ with $|\lambda| = 1$ such that $|\lambda \mu| = \lambda \mu$. Now we define $\tilde{y} := \lambda \tilde{w}$ and obtain
	\begin{align*}
		(\tilde{y}, \tilde{x})_{H_2} = \lambda (\tilde{w}, \tilde{x})_{H_2} = \lambda \mu = |\lambda \mu| = |\lambda (\tilde{w}, \tilde{x})_{H_2}| = |(y,x)_{H_2}|
	\end{align*}
	and $\norm[H_2]{\tilde{y}} = |\lambda| \norm[H_2]{\tilde{w}} = \norm[H_1]{y}$. 
\end{proof}


\begin{lemma} \label{lemma:aux_main}
	Let $H_1$ and $H_2$ be Hilbert spaces and $g: \mathcal{P}(H_1) \to \mathcal{P}(H_2)$ an isometry. Let $M := \{e_i \mid i \in I\}$ be an orthonormal basis and $x \in H_1 \setminus \{0\}$ with
	\begin{align*}
		x = \sum_{i \in I} \lambda_i e_i
	\end{align*} 
	and $l \in I$ with $\lambda_l \in \C \setminus \{0\}$. For every $i \in I$ let $\tilde{e}_i \in g(\C e_i)$ be a normalized vector. Then there exists a unique $\tilde{x} \in g(\C x)$ with 
	\begin{align*}
		\tilde{x} = \lambda_l \tilde{e}_l + \sum_{i \in I \setminus \{l\}} \mu_i \tilde{e}_i
	\end{align*}
	where for every $i \in I \setminus \{l\}$
	\begin{align*}
		\mu_i = (\tilde{x}, \tilde{e}_i)_{H_2} \quad \text{and} \quad |\mu_i| = |\lambda_i|.
	\end{align*}
\end{lemma}

\begin{proof}
	We first observe that for every $j \in I$ 
	\begin{align}
		\pbraces{x, e_j}_{H_1} &= \pbraces{\sum_{i \in I} \lambda_i e_i, e_j}_{H_1} = \sum_{i \in I} \lambda_i \pbraces{e_i, e_j}_{H_1} = \lambda_j \label{eq:aux_fouriercoef}
	\end{align}
	Since $\lambda_l \neq 0$ we can use \ref{lemma:phase_adjustment_ray} and we obtain a unique $\tilde{y} \in g(\C x)$ with $\norm[H_2]{\tilde{y}} = \norm[H_1]{x}$ and 
	\begin{align*}
		(\tilde{y}, \tilde{e}_l)_{H_2} = |(\tilde{y}, \tilde{e}_l)_{H_2}| = |(x, e_l)_{H_1}| = \vbraces{\lambda_l}
	\end{align*}
	whereby we also used \eqref{eq:vector_isometry}. From \ref{lemma:phase_adjustment_complex} we know there exists a normalized $\nu \in \C$ with $|\lambda_l| = |\nu \lambda_l| = \nu \lambda_l$. We define $\tilde{x} := \frac{1}{\nu} \tilde{y}$ and find
	\begin{align*}
		(\tilde{x}, \tilde{e}_l)_{H_2} = \frac{1}{\nu} (\tilde{y}, \tilde{e}_l)_{H_2} = \frac{1}{\nu} |\lambda_l| = \frac{1}{\nu} \nu \lambda_l = \lambda_l.
	\end{align*} 
	Finally, using \eqref{eq:ran_fourier}, we obtain
	\begin{align*}
		\tilde{x} = \sum_{i \in I} (\tilde{x}, \tilde{e}_i)_{H_2} \tilde{e}_i = \lambda_l \tilde{e}_l + \sum_{i \in I \setminus \{l\}} (\tilde{x}, \tilde{e}_i)_{H_2} \tilde{e}_i 
	\end{align*}
	and for every $i \in A \setminus \{l\}$ we conclude, using \eqref{eq:vector_isometry} and \eqref{eq:aux_fouriercoef}, that
	\begin{align*}
		\vbraces{(\tilde{x}, \tilde{e}_i)_{H_2}} = |(x, e_i)_{H_1}| = |\lambda_i´|.
	\end{align*}
\end{proof}


\begin{lemma} \label{lemma:function_on_onb}
	Let $H_1$ and $H_2$ be Hilbert spaces with $\dim H_1 > 1$ and $g: \mathcal{P}(H_1) \to \mathcal{P}(H_2)$ an isometry. Let furthermore $M := \{e_j \mid j \in J\}$ be a non-empty orthonormal basis of $H_1$ with $q \in J$ and some normalized $\tilde{e}_q \in g(\C e_q)$  and for all $j \in J \setminus \{q\}$ the vectors $v_{qj} := \frac{1}{\sqrt{2}} (e_q + e_j)$ and $w_{qj} := \frac{1}{\sqrt{2}}(e_q + ie_j)$ and $w_{jq} := \frac{1}{\sqrt{2}}(e_j + ie_q)$. Then for every $k \in J \setminus \{q\}$ there exists a normalized $\tilde{e}_k \in g(\C e_k)$ and a normalized $\tilde{v}_{qk} \in g(\C x_{qk})$ and a normalized $\tilde{w}_{qk} \in g(\C w_{qk})$ and a normalized $\tilde{w}_{kq} \in g(\C w_{kq})$ and $\lambda_k \in \{i, -i\}$ with
	\begin{align*}
		\tilde{v}_{qk} = \frac{1}{\sqrt{2}}(\tilde{e}_{q} + \tilde{e}_k) \quad \text{and} \quad \tilde{w}_{qk} = \frac{1}{\sqrt{2}}(\tilde{e}_q + \lambda_k \tilde{e}_k) \quad \text{and} \quad \tilde{w}_{kq} = \frac{1}{\sqrt{2}} (\tilde{e}_k + \lambda_k \tilde{e}_q).
	\end{align*}
\end{lemma}

\begin{proof}
	We observe that for every $i \in J$ and every $j \in J \setminus q$ we have
	\begin{align*}
		\pbraces{v_{qj}, e_i}_{H_1} = \pbraces{\frac{1}{\sqrt{2}}(e_q + e_j), e_i}_{H_1} =
		\begin{cases}
			\frac{1}{\sqrt{2}} &, \text{if } i \in \{q,j\} \\
			0 &, \text{else}
		\end{cases}
	\end{align*}
	Hence we can use \ref{lemma:phase_adjustment_ray} and obtain $\tilde{v}_{qj} \in g(\C v_{qj})$ with
	\begin{align*}
		(\tilde{v}_{qj}, \tilde{e}_q)_{H_2} = \vbraces{(\tilde{v}_{qj}, \tilde{e}_q)_{H_2}} = \vbraces{(v_{qj}, e_q)_{H_1}} = \frac{1}{\sqrt{2}}.
	\end{align*}
	Using \ref{lemma:phase_adjustment_ray} again, we find $\tilde{e}_{j} \in g(\C e_j)$ with
	\begin{align*}
		\pbraces{\tilde{v}_{qj}, \tilde{e}_j}_{H_2} = \vbraces{\pbraces{\tilde{v}_{qj}, \tilde{e}_j}_{H_2}} = \vbraces{\pbraces{v_{qj}, e_j}_{H_1}} = \frac{1}{\sqrt{2}}.
	\end{align*}
	For all $i \in J \setminus \{q, j\}$ we obtain
	\begin{align*}
		\vbraces{\pbraces{\tilde{v}_{q,j}, \tilde{e}_i}_{H_2}} = \vbraces{\pbraces{v_{qj}, e_i}_{H_1}} = 0
	\end{align*}
	and thus when using \eqref{eq:ran_fourier} we obtain
	\begin{align*}
		\tilde{v}_{qj} = \sum_{i \in I} (\tilde{v}_{qj}, \tilde{e}_i)_{H_2} \tilde{e}_i = \frac{1}{\sqrt{2}} (\tilde{e}_q + \tilde{e}_j).
	\end{align*}
	We choose some arbitrary $k \in J$. Using \ref{lemma:aux_main} we obtain $\tilde{w}_{qj} \in g(\C w_{qj})$ and $\tilde{w}_{jq} \in g(\C w_{jq})$ with
	\begin{align*}
		\tilde{w}_{qj} = \frac{1}{\sqrt{2}}(\tilde{e}_q + \lambda_j \tilde{e}_j) \quad \text{and} \quad \tilde{w}_{jq} = \frac{1}{\sqrt{2}}(\tilde{e}_j + \lambda_{q} \tilde{e}_q)
	\end{align*}
	and $\vbraces{\lambda_q} = \vbraces{\lambda_j} = 1$. Next we find
	\begin{align*}
		\frac{1}{\sqrt{2}}\vbraces{1 + \lambda_j} &= \vbraces{\pbraces{\frac{1}{\sqrt{2}}(\tilde{e}_q + \lambda_j \tilde{e}_j), \frac{1}{\sqrt{2}} (\tilde{e}_q + \tilde{e}_j)}_{H_2}} = \vbraces{\pbraces{\tilde{w}_{qj}, \tilde{v}_{qj}}_{H_2}} \\
		&= \vbraces{\pbraces{w_{qj}, v_{qj}}_{H_1}} = \vbraces{\pbraces{\frac{1}{\sqrt{2}}(e_q + ie_j), \frac{1}{\sqrt{2}} (e_q + e_j)}_{H_1}} = \frac{1}{\sqrt{2}} \vbraces{1 + i} = 1
	\end{align*}
	and similarly $\vbraces{1 + \lambda_q} = \sqrt{2}$ thus with \ref{lemma:complex_geom} we obtain $\lambda_j, \lambda_q \in \{i, -i\}$. Now we have a look at
	\begin{align*}
		\frac{1}{2}\vbraces{\lambda_j + \overline{\lambda_q}} &= \vbraces{\pbraces{\frac{1}{\sqrt{2}}(\tilde{e}_q + \lambda_j \tilde{e}_j), \frac{1}{\sqrt{2}}(\tilde{e}_j + \lambda_{q} \tilde{e}_q)}_{H_2}} = \vbraces{\pbraces{\tilde{w}_{qj}, \tilde{w}_{jq}}_{H_2}} \\
		&= \vbraces{\pbraces{w_{qj}, w_{jq}}_{H_1}} = \vbraces{\pbraces{\frac{1}{\sqrt{2}}(e_q + i e_j), \frac{1}{\sqrt{2}}(e_j + i e_q)}_{H_1}} = \frac{1}{2}\vbraces{i - i} = 0
	\end{align*}
	and conclude that $\lambda_j = \lambda_q$ because else we would have the contradiction $1 = 0$. 
\end{proof}


\begin{example} \label{example:twodim}
	Let $H_1$ and $H_2$ be Hilbert spaces with $\dim H_1 = 2$ and $g: \mathcal{P}(H_1) \to \mathcal{P}(H_2)$ an isometry. We consider a orthonormal basis $M = {e_1, e_2}$ of $H_1$ and define
	\begin{align*}
		v := \frac{1}{\sqrt{2}}(e_1 + e_2), \qquad w_{12} := \frac{1}{\sqrt{2}}(e_1 + i e_2), \qquad  w_{21} := \frac{1}{\sqrt{2}}(e_2 + i e_1).
	\end{align*} 
	From \ref{lemma:function_on_onb} we know that there exist $\tilde{e}_1 \in g(\C e_1)$, $\tilde{e}_2 \in g(\C e_2)$, $\tilde{v} \in g(\C v)$, $\tilde{w}_{12} \in g(\C w_{12})$, $\tilde{w}_{21} \in g(\C w_{21})$ and $\lambda \in \{i, -i\}$ with
	\begin{align*}
		\tilde{v} = \frac{1}{\sqrt{2}}(\tilde{e}_1 + \tilde{e}_2), \qquad \tilde{w}_{12} = \frac{1}{\sqrt{2}}(\tilde{e}_1 + \lambda \tilde{e}_2), \qquad \tilde{w}_{21} = \frac{1}{\sqrt{2}}(\tilde{e}_2 + \lambda \tilde{e}_1)
	\end{align*} 
	If $\lambda = i$ the we define $\zeta = \id_C$ and if $\lambda = -i$ we define $\zeta$ as the complex conjugation. Either way we hae $\lambda = \zeta(i)$.
	
	Now we are ready to define $U$. First of all $U0 := 0$. For an arbitrary $z \in H_1 \setminus \{0\}$ we know there exist $\lambda_1, \lambda_2 \in \C$ with $z = \lambda_1 e_1 + \lambda_2 e_2$ and there exists $r \in \{1,2\}$ with $\lambda_r \neq 0$. We then find $s \in \{1, 2\} \setminus \{r\}$. From \ref{lemma:aux_main} we know there exists $\tilde{z} \in g(\C z)$ with 
	\begin{align*}
		\tilde{z} = \lambda_r e_r + \nu_s e_s \quad \text{where} \quad \vbraces{\nu_s} = \vbraces{\lambda_s}.
	\end{align*}
	and we define
	\begin{align*}
		Uz := \zeta(\lambda_r) \tilde{e}_r + \zeta(\nu_s) \tilde{e}_s
	\end{align*}
	We find 
	\begin{align*}
		\frac{1}{\sqrt{2}} \vbraces{\lambda_r + \nu_s} = \frac{1}{\sqrt{2}} \vbraces{\zeta(\lambda_r + \nu_s)}= \vbraces{\pbraces{\tilde{z}, \tilde{v}}_{H_2}} = \vbraces{\pbraces{z, v}_{H_1}} = \frac{1}{\sqrt{2}} \vbraces{\lambda_r + \lambda_s}
	\end{align*}
	and for $y := \frac{1}{\sqrt{2}}(e_r + ie_s)$ and $\tilde{y} := \frac{1}{\sqrt{2}} (e_r + \zeta(i) e_s)$ we find
	\begin{align*}
		\frac{1}{\sqrt{2}}\vbraces{\lambda_r - i\nu_s} &= \frac{1}{\sqrt{2}} \vbraces{\zeta(\lambda_r) + \overline{\zeta(i)} \zeta(\nu_s)}= \vbraces{\pbraces{\tilde{z}, \tilde{y}}_{H_2}} \\
		&= \vbraces{\pbraces{z, y}_{H_1}} = \vbraces{\pbraces{\lambda_r e_r + \lambda_s e_2, \frac{1}{\sqrt{2}}(e_r + ie_s)}_{H_1}} = \frac{1}{\sqrt{2}} \vbraces{\lambda_r - i\lambda_s}
	\end{align*}
	and with \ref{lemma:complex_alg} we obtain $\lambda_s = \nu_s$. Hence we know
	\begin{align*}
		Uz = \zeta(\lambda_1) \tilde{e}_1 + \zeta(\lambda_2) \tilde{e}_2.
	\end{align*}
	
	For arbitrary $a,b \in H_2$ with $a = \lambda_1 e_1 + \lambda_2 e_2$ and $b = \mu_1 e_1 + \mu_2 e_2$ we obtain
	\begin{align*}
		\pbraces{Ua, Ub}_{H_2} = \pbraces{\zeta(\lambda_1) \tilde{e}_1 + \zeta(\lambda_2) \tilde{e}_2, \zeta(\mu_1) \tilde{e}_1 + \zeta(\mu_2) \tilde{e}_2}_{H_2} = \zeta(\lambda_1 \overline{\mu_1} + \lambda_2 \overline{\mu_2}) = \zeta\pbraces{\pbraces{a, b}_{H_1}}.
	\end{align*}
\end{example}


\begin{theorem}
	Let $H_1$ and $H_2$ be Hilbert spaces and $g: \mathcal{P}(H_1) \to \mathcal{P}(H_2)$ be an isometry. Then there exists a function $U: H_1 \to H_2$ that is either linear and fulfills $(Ux, Uy)_{H_2} = (x,y)_{H_1}$ for all $x,y \in H_1$ or antilinear and fulfills $(Ux, Uy)_{H_2} = \overline{(x,y)_{H_1}}$ for all $x,y \in H_1$. Furthermore for every ray $R \in \mathcal{P}(H_1)$ the implication
	\begin{align*}
		x \in R \Rightarrow Ux \in g(R).
	\end{align*}
	holds.
\end{theorem}

\begin{proof}
	We already showed the theorem for $\dim H_1 = 0$ in \ref{example:zerodim}, for $\dim H_1 = 1$ in \ref{example:onedim} and for $\dim H_1 = 2$ in \ref{example:twodim}, hence from now on we assume $\dim H_1 > 2$. We know from \ref{lemma:onb} that there exists an orthonormal basis $M := \{e_j \mid j \in J\}$ of $H_1$. First we choose some $q \in J$ and a normalized vector $\tilde{e}_q \in g(\C e_q)$. For all distinct $r,s \in J$ we define
	\begin{align*}
		v_{rs} := \frac{1}{\sqrt{2}}(e_r + e_s) \quad \text{and} \quad w_{rs} := \frac{1}{\sqrt{2}}(e_r + ie_s).
	\end{align*}
	We know from \ref{lemma:function_on_onb} that for every $j \in J \setminus \{q\}$ there exist $\tilde{e}_j \in g(\C e_j)$, $\tilde{v}_{qj} \in g(\C v_{qj})$, $\tilde{w}_{qj} \in g(\C w_{qj})$, $\tilde{w}_{jq} \in g(\C w_{jq})$ and $\lambda_j \in \{i, -i\}$ with
	\begin{align*}
		\tilde{v}_{qj} = \frac{1}{\sqrt{2}}(e_q + e_j), \qquad \tilde{w}_{qj} = \frac{1}{\sqrt{2}}(e_q + ie_j), \qquad \tilde{w}_{jq} = \frac{1}{\sqrt{2}}(e_j + ie_q).
	\end{align*} 
	We define $\tilde{v}_{jq} := \tilde{v}_{qj}$ and arbitrarily choose distinct $k,l \in J \setminus \{q\}$ and define
	\begin{align*}
		x_{kl} := \frac{1}{\sqrt{3}}(e_q + e_k + e_l) \quad \text{and} \quad y_{kl}:= \frac{1}{\sqrt{3}}(e_q + e_k + ie_l)
	\end{align*} 
	We can now use \ref{lemma:aux_main} and know that there exists a unique $\tilde{x}_{kl} \in g(\C x_{kl})$ and a unique $\tilde{y}_{kl} \in g(\C y_{kl})$ with 
	\begin{align*}
		\tilde{w}_{kl} = \frac{1}{\sqrt{3}} \pbraces{\tilde{e}_q + \lambda_{k} \tilde{e}_k + \lambda_{l} \tilde{e}_l} \quad \text{and} \quad \tilde{y}_{kl} = \frac{1}{\sqrt{3}}(\tilde{e}_q + \mu_k \tilde{e}_k + \mu_l \tilde{e}_l)
	\end{align*}
	and $|\lambda_k| = |\lambda_l| = \vbraces{\mu_k} = \vbraces{\mu_l} = 1$. Next, we can use \ref{bullet:ran_ons} and observe that for $j \in \{k,l\}$
	\begin{align*}
		\frac{1}{\sqrt{6}} \vbraces{1 + \lambda_j} &= \vbraces{\pbraces{\frac{1}{\sqrt{3}} \pbraces{\tilde{e}_q + \lambda_k \tilde{e}_k + \lambda_l \tilde{e}_l}, \frac{1}{\sqrt{2}} \pbraces{\tilde{e}_q + \tilde{e}_j}}_{H_2}} = |(\tilde{x}_{kl}, \tilde{v}_{qj})_{H_2}| \\
		&= |(x_{kl}, v_{qj})_{H_1}| = \vbraces{\pbraces{\frac{1}{\sqrt{3}} \pbraces{e_q + e_k + e_l}, \frac{1}{\sqrt{2}} (e_q + e_j)}_{H_1}} = \frac{2}{\sqrt{6}}.
	\end{align*}
	and 
	\begin{align*}
	\frac{1}{\sqrt{6}} \vbraces{1 + \mu_j} &= \vbraces{\pbraces{\frac{1}{\sqrt{3}} \pbraces{\tilde{e}_q + \mu_k \tilde{e}_k + \mu_l \tilde{e}_l}, \frac{1}{\sqrt{2}} \pbraces{\tilde{e}_q + \tilde{e}_j}}_{H_2}} = |(\tilde{y}_{kl}, \tilde{v}_{qj})_{H_2}| \\
	&= |(y_{kl}, v_{qj})_{H_1}| = \vbraces{\pbraces{\frac{1}{\sqrt{3}} \pbraces{e_q + e_k + ie_l}, \frac{1}{\sqrt{2}} (e_q + e_j)}_{H_1}} =
	\begin{cases}
		\frac{2}{\sqrt{6}} &, \text{if } j = k \\
		\frac{\sqrt{2}}{\sqrt{6}} &, \text{if } j = l
	\end{cases} .
	\end{align*}
	Using \ref{lemma:complex_geom} we obtain $\lambda_k = \lambda_l = \mu_k = 1$ and $\mu_l \in \{i, -i\}$, thus
	\begin{align*}
		\tilde{x}_{kl} = \frac{1}{\sqrt{3}}(e_q + e_k + e_l) \quad \text{and} \quad \tilde{y}_{kl} = \frac{1}{\sqrt{3}}(e_q + e_k + \mu_l e_l)
	\end{align*}
	
	Using \ref{lemma:aux_main} again we find $\tilde{v}_{kl} \in g(\C v_{kl})$ with
	\begin{align*}
		\tilde{v}_{kl} = \frac{1}{\sqrt{2}} \pbraces{\tilde{e}_k + \lambda_l Ue_l}.
	\end{align*}
	where $|\lambda_l|= \frac{\sqrt{2}}{\sqrt{2}} = 1$. We find
	\begin{align*}
		\frac{1}{\sqrt{6}} \vbraces{1 + \overline{\lambda_l}} &= \vbraces{\pbraces{\frac{1}{\sqrt{3}} \pbraces{\tilde{e}_q + \tilde{e}_k + \tilde{e}_l}, \frac{1}{\sqrt{2}} \pbraces{\tilde{e}_k + \lambda_l \tilde{e}_l}}_{H_2}} = |(\tilde{w}_{kl}, \tilde{v}_{kl})_{H_2}| \\
		&= |(w_{kl}, v_{kl})_{H_1}| = \vbraces{\pbraces{\frac{1}{\sqrt{3}} \pbraces{e_q + e_k + e_l}, \frac{1}{\sqrt{2}} (e_k + e_l)}_{H_1}} = \frac{2}{\sqrt{6}}
	\end{align*}
	and with \ref{lemma:complex_geom} we obtain $\lambda_l = 1$ and hence
	\begin{align*}
		\tilde{v}_{kl} = \frac{1}{\sqrt{2}}(\tilde{e}_k + \tilde{e}_l).
	\end{align*}
	
	For distinct $r,s \in J$ we define yet another vector
	\begin{align*}
		x_{rs} := \frac{1}{\sqrt{2}}(e_r + ie_s).
	\end{align*}
	We can use \ref{lemma:aux_main} and find a unique normalized vector $\tilde{x}_{rs} \in g(\C v_{rs})$ with
	\begin{align*}
		\tilde{x}_{rs} = \frac{1}{\sqrt{2}}(\tilde{e}_r + \lambda_{rs} \tilde{e}_s)
	\end{align*}
	and $|\mu_{\alpha \beta}| = \frac{\sqrt{3}}{\sqrt{3}} = 1$ only this time
	\begin{align*}
		\frac{1}{\sqrt{6}} \vbraces{1 + \overline{\mu_{\alpha \beta}}} &= \vbraces{\pbraces{\frac{1}{\sqrt{3}} \pbraces{Ue_\rho + Ue_\alpha + Ue_\beta}, \frac{1}{\sqrt{2}} \pbraces{Ue_\alpha + \mu_{\alpha \beta}Ue_\beta}}_{H_2}} = |(Uy_{\alpha \beta}, \tilde{v}_{\alpha \beta})_{H_2}| \\
		&= |(y_{\alpha \beta}, v_{\alpha \beta})_{H_1}| = \vbraces{\pbraces{\frac{1}{\sqrt{3}} \pbraces{e_\rho + e_\alpha + e_\beta}, \frac{1}{\sqrt{2}} (e_\alpha + ie_\beta)}_{H_1}} = \frac{1}{\sqrt{6}}\vbraces{1 + i} = \frac{\sqrt{2}}{\sqrt{6}}
	\end{align*}
	and with \ref{lemma:complex_geom} we find $\mu_{\alpha \beta} = i$ or $\mu_{\alpha \beta} = -i$. Having a look at
	\begin{align*}
		\frac{1}{2}\vbraces{\mu_{\alpha \beta} + \overline{\mu_{\beta \alpha}}} &= \vbraces{\pbraces{\frac{1}{\sqrt{2}}(Ue_\alpha + \mu_{\alpha \beta} Ue_\beta), \frac{1}{\sqrt{2}}(Ue_\beta + \mu_{\beta \alpha} Ue_\alpha)}} = |(\tilde{v}_{\alpha \beta}, \tilde{v}_{\beta \alpha})_{H_2}| \\
		&= |(v_{\alpha \beta}, v_{\beta \alpha})_{H_1}| = \vbraces{\pbraces{\frac{1}{\sqrt{2}}(e_\alpha + ie_\beta), \frac{1}{\sqrt{2}}(e_\beta + ie_\alpha)}_{H_1}} \\
		&= \frac{1}{2} \vbraces{ (ie_\alpha, e_\alpha)_{H_1} + (e_\beta, ie_\beta)_{H_1}} = \frac{1}{2} \vbraces{i - i} = 0
	\end{align*}
	we find $\mu_{\alpha \beta} = \mu_{\beta \alpha}$, because else we would have the contradiction $1 = 0$. We define
	\begin{align*}
		Uv_{\alpha \beta} := \tilde{v}_{\alpha \beta} = \frac{1}{\sqrt{2}}(Ue_\alpha + \mu_{\alpha \beta} Ue_\beta).
	\end{align*}
	Now we take $\gamma \in A \setminus{\alpha \beta}$ and define yet another vector
	\begin{align*}
		w_{\alpha \beta \gamma} := \frac{1}{\sqrt{3}} \pbraces{e_\alpha + ie_\beta + e_\gamma}
	\end{align*}
	and we find a unique $\tilde{w}_{\alpha \beta \gamma} \in g(\C w_{\alpha \beta \gamma})$ with
	\begin{align*}
		\tilde{w}_{\alpha \beta \gamma} = \frac{1}{\sqrt{3}} \pbraces{Ue_\alpha + \lambda_{\alpha \beta \gamma} Ue_\beta + \mu_{\alpha \beta \gamma} Ue_\gamma}.
	\end{align*}
	and $|\lambda_{\alpha \beta \gamma}| = |\mu_{\alpha \beta \gamma}| = \frac{\sqrt{3}}{\sqrt{3}} = 1$. For $\xi \in \{\beta, \gamma\}$ we have
	\begin{align*}
		\begin{rcases}
			\frac{1}{\sqrt{6}} |1 + \lambda_{\alpha \beta \gamma}| &, \text{if } \xi = \beta \\
			\frac{1}{\sqrt{6}} |1 + \mu_{\alpha \beta \gamma}| &, \text{if } \xi = \gamma
		\end{rcases}
		&= \vbraces{\pbraces{\frac{1}{\sqrt{3}} \pbraces{Ue_\alpha + \lambda_{\alpha \beta \gamma} Ue_\beta + \mu_{\alpha \beta \gamma} Ue_\gamma}, \frac{1}{\sqrt{2}} (Ue_\alpha + Ue_\xi)}_{H_2}} \\
		&= \vbraces{(\tilde{w}_{\alpha \beta \gamma}, Ux_{\alpha \xi})_{H_2}} = \vbraces{(w_{\alpha \beta \gamma}, x_{\alpha \xi})_{H_1}} \\
		&= \vbraces{\pbraces{\frac{1}{\sqrt{3}} \pbraces{e_\alpha + ie_\beta + e_\gamma}, \frac{1}{\sqrt{2}} (e_\alpha + e_\xi)}_{H_1}} = 
		\begin{cases}
			\frac{\sqrt{2}}{\sqrt{6}} &, \text{if } \xi = \beta \\
			\frac{2}{\sqrt{6}} &, \text{if } \xi = \gamma
		\end{cases}
	\end{align*}
	and thus $\mu_{\alpha \beta \gamma} = 1$ and $\lambda_{\alpha \beta \gamma} = i$ or $\lambda_{\alpha \beta \gamma} = -i$. We define
	\begin{align*}
		Uw_{\alpha \beta \gamma} := \tilde{w}_{\alpha \beta \gamma} = \frac{1}{\sqrt{3}} \pbraces{Ue_\alpha + \lambda_{\alpha \beta \gamma} Ue_\beta + Ue_\gamma}
	\end{align*}
	For $\xi \in \{\alpha, \gamma\}$ we do some further calculations and obtain
	\begin{align}
		\frac{1}{\sqrt{6}}\vbraces{\lambda_{\alpha \beta \gamma} + \overline{\mu_{\xi \beta}}}&= \vbraces{\pbraces{\frac{1}{\sqrt{3}} \pbraces{Ue_\alpha + \lambda_{\alpha \beta \gamma} Ue_\beta + Ue_\gamma}, \frac{1}{\sqrt{2}}(Ue_\xi + \mu_{\xi \beta} Ue_\beta)}} \\
		&= \vbraces{\pbraces{Uw_{\alpha \beta \gamma}, Uv_{\xi \beta}}_{H_2}} = \vbraces{\pbraces{w_{\alpha \beta \gamma}, v_{\xi \beta}}_{H_1}} \label{eq:aux_chain} \\
		&= \vbraces{\pbraces{\frac{1}{\sqrt{3}} \pbraces{e_\alpha + ie_\beta + e_\gamma}, \frac{1}{\sqrt{2}}(e_\xi + ie_\beta)}_{H_1}} = \frac{1}{\sqrt{6}} \vbraces{1 + (ie_{\beta}, ie_\beta)_{H_1}} = \frac{2}{\sqrt{6}}.
	\end{align}
	We can conclude that $\mu_{\alpha \beta} = \mu_{\gamma \beta}$ because if they were different than we would have one option to plug into \eqref{eq:aux_chain} for the contradiction $0 = \frac{2}{\sqrt{6}}$. Now, considering $\delta \in A \setminus \{\alpha, \beta, \gamma\}$ we can conclude that
	\begin{align*}
		\mu_{\alpha \beta} = \mu_{\gamma \beta} = \mu_{\beta \gamma} = \mu_{\delta \gamma} = \mu_{\gamma \delta}.
	\end{align*}
	At this point we distinguish between two cases.
	\begin{enumerate}
		\item $\mu_{\alpha \beta} = i$. Then we define $\zeta:\C \to \C: \lambda \mapsto \lambda$.
		\item $\mu_{\alpha \beta} = -i$. Then we define $\zeta:\C \to \C: \lambda \mapsto \overline{\lambda}$.
	\end{enumerate}
	
	We are ready to define $U$ on an arbitrary $x \in H_1 \setminus \{0\}$. We know that with the definition $\lambda_\xi := (x,e_\xi)_{H_1}$ for each $\xi \in A$ the equality
	\begin{align*}
		x = \sum_{\xi \in A} \lambda_\xi e_\xi
	\end{align*}
	holds. As $x \neq 0$ there exists some $\eta \in A$ with $\lambda_\eta \neq 0$ and we know from \ref{lemma:aux_main} that there exists a unique normalized $\tilde{x} \in g(\C x)$ with
	\begin{align*}
		\tilde{x} = \lambda_\eta Ue_\eta + \sum_{\xi \in A \setminus \{\eta\}} \nu_\xi Ue_\xi
	\end{align*} 
	with $|\lambda_\xi| = |\nu_\xi|$ for all $\xi \in A \setminus \{\eta\}$. We define
	\begin{align*}
		Ux := \zeta(\lambda_\eta) Ue_\eta + \sum_{\xi \in A} \zeta(\nu_\xi) Ue_\xi
	\end{align*}
	and remark that this definition is compatible with the previous definitions we made. Now we consider some $\delta \in A \setminus \{\eta\}$ and calculate
	\begin{align*}
		\frac{1}{\sqrt{2}}\vbraces{\lambda_\eta + \nu_\delta} &= \frac{1}{\sqrt{2}} \vbraces{\zeta(\lambda_\eta) + \zeta(\nu_\delta)} \\
		&= \vbraces{\pbraces{\zeta(\lambda_\eta) Ue_\eta + \sum_{\xi \in A} \zeta(\nu_\xi) Ue_\xi, \frac{1}{\sqrt{2}}(Ue_\eta + Ue_\delta)}_{H_2}} = |(Ux, Ux_{\eta \delta})_{H_2}| \\
		&= \vbraces{\pbraces{x, x_{\eta \delta}}_{H_1}} = \vbraces{\pbraces{\sum_{\xi \in A} \lambda_\xi e_\xi, \frac{1}{\sqrt{2}}(e_\eta + e_\delta)}_{H_1}} = \frac{1}{\sqrt{2}} \vbraces{\lambda_\eta + \lambda_\delta} 
	\end{align*}
	and furthermore
	\begin{align*}
		\frac{1}{\sqrt{2}}\vbraces{\lambda_\eta -i \nu_\delta} &= \frac{1}{\sqrt{2}} \vbraces{\zeta(\lambda_\eta) + \overline{\zeta(i)} \zeta(\nu_\delta)} \\
		&= \vbraces{\pbraces{\zeta(\lambda_\eta) Ue_\eta + \sum_{\xi \in A} \zeta(\nu_\xi) Ue_\xi, \frac{1}{\sqrt{2}}(Ue_\eta + \zeta(i) Ue_\delta)}_{H_2}} = |(Ux, Uv_{\eta \delta})_{H_2}| \\
		&= \vbraces{\pbraces{x, v_{\eta \delta}}_{H_1}} = \vbraces{\pbraces{\sum_{\xi \in A} \lambda_\xi e_\xi, \frac{1}{\sqrt{2}}(e_\eta + ie_\delta)}_{H_1}} = \frac{1}{\sqrt{2}} \vbraces{\lambda_\eta -i \lambda_\delta}.
	\end{align*}
	Now we can use \ref{lemma:complex_alg} and obtain $\nu_\delta = \lambda_\delta$. Now we observe that 
	\begin{align*}
		U(x) = \sum_{\xi \in A} \zeta(\lambda_\xi) Ue_\xi = \sum_{\xi \in A} \zeta\pbraces{(x, e_\xi)_{H_1}} Ue_\xi.
	\end{align*}
	Hence for arbitrary $v,w \in H_1$ we obtain
	\begin{align*}
		\pbraces{Uv, Uw}_{H_2} &= \pbraces{\sum_{\xi \in A} \zeta((v, e_\xi)_{H_2}) Ue_\xi, \sum_{\eta \in A} \zeta((w, e_\eta)_{H_1}) Ue_\eta}_{H_2} = \sum_{\xi \in A} \zeta\pbraces{(v, e_\xi)_{H_1} \overline{(w,e_\xi)}_{H_1}} \\
		&= \zeta \pbraces{\sum_{\xi \in A} (v, e_\xi)_{H_1} \overline{(w, e_\xi)_{H_1}}} = \zeta\pbraces{\pbraces{\sum_{\xi \in A} (v, e_\xi)_{H_1} e_\xi, \sum_{\eta \in A} (w, e_\eta)_{H_1} e_\eta}_{H_1}} = \zeta\pbraces{\pbraces{v, w}_{H_1}}.
	\end{align*}
\end{proof}

\section{Concluding remarks}
The proof given here is not particularly short and it involves quite a few calculations. Despite these drawbacks the proof given has the merit that it proves a very general form of Wigner's theorem where the two Hilbert spaces involved can be different ones and do not have to be separable. Furthermore we constructed the desired function step by step in the proof which might be very insightful and we did not have to use very deep mathematical results. Finally it is worth mentioning that the paper does give a lot of detailed calculations which should make it easy to read.

\section{Acknowledgments}
I want to thank my advisor Professor Michael Kaltenbäck for his support and for correcting the paper. 

\printbibliography

\end{document}
