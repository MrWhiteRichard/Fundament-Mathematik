\documentclass{article}

% Hier befinden sich Pakete, die wir beinahe immer benutzen ...

\usepackage[utf8]{inputenc}

% Sprach-Paket:
\usepackage[english]{babel}

% damit's nicht so, wie beim Grill aussieht:
\usepackage{fullpage}
\usepackage{changepage}

% Mathematik:
\usepackage{amsmath, amssymb, amsfonts, amsthm}
\usepackage{bbm, mathrsfs, stmaryrd}
\usepackage{mathtools, mathdots}

% Doppel-Klammern:
\usepackage{stmaryrd}

% Makros mit mehereren Default-Argumenten:
\usepackage{twoopt}

% Anführungszeichen (Makro \Quote{}):
\usepackage{babel}

% if's für Makros:
\usepackage{xifthen}
\usepackage{etoolbox}

% *'s für Makros:
\usepackage{suffix}

% tikz ist kein Zeichenprogramm (doch!):
\usepackage{tikz}
\usetikzlibrary{arrows}
\usetikzlibrary{decorations.markings}

% bessere Aufzählungen:
\usepackage{enumitem}

% (bessere) Umgebung für Bilder:
\usepackage{graphicx, subfig, float}
\usepackage{tcolorbox}

% Umgebung für Code:
\usepackage{listings}

% Farben:
\usepackage{xcolor}

% Umgebung für "plain text":
\usepackage{verbatim}

% Umgebung für mehrerer Spalten:
\usepackage{multicol}
\setlength{\columnsep}{1cm}

% diagonale Brüche
\usepackage{nicefrac}

% faktorisieren
\usepackage{faktor}

% Spaltentypen verschiedener Dicke
\usepackage{tabularx}
\usepackage{makecell}

% Für Vektoren
\usepackage{esvect}

% (Web-)Links
\usepackage{hyperref}

% Zitieren & Literatur-Verzeichnis
\usepackage[
backend=biber,
style=numeric
]{biblatex}
\usepackage{csquotes}

% so ähnlich wie mathbb
%\usepackage{mathds}

% Keine Ahnung, was das macht ...
\usepackage{booktabs}
\usepackage{ngerman}
\usepackage{placeins}

% Bäume
\usepackage[noeepic]{qtree}

% Algorithmen 
\usepackage{algorithm}
\usepackage{algpseudocode}

% special letters:

\newcommand{\N}{\mathbb{N}}
\newcommand{\Z}{\mathbb{Z}}
\newcommand{\Q}{\mathbb{Q}}
\newcommand{\R}{\mathbb{R}}
\newcommand{\C}{\mathbb{C}}
\newcommand{\K}{\mathbb{K}}
\newcommand{\T}{\mathbb{T}}
\newcommand{\E}{\mathbb{E}}
\newcommand{\V}{\mathbb{V}}
\renewcommand{\S}{\mathbb{S}}
\renewcommand{\P}{\mathbb{P}}
\newcommand{\1}{\mathbbm{1}}

% quantors:

\newcommand{\Forall}{\forall \,}
\newcommand{\Exists}{\exists \,}
\newcommand{\ExistsOnlyOne}{\exists! \,}
\newcommand{\nExists}{\nexists \,}
\newcommand{\ForAlmostAll}{\forall^\infty \,}

% MISC symbols:

\newcommand{\landau}{{\scriptstyle \mathcal{O}}}
\newcommand{\Landau}{\mathcal{O}}


\newcommand{\eps}{\mathrm{eps}}

% graphics in a box:

\newcommandtwoopt
{\includegraphicsboxed}[3][][]
{
  \begin{figure}[!h]
    \begin{boxedin}
      \ifthenelse{\isempty{#1}}
      {
        \begin{center}
          \includegraphics[width = 0.75 \textwidth]{#3}
          \label{fig:#2}
        \end{center}
      }{
        \begin{center}
          \includegraphics[width = 0.75 \textwidth]{#3}
          \caption{#1}
          \label{fig:#2}
        \end{center}
      }
    \end{boxedin}
  \end{figure}
}

% braces:

\newcommand{\pbraces}[1]{{\left  ( #1 \right  )}}
\newcommand{\bbraces}[1]{{\left  [ #1 \right  ]}}
\newcommand{\Bbraces}[1]{{\left \{ #1 \right \}}}
\newcommand{\vbraces}[1]{{\left  | #1 \right  |}}
\newcommand{\Vbraces}[1]{{\left \| #1 \right \|}}
\newcommand{\abraces}[1]{{\left \langle #1 \right \rangle}}
\newcommand{\round}[1]{\bbraces{#1}}

\newcommand
{\floorbraces}[1]
{{\left \lfloor #1 \right \rfloor}}

\newcommand
{\ceilbraces} [1]
{{\left \lceil  #1 \right \rceil }}

% special functions:

\newcommand{\norm}  [2][]{\Vbraces{#2}_{#1}}
\newcommand{\diam}  [2][]{\mathrm{diam}_{#1} \: #2}
\newcommand{\diag}  [1]{\mathrm{diag} \: #1}
\newcommand{\dist}  [1]{\mathrm{dist} \: #1}
\newcommand{\mean}  [1]{\mathrm{mean} \: #1}
\newcommand{\erf}   [1]{\mathrm{erf} \: #1}
\newcommand{\id}    [1]{\mathrm{id} \: #1}
\newcommand{\sgn}   [1]{\mathrm{sgn} \: #1}
\newcommand{\supp}  [1]{\mathrm{supp} \: #1}
\newcommand{\arsinh}[1]{\mathrm{arsinh} \: #1}
\newcommand{\arcosh}[1]{\mathrm{arcosh} \: #1}
\newcommand{\artanh}[1]{\mathrm{artanh} \: #1}
\newcommand{\card}  [1]{\mathrm{card} \: #1}
\newcommand{\Span}  [1]{\mathrm{span} \: #1}
\newcommand{\Aut}   [1]{\mathrm{Aut} \: #1}
\newcommand{\End}   [1]{\mathrm{End} \: #1}
\newcommand{\ggT}   [1]{\mathrm{ggT} \: #1}
\newcommand{\kgV}   [1]{\mathrm{kgV} \: #1}
\newcommand{\ord}   [1]{\mathrm{ord} \: #1}
\newcommand{\grad}  [1]{\mathrm{grad} \: #1}
\newcommand{\ran}   [1]{\mathrm{ran} \: #1}
\newcommand{\graph} [1]{\mathrm{graph} \: #1}
\newcommand{\Inv}   [1]{\mathrm{Inv} \: #1}
\newcommand{\pv}    [1]{\mathrm{pv} \: #1}
\newcommand{\GL}    [1]{\mathrm{GL} \: #1}
\newcommand{\Mod}{\mathrm{Mod} \:}
\newcommand{\Th}{\mathrm{Th} \:}
\newcommand{\Char}{\mathrm{char}}
\newcommand{\At}{\mathrm{At}}
\newcommand{\Ob}{\mathrm{Ob}}
\newcommand{\Hom}{\mathrm{Hom}}
\newcommand{\orthogonal}[3][]{#2 ~\bot_{#1}~ #3}
\newcommand{\Rang}{\mathrm{Rang}}
\newcommand{\NIL}{\mathrm{NIL}}
\newcommand{\Res}{\mathrm{Res}}
\newcommand{\lxor}{\dot \lor}
\newcommand{\Div}{\mathrm{div} \:}
\newcommand{\meas}{\mathrm{meas} \:}

% fractions:

\newcommand{\Frac}[2]{\frac{1}{#1} \pbraces{#2}}
\newcommand{\nfrac}[2]{\nicefrac{#1}{#2}}

% derivatives & integrals:

\newcommandtwoopt
{\Int}[4][][]
{\int_{#1}^{#2} #3 ~\mathrm{d} #4}

\newcommandtwoopt
{\derivative}[3][][]
{
  \frac
  {\mathrm{d}^{#1} #2}
  {\mathrm{d} #3^{#1}}
}

\newcommandtwoopt
{\pderivative}[3][][]
{
  \frac
  {\partial^{#1} #2}
  {\partial #3^{#1}}
}

\newcommand
{\primeprime}
{{\prime \prime}}

\newcommand
{\primeprimeprime}
{{\prime \prime \prime}}

% Text:

\newcommand{\Quote}[1]{\glqq #1\grqq{}}
\newcommand{\Text}[1]{{\text{#1}}}
\newcommand{\fastueberall}{\text{f.ü.}}
\newcommand{\fastsicher}{\text{f.s.}}

% -------------------------------- %
% amsthm-stuff:

\theoremstyle{definition}

% numbered theorems
\newtheorem{theorem}{Theorem}
\newtheorem{lemma}{Lemma}
\newtheorem{corollary}{Corollary}
\newtheorem{proposition}{Proposition}
\newtheorem{remark}{Remark}
\newtheorem{definition}{Definition}
\newtheorem{example}{Example}

% unnumbered theorems
\newtheorem*{theorem*}{Theorem}
\newtheorem*{lemma*}{Lemma}
\newtheorem*{corollary*}{corollary}
\newtheorem*{proposition*}{Proposition}
\newtheorem*{remark*}{Remark}
\newtheorem*{definition*}{Definition}
\newtheorem*{example*}{Example}

% Please define this stuff in project ("main.tex"):

% \def \lastexercisenumber {...}
% This will be 0 by default

% \setcounter{section}{...}
% This will be 0 by default
% and hence, completely ignored

\ifnum \thesection = 0
{
  \newtheorem{exercise}{exercise}
}
\else
{
  \newtheorem{exercise}{exercise}[section]
}
\fi

\ifdef
{\lastexercisenumber}
{\setcounter{exercise}{\lastexercisenumber}}

\newenvironment{solution}
{
  \begin{proof}[solution]
}{
  \end{proof}
}

\renewcommand{\proofname}{Proof}

% -------------------------------- %
% environment zum einkasteln:

% dickere vertical lines
\newcolumntype
{x}
[1]
{
  !{
    \centering
    \arraybackslash
    \vrule
    width #1}
}

% environment selbst (the big cheese)
\newenvironment
{boxedin}
{
  \begin{tabular}
  {
    x{1 pt}
    p{\textwidth}
    x{1 pt}
  }
  \Xhline
  {2 \arrayrulewidth}
}
{
  \\
  \Xhline{2 \arrayrulewidth}
  \end{tabular}
}

% -------------------------------- %



\addbibresource{../../../Fundament-LaTeX/references.bib}
\addbibresource{wigner_bibliogarphy.bib}

\parindent 0pt

\title
{
  Wigner's Theorem
}
\author
{
  Fabian Zehetgruber \\ [1cm]{\small Advisor: Prof. Michael Kaltenbäck}
}
\date{01.01.2021}

\begin{document}
	
\maketitle

\begin{abstract}
	This paper presents a detailed proof of Wigner's Theorem. The proof given is not new but is the same as the proof that was given by Daniel D. Spiegel in 2018. As this is a seminar-paper it comprises a lot of detailed calculations that are needed for the proof.
\end{abstract}

\section{Introduction}

Wigner's theorem has it's motivation in physics. It plays a role in the mathematical formulation of quantum mechanics. In this paper we proof, in comparison to other papers I found, a rather general form of Wigner's theorem. As already mentioned in the abstract, most of the ideas in this paper are from \cite{spiegel2018constructive} and there are no completely new ideas. This can be put down to the fact that the present paper is a seminar paper, which implies that this paper was written with the intention of practicing the writing process and not so much with the intention to present new results. Nevertheless the paper might be interesting especially for less experienced mathematicians, because everything is presented in great detail. Furthermore the paper comprises some additional ideas from \cite{Geh_r_2014} or \cite{Bargmann_1964}. 






\section{Complex numbers $\C$}

As we will have to work a lot with the complex numbers $\C$ we want to start with some properties of them. 

\begin{definition}
	Let $K$ be a field and $\zeta: K \to K$ a bijective function. We call $\zeta$ an \textit{automorphism} of $K$ if for all $\lambda, \mu \in K$ the equalities
	\begin{enumerate}
		\item $\zeta(\lambda + \mu) = \zeta(\lambda) + \zeta(\mu)$
		\item $\zeta(\lambda \mu) = \zeta(\lambda) \zeta(\mu)$
	\end{enumerate}
	are true.
	
\end{definition}

\begin{definition}
	Throughout this paper $\overline{\cdot}: \C \to \C: \lambda_1 + i \lambda_2 \mapsto \lambda_1 - i \lambda_2$ will be the \textit{complex conjugation}.
\end{definition}

\begin{remark}
	The complex conjugation is a continuous automorphism of $\C$ and is it's own inverse. Some of this can be found in \cite[p.40]{LinAG1&2}.
\end{remark}

\begin{lemma} \label{lemma:complex_geom}
	Let $\lambda \in \C$ and $|\lambda| = 1$. Then the following statements are true.
	\begin{enumerate}
		\item $|1 + \lambda| = 2 \Rightarrow \lambda = 1$
		\item $|1 + \lambda| = \sqrt{2} \Rightarrow \lambda = i \lor \lambda = -i$
	\end{enumerate}
\end{lemma}

\begin{proof}
	We will proof the two statements separately. We always have $\lambda = \mu + i \nu$ with $\mu, \nu \in \R$. We start calculating and obtain
	\begin{align*}
		|1 + \lambda|^2 = |(1 + \mu) + i \nu|^2 = (1 + \mu)^2 + \nu^2 = 1 + 2\mu + \mu^2 + \nu^2 = 1 + 2\mu + |\lambda|^2 = 2 + 2 \mu.
	\end{align*}
	\begin{enumerate}
		\item We know that $4 = |1 + \lambda|^2 = 2 + 2\mu$ and this implies $\mu = 1$ and using this knowledge we obtain
		\begin{align*}
			1 + \nu^2 = \mu^2 + \nu^2 = |\lambda|^2 = 1
		\end{align*}
		which lets us conclude $\nu^2 = 0$ and hence $\nu = 0$ thus $\lambda = \mu + i \nu = 1$. 
		
		\item Here we know that $2 = |1 + \lambda|^2 = 2 + 2\mu$ and hence $\mu = 0$. Using this knowledge we observe
		\begin{align*}
			\nu^2 = \mu^2 + \nu^2 = |\lambda|^2 = 1
		\end{align*}
		and thus $\nu = 1 \lor \nu = -1$. This implies $\lambda = \mu + i \nu = i \lor \lambda = -i$.  
	\end{enumerate}
\end{proof}

\begin{lemma} \label{lemma:phase_adjustment_complex}
	For any complex number $\mu \in \C \setminus \{0\}$ there exists a unique $\lambda \in \C$ with $|\lambda| = 1$ such that $|\lambda \mu| = \lambda \mu$. 
\end{lemma}

\begin{proof}
	Let $\mu \in \C \setminus \{0\}$ be an arbitrary number. We define $\lambda := \frac{|\mu|}{\mu}$ which implies $|\lambda| = \frac{|\mu|}{|\mu|} = 1$ and obtain
	\begin{align*}
		|\lambda \mu| = |\lambda| |\mu| = |\mu| = \frac{|\mu|}{\mu} \mu = \lambda \mu.
	\end{align*}
	For another $\nu \in \C$ with $|\nu| = 1$ and $|\nu \mu| = \nu \mu$ we obtain
	\begin{align*}
		\nu = \frac{|\nu \mu|}{\mu} = \frac{|\nu| |\mu|}{\mu} = \frac{|\mu|}{\mu} = \lambda,
	\end{align*}
	hence $\lambda$ is unique. 
\end{proof}

\begin{lemma} \label{lemma:complex_alg}
	Let $\lambda, \mu, \nu \in \C$ where $\lambda \neq 0$ and $|\mu| = |\nu|$. Then the following implication is  true. 
	\begin{align*}
		|\lambda + \nu| = |\lambda + \mu| \land |\lambda - i\nu| = |\lambda - i\mu| \Rightarrow \nu = \mu.
	\end{align*}
\end{lemma}

\begin{proof}
	We write $\lambda = \lambda_1 + i \lambda_2$, $\mu = \mu_1 + i \mu_2$ and $\nu = \nu_1 + i \nu_2$ with $\lambda_1, \lambda_2, \mu_1, \mu_2, \nu_1, \nu_2 \in \R$ and start calculating.
	\begin{align*}
		|\lambda|^2 + |\nu|^2 + 2 \lambda_1 \nu_1 + 2 \lambda_2 \nu_2  &= (\lambda_1 + \nu_1)^2 + (\lambda_2 + \nu_2)^2 = |\lambda + \nu|^2 \\
		&= |\lambda + \mu|^2 = |\lambda|^2 + |\mu|^2 + 2 \lambda_1 \mu_1 + 2 \lambda_2 \mu_2
	\end{align*}
	thus $\lambda_1 \nu_1 + \lambda_2 \nu_2 - \lambda_1 \mu_1 - \lambda_2 \nu_2 = 0$ and
	\begin{align*}
		|\lambda|^2 + |\nu|^2 + 2 \lambda_1 \nu_2 - 2 \lambda_2 \nu_1  &= (\lambda_1 + \nu_2)^2 + (\lambda_2 - \nu_1)^2 = |\lambda - i\nu|^2 \\
		&= |\lambda - i\mu|^2 = |\lambda|^2 + |\mu|^2 + 2 \lambda_1 \mu_2 - 2 \lambda_2 \mu_1
	\end{align*}
	thus $\lambda_1 \nu_2 - \lambda_2 \nu_1 - \lambda_1 \mu_2 + \lambda_2 \mu_1 = 0$.
	Now we find that
	\begin{align*}
		\overline{\lambda}(\nu - \mu) &= (\lambda_1 - i\lambda_2)(\nu_1 + i \nu_2 - \mu_1 - i\mu_2) \\
		&= \lambda_1\nu_1 + i\lambda_1 \nu_2 - \lambda_1\mu_1 - i\lambda_1\mu_2 -i\lambda_2\nu_1 + \lambda_2\nu_2 + i\lambda_2\mu_1 - \lambda_2\mu_2 \\
		&= (\lambda_1 \nu_1 + \lambda_2\nu_2 - \lambda_1\mu_1 - \lambda_2\mu_2) + i(\lambda_1 \nu_2 - \lambda_2\nu_1 -\lambda_1\mu_2 + \lambda_2\mu_1) = 0
	\end{align*}
	and because $\overline{\lambda} \neq 0$ we know that $\nu = \mu$.
\end{proof}

\section{Hilbert spaces}

\begin{definition}
	Let $H$ be a Vector space over $\C$. A function $(\cdot, \cdot): H \times H \to \C$ is called \textit{inner product} if 
	\begin{enumerate}
		\item $(x,x) > 0$ for all $x \in H \setminus \{0\}$.
		\item $(x,y) = \overline{(y,x)}$ for all $x,y \in H$.
		\item $(x + y, z) = (x,z) + (y,z)$ for all $x,y,z \in H$, and $(\lambda x, y) = \lambda (x,y)$ for all $\lambda \in \C$, $x,y \in H$. 
	\end{enumerate}
\end{definition}


\begin{remark}
	We know from \cite[p.41]{FAna1} that an inner product induces a norm. Throughout this paper a vector space $H$ with an inner product will always be a normed spaces with this norm.
\end{remark}


\begin{remark} \label{remark:csb}
	Let $V$ be a vector space and $(\cdot, \cdot)_H$ an inner product on $V$. Then for all $x,y \in V$ the inequality $\vbraces{(x,y)} \leq \norm{x} \norm{y}$ holds. Equality holds if and only if $x$ and $y$ are linearly independent. The inequality is called \textit{Cauchy-Schwarz inequality}. The proof can be found in \cite[p. 41]{FAna1}.
\end{remark}


\begin{remark}
	For a vector space with inner product $(\cdot, \cdot): V \times V \to \C$ the inner product is continuous when $V \times V$ in endowed with the product topology. Furthermore for every $y \in V$ we know that $f_y:V \to \C: x \mapsto (x,y)$ is continuous. The proof of these facts can be found in \cite[p.43]{FAna1} 
\end{remark}

\begin{definition}
	A vector space $H$ over $\C$ with a scalar product that is complete as a normed space endowed with the norm induced by the scalar product is called \textit{Hilbert space}.
\end{definition}

In this paper we are only going to consider Hilbert spaces over the field $\C$ and not over $\R$. 

\begin{definition}
	Let $V$ be a vector space with an inner product $(\cdot, \cdot)$. We call two subsets $M,N \subseteq V$ \textit{orthogonal}, denoted $M \perp N$, if for all $x \in M$ and all $y \in N$ we have $(x,y) = 0$. Two vectors $v,w \in V$ are called \textit{orthogonal} if $(v,w) = 0$. 
\end{definition}

\begin{definition}
	Let $H$ be a Hilbert space. A subset $M \subseteq H$ is called an \textit{orthonormal system} if for all $u,v \in M$
	\begin{align*}
		(u,v) = 
		\begin{cases}
			1 &, \text{if } u = v \\
			0 &, \text{if } u \neq v
		\end{cases}.
	\end{align*}
	If every orthonormal system $\tilde{M}$ fulfills
	\begin{align*}
		\tilde{M} \supseteq M \Rightarrow \tilde{M} = M
	\end{align*}
	then $M$ is called an \textit{orthonormal basis}.
\end{definition}

\begin{remark}
	Whenever we denote an orthonormal system $M = \{e_i \mid i \in I\}$ in this paper we require that for all $j,k \in M$ the implication
	\begin{align*}
		j \neq k \Rightarrow e_j \neq e_k
	\end{align*}
	holds.
\end{remark}

\begin{lemma}\label{lemma:onb}
	Let $H$ be a Hilbert space and $M$ an orthonormal system. Then there exists an orthonormal basis $\tilde{M} \supseteq M$. Particularly there exists an orthonormal basis of $H$. 
\end{lemma}

\begin{proof}
	The proof can be found in \cite[p.52]{FAna1}.
\end{proof}

\begin{theorem}
	Let $H$ be a Hilbert space and $M = \{e_i \mid i \in I\}$ an orthonromal system. Then the following statements are equivalent.
	\begin{enumerate}
		\item $M$ is an orthonormal basis.
		
		\item For every $x \in H$
		\begin{align}\label{eq:parzeval}
		\sum_{i \in I} \vbraces{(x, e_i)_H}^2 = \norm[H]{x}^2.
		\end{align}
		
		\item For all $x,y \in H$ the equality
		\begin{align*}
			\sum_{i \in I} (x, e_i)_H \overline{(y,e_i)_H} = (x,y)_H
		\end{align*}
		holds.
		
		\item For every $x \in H$  
		\begin{align} \label{eq:fourierseries}
		x = \sum_{i \in I} (x, e_i)_H e_i
		\end{align}
	\end{enumerate} 
\end{theorem}


\begin{proof}
	The proof can be found in \cite[p. 54]{FAna1}.
\end{proof}


\begin{definition}
	For a Hilbert space $H$ and an orthonormal basis $M = \{e_i \mid i \in I\}$ and an $x \in H$ the equality \eqref{eq:parzeval} is called \textit{Parseval's equality}. The series in \eqref{eq:fourierseries} is called \textit{fourierseries} of $x$ with regard to the orthonormal basis $M$. 
\end{definition}

\begin{lemma}
	Let $H$ be a Hilbert space and $M := \{e_i \mid i \in I\}$ be a non-empty orthonormal system. Then for every $x \in H$ the equivalence
	\begin{align}
		\norm[H]{x}^2 = \sum_{i \in I} \vbraces{(x, e_i)_H}^2 \Leftrightarrow x = \sum_{i \in I} (x,e_i) e_i \label{eq:parceval_to_fourier}
	\end{align}
	holds.
\end{lemma}

\begin{proof}
	We proof the two implications separately. For both directions we consider an orhtonormal basis $\{f_j \mid j \in J\} \supseteq M$ that exists according to lemma \ref{lemma:onb}. 
	\begin{enumerate}
		\item[\Quote{$\Rightarrow$}]  Using Parzeval's equality \eqref{eq:parzeval} we obtain
		\begin{align*}
		\sum_{i \in I} |(x, e_i)_H|^2 = \norm[H]{x}^2 = \sum_{j \in J} |(x,f_j)_H|^2
		\end{align*}
		and hence for all $j \in J$ where there exists no $i \in I$ with $e_i = f_j$ the equality $(x, f_j)_H = 0$ must be true. Finally, using the representation as a Fourier series \eqref{eq:fourierseries} we obtain
		\begin{align*}
		x = \sum_{j \in J} (x,f_j)_H f_j = \sum_{i \in I} (x,e_i)_H e_i.
		\end{align*}
		
		\item[\Quote{$\Leftarrow$}] This time we immediately observe that for all $j \in J$ with $f_j \notin M$ we have 
		\begin{align*}
			\pbraces{x, f_j}_H = \pbraces{\sum_{i \in I} \pbraces{x, e_i}_H e_i, f_j}_H = \sum_{i \in I} \pbraces{x, e_i}_H \pbraces{e_i, f_j}_H = 0
		\end{align*}
		and hence with Parseval's equality
		\begin{align*}
			\sum_{i \in I} \vbraces{\pbraces{x,e_i}_H}^2 = \sum_{j \in J} \vbraces{\pbraces{x,f_j}_H}^2 = \norm[H]{x}^2.
		\end{align*}
	\end{enumerate}
	
\end{proof}



\begin{definition}
	Let $V$ and $W$ be two vector spaces over the same field $K$ and $\zeta$ an automorphism of $K$. A function $f: V \to W$ is called \textit{semilinear} with regard to $\zeta$, if all $x,y \in V$ and all $\lambda \in K$ fulfill
	\begin{enumerate}
		\item $f(x + y) = f(x) + f(y)$
		\item $f(\lambda x) = \zeta(\lambda) f(x)$.
	\end{enumerate}
	We also call $f$ a $\zeta$\textit{-linear function}. If $\zeta = id_K$ then $f$ is simply a \textit{linear function} and if $K = \C$ and $\zeta$ is the complex conjugation then $f$ is called an \textit{antilinear function}.
\end{definition}


\begin{remark}
	The properties of $\zeta$-linear functions are very similar to the ones we know from linear function. See \cite[p. 138]{LinAG1&2} for these results. We will use the property that a $\zeta$-linear function $f$ is injective if $\ker f = \{0\}$.
\end{remark}


\begin{definition}
	Let $(X,\mathcal{T}_X)$ and $(Y,\mathcal{T}_Y)$ be topological vector spaces. We denote the set of all $\zeta$-linear and continuous functions from $X$ to $Y$ with $\zeta$-$L_b(X,Y)$. If $(X, \mathcal{T}_X) = (Y, \mathcal{T}_Y)$ then we write $L_b(X) = L_b(X,Y)$. If $\zeta$ is the identity function then we write $L_b(X,Y)$ and $L_b(X)$.
\end{definition}


\begin{remark}
	The scalar products in this paper are linear in the first and antilinear in the second argument. This result can be found in \cite[p. 41]{FAna1}.
\end{remark}

\begin{definition}
	Let $(X, \mathcal{T})$ be a topological vector space over $\C$. Then we denote $(X, \mathcal{T})^\prime$ for the set of all linear and continuous functions from $X$ into the field $\C$. We call this set the \textit{continuous dual space} of $(X, \mathcal{T})$.
\end{definition}

\begin{remark}
	It is not necessary to precisely define a topological vector space here. We only need to know that every normed space is a topological vector space. This result can be found in \cite[p. 18]{FAna1}
\end{remark}

\begin{remark}
	Let $X$ be a normed space. Then $X^\prime$ with the operator norm is a Banach space. See \cite[p. 25]{FAna1} for this result.
\end{remark}

\begin{proposition} \label{prop:riesz}
	Let $H$ be a Hilbert space. Then the function
	\begin{align*}
		\Phi: 
		\begin{cases}
			H \to H^\prime \\
			y \mapsto f_y
		\end{cases}
	\end{align*}
	where $f_y: H \to \C: x \mapsto (x,y)_H$ is an isometric and antilinear bijection from $H$ to $H^\prime$. 
\end{proposition}

\begin{proof}
	The proof can be found in \cite[p. 50]{FAna1}
\end{proof}

\begin{definition}
	Let $A$ be an algebra with an identity element $e$. An element $a \in A$ is called \textit{inveritble}, if there exists $b \in A$ with $ab = ba = e$. We define
	\begin{align*}
		\Inv(A) := \{a \in A \mid a \text{ is invertible}\}
	\end{align*}
	and based on this we define the \textit{spectrum} of an element $a \in A$ as
	\begin{align*}
		\sigma(a) = \{\lambda \in \C \mid (a - \lambda e) \notin \Inv(A)\}.
	\end{align*}
\end{definition}

\begin{remark}
	A precise definition of an Algebra is not required in this paper. It can be found in \cite[p. 122]{FAna1}. We only need to know that for a Banach space $X$ the space $L_b(X)$ is a Banach algebra with identity element, thus for every $T \in L_b(X)$ the spectrum $\sigma(T)$ is defined. See \cite[p.121-122]{FAna1} for this result. 
\end{remark}

\begin{definition}
	Let $X$ be a Banach space and $T \in L_b(X)$. Then $\lambda \in \C$ is called \textit{eigenvalue} of $T$ if $\ker(T - \lambda I) \neq \{0\}$. 
\end{definition}

\begin{definition}
	Let $A$ be a Banach algebra with an identity element and $a \in A$. Then we define the \textit{spectral radius}
	\begin{align*}
		r(a) := \max\{|\lambda| : \lambda \in \sigma(a)\}
	\end{align*}
	where $\sigma(a)$ is the spectrum of $a$.
\end{definition}

\begin{definition}
	Let $X, Y$ be Banach spaces and $K := \{x \in X: |x| \leq 1\}$. Then a linear function $T: X \to Y$ is called compact, if $T(K)$ is relatively compact in $Y$. 
\end{definition}

\begin{remark} \label{remark:compact}
	Let $X, Y$ be Banach spaces and $T:X \to Y$ a linear and continuous function with $\dim \ran T < \infty$. Then $T$ is compact. This result can be found in \cite[p. 133]{FAna1}.
\end{remark}

\begin{remark} \label{remark:compact_spectrum}
	Let $X$ be a Banach space and $T: X \to X$ compact. Then every $\lambda \in \sigma(T) \setminus\{0\}$, where $\sigma(T)$ is the spectrum of $T$, is an eigenvalue of $T$. This result can be found in \cite[p.138]{FAna1}.
\end{remark}

\begin{lemma}
	Let $H_1$ and $H_2$ be Hilbert spaces and $\zeta$ an automorphism of $\C$ with continuous inverse $\zeta^{-1}$. Let furthermore $T \in \zeta\text{-}L_b(H_1, H_2)$. Then there exists a unique function $T_\zeta^\ast: H_2 \to H_1$ such that for all $x \in H_1$ and $y \in H_2$ the equation 
	\begin{align*}
		(Tx, y)_{H_2} = \zeta\pbraces{(x, T_\zeta^\ast y)_{H_1}}
	\end{align*}
	holds. 
\end{lemma}

\begin{proof}
	For an arbitrary $y \in H_2$ we define $f_y: H_1 \to \C: x \mapsto \zeta^{-1}\pbraces{(Tx,y)_{H_2}}$. For arbitrary $u,v \in H_1$ and arbitrary $\lambda \in \C$ we obtain
	\begin{align*}
		f_y(u + \lambda v) &= \zeta^{-1}\pbraces{(T(u + \lambda v), y)_{H_2}} = \zeta^{-1}\pbraces{(Tu, y)_{H_2} + \zeta(\lambda) (Tv, y)_{H_2}} \\
		&= \zeta^{-1}\pbraces{(Tu,y)_{H_2}} + \lambda \zeta^{-1}\pbraces{(Tv,y)_{H_2}} = f_y(u) + \lambda f_y(v),
	\end{align*}
	hence $f_y$ is a linear function. Furthermore, from Fana corollary 3.1.4 we know that $(\cdot, \cdot)_{H_2}: H_2 \times H_2 \to \C$ is continuous and we have $\zeta^{-1}$ continuous hence $f_y$ is continuous. Now, using proposition \ref{prop:riesz}, we know that there exists a unique $z_y \in H_1$ which fulfills $f_y(x) = (x,z_y)_{H_1}$ for all $x \in H_1$. This allows us to uniquely define a function
	\begin{align*}
		T_\zeta^\ast: H_2 \to H_1: y \mapsto z_y
	\end{align*}
	that fulfills for all $x \in H_1$ and all $y \in H_2$ the equalities
	\begin{align*}
		(Tx,y)_{H_2} = \zeta\pbraces{\zeta^{-1}\pbraces{(Tx,y)_{H_2}}} = \zeta\pbraces{f_y(x)} = \zeta \pbraces{(x, T_\zeta^\ast y)_{H_1}}.
	\end{align*}
\end{proof}


\begin{remark}
	We certainly know that the identity function and the complex conjugation are automorphisms of $\C$ with continuous inverse. The question how many other function of this kind exist is not answered here.
\end{remark}


\begin{definition}
	Let $H$ be a Hilbert space and $T\in L_b(H)$. Then $T$ is called \textit{normal} if $TT^\ast = T^\ast T$. 
\end{definition}


\begin{definition}
	Let $H_1$ and $H_2$ be Hilbert spaces, $\zeta$ an automorphism of $\C$ with continuous inverse and $U \in \zeta$-$L_b(H_1, H_2)$. Then $U$ is called $\zeta$-unitary, if $U_\zeta^\ast U = I_{H_1}$ and $U U_\zeta^\ast = I_{H_2}$. If $\zeta$ is the identity mapping then $U$ is called \textit{unitary} and if $\zeta$ is the complex conjugation then $U$ is called \textit{antiunitary}.
\end{definition}


\begin{remark}\label{remark:operator_equivalence_hilbert}
		Let $H$ be a Hilbert space and $T\in L_b(H)$ with $(Tx,x)_H = 0$ for all $x \in H$. Then $T = 0$. The proof of this can be found in \cite[p.142]{FAna1}.
\end{remark}


\begin{proposition} \label{prop:unitary}
	Let $H_1$ and $H_2$ be Hilbert spaces $U \in \zeta$-$L_b{H_1, H_2}$, where $\zeta$ is an automorphism of $\C$ with continuous inverse $\zeta^{-1}$. Then the following statements are equivalent.
	\begin{enumerate}[label = (\roman*)]
		\item $U$ is $\zeta$-unitary. 
		\item $\ran U = H_2$ and $(Ux , Uy)_{H_2} = \zeta\pbraces{(x,y)_{H_1}}$ for all $x,y \in H_1$.
		\item $\ran U = H_2$ and $\norm[H_2]{Ux} = \norm[H_1]{x}$ for all $x \in H_1$. 
	\end{enumerate}
\end{proposition}

\begin{proof}
	We will split the proof in two implications.
	\begin{enumerate}
		\item[\Quote{$(\mathrm{i}) \Rightarrow \ (\mathrm{ii})$}] Due to the fact that $U U_\zeta^\ast = I_{H_2}$ we know that $\ran U = H_2$ and using $U_\zeta^\ast U = I_{H_1}$, we obtain for every $x,y \in H_1$ 
		\begin{align*}
			(Ux, Uy)_{H_2} = \zeta\pbraces{(x, U_\zeta^\ast U y)_{H_1}} = \zeta \pbraces{(x,y)_{H_1}}
		\end{align*}
		which is exactly what we wanted to show.
		
		\item[\Quote{$(\mathrm{ii}) \Rightarrow \ (\mathrm{iii})$}] This implication is trivially true.
		
		\item[\Quote{$(\mathrm{iii}) \Rightarrow \ (\mathrm{i})$}] Now we have for every $x \in H_1$ that 
		\begin{align*}
			(x, U_\zeta^\ast U x)_{H_1} = \zeta^{-1}\pbraces{(Ux, Ux)_{H_2}} = \zeta^{-1}\pbraces{\zeta\pbraces{(x,x)_{H_1}}} = (x,x)_{H_1}.
		\end{align*}
		Using remark \ref{remark:operator_equivalence_hilbert} we now know that $U_\zeta^\ast U = I_{H_1}$. For $x \in H_1$ and $Ux = 0$ we know that 
		\begin{align*}
			0 = (Ux, Ux)_{H_2} = \zeta((x,x)_{H_1})
		\end{align*}
		and hence $x = 0$. We conclude that $U$ is bijective and from $U_\zeta^\ast U = I_{H_1}$ we obtain
		\begin{align*}
			U U_\zeta^\ast = U U_\zeta^\ast UU^{-1} = UI_{H_1}U^{-1} = UU^{-1} = I_{H_2}.
		\end{align*}
	\end{enumerate}
\end{proof}


\begin{remark} \label{remark:spectral_radius}
	Let $H$ be a Hilbert space and $N: H \to H$ normal. Then $r(N) = \norm{N}$ where $r$ is the spectral radius. The proof of this statement can be found in \cite[p.142]{FAna1}.
\end{remark}


\begin{definition}
	Let $V$ be a vector space with an inner product $(\cdot, \cdot)$. We call a linear function $P: V \to V$ \textit{orthogonal projection}, if
	\begin{enumerate}
		\item $P = P^2$
		\item $\ran P \perp \ker P$
	\end{enumerate}
\end{definition}


\begin{remark}
	In a vector space $V$ with an inner product $(\cdot, \cdot)$ a linear function $P: V \to V$ with $P^2 = P$ is an orthogonal projection if and only if for all $x,y \in V$ the equality
	\begin{align*}
		(Px, y) = (x,Py)
	\end{align*}
	holds. This result can be found in \cite[p. 47]{FAna1}.
\end{remark}


\begin{remark} \label{remark:orth_proj_uniqueness}
	Let $H$ be a Hilbert space and $M \subseteq H$ a closed subspace. Then there exists a unique orthogonal projection $P$ with $\ran P = M$. The proof of this statement can be found in \cite[p. 48]{FAna1}.
\end{remark}

\section{Projective Hilbert spaces}

\begin{definition}
	Let $V$ be a vector space over the field $K$. The set $\mathcal{P}(V) = \{Kx \mid x \in V \setminus \{0\}\}$ consisting of all onedimensional subspaces of $V$ is called the \textit{projective space} of $V$. If $V$ is a Hilbert space then $\mathcal{P}(V)$ is called \textit{projective Hilbert space}. We call the elements of a projective Hilbert space \textit{rays}.
\end{definition}


\begin{lemma} \label{lemma:ray_prod}
	Let $R_1$ and $R_2$ be rays of the projective Hilbert space $\mathcal{P}(H)$. Then there exists a unique $\rho \in [0, 1]$ such that for all $x_1 \in R_1 \setminus \{0\}$ and $x_2 \in R_2 \setminus \{0\}$ the equation
	\begin{align*}
		\frac{\vbraces{(x_1, x_2)_H}}{\norm[H]{x_1} \norm[H]{x_2}} = \rho
	\end{align*}
	holds.
\end{lemma}

\begin{proof}
	Let $x_1, y_1 \in R_1 \setminus \{0\}$ and $x_2, y_2 \in R_2 \setminus \{0\}$. We know that $y_1 = \lambda_1 x_1$ and $y_2 = \lambda_2 x_2$ for some $\lambda_1, \lambda_2 \in \C \setminus \{0\}$. Now we just start calculating and obtain
	\begin{align*}
		\rho := \frac{\vbraces{(y_1, y_2)_H}}{\norm[H]{y_1} \norm[H]{y_2}} = \frac{\vbraces{(\lambda_1 x_1, \lambda_2 x_2)_H}}{\norm[H]{\lambda_1 x_1} \norm[H]{\lambda_2 x_2}} = \frac{\vbraces{\lambda_1 \lambda_2}\vbraces{(x_1, x_2)_H}}{\vbraces{\lambda_1 \lambda_2}\norm[H]{x_1} \norm[H]{x_2}} = \frac{\vbraces{(x_1, x_2)_H}}{\norm[H]{x_1} \norm[H]{x_2}}.
	\end{align*}
	Because of the Cauchy-Schwarz inequality it is clear that $\rho \in [0,1]$. 
\end{proof}


\begin{definition}
	The previous lemma \ref{lemma:ray_prod} allows us to define a \textit{ray-product} on a projective Hilbert space $\mathcal{P}(H)$.
	\begin{align*}
		(\cdot, \cdot)_{\mathcal{P}(H)}: \mathcal{P}(H) \times \mathcal{P}(H) \to [0,1] : (\C x, \C y) \mapsto \frac{\vbraces{(x, y)_H}}{\norm[H]{x} \norm[H]{y}}
	\end{align*}
\end{definition}


\begin{lemma} \label{lemma:projective_metric}
	Let $\mathcal{P}(H)$ be a projective Hilbert space and
	\begin{align*}
		f: \mathcal{P}(H) \to L_b(H): R \mapsto 
		\begin{cases}
			H \to H \\
			x \mapsto (x,v_R)_H v_R
		\end{cases},
	\end{align*} 
	where $v_R \in R$ is a normalized vector. Then for all $R \in \mathcal{P}(H)$ we find that $f(R)$ is the orthogonal projection with $\ran f(R) = R$ and $d: \mathcal{P}(H) \times \mathcal{P}(H) \to [0, \infty): (R,S) \mapsto \norm{f(R) - f(S)}$ is a metric. 
\end{lemma}

\begin{proof}
	First we consider some $R \in \mathcal{P}(H)$ and define $P := f(R): H \to H: x \mapsto (x,v_R)_H v_R$. We find that for every $x \in H$
	\begin{align*}
		P^2x = P(x, v_R)_H v_R = \pbraces{(x, v_R)_H v_R, v_R}_H v_R = (x,v_R)_H v_R = Px
	\end{align*}
	and for all $x,y \in H$
	\begin{align*}
		(Px, y)_H = \pbraces{(x, v_R)_H v_R, y}_H = (x, v_R)_H (v_R, y)_H = \pbraces{x, (y, v_R)_H v_R}_H = (x, Py)_H
	\end{align*}
	and we observe that $P = f(R)$ is a linear function and hence an orthogonal projection. Due to the uniqueness of the orthogonal projection that we know from remark \ref{remark:orth_proj_uniqueness} it is clear that $d$ is a metric.
\end{proof}


\begin{remark}
	Throughout this paper a projective Hilbert space will be endowed with the metric from lemma \ref{lemma:projective_metric}. 
\end{remark}


\begin{lemma} \label{lemma:metric_representation}
	In a projective Hilbert space $\mathcal{P}(H)$ for all rays $R$ and $S$ the equality 
	\begin{align*}
		d(R,S) = \sqrt{1 - (R,S)_{\mathcal{P}(H)}^2}
	\end{align*}
	holds.
\end{lemma}

\begin{proof}
	Let $R,S \in\mathcal{P}(H)$ be arbitrary rays and $P: H \to R: x \mapsto (x, u)_H u$ as well as $Q: H \to S: x \mapsto (x, v)_H v$, where $u \in R$ and $v \in S$ are normalized vectors. If $R = S$ then, using remark \ref{remark:csb}, we easily observe that the equation holds, thus from now on we assume $R \neq S$. Now we are going to have a look at the spectrum of $T: H \to H: x  \mapsto Px - Qx$. According to lemma \ref{lemma:projective_metric} $P$ and $Q$ are orthogonal projections and we observe
	\begin{align*}
		T^\ast = P^\ast - Q^\ast = P - Q = T
	\end{align*}
	hence $T$ is normal. We also observe that $\ran T \subseteq \mathrm{span}\{u, v\} =: W$ thus $\dim \ran T < \infty$ and according to remark \ref{remark:compact}¸ this implies that $T$ is compact.
	
	Let us now assume $\lambda \in \C \setminus \{0\}$ is in the spectrum of $T$. Due to the fact that $T$ is compact we know from remark \ref{remark:compact_spectrum} that $\lambda$ is eigenvalue of $T$ which gives us $T x = \lambda x$ for some $x \in H \setminus \{0\}$. This equation lets us conclude that $x \in \mathrm{span 
	}\ T$ which means there exist $\mu, \nu \in \C$ with $x = \mu u + \nu v$. As we know that $x \neq 0$ we can also conclude that $\mu \neq 0$ or $\nu \neq 0$ and without loss of generality we assume $\mu \neq 0$. Now we start calculating.
	\begin{align*}
		\lambda \mu u + \lambda \nu v = \lambda x = Tx = Px - Qx = \mu Pu + \nu Pv - \mu Qu - \nu Qv = \mu u + \nu Pv - \mu Qu - \nu v .
	\end{align*}
	As $R \neq S$ we know that $u$ and $v$ are linearly independent and $Qu = (u,v)_H v$ as well as $Pv = (v,u)_H u$ hence the two equations
	\begin{align}
		\lambda \mu  = \mu  + \nu (v,u)_H \label{eq:metric1}\\
		 \lambda \nu  = - \nu  - \mu (u,v)_H \label{eq:metric2}
	\end{align}
	must be fulfilled. 
	
	If $(v,u)_H = 0$ then from \eqref{eq:metric1} we conclude that $\lambda \mu = \mu$ and hence $\lambda = 1$. In this case we also observe that $(R,S)_{\mathcal{P}(H)} = |(u,v)_H| = 0$ and hence according to remark \ref{remark:spectral_radius}
	\begin{align*}
		d(R,S) = \norm{P - Q} = r(P - Q) = 1 = \sqrt{1 - (R,S)_{\mathcal{P}(H)}}
	\end{align*}
	which is what we had to show.
	
	Now assuming $(v,u)_H \neq 0$ we can do further calculations. First we use \eqref{eq:metric1} and obtain
	\begin{align*}
		\lambda = \frac{\mu + \nu (v,u)_H}{\mu} = 1 + \frac{\nu}{\mu} (v,u)_H 
	\end{align*}
	which lets us conclude that 
	\begin{align}
		\frac{\nu}{\mu} = \frac{\lambda - 1}{(v,u)_H}. \label{eq:metric3}
	\end{align}
	Now, using \eqref{eq:metric2},we obtain
	\begin{align*}
		 (\lambda + 1) \frac{\nu}{\mu} =  -(u,v)_H 
	\end{align*}
	and plugging in \eqref{eq:metric3} we conclude that
	\begin{align*}
			\frac{\lambda^2 - 1}{(v,u)_H} = (\lambda + 1) \frac{\lambda - 1}{(v,u)_H} =  - (u,v)_H.
	\end{align*}
	With a simple transformation we get
	\begin{align*}
		\lambda = \pm \sqrt{1 - |(u,v)|^2} = \pm \sqrt{1 - (R,S)_{\mathcal{P}(H)}^2}.
	\end{align*}
	Although we do not know for sure whether $0$ is in the spectrum of $T$ we now know the spectral radius of $T$ which finishes the proof because using remark \ref{remark:spectral_radius} we obtain
	\begin{align*}
		d(R,S) = \norm{P - Q} = r(P - Q) = \sqrt{1 - (R,S)_{\mathcal{P}(H)}^2}.
	\end{align*}
\end{proof}


\begin{lemma}
	Let $\mathcal{P}(H_1)$ and $\mathcal{P}(H_2)$ be two projective Hilbert spaces and $g: \mathcal{P}(H_1) \to \mathcal{P}(H_2)$ an isometry. Let furthermore $M := \{e_i \mid i \in I\}$ be an orthonormal basis of $H_1$ and $x,y \in H_1 $. Let $\tilde{x} \in g(\C x)$ and $\tilde{y} \in g(\C y)$  be vectors with $\norm[H_1]{x} = \norm[H_2]{\tilde{x}}$ and $\norm[H_1]{y} = \norm[H_2]{\tilde{y}}$. Lastly, for every $i \in I$ let $\tilde{e}_i \in g(\C e_i)$ be a normalized vector. Then the  following statements are true.
	
	\begin{enumerate}
		\item The equality
		\begin{align} 
			\vbraces{(\tilde{x}, \tilde{y})_{H_1}} = \vbraces{(x, y)_{H_2}} \label{eq:vector_isometry}
		\end{align}
		holds.		
		
		\item The set $L:=\{e_i \mid i \in I\}$ is an orthonormal system of $H_2$. \label{bullet:ran_ons}
		
		\item The equality
		\begin{align}
			\tilde{x} = \sum_{i \in I} (\tilde{x}, \tilde{e}_i)_{H_2} \tilde{e}_i. \label{eq:ran_fourier}
		\end{align}
		holds.
		
	\end{enumerate} 
\end{lemma}

\begin{proof}
	We will proof the statements separately.
	\begin{enumerate}
		\item In case $x = 0$ or $y = 0$ the statement is clearly true. From now on we assume $x,y \in H_1 \setminus \{0\}$. Using lemma \ref{lemma:metric_representation} we obtain
		\begin{align*}
			\sqrt{1 - \dfrac{|(x,y)_{H_1}|^2}{\norm[H_1]{x}^2 \norm[H_1]{y}^2}} &= \sqrt{1 - (\C x,\C y)_{\mathcal{P}(H_1)}^2} = d(\C x,\C y) \\
			&= d\pbraces{g(\C x), g(\C y)} = \sqrt{1 - (g(\C x), g(\C y))_{\mathcal{P}(H_2)}^2} = \sqrt{1 - \dfrac{|(\tilde{x}, \tilde{y})_{H_2}|^2}{\norm[H_2]{\tilde{x}}^2 \norm[H_2]{\tilde{y}}^2}}
		\end{align*}
		and we immediately observe that $|(x,y)_{H_1}| = |(\tilde{x},\tilde{y})_{H_2}|$. 
		
		\item Using what we just proofed \eqref{eq:vector_isometry} and the fact that $M$ is an orthonormal basis of $H_1$ we pbtain for every $i,j \in I$ the equality
		\begin{align*}
			|(\tilde{e}_i, \tilde{e}_j)_{H_2}| = |(e_i, e_j)_{H_1}| = 
			\begin{cases}
				0 &, \text{if } i \neq j \\
				1 &, \text{if } i = j
			\end{cases}.
		\end{align*}
		Hence $L$ is an orthonormal system.
		
		\item In case $\tilde{x} = 0$ the statement is clearly true. From now on we assume $\tilde{x} \neq 0$. Using \eqref{eq:vector_isometry} and Parzeval's equality \eqref{eq:parzeval} we obtain
		\begin{align*}
			\norm[H_2]{\tilde{x}}^2 = \norm[H_1]{x}^2 = \sum_{i \in I} |(x, e_i)_{H_1}|^2 = \sum_{i \in I} |(\tilde{x}, \tilde{e}_i)_{H_2}|^2.
		\end{align*}
		Because of \ref{bullet:ran_ons} we can now use \eqref{eq:parceval_to_fourier} and immediately obtain \eqref{eq:ran_fourier}, which is just what we wanted to show.
	\end{enumerate}
\end{proof}

\section{Statement and proof of Wigner's Theorem}

\begin{example} \label{example:zerodim}
	Let $H_1$ and $H_2$ be Hilbert spaces where $\dim H_1 = 0$ and $g: \mathcal{P}(H_1) \to \mathcal{P}(H_2)$ be an isometry. As $H_1 = \{0\}$ we obtain $\mathcal{P}(H_1) = \emptyset$. Now by defining $U: H_1 \to H_2: 0 \mapsto 0$ we observe that $U$ is linear as well as antilinear and also both unitary and antiunitary. Furthermore for every $R \in \mathcal{P}(H_1)$ and every $x \in H_1$ the implication 
	\begin{align*}
		x \in R \Rightarrow Ux \in g(R)
	\end{align*} 
	is true simply because $\mathcal{P}(H_1) = \emptyset$.
\end{example}

\begin{example} \label{example:onedim}
	Let $H_1$ and $H_2$ be Hilbert sapces where $\dim H_1 = 1$ and $g: \mathcal{P}(H_1) \to \mathcal{P}(H_2)$ be an isometry. Now we have $\mathcal{P}(H_1) = \{H_1\}$ which means there exists only one ray in $\mathcal{P}(H_1)$. Now we take a normalized $x \in H_1$ and a normalized $y \in g(H_1)$ and define $U: H_1 \to H_2: \lambda x \mapsto \lambda y$ and $T: H_1 \to H_2: \lambda x \mapsto \overline{\lambda} y$. For any $\lambda, \mu \in \C$
	\begin{align*}
		(U \lambda x, U \mu x)_{H_2} = (\lambda y, \mu y)_{H_2} = \lambda \overline{\mu} \norm[H_2]{y} = \lambda \overline{\mu} = \lambda \overline{\mu} \norm[H_1]{x} = (\lambda x, \mu x)_{H_2}
	\end{align*}
	and 
	\begin{align*}
		(T\lambda x, T \mu x)_{H_2} = (\overline{\lambda} y, \overline{\mu} y)_{H_2} = \overline{\lambda} \mu \norm[H_2]{y} = \overline{\lambda} \mu = \overline{\lambda } \mu \norm[H_1]{x} = \overline{(\lambda x, \mu x)_{H_2}}.
	\end{align*}
	Hence $U$ is unitary and $T$ is antiunitary. Furthermore by definition for any ray $R \in \mathcal{P}(H)$ and any $z \in R$ we have $Uz, Tz \in g(R)$. 
\end{example}

\begin{lemma} \label{lemma:phase_adjustment_ray}
	Let $H_1$ and $H_2$ be Hilbert spaces and $g: \mathcal{P}(H_1) \to \mathcal{P}(H_2)$ be an isometry. For two vectors $x,y \in H_1$ where $(x,y)_{H_1} \neq 0$ and a normalized vector $\tilde{x} \in g(\C x)$ with $\norm[H_2]{\tilde{x}} = \norm[H_1]{x}$ there exists a unique $\tilde{y} \in g(\C y)$ that fulfills $\norm[H_2]{\tilde{y}} = \norm[H_1]{y}$ and $(v,u)_{H_2} = |(v,u)_{H_2}| $.
\end{lemma}

\begin{proof}
	We take some arbitrary vector $\tilde{w} \in g(\C y)$ with $\norm[H_2]{\tilde{w}} = \norm[H_1]{\tilde{w}}$ and define $\mu := (\tilde{w}, \tilde{x})_{H_2}$. From \eqref{eq:vector_isometry} we know that $|\mu| = |(\tilde{w},\tilde{x})_{H_2}| = |(y,x)_{H_1}| \neq 0$ and hence we know that $\mu \in \C \setminus \{0\}$. Now we know from lemma \ref{lemma:phase_adjustment_complex} that there exists a unique $\lambda \in \C$ with $|\lambda| = 1$ such that $|\lambda \mu| = \lambda \mu$. Now we define $\tilde{y} := \lambda \tilde{w}$ and obtain
	\begin{align*}
		(\tilde{y}, \tilde{x})_{H_2} = \lambda (\tilde{w}, \tilde{x})_{H_2} = \lambda \mu = |\lambda \mu| = |\lambda (\tilde{w}, \tilde{x})_{H_2}| = |(y,x)_{H_2}|
	\end{align*}
	and $\norm[H_2]{\tilde{y}} = |\lambda| \norm[H_2]{\tilde{w}} = \norm[H_1]{y}$. 
\end{proof}


\begin{lemma} \label{lemma:aux_main}
	Let $H_1$ and $H_2$ be Hilbert spaces and $g: \mathcal{P}(H_1) \to \mathcal{P}(H_2)$ an isometry. Let $M := \{e_i \mid i \in I\}$ be an orthonormal basis and $x \in H_1 \setminus \{0\}$ with
	\begin{align*}
		x = \sum_{i \in I} \lambda_i e_i
	\end{align*} 
	and $l \in I$ with $\lambda_l \in \C \setminus \{0\}$. For every $i \in I$ let $\tilde{e}_i \in g(\C e_i)$ be a normalized vector. Then there exists a unique $\tilde{x} \in g(\C x)$ with 
	\begin{align*}
		\tilde{x} = \lambda_l \tilde{e}_l + \sum_{i \in I \setminus \{l\}} \mu_i \tilde{e}_i
	\end{align*}
	where for every $i \in I \setminus \{l\}$
	\begin{align*}
		\mu_i = (\tilde{x}, \tilde{e}_i)_{H_2} \quad \text{and} \quad |\mu_i| = |\lambda_i|.
	\end{align*}
\end{lemma}

\begin{proof}
	We first observe that for every $j \in I$ 
	\begin{align}
		\pbraces{x, e_j}_{H_1} &= \pbraces{\sum_{i \in I} \lambda_i e_i, e_j}_{H_1} = \sum_{i \in I} \lambda_i \pbraces{e_i, e_j}_{H_1} = \lambda_j \label{eq:aux_fouriercoef}
	\end{align}
	Since $\lambda_l \neq 0$ we can use lemma \ref{lemma:phase_adjustment_ray} and we obtain a unique $\tilde{y} \in g(\C x)$ with $\norm[H_2]{\tilde{y}} = \norm[H_1]{x}$ and 
	\begin{align*}
		(\tilde{y}, \tilde{e}_l)_{H_2} = |(\tilde{y}, \tilde{e}_l)_{H_2}| = |(x, e_l)_{H_1}| = \vbraces{\lambda_l}
	\end{align*}
	whereby we also used \eqref{eq:vector_isometry}. From lemma \ref{lemma:phase_adjustment_complex} we know there exists a normalized $\nu \in \C$ with $|\lambda_l| = |\nu \lambda_l| = \nu \lambda_l$. We define $\tilde{x} := \frac{1}{\nu} \tilde{y}$ and find
	\begin{align*}
		(\tilde{x}, \tilde{e}_l)_{H_2} = \frac{1}{\nu} (\tilde{y}, \tilde{e}_l)_{H_2} = \frac{1}{\nu} |\lambda_l| = \frac{1}{\nu} \nu \lambda_l = \lambda_l.
	\end{align*} 
	Finally, using \eqref{eq:ran_fourier}, we obtain
	\begin{align*}
		\tilde{x} = \sum_{i \in I} (\tilde{x}, \tilde{e}_i)_{H_2} \tilde{e}_i = \lambda_l \tilde{e}_l + \sum_{i \in I \setminus \{l\}} (\tilde{x}, \tilde{e}_i)_{H_2} \tilde{e}_i 
	\end{align*}
	and for every $i \in I \setminus \{l\}$ we conclude, using \eqref{eq:vector_isometry} and \eqref{eq:aux_fouriercoef}, that
	\begin{align*}
		\vbraces{(\tilde{x}, \tilde{e}_i)_{H_2}} = |(x, e_i)_{H_1}| = |\lambda_i´|.
	\end{align*}
\end{proof}


\begin{lemma} \label{lemma:function_on_onb}
	Let $H_1$ and $H_2$ be Hilbert spaces with $\dim H_1 > 1$ and $g: \mathcal{P}(H_1) \to \mathcal{P}(H_2)$ an isometry. Let furthermore $M := \{e_j \mid j \in J\}$ be a non-empty orthonormal basis of $H_1$ with $q \in J$ and some normalized $\tilde{e}_q \in g(\C e_q)$  and for all $j \in J \setminus \{q\}$ the vectors $v_{qj} := \frac{1}{\sqrt{2}} (e_q + e_j)$ and $w_{qj} := \frac{1}{\sqrt{2}}(e_q + ie_j)$ and $w_{jq} := \frac{1}{\sqrt{2}}(e_j + ie_q)$. Then for every $k \in J \setminus \{q\}$ there exists a normalized $\tilde{e}_k \in g(\C e_k)$ and a normalized $\tilde{v}_{qk} \in g(\C x_{qk})$ and a normalized $\tilde{w}_{qk} \in g(\C w_{qk})$ and a normalized $\tilde{w}_{kq} \in g(\C w_{kq})$ and $\lambda_k \in \{i, -i\}$ with
	\begin{align*}
		\tilde{v}_{qk} = \frac{1}{\sqrt{2}}(\tilde{e}_{q} + \tilde{e}_k) \quad \text{and} \quad \tilde{w}_{qk} = \frac{1}{\sqrt{2}}(\tilde{e}_q + \lambda_k \tilde{e}_k) \quad \text{and} \quad \tilde{w}_{kq} = \frac{1}{\sqrt{2}} (\tilde{e}_k + \lambda_k \tilde{e}_q).
	\end{align*}
\end{lemma}

\begin{proof}
	We observe that for every $i \in J$ and every $j \in J \setminus q$ we have
	\begin{align*}
		\pbraces{v_{qj}, e_i}_{H_1} = \pbraces{\frac{1}{\sqrt{2}}(e_q + e_j), e_i}_{H_1} =
		\begin{cases}
			\frac{1}{\sqrt{2}} &, \text{if } i \in \{q,j\} \\
			0 &, \text{else}
		\end{cases}
	\end{align*}
	Hence we can use lemma \ref{lemma:phase_adjustment_ray} and obtain $\tilde{v}_{qj} \in g(\C v_{qj})$ with
	\begin{align*}
		(\tilde{v}_{qj}, \tilde{e}_q)_{H_2} = \vbraces{(\tilde{v}_{qj}, \tilde{e}_q)_{H_2}} = \vbraces{(v_{qj}, e_q)_{H_1}} = \frac{1}{\sqrt{2}}.
	\end{align*}
	Using lemma \ref{lemma:phase_adjustment_ray} again, we find $\tilde{e}_{j} \in g(\C e_j)$ with
	\begin{align*}
		\pbraces{\tilde{v}_{qj}, \tilde{e}_j}_{H_2} = \vbraces{\pbraces{\tilde{v}_{qj}, \tilde{e}_j}_{H_2}} = \vbraces{\pbraces{v_{qj}, e_j}_{H_1}} = \frac{1}{\sqrt{2}}.
	\end{align*}
	For all $i \in J \setminus \{q, j\}$ we obtain
	\begin{align*}
		\vbraces{\pbraces{\tilde{v}_{q,j}, \tilde{e}_i}_{H_2}} = \vbraces{\pbraces{v_{qj}, e_i}_{H_1}} = 0
	\end{align*}
	and thus when using \eqref{eq:ran_fourier} we obtain
	\begin{align*}
		\tilde{v}_{qj} = \sum_{i \in I} (\tilde{v}_{qj}, \tilde{e}_i)_{H_2} \tilde{e}_i = \frac{1}{\sqrt{2}} (\tilde{e}_q + \tilde{e}_j).
	\end{align*}
	We choose some arbitrary $k \in J$. Using lemma \ref{lemma:aux_main} we obtain $\tilde{w}_{qj} \in g(\C w_{qj})$ and $\tilde{w}_{jq} \in g(\C w_{jq})$ with
	\begin{align*}
		\tilde{w}_{qj} = \frac{1}{\sqrt{2}}(\tilde{e}_q + \lambda_j \tilde{e}_j) \quad \text{and} \quad \tilde{w}_{jq} = \frac{1}{\sqrt{2}}(\tilde{e}_j + \lambda_{q} \tilde{e}_q)
	\end{align*}
	and $\vbraces{\lambda_q} = \vbraces{\lambda_j} = 1$. Next we find
	\begin{align*}
		\frac{1}{\sqrt{2}}\vbraces{1 + \lambda_j} &= \vbraces{\pbraces{\frac{1}{\sqrt{2}}(\tilde{e}_q + \lambda_j \tilde{e}_j), \frac{1}{\sqrt{2}} (\tilde{e}_q + \tilde{e}_j)}_{H_2}} = \vbraces{\pbraces{\tilde{w}_{qj}, \tilde{v}_{qj}}_{H_2}} \\
		&= \vbraces{\pbraces{w_{qj}, v_{qj}}_{H_1}} = \vbraces{\pbraces{\frac{1}{\sqrt{2}}(e_q + ie_j), \frac{1}{\sqrt{2}} (e_q + e_j)}_{H_1}} = \frac{1}{\sqrt{2}} \vbraces{1 + i} = 1
	\end{align*}
	and similarly $\vbraces{1 + \lambda_q} = \sqrt{2}$ thus with lemma \ref{lemma:complex_geom} we obtain $\lambda_j, \lambda_q \in \{i, -i\}$. Now we have a look at
	\begin{align*}
		\frac{1}{2}\vbraces{\lambda_j + \overline{\lambda_q}} &= \vbraces{\pbraces{\frac{1}{\sqrt{2}}(\tilde{e}_q + \lambda_j \tilde{e}_j), \frac{1}{\sqrt{2}}(\tilde{e}_j + \lambda_{q} \tilde{e}_q)}_{H_2}} = \vbraces{\pbraces{\tilde{w}_{qj}, \tilde{w}_{jq}}_{H_2}} \\
		&= \vbraces{\pbraces{w_{qj}, w_{jq}}_{H_1}} = \vbraces{\pbraces{\frac{1}{\sqrt{2}}(e_q + i e_j), \frac{1}{\sqrt{2}}(e_j + i e_q)}_{H_1}} = \frac{1}{2}\vbraces{i - i} = 0
	\end{align*}
	and conclude that $\lambda_j = \lambda_q$ because else we would have the contradiction $1 = 0$. 
\end{proof}


\begin{example} \label{example:twodim}
	Let $H_1$ and $H_2$ be Hilbert spaces with $\dim H_1 = 2$ and $g: \mathcal{P}(H_1) \to \mathcal{P}(H_2)$ an isometry. We consider a orthonormal basis $M = {e_1, e_2}$ of $H_1$ and define
	\begin{align*}
		v := \frac{1}{\sqrt{2}}(e_1 + e_2), \qquad w_{12} := \frac{1}{\sqrt{2}}(e_1 + i e_2), \qquad  w_{21} := \frac{1}{\sqrt{2}}(e_2 + i e_1).
	\end{align*} 
	From lemma \ref{lemma:function_on_onb} we know that there exist $\tilde{e}_1 \in g(\C e_1)$, $\tilde{e}_2 \in g(\C e_2)$, $\tilde{v} \in g(\C v)$, $\tilde{w}_{12} \in g(\C w_{12})$, $\tilde{w}_{21} \in g(\C w_{21})$ and $\lambda \in \{i, -i\}$ with
	\begin{align*}
		\tilde{v} = \frac{1}{\sqrt{2}}(\tilde{e}_1 + \tilde{e}_2), \qquad \tilde{w}_{12} = \frac{1}{\sqrt{2}}(\tilde{e}_1 + \lambda \tilde{e}_2), \qquad \tilde{w}_{21} = \frac{1}{\sqrt{2}}(\tilde{e}_2 + \lambda \tilde{e}_1)
	\end{align*} 
	If $\lambda = i$ the we define $\zeta = \id_C$ and if $\lambda = -i$ we define $\zeta$ as the complex conjugation. Either way we hae $\lambda = \zeta(i)$.
	
	Now we are ready to define $U$. First of all $U0 := 0$. For an arbitrary $z \in H_1 \setminus \{0\}$ we know there exist $\lambda_1, \lambda_2 \in \C$ with $z = \lambda_1 e_1 + \lambda_2 e_2$ and there exists $r \in \{1,2\}$ with $\lambda_r \neq 0$. We then find $s \in \{1, 2\} \setminus \{r\}$. From lemma \ref{lemma:aux_main} we know there exists $\tilde{z} \in g(\C z)$ with 
	\begin{align*}
		\tilde{z} = \lambda_r e_r + \nu_s e_s \quad \text{where} \quad \vbraces{\nu_s} = \vbraces{\lambda_s}.
	\end{align*}
	and we define
	\begin{align*}
		Uz := \zeta(\lambda_r) \tilde{e}_r + \zeta(\nu_s) \tilde{e}_s
	\end{align*}
	We find 
	\begin{align*}
		\frac{1}{\sqrt{2}} \vbraces{\lambda_r + \nu_s} = \frac{1}{\sqrt{2}} \vbraces{\zeta(\lambda_r + \nu_s)}= \vbraces{\pbraces{\tilde{z}, \tilde{v}}_{H_2}} = \vbraces{\pbraces{z, v}_{H_1}} = \frac{1}{\sqrt{2}} \vbraces{\lambda_r + \lambda_s}
	\end{align*}
	and for $y := \frac{1}{\sqrt{2}}(e_r + ie_s)$ and $\tilde{y} := \frac{1}{\sqrt{2}} (e_r + \zeta(i) e_s)$ we find
	\begin{align*}
		\frac{1}{\sqrt{2}}\vbraces{\lambda_r - i\nu_s} &= \frac{1}{\sqrt{2}} \vbraces{\zeta(\lambda_r) + \overline{\zeta(i)} \zeta(\nu_s)}= \vbraces{\pbraces{\tilde{z}, \tilde{y}}_{H_2}} \\
		&= \vbraces{\pbraces{z, y}_{H_1}} = \vbraces{\pbraces{\lambda_r e_r + \lambda_s e_2, \frac{1}{\sqrt{2}}(e_r + ie_s)}_{H_1}} = \frac{1}{\sqrt{2}} \vbraces{\lambda_r - i\lambda_s}
	\end{align*}
	and with lemma \ref{lemma:complex_alg} we obtain $\lambda_s = \nu_s$. Hence we know
	\begin{align*}
		Uz = \zeta(\lambda_1) \tilde{e}_1 + \zeta(\lambda_2) \tilde{e}_2.
	\end{align*}
	Furthermore we have
	\begin{align*}
		\norm[H_2]{Uz}^2 = \norm[H_2]{\zeta\pbraces{\lambda_1} e_1 + \zeta\pbraces{\lambda_2} e_2}^2 = \vbraces{\zeta\pbraces{\lambda_1}}^2 + \vbraces{\zeta\pbraces{\lambda_2}}^2 = \vbraces{\lambda_1}^2 + \vbraces{\lambda_2}^2 = \norm[H_1]{\lambda_1 e_1 + \lambda_2 e_2} = \norm[H_1]{z}
	\end{align*}
\end{example}


\begin{lemma} \label{lemma:aux_last}
	Let $H_1$ and $H_2$ be Hilbert spaces with $\dim H_1 >2$ and $g: \mathcal{P}(H_1) \to \mathcal{P}(H_2)$ an isometry. Let furthermore $\{e_j \mid j \in J\}$ be an orthonormal basis of $H_1$ and for all distinct $k,l \in J$ 
	\begin{align*}
		v_{kl} := \frac{1}{\sqrt{2}}(e_k + e_l) \quad \text{and} \quad w_{kl} := \frac{1}{\sqrt{2}}(e_k + ie_l)
	\end{align*}
	Then for all $j \in J$ there exists $\tilde{e}_j \in g(\C e_j)$ and for all distinct $r,s \in J$ there exists a normalized $\tilde{v}_{rs} \in g(\C v_{rs})$, a normalized $\tilde{w}_{rs} \in g(\C w_{rs})$ and a $\lambda_{rs} \in \{-i, i\}$ with 
	\begin{align*}
		\tilde{v}_{rs} = \frac{1}{\sqrt{2}}(\tilde{e}_r + \tilde{e}_s) \quad \text{and} \quad \tilde{w}_{rs} = \frac{1}{\sqrt{2}}(\tilde{e}_r + \lambda_{rs} \tilde{e}_s)
	\end{align*}
	and $\lambda_{rs} = \lambda_{sr}$.
\end{lemma}

\begin{proof}
	We know from lemma \ref{lemma:function_on_onb} that for every $j \in J \setminus \{q\}$ there exist $\tilde{e}_j \in g(\C e_j)$, $\tilde{v}_{qj} \in g(\C v_{qj})$, $\tilde{w}_{qj} \in g(\C w_{qj})$, $\tilde{w}_{jq} \in g(\C w_{jq})$ and $\lambda_{jq} = \lambda_{qj} \in \{i, -i\}$ with
	\begin{align*}
	\tilde{v}_{qj} = \frac{1}{\sqrt{2}}(\tilde{e}_q + \tilde{e}_j), \qquad \tilde{w}_{qj} = \frac{1}{\sqrt{2}}(\tilde{e}_q + \lambda_{qj}\tilde{e}_j), \qquad \tilde{w}_{jq} = \frac{1}{\sqrt{2}}(\tilde{e}_j + \lambda_{jq} \tilde{e}_q).
	\end{align*} 
	We define $\tilde{v}_{jq} := \tilde{v}_{qj}$ and arbitrarily choose distinct $k,l \in J \setminus \{q\}$ and define
	\begin{align*}
	x_{kl} := \frac{1}{\sqrt{3}}(e_q + e_k + e_l) 
	\end{align*} 
	We can now use lemma \ref{lemma:aux_main} and know that there exists a unique $\tilde{x}_{kl} \in g(\C x_{kl})$ with 
	\begin{align*}
	\tilde{x}_{kl} = \frac{1}{\sqrt{3}} \pbraces{\tilde{e}_q + \mu_{k} \tilde{e}_k + \mu_{l} \tilde{e}_l} 
	\end{align*}
	and $\vbraces{\mu_k} = \vbraces{\mu_l} = 1$. Next, we can use \ref{bullet:ran_ons} and observe that for $j \in \{k,l\}$
	\begin{align*}
	\frac{1}{\sqrt{6}} \vbraces{1 + \mu_j} &= \vbraces{\pbraces{\frac{1}{\sqrt{3}} \pbraces{\tilde{e}_q + \mu_k \tilde{e}_k + \mu_l \tilde{e}_l}, \frac{1}{\sqrt{2}} \pbraces{\tilde{e}_q + \tilde{e}_j}}_{H_2}} = |(\tilde{x}_{kl}, \tilde{v}_{qj})_{H_2}| \\
	&= |(x_{kl}, v_{qj})_{H_1}| = \vbraces{\pbraces{\frac{1}{\sqrt{3}} \pbraces{e_q + e_k + e_l}, \frac{1}{\sqrt{2}} (e_q + e_j)}_{H_1}} = \frac{2}{\sqrt{6}}.
	\end{align*}
	Using lemma \ref{lemma:complex_geom} we obtain $\mu_k = \mu_l = 1$, thus
	\begin{align*}
	\tilde{x}_{kl} = \frac{1}{\sqrt{3}}(\tilde{e}_q + \tilde{e}_k + \tilde{e}_l).
	\end{align*}
	Next we make use of lemma \ref{lemma:aux_main} and obtain $\tilde{v}_{kl} \in g(\C v_{kl})$ and $\tilde{w}_{kl} \in g(\C w_{kl})$ with
	\begin{align*}
	\tilde{v}_{kl} = \frac{1}{\sqrt{2}}(\tilde{e}_k + \nu_l \tilde{e}_l) \quad \text{and} \quad \tilde{w}_{kl} = \frac{1}{\sqrt{2}}(\tilde{e}_k + \lambda_{kl} \tilde{e}_l)
	\end{align*}
	and $\vbraces{\nu_l} = \vbraces{\lambda_{kl}}$. We have a look at 
	\begin{align*}
		\frac{1}{\sqrt{6}}(1 + \nu_l) &= \vbraces{\pbraces{\frac{1}{\sqrt{2}}(\tilde{e}_k + \nu_l \tilde{e}_l), \frac{1}{\sqrt{3}}(\tilde{e}_q + \tilde{e}_k + \tilde{e}_l)}_{H_2}} = \vbraces{\pbraces{\tilde{v}_{kl}, \tilde{x}_{kl}}_{H_2}} \\
		&= \vbraces{\pbraces{v_{kl}, x_{kl}}_{H_1}} = \vbraces{\pbraces{\frac{1}{\sqrt{2}}(e_k + e_l), \frac{1}{\sqrt{3}}(e_q + e_k + e_l)}_{H_1}} = \frac{2}{\sqrt{6}}
	\end{align*}
	and 
	\begin{align*}
		\frac{1}{\sqrt{6}}(1 + \lambda_{kl}) &= \vbraces{\pbraces{\frac{1}{\sqrt{2}}(\tilde{e}_k + \lambda_{kl} \tilde{e}_l), \frac{1}{\sqrt{3}}(\tilde{e}_q + \tilde{e}_k + \tilde{e}_l)}_{H_2}} = \vbraces{\pbraces{\tilde{w}_{kl}, \tilde{x}_{kl}}_{H_2}} \\
		&= \vbraces{\pbraces{w_{kl}, x_{kl}}_{H_1}} = \vbraces{\pbraces{\frac{1}{\sqrt{2}}(e_k + i e_l), \frac{1}{\sqrt{3}}(e_q + e_k + e_l)}_{H_1}} = \frac{\sqrt{2}}{\sqrt{6}}
	\end{align*}
	and obtain with lemma \ref{lemma:complex_geom} that $\nu_l = 1$ and $\lambda_{kl} \in \{i, -i\}$. Finally we find
	\begin{align*}
		\frac{1}{2}\vbraces{\lambda_{kl} - \lambda_{lk}} &= \frac{1}{2}\vbraces{\lambda_{kl} + \overline{\lambda_{lk}}} = \vbraces{\pbraces{\frac{1}{\sqrt{2}}(\tilde{e}_k + \lambda_{kl} \tilde{e}_l), \frac{1}{\sqrt{2}}(\tilde{e}_l + \lambda_{lk} \tilde{e}_k)}_{H_2}} = \vbraces{\pbraces{\tilde{w}_{kl}, \tilde{w}_{lk}}_{H_2}} \\
		&= \vbraces{\pbraces{w_{kl}, w_{lk}}_{H_1}} = \vbraces{\pbraces{\frac{1}{\sqrt{2}}(e_k + i e_l), \frac{1}{\sqrt{2}}(e_l + i e_k)}_{H_1}} = \frac{1}{2} \vbraces{i + \overline{i}} = 0
	\end{align*}
	and from this we conclude $\lambda_{kl} = \lambda_{lk}$ because else we would have the contradiction $1 = 0$. To sum up, we showed for arbitrary $r,s \in J$ the equalities
	\begin{align*}
		\tilde{v}_{rs} = \frac{1}{\sqrt{2}}(\tilde{e}_r + \tilde{e}_s) \quad \text{and} \quad \tilde{w}_{rs} = \frac{1}{\sqrt{2}}(\tilde{e}_r + \lambda_{rs} \tilde{e}_s)
	\end{align*}
	where $\lambda_{rs} \in \{i, -i\}$ hold and that $\lambda_{rs} = \lambda_{sr}$. 
\end{proof}


\begin{theorem} \label{theorem:wigner}
	Let $H_1$ and $H_2$ be Hilbert spaces and $g: \mathcal{P}(H_1) \to \mathcal{P}(H_2)$ be an isometry. Then there exists an isometry $U: H_1 \to H_2$ that is either linear or antilinear. Furthermore for every ray $R \in \mathcal{P}(H_1)$ the implication
	\begin{align*}
		x \in R \Rightarrow Ux \in g(R).
	\end{align*}
	holds.
\end{theorem}

\begin{proof}
	We already showed the theorem for $\dim H_1 = 0$ in \ref{example:zerodim}, for $\dim H_1 = 1$ in \ref{example:onedim} and for $\dim H_1 = 2$ in \ref{example:twodim}, hence from now on we assume $\dim H_1 > 2$. We know from lemma \ref{lemma:onb} that there exists an orthonormal basis $M := \{e_j \mid j \in J\}$ of $H_1$. For all distinct $r,s \in J$ we define
	\begin{align*}
		v_{rs} := \frac{1}{\sqrt{2}}(e_r + e_s) \quad \text{and} \quad w_{rs} := \frac{1}{\sqrt{2}}(e_r + ie_s).
	\end{align*}
	 We know from lemma \ref{lemma:aux_last} that for all $j \in J$ there exists $\tilde{e}_j \in g(\C e_j)$ and for all distinct $r,s \in J$ there exists a normalized $\tilde{v}_{rs} \in g(\C v_{rs})$, a normalized $\tilde{w}_{rs} \in g(\C w_{rs})$ and a $\lambda_{rs} \in \{-i, i\}$ with 
	 \begin{align*}
	 \tilde{v}_{rs} = \frac{1}{\sqrt{2}}(\tilde{e}_r + \tilde{e}_s) \quad \text{and} \quad \tilde{w}_{rs} = \frac{1}{\sqrt{2}}(\tilde{e}_r + \lambda_{rs} \tilde{e}_s)
	 \end{align*}
	 and $\lambda_{rs} = \lambda_{sr}$. Now we take distinct $k,l,m \in J$ and define
	 \begin{align*}
	 	y_{klm} := \frac{1}{\sqrt{3}}(e_k + e_l + ie_m).
	 \end{align*}
	 We know from lemma \ref{lemma:aux_main} that there exists a normalized $\tilde{y}_{klm} \in g(\C y_{klm})$ with 
	 \begin{align*}
	 	\tilde{y}_{klm} = \frac{1}{\sqrt{3}}(\tilde{e}_k + \mu_l \tilde{e}_l + \mu_m \tilde{e}_m)
	 \end{align*}
	 with $\vbraces{\mu_l} = \vbraces{\mu_m} = 1$. For $j \in \{l,m\}$ we find
	 \begin{align*}
	 	\frac{1}{\sqrt{6}} \vbraces{1 + \mu_j} &= \vbraces{\pbraces{\frac{1}{\sqrt{3}}(\tilde{e}_k + \mu_l \tilde{e}_l + \mu_m \tilde{e}_m), \frac{1}{\sqrt{2}}(\tilde{e}_k + \tilde{e}_j)}} = \vbraces{\pbraces{\tilde{y}_{klm}, \tilde{v}_{kj}}_{H_2}} \\
	 	&= \vbraces{\pbraces{y_{klm}, v_{kj}}_{H_1}} = \vbraces{\pbraces{\frac{1}{\sqrt{3}}(e_k + e_l + ie_m), \frac{1}{\sqrt{2}}(e_k + e_j)}_{H_1}} = 
	 	\begin{cases}
	 		\frac{2}{\sqrt{6}} &, \text{if } j = l \\
	 		\frac{\sqrt{2}}{\sqrt{6}} &, \text{if } j = m
	 	\end{cases}
	\end{align*}
	and using lemma \ref{lemma:complex_geom} we obtain $\mu_l = 1$ and $\mu_m \in \{-i, i\}$. Next, we find for $j \in \{k,l\}$ that
	\begin{align*}
		\frac{1}{\sqrt{6}} \vbraces{\mu_m - \lambda_{mj}} &= \frac{1}{\sqrt{6}} \vbraces{\mu_m + \overline{\lambda_{mj}}} = \vbraces{\pbraces{\frac{1}{\sqrt{3}}(\tilde{e}_k + \tilde{e}_l + \mu_m \tilde{e}_m), \frac{1}{\sqrt{2}}(\tilde{e}_m + \lambda_{mj} \tilde{e}_j)}_{H_2}} = \vbraces{\pbraces{\tilde{y}_{klm}, \tilde{w}_{mj}}_{H_2}} \\
		&= \vbraces{\pbraces{y_{klm}, w_{mj}}_{H_1}} = \vbraces{\pbraces{\frac{1}{\sqrt{3}}(e_k + e_l + ie_m), \frac{1}{\sqrt{2}}(e_m + ie_j)}_{H_1}} = \frac{1}{\sqrt{6}}\vbraces{i + \overline{i}} = 0
	\end{align*}
	and conclude that $\lambda_{mk} = \mu_m = \lambda_{ml}$. This is where we wanted to end up, because now we obtain for distinct $k,l,m,n \in J$
	\begin{align*}
		\lambda_{kl} = \lambda_{kn} = \lambda_{nk} = \lambda_{nm} = \lambda_{mn}
	\end{align*}
	and hence for all $k,l,m,n \in J$ with $k \neq l$ and $m \neq n$ that $\lambda_{kl} = \lambda_{mn}$. This allows us to define $\zeta: \C \to \C$ as the identitiy function if $\lambda_{kl} = i$ or the complex conjugation if $\lambda_{kl} = -i$ and write for all distinct $r,s \in J$ 
	\begin{align*}
		\tilde{w}_{rs} = \frac{1}{\sqrt{2}}(e_r + \zeta(i) e_s).
	\end{align*}
	
	We are ready to define $U$ on an arbitrary $z \in H_1 \setminus \{0\}$. Of course we define $U0 := 0$.  We know that with the definition $\lambda_j := (z,e_j)_{H_1}$ for all $j \in J$ the equality
	\begin{align*}
		z = \sum_{j \in J} \lambda_j e_j
	\end{align*}
	holds. As $x \neq 0$ there exists some $k \in J$ with $\lambda_k \neq 0$ and we know from lemma \ref{lemma:aux_main} that there exists a unique normalized $\tilde{z} \in g(\C z)$ with
	\begin{align*}
		\tilde{z} = \lambda_k \tilde{e}_k + \sum_{j \in J \setminus \{k\}} \nu_j \tilde{e}_j
	\end{align*} 
	with $|\lambda_j| = |\nu_j|$ for all $j \in J \setminus \{k\}$. We define
	\begin{align*}
		Uz := \zeta(\lambda_k) \tilde{e}_k + \sum_{j \in J \setminus{k}} \zeta(\nu_j) \tilde{e}_j.
	\end{align*}
	Now we consider some $l \in J \setminus \{k\}$ and calculate
	\begin{align*}
		\frac{1}{\sqrt{2}}\vbraces{\lambda_k + \nu_l} &= \frac{1}{\sqrt{2}} \vbraces{\zeta(\lambda_k) + \zeta(\nu_l)} = \vbraces{\pbraces{\zeta(\lambda_k) \tilde{e}_k + \sum_{j \in J \setminus{k}} \zeta(\nu_j) \tilde{e}_j, \frac{1}{\sqrt{2}}(\tilde{e}_k + \tilde{e}_l)}_{H_2}} = |(Uz, \tilde{v}_{kl})_{H_2}| \\
		&= \vbraces{\pbraces{z, v_{kl}}_{H_1}} = \vbraces{\pbraces{\sum_{j \in J} \lambda_j e_j, \frac{1}{\sqrt{2}}(e_k + e_l)}_{H_1}} = \frac{1}{\sqrt{2}} \vbraces{\lambda_k + \lambda_l} 
	\end{align*}
	and furthermore
	\begin{align*}
		\frac{1}{\sqrt{2}}\vbraces{\lambda_k - i \nu_l} &= \frac{1}{\sqrt{2}} \vbraces{\zeta(\lambda_k) + \overline{\zeta(i)} \zeta(\nu_l)} = \vbraces{\pbraces{\zeta(\lambda_k) \tilde{e}_k + \sum_{j \in J \setminus{k}} \zeta(\nu_j) \tilde{e}_j, \frac{1}{\sqrt{2}}(e_k + \zeta(i) e_l)}_{H_2}} = |(Uz, \tilde{w}_{kl})_{H_2}| \\
		&= \vbraces{\pbraces{z, w_{kl}}_{H_1}} = \vbraces{\pbraces{\sum_{j \in J} \lambda_j e_j, \frac{1}{\sqrt{2}}(e_k + ie_l)}_{H_1}} = \frac{1}{\sqrt{2}} \vbraces{\lambda_k - i \lambda_l}.
	\end{align*}
	Now we can use lemma \ref{lemma:complex_alg} and obtain $\nu_l = \lambda_l$. Now we observe that 
	\begin{align*}
		Uz = \sum_{j \in J} \zeta(\lambda_j) \tilde{e}_j = \sum_{j \in J} \zeta\pbraces{\pbraces{z, e_j}_{H_1}}\tilde{e}_j
	\end{align*}
	By definition $Uz \in g(\C z)$ and for arbitrary $x,y \in H_1$ and arbitrary $\lambda \in \C$ we obtain
	\begin{align*}
		U(x + \lambda y) &= U \pbraces{\sum_{j \in J} \pbraces{\pbraces{x, e_j}_{H_1} + \lambda \pbraces{y, e_j}_{H_1}} e_j} \\
		&= \sum_{j \in J} \zeta\pbraces{\pbraces{x, e_j}_{H_1}} \tilde{e}_j + \zeta(\lambda) \sum_{j \in J} \zeta\pbraces{\pbraces{y, e_j}_{H_1}} \tilde{e}_j = Ux + \zeta\pbraces{\lambda} Uy
	\end{align*}
	thus $U$ is $\zeta$-linear and 
	\begin{align*}
		\norm[H_2]{Ux}^2 &= \norm[H_2]{\sum_{j \in J} \zeta\pbraces{\pbraces{x, e_j}_{H_1}} \tilde{e}_j}^2 = \sum_{j \in J} \vbraces{\zeta\pbraces{\pbraces{x, e_j}_{H_1}}}^2  \\
		&= \sum_{j \in J} \vbraces{\pbraces{x, e_j}_{H_1}}^2 = \norm[H_1]{\sum_{j \in J} \zeta\pbraces{\pbraces{x, e_j}_{H_1}} e_j}^2 = \norm[H_1]{x}^2
	\end{align*}
\end{proof}


\begin{corollary}
	Let $H_1$ and $H_2$ be Hilbert spaces and $g: \mathcal{P}(H_1) \to \mathcal{P}(H_2)$ be a surjective isometry. Then there exists a function $U: H_1 \to H_2$ that is either linear and unitary or antilinear and antiunitary. Furthermore for every ray $R \in \mathcal{P}(H_1)$ the implication
	\begin{align*}
	x \in R \Rightarrow Ux \in g(R).
	\end{align*}
	holds.
\end{corollary}

\begin{proof}
	We know theorem \ref{theorem:wigner} that there exists $\zeta: \C \to \C$ that is either the identity function or the complex conjugation and a $\zeta$-linear isometry $U: H_1 \to H_2$. Furthermore for all $R \in \mathcal{P}(H_1)$ the implication
	\begin{align*}
		x \in R \Rightarrow Ux \in g(R)
	\end{align*}
	holds. Let us consider some arbitrary $y \in H_2$. If $y = 0$ then we know $U0 = 0 = y$ hence from now on we assume $y \neq 0$. Due to the fact that $g$ is surjective we know there exists $R \in \mathcal{P}(H_1)$ with $g(R) = \C y$. From this we conclude that $U(R) = \C y$ and hence there exists $x \in R$ with $Ux = y$, thus $\ran U = H_2$. From proposition \ref{prop:unitary} we obtain that $U$ is $\zeta$-unitary.
\end{proof}

\section{Concluding remarks}
The proof given here is not particularly short and it involves quite a few calculations. Despite these drawbacks the proof given has the merit that it proofs a very general form of Wigner's theorem where the two Hilbert spaces involved can be different ones and do not have to be separable. Furthermore we constructed the desired function step by step in the proof which might be very insightful and we did not have to use very deep mathematical results. Finally it is worth mentioning that the paper does give a lot of detailed calculations which should make it easy to read.

\section{Acknowldegements}
I want to thank my advisor Prof. Michael Kaltenbäck for his support and for correcting the paper. 

\printbibliography

\end{document}
