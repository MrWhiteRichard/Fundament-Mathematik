\section{Complex numbers $\C$}

As we will have to work a lot with the complex numbers $\C$ we want to start with some of their properties.

\begin{definition}
	Let $K$ be a field and $\zeta: K \to K$ a bijective function. We call $\zeta$ an \textit{automorphism} on $K$ if for all $\lambda, \mu \in K$ the equalities
	\begin{align*}
		\zeta(\lambda + \mu) = \zeta(\lambda) + \zeta(\mu) \quad \text{and} \quad \zeta(\lambda \mu) = \zeta(\lambda) \zeta(\mu)
	\end{align*}
	hold true.
\end{definition}


\begin{definition}
	Throughout this paper $\overline{\cdot}: \C \to \C: \lambda_1 + i \lambda_2 \mapsto \lambda_1 - i \lambda_2$ will be the \textit{complex conjugation}.
\end{definition}


\begin{lemma} \label{lemma:continuous_auto}
	There exist only two continuous automorphisms on $\C$, namely the identity function and the complex conjugation. These two functions both conincide with their own inverse and act as the identity function on the real line.
\end{lemma}

\begin{proof}
	The identity function and the complex conjugation are both isometries. Thus, they are continuous and they are clearly automorphisms on $\C$. They both coincide with their own inverse and act as the identity function on the real line.
	
	For any continuous automorphism $\zeta$ on $\C$ we have $\zeta(0) = 0$ and $\zeta(1) = 1$. Assume $\zeta(\alpha) = \alpha$ for some $\alpha \in \N$. We conclude $\zeta(\alpha + 1) = \zeta(\alpha) + \zeta(1) = \alpha + 1$. Hence, we showed by induction that for all $\alpha \in \N$ the equality $\zeta(\alpha) = \alpha$ holds true. For an arbitrary $\lambda \in \C \setminus \{0\}$ we have
	\begin{align*}
		0 = \zeta(0) = \zeta(\lambda - \lambda) = \zeta(\lambda) + \zeta(-\lambda) \quad \text{and} \quad 1 = \zeta(1) = \zeta\pbraces{\frac{\lambda}{\lambda}} = \zeta(\lambda) \zeta(\lambda^{-1}).
	\end{align*}
	Thus, $\zeta(- \lambda) = -\zeta(\lambda)$ and $\zeta\pbraces{\lambda^{-1}}= \zeta(\lambda)^{-1}$. We conclude $\zeta(\beta) = \beta$ for all $\beta \in \Z$ and, in turn, $\zeta(\gamma) = \gamma$ for all $\gamma \in \Q$. For any $\delta \in \R$ there exists a sequence of rational numbers $\pbraces{\gamma_n}_{n \in \N}$ that converges to $\delta$. Due to continuity of $\zeta$ we have 
	\begin{align*}
		\zeta(\delta) = \zeta\pbraces{\lim_{n \to \infty} \gamma_n} = \lim_{n \to \infty} \zeta(\gamma_n) = \lim_{n \to \infty} \gamma_n = \delta.
	\end{align*}
	From
	\begin{align*}
		-1 = \zeta(-1) = \zeta\pbraces{i^2} = \zeta(i)^2
	\end{align*}
	we conclude that $\zeta(i) \in \Bbraces{i, -i}$. For a complex number $\mu = \mu_1 + i \mu_2$, where $\mu_1, \mu_2 \in \R$, we obtain
	\begin{align*}
		\zeta(\mu) = \zeta\pbraces{\mu_1 + i \mu_2} = \zeta\pbraces{\mu_1} + \zeta(i) \zeta\pbraces{\mu_2} = \mu_1 + \zeta(i) \mu_2.
	\end{align*}
	Thus, $\zeta$ is either the identity function or the complex conjugation.
\end{proof}


\begin{figure}[!h]
	\centering
	
	\begin{tikzpicture}[scale=3]
	\draw[->] (-1.5cm,0cm) -- (1.5cm,0cm) node[right,fill=white] {$\Re$};
	\draw[->] (0cm,-1.5cm) -- (0cm,1.5cm) node[above,fill=white] {$\Im$};
	
	\draw[thick] (0cm,0cm) circle(1cm);
	
	\draw (0cm, 0cm) node[anchor=north west] {$0$};
	\draw (1cm, 0cm) node[anchor=north west] {$1$};
	\draw (0cm, 1cm) node[anchor=south east] {$i$};
	\draw (-1cm, 0cm) node[anchor=north west] {$-1$};
	\draw (0cm, -1cm) node[anchor=south east] {$-i$};
	
	\draw (1cm, 0cm) -- node[anchor=north east] {$\sqrt{2}$} (0cm, 1cm) ;
	\draw (1cm, 0cm) -- node[anchor=south east] {$\sqrt{2}$} (0cm, -1cm) ;
	
	\draw[line width=0.5mm, blue, dashed] (1cm, 0cm) -- node[anchor=south east] {$2$} (-1cm, 0cm);
	\end{tikzpicture}
	\caption{Geometric interpretaion of Lemma \ref{lemma:complex_geom}}
\end{figure}


\begin{lemma} \label{lemma:complex_geom}
	For $\lambda \in \C$ with $|\lambda| = 1$ the following statements are true.
	\begin{enumerate}
		\item If $-\lambda$ has euclidean distance $2$ from the complex number $1$ then $\lambda = 1$. Thus,
			\begin{align*}
				|1 + \lambda| = 2 \Rightarrow \lambda = 1.
			\end{align*}
		\item If $-\lambda$ has euclidean distance $\sqrt{2}$ from the complex number $1$ then either $\lambda = i$ or $\lambda = -i$ holds true, i.e.
			\begin{align*}
				|1 + \lambda| = \sqrt{2} \Rightarrow \lambda = i \lor \lambda = -i.
			\end{align*}
	\end{enumerate}
\end{lemma}

\begin{proof}
	For any $\lambda \in \C$, $\vbraces{\lambda} = 1$, we have
	\begin{align*}
		|1 + \lambda|^2 \lambda = (1 + \lambda) \pbraces{1 + \overline{\lambda}} \lambda = (1 + \lambda)^2 = 1 + 2 \lambda + \lambda^2.
	\end{align*}
	\begin{enumerate}
		\item If $|1 + \lambda|^2 = 4$, then we obtain
			\begin{align*}
				0 = 1 - 2 \lambda + \lambda^2 = (1 - \lambda)^2,
			\end{align*} 
		and in turn $\lambda = 1$. 
		\item If $|1 + \lambda|^2 = 2$, then we obtain 
		\begin{align*}
			0 = 1 + \lambda^2.
		\end{align*}
		Hence, $\lambda \in \Bbraces{i, -i}$.  
	\end{enumerate}
\end{proof}

\begin{lemma} \label{lemma:phase_adjustment_complex}
	Let $\mu \in \C \setminus \{0\}$. If a function $\zeta:\C \to \C$ satisfies $\vbraces{\zeta(\mu)} = \vbraces{\mu}$, then there exists a unique $\lambda \in \C$ with $|\lambda| = 1$ such that $|\mu| = \lambda \zeta(\mu)$. 
\end{lemma}

\begin{proof}
	Let $\mu \in \C \setminus \{0\}$ be an arbitrary number. Defining
	\begin{align*}
		\lambda := \frac{|\mu|}{\zeta(\mu)} \quad \text{we have} \quad |\lambda| = \frac{|\mu|}{|\zeta(\mu)|} = \frac{\vbraces{\mu}}{\vbraces{\mu}} = 1
	\end{align*}
	and
	\begin{align*}
		|\mu| = \frac{|\mu|}{\zeta(\mu)} \zeta(\mu) = \lambda \zeta(\mu).
	\end{align*}
	For another $\nu \in \C$ with $|\nu| = 1$ and $|\mu| = \nu \zeta(\mu)$ we obtain
	\begin{align*}
		\nu = \frac{|\mu|}{\zeta(\mu)} = \lambda.
	\end{align*}
\end{proof}

\begin{lemma} \label{lemma:complex_alg}
	Let $\lambda, \mu, \nu \in \C$ where $\lambda \neq 0$ and $|\mu| = |\nu|$. Then the following implication holds true. 
	\begin{align*}
		|\lambda + \nu| = |\lambda + \mu| \land |\lambda - i\nu| = |\lambda - i\mu| \Rightarrow \nu = \mu.
	\end{align*}
\end{lemma}

\begin{proof}
	We have
	\begin{align*}
		|\lambda|^2 + \lambda \overline{\mu} + \overline{\lambda} \mu + \vbraces{\mu}^2 = \pbraces{\lambda + \mu}\pbraces{\overline{\lambda} + \overline{\mu}} = \vbraces{\lambda + \mu}^2 = \vbraces{\lambda + \nu}^2 = \pbraces{\lambda + \nu}\pbraces{\overline{\lambda} + \overline{\nu}} = |\lambda|^2 + \lambda \overline{\nu} + \overline{\lambda} \nu + \vbraces{\nu}^2
	\end{align*}
	and because of $\vbraces{\mu} = \vbraces{\nu}$ we conclude $\lambda\pbraces{\overline{\nu} - \overline{\mu}} = -\overline{\lambda} \pbraces{\nu - \mu}$. Furthermore,
	\begin{align*}
		|\lambda|^2 + i\lambda \overline{\mu} - i\overline{\lambda} \mu + \vbraces{\mu}^2 = \pbraces{\lambda - i\mu}\pbraces{\overline{\lambda} + i\overline{\mu}} = \vbraces{\lambda - i\mu}^2 = \vbraces{\lambda - i\nu}^2 = \pbraces{\lambda - i\nu}\pbraces{\overline{\lambda} + i\overline{\nu}} = |\lambda|^2 + i\lambda \overline{\nu} - i\overline{\lambda} \nu + \vbraces{\nu}^2.
	\end{align*}
	Again because of the assumption $\vbraces{\mu} = \vbraces{\nu}$ we obtain
	\begin{align*}
		\overline{\lambda} \pbraces{\nu - \mu} = \lambda\pbraces{\overline{\nu} - \overline{\mu}} = -\overline{\lambda} \pbraces{\nu - \mu}.
	\end{align*}
	$\overline{\lambda} \neq 0$ finally implies $\nu = \mu$.
\end{proof}