\section{Projective Hilbert spaces}

\begin{definition}
	Let $V$ be a vector space over the field $K$. The set $\mathcal{P}(V) = \{Kx \mid x \in V \setminus \{0\}\}$ consisting of all onedimensional subspaces of $V$ is called the \textit{projective space} of $V$. If $V$ is a Hilbert space then $\mathcal{P}(V)$ is called \textit{projective Hilbert space}. We call the elements of a projective Hilbert space \textit{rays}.
\end{definition}


\begin{lemma} \label{lemma:ray_prod}
	Let $R_1$ and $R_2$ be rays of the projective Hilbert space $\mathcal{P}(H)$. Then there exists a unique $\rho \in [0, 1]$ such that for all $x_1 \in R_1 \setminus \{0\}$ and $x_2 \in R_2 \setminus \{0\}$ the equation
	\begin{align*}
		\frac{\vbraces{(x_1, x_2)_H}}{\norm[H]{x_1} \norm[H]{x_2}} = \rho
	\end{align*}
	holds.
\end{lemma}

\begin{proof}
	Let $x_1, y_1 \in R_1 \setminus \{0\}$ and $x_2, y_2 \in R_2 \setminus \{0\}$. We know that $y_1 = \lambda_1 x_1$ and $y_2 = \lambda_2 x_2$ for some $\lambda_1, \lambda_2 \in \C \setminus \{0\}$. Now we just start calculating and obtain
	\begin{align*}
		\rho := \frac{\vbraces{(y_1, y_2)_H}}{\norm[H]{y_1} \norm[H]{y_2}} = \frac{\vbraces{(\lambda_1 x_1, \lambda_2 x_2)_H}}{\norm[H]{\lambda_1 x_1} \norm[H]{\lambda_2 x_2}} = \frac{\vbraces{\lambda_1 \lambda_2}\vbraces{(x_1, x_2)_H}}{\vbraces{\lambda_1 \lambda_2}\norm[H]{x_1} \norm[H]{x_2}} = \frac{\vbraces{(x_1, x_2)_H}}{\norm[H]{x_1} \norm[H]{x_2}}.
	\end{align*}
	Because of the Cauchy-Schwarz inequality it is clear that $\rho \in [0,1]$. 
\end{proof}


\begin{definition}
	The previous lemma \ref{lemma:ray_prod} allows us to define a \textit{ray-product} on a projective Hilbert space $\mathcal{P}(H)$.
	\begin{align*}
		(\cdot, \cdot)_{\mathcal{P}(H)}: \mathcal{P}(H) \times \mathcal{P}(H) \to [0,1] : (\C x, \C y) \mapsto \frac{\vbraces{(x, y)_H}}{\norm[H]{x} \norm[H]{y}}
	\end{align*}
\end{definition}


\begin{lemma} \label{lemma:projective_metric}
	Let $\mathcal{P}(H)$ be a projective Hilbert space and
	\begin{align*}
		f: \mathcal{P}(H) \to L_b(H): R \mapsto 
		\begin{cases}
			H \to H \\
			x \mapsto (x,v_R)_H v_R
		\end{cases},
	\end{align*} 
	where $v_R \in R$ is a normalized vector. Then for all $R \in \mathcal{P}(H)$ we find that $f(R)$ is the orthogonal projection with $\ran f(R) = R$ and $d: \mathcal{P}(H) \times \mathcal{P}(H) \to [0, \infty): (R,S) \mapsto \norm{f(R) - f(S)}$ is a metric. 
\end{lemma}

\begin{proof}
	First we consider some $R \in \mathcal{P}(H)$ and define $P := f(R): H \to H: x \mapsto (x,v_R)_H v_R$. We find that for every $x \in H$
	\begin{align*}
		P^2x = P(x, v_R)_H v_R = \pbraces{(x, v_R)_H v_R, v_R}_H v_R = (x,v_R)_H v_R = Px
	\end{align*}
	and for all $x,y \in H$
	\begin{align*}
		(Px, y)_H = \pbraces{(x, v_R)_H v_R, y}_H = (x, v_R)_H (v_R, y)_H = \pbraces{x, (y, v_R)_H v_R}_H = (x, Py)_H
	\end{align*}
	and we observe that $P = f(R)$ is a linear function and hence an orthogonal projection. Due to the uniqueness of the orthogonal projection that we know from remark \ref{remark:orth_proj_uniqueness} it is clear that $d$ is a metric.
\end{proof}


\begin{remark}
	Throughout this paper a projective Hilbert space will be endowed with the metric from lemma \ref{lemma:projective_metric}. 
\end{remark}


\begin{lemma} \label{lemma:metric_representation}
	In a projective Hilbert space $\mathcal{P}(H)$ for all rays $R$ and $S$ the equality 
	\begin{align*}
		d(R,S) = \sqrt{1 - (R,S)_{\mathcal{P}(H)}^2}
	\end{align*}
	holds.
\end{lemma}

\begin{proof}
	Let $R,S \in\mathcal{P}(H)$ be arbitrary rays and $P: H \to R: x \mapsto (x, u)_H u$ as well as $Q: H \to S: x \mapsto (x, v)_H v$, where $u \in R$ and $v \in S$ are normalized vectors. If $R = S$ then, using remark \ref{remark:csb}, we easily observe that the equation holds, thus from now on we assume $R \neq S$. Now we are going to have a look at the spectrum of $T: H \to H: x  \mapsto Px - Qx$. According to lemma \ref{lemma:projective_metric} $P$ and $Q$ are orthogonal projections and we observe
	\begin{align*}
		T^\ast = P^\ast - Q^\ast = P - Q = T
	\end{align*}
	hence $T$ is normal. We also observe that $\ran T \subseteq \mathrm{span}\{u, v\} =: W$ thus $\dim \ran T < \infty$ and according to remark \ref{remark:compact}¸ this implies that $T$ is compact.
	
	Let us now assume $\lambda \in \C \setminus \{0\}$ is in the spectrum of $T$. Due to the fact that $T$ is compact we know from remark \ref{remark:compact_spectrum} that $\lambda$ is eigenvalue of $T$ which gives us $T x = \lambda x$ for some $x \in H \setminus \{0\}$. This equation lets us conclude that $x \in \mathrm{span 
	}\ T$ which means there exist $\mu, \nu \in \C$ with $x = \mu u + \nu v$. As we know that $x \neq 0$ we can also conclude that $\mu \neq 0$ or $\nu \neq 0$ and without loss of generality we assume $\mu \neq 0$. Now we start calculating.
	\begin{align*}
		\lambda \mu u + \lambda \nu v = \lambda x = Tx = Px - Qx = \mu Pu + \nu Pv - \mu Qu - \nu Qv = \mu u + \nu Pv - \mu Qu - \nu v .
	\end{align*}
	As $R \neq S$ we know that $u$ and $v$ are linearly independent and $Qu = (u,v)_H v$ as well as $Pv = (v,u)_H u$ hence the two equations
	\begin{align}
		\lambda \mu  = \mu  + \nu (v,u)_H \label{eq:metric1}\\
		 \lambda \nu  = - \nu  - \mu (u,v)_H \label{eq:metric2}
	\end{align}
	must be fulfilled. 
	
	If $(v,u)_H = 0$ then from \eqref{eq:metric1} we conclude that $\lambda \mu = \mu$ and hence $\lambda = 1$. In this case we also observe that $(R,S)_{\mathcal{P}(H)} = |(u,v)_H| = 0$ and hence according to remark \ref{remark:spectral_radius}
	\begin{align*}
		d(R,S) = \norm{P - Q} = r(P - Q) = 1 = \sqrt{1 - (R,S)_{\mathcal{P}(H)}}
	\end{align*}
	which is what we had to show.
	
	Now assuming $(v,u)_H \neq 0$ we can do further calculations. First we use \eqref{eq:metric1} and obtain
	\begin{align*}
		\lambda = \frac{\mu + \nu (v,u)_H}{\mu} = 1 + \frac{\nu}{\mu} (v,u)_H 
	\end{align*}
	which lets us conclude that 
	\begin{align}
		\frac{\nu}{\mu} = \frac{\lambda - 1}{(v,u)_H}. \label{eq:metric3}
	\end{align}
	Now, using \eqref{eq:metric2},we obtain
	\begin{align*}
		 (\lambda + 1) \frac{\nu}{\mu} =  -(u,v)_H 
	\end{align*}
	and plugging in \eqref{eq:metric3} we conclude that
	\begin{align*}
			\frac{\lambda^2 - 1}{(v,u)_H} = (\lambda + 1) \frac{\lambda - 1}{(v,u)_H} =  - (u,v)_H.
	\end{align*}
	With a simple transformation we get
	\begin{align*}
		\lambda = \pm \sqrt{1 - |(u,v)|^2} = \pm \sqrt{1 - (R,S)_{\mathcal{P}(H)}^2}.
	\end{align*}
	Although we do not know for sure whether $0$ is in the spectrum of $T$ we now know the spectral radius of $T$ which finishes the proof because using remark \ref{remark:spectral_radius} we obtain
	\begin{align*}
		d(R,S) = \norm{P - Q} = r(P - Q) = \sqrt{1 - (R,S)_{\mathcal{P}(H)}^2}.
	\end{align*}
\end{proof}


\begin{lemma}
	Let $\mathcal{P}(H_1)$ and $\mathcal{P}(H_2)$ be two projective Hilbert spaces and $g: \mathcal{P}(H_1) \to \mathcal{P}(H_2)$ an isometry. Let furthermore $M := \{e_j \mid j \in J\}$ be an orthonormal basis of $H_1$ and $x,y \in H_1 \setminus \{0\} $. Let $\tilde{x} \in g(\C x)$ and $\tilde{y} \in g(\C y)$  be vectors with $\norm[H_1]{x} = \norm[H_2]{\tilde{x}}$ and $\norm[H_1]{y} = \norm[H_2]{\tilde{y}}$. Lastly, for every $j \in J$ let $\tilde{e}_j \in g(\C e_j)$ be a normalized vector. Then the  following statements are true.
	
	\begin{enumerate}
		\item The equality
		\begin{align} 
			\vbraces{(\tilde{x}, \tilde{y})_{H_1}} = \vbraces{(x, y)_{H_2}} \label{eq:vector_isometry}
		\end{align}
		holds.		
		
		\item The set $L:=\{e_j \mid j \in J\}$ is an orthonormal system of $H_2$.
		
		\item The equality
		\begin{align}
			\tilde{x} = \sum_{j \in J} (\tilde{x}, \tilde{e}_j)_{H_2} \tilde{e}_j. \label{eq:ran_fourier}
		\end{align}
		holds.
		
	\end{enumerate} 
\end{lemma}

\begin{proof}
	We will proof the statements separately.
	\begin{enumerate}
		\item Using lemma \ref{lemma:metric_representation} we obtain
		\begin{align*}
			\sqrt{1 - \dfrac{|(x,y)_{H_1}|^2}{\norm[H_1]{x}^2 \norm[H_1]{y}^2}} &= \sqrt{1 - (\C x,\C y)_{\mathcal{P}(H_1)}^2} = d(\C x,\C y) \\
			&= d\pbraces{g(\C x), g(\C y)} = \sqrt{1 - (g(\C x), g(\C y))_{\mathcal{P}(H_2)}^2} = \sqrt{1 - \dfrac{|(\tilde{x}, \tilde{y})_{H_2}|^2}{\norm[H_2]{\tilde{x}}^2 \norm[H_2]{\tilde{y}}^2}}
		\end{align*}
		and we immediately observe that $|(x,y)_{H_1}| = |(\tilde{x},\tilde{y})_{H_2}|$. 
		
		\item Using what we just proofed \eqref{eq:vector_isometry} and the fact that $M$ is an orthonormal basis of $H_1$ we obtain for every $i,j \in $ the equality
		\begin{align*}
			|(\tilde{e}_i, \tilde{e}_j)_{H_2}| = |(e_i, e_j)_{H_1}| = 
			\begin{cases}
				0 &, \text{if } i \neq j \\
				1 &, \text{if } i = j
			\end{cases}.
		\end{align*}
		Hence $L$ is an orthonormal system.
		
		\item In case $\tilde{x} = 0$ the statement is clearly true. From now on we assume $\tilde{x} \neq 0$. Using \eqref{eq:vector_isometry} and Parzeval's equality \eqref{eq:parzeval} we obtain
		\begin{align*}
			\norm[H_2]{\tilde{x}}^2 = \norm[H_1]{x}^2 = \sum_{i \in I} |(x, e_i)_{H_1}|^2 = \sum_{i \in I} |(\tilde{x}, \tilde{e}_i)_{H_2}|^2.
		\end{align*}
		Because $L$ is an orthonormal system we can now use \eqref{eq:parceval_to_fourier} and immediately obtain \eqref{eq:ran_fourier}, which is just what we wanted to show.
	\end{enumerate}
\end{proof}