\section{Projective Hilbert spaces}

\begin{definition}
	Let $V$ be a vector space over the field $K$. The set $\mathcal{P}(V) = \{Kx \mid x \in V \setminus \{0\}\}$ consisting of all onedimensional subspaces of $V$ is called the \textit{projective space} of $V$. If $V$ is a Hilbert space then we call $\mathcal{P}(V)$ \textit{projective Hilbert space} and its elements \textit{rays}.
\end{definition}


\begin{lemma} \label{lemma:ray_prod}
	If $R_1$ and $R_2$ are rays of a projective Hilbert space $\mathcal{P}(H)$, then there exists a unique $\rho \in [0, 1]$ such that for all $x_1 \in R_1 \setminus \{0\}$ and $x_2 \in R_2 \setminus \{0\}$
	\begin{align*}
		\frac{\vbraces{(x_1, x_2)}}{\norm[]{x_1} \norm[]{x_2}} = \rho.
	\end{align*}
\end{lemma}

\begin{proof}
	Let $x_1, y_1 \in R_1 \setminus \{0\}$ and $x_2, y_2 \in R_2 \setminus \{0\}$. Then we can write $y_1 = \lambda_1 x_1$ and $y_2 = \lambda_2 x_2$ for some $\lambda_1, \lambda_2 \in \C \setminus \{0\}$. Hence,
	\begin{align*}
		\rho := \frac{\vbraces{(y_1, y_2)}}{\norm[]{y_1} \norm[]{y_2}} = \frac{\vbraces{(\lambda_1 x_1, \lambda_2 x_2)_H}}{\norm[]{\lambda_1 x_1} \norm[]{\lambda_2 x_2}} = \frac{\vbraces{\lambda_1 \lambda_2}\vbraces{(x_1, x_2)}}{\vbraces{\lambda_1 \lambda_2}\norm[]{x_1} \norm[]{x_2}} = \frac{\vbraces{(x_1, x_2)}}{\norm[]{x_1} \norm[]{x_2}}.
	\end{align*}
	Because of the Cauchy-Schwarz inequality we have $\rho \in [0,1]$. 
\end{proof}


\begin{definition}
	The previous Lemma \ref{lemma:ray_prod} allows us to define the \textit{ray-product} $(\cdot, \cdot)_{\mathcal{P}(H)}: \mathcal{P}(H) \times \mathcal{P}(H) \to [0,1]$ on a projective Hilbert space $\mathcal{P}(H)$ by
	\begin{align*}
		 (\C x, \C y)_{\mathcal{P}(H)} := \frac{\vbraces{(x, y)}}{\norm[]{x} \norm[]{y}}.
	\end{align*}
\end{definition}


\begin{lemma} \label{lemma:projective_metric}
	Let $\mathcal{P}(H)$ be a projective Hilbert space and let $f: \mathcal{P}(H) \to L_b(H)$ be defined by $f(R)(x) := (x,v_R)v_R$ for $x \in H$ and $R \in \mathcal{P}(H)$, where $v_R \in R$ is a vector of norm one. Then for all $R \in \mathcal{P}(H)$ the operator $f(R)$ is the orthogonal projection with $\ran f(R) = R$. Moreover, $d: \mathcal{P}(H) \times \mathcal{P}(H) \to [0, \infty)$ defined by $(R,S) \mapsto \norm{f(R) - f(S)}$ is a metric. 
\end{lemma}

\begin{proof}
	Given $u,v \in R$ with $\norm{u} = \norm{v}$ we have $v = \lambda u$ for some $\lambda \in \C$ with $\vbraces{\lambda} = 1$. Hence,
	\begin{align*}
		\pbraces{x,v}v = \pbraces{x, \lambda u} \lambda u = \lambda \overline{\lambda} \pbraces{x, u} u = \pbraces{x, u} u 
	\end{align*}
	for $x \in H$. Thus, $f(R): H \to H$ is an obviously linear operator which does not depend on the choice of $v_R$. Moreover,
	\begin{align*}
		P^2x = P\pbraces{(x, v_R) v_R} = \pbraces{(x, v_R) v_R, v_R} v_R = (x,v_R) v_R = Px.
	\end{align*}
	Since also for $x,y \in H$ we have
	\begin{align*}
		(Px, y) = \pbraces{(x, v_R) v_R, y} = (x, v_R) (v_R, y) = \pbraces{x, (y, v_R) v_R} = (x, Py),
	\end{align*}
	the operator $P = f(R)$ is an orthogonal projection. The map $d$ is a metric, because the operator norm induces a metric in the well known way. 
\end{proof}


\begin{remark}
	Throughout this paper a projective Hilbert space will be endowed with the metric from Lemma \ref{lemma:projective_metric}. 
\end{remark}


\begin{lemma} \label{lemma:metric_representation}
	In a projective Hilbert space $\mathcal{P}(H)$ the equality 
	\begin{align*}
		d(R,S) = \sqrt{1 - (R,S)_{\mathcal{P}(H)}^2}
	\end{align*}
	holds true for all rays $R, S \in \mathcal{P}\pbraces{H}$.
\end{lemma}

\begin{proof}
	Let $R,S \in\mathcal{P}(H)$ be arbitrary rays and let $P := f(R)$ and $Q:= f(S)$ be the orthogonal projections onto $R$ and $S$ respectively as in Lemma \ref{lemma:projective_metric} and set $u := v_R$ and $v := v_S$. If $R = S$ then the equation holds true because of Remark \ref{remark:csb}. Thus, from now on we assume $R \neq S$. We are going to have a look at the spectrum of $T: H \to H$ defined by $Tx = Px - Qx = (x,u)u - (x,v)v$. From
	\begin{align*}
	T\pbraces{u - \frac{\pbraces{u,v}}{1 + \sqrt{1 - \vbraces{\pbraces{u,v}}^2}}v} &= Tu - \frac{\pbraces{u,v}}{1 + \sqrt{1 - \vbraces{\pbraces{u,v}}^2}} Tv = u - (u,v)v - \frac{\pbraces{u,v}}{1 + \sqrt{1 - \vbraces{\pbraces{u,v}}^2}} \pbraces{(v,u) u - v} \\
	&= \pbraces{1 - \frac{\vbraces{\pbraces{u,v}}^2}{1 + \sqrt{1 - \vbraces{\pbraces{u,v}}^2}}}u - \pbraces{(u,v) - \frac{\pbraces{u,v}}{1 + \sqrt{1 - \vbraces{\pbraces{u,v}}^2}}}v \\
	&= \frac{1 - \vbraces{\pbraces{u,v}}^2 + \sqrt{1 - \vbraces{\pbraces{u,v}}^2}}{1 + \sqrt{1 - \vbraces{\pbraces{u,v}}^2}} u - \frac{(u,v) \sqrt{1 - \vbraces{\pbraces{u,v}}^2}}{1 + \sqrt{1 - \vbraces{\pbraces{u,v}}^2}}v \\
	&= \sqrt{1 - \vbraces{\pbraces{u,v}}^2} \pbraces{u - \frac{\pbraces{u,v}}{1 + \sqrt{1 - \vbraces{\pbraces{u,v}}^2}}v}
	\end{align*}
	we conclude that $\sqrt{1 - \vbraces{\pbraces{u,v}}^2} = \sqrt{1 - (R,S)_{\mathcal{P}(H)}^2}$ is an eigenvalue of $T$. As
	\begin{align*}
		T^\ast = P^\ast - Q^\ast = P - Q = T
	\end{align*}
	the operator $T$ is selfadjoint and therefore normal. We also observe that $\ran T \subseteq \mathrm{span}\{u, v\}$. According to Remark \ref{remark:compact} this implies that $T$ is compact.  
	Let us now assume that $\lambda \in \C \setminus \{0\}$ belongs to the spectrum of $T$. Due to the fact that $T$ is compact and by Remark \ref{remark:compact_spectrum} the complex number $\lambda$ is eigenvalue of $T$, which gives $T x = \lambda x$ for some $x \in H \setminus \{0\}$. Hence, $x \in \ran T$ which implies the existence of $\mu, \nu \in \C$ with $x = \mu u + \nu v$. From $x \neq 0$ we conclude $\mu \neq 0$ or $\nu \neq 0$. With no loss of generality we assume $\mu \neq 0$.
	
	As $R \neq S$, the vectors $u$ and $v$ are linearly independent. We conclude
	\begin{align}
		\lambda \mu  = \mu  + \nu (v,u) \label{eq:metric1}\\
		 \lambda \nu  = - \nu  - \mu (u,v) \label{eq:metric2}
	\end{align}
	from
	\begin{align*}
		\lambda \mu u + \lambda \nu v = \lambda x = Tx = \mu Tu + \nu Tv = \mu \pbraces{u - (u,v)v} + \nu \pbraces{(v,u)u - v} = \pbraces{\mu + \nu (v,u)} u - \pbraces{\nu + \mu(u,v)} v. 
	\end{align*}
	
	If $(v,u) = 0$, then \eqref{eq:metric1} yields $\lambda \mu = \mu$ and hence $\lambda = 1$. In this case we have $(R,S)_{\mathcal{P}(H)} = |(u,v)_H| = 0$ which, according to Remark \ref{remark:spectral_radius}, yields 
	\begin{align*}
		d(R,S) = \norm{P - Q} = r(P - Q) = 1 = \sqrt{1 - (R,S)_{\mathcal{P}(H)}}.
	\end{align*}
	
	Assuming $(v,u) \neq 0$ we can do further calculations. From \eqref{eq:metric1} we conclude that
	\begin{align*}
		\lambda = \frac{\mu + \nu (v,u)}{\mu} = 1 + \frac{\nu}{\mu} (v,u) 
	\end{align*}
	and hence
	\begin{align}
		\frac{\nu}{\mu} = \frac{\lambda - 1}{(v,u)}. \label{eq:metric3}
	\end{align}
	Using \eqref{eq:metric2} we obtain
	\begin{align*}
		 (\lambda + 1) \frac{\nu}{\mu} =  -(u,v), 
	\end{align*}
	what together with \eqref{eq:metric3} implies
	\begin{align*}
			\frac{\lambda^2 - 1}{(v,u)} = (\lambda + 1) \frac{\lambda - 1}{(v,u)} =  - (u,v).
	\end{align*}
	With a simple transformation we find
	\begin{align*}
		\lambda = \pm \sqrt{1 - |(u,v)|^2} = \pm \sqrt{1 - (R,S)_{\mathcal{P}(H)}^2}.
	\end{align*}
	Although we do not know for sure, whether $0$ belongs to the spectrum of $T$, we know its spectral radius $\vbraces{\lambda}$. Finally, by Remark \ref{remark:spectral_radius} we obtain
	\begin{align*}
		d(R,S) = \norm{P - Q} = r(P - Q) = \sqrt{1 - (R,S)_{\mathcal{P}(H)}^2}.
	\end{align*}
\end{proof}


\begin{lemma}
	Let $\mathcal{P}(H_1)$ and $\mathcal{P}(H_2)$ be two projective Hilbert spaces and $g: \mathcal{P}(H_1) \to \mathcal{P}(H_2)$ an isometry with respect to the metric from Lemma \ref{lemma:projective_metric}. Let $M := \{e_j \mid j \in J\}$ be an orthonormal basis of $H_1$ and $x,y \in H_1 \setminus \{0\} $. Let $\tilde{x} \in g(\C x)$ and $\tilde{y} \in g(\C y)$  be vectors satisfying $\norm[H_1]{x} = \norm[H_2]{\tilde{x}}$ and $\norm[H_1]{y} = \norm[H_2]{\tilde{y}}$. Lastly, for every $j \in J$ let $\tilde{e}_j \in g(\C e_j)$ be a normalized vector. Then
	\begin{align} 
		\vbraces{(\tilde{x}, \tilde{y})_{H_1}} = \vbraces{(x, y)_{H_2}} \label{eq:vector_isometry},
	\end{align}
	the set $L:=\{\tilde{e}_j \mid j \in J\}$ is an orthonormal system in $H_2$, and 
	\begin{align}
	\tilde{x} = \sum_{j \in J} (\tilde{x}, \tilde{e}_j)_{H_2} \tilde{e}_j. \label{eq:ran_fourier}
	\end{align} 
\end{lemma}

\begin{proof}
	Employing Lemma \ref{lemma:metric_representation} we obtain
	\begin{align*}
		\sqrt{1 - \dfrac{|(x,y)_{H_1}|^2}{\norm[H_1]{x}^2 \norm[H_1]{y}^2}} &= \sqrt{1 - (\C x,\C y)_{\mathcal{P}(H_1)}^2} = d(\C x,\C y) \\
		&= d\pbraces{g(\C x), g(\C y)} = \sqrt{1 - (g(\C x), g(\C y))_{\mathcal{P}(H_2)}^2} = \sqrt{1 - \dfrac{|(\tilde{x}, \tilde{y})_{H_2}|^2}{\norm[H_2]{\tilde{x}}^2 \norm[H_2]{\tilde{y}}^2}}
	\end{align*}
	which immediately implies \eqref{eq:vector_isometry}. Using this equation and the fact that $M$ is an orthonormal basis of $H_1$ we obtain for every $i,j \in J$ 
	\begin{align*}
	|(\tilde{e}_i, \tilde{e}_j)_{H_2}| = |(e_i, e_j)_{H_1}| = 
	\begin{cases}
	0 &, \text{if } i \neq j, \\
	1 &, \text{if } i = j.
	\end{cases}
	\end{align*}
	Hence, $L$ is an orthonormal system. Using \eqref{eq:vector_isometry} and Parzeval's equality \eqref{eq:parzeval} we obtain
	\begin{align*}
	\norm[H_2]{\tilde{x}}^2 = \norm[H_1]{x}^2 = \sum_{j \in J} |(x, e_j)_{H_1}|^2 = \sum_{j \in J} |(\tilde{x}, \tilde{e}_j)_{H_2}|^2.
	\end{align*}
	Due to the fact that $L$ is an orthonormal system by \eqref{eq:parceval_to_fourier} we obtain \eqref{eq:ran_fourier}.
\end{proof}