\section{Concluding remarks}
The proof given here is not particularly short and it involves quite a few calculations. Despite these drawbacks it has the merit that it proves a very general form of Wigner's theorem where the two Hilbert spaces involved can be different ones and do not have to be separable. Furthermore, we constructed the desired function step by step in the proof which might be very insightful and we did not have to use very deep mathematical results. Finally, it is worth mentioning that the paper does give a lot of detailed calculations which should make it easy to read.