% --------------------------------------------------------------------------------

\begin{exercise}[220]

\phantom{}
	Wenn $f: \N^k \to \N$ eine totale Funktion ist, die (als Relation) eine $\Sigma_1$-Menge ist, dann ist $f$ auch $\Delta_1$ (d.h., die Menge $(\N^k \times \N) \setminus f$ ist auch $\Sigma_1$).

\end{exercise}

% --------------------------------------------------------------------------------

\begin{solution}

\begin{align*}
	(x_1,\dots,x_k,y) \in (\N^k \times \N) \setminus f \iff
	\exists x^{\prime}_1\, \cdots \exists x^{\prime}_k\, \exists  y^{\prime}\,
	(x^{\prime}_1,\dots,x^{\prime}_k,y^{\prime}) \in f \land x_1 = x^{\prime}_1
	\land \dots x_k = x^{\prime}_k \land y \neq y^{\prime}.
\end{align*}

\end{solution}
