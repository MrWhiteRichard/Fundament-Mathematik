% --------------------------------------------------------------------------------
\subsection*{72//73//74}

\begin{exercise}[72]

\phantom{}
	Zeigen Sie, dass $\mathscr{S}$ unter Durchschnitt abgeschlossen ist. (Anleitung: Sei $A = Sp(\varphi_1)$, $B = Sp(\varphi_2)$. Erklären Sie, warum Sie ohne Beschränkung der Allgemeinheit annehmen dürfen, dass die Sprachen zu $\varphi_1$ und $\varphi_2$ disjunkt sind...)

\end{exercise}

% --------------------------------------------------------------------------------

\begin{solution}

\phantom{}
	Da $\varphi_1$ und $\varphi_2$ nur endlich viele Funktionssymbole, Relationssymbole, Konstantensymbole und Variable verwenden können wir ohne Beschränkung der Allgemeinheit annehmen, dass die Sprachen zu den beiden Formeln disjunkt sind.\newline
	Sei nun $n \in A \cap B$. Wähle ein Modell $\mathfrak{M}_1$ mit $\vbraces{M_1} = n$ und $\mathfrak{M}_1 \vDash \varphi_1$ sowie ein Modell $\mathfrak{M}_2$ mit $\vbraces{M_2} = n$ und $\mathfrak{M}_2 \vDash \varphi_2$. Wir können ohne Beschränkung der Allgemeinheit sagen, dass $M_1 = M_2$ gilt, da es eine Bijektion zwischen den beiden Mengen gibt. Wegen der Disjunktheit der Sprachen können wir nun die Modelle \Quote{vereinigen}: 
	\begin{align*}
	\mathfrak{M} := \mathfrak{M}_1 \cup \mathfrak{M}_2
	\end{align*}
	Wobei es keine Probleme bei den Interpretationen gibt. Wir erhalten nun klarerweise $\mathfrak \vDash \varphi_1 \land \varphi_2$.  \newline
	Betrachten wir umgekehrt $n \in Sp(\varphi_1 \land \varphi_2)$ dann gibt es ein Modell $\mathfrak{M}$ mit $\vbraces{M} = n$ und $\mathfrak{M} \vDash \varphi_1 \land \varphi_2$ und damit gilt klarerweise auch $\mathfrak{M} \vDash \varphi_1$ und $\mathfrak{M} \vDash \varphi_2$. 
\end{solution}

\begin{exercise}[73]

\phantom{}
	Zeigen Sie, dass $\mathscr{S}$ unter Vereinigung abgeschlossen ist.

\end{exercise}

% --------------------------------------------------------------------------------

\begin{solution}

\phantom{}

\end{solution}

\begin{exercise}[74]

\phantom{}
	Zeigen Sie, dass $\mathscr{S}$ unter Komplement abgeschlossen ist.

\end{exercise}

% --------------------------------------------------------------------------------

\begin{solution}

\phantom{}

\end{solution}
