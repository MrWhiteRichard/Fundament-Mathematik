\documentclass{article}

% Hier befinden sich Pakete, die wir beinahe immer benutzen ...

\usepackage[utf8]{inputenc}

% Sprach-Paket:
\usepackage[ngerman]{babel}

% damit's nicht so, wie beim Grill aussieht:
\usepackage{fullpage}

% Mathematik:
\usepackage{amsmath, amssymb, amsfonts, amsthm}
\usepackage{bbm, mathrsfs, stmaryrd}
\usepackage{mathtools, mathdots}

% Makros mit mehereren Default-Argumenten:
\usepackage{twoopt}

% Anführungszeichen (Makro \Quote{}):
\usepackage{babel}

% if's für Makros:
\usepackage{xifthen}
\usepackage{etoolbox}

% tikz ist kein Zeichenprogramm (doch!):
\usepackage{tikz}

% bessere Aufzählungen:
\usepackage{enumitem}

% (bessere) Umgebung für Bilder:
\usepackage{graphicx, subfig, float}

% Umgebung für Code:
\usepackage{listings}

% Farben:
\usepackage{xcolor}

% Umgebung für "plain text":
\usepackage{verbatim}

% Umgebung für mehrerer Spalten:
\usepackage{multicol}

% "nette" Brüche
\usepackage{nicefrac}

% Spaltentypen verschiedener Dicke
\usepackage{tabularx}
\usepackage{makecell}

% Für Vektoren
\usepackage{esvect}

% (Web-)Links
\usepackage{hyperref}

% Zitieren & Literatur-Verzeichnis
\usepackage[style = authoryear]{biblatex}
\usepackage{csquotes}

% so ähnlich wie mathbb
%\usepackage{mathds}

% Keine Ahnung, was das macht ...
\usepackage{booktabs}
\usepackage{ngerman}
\usepackage{placeins}

% special letters:

\newcommand{\N}{\mathbb{N}}
\newcommand{\Z}{\mathbb{Z}}
\newcommand{\Q}{\mathbb{Q}}
\newcommand{\R}{\mathbb{R}}
\newcommand{\C}{\mathbb{C}}
\newcommand{\K}{\mathbb{K}}
\newcommand{\T}{\mathbb{T}}
\newcommand{\E}{\mathbb{E}}
\newcommand{\V}{\mathbb{V}}
\renewcommand{\P}{\mathbb{P}}
\newcommand{\1}{\mathbbm{1}}

\newcommand  {\B}{\mathfrak{B}}
\renewcommand{\S}{\mathfrak{S}}

% quantors:

\newcommand{\Forall}{\forall \,}
\newcommand{\Exists}{\exists \,}
\newcommand{\ExistsOnlyOne}{\exists! \,}
\newcommand{\nExists}{\nexists \,}

% MISC symbols:

\newcommand{\landau}[1]
{
  {\scriptstyle \mathcal{O}}
  \pbraces{#1}
}

\newcommand{\Landau}[1]
{
  \mathcal{O}
  \pbraces{#1}
}

\newcommand{\eps}{\mathrm{eps}}

% graphics in a box:

\newcommandtwoopt
{\includegraphicsboxed}[3][][]
{
  \begin{figure}[!h]
    \begin{boxedin}
      \ifthenelse{\isempty{#2}}
      {
        \begin{center}
          \includegraphics[width = 0.75 \textwidth]{#3}
          \label{fig:#1}
        \end{center}
      }{
        \begin{center}
          \includegraphics[width = 0.75 \textwidth]{#3}
          \caption{#2}
          \label{fig:#1}
        \end{center}
      }
    \end{boxedin}
  \end{figure}
}

% braces:

\newcommand{\pbraces}[1]{{\left  ( #1 \right  )}}
\newcommand{\bbraces}[1]{{\left  [ #1 \right  ]}}
\newcommand{\Bbraces}[1]{{\left \{ #1 \right \}}}
\newcommand{\vbraces}[1]{{\left  | #1 \right  |}}
\newcommand{\Vbraces}[1]{{\left \| #1 \right \|}}
\newcommand{\abraces}[1]{{\left \langle #1 \right \rangle}}
\newcommand{\round}[1]{\bbraces{#1}}

\newcommand
{\floor}[1]
{{\left \lfloor #1 \right \rfloor}}

\newcommand
{\ceil} [1]
{{\left \lceil  #1 \right \rceil }}

% special functions:

\newcommand{\norm}  [2][]{\Vbraces{#2}_{#1}}
\newcommand{\diag}  [1]{\mathrm{diag} \: #1}
\newcommand{\dist}  [1]{\mathrm{dist} \: #1}
\newcommand{\mean}  [1]{\mathrm{mean} \: #1}
\newcommand{\erf}   [1]{\mathrm{erf} \: #1}
\newcommand{\id}    [1]{\mathrm{id} \: #1}
\newcommand{\sgn}   [1]{\mathrm{sgn} \: #1}
\newcommand{\supp}  [1]{\mathrm{supp} \: #1}
\newcommand{\arsinh}[1]{\mathrm{arsinh} \: #1}
\newcommand{\arcosh}[1]{\mathrm{arcosh} \: #1}
\newcommand{\artanh}[1]{\mathrm{artanh} \: #1}
\newcommand{\card}  [1]{\mathrm{card} \: #1}
\newcommand{\Span}  [1]{\mathrm{span} \: #1}
\newcommand{\Aut}   [1]{\mathrm{Aut} \: #1}
\newcommand{\End}   [1]{\mathrm{End} \: #1}
\newcommand{\ggT}   [1]{\mathrm{ggT} \: #1}
\newcommand{\kgV}   [1]{\mathrm{kgV} \: #1}
\newcommand{\ord}   [1]{\mathrm{ord} \: #1}
\newcommand{\grad}  [1]{\mathrm{grad} \: #1}
\newcommand{\ran}   [1]{\mathrm{ran} \: #1}
\newcommand{\graph} [1]{\mathrm{graph} \: #1}
\newcommand{\Inv}   [1]{\mathrm{Inv} \: #1}
\newcommand{\pv}    [1]{\mathrm{pv} \: #1}
\newcommand{\Mod}{\: \mathrm{mod} \:}
\newcommand{\Char}{\mathrm{char}}
\newcommand{\At}{\mathrm{At}}
\newcommand{\Ob}{\mathrm{Ob}}
\newcommand{\Hom}{\mathrm{Hom}}
\newcommand{\orthogonal}[3][]{#2 ~\bot_{#1}~ #3}
\newcommand{\Rang}{\mathrm{Rang}}

\newcommand
{\GL}[2][]
{\mathrm{GL}_{#1} \pbraces{#2}}

% fractions:

\newcommand{\Frac}[2]{\frac{1}{#1} \pbraces{#2}}
\newcommand{\nfrac}[2]{\nicefrac{#1}{#2}}

% derivatives & integrals:

\newcommandtwoopt
{\Int}[4][][]
{\int_{#1}^{#2} #3 ~\mathrm{d} #4}

\newcommandtwoopt
{\derivative}[3][][]
{
  \frac
  {\mathrm{d}^{#1} #2}
  {\mathrm{d} #3^{#1}}
}

\newcommandtwoopt
{\pderivative}[3][][]
{
  \frac
  {\partial^{#1} #2}
  {\partial #3^{#1}}
}

\newcommand
{\primeprime}
{{\prime \prime}}

\newcommand
{\primeprimeprime}
{{\prime \prime \prime}}

% Text:

\newcommand{\Quote}[1]{\glqq #1\grqq{}}
\newcommand{\Text}[1]{{\text{#1}}}
\newcommand{\fastueberall}{\text{f.ü.}}
\newcommand{\fastsicher}{\text{f.s.}}

% -------------------------------- %
% amsthm-stuff:

\theoremstyle{definition}

% numbered theorems
\newtheorem{theorem}    {Satz}   [section]
\newtheorem{lemma}      [theorem]{Lemma}
\newtheorem{corollary}  [theorem]{Korollar}
\newtheorem{proposition}[theorem]{Proposition}
\newtheorem{remark}     [theorem]{Bemerkung}
\newtheorem{definition} [theorem]{Definition}
\newtheorem{example}    [theorem]{Beispiel}

% unnumbered theorems
\newtheorem*{theorem*}    {Satz}
\newtheorem*{lemma*}      {Lemma}
\newtheorem*{corollary*}  {Korollar}
\newtheorem*{proposition*}{Proposition}
\newtheorem*{remark*}     {Bemerkung}
\newtheorem*{definition*} {Definition}
\newtheorem*{example*}    {Beispiel}

% Please define this stuff in project ("main.tex"):

% \def \lastexercisenumber {...}
% This will be 0 by default

% \setcounter{section}{...}
% This will be 0 by default
% and hence, completely ignored

\ifnum \thesection = 0
{
  \newtheorem{exercise}{Aufgabe}
}
\else
{
  \newtheorem{exercise}{Aufgabe}[section]
}
\fi

\ifdef
{\lastexercisenumber}
{\setcounter{exercise}{\lastexercisenumber}}

\newenvironment{solution}
{
  \begin{proof}[Lösung]
}{
  \end{proof}
}

\renewcommand{\proofname}{Beweis}

% -------------------------------- %
% environment zum einkasteln:

% dickere vertical lines
\newcolumntype
{x}
[1]
{
  !{
    \centering
    \arraybackslash
    \vrule
    width #1}
}

% environment selbst (the big cheese)
\newenvironment
{boxedin}
{
  \begin{tabular}
  {
    x{1 pt}
    p{\textwidth}
    x{1 pt}
  }
  \Xhline
  {2 \arrayrulewidth}
}
{
  \\
  \Xhline{2 \arrayrulewidth}
  \end{tabular}
}

% -------------------------------- %
% MISC "Ein-Deutschungen"

\renewcommand{\figurename}{Abbildung}
\renewcommand{\tablename} {Tabelle}

% -------------------------------- %


\parindent 0pt

\title
{
  Logik und Grundlagen der Mathematik \\
  \vspace{4pt}
  \normalsize
  \textit{10. Übung am 10.12.2020}
}
\author
{
  Richard Weiss
  \and
  Florian Schager
  \and
  Fabian Zehetgruber
}
\date{}

\begin{document}

\maketitle

\section*{Berechenbare Funktionen und (semi-)entscheidbare Mengen}

Die Menge der $\mu$-rekursiven Funktionen ist die kleinste Menge von (möglicherweise partiellen)
Funktionen, die alle primitiv rekursiven Funktionen enthält, unter Komposition und
primitiver Rekursion abgeschlossen ist, und außerdem Folgendes erfüllt:

\begin{adjustwidth}{1cm}{}
Wenn $f: \N^k \times \N$ total und $\mu$-rekursiv ist, \\
dann ist die partielle Funktion $\vv{x} \mapsto \min\{y: f(\vv{x},y) = 0\}$ auch
$\mu$-rekursiv.
\end{adjustwidth}

\begin{exercise}
Betrachten Sie für $r \in \R$ die autonome ODE
\begin{align*}
  y^{\prime} = y + \tanh(ry).
\end{align*}
Wieviele Ruhelagen hat die Gleichung abhängig von $r$? Um welchem Typ von Bifurkation
handelt es sich beim Bifurkationspunkt $r = 1$?
\end{exercise}
\begin{solution}
Bestimmen wir zuerst die Ruhelagen der ODE:
Für alle $r$ sehen wir aufgrund $\tanh(0) = 0$, dass $y^* = 0$ eine Ruhelage der ODE
sein muss.
Die Nullstellen direkt zu berechnen sieht eher schwierig aus, daher betrachten wir
einmal die Extremwerte.
\begin{align*}
  f^{\prime}(y) = 1 + r(1 - \tanh(ry)^2) = 1 + r - r\tanh(ry)^2.
\end{align*}
Fall 1: $r \geq 0:$
\begin{align*}
  f^{\prime}(y) \geq 1
\end{align*}
Fall 2: $-1 < r < 0$:
\begin{align*}
  f^{\prime}(y) > -r\tanh(ry)^2 \geq 0
\end{align*}
Fall 3: $r = -1$:
\begin{align*}
  f^{\prime}(y) &\geq 0 \\
  f^{\prime}(0) &= 0.
\end{align*}
In allen Fällen ist die Funktion also streng monoton steigend und kann damit
nur die Nullstelle $y^* = 0$ besitzen. \\
Fall 4: $r < -1$:
\begin{align*}
  f^{\prime}(0) &= 1 + r - r\tanh(0)^2 = 1 + r < 0 \\
  \lim_{y \to \pm\infty}f^{\prime}(y) &= 1 + r - r = 1 > 0 \\
\end{align*}
Also ist $f$ in einer Umgebung von $0$ streng monton fallend und es exisitieren
$y_0 > 0: f(y_0) < 0$ und $y_1 < 0: f(y_1) > 0$. Gleichzeitig gilt
\begin{align*}
  \lim_{y \to \infty} f(y) &= \infty \\
  \lim_{y \to -\infty} f(y) & -\infty.
\end{align*}
Laut dem Zwischenwertsatz hat $f$ für $r < -1$ also mindestens drei Nullstellen.
Seien $y_-^*,y_+^*$ die erste negative, beziehungsweise positive Nullstelle von $f$.
Dann folgt für $y > y_+^*$ aufgrund der Monotonie des Tangens Hyperbolicus:
\begin{align*}
  f(y) = y + \tanh(ry) > y_+^* + \tanh(ty_+^*) = 0.
\end{align*}
Analog folgt für $y < y_-^*: f(y) < 0$. Also kann $f$ nur genau diese drei Nullstellen haben. \\
Der Bifurkationspunkt ist also $r = -1$, in welchem aus einer Ruhelage 3 Ruhelagen werden.
Das nennt man auch \glqq supercritical pitchfork bifurcation \grqq. \\
Jetzt kann man noch die Stabilität der Ruhelagen untersuchen, bin mir aber
nicht sicher, ob das gefordert ist.
\FloatBarrier
\begin{figure}
    \centering
    \includegraphics[width=\linewidth]{bifurcation_plot.png}
    \caption{Ruhelagen in Abhängigkeit von $r$}
\end{figure}
\FloatBarrier
\end{solution}

\pagebreak
% --------------------------------------------------------------------------------

\begin{exercise}
\textit{(Periodic Sobolev Spaces)} Let $\Omega=(0,2 \pi)$ and consider the
complete orthonormal system of $L^{2}(\Omega)$ given by
\begin{align*}
  \left\{C_{0}=\frac{1}{\sqrt{2 \pi}}, C_{n}(x)=\frac{1}{\sqrt{\pi}} \cos (n x), S_{n}(x)=\frac{1}{\sqrt{\pi}} \sin (n x) \,\Bigg|\, n \in \mathbb{N}\right\}.
\end{align*}
\begin{enumerate}[label = (\roman*)]
  \item Show that for $k \in \mathbb{N}$ the space
  \begin{align*}
      H_{per}^{k}(\Omega):=\left\{f \in H^{k}(\Omega) \mid f^{(j)}(0)=f^{(j)}(2 \pi) \text { for } j=0, \ldots, k-1\right\}
  \end{align*}
  is a well-defined Hilbert space.
  \item Show that $f \in H_{\text {per}}^{1}(\Omega)$ if and only if
  \begin{align*}
      f=\sum_{m=1}^{\infty} a_{m} S_{m}+\sum_{m=0}^{\infty} b_{m} C_{m} \quad \text { with } \quad
      \sum_{m=1}^{\infty} m^{2}\left(\left|a_{m}\right|^{2}+\left|b_{m}\right|^{2}\right)<\infty.
  \end{align*}
  In this case, $f$ can be differentiated \glqq term-wise\grqq.
  \item For $n \in \mathbb{N}$ consider the projection

  \begin{align*}
    P_{n}&: H_{p e r}^{k}(\Omega) \rightarrow H_{p e r}^{k}(\Omega) \\
    f&=\sum_{m=1}^{\infty} a_{m} S_{m}+\sum_{m=0}^{\infty} b_{m} C_{m}
    \mapsto P_{n} f=\sum_{m=1}^{n} a_{m} S_{m}+\sum_{m=0}^{n} b_{m} C_{m}.
  \end{align*}

  Show that for $f \in H_{\text {per}}^{k}(\Omega)$ it holds that
  \begin{align*}
  \left\|f-P_{n} f\right\|_{L^{2}(\Omega)} \leq \frac{1}{(n+1)^{k}}\left\|f^{(k)}\right\|_{L^{2}(\Omega)}.
  \end{align*}

\end{enumerate}
\end{exercise}

% --------------------------------------------------------------------------------

\begin{solution}

\phantom{}

\end{solution}

% --------------------------------------------------------------------------------

\begin{algebraUE}{336}

Zeigen Sie, dass der Ring $K[[x]]$ der formalen Potenzreihen über einem Körper $K$ euklidisch, folglich auch ein Hauptidealring und faktoriell ist. Bestimmen Sie alle irreduziblen Elemente modulo Assoziiertheit und geben Sie sämtliche Ideale durch Erzeugende an, jedes genau einmal.

\end{algebraUE}

\begin{solution}

Für die euklidische Bewertung wählen wir in dem Fall

\begin{align*}
  H:K[[x]] \setminus \{0\} \to \N: p \mapsto \ord(p)
\end{align*}

wohldefiniert, da für $p \in K[[x]] \setminus \{0\}$, $\ord(p) \geq 0$.

Wir müssen noch zeigen, dass Division mit Rest möglich ist, Also

\begin{align*}
  \Forall g \in K[[x]] \setminus \{0\}, f \in K[[x]] \Exists q,r \in K[[x]]: f = gq + r \text{~mit~} H(r) < H(g).
\end{align*}

Seien also $g \in K[[x]] \setminus \{0\}, f \in K[[x]]$ beliebig, dann brauchen wir eine Fallunterscheidung

Fall 1: $\ord(f) < \ord(g) \lor f = 0$:

Dann gilt

\begin{align*}
  f = g \cdot 0 + f
\end{align*}

Mit $H(r) = \ord(f) < \ord(g) = H(g) \lor f = r = 0$.

Fall 2: $\underbrace{\ord(f)}_{m} \geq \underbrace{\ord(g)}_{n}$:

Wir wissen schon, dass wir $f$ und $g$ umschreiben können als

\begin{align*}
  f = x^m\tilde{f}, \ord(\tilde{f}) = 0 \\
  g = x^n\tilde{g}, \ord(\tilde{g}) = 0
\end{align*}

Nach Proposition 3.3.6.5 (6) wissen wir bereits, dass $\tilde{g}$ eine Inverse besitzt, also wählen wir $q = \tilde{g}^{-1}x^{m-n}\tilde{f}$ und erhalten

\begin{align*}
  f = x^m \tilde{f} = x^n\tilde{g} \cdot \tilde{g}^{-1}x^{m-n}\tilde{f} = gq + 0
\end{align*}

also $r = 0$.

Der Ring ist also euklidisch und somit bekanntlich auch Hauptideal- und faktorieller Ring.

Die Einheiten sind, wie bereits vorhin erwähnt $E := E(K[[x]]) = \{p \in K[[x]]: \ord(p) = 0\}$. Wir behaupten, dass alle irreduziblen Elemente gegeben sind durch

\begin{align*}
  \{p \in K[[x]]: \ord(p) = 1\} = xE ( = [x]_{\sim})
\end{align*}

Zunächst zur Gleichheit der beiden Ausdrücke. Wie oben schon verwendet kann man ein $p \in K[[x]]$ genau dann als $x\tilde{p}$ mit $\tilde{p} \in E$ schreiben, wenn $\ord(p) = 1$.

Dass $x$ (und somit alle assoziierten) irreduzibel sind, folgt für $x = pq$ mit beliebigen Potenzreihen $p,q$, da $1 = \ord(x) = \ord(pq) = \ord(p) + \ord(q)$. Also muss $\ord(p) = 0 \lor \ord(q) = 0$ und somit ein Faktor eine Einheit.

Angenommen, es gäbe ein irreduzibles Element $p \neq xe, e \in E$, also $\ord(p) = n \geq 2$. Dann können wir $p = x^n\tilde{p} = x^{n-1} (x\tilde{p})$ schreiben, wobei beide Faktoren Ordnung größer 0 haben und somit keine Einheiten sind.

Nun zu den Idealen. Wir wissen schon, dass $K[[x]]$ ein Hauptidealring ist, und somit können wir nach Proposition 5.2.2.6 jedes Ideal mit einer Äquivalenzklasse aus der Teilerhalbordnung (in dem Fall sogar Totalordnung oder?) $(K[[x]]/_{\sim},|)$ identifizieren.

Diese besteht genau aus den Äquivalenzklassen der Monome (bzw. Potenzreihen mit Ordnung $n$, Fall $n=1$ oben schon behandelt). Alle Ideale sind also eindeutig gegeben durch

\begin{align*}
  (x^n) = \{px^n: p \in K[[x]]\} = \{q \in K[[x]]: \ord(q) \geq n\}, n \in \N
\end{align*}

Bin mir ziemlich sicher, dass es so ist, weiß nicht ob die Argumentation zu 100Prozent schlüssig ist.


\end{solution}

\begin{algebraUE}{342}
Sei $R$ ein faktorieller Ring. Ist $f = \sum_{i=0}^na_ix^i \in R[x]$ mit Grad $\geq 1$
ein primitives Polynom und $p \in R$ irreduzibel mit
\begin{align*}
  p \nmid a_n, p | a_i, \text{ für } i = 0,\dots,n-1, \text{ und } p^2 \nmid a_0,
\end{align*}
dann ist $f$ irreduzibel in $R[x]$.
\end{algebraUE}

\begin{solution}
Angenommen $f$ wäre nicht irreduzibel, also existieren Nicht-Einheiten $q,r \in R[x]$ mit $f = qr$.
Da $f$ ein primitives Polynom ist gilt $\ggT\{a_i: i = 1,\dots,n\} = 1_R$ und $q,r$
müssen ebenso primitiv sein. Es gilt also
\begin{align*}
  f = \sum_{i=0}^na_ix^i = \sum_{i = 0}^n \sum_{k= 0}^i q_kr_{i-k}x^i.
\end{align*}
Durch Koeffizientenvergleich erhalten wir
\begin{align*}
  a_0 &= q_0r_0 \\
  a_1 &= q_0r_1 + q_1r_0 \\
  &\vdots \\
  a_n &= q_0r_n + \dots + q_nr_0.
\end{align*}
Da $p$ ein irreduzibles Element eines faktoriellen Ringes ist, ist $p$ prim und es folgt
aus $p| a_0 = q_0r_0$, dass $p|q_0 \lor p|r_0$. Da zusätzlich $p^2 \nmid a_0$ gilt,
kann $p$ nur genau einen der Faktoren $q_0,r_0$ teilen. Gelte also o.B.d.A. $p | q_0, p \nmid r_0$.
Jetzt gehen wir induktiv vor und zeigen $p | q_k, k = 0,\dots,n$. Der Anfang ist bereits getan,
gelte nun $p | q_{i}, i = 0,\dots,k-1$:
\begin{align*}
  p | a_k = q_kr_n + \dots + q_{k-1}r_1 + q_kr_0.
\end{align*}
Mit der Induktionsvoraussetzung erhalten wir $p | q_kr_n + \dots + q_{k-1}r_1$
und daraus folgt $p | q_kr_0$. Da $p \nmid r_0$ gilt damit schon $p | q_k$. \\
Also können wir $p$ aus $q$ herausheben und $q$ kann nicht primitiv sein. Widerspruch!

Alternativ: (erster Teil fast identisch)

Angenommen $f$ wäre nicht irreduzibel, also existieren Nicht-Einheiten $q,r \in R[x]$ mit $f = qr$.
Es gilt
\begin{align*}
  f = \sum_{i=0}^na_ix^i = \sum_{i = 0}^n \sum_{k= 0}^i q_kr_{i-k}x^i.
\end{align*}
Durch Koeffizientenvergleich erhalten wir
\begin{align*}
  a_0 &= q_0r_0 \\
  a_1 &= q_0r_1 + q_1r_0 \\
  &\vdots \\
  a_n &= q_0r_n + \dots + q_nr_0.
\end{align*}
Da $p$ ein irreduzibles Element eines faktoriellen Ringes ist, ist $p$ prim und es folgt
aus $p| a_0 = q_0r_0$, dass $p|q_0 \lor p|r_0$. Da zusätzlich $p^2 \nmid a_0$ gilt,
kann $p$ nur genau einen der Faktoren $q_0,r_0$ teilen. Gelte also o.B.d.A. $p | q_0, p \nmid r_0$.

Nun können sicher nicht alle Koeffizienten $q_0,...,q_n$ durch $p$ teilbar sein, da dies sonst auch $a_n$ wäre. Es gibt also ein minimales $s$, sodass $q_s$ nicht durch $p$ teilbar ist mit

\begin{align*}
  s \leq grad(q) < grad(f)
\end{align*}

da $r$ einen Grad größer $0$ hat.

Nun gilt aber für $a_s$

\begin{align*}
  a_s = q_0 r_s + \dots + q_{s-1}r_1 + q_s r_0
\end{align*}

Nach Voraussetzung sind alle bis auf den letzten Summanden durch $p$ teilbar (da $p \nmid q_s \land p \nmid r_0$). Das ist aber ein Widerspruch zu $p|a_s$.

\end{solution}

\begin{algebraUE}{344}
Seien $R$ ein faktorieller Ring, $f \in R[x]$ mit führendem Koeffizienten $a_n$
und konstantem Koeffizienten $a_0$ und $p,q \in R$ teilerfremd und das Element
$\frac{p}{q}$ des Quotientenkörpers $Q$ eine Nullstelle von $f$. Dann gilt $p|a_0$
und $q | a_n$.
\end{algebraUE}

\begin{solution}
\begin{align*}
  a_n \frac{p^n}{q^n} + \dots + a_0 = 0 \\
  \iff a_np^n + a_{n-1}p^{n-1}q^1 + \dots + a_1p^1q^{n-1} + a_0q^n = 0 \\
  \iff a_np^n + a_{n-1}p^{n-1}q^1 + \dots + a_1p^1q^{n-1} = -a_0q^n.
\end{align*}
Jetzt gilt $p | p(a_np^{n-1} + a_{n-1}p^{n-2}q^1 + \dots + a_1q^{n-1}) = -a_0q^n$.
Da $p,q$ teilerfremd sind folgt daraus bereits $p | a_0$.
Analog sieht man $q |q(a_{n-1}p^{n-1} + \dots + a_1p^1q^{n-2} + a_0q^{n-1}) = -a_np^n$
und damit $q | a_n$.

\end{solution}

\begin{exercise}
Betrachten Sie die Schwingung einer einseitig eingespannten Saite, welche die
Differentialgleichung
\begin{align*}
  \frac{1}{c^2}\frac{\partial^2}{\partial t ^2}y(x,t) = \frac{\partial^2}{\partial x^2}
  y(x,t), \qquad x \in (0,1), \qquad t > 0
\end{align*}
mit den Randbedingungen
\begin{align*}
  y(0,t) = 0, \qquad \frac{\partial}{\partial x}y(1,t) = 0, \qquad t > 0,
\end{align*}
erfüllt. Die Anfangsbedingungen seien $y(\cdot,0) = y_0(\cdot)$ und
$\frac{\partial}{\partial t}y(\cdot,0) = y_1(\cdot)$. Formulieren und lösen Sie
das Sturm-Liouville Eigenwertproblem, welches durch den Ansatz der Separation
der Variablen entsteht. Geben Sie eine (formale) Lösung als Reihe an.
\end{exercise}
\begin{solution}
Machen wir den Ansatz mit Seperation der Variablen. Das bedeutet wir setzen

\begin{align*}
  y(x,t)
  =
  v(x)w(t)
\end{align*}

Setzen wir nun in unsere Differentialgleichung ein erhalten wir

\begin{align*}
  \frac{1}{c^2}w^\primeprime(t)v(x)
  \stackrel{!}{=}
  v^\primeprime(x)w(t) \\
  \implies
  \frac{1}{c^2}\frac{w^\primeprime(t)}{w(t)}
  =
  \frac{v^\primeprime(x)}{v(x)}
\end{align*}

Dabei hängt die linke Seite nur von $t$ und die rechte seite nur von $x$ ab. Deswegen
muss es also eine Konstante $\lambda$ geben, die folgendes erfüllt:

\begin{align*}
  \frac{1}{c^2}\frac{w^\primeprime(t)}{w(t)}
  =
  -\lambda
  =
  \frac{v^\primeprime(x)}{v(x)}
  \implies
  \begin{cases}
    -\lambda v - v^\primeprime = 0 & \text{auf } (0,1), \quad v(0)=0=v^\prime(1) \\
    -\lambda c^2 w - w^\primeprime = 0 & \text{auf }(0,\infty)
  \end{cases}
\end{align*}

Aus der Vorlesung kennen wir die allgemeinen Lösungen für $v$ und  $w$:

\begin{align*}
  v(x) = c_1 \sin(\sqrt{\lambda}x) + c_2 \cos(\sqrt{\lambda}x) \\
  w(t) = c_3 \sin(c\sqrt{\lambda}t) + c_4 \cos(c\sqrt{\lambda}t)
\end{align*}

Um nun die Randbedingungen zu erfüllen, setzen wir $c_2 = 0$ und sehen uns die erste Ableitung
von $v$ an (o.B.d.A $c_1 \neq 0$)
\begin{align*}
  v^\prime(1)
  =
  c_1 \sqrt{\lambda}\cos(\sqrt{\lambda})
  \stackrel{!}{=}
  0 \\
  \implies
  \lambda = 0
  \lor
  \lambda
  =
  \frac{\pi^2}{4}n^2 \quad \text{mit } n\in \Z \backslash\{0\}
\end{align*}

Wir haben nun mit

\begin{align*}
  v(x)w(t)
  =
  c_1\sin(n\frac{\pi}{2} x)
  (c_3 \sin(c\frac{\pi}{2}nt) +
  c_4 \cos(c\frac{\pi}{2}nt))
\end{align*}

eine Lösung unserer Differentialgleichung die auch die Randbedingungen erfüllt.
Machen wir nun den Ansatz für die Lösung mit dem Superpositionsprinzip

\begin{align*}
  y(x,t)
  =
  \sum_{n=1}^{\infty} \sin(\frac{nx\pi}{2})
  (c_{1,n}\sin(\frac{nct\pi}{2})+c_{2,n}\cos(\frac{nct\pi}{2}))
\end{align*}

Um nun die Koeffizienten $c_{1,n},c_{2,n}$ zu erhalten, sehen wir uns die
Anfangsbedingungen an.

\begin{align*}
  y_0(x)
  \stackrel{!}{=}
  \sum_{n=1}^\infty c_{2,n}\sin(\frac{nx\pi}{2}),       \quad x \in (0,1) \\
  y_1(x)
  \stackrel{!}{=}
  \sum_{n=1}^\infty c_{1,n}\frac{nc\pi}{2}\sin(\frac{nx\pi}{2}),  \quad x \in (0,1)
\end{align*}

Falls man nun $y_0$ und $y_1$ antisymmetrisch auf $(-1,1)$
fortsetzt und dann die Sinusreihen bildet (falls diese dann auch gegen die entsprechenden Funktionen
konvergieren) erhält man ebenso die Darstellung:

\begin{align*}
  y_0(x)
  =
  \sum_{n=1}^\infty a_n \sin(\pi nx) \\
  y_1(x)
  =
  \sum_{n=1}^\infty b_n \sin(n \pi x), \quad \text{mit} \\
  a_n
  =
  \int_0^2 y_0(x)\sin(n\pi x) dx \\
  b_n
  =
  \int_0^2 y_1(x)\sin(n\pi x) dx \\
\end{align*}

Womit man zumindest $c_{1,n}$ und $c_{2,n}$ für $n \in 2\N$ bestimmen könnte(?)

\end{solution}

\begin{exercise}
\leavevmode \\
\begin{enumerate}[label = \textbf{\alph*)}]
  \item Zeigen Sie, dass für jedes $f \in C([0,1])$ die Randwertaufgabe
  \begin{align*}
    -y^{\primeprime} + ay = f, \qquad y^{\prime}(0) = y^{\prime}(1) = 0, \qquad a \in (0,\infty)
  \end{align*}
  eine eindeutige Lösung besitzt.
  \item Bestimmen Sie die Greensche Funktion für das Randwertproblem.
  \item Gegeben sei eine stetige Funktion $F: [0,1] \times \R \to \R$, die
  global Lipschitz-stetig bezüglich $u$ ist, also
  \begin{align*}
    |F(t,u) - F(t,\widetilde{u})| \leq L|u - \widetilde{u}|, \qquad
    \forall t \in [0,1], u , \widetilde{u} \in \R
  \end{align*}
  mit Lipschitzkonstante $0 < L < a$. Zeigen Sie, dass unter diesen Voraussetzungen
  das nichtlineare Randwertproblem
  \begin{align*}
    -y^{\primeprime} + ay = F(t,y), \qquad y^{\prime}(0) = y^{\prime}(1) = 0
  \end{align*}
  eine eindeutige Lösung besitzt.
  \item Falls $L \geq a$, dann besitzt dieses Randwertproblem im Allgemeinen keine
  eindeutige Lösung.
\end{enumerate}
\end{exercise}
\begin{solution}
\leavevmode \\
\begin{enumerate}[label = \textbf{\alph*)}]
  \item Die ODE ist bereits ist selbstadjungierter Form mit $p(x) = 1, q(x) = a$.
  Bestimmen wir zunächst ein Fundamentalsystem für $-y^{\primeprime} + ay = 0$.
  Zwei linear unabhängige Lösungen kann man direkt ablesen mit $y_1(x) = \exp(\sqrt{a}x), y_2(x) = \exp(-\sqrt{a}x)$.
  Das Randwertproblem ist genau dann für alle \\
  $f \in C([a,b];\R), \rho_1,\rho_2 \in \R$
  eindeutig lösbar, wenn
  \begin{align*}
    0 \stackrel{!}{\neq} \det\left(\begin{pmatrix}
      R_1y_1 & R_1y_2 \\ R_2y_1 & R_2y_2
    \end{pmatrix}\right) =
    \det\left(\begin{pmatrix}
      \sqrt{a} & -\sqrt{a} \\ \sqrt{a}\exp(\sqrt{a}) & -\sqrt{a}\exp(-\sqrt{a})
    \end{pmatrix}\right)
    = a\exp(\sqrt{a}) - a\exp(-\sqrt{a}).
  \end{align*}
  Dies wird in offensichtlicher Weise von allen $a > 0$ erfüllt.
  \item Additionstherome:
  \begin{align}
    \sinh(x \pm y) = \sinh(x)\cosh(y) \pm \cosh(x)\sinh(y) \label{sinh} \\
    \cosh(x \pm y) = \cosh(x)\cosh(y) \pm \sinh(x)\sinh(y) \label{cosh}
  \end{align}
  Für die Konstruktion der Greenschen Funktion verwenden wir das (hoffentlich)
  bequemere Fundamentalsystem
  \begin{align*}
  y_1(x) &= \frac{\exp(\sqrt{a}x)+\exp(-\sqrt{a}x)}{2} = \cosh(\sqrt{a}x), \\
  y_2(x) &= \frac{\exp(\sqrt{a}x)+\exp(-\sqrt{a}x)}{2} = \sinh(\sqrt{a}x)
  \end{align*}
  oder das noch bequemere
  \begin{align*}
  y_1(x) &= \cosh(\sqrt{a}x), \\
  y_2(x) &= \cosh(-\sqrt{a})\cosh(\sqrt{a}x) + \sinh(-\sqrt{a})\sinh(\sqrt{a}x) =
  \cosh(\sqrt{a}(x-1)).
  \end{align*}
  Wir machen den Ansatz
  \begin{align*}
    G(x,t) = \begin{cases}
      a_1(t)\cosh(\sqrt{a}x) + a_2(t)\cosh(\sqrt{a}(x-1)), & x \leq t \\
      b_1(t)\cosh(\sqrt{a}x) + b_2(t)\cosh(\sqrt{a}(x-1)), & x > t
    \end{cases},
  \end{align*}
  welcher nach Konstruktion sicher die Bedingung
  \begin{itemize}
    \item $LG(\cdot,t) = 0$ auf $(a,t) \cup (t,b)$ erfüllt.
    \item $R_1G(\cdot,t) = 0$:
    \begin{align*}
      R_1G(\cdot,t) = \partial_xG(0,t) = a_1(t)\sqrt{a}\sinh(0) - a_2(t)\sqrt{a}\sinh(-\sqrt{a}) =  - a_2(t)\sqrt{a}\sinh(-\sqrt{a}) \stackrel{!}{=} 0.
    \end{align*}
    Also erhalten wir $a_2(t) = 0$.
    \item $R_2G(\cdot,t) = 0$:
    \begin{align*}
      R_2G(\cdot,t) &= \partial_xG(1,t) = b_1(t)\sqrt{a}\sinh(\sqrt{a}) - b_2(t)\sqrt{a}\sinh(0)
      = b_1(t)\sqrt{a}\sinh(\sqrt{a}) \stackrel{!}{=} 0
    \end{align*}
    Wir erhalten $b_1(t) = 0$.
    \item $G(\cdot,t)$ ist stetig bei $x = t$: \\
    \begin{align*}
      G(t^-,t) = a_1(t)\cosh(\sqrt{a}t) \stackrel{!}{=}
      b_2(t)\cosh(\sqrt{a}(t-1)) = G(t^+,t).
    \end{align*}
    Wir erhalten $a_1(t) = b_2(t)\frac{\cosh(\sqrt{a}(t-1))}{\cosh(\sqrt{a}t)}$.
    \item $\partial_x G(t^+,t)- \partial_x G(t^-,t) = - \frac{1}{p(t)} = -1$:
    \begin{align*}
      \partial_x G(t^+,t)- \partial_x G(t^-,t) =
      b_2(t)\sqrt{a}\sinh(\sqrt{a}(t-1)) - a_1(t)\sqrt{a}\sinh(\sqrt{a}t) \stackrel{!}{=} -1.
    \end{align*}
    Es folgt
    \begin{align*}
      b_2(t) &= -\frac{1}{\sqrt{a}\sinh(\sqrt{a}(t-1)) - \frac{\cosh(\sqrt{a}(t-1))}{\cosh(\sqrt{a}t)}\sqrt{a}\sinh(\sqrt{a}t)} \\
      &= -\frac{\cosh(\sqrt{a}t)}{\sqrt{a}\left(\sinh(\sqrt{a}(t-1))\cosh(\sqrt{a}t) - \cosh(\sqrt{a}(t-1))\sinh(\sqrt{a}t)\right)} \\
      &= -\frac{\cosh(\sqrt{a}t)}{\sqrt{a}\sinh(-\sqrt{a})}
      = \frac{\cosh(\sqrt{a}t)}{\sqrt{a}\sinh(\sqrt{a})}
    \end{align*}
    \item Insgesamt liefert das
    \begin{align*}
      G(x,t) = \begin{cases}
      \frac{\cosh(\sqrt{a}(t-1))}{\sqrt{a}\sinh(\sqrt{a})}\cosh(\sqrt{a}x), & x \leq t \\
      \frac{\cosh(\sqrt{a}t)}{\sqrt{a}\sinh(\sqrt{a})}\cosh(\sqrt{a}(x-1)), & x > t
      \end{cases}.
    \end{align*}
  \end{itemize}
  \item Wir betrachten das äquivalente System erster Ordnung:
  \begin{align*}
    \begin{pmatrix}
      y \\ z
    \end{pmatrix}^{\prime} =
    \begin{pmatrix}
      z \\ ay - F(t,y)
    \end{pmatrix}
  \end{align*}
  Wir definieren die Funktion
  \begin{align*}
    f(t,y,z) = \begin{pmatrix}
      z \\ ay - F(t,y)
    \end{pmatrix}.
  \end{align*}
  Gelte $\|(y,z)^{\top} - (\widetilde{y},\widetilde{z})\|_{\infty} \leq \epsilon$. Dann folgt
  \begin{align*}
    \|f(t,y,z) - f(t,\widetilde{y},\widetilde{z})\|_{\infty} \leq \max\{\epsilon, |a\epsilon| + |L\epsilon|\} \leq \max\{1,2a\}\epsilon
  \end{align*}
  damit die Lipschitz-Stetigkeit von $f$. \\
  Alternativ können wir das Problem auch in eine Integraldarstellung umformen:
  \begin{align*}
    y(t) = \int_0^t\int_0^say(\tau) - F(\tau,y(\tau))d\tau ds
  \end{align*}
  Definiere die Funktion
  \begin{align*}
    f(t,y) = \int_0^ty - F(\tau,y) d\tau
  \end{align*}
  \item Betrachte das Gegenbeispiel $F(t,y) = ay$ mit Lipschitzkonstante $L = a$. Das Randwertproblem lautet nun
  \begin{align*}
    y^{\primeprime} = 0, \qquad y^{\prime}(0) = 0 = y^{\prime}(1),
  \end{align*}
  welches klarerweise von jeder beliebigen konstanten Funktion gelöst wird.
\end{enumerate}
\end{solution}

% --------------------------------------------------------------------------------

\begin{exercise}[75]

\phantom{}

\end{exercise}

% --------------------------------------------------------------------------------

\begin{solution}

\phantom{}

\end{solution}


\end{document}
