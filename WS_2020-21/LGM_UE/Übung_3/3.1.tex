% --------------------------------------------------------------------------------

\begin{exercise}[66]

Sei $\mathscr{M}$ eine Struktur, sei $b$ eine Belegung, seien $s$ und $t$ Terme,
und $x$ eine Variable. \\
Sei $a := \overline{b}(t)$. dann ist $\overline{b}(s[x/t])
= \overline{b_{x/a}}(s)$.

\end{exercise}

% --------------------------------------------------------------------------------

\begin{solution}

Wir zeigen die Aussage mittels der induktiven Definition der Substitution.
\begin{enumerate}
  \item Sei $s = x$:
  \begin{align*}
    \overline{b}(s[x/t]) = \overline{b}(t) = \overline{b_{x/a}}(s).
  \end{align*}
  \item Sei $s = y$ eine von $x$ verschiedene Variable:
  \begin{align*}
    \overline{b}(s[x/t]) = \overline{b}(y) = \overline{b_{x/a}}(s).
  \end{align*}
  \item Sei $s = c$ eine Konstante:
  \begin{align*}
    \overline{b}(s[x/t]) = \overline{b}(c) = \overline{b_{x/a}}(s).
  \end{align*}
  \item Sei $s = f(s_1,\dots,s_k)$ mit einem $k$-stelligen Funktionssymbol $f$
  und Termen $s_1,\dots,s_k$:
  \begin{align*}
    \overline{b}(s[x/t]) &= \overline{b}(f(s_1,\dots,s_k)[x/t])
    = \overline{b}(f(s_1[x/t],\dots,s_k[x/t])) \\
    &= f^{\mathscr{M}}(\overline{b}(s_1[x/t]),\dots,\overline{b}(s_k[x/t])) \\
    &= f^{\mathscr{M}}(\overline{b_{x/a}}(s_1),\dots,\overline{b_{x/a}}(s_k)) \\
    &= \overline{b_{x/a}}(f(s_1,\dots,s_k)).
  \end{align*}
\end{enumerate}
Damit enthält die Menge aller Terme $s$, sodass für alle Terme $t$ und Belegungen $b$
\begin{align*}
  \overline{b}(s[x/t]) = \overline{b_{x/a}}(s)
\end{align*}
gilt, alle Konstanten und Variablen und ist unter Anwendung von Funktionssymbolen
abgeschlossen. \\
Also gilt die Aussage für alle Terme $s$.
\end{solution}

% --------------------------------------------------------------------------------
