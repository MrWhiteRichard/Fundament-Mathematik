\documentclass{article}

% Hier befinden sich Pakete, die wir beinahe immer benutzen ...

\usepackage[utf8]{inputenc}

% Sprach-Paket:
\usepackage[ngerman]{babel}

% damit's nicht so, wie beim Grill aussieht:
\usepackage{fullpage}

% Mathematik:
\usepackage{amsmath, amssymb, amsfonts, amsthm}
\usepackage{bbm, mathrsfs, stmaryrd}
\usepackage{mathtools, mathdots}

% Makros mit mehereren Default-Argumenten:
\usepackage{twoopt}

% Anführungszeichen (Makro \Quote{}):
\usepackage{babel}

% if's für Makros:
\usepackage{xifthen}
\usepackage{etoolbox}

% tikz ist kein Zeichenprogramm (doch!):
\usepackage{tikz}

% bessere Aufzählungen:
\usepackage{enumitem}

% (bessere) Umgebung für Bilder:
\usepackage{graphicx, subfig, float}

% Umgebung für Code:
\usepackage{listings}

% Farben:
\usepackage{xcolor}

% Umgebung für "plain text":
\usepackage{verbatim}

% Umgebung für mehrerer Spalten:
\usepackage{multicol}

% "nette" Brüche
\usepackage{nicefrac}

% Spaltentypen verschiedener Dicke
\usepackage{tabularx}
\usepackage{makecell}

% Für Vektoren
\usepackage{esvect}

% (Web-)Links
\usepackage{hyperref}

% Zitieren & Literatur-Verzeichnis
\usepackage[style = authoryear]{biblatex}
\usepackage{csquotes}

% so ähnlich wie mathbb
%\usepackage{mathds}

% Keine Ahnung, was das macht ...
\usepackage{booktabs}
\usepackage{ngerman}
\usepackage{placeins}

% special letters:

\newcommand{\N}{\mathbb{N}}
\newcommand{\Z}{\mathbb{Z}}
\newcommand{\Q}{\mathbb{Q}}
\newcommand{\R}{\mathbb{R}}
\newcommand{\C}{\mathbb{C}}
\newcommand{\K}{\mathbb{K}}
\newcommand{\T}{\mathbb{T}}
\newcommand{\E}{\mathbb{E}}
\newcommand{\V}{\mathbb{V}}
\renewcommand{\P}{\mathbb{P}}
\newcommand{\1}{\mathbbm{1}}

\newcommand  {\B}{\mathfrak{B}}
\renewcommand{\S}{\mathfrak{S}}

% quantors:

\newcommand{\Forall}{\forall \,}
\newcommand{\Exists}{\exists \,}
\newcommand{\ExistsOnlyOne}{\exists! \,}
\newcommand{\nExists}{\nexists \,}

% MISC symbols:

\newcommand{\landau}[1]
{
  {\scriptstyle \mathcal{O}}
  \pbraces{#1}
}

\newcommand{\Landau}[1]
{
  \mathcal{O}
  \pbraces{#1}
}

\newcommand{\eps}{\mathrm{eps}}

% graphics in a box:

\newcommandtwoopt
{\includegraphicsboxed}[3][][]
{
  \begin{figure}[!h]
    \begin{boxedin}
      \ifthenelse{\isempty{#2}}
      {
        \begin{center}
          \includegraphics[width = 0.75 \textwidth]{#3}
          \label{fig:#1}
        \end{center}
      }{
        \begin{center}
          \includegraphics[width = 0.75 \textwidth]{#3}
          \caption{#2}
          \label{fig:#1}
        \end{center}
      }
    \end{boxedin}
  \end{figure}
}

% braces:

\newcommand{\pbraces}[1]{{\left  ( #1 \right  )}}
\newcommand{\bbraces}[1]{{\left  [ #1 \right  ]}}
\newcommand{\Bbraces}[1]{{\left \{ #1 \right \}}}
\newcommand{\vbraces}[1]{{\left  | #1 \right  |}}
\newcommand{\Vbraces}[1]{{\left \| #1 \right \|}}
\newcommand{\abraces}[1]{{\left \langle #1 \right \rangle}}
\newcommand{\round}[1]{\bbraces{#1}}

\newcommand
{\floor}[1]
{{\left \lfloor #1 \right \rfloor}}

\newcommand
{\ceil} [1]
{{\left \lceil  #1 \right \rceil }}

% special functions:

\newcommand{\norm}  [2][]{\Vbraces{#2}_{#1}}
\newcommand{\diag}  [1]{\mathrm{diag} \: #1}
\newcommand{\dist}  [1]{\mathrm{dist} \: #1}
\newcommand{\mean}  [1]{\mathrm{mean} \: #1}
\newcommand{\erf}   [1]{\mathrm{erf} \: #1}
\newcommand{\id}    [1]{\mathrm{id} \: #1}
\newcommand{\sgn}   [1]{\mathrm{sgn} \: #1}
\newcommand{\supp}  [1]{\mathrm{supp} \: #1}
\newcommand{\arsinh}[1]{\mathrm{arsinh} \: #1}
\newcommand{\arcosh}[1]{\mathrm{arcosh} \: #1}
\newcommand{\artanh}[1]{\mathrm{artanh} \: #1}
\newcommand{\card}  [1]{\mathrm{card} \: #1}
\newcommand{\Span}  [1]{\mathrm{span} \: #1}
\newcommand{\Aut}   [1]{\mathrm{Aut} \: #1}
\newcommand{\End}   [1]{\mathrm{End} \: #1}
\newcommand{\ggT}   [1]{\mathrm{ggT} \: #1}
\newcommand{\kgV}   [1]{\mathrm{kgV} \: #1}
\newcommand{\ord}   [1]{\mathrm{ord} \: #1}
\newcommand{\grad}  [1]{\mathrm{grad} \: #1}
\newcommand{\ran}   [1]{\mathrm{ran} \: #1}
\newcommand{\graph} [1]{\mathrm{graph} \: #1}
\newcommand{\Inv}   [1]{\mathrm{Inv} \: #1}
\newcommand{\pv}    [1]{\mathrm{pv} \: #1}
\newcommand{\Mod}{\: \mathrm{mod} \:}
\newcommand{\Char}{\mathrm{char}}
\newcommand{\At}{\mathrm{At}}
\newcommand{\Ob}{\mathrm{Ob}}
\newcommand{\Hom}{\mathrm{Hom}}
\newcommand{\orthogonal}[3][]{#2 ~\bot_{#1}~ #3}
\newcommand{\Rang}{\mathrm{Rang}}

\newcommand
{\GL}[2][]
{\mathrm{GL}_{#1} \pbraces{#2}}

% fractions:

\newcommand{\Frac}[2]{\frac{1}{#1} \pbraces{#2}}
\newcommand{\nfrac}[2]{\nicefrac{#1}{#2}}

% derivatives & integrals:

\newcommandtwoopt
{\Int}[4][][]
{\int_{#1}^{#2} #3 ~\mathrm{d} #4}

\newcommandtwoopt
{\derivative}[3][][]
{
  \frac
  {\mathrm{d}^{#1} #2}
  {\mathrm{d} #3^{#1}}
}

\newcommandtwoopt
{\pderivative}[3][][]
{
  \frac
  {\partial^{#1} #2}
  {\partial #3^{#1}}
}

\newcommand
{\primeprime}
{{\prime \prime}}

\newcommand
{\primeprimeprime}
{{\prime \prime \prime}}

% Text:

\newcommand{\Quote}[1]{\glqq #1\grqq{}}
\newcommand{\Text}[1]{{\text{#1}}}
\newcommand{\fastueberall}{\text{f.ü.}}
\newcommand{\fastsicher}{\text{f.s.}}

% -------------------------------- %
% amsthm-stuff:

\theoremstyle{definition}

% numbered theorems
\newtheorem{theorem}    {Satz}   [section]
\newtheorem{lemma}      [theorem]{Lemma}
\newtheorem{corollary}  [theorem]{Korollar}
\newtheorem{proposition}[theorem]{Proposition}
\newtheorem{remark}     [theorem]{Bemerkung}
\newtheorem{definition} [theorem]{Definition}
\newtheorem{example}    [theorem]{Beispiel}

% unnumbered theorems
\newtheorem*{theorem*}    {Satz}
\newtheorem*{lemma*}      {Lemma}
\newtheorem*{corollary*}  {Korollar}
\newtheorem*{proposition*}{Proposition}
\newtheorem*{remark*}     {Bemerkung}
\newtheorem*{definition*} {Definition}
\newtheorem*{example*}    {Beispiel}

% Please define this stuff in project ("main.tex"):

% \def \lastexercisenumber {...}
% This will be 0 by default

% \setcounter{section}{...}
% This will be 0 by default
% and hence, completely ignored

\ifnum \thesection = 0
{
  \newtheorem{exercise}{Aufgabe}
}
\else
{
  \newtheorem{exercise}{Aufgabe}[section]
}
\fi

\ifdef
{\lastexercisenumber}
{\setcounter{exercise}{\lastexercisenumber}}

\newenvironment{solution}
{
  \begin{proof}[Lösung]
}{
  \end{proof}
}

\renewcommand{\proofname}{Beweis}

% -------------------------------- %
% environment zum einkasteln:

% dickere vertical lines
\newcolumntype
{x}
[1]
{
  !{
    \centering
    \arraybackslash
    \vrule
    width #1}
}

% environment selbst (the big cheese)
\newenvironment
{boxedin}
{
  \begin{tabular}
  {
    x{1 pt}
    p{\textwidth}
    x{1 pt}
  }
  \Xhline
  {2 \arrayrulewidth}
}
{
  \\
  \Xhline{2 \arrayrulewidth}
  \end{tabular}
}

% -------------------------------- %
% MISC "Ein-Deutschungen"

\renewcommand{\figurename}{Abbildung}
\renewcommand{\tablename} {Tabelle}

% -------------------------------- %

\input{../../../Fundament-LaTeX/listings.tex}

\parskip 0pt
\parindent 0pt

\title
{
  Logik und Grundlagen der Mathematik \\
  \vspace{4pt}
  \normalsize
  \textit{3. Übung am 22.10.2020}
}
\author
{
  Richard Weiss
  \and
  Florian Schager
  % \and
  % Christian Sallinger
  \and
  Fabian Zehetgruber
  % \and
  % Paul Winkler
  % \and
  % Christian Göth
}
\date{}

\begin{document}

\maketitle

\section*{Substitution}

% --------------------------------------------------------------------------------

\begin{exercise}[Random walk of a robot]

A robot is placed at the origin (the point $(0, 0)$) on a two-dimension integer grid (see the figure below).
Denote the position of the robot by $(x, y)$.
The robot can either move right to $(x + 1, y)$ or move up to $(x, y + 1)$.

\begin{center}
    \begin{tikzpicture}[scale = 0.5]

        \draw [->] (0, 0) -- (0, 11);
        \draw [->] (0, 0) -- (11, 0);
        \draw (0, 0) grid (10, 10);

        \filldraw [color = blue] (0, 0) circle (4pt);
        \draw node [below left] {$(0, 0)$};

        \filldraw [color = red] (8, 6) circle (4pt);
        \draw [->] (11, 7) node [above right] {$(8, 6)$} -- (8.25, 6.25);

    \end{tikzpicture}
\end{center}

\begin{enumerate}[label = (\alph*)]

    \item Suppose each time the robot randomly moves right of up with equal chance.
    What is the probability that the robot will ever reach the point $(8, 6)$?

    \item Suppose another robot has a $\frac{2}{3}$ chance to move right and a $\frac{1}{3}$ chance to move up when $x + y$ is even, otherwise it has a $\frac{1}{4}$ chance to move right and a $\frac{3}{4}$ chance to move up.
    It stops whenever $|x - y| \geq 2$.
    Find the probability that $x - y = 2$ when it stops.

\end{enumerate}

\end{exercise}

% --------------------------------------------------------------------------------

\begin{solution}

\phantom{}

\begin{center}
    \begin{tikzpicture}[scale = 0.5]

        \draw [->] (0, 0) -- (0, 11);
        \draw [->] (0, 0) -- (11, 0);
        \draw (0, 0) grid (10, 10);

        \draw [<->, color = yellow, thick] (11, 3) -- (3, 11);

        \draw [->, color = green, thick] (0, 0) -- (11, 11);
        \draw [->, color = green, thick] (0, 2) -- (9, 11);
        \draw [->, color = green, thick] (2, 0) -- (11, 9);

        \filldraw [color = blue] (0, 0) circle (4pt);
        \draw node [below left] {$(0, 0)$};

        \filldraw [color = red] (8, 6) circle (4pt);
        \draw [->] (11, 7) node [above right] {$(8, 6)$} -- (8.25, 6.25);

    \end{tikzpicture}
\end{center}

\begin{enumerate}[label = (\alph*)]

    \item Each path that the robot moves along is equally likely.
    
    There are $\binom{a}{b}$ paths ($a = x + y$, $b \in \Bbraces{x, y}$) that lead from $(0, 0)$ to $(x, y)$.
    This can be observed by turning one of the figures above by $90 + 45$ degrees clockwise and modifying it to be a Pascal's triangle.
    The number of downwards paths that lead from the root of a Pascal's to one of its nodes with value $z$ is $z$.
    
    \begin{figure}[H]
        \centering
        \subfloat[\href{https://de.m.wikipedia.org/wiki/Datei:Pascal_triangle_small.svg}{wikipedia}]{
            \includegraphics[width = 0.35 \textwidth]{pascals_triangle_wikipedia.png}
        }
        \hspace{1cm}
        \subfloat[\href{https://stackoverflow.com/questions/47614514/how-can-i-modify-my-program-to-print-out-pascals-triangle}{stackoverflow}]{
            \includegraphics[width = 0.35 \textwidth]{pascals_triangle_stackoverflow.png}
        }
        \caption{Pascale's triangles}
    \end{figure}
    
    Let the robot move $8 + 6$ steps.
    It must land on the yellow diagonal $\Bbraces{(10, 4), \dots, (4, 10)}$.
    The position on that diagonal completely determines, whether the robot reaches $(8, 6)$ or not (since it can only move right and up).
    Consider the Binomial Theorem for $(1 + 1)^n$.
    
    \begin{align*}
        P(\text{robot passes through $(8, 6)$})
        & =
        P(\text{robot sits on $(8, 6)$} \mid \text{robot moved exactly $8 + 6$ steps}) \\
        & =
        \frac
        {
            \binom{14}{7 \pm 1}
        }{
            \sum_{n=0}^{14}
                \binom{14}{n}
        } \\
        & =
        \frac{\binom{14}{7 \pm 1}}{2^n} \\
        & \approx
        0.183288574219
    \end{align*}

    \item On the middle green diagonal $\Bbraces{(x, x)}_{x=0}^{10}$ every element $(x, x)$ has an even value $x + x$.

    $\Forall x = 0, \dots, 8:$

    \begin{align*}
        &
        P(\text{robot will move to $(x+1, x+1)$} \mid \text{robot sits at $(x, x)$}) \\
        & =
        P(\text{robot will move up and then right} \mid \text{robot sits at $(x, x)$}) \\
        & +
        P(\text{robot will move right and then up} \mid \text{robot sits at $(x, x)$}) \\
        & =
        \frac{1}{3} \frac{1}{4} + \frac{2}{3} \frac{3}{4} \\
        & =
        \frac{7}{12} =: p_0,
    \end{align*}

    \begin{align*}
        P(\text{robot will move to $(x+2, x)$} \mid \text{robot sits at $(x, x)$})
        =
        \frac{1}{3} \frac{3}{4}
        =
        \frac{3}{12} =: p_{-1},
    \end{align*}

    \begin{align*}
        P(\text{robot will move to $(x, x+2)$} \mid \text{robot sits at $(x, x)$})
        =
        \frac{2}{3} \frac{1}{4}
        =
        \frac{2}{12} =: p_1
    \end{align*}

    \begin{multline*}
        \pi_n
        :=
        P(\text{ending position $(x, y)$ fulfills $x - y = 2$} \mid x + y \leq 2 n) \\
        \xrightarrow{n \to \infty}
        P(\text{ending position $(x, y)$ fulfills $x - y = 2$}) =: \pi_\infty
    \end{multline*}

    \begin{align*}
        \pi_1 & = p_1 \\
        \pi_2 & = \pi_1 + p_0 p_1 = p_1 + p_0 p_1 = p_1 (1 + p_1) \\
        \pi_3 & = \pi_2 + p_0^2 p_1 = p_1 (1 + p_0) + p_0^2 p_1 = p_1 (1 + p_0 + p_0^2) \\
        & \vdots \\
        \pi_\infty & = p_1 \sum_{k=0}^\infty p_0^k = \frac{p_1}{1 - p_0} = \frac{2 / 12}{1 - 7 / 12} = \frac{2 / 12}{5 / 12} = \frac{2}{5} = 0.4
    \end{align*}

    \begin{comment}

        \begin{align*}
            &
            P(\text{ending position $(x, y)$ fulfills $|x - y| \neq 2$}) \\
            & =
            P
            \pbraces
            {
                \begin{array}{l}
                    \text{ending position is $(10, 10)$ and} \\
                    \text{robot only moves between diagonals $\Bbraces{(x+2, x)}_{x=0}^8$ and $\Bbraces{(x, x+2)}_{x=0}^8$}
                \end{array}
            } \\
            & =
            \underbrace
            {
                P(\text{robot will move to $(10, 10)$} \mid \text{robot sits at $(9, 9)$})
            }_1 \\
            & \quad
            P(\text{robot will move to $(9, 9)$} \mid \text{robot sits at $(8, 8)$}) \\
            & \quad
            \vdots \\
            & \quad
            P(\text{robot will move to $(1, 1)$} \mid \text{robot sits at $(0, 0)$}) \\
            & =
            \prod_{x=0}^8
                P(\text{robot will move to $(x+1, x+1)$} \mid \text{robot sits at $(x, x)$}) \\
            & =
            \pbraces{\frac{7}{12}}^9
        \end{align*}

        \begin{align*}
            P(\text{ending position $(x, y)$ fulfills $x - y = 2$})
            =
            P(\text{ending position $(x, y)$ fulfills $|x - y| = 2$}) / 2
            =
            \pbraces{1 - \pbraces{\frac{7}{12}}^9} / 2
            \approx
            0.496089600308
        \end{align*}

    \end{comment}

\end{enumerate}

\end{solution}

% --------------------------------------------------------------------------------

% -------------------------------------------------------------------------------- %

\begin{exercise}

Zeigen Sie:

\begin{enumerate}[label = (\alph*)]

    \item Bekanntlich ist die Funktion
    
    \begin{align*}
        f(x)
        =
        \begin{cases}
            \sin(1 / x) & \text{für}~ x > 0 \\
            0         & \text{für}~ x = 0
        \end{cases}
    \end{align*}

    Im Intervall $[0, 1]$ Riemann-integrierbar.
    Zeigen Sie

    \begin{align*}
        g(x)
        =
        \Int[0][x]{f(u)}{u}
        =
        x^2 \cos(1 / x) - \Int[0][x]{2 u \cos(1/u)}{u}
    \end{align*}

    und damit $g^\prime(x) = f(x)$ für alle $x \in [0, 1]$ (wobei in $0$ eigentlich nur die einseitige Ableitung existiert, Sie können sich aber vorstellen, dass $g$ (und $f$) auf er negativen Achse einfach null sind.)

    \item Für $a > 0$ wählen wir $n$ so, dass
    
    \begin{align*}
        \frac{1}{n \pi} < \frac{a}{2}
    \end{align*}

    gilt.
    Dann ist für $a > 0$ die Funktion

    \begin{align*}
        h_{0, a}(x)
        =
        \begin{cases}
            g(x)               & \text{für}~ 0 \leq x \leq \frac{1}{n \pi}, \\
            g(\frac{1}{n \pi}) & \text{für}~ \frac{1}{n \pi} < x < a - \frac{1}{n \pi}, \\
            g(a - x)           & \text{für}~ a - \frac{1}{n \pi} \leq x \leq a, \\
            0                  & \text{sonst}
        \end{cases}
    \end{align*}

    überall differenzierbar, die Ableitung $h_{0, a}^\prime$ ist überall stetig außer in den Punkten $0$ und $a$, und in jeder Umbegung von $0$ und $a$ gibt es Punkte mit $h^\prime = 1$ und $h^\prime = -1$.

    \item $h_{a, b}(x) = h_{0, b - a}(x - a)$ ist überall differenzierbar mit einer Ableitung, die überall stetig ist, außer in den Punkten $a$ und $b$ (es soll natürlich $a < b$ gelten.)

\end{enumerate}

\end{exercise}

% -------------------------------------------------------------------------------- %

\begin{solution}

\phantom{}

\begin{enumerate}[label = (\alph*)]

    \item Die linke und rechte Seite verschwinden beide für $x \to 0$.
    Weiters stimmen ihre Ableitungen in $x \in ]0, 1]$ überein.
    Unter Verwendung des rechtsseitigen Differentialquotienten, und dieser Identität, zeigt man unmittelbar $g^\prime(0) = 0$.

    \item Zwischen den Sprüngen ist die Funktion offensichtlich stetig differenzierbar.
    Darauf verschwinden die jeweiligen linksseitigen und rechtsseitigen Differentialquotienten, stimmen also insbesondere überein.

    Sei $\epsilon > 0$ hinreichend klein.

    \begin{align*}
        \implies
        \Forall x \in B_\epsilon(0) \cap (0, a):
            h_{0, 1}^\prime(x)
            =
            g^\prime(x)
            =
            f(x)
            =
            \sin(1 / x)
            \stackrel{!}{=}
            \pm 1
    \end{align*}

    Letztere Gleichheit gilt für $x := \pm \frac{1}{m \pi}$ und $m \in \N$ hinreichend groß.
    Analoges Spiel gilt für eine $\epsilon$-Umgebung von $a$.

    \item Nachdem $h_{a, b}$ eine affine Transformation der Funktion aus (b) ist, übertragen sich alle Regularitätseigenschaften.

\end{enumerate}

\end{solution}

% -------------------------------------------------------------------------------- %


\section*{Ableitungskalküle}

In den folgenden Beispielen betrachten wir Zeichenfolgen (Strings), die aus
den Zeichen $1,+,=$ zusammengesetzt sind. Auch die leere Folge gilt als
Zeichenfolge; sie hat Länge $0$. Meist wird sie mit $\epsilon$ oder mit $\Lambda$
bezeichnet. \\
Gewisse Zeichenfolgen zeichnen wir als \blockquote{ableitbar} aus. \\
Gewisse Zeichenfolgen nennen wir \blockquote{Axiome}. Axiome $A$ schreiben wir in
der Form

\begin{align*}
  \frac{\emptyset}{A}
\end{align*}

an. Wir lesen dies als \blockquote{A ist ableitbar}. Eine \blockquote{Regel}, die wir in
der Form

\begin{align*}
  \frac{A_1,\dots,A_n}{B}
\end{align*}

schreiben, lesen wir als \blockquote{Wenn $A_1,\dots,A_n$ ableitbar sind, dann auch $B$}. \\
Die Menge der ableitbaren Zeichenfolgen ist die kleinste Menge $M$, die alle
Axiome enthält und unter allen Regeln abgeschlossen ist.
% --------------------------------------------------------------------------------

\begin{exercise}

\phantom{}

\begin{enumerate}[label = (\roman*)]
    \item Zeigen Sie, dass die Funktion
    
    \begin{align*}
        f(x)
        =
        \begin{cases}
            \ln{|x|} & x \neq 0 \\
            0        & x = 0
        \end{cases}
    \end{align*}

    eine reguläre Distribution definiert, die punktweise Ableitung

    \begin{align*}
        f^\prime(x)
        =
        \begin{cases}
            \frac{1}{x}        & x \neq 0 \\
            \text{undefiniert} & x = 0
        \end{cases}
    \end{align*}

    jedoch nicht.

    \item Es bezeichne $\pv{(\frac{1}{x})}$ die Distribution
    
    \begin{align*}
        \langle \pv{\pbraces{\frac{1}{x}}}, \varphi \rangle
        =
        \lim_{\varepsilon \to 0+}
        \pbraces
        {
            \Int[-\infty][-\varepsilon]
            {
                \frac{\varphi(x)}{x}
            }{x}
            +
            \Int[\varepsilon][\infty]
            {
                \frac{\varphi(x)}{x}
            }{x}
        }
        =
        \lim_{\varepsilon \to 0+}
        \Int[|x| > \varepsilon]
        {
            \frac{\varphi(x)}{x}
        }{x}.
    \end{align*}

    Zeigen Sie, dass $\abraces{\pv{(\frac{1}{x})}, \varphi} = \Int[0][\infty]{\frac{\varphi(x) - \varphi(-x)}{x}}{x}$.

    \item Überprüfen Sie, dass $(\ln{|x|})^\prime = \pv{(\frac{1}{x})}$ in $\mathcal{D}^\prime(\R)$ gilt.

\end{enumerate}

\end{exercise}

% --------------------------------------------------------------------------------

\begin{solution}
\phantom{}
\begin{enumerate}[label = (\roman*)]
	\item Gegeben sei eine beliebige kompakte Menge $K \subseteq \R$. Wir wählen $a \in [1, \infty]$ mit $K \subseteq [-a, a]$. Nun ist
	\begin{align*}
	\int_K \vbraces{f(x)} dx &\leq 2 \pbraces{\int_1^a \log(x) dx - \int_0^1 \log(x) dx} \\
	&= 2 \pbraces{(\log(a)a - a) - (\log(1)1 - 1) - \pbraces{(\log(1)1 - 1) - \lim_{\epsilon \to 0+} (\log(\epsilon)\epsilon - \epsilon )}} \\
	&= \log(a)a - a + 2 < \infty.
	\end{align*}
	Betrachte hingegen 
	\begin{align*}
	\int_0^1 \vbraces{f^\prime(x)} dx = \int_0^1 \frac{1}{x} dx = \log(1) - \lim_{\epsilon \to 0+} \log(\epsilon) = \infty.
	\end{align*}
	\item Wir berechnen
	\begin{align*}
	\abraces{\pv\pbraces{\frac{1}{x}}, \phi} &= \lim_{\varepsilon \to 0+}
	\pbraces
	{
		\Int[-\infty][-\varepsilon]
		{
			\frac{\varphi(t)}{t}
		}{t}
		+
		\Int[\varepsilon][\infty]
		{
			\frac{\varphi(x)}{x}
		}{x}
	} \\
	&= \lim_{\varepsilon \to 0+}
	\pbraces
	{
	-\Int[\varepsilon][\infty]
	{
		\frac{\varphi(-x)}{x}
	}{x}
	+
	\Int[\varepsilon][\infty]
	{
		\frac{\varphi(x)}{x}
	}{x}
	} = \Int[0][\infty]{\frac{\varphi(x) - \varphi(-x)}{x}}{x}
	\end{align*}
	\item 
	\begin{align*}
	\abraces{\pbraces{\log|x|}^\prime, \phi} &= -\abraces{\log|x|, \phi^\prime} = -\Int[\R][]{\log|x| \phi^\prime(x)}{x} = -\Int[0][\infty]{\log|x| (\phi^\prime(x) + \phi^\prime(-x))}{x} \\
	&= -\Int[0][\infty]{\log|x| (\phi(x) - \phi(-x))^\prime}{x} = \Int[0][\infty]{\frac{\varphi(x) - \varphi(-x)}{x}}{x} = \abraces{\pv\pbraces{\frac{1}{x}}, \phi}
	\end{align*}
\end{enumerate}

\end{solution}

% --------------------------------------------------------------------------------

% --------------------------------------------------------------------------------

\begin{exercise}

Zeigen Sie, dass

\begin{align*}
    \lim_{\varepsilon \to 0+}
    \frac{1}{x - i \varepsilon}
    =
    \pv
    {
        \pbraces
        {
            \frac{1}{x}
        }
    }
    +
    \pi i \delta.
\end{align*}

\end{exercise}

% --------------------------------------------------------------------------------

\begin{solution}

Wir beginnen mit der Rechnung 
\begin{align*}
\frac{1}{x - i\varepsilon} = \frac{x + i \varepsilon}{x^2 + \varepsilon^2} = \frac{x}{x^2 + \varepsilon^2} + i\frac{\varepsilon}{x^2 + \varepsilon^2}.
\end{align*}
Gegen was der zweite Summand konvergiert wissen wir schon aus der zweiten Aufgabe. K"ummern wir uns also um den Ersten. Könnten wir beispielsweise für ein beliebiges $\phi \in \mathcal{D}(\R)$ zeigen, dass (evtl Satz von der monotonen Konvergenz?)
\begin{align*}
\lim_{\varepsilon \to 0+} \abraces{\log\pbraces{\sqrt{x^2 + \varepsilon^2}}, \phi}&= \lim_{\varepsilon \to 0+} \Int[\R][]{\log\pbraces{\sqrt{x^2 + \varepsilon^2}} \phi(x)}{x} = \Int[\R][]{\lim_{\varepsilon \to 0+}\log\pbraces{\sqrt{x^2 + \varepsilon^2}} \phi(x)}{x} = \\
&= \Int[\R][]{\log|x| \phi(x)}{x} = \abraces{\log|x|, \phi},
\end{align*}
so würde mit Lemma 3.13 schon 
\begin{align*}
\frac{x}{x^2 + \varepsilon^2} = \pbraces{\log\pbraces{\sqrt{x^2 + \varepsilon^2}}}^\prime \to \pbraces{\log|x|} = \pv\pbraces{\frac{1}{x}}
\end{align*}
folgen.
\end{solution}

% --------------------------------------------------------------------------------


In den folgenden Beispielen betrachten wir Zeichenfolgen, die aus $1,+,!,=$
zusammengesetzt sind. Unser Ableitungssystem enthält nun das einzige Axiom
$\frac{\emptyset}{1!}$ und die folgenden Regeln
\begin{align*}
  \frac{A!}{1A!} \quad \frac{A!}{A + 1 = A} \quad \frac{A + B = C}{A + B1 = CA}
\end{align*}
für beliebige Zeichenfolgen $A,B,C$.

% --------------------------------------------------------------------------------

\begin{exercise}[Transformations]

Suppose $X$ and $Y$ are independent gamma distributed random variables with $X \sim \operatorname{Gamma}(\alpha_1, \beta)$ and $Y \sim \operatorname{Gamma}(\alpha_2, \beta)$.
Consider the following two random variables

\begin{align*}
    U = X + Y
    \quad
    \text{and}
    \quad
    V = \frac{X}{X + Y}.
\end{align*}

\begin{enumerate}[label = (\alph*)]
    \item Show that $U \sim \operatorname{Gamma}(\alpha_1 + \alpha_2, \beta)$.
    \item Show that $U$ and $V$ are also independent random variables.
\end{enumerate}

\end{exercise}

% --------------------------------------------------------------------------------

\begin{solution}

ToDo!

\end{solution}

% --------------------------------------------------------------------------------

% --------------------------------------------------------------------------------

\begin{exercise}

Eine Distribution heißt \textit{positiv} wenn $\abraces{u, \varphi} \geq 0$ für alle $\varphi \geq 0$ gilt.

\begin{enumerate}[label = (\roman*)]
    \item Zeigen Sie, dass jede positive Distribution Ordnung $0$ hat.
    \item Zeigen Sie, dass folgende Distribution $T \in \mathcal{D}^\prime(\R)$ nicht positiv ist:
    
    \begin{align*}
        \abraces{T, \varphi}
        =
        \Int[-\infty][-1]
        {
            \frac{\varphi(x)}{|x|}
        }{x}
        +
        \Int[1][\infty]
        {
            \frac{\varphi(x)}{|x|}
        }{x}
        +
        \Int[-1][1]
        {
            \frac{\varphi(x) - \varphi(0)}{|x|}
        }{x}.
    \end{align*}

\end{enumerate}

\end{exercise}

% --------------------------------------------------------------------------------

\begin{solution}
\phantom{}
\begin{enumerate}[label = (\roman*)]
	\item Seien $K \subseteq \Omega$ und $\phi \in \mathcal{D}(K)$ beliebig.
	\item Wir wählen die Funktion 
	\begin{align*}
	\phi(x) =
	\begin{cases}
	\exp\pbraces{\frac{1}{x^2 - 1}} &, \vbraces{x} < 1 \\
	0 &, \vbraces{x} \geq 1,
	\end{cases}
	\end{align*}
	welche, wie wir aus der Vorlesung wissen, aus $\mathcal{D}(\R)$ ist und $\phi \geq 0$ erfüllt. Außerdem nimmt die Funktion im Punkt $0$ ihr Maximum an. Deshalb ist sicher
	\begin{align*}
	\abraces{T, \varphi} = \Int[-1][1]
	{
		\frac{\varphi(x) - \varphi(0)}{|x|}
	}{x} < 0 
	\end{align*}
	und damit $T$ nicht positiv.
\end{enumerate}

\end{solution}

% --------------------------------------------------------------------------------


\section*{Logische Axiome, MP}
% --------------------------------------------------------------------------------

\begin{exercise}[Exercise 4.10]

What is the analog of the value iteration update (4.10) for action values $q_{k+1}(s,a)$?

\begin{align}
  v_{k+1}(s) \ &\dot= \ \max_a \E[R_{t+1} + \gamma v_k(S_{t+1}) | S_t = s, A_t = a] \notag \\
  &= \max_a \sum_{s',r} p(s',r,a|s,a)[r + \gamma v_k(s')] \tag{4.10}.
\end{align}

\end{exercise}

% --------------------------------------------------------------------------------

\begin{solution}

\begin{align*}
  q_{k+1}(s,a) \ &\dot= \ \max_{a} \E_\pi[R_{t+1} + \gamma \sum_{a'}q_\pi(S_{t+1},a') | S_t = s, A_t = a] \\
  &= \max_{a} \sum_{s',r} p(s',r|s,a)\left[r + \gamma \sum_{a'} \pi(a'|s')q_k(s',a')\right].
\end{align*}

\end{solution}

% --------------------------------------------------------------------------------


\end{document}
