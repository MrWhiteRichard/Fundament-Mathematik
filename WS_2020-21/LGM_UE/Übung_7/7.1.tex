% --------------------------------------------------------------------------------

\begin{exercise}[142]
Finden Sie ein möglichst interessantes Beispiel $\mathscr{L},\mathscr{M}_1,\mathscr{M}_2,\varphi,b$,
sodass zwar $\mathscr{M}_1 \leq \mathscr{M}_2$ gilt, aber nicht $\mathscr{M}_1 \preccurlyeq \mathscr{M}_2$
(letzteres belegt durch $\varphi$ und $b$).
\end{exercise}

% --------------------------------------------------------------------------------

\begin{solution}
Sei $\mathscr{L} = (+,0)$ die Sprache der Monoide,
\begin{align*}
  \mathscr{M}_1 &= (\N,+_\N,0_\N), \\
  \mathscr{M}_2 &= (\Z,+_\Z,0_\Z)
\end{align*}
Es gilt mit Sicherheit
\begin{align*}
  \N \subseteq \Z, 0_\N = 0_\Z, +_\N = +_\Z\upharpoonright \N \implies \mathscr{M}_1 \leq \mathscr{M}_2.
\end{align*}
Die Formel
\begin{align*}
  \varphi = \exists y\,(x+y = 0)
\end{align*}
ist sicher allgemeingültig in $\Z$, aber in $\N$ unter jeder Belegung $b$ mit $b(x) \neq 0$ falsch.
\end{solution}

% --------------------------------------------------------------------------------
