% --------------------------------------------------------------------------------

\begin{exercise}[173]
Finden Sie (durch Skolemisierung) eine Formel in Pränexform ohne Existenzquantoren,
die zur Formel
\begin{align*}
  \exists x \exists y \forall z\, ((P(x) \lor P(y))\land \neg P(z))
\end{align*}
erfüllungsäquivalent ist.
\end{exercise}

% --------------------------------------------------------------------------------

\begin{solution}
Wir wissen aus der Vorlesung, dass für eine Formel $\psi$ und eine Konstante $c$,
welche nicht in $\psi$ vorkommt, die Formeln
\begin{align*}
  &\exists x\, \psi \quad \text{und}\\
  &\psi[x/c]
\end{align*}
erfüllungsäquivalent ist. Wenden wir diese Erfüllungsäquivalent zweimal an,
dann erhalten wir mit Konstanten $c,d$, welche nicht in $P$ vorkommen:
\begin{align*}
  \forall z\, ((P(c) \lor P(d))\land \neg P(z))
\end{align*}
\end{solution}

% --------------------------------------------------------------------------------

% --------------------------------------------------------------------------------

\begin{exercise}[174]
Finden Sie (durch Skolemisierung) eine Formel in Pränexform ohne Existenzquantoren,
die zur Formel
\begin{align*}
  \forall z \exists x \exists y \, ((P(x) \lor P(y))\land \neg P(z))
\end{align*}
erfüllungsäquivalent ist.
\end{exercise}

% --------------------------------------------------------------------------------

\begin{solution}
Hier verwenden wir zweimal die Erfüllungsäquivalenz von
\begin{align*}
  &\forall x\, \exists y\, \psi \quad \text{und} \\
  &\forall x\, \psi[y/f(x)],
\end{align*}
wobei wir voraussetzen, dass $P$ eine bereinigte Formel ist. Wir erhalten also:
\begin{align*}
  \forall z \, ((P(g(z)) \lor P(f(z)))\land \neg P(z)).
\end{align*}
\end{solution}

% --------------------------------------------------------------------------------
