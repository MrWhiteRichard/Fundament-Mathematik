% --------------------------------------------------------------------------------

\begin{exercise}[143]
Finden Sie ein Beispiel $\mathscr{L},\mathscr{M}_1,\mathscr{M}_2$, sodass
$\mathscr{M}_1 \preccurlyeq \mathscr{M}_2$ und $M_1 \neq M_2$.
\end{exercise}

% --------------------------------------------------------------------------------

\begin{solution}
Wir betrachten eine Sprache ohne Gleichheitsrelation und ohne Konstantensymbole, allerdings
mit einer einstelligen Relation $R$.
\begin{align*}
  \mathscr{M}_1 &= (\{0\}, R^{\mathscr{M}_1}= \emptyset) \\
  \mathscr{M}_2 &= (\{0,1\}, R^{\mathscr{M}_2}= \emptyset) \\
\end{align*}
Jede Atomformel hat also die Form $R(x)$ und ist sowohl in $\mathscr{M}_1$,
als auch in $\mathscr{M}_2$ unter jeder Belegung $b$ falsch. \\
Wir zeigen mittels Induktion nach dem Formelaufbau, dass jede Formel
entweder unter jeder Belegung in $\mathscr{M}_1$ und $\mathscr{M}_1$ wahr oder
unter jeder Belegung $\mathscr{M}_1$ und $\mathscr{M}_1$ falsch ist. \\
Induktiv über den Formelaufbau zeigt man dann, dass das bereits für alle Formeln gelten muss.
\end{solution}

% --------------------------------------------------------------------------------

\begin{solution}
	Ein weiterer Vorschlag ist $\mathscr{L} = \{\leq\}$ und $\mathscr{M}_1 = 2\N$ sowie $\mathscr{M}_2 = \N$. Auch hier ist mir noch kein sauberer Beweis gelungen.
\end{solution}
