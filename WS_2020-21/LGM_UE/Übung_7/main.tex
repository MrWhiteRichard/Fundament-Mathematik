\documentclass{article}

% Hier befinden sich Pakete, die wir beinahe immer benutzen ...

\usepackage[utf8]{inputenc}

% Sprach-Paket:
\usepackage[ngerman]{babel}

% damit's nicht so, wie beim Grill aussieht:
\usepackage{fullpage}

% Mathematik:
\usepackage{amsmath, amssymb, amsfonts, amsthm}
\usepackage{bbm, mathrsfs, stmaryrd}
\usepackage{mathtools, mathdots}

% Makros mit mehereren Default-Argumenten:
\usepackage{twoopt}

% Anführungszeichen (Makro \Quote{}):
\usepackage{babel}

% if's für Makros:
\usepackage{xifthen}
\usepackage{etoolbox}

% tikz ist kein Zeichenprogramm (doch!):
\usepackage{tikz}

% bessere Aufzählungen:
\usepackage{enumitem}

% (bessere) Umgebung für Bilder:
\usepackage{graphicx, subfig, float}

% Umgebung für Code:
\usepackage{listings}

% Farben:
\usepackage{xcolor}

% Umgebung für "plain text":
\usepackage{verbatim}

% Umgebung für mehrerer Spalten:
\usepackage{multicol}

% "nette" Brüche
\usepackage{nicefrac}

% Spaltentypen verschiedener Dicke
\usepackage{tabularx}
\usepackage{makecell}

% Für Vektoren
\usepackage{esvect}

% (Web-)Links
\usepackage{hyperref}

% Zitieren & Literatur-Verzeichnis
\usepackage[style = authoryear]{biblatex}
\usepackage{csquotes}

% so ähnlich wie mathbb
%\usepackage{mathds}

% Keine Ahnung, was das macht ...
\usepackage{booktabs}
\usepackage{ngerman}
\usepackage{placeins}

% special letters:

\newcommand{\N}{\mathbb{N}}
\newcommand{\Z}{\mathbb{Z}}
\newcommand{\Q}{\mathbb{Q}}
\newcommand{\R}{\mathbb{R}}
\newcommand{\C}{\mathbb{C}}
\newcommand{\K}{\mathbb{K}}
\newcommand{\T}{\mathbb{T}}
\newcommand{\E}{\mathbb{E}}
\newcommand{\V}{\mathbb{V}}
\renewcommand{\P}{\mathbb{P}}
\newcommand{\1}{\mathbbm{1}}

\newcommand  {\B}{\mathfrak{B}}
\renewcommand{\S}{\mathfrak{S}}

% quantors:

\newcommand{\Forall}{\forall \,}
\newcommand{\Exists}{\exists \,}
\newcommand{\ExistsOnlyOne}{\exists! \,}
\newcommand{\nExists}{\nexists \,}

% MISC symbols:

\newcommand{\landau}[1]
{
  {\scriptstyle \mathcal{O}}
  \pbraces{#1}
}

\newcommand{\Landau}[1]
{
  \mathcal{O}
  \pbraces{#1}
}

\newcommand{\eps}{\mathrm{eps}}

% graphics in a box:

\newcommandtwoopt
{\includegraphicsboxed}[3][][]
{
  \begin{figure}[!h]
    \begin{boxedin}
      \ifthenelse{\isempty{#2}}
      {
        \begin{center}
          \includegraphics[width = 0.75 \textwidth]{#3}
          \label{fig:#1}
        \end{center}
      }{
        \begin{center}
          \includegraphics[width = 0.75 \textwidth]{#3}
          \caption{#2}
          \label{fig:#1}
        \end{center}
      }
    \end{boxedin}
  \end{figure}
}

% braces:

\newcommand{\pbraces}[1]{{\left  ( #1 \right  )}}
\newcommand{\bbraces}[1]{{\left  [ #1 \right  ]}}
\newcommand{\Bbraces}[1]{{\left \{ #1 \right \}}}
\newcommand{\vbraces}[1]{{\left  | #1 \right  |}}
\newcommand{\Vbraces}[1]{{\left \| #1 \right \|}}
\newcommand{\abraces}[1]{{\left \langle #1 \right \rangle}}
\newcommand{\round}[1]{\bbraces{#1}}

\newcommand
{\floor}[1]
{{\left \lfloor #1 \right \rfloor}}

\newcommand
{\ceil} [1]
{{\left \lceil  #1 \right \rceil }}

% special functions:

\newcommand{\norm}  [2][]{\Vbraces{#2}_{#1}}
\newcommand{\diag}  [1]{\mathrm{diag} \: #1}
\newcommand{\dist}  [1]{\mathrm{dist} \: #1}
\newcommand{\mean}  [1]{\mathrm{mean} \: #1}
\newcommand{\erf}   [1]{\mathrm{erf} \: #1}
\newcommand{\id}    [1]{\mathrm{id} \: #1}
\newcommand{\sgn}   [1]{\mathrm{sgn} \: #1}
\newcommand{\supp}  [1]{\mathrm{supp} \: #1}
\newcommand{\arsinh}[1]{\mathrm{arsinh} \: #1}
\newcommand{\arcosh}[1]{\mathrm{arcosh} \: #1}
\newcommand{\artanh}[1]{\mathrm{artanh} \: #1}
\newcommand{\card}  [1]{\mathrm{card} \: #1}
\newcommand{\Span}  [1]{\mathrm{span} \: #1}
\newcommand{\Aut}   [1]{\mathrm{Aut} \: #1}
\newcommand{\End}   [1]{\mathrm{End} \: #1}
\newcommand{\ggT}   [1]{\mathrm{ggT} \: #1}
\newcommand{\kgV}   [1]{\mathrm{kgV} \: #1}
\newcommand{\ord}   [1]{\mathrm{ord} \: #1}
\newcommand{\grad}  [1]{\mathrm{grad} \: #1}
\newcommand{\ran}   [1]{\mathrm{ran} \: #1}
\newcommand{\graph} [1]{\mathrm{graph} \: #1}
\newcommand{\Inv}   [1]{\mathrm{Inv} \: #1}
\newcommand{\pv}    [1]{\mathrm{pv} \: #1}
\newcommand{\Mod}{\: \mathrm{mod} \:}
\newcommand{\Char}{\mathrm{char}}
\newcommand{\At}{\mathrm{At}}
\newcommand{\Ob}{\mathrm{Ob}}
\newcommand{\Hom}{\mathrm{Hom}}
\newcommand{\orthogonal}[3][]{#2 ~\bot_{#1}~ #3}
\newcommand{\Rang}{\mathrm{Rang}}

\newcommand
{\GL}[2][]
{\mathrm{GL}_{#1} \pbraces{#2}}

% fractions:

\newcommand{\Frac}[2]{\frac{1}{#1} \pbraces{#2}}
\newcommand{\nfrac}[2]{\nicefrac{#1}{#2}}

% derivatives & integrals:

\newcommandtwoopt
{\Int}[4][][]
{\int_{#1}^{#2} #3 ~\mathrm{d} #4}

\newcommandtwoopt
{\derivative}[3][][]
{
  \frac
  {\mathrm{d}^{#1} #2}
  {\mathrm{d} #3^{#1}}
}

\newcommandtwoopt
{\pderivative}[3][][]
{
  \frac
  {\partial^{#1} #2}
  {\partial #3^{#1}}
}

\newcommand
{\primeprime}
{{\prime \prime}}

\newcommand
{\primeprimeprime}
{{\prime \prime \prime}}

% Text:

\newcommand{\Quote}[1]{\glqq #1\grqq{}}
\newcommand{\Text}[1]{{\text{#1}}}
\newcommand{\fastueberall}{\text{f.ü.}}
\newcommand{\fastsicher}{\text{f.s.}}

% -------------------------------- %
% amsthm-stuff:

\theoremstyle{definition}

% numbered theorems
\newtheorem{theorem}    {Satz}   [section]
\newtheorem{lemma}      [theorem]{Lemma}
\newtheorem{corollary}  [theorem]{Korollar}
\newtheorem{proposition}[theorem]{Proposition}
\newtheorem{remark}     [theorem]{Bemerkung}
\newtheorem{definition} [theorem]{Definition}
\newtheorem{example}    [theorem]{Beispiel}

% unnumbered theorems
\newtheorem*{theorem*}    {Satz}
\newtheorem*{lemma*}      {Lemma}
\newtheorem*{corollary*}  {Korollar}
\newtheorem*{proposition*}{Proposition}
\newtheorem*{remark*}     {Bemerkung}
\newtheorem*{definition*} {Definition}
\newtheorem*{example*}    {Beispiel}

% Please define this stuff in project ("main.tex"):

% \def \lastexercisenumber {...}
% This will be 0 by default

% \setcounter{section}{...}
% This will be 0 by default
% and hence, completely ignored

\ifnum \thesection = 0
{
  \newtheorem{exercise}{Aufgabe}
}
\else
{
  \newtheorem{exercise}{Aufgabe}[section]
}
\fi

\ifdef
{\lastexercisenumber}
{\setcounter{exercise}{\lastexercisenumber}}

\newenvironment{solution}
{
  \begin{proof}[Lösung]
}{
  \end{proof}
}

\renewcommand{\proofname}{Beweis}

% -------------------------------- %
% environment zum einkasteln:

% dickere vertical lines
\newcolumntype
{x}
[1]
{
  !{
    \centering
    \arraybackslash
    \vrule
    width #1}
}

% environment selbst (the big cheese)
\newenvironment
{boxedin}
{
  \begin{tabular}
  {
    x{1 pt}
    p{\textwidth}
    x{1 pt}
  }
  \Xhline
  {2 \arrayrulewidth}
}
{
  \\
  \Xhline{2 \arrayrulewidth}
  \end{tabular}
}

% -------------------------------- %
% MISC "Ein-Deutschungen"

\renewcommand{\figurename}{Abbildung}
\renewcommand{\tablename} {Tabelle}

% -------------------------------- %

\input{../../../Fundament-LaTeX/listings.tex}

\parskip 0pt
\parindent 0pt

\title
{
  Logik und Grundlagen der Mathematik \\
  \vspace{4pt}
  \normalsize
  \textit{7. Übung am 19.11.2020}
}
\author
{
  Richard Weiss
  \and
  Florian Schager
  \and
  Fabian Zehetgruber
}
\date{}

\begin{document}

\maketitle
\section*{Skolemtheorien}
Sei $\Sigma$ eine Theorie in der Sprache $\mathscr{L}$. Wir nennen $\Sigma$ eine
Skolem-Theorie, wenn es für alle $n \geq 0$ und alle Formeln der Form $\exists y \psi$
mit den freien Variablen $x_1,\dots,x_n$ ein $n$-stelliges Funktionssymbol $f$ gibt,
sodass $\Sigma \vdash \exists y \psi \rightarrow \psi[y/f(\overline{x})]$ (wobei wir
$f(\overline{x})$ statt $f(x_1,\dots,x_n)$ schreiben). \\
Für zwei $\mathscr{L}$-Strukturen $\mathscr{M}_1$ und $\mathscr{M}_2$ mit $M_1 \subseteq M_2$
sagen wir $\mathscr{M}_1 \leq \mathscr{M}_2$ (``$\mathscr{M}_1$ ist Unterstruktur von $\mathscr{M}_2$''),
wenn für alle Funktionssymbole $f$ in $\mathscr{L}$ gilt, dass
$f^{\mathscr{M}_1} = f^{\mathscr{M}_2}\upharpoonright M_1$ (genauer:
$f^{\mathscr{M}_1} = f^{\mathscr{M}_2}\upharpoonright M_1^k$) wenn $k$ die Stelligkeit von $f$ ist),
sowie $c^{\mathscr{M}_1} = c^{\mathscr{M}_2}$ für alle Konstantensymbole, und
$R^{\mathscr{M}_1} = R^{\mathscr{M}_2}\cap M^k$ für alle ($k$-stelligen) Relationssymbole $R$. \\
Wir schreiben $\mathscr{M}_1 \preccurlyeq \mathscr{M}_2$, wenn erstens $\mathscr{M}_1 \leq \mathscr{M}_2$
gilt, und überdies für jede Formel $\varphi$ und jede Belegung $b$ mit Werten in
$M_1: \mathscr{M}_1 \vDash \varphi[b] \iff \mathscr{M}_2 \vDash \varphi[b]$.
% -------------------------------------------------------------------------------- %

\begin{exercise}[\textbf{Uniform distribution}]

Let $X_1,\dots,X_n$ be a random sample from uniform $(\theta,1)$ distribution,
where $\theta < 1$ is an unknown parameter.

\begin{enumerate}[label = (\alph*)]
  \item Find the MLE $\hat{\theta}$ of $\theta$.
  \item Is $\hat{\theta}$ asymptotically normal? If yes, find the asymptotic
  mean and variance. Otherwise, find a sequence $r_n$ and $a_n$ such that
  $r_n(\hat{\theta} - a_n)$ converges in distribution to a non-degenerate
  (not pointmass) distribution.
\end{enumerate}

\end{exercise}

% -------------------------------------------------------------------------------- %

\begin{solution}

\phantom{}

\begin{enumerate}[label = (\alph*)]
  \item For the maximum likelihood estimation we first determine the
  joint distribution $f_X$.
  \begin{align*}
    L_n(\theta) &= f_X(x_1,\dots,x_n | \theta) 
    = \prod_{i=1}^n f_{X_i}(x_i | \theta)
    = \prod_{i=1}^n \frac{1}{1-\theta}\1_{(\theta,1)}(x_i) \\
    &= \frac{1}{(1-\theta)^n}\1_{(\theta,1)^n}(x_1,\dots,x_n)\\
    &= \begin{cases}
      \frac{1}{(1-\theta)^n}, & \theta < \min_{i=1}^n x_i \\
      0, & \theta \geq \min_{i=1}^n x_i
    \end{cases}.
  \end{align*}

  Clearly, the maximum is attained at the point 
  $\hat{\theta} = \min_{i=1}^n x_i$.

  \item We calculate
  
  \begin{align*}
    F_{\hat{\theta}}(x) &= 1 - \P(\hat{\theta}(X) > x)
    = 1 - \prod_{i=1}^n \frac{(1-x)}{(1-\theta)}
    = 1 - \frac{(1-x)^n}{(1-\theta)^n}
  \end{align*}

  and derive

  \begin{align*}
    f_{\hat{\theta}}(x) = n\frac{(1-x)^{n-1}}{(1-\theta)^{n}}.
  \end{align*}

  

  With the affine transformation $f(x) = r_n(x - a_n)$
  we obtain the distribution

  \begin{align*}
    f_{r_n(\hat{\theta} - a_n)}(y)
    &= f_{\hat{\theta}}(y/r_n + a_n)\frac{1}{r_n} \\
    &= n\frac{(1-y/r_n + a_n)^{n-1}}{(1-\theta)^{n}r_n} \\
    &= \frac{n}{(1- \theta)r_n} \left(\frac{1 - \frac{y}{r_n} - a_n}{1 - \theta}\right)^{n-1}
  \end{align*}

  Setting $r_n = \frac{n - 1}{1 - \theta}$ and $a_n = \theta$ the expression simplifies to
  
  \begin{align*}
    f_{\frac{n - 1}{1 - \theta}(\hat{\theta} - \theta)}(y) &= \frac{n}{n - 1}
    \left(\frac{1 - \theta  - \frac{y(1 - \theta)}{n - 1}}{1 - \theta}\right)^{n-1} \\
    &= \frac{n}{n - 1}\left(1 - \frac{y}{n-1}\right)^{n-1}
    \xrightarrow{n \to \infty} \exp(-y).
  \end{align*}

  Therefore $\frac{n - 1}{1 - \theta}(\hat{\theta} - \theta)$ converges
  in distribution to an exponential distribution with parameter $\lambda = 1$.

\end{enumerate}

\end{solution}

% -------------------------------------------------------------------------------- %

% --------------------------------------------------------------------------------

\begin{exercise}[143]
Finden Sie ein Beispiel $\mathscr{L},\mathscr{M}_1,\mathscr{M}_2$, sodass
$\mathscr{M}_1 \preccurlyeq \mathscr{M}_2$ und $M_1 \neq M_2$.
\end{exercise}

\begin{solution}
	Da wir das Gleichheitssymbol ohnehin in unserer Sprache haben wollen wir uns nicht weiter belasten und nicht mehr in unsere Sprache aufnehmen, also keine Konstantensymbole, Funktionssymbole oder weitere Relationssymbole. Wäre $M_1$ endlich mit $n \in \N$ Elementen, dann gelten wegen $ |M_2| > |M_1| = n$ die Aussagen
	\begin{align*}
	\mathscr{M}_1 \nvDash \exists^{\geq n + 1} \ [b], \quad \mathscr{M}_2 \vDash \exists^{\geq n + 1} \ [b]
	\end{align*}
	 Also müssen $M_1$ und $M_2$ mindestens abzählbar undendlich viele Elemente haben. Wir wählen $\mathscr{M}_1 = \N \setminus \{0\}$ sowie $\mathscr{M}_2 = \N$.
\end{solution}

% --------------------------------------------------------------------------------

\begin{exercise}[Minimum variance estimator]

Let $W_1, \dots, W_k$ be unbiased estimators of a parameter $\theta$ with $\Var \sigma_i^2$ and $\Cov(W_i, W_j) = 0$ if $i \neq j$.
Show that, of all estimators of the form $\sum a_i W_i$ where $a_i$s are constant and $\E_\theta \pbraces{\sum a_i W_i} = \theta$, the estimator

\begin{align*}
    W^\ast
    =
    \frac
    {
        \sum W_i / \sigma_i^2
    }{
        \sum (1 / \sigma_i^2)
    }
\end{align*}

has minimum variance.
Show that

\begin{align*}
    \Var W^\ast
    =
    \frac
    {
        1
    }{
        \sum (1 / \sigma_i^2)
    }.
\end{align*}

\end{exercise}

% --------------------------------------------------------------------------------

\begin{solution}

Let $W = \sum_{i=1}^n a_i W_i$ be another estimator of the upper form.
Because $W_1, \dots, W_n$ are unbiased, $\E W_1 = \cdots = \E W_n = \theta$, and thus, for $W$ to be unbiased as well,

\begin{align*}
    \theta
    & \stackrel{!}{=}
    \E W \\
    & =
    \E \pbraces{\sum_{i=1}^n a_i W_i} \\
    & =
    \sum_{i=1}^n a_i \E W_i \\
    & =
    \theta \sum_{i=1}^n a_i.
\end{align*}

Because $\Cov(W_i, W_j) = 0$ if $i \neq j$,

\begin{align*}
    \Var W
    & =
    \Var \pbraces{\sum_{i=1}^n a_i W_i} \\
    & =
    \sum_{i=1}^n a_i^2 \Var W_i \\
    & =
    \sum_{i=1}^n a_i^2 \sigma_i^2.
\end{align*}

Hence, we get the optimisation problem

\begin{align*}
    W^\ast
    \stackrel{!}{=}
    \sum_{i=1}^n a_i^2 \sigma_i^2
    \stackrel{!}{=}
    \min,
\end{align*}

subject to the constraint

\begin{align*}
    \sum_{i=1}^n a_i = 1.
\end{align*}

Choosing the coefficients

\begin{align*}
    a_i
    :=
    \frac
    {
        1 / \sigma_i^2
    }{
        \sum_{j=1}^n 1 / \sigma_j^2
    },
    \quad
    \text{for}
    \quad
    i = 1, \dots, n,
\end{align*}

we see, that $W^\ast = \sum_{i=1}^n a_i W_i$ is indeed a convex combination.
Plugging $a_1, \dots, a_n$ into the variance formula derived above immediately yields the desired result for $\Var W^\ast$.

In order to show the minimisation property of $W^\ast$, let

\begin{align*}
    \varepsilon_i \in \R,
    \quad
    \delta_i
    :=
    \varepsilon_i \sum_{j=1}^n 1 / \sigma_j^2
    =
    \varepsilon_i a_i \sigma_i^2 \pbraces{\sum_{i=1}^n 1 / \sigma_i^2}^2,
    \quad
    \text{for}
    \quad
    i = 1, \dots, n,
\end{align*}

such that

\begin{align*}
    1
    & =
    \sum_{i=1}^n (a_i + \varepsilon_i) \\
    & =
    \underbrace
    {
        \sum_{i=1}^n a_i
    }_1
    +
    \sum_{i=1}^n \varepsilon_i.
\end{align*}

We ths get

\begin{align*}
    \sum_{i=1}^n \delta_i
    & =
    \underbrace
    {
        \sum_{i=1}^n \varepsilon_i
    }_0
    \sum_{j=1}^n 1 / \sigma_j^2 \\
    & =
    0,
\end{align*}

and finally,

\begin{align*}
    \sum_{i=1}^n (c_i + \varepsilon_i)^2 \sigma_i^2
    & =
    \sum_{i=1}^n a_i^2 \sigma_i^2
    +
    2 \sum_{i=1}^n a_i \varepsilon_i \sigma_i^2
    +
    \underbrace
    {
        \sum_{i=1}^n \varepsilon_i^2 \sigma_i^2
    }_0 \\
    & \geq
    W^\ast
    +
    2
    \underbrace
    {
        \sum_{i=1}^n \delta_i
    }_0
    \Bigg / \sum_{j=1}^n 1 / \sigma_i^2 \\
    & =
    W^\ast.
\end{align*}

\end{solution}

% --------------------------------------------------------------------------------

% --------------------------------------------------------------------------------

\begin{exercise}

  Zeigen Sie die \textit{2. poincarésche Ungleichung:} Sei $\Omega \subseteq \R^n$ ein beschränktes Gebiet mit $\partial\Omega \in C^1.$ Dann existiert eine Konstante $C > 0,$ sodass für alle $u \in H^1(\Omega)$
  \begin{align*}
      \| u - \overline{u} \|_{L^2(\Omega)} \leq C \| \nabla u \|_{L^2(\Omega)}
  \end{align*}
  gilt, wobei $\overline{u} := \frac{1}{|\Omega|} \int_\Omega u(x) \mathrm{~d}x$.

  \textit{Hinweis:} Sie dürfen folgende Aussage ohne Beweis verwenden: Sei $\Omega \subseteq \R^n$ ein beschränktes Gebiet mit $\partial\Omega \in C^1$ und $u \in H^1(\Omega)$
  mit $\| \nabla u \|_{L^2(\Omega)} = 0.$ Dann ist $u$ eine konstante Funktion.

\end{exercise}

% --------------------------------------------------------------------------------

\begin{solution}

Wir führen einen Widerspruchsbeweis. Angenommen, es gäbe keine solche Konstante,
dann finden wir eine Folge $(u_n)_{n \in \N} \subset H^1(\Omega)$ mit
\begin{align*}
  \|u_n - \overline{u_n}\|_{L^2(\Omega)} \geq n\|\nabla u_n\|_{L^2(\Omega)}
\end{align*}
Definiere nun $v_n := \frac{u_n - \overline{u_n}}{\|u_n - \overline{u_n}\|_{L^2(\Omega)}}$.
Dann folgt
\begin{align*}
  \|v_n\|_{L^2(\Omega)} &= 1, \\
  \qquad \|\nabla v_n\|_{L^2(\Omega)} &\leq \frac{1}{n}, \\
  \int_\Omega v_n &= \frac{1}{\|u_n -\overline{u_n}\|_{L^2(\Omega)}}
  \left(\int_{\Omega}u_n - \frac{1}{|\Omega|}\int_\Omega u_n dx dy\right)
  = \frac{1}{\|u_n -\overline{u_n}\|_{L^2(\Omega)}}
  \left(\int_{\Omega}u_n dy - \int_\Omega u_n dx\right) = 0.
\end{align*}
Insbesondere ist $(v_n)_{n \in \N}$ eine beschränkte Folge in $H^1(\Omega)$.
Aus dem Satz von Rellich-Kondrachov erhalten wir vermöge der kompakten Einbettung
$H^1(\Omega) \hookrightarrow H^0(\Omega) = L^2(\Omega)$ eine konvergente Teilfolge
$(v_{n_k})_{k\in \N}$ in $L^2(\Omega)$. Für den Grenzwert $v := \lim_{k \to \infty} v_{n_k}$ gilt
\begin{align*}
  \|v\|_{L^2(\Omega)} = 1, \quad \int_\Omega v = 0.
\end{align*}
Außerdem ist $(v_{n_k})_{k \in \N}$ sogar eine Cauchy-Folge in $H^1(\Omega)$
und konvergiert daher in $H^1(\Omega)$ gegen den selben Grenzwert.
Aus der Stetigkeit von $\|\nabla(\cdot)\|_{L^2(\Omega)}$ bezüglich der $H^1(\Omega)$-Norm
folgt also
\begin{align*}
  \|\nabla v\| = \lim_{k \to \infty}\|\nabla v_{n_k}\|_{L^2(\Omega)} = 0.
\end{align*}
Laut Hinweis dürfen wir nun behaupten, dass $v$ bereits konstant sein muss.
Aus $\int_\Omega v = 0$ folgt damit sogar $v \equiv 0$, im Widerspruch zu $\|v\|_{L^2(\Omega)} = 1$.
\end{solution}

% --------------------------------------------------------------------------------

\section*{Skolemisierung}
% --------------------------------------------------------------------------------

\begin{exercise}[\textbf{Exponential family}]

Show that a Poisson family of distributions $\Poi(\lambda)$, with unknown
$\lambda > 0$ belongs to the exponential family.
    
\end{exercise}
    
% --------------------------------------------------------------------------------
    
\begin{solution}
    
The pdf of the Poisson family with parameter $\lambda$ reads

\begin{align*}
    f_\lambda(x) = \frac{\lambda^x}{x!}\exp(-\lambda)
    =  \frac{1}{x!}\exp(\log(\lambda)x-\lambda).
\end{align*}

If we now define $h(x) = \frac{1}{x!}, w_1(\lambda) = \log(\lambda), 
t_1(x) = x, w_2(\lambda) = - \lambda, t_2(x) = 1$,
we see the the Poisson family indeed belongs to the exponential family.
    
\end{solution}
    
% --------------------------------------------------------------------------------
    
\begin{algebraUE}{215}
Jede archimedisch angeordnete abelsche Gruppe $G$ lässt sich als solche isomorph
in die geordnete additive Gruppe $\R$ der reellen Zahlen einbetten. (Umgekehrt
ist jede additive Untergruppe von $\R$ archimedisch angeordnet.)
Ist $\iota: G \mapsto \R$ eine solche isomorphe Einbettung, so sind alle anderen
gegeben durch sämtliche Abbildungen $\lambda_l$ mit reellem $\lambda > 0$. \\
\textit{Anleitung für die Existenz von $\iota$}: Gehen Sie für nichttriviales
$G$ von einem positiven Element $g \in G$ aus, das Sie auf die reelle Zahl
$1 = \iota(g)$ abbilden. Wegen der archimedischen Eigenschaft definiert das
einen eindeutigen ordnungsverträglichen Homomorphismus $\iota$, der
(wieder wegen der archimedischen Eigenschaft) sogar injektiv sein muss.
\end{algebraUE}
\begin{solution}
Wir setzen o.B.d.A. $G \neq \{0\}$, also $\exists 0 \neq g \in G^+$.
Wir setzen $\iota(g) = 1$ und haben damit
\begin{align*}
  \forall k \in \Z: \iota(kg) = k\iota(g) = k.
\end{align*}
Für $G = \Z g$ sind wir damit bereits fertig.
Anderenfalls, wähle $h \in G^+\backslash \Z g$ und setze
\begin{align*}
  \forall n \in \N\backslash \{0\}: m_n^h &:= \min\{k \in \N: 2^nh \leq kg\}.
\end{align*}
Damit $\iota$ ein Ordnungisomorphismus werden kann, muss gelten
\begin{align*}
  2^n \iota(h) &= \iota(2^nh) \leq \iota(m_n^hg) = m_n^h \iff \iota(h) \leq \frac{m_n^h}{2^n} \\
  2^n \iota(h) &= \iota(2^nh) \geq \iota((m_n^h - 1)g) = m_n^h - 1 \iff \iota(h) \geq \frac{m_n^h - 1}{2^n},
\end{align*}
also
\begin{align*}
  \forall n \in \N: \iota(h) \in \left[\frac{m_n^h-1}{2^n},\frac{m_n^h}{2^n}\right]
  \iff \iota(h) \in \bigcap_{n \in \N} \left[\frac{m_n^h-1}{2^n},\frac{m_n^h}{2^n}\right].
\end{align*}
Es gilt
\begin{align*}
  (m_n - 1)g \leq 2^nh \leq m_ng.
\end{align*}
Es folgt
\begin{align*}
  2^{n+1}h &\leq 2\cdot2^nh \leq 2 m_n g \implies m_{n+1} \leq 2m_n
  \iff \frac{m_{n+1}}{2^{n+1}} \leq \frac{m_n}{2^n} \\
  2^{n+1}h &\geq 2 (m_n - 1)g \implies m_{n+1} - 1 \geq 2 (m_n - 1)
  \iff \frac{m_{n+1}-1}{2^{n+1}} \geq \frac{m_n-1}{2^n}\\
  &\iff \left[\frac{m_{n+1} - 1}{2^{n+1}},\frac{m_{n+1}}{2^{n+1}}\right] \subset \left[\frac{m_{n} - 1}{2^{n}},\frac{m_{n}}{2^{n}}\right]
\end{align*}
Induktiv folgt dann, dass mit $[\frac{m_n^h-1}{2^n},\frac{m_n^h}{2^n}]_{n \in \N}$
eine absteigende Mengenfolge vorliegt und der Schnitt darüber ist somit nicht leer.
Aufgrund der Vollständigkeit der reellen Zahlen existiert also
\begin{align*}
  \iota(h) = x = \lim_{n \rightarrow \infty} \frac{m_n^h}{2^n} = \lim_{n \rightarrow \infty} \frac{m_n^h-1}{2^n}
\end{align*}
und somit wird $\iota$ bereits auf ganz $G$ eindeutig festgelegt. \\
Jetzt müssen wir noch zeigen, dass diese Konstruktion auch ein Homomorphismus ist.
\begin{align*}
  \iota(0) &= \iota(0g) = 0\iota(g) = 0 \\
  \iota(-h) &= -\iota(h)
\end{align*}
Für die Verträglichkeit mit der Addition müssen wir etwas mehr arbeiten:
\begin{align*}
  \forall n \in \N:& 2^nh_1 \leq m_n^{h_1}g \land 2^nh_2 \leq m_n^{h_2}g \\
  &\implies 2^n(h_1 + h_2) \leq (m_n^{h_1} + m_n^{h_2})g \implies m_n^{h_1} + m_n^{h_2} \geq m_n^{h_1 + h_2} \\
  \forall n \in \N:& 2^nh_1 \geq (m_n^{h_1}-1)g \land 2^nh_2 \geq (m_n^{h_2}-1)g \\
  &\implies 2^n(h_1 + h_2) \geq (m_n^{h_1} + m_n^{h_2} - 2)g \implies m_n^{h_1} + m_n^{h_2} \leq m_n^{h_1 + h_2} + 1
\end{align*}
Daraus folgt
\begin{align*}
  \iota(h_1) + \iota(h_2) = \lim_{n \rightarrow \infty} \frac{m_n^{h_1} + m_n^{h_2}}{2^n}
  = \lim_{n \rightarrow \infty} \frac{m_n^{h_1 + h_2}}{2^n} = \iota(h_1 + h_2).
\end{align*}
Nun zur Injektivität von $\iota$: Sei $h \in G^+\backslash\{0\}$ beliebig.
Dann existiert ein $n \in \N: g < nh$, weil die Gruppe $G$
archimedisch angeordnet ist. Daraus folgt $0 < 1 = \iota(g) < \iota(nh) = n\iota(h)$.
Daraus erhalten wir schließlich die Injektivität
und gleichzeitig auch die Ordnungsverträglichkeit, denn für $a > b \in G$ gilt
\begin{align*}
  \iota(a) - \iota(b) = \iota(a-b) > 0 \implies \iota(a) > \iota(b).
\end{align*}
Schließlich noch zu Eindeutigkeit von $\iota$: \\
Sei nun $\alpha$ eine weiter isomorphe Einbettung mit
\begin{align*}
  \alpha(g) = \lambda = \lambda \iota(g) = \iota(\lambda g).
\end{align*}
Wie zuvor ist dadurch schon
\begin{align*}
  \forall n \in \N: \lambda \frac{m_n^h - 1}{2^n} = \frac{m_n^h - 1}{2^n}\alpha(g) \leq \alpha(h) \leq \frac{m_n^h}{2^n}\alpha(g)
  = \lambda \frac{m_n^h}{2^n}
\end{align*}
$\alpha(h)$ eindeutig festgelegt und es gilt $\alpha = \lambda \iota$.
\end{solution}




\end{document}
