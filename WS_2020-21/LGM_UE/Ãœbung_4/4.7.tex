% --------------------------------------------------------------------------------

\begin{exercise}[98]

Der \blockquote{indirekte Beweis}: Wenn $\Sigma \cup \{\varphi\} \vdash \bot$, dann
$\Sigma \vdash \neg \varphi$. Erklären Sie, wie man aus einem formalen Beweis
$P_1$ für $\Sigma \cup \{\varphi\} \vdash \bot$ einen formalen Beweis $P_2$
für $\Sigma \vdash \neg \varphi$ machen kann, und geben Sie eine obere
Abschätzung für die Länge von $P_2$ verglichen mit der Länge von $P_1$ an. \\
Ebenso: Wenn $\Sigma \cup \{\neg \varphi\} \vdash \bot$, dann $\Sigma \vdash \varphi$.

\end{exercise}

% --------------------------------------------------------------------------------

\begin{solution}

Sei $\varphi_1,\dots,\varphi_n = \bot$ ein formaler Beweis aus $\Sigma \cup \{\varphi\}$.
Mit dem Deduktionstheorem erhalten wir in $3n$ Schritten
\begin{align*}
  \Sigma \vdash \varphi \rightarrow \bot.
\end{align*}
Mit der Tautologie $(\varphi \rightarrow \bot) \rightarrow \neq \varphi$
und einmaligen Anwenden von Modus Ponens erhalten wir $\Sigma \vdash \neg \varphi$
mit höchstens $3n + 2$ Ableitungsschritten.

\end{solution}

% --------------------------------------------------------------------------------
