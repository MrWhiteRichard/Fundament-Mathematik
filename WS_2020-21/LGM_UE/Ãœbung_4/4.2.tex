% -------------------------------------------------------------------------------- %

\begin{exercise}[78]

$(\forall x\, \exists y\, R(x,y)) \rightarrow (\forall y\, \exists x\, R(y,x)), \quad
(\forall x\, \exists y\, x \leq y) \rightarrow (\forall y\, \exists x\, y \leq x)$

\end{exercise}

% -------------------------------------------------------------------------------- %

\begin{solution}
Wir zeigen, dass die beiden Formeln allgemeingültig sind. \\
Sei dazu $\mathscr{M}$ eine $\mathscr{L}$-Struktur, $b$ eine beliebige Belegung.
Im Falle, dass $\hat{b}(\forall x\, \exists y\, R(x,y)) = 0$ sind wir bereits fertig. \\
Gelte also
\begin{align*}
  1 = \hat{b}(\forall x\, \exists y\, R(x,y)).
\end{align*}
Dann folgt mit
\begin{align*}
  \hat{b}(\forall x\, \exists y\, R(x,y))
  &= \inf\{\widehat{b_{x/m}}(\exists y\, R(x,y): m \in M)\} \\
  &= \inf\{\sup\{\widehat{(b_{x/m})_{y/n}}(R(x,y): n \in M\} m \in M)\} \\
  &= \inf\{\sup\{\widehat{(b_{y/m})_{x/n}}(R(y,x): n \in M\} m \in M)\} \\
  &= \hat{b}(\forall y\, \exists x\, R(y,x)),
\end{align*}
dass auch $1 = \hat{b}(\forall y\, \exists x\, R(y,x))$. \\
Für die zweite Formel setze einfach $R = \leq$.

\end{solution}

% -------------------------------------------------------------------------------- %
