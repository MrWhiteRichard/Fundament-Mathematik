\documentclass{article}

% ---------------------------------------------------------------- %
% short package descriptions are copied from
% https://ctan.org/

% ---------------------------------------------------------------- %

% Accept different input encodings
\usepackage[utf8]{inputenc}

% Standard package for selecting font encodings
\usepackage[T1]{fontenc}

% ---------------------------------------------------------------- %

% Multilingual support for Plain TEX or LATEX
\usepackage[ngerman]{babel}

% ---------------------------------------------------------------- %

% Set all page margins to 1.5cm
\usepackage{fullpage}

% Margin adjustment and detection of odd/even pages
\usepackage{changepage}

% Flexible and complete interface to document dimensions
\usepackage{geometry}

% ---------------------------------------------------------------- %
% mathematics

\usepackage{amsmath}  % AMS mathematical facilities for LATEX
\usepackage{amssymb}
\usepackage{amsfonts} % TEX fonts from the American Mathematical Society
\usepackage{amsthm}   % Typesetting theorems (AMS style)

% Mathematical tools to use with amsmath
\usepackage{mathtools}

% Support for using RSFS fonts in maths
\usepackage{mathrsfs}

% Commands to produce dots in math that respect font size
\usepackage{mathdots}

% "Blackboard-style" cm fonts
\usepackage{bbm}

% Typeset in-line fractions in a "nice" way
\usepackage{nicefrac}

% Typeset quotient structures with LATEX
\usepackage{faktor}

% Vector arrows
\usepackage{esvect}

% St Mary Road symbols for theoretical computer science
\usepackage{stmaryrd}

% Three series of mathematical symbols
\usepackage{mathabx}

% ---------------------------------------------------------------- %
% algorithms

% Package for typesetting pseudocode
\usepackage{algpseudocode}

% Typeset source code listings using LATEX
\usepackage{listings}

% Reimplementation of and extensions to LATEX verbatim
\usepackage{verbatim}

% If necessary, please use the following 2 packages locally, but never both.
% This is because the algorithm environment gets defined in both packages, which leads to name conflicts.
% \usepackage{algorithm2e}
% \usepackage{algorithm}

% ---------------------------------------------------------------- %
% utilities

% A generic document command parser
\usepackage{xparse}

% Extended conditional commands
\usepackage{xifthen}

% e-TEX tools for LATEX
\usepackage{etoolbox}

% Define commands with suffixes
\usepackage{suffix}

% Extensive support for hypertext in LATEX
\usepackage{hyperref}

% Driver-independent color extensions for LATEX and pdfLATEX
\usepackage{xcolor}

% ---------------------------------------------------------------- %
% graphics

% -------------------------------- %

\usepackage{tikz}

% MISC
\usetikzlibrary{patterns}
\usetikzlibrary{decorations.markings}
\usetikzlibrary{positioning}
\usetikzlibrary{arrows}
\usetikzlibrary{arrows.meta}
\usetikzlibrary{overlay-beamer-styles}

% finite state machines
\usetikzlibrary{automata}

% turing machines
\usetikzlibrary{calc}
\usetikzlibrary{chains}
\usetikzlibrary{decorations.pathmorphing}

% -------------------------------- %

% Draw tree structures
\usepackage[noeepic]{qtree}

% Enhanced support for graphics
\usepackage{graphicx}

% Figures broken into subfigures
\usepackage{subfig}

% Improved interface for floating objects
\usepackage{float}

% Control float placement
\usepackage{placeins}

% Include PDF documents in LATEX
\usepackage{pdfpages}

% ---------------------------------------------------------------- %

% Control layout of itemize, enumerate, description
\usepackage[inline]{enumitem}

% Intermix single and multiple columns
\usepackage{multicol}
\setlength{\columnsep}{1cm}

% Coloured boxes, for LATEX examples and theorems, etc
\usepackage{tcolorbox}

% ---------------------------------------------------------------- %
% tables

% Tabulars with adjustable-width columns
\usepackage{tabularx}

% Tabular column heads and multilined cells
\usepackage{makecell}

% Publication quality tables in LATEX
\usepackage{booktabs}

% ---------------------------------------------------------------- %
% bibliography and quoting

% Sophisticated Bibliographies in LATEX
\usepackage[backend = biber, style = alphabetic]{biblatex}

% Context sensitive quotation facilities
\usepackage{csquotes}

% ---------------------------------------------------------------- %

% ---------------------------------------------------------------- %
% special letters

\newcommand{\N}{\mathbb N}
\newcommand{\Z}{\mathbb Z}
\newcommand{\Q}{\mathbb Q}
\newcommand{\R}{\mathbb R}
\newcommand{\C}{\mathbb C}
\newcommand{\K}{\mathbb K}
\newcommand{\T}{\mathbb T}
\newcommand{\E}{\mathbb E}
\newcommand{\V}{\mathbb V}
\renewcommand{\S}{\mathbb S}
\renewcommand{\P}{\mathbb P}
\newcommand{\1}{\mathbbm 1}
\newcommand{\G}{\mathbb G}

\newcommand{\iu}{\mathrm i}

% ---------------------------------------------------------------- %
% quantors

\newcommand{\Forall}        {\forall ~}
\newcommand{\Exists}        {\exists ~}
\newcommand{\nExists}       {\nexists ~}
\newcommand{\ExistsOnlyOne} {\exists! ~}
\newcommand{\nExistsOnlyOne}{\nexists! ~}
\newcommand{\ForAlmostAll}  {\forall^\infty ~}

% ---------------------------------------------------------------- %
% graphics boxed

\newcommand
{\includegraphicsboxed}
[2][0.75]
{
    \begin{center}
        \begin{tcolorbox}[standard jigsaw, opacityback = 0]

            \centering
            \includegraphics[width = #1 \textwidth]{#2}

        \end{tcolorbox}
    \end{center}
}

\newcommand
{\includegraphicsunboxed}
[2][0.75]
{
    \begin{center}
        \includegraphics[width = #1 \textwidth]{#2}
    \end{center}
}

\NewDocumentCommand
{\includegraphicsgraphicsboxed}
{ O{0.75} O{0.25} m m}
{
    \begin{center}
        \begin{tcolorbox}[standard jigsaw, opacityback = 0]

            \centering
            \includegraphics[width = #1 \textwidth]{#3} \\
            \vspace{#2 cm}
            \includegraphics[width = #1 \textwidth]{#4}

        \end{tcolorbox}
    \end{center}
}

\NewDocumentCommand
{\includegraphicsgraphicsunboxed}
{ O{0.75} O{0.25} m m}
{
    \begin{center}

        \centering
        \includegraphics[width = #1 \textwidth]{#3} \\
        \vspace{#2 cm}
        \includegraphics[width = #1 \textwidth]{#4}

    \end{center}
}

% ---------------------------------------------------------------- %
% braces

\newcommand{\pbraces}[1]{{\left  ( #1 \right  )}}
\newcommand{\bbraces}[1]{{\left  [ #1 \right  ]}}
\newcommand{\Bbraces}[1]{{\left \{ #1 \right \}}}
\newcommand{\vbraces}[1]{{\left  | #1 \right  |}}
\newcommand{\Vbraces}[1]{{\left \| #1 \right \|}}

\newcommand{\abraces}[1]{{\left \langle #1 \right \rangle}}

\newcommand{\floorbraces}[1]{{\left \lfloor #1 \right \rfloor}}
\newcommand{\ceilbraces} [1]{{\left \lceil  #1 \right \rceil }}

\newcommand{\dbbraces}    [1]{{\llbracket     #1 \rrbracket}}
\newcommand{\dpbraces}    [1]{{\llparenthesis #1 \rrparenthesis}}
\newcommand{\dfloorbraces}[1]{{\llfloor       #1 \rrfloor}}
\newcommand{\dceilbraces} [1]{{\llceil        #1 \rrceil}}

\newcommand{\dabraces}[1]{{\left \langle \left \langle #1 \right \rangle \right \rangle}}

\newcommand{\abs}  [1]{\vbraces{#1}}
\newcommand{\round}[1]{\bbraces{#1}}
\newcommand{\floor}[1]{\floorbraces{#1}}
\newcommand{\ceil} [1]{\ceilbraces{#1}}

% ---------------------------------------------------------------- %

% MISC

% metric spaces
\newcommand{\norm}[2][]{\Vbraces{#2}_{#1}}
\DeclareMathOperator{\metric}{d}
\DeclareMathOperator{\dist}  {dist}
\DeclareMathOperator{\diam}  {diam}

% O-notation
\newcommand{\landau}{{\scriptstyle \mathcal{O}}}
\newcommand{\Landau}{\mathcal{O}}

% ---------------------------------------------------------------- %

% math operators

% hyperbolic trigonometric function inverses
\DeclareMathOperator{\areasinh}{areasinh}
\DeclareMathOperator{\areacosh}{areacosh}
\DeclareMathOperator{\areatanh}{areatanh}

% special functions
\DeclareMathOperator{\id} {id}
\DeclareMathOperator{\sgn}{sgn}
\DeclareMathOperator{\Inv}{Inv}
\DeclareMathOperator{\erf}{erf}
\DeclareMathOperator{\pv} {pv}

% exponential function as power
\WithSuffix \newcommand \exp* [1]{\mathrm{e}^{#1}}

% operations on sets
\DeclareMathOperator{\meas}{meas}
\DeclareMathOperator{\card}{card}
\DeclareMathOperator{\Span}{span}
\DeclareMathOperator{\conv}{conv}
\DeclareMathOperator{\cof}{cof}
\DeclareMathOperator{\mean}{mean}
\DeclareMathOperator{\avg}{avg}
\DeclareMathOperator*{\argmax}{argmax}
\DeclareMathOperator*{\argsmax}{argsmax}

% number theory stuff
\DeclareMathOperator{\ggT}{ggT}
\DeclareMathOperator{\kgV}{kgV}
\DeclareMathOperator{\modulo}{mod}

% polynomial stuff
\DeclareMathOperator{\ord}{ord}
\DeclareMathOperator{\grad}{grad}

% function properties
\DeclareMathOperator{\ran}{ran}
\DeclareMathOperator{\supp}{supp}
\DeclareMathOperator{\graph}{graph}
\DeclareMathOperator{\dom}{dom}
\DeclareMathOperator{\Def}{def}
\DeclareMathOperator{\rg}{rg}

% matrix stuff
\DeclareMathOperator{\GL}{GL}
\DeclareMathOperator{\SL}{SL}
\DeclareMathOperator{\U}{U}
\DeclareMathOperator{\SU}{SU}
\DeclareMathOperator{\PSU}{PSU}
% \DeclareMathOperator{\O}{O}
% \DeclareMathOperator{\PO}{PO}
% \DeclareMathOperator{\PSO}{PSO}
\DeclareMathOperator{\diag}{diag}

% algebra stuff
\DeclareMathOperator{\At}{At}
\DeclareMathOperator{\Ob}{Ob}
\DeclareMathOperator{\Hom}{Hom}
\DeclareMathOperator{\End}{End}
\DeclareMathOperator{\Aut}{Aut}
\DeclareMathOperator{\Lin}{L}

% other function classes
\DeclareMathOperator{\Lip}{Lip}
\DeclareMathOperator{\Mod}{Mod}
\DeclareMathOperator{\Dil}{Dil}

% constants
\DeclareMathOperator{\NIL}{NIL}
\DeclareMathOperator{\eps}{eps}

% ---------------------------------------------------------------- %
% doubble & tripple powers

\newcommand
{\primeprime}
{{\prime \prime}}

\newcommand
{\primeprimeprime}
{{\prime \prime \prime}}

\newcommand
{\astast}
{{\ast \ast}}

\newcommand
{\astastast}
{{\ast \ast \ast}}

% ---------------------------------------------------------------- %
% derivatives

\NewDocumentCommand
{\derivative}
{ O{} O{} m m}
{
    \frac
    {\mathrm d^{#2} {#1}}
    {\mathrm d {#3}^{#2}}
}

\NewDocumentCommand
{\pderivative}
{ O{} O{} m m}
{
    \frac
    {\partial^{#2} {#1}}
    {\partial {#3}^{#2}}
}

\DeclareMathOperator{\Div}{div}
\DeclareMathOperator{\rot}{rot}

% ---------------------------------------------------------------- %
% integrals

\NewDocumentCommand
{\Int}
{ O{} O{} m m}
{\int_{#1}^{#2} #3 ~ \mathrm d #4}

\NewDocumentCommand
{\Iint}
{ O{} O{} m m m}
{\iint_{#1}^{#2} #3 ~ \mathrm d #4 ~ \mathrm d #5}

\NewDocumentCommand
{\Iiint}
{ O{} O{} m m m m}
{\iiint_{#1}^{#2} #3 ~ \mathrm d #4 ~ \mathrm d #5 ~ \mathrm d #6}

\NewDocumentCommand
{\Iiiint}
{ O{} O{} m m m m m}
{\iiiint_{#1}^{#2} #3 ~ \mathrm d #4 ~ \mathrm d #5 ~ \mathrm d #6 ~ \mathrm d #7}

\NewDocumentCommand
{\Idotsint}
{ O{} O{} m m m}
{\idotsint_{#1}^{#2} #3 ~ \mathrm d #4 \dots ~ \mathrm d #5}

\NewDocumentCommand
{\Oint}
{ O{} O{} m m}
{\oint_{#1}^{#2} #3 ~ \mathrm d #4}

% ---------------------------------------------------------------- %

% source:
% https://tex.stackexchange.com/questions/203257/tikz-chains-with-one-side-of-the-leftmost-node-thickbold

% #1 (optional): current state, e.g. $q_0$
% #2: cursor position, e.g. 1
% #3: number of displayed cells, e.g. 5
% #4: contents of cells, e.g. {$\triangleright$, $x_1$, \dots, $x_n$, \textvisiblespace}

\newcommand{\turingtape}[4][]
{
    \begin{tikzpicture}

        \tikzset{tape/.style={minimum size=.7cm, draw}}

        \begin{scope}[start chain=0 going right, node distance=0mm]
            \foreach \x [count=\i] in #4
            {
                \ifnum\i=#3 % if last node reset outer sep to 0pt
                    \node [on chain=0, tape, outer sep=0pt] (n\i) {\x};
                    \draw (n\i.north east) -- ++(.1,0) decorate [decoration={zigzag, segment length=.12cm, amplitude=.02cm}] {-- ($(n\i.south east)+(+.1,0)$)} -- (n\i.south east) -- cycle;
                \else
                    \node [on chain=0, tape] (n\i) {\x};
                \fi

                \ifnum\i=1 % if first node draw a thick line at the left
                    \draw [line width=.1cm] (n\i.north west) -- (n\i.south west);
                \fi
            }
 
            \node [right=.25cm of n#3] {$\cdots$};
            \node [tape, above left=.25cm and 1cm of n1] (q) {#1};
            \draw [>=latex, ->] (q) -| (n#2);

        \end{scope}

    \end{tikzpicture}
}

% ---------------------------------------------------------------- %

% ---------------------------------------------------------------- %
% amsthm-environments:

\theoremstyle{definition}

% numbered theorems
\newtheorem{theorem}             {Satz}[section]
\newtheorem{lemma}      [theorem]{Lemma}
\newtheorem{corollary}  [theorem]{Korollar}
\newtheorem{proposition}[theorem]{Proposition}
\newtheorem{remark}     [theorem]{Bemerkung}
\newtheorem{definition} [theorem]{Definition}
\newtheorem{example}    [theorem]{Beispiel}
\newtheorem{heuristics} [theorem]{Heuristik}

% unnumbered theorems
\newtheorem*{theorem*}    {Satz}
\newtheorem*{lemma*}      {Lemma}
\newtheorem*{corollary*}  {Korollar}
\newtheorem*{proposition*}{Proposition}
\newtheorem*{remark*}     {Bemerkung}
\newtheorem*{definition*} {Definition}
\newtheorem*{example*}    {Beispiel}
\newtheorem*{heuristics*} {Heuristik}

% ---------------------------------------------------------------- %
% exercise- and solution-environments:

% Please define this stuff in project ("main.tex"):
% \def \lastexercisenumber {...}

\newtheorem{exercise}{Aufgabe}
\setcounter{exercise}{\lastexercisenumber}

\newenvironment{solution}
{
  \begin{proof}[Lösung]
}{
  \end{proof}
}

% ---------------------------------------------------------------- %
% MISC translations for environment-names

\renewcommand{\proofname} {Beweis}
\renewcommand{\figurename}{Abbildung}
\renewcommand{\tablename} {Tabelle}

% ---------------------------------------------------------------- %

% ---------------------------------------------------------------- %
% https://www.overleaf.com/learn/latex/Code_listing

\definecolor{codegreen} {rgb}{0, 0.6, 0}
\definecolor{codegray}    {rgb}{0.5, 0.5, 0.5}
\definecolor{codepurple}{rgb}{0.58, 0, 0.82}
\definecolor{backcolour}{rgb}{0.95, 0.95, 0.92}

\lstdefinestyle{overleaf}
{
    backgroundcolor = \color{backcolour},
    commentstyle = \color{codegreen},
    keywordstyle = \color{magenta},
    numberstyle = \tiny\color{codegray},
    stringstyle = \color{codepurple},
    basicstyle = \ttfamily \footnotesize,
    breakatwhitespace = false,
    breaklines = true,
    captionpos = b,
    keepspaces = true,
    numbers = left,
    numbersep = 5pt,
    showspaces = false,
    showstringspaces = false,
    showtabs = false,
    tabsize = 2
}

% ---------------------------------------------------------------- %
% https://en.wikibooks.org/wiki/LaTeX/Source_Code_Listings

\lstdefinestyle{customc}
{
    belowcaptionskip = 1 \baselineskip,
    breaklines = true,
    frame = L,
    xleftmargin = \parindent,
    language = C,
    showstringspaces = false,
    basicstyle = \footnotesize \ttfamily,
    keywordstyle = \bfseries \color{green!40!black},
    commentstyle = \itshape \color{purple!40!black},
    identifierstyle = \color{blue},
    stringstyle = \color{orange},
}

\lstdefinestyle{customasm}
{
    belowcaptionskip = 1 \baselineskip,
    frame = L,
    xleftmargin = \parindent,
    language = [x86masm] Assembler,
    basicstyle = \footnotesize\ttfamily,
    commentstyle = \itshape\color{purple!40!black},
}

% ---------------------------------------------------------------- %
% https://tex.stackexchange.com/questions/235731/listings-syntax-for-literate

\definecolor{maroon}        {cmyk}{0, 0.87, 0.68, 0.32}
\definecolor{halfgray}      {gray}{0.55}
\definecolor{ipython_frame} {RGB}{207, 207, 207}
\definecolor{ipython_bg}    {RGB}{247, 247, 247}
\definecolor{ipython_red}   {RGB}{186, 33, 33}
\definecolor{ipython_green} {RGB}{0, 128, 0}
\definecolor{ipython_cyan}  {RGB}{64, 128, 128}
\definecolor{ipython_purple}{RGB}{170, 34, 255}

\lstdefinestyle{stackexchangePython}
{
    breaklines = true,
    %
    extendedchars = true,
    literate =
    {á}{{\' a}} 1 {é}{{\' e}} 1 {í}{{\' i}} 1 {ó}{{\' o}} 1 {ú}{{\' u}} 1
    {Á}{{\' A}} 1 {É}{{\' E}} 1 {Í}{{\' I}} 1 {Ó}{{\' O}} 1 {Ú}{{\' U}} 1
    {à}{{\` a}} 1 {è}{{\` e}} 1 {ì}{{\` i}} 1 {ò}{{\` o}} 1 {ù}{{\` u}} 1
    {À}{{\` A}} 1 {È}{{\' E}} 1 {Ì}{{\` I}} 1 {Ò}{{\` O}} 1 {Ù}{{\` U}} 1
    {ä}{{\" a}} 1 {ë}{{\" e}} 1 {ï}{{\" i}} 1 {ö}{{\" o}} 1 {ü}{{\" u}} 1
    {Ä}{{\" A}} 1 {Ë}{{\" E}} 1 {Ï}{{\" I}} 1 {Ö}{{\" O}} 1 {Ü}{{\" U}} 1
    {â}{{\^ a}} 1 {ê}{{\^ e}} 1 {î}{{\^ i}} 1 {ô}{{\^ o}} 1 {û}{{\^ u}} 1
    {Â}{{\^ A}} 1 {Ê}{{\^ E}} 1 {Î}{{\^ I}} 1 {Ô}{{\^ O}} 1 {Û}{{\^ U}} 1
    {œ}{{\oe}}  1 {Œ}{{\OE}}  1 {æ}{{\ae}}  1 {Æ}{{\AE}}  1 {ß}{{\ss}}  1
    {ç}{{\c c}} 1 {Ç}{{\c C}} 1 {ø}{{\o}} 1 {å}{{\r a}} 1 {Å}{{\r A}} 1
    {€}{{\EUR}} 1 {£}{{\pounds}} 1
}


% Python definition (c) 1998 Michael Weber
% Additional definitions (2013) Alexis Dimitriadis
% modified by me (should not have empty lines)

\lstdefinelanguage{iPython}{
    morekeywords = {access, and, break, class, continue, def, del, elif, else, except, exec, finally, for, from, global, if, import, in, is, lambda, not, or, pass, print, raise, return, try, while}, %
    %
    % Built-ins
    morekeywords = [2]{abs, all, any, basestring, bin, bool, bytearray, callable, chr, classmethod, cmp, compile, complex, delattr, dict, dir, divmod, enumerate, eval, execfile, file, filter, float, format, frozenset, getattr, globals, hasattr, hash, help, hex, id, input, int, isinstance, issubclass, iter, len, list, locals, long, map, max, memoryview, min, next, object, oct, open, ord, pow, property, range, raw_input, reduce, reload, repr, reversed, round, set, setattr, slice, sorted, staticmethod, str, sum, super, tuple, type, unichr, unicode, vars, xrange, zip, apply, buffer, coerce, intern}, %
    %
    sensitive = true, %
    morecomment = [l] \#, %
    morestring = [b]', %
    morestring = [b]", %
    %
    morestring = [s]{'''}{'''}, % used for documentation text (mulitiline strings)
    morestring = [s]{"""}{"""}, % added by Philipp Matthias Hahn
    %
    morestring = [s]{r'}{'},     % `raw' strings
    morestring = [s]{r"}{"},     %
    morestring = [s]{r'''}{'''}, %
    morestring = [s]{r"""}{"""}, %
    morestring = [s]{u'}{'},     % unicode strings
    morestring = [s]{u"}{"},     %
    morestring = [s]{u'''}{'''}, %
    morestring = [s]{u"""}{"""}, %
    %
    % {replace}{replacement}{lenght of replace}
    % *{-}{-}{1} will not replace in comments and so on
    literate = 
    {á}{{\' a}} 1 {é}{{\' e}} 1 {í}{{\' i}} 1 {ó}{{\' o}} 1 {ú}{{\' u}} 1
    {Á}{{\' A}} 1 {É}{{\' E}} 1 {Í}{{\' I}} 1 {Ó}{{\' O}} 1 {Ú}{{\' U}} 1
    {à}{{\` a}} 1 {è}{{\` e}} 1 {ì}{{\` i}} 1 {ò}{{\` o}} 1 {ù}{{\` u}} 1
    {À}{{\` A}} 1 {È}{{\' E}} 1 {Ì}{{\` I}} 1 {Ò}{{\` O}} 1 {Ù}{{\` U}} 1
    {ä}{{\" a}} 1 {ë}{{\" e}} 1 {ï}{{\" i}} 1 {ö}{{\" o}} 1 {ü}{{\" u}} 1
    {Ä}{{\" A}} 1 {Ë}{{\" E}} 1 {Ï}{{\" I}} 1 {Ö}{{\" O}} 1 {Ü}{{\" U}} 1
    {â}{{\^ a}} 1 {ê}{{\^ e}} 1 {î}{{\^ i}} 1 {ô}{{\^ o}} 1 {û}{{\^ u}} 1
    {Â}{{\^ A}} 1 {Ê}{{\^ E}} 1 {Î}{{\^ I}} 1 {Ô}{{\^ O}} 1 {Û}{{\^ U}} 1
    {œ}{{\oe}}  1 {Œ}{{\OE}}  1 {æ}{{\ae}}  1 {Æ}{{\AE}}  1 {ß}{{\ss}}  1
    {ç}{{\c c}} 1 {Ç}{{\c C}} 1 {ø}{{\o}} 1 {å}{{\r a}} 1 {Å}{{\r A}} 1
    {€}{{\EUR}} 1 {£}{{\pounds}} 1
    %
    {^}{{{\color{ipython_purple}\^ {}}}} 1
    { = }{{{\color{ipython_purple} = }}} 1
    %
    {+}{{{\color{ipython_purple}+}}} 1
    {*}{{{\color{ipython_purple}$^\ast$}}} 1
    {/}{{{\color{ipython_purple}/}}} 1
    %
    {+=}{{{+=}}} 1
    {-=}{{{-=}}} 1
    {*=}{{{$^\ast$ = }}} 1
    {/=}{{{/=}}} 1,
    literate = 
    *{-}{{{\color{ipython_purple} -}}} 1
     {?}{{{\color{ipython_purple} ?}}} 1,
    %
    identifierstyle = \color{black}\ttfamily,
    commentstyle = \color{ipython_cyan}\ttfamily,
    stringstyle = \color{ipython_red}\ttfamily,
    keepspaces = true,
    showspaces = false,
    showstringspaces = false,
    %
    rulecolor = \color{ipython_frame},
    frame = single,
    frameround = {t}{t}{t}{t},
    framexleftmargin = 6mm,
    numbers = left,
    numberstyle = \tiny\color{halfgray},
    %
    %
    backgroundcolor = \color{ipython_bg},
    % extendedchars = true,
    basicstyle = \scriptsize,
    keywordstyle = \color{ipython_green}\ttfamily,
}

% ---------------------------------------------------------------- %
% https://tex.stackexchange.com/questions/417884/colour-r-code-to-match-knitr-theme-using-listings-minted-or-other

\geometry{verbose, tmargin = 2.5cm, bmargin = 2.5cm, lmargin = 2.5cm, rmargin = 2.5cm}

\definecolor{backgroundCol}  {rgb}{.97, .97, .97}
\definecolor{commentstyleCol}{rgb}{0.678, 0.584, 0.686}
\definecolor{keywordstyleCol}{rgb}{0.737, 0.353, 0.396}
\definecolor{stringstyleCol} {rgb}{0.192, 0.494, 0.8}
\definecolor{NumCol}         {rgb}{0.686, 0.059, 0.569}
\definecolor{basicstyleCol}  {rgb}{0.345, 0.345, 0.345}

\lstdefinestyle{stackexchangeR}
{
    language = R,                                        % the language of the code
    basicstyle = \small \ttfamily \color{basicstyleCol}, % the size of the fonts that are used for the code
    % numbers = left,                                      % where to put the line-numbers
    numberstyle = \color{green},                         % the style that is used for the line-numbers
    stepnumber = 1,                                      % the step between two line-numbers. If it is 1, each line will be numbered
    numbersep = 5pt,                                     % how far the line-numbers are from the code
    backgroundcolor = \color{backgroundCol},             % choose the background color. You must add \usepackage{color}
    showspaces = false,                                  % show spaces adding particular underscores
    showstringspaces = false,                            % underline spaces within strings
    showtabs = false,                                    % show tabs within strings adding particular underscores
    % frame = single,                                      % adds a frame around the code
    % rulecolor = \color{white},                           % if not set, the frame-color may be changed on line-breaks within not-black text (e.g. commens (green here))
    tabsize = 2,                                         % sets default tabsize to 2 spaces
    captionpos = b,                                      % sets the caption-position to bottom
    breaklines = true,                                   % sets automatic line breaking
    breakatwhitespace = false,                           % sets if automatic breaks should only happen at whitespace
    keywordstyle = \color{keywordstyleCol},              % keyword style
    commentstyle = \color{commentstyleCol},              % comment style
    stringstyle = \color{stringstyleCol},                % string literal style
    literate = %
    *{0}{{{\color{NumCol} 0}}} 1
     {1}{{{\color{NumCol} 1}}} 1
     {2}{{{\color{NumCol} 2}}} 1
     {3}{{{\color{NumCol} 3}}} 1
     {4}{{{\color{NumCol} 4}}} 1
     {5}{{{\color{NumCol} 5}}} 1
     {6}{{{\color{NumCol} 6}}} 1
     {7}{{{\color{NumCol} 7}}} 1
     {8}{{{\color{NumCol} 8}}} 1
     {9}{{{\color{NumCol} 9}}} 1
}

% ---------------------------------------------------------------- %
% Fundament Mathematik

\lstdefinestyle{fundament}{basicstyle = \ttfamily}

% ---------------------------------------------------------------- %


\parskip 0pt
\parindent 0pt

\title
{
  Logik und Grundlagen der Mathematik \\
  \vspace{4pt}
  \normalsize
  \textit{4. Übung am 29.10.2020}
}
\author
{
  Richard Weiss
  \and
  Florian Schager
  % \and
  % Christian Sallinger
  \and
  Fabian Zehetgruber
  % \and
  % Paul Winkler
  % \and
  % Christian Göth
}
\date{}

\begin{document}

\maketitle

\section*{Spektren}

Für jede geschlossene Formel $\varphi$ (das heißt: $\varphi$ hat keine freien Variablen)
definieren wir das Spektrum $Sp(\varphi)$ als die Menge aller natürlichen Zahlen,
sodass es ein endliches Modell von $\varphi$ mit genau $n$ Elementen gibt. \\
(Wir sagen, dass $\mathcal{M}$ ein Modell von $\varphi$ ist, wenn für alle Belegungen
$b$ die Gleichung $\hat{b}(\varphi) = 1$ gilt.)

% --------------------------------------------------------------------------------

\begin{exercise}[The mean of independent normal distributions]

\phantom{}

\begin{enumerate}[label = (\alph*)]

    \item Show that the moment generating function (mgf) of $X \sim \mathcal N(\mu \sigma^2)$ is of the form
    
    \begin{align*}
        M_X(t) = \exp*{\mu t + \frac{\sigma^2 t^2}{2}}.
    \end{align*}

    \item Let $X \sim \mathcal N(\mu, \sigma^2)$ and let $Y = a X + b$ with fixed real constants $a$ and $b$.
    Show that $Y \sim \mathcal N(a \mu + b, a^2 \sigma^2)$.

    \item Let $X_1, \dots, X_n$ be independent identically distributed random variables with $X_1 \sim \mathcal N(\mu, \sigma^2)$.
    Show that the mean $\bar X = \frac{1}{n} (X_1 + \cdots + X_n)$ is also normally distributed and $\bar X \sim \mathcal N(\mu, \frac{\sigma^2}{n})$.

\end{enumerate}

\end{exercise}

% --------------------------------------------------------------------------------

\begin{solution}

ToDo!

\end{solution}

% --------------------------------------------------------------------------------


\section*{Prädikatenlogik: Gültigkeit}

Wir betrachten in den folgenden Übungsbeispielen eine prädikatenlogische Sprache
mit Relationssymbolen $P, Q, R, \leq$, sowie (wenn nötig oder sinnvoll) weiteren
Funktions- und Konstantensymbolen $f, g, +, 0, c, d, \dots$
(Die Stelligkeit ist jeweils dem Kontext zu entnehmen.) \\
Welche der folgenden Formeln sind allgemeingültig (gelten also in jeder Struktur
unserer Sprache, unter jeder Belegung)?
Geben Sie gegebenfalls ein Gegenbeispiel an (wenn möglich, ein endliches).

% --------------------------------------------------------------------------------

\begin{exercise}[78]

$(\forall x\, \exists y\, R(x,y)) \rightarrow (\forall y\, \exists x\, R(y,x)), \quad
(\forall x\, \exists y\, x \leq y) \rightarrow (\forall y\, \exists x\, y \leq x)$

\end{exercise}

% --------------------------------------------------------------------------------

\begin{solution}
Wir zeigen, dass die beiden Formeln allgemeingültig sind. \\
Sei dazu $\mathscr{M}$ eine $\mathscr{L}$-Struktur, $b$ eine beliebige Belegung.
Im Falle, dass $\hat{b}(\forall x\, \exists y\, R(x,y)) = 0$ sind wir bereits fertig. \\
Gelte also
\begin{align*}
  1 = \hat{b}(\forall x\, \exists y\, R(x,y)).
\end{align*}
Dann folgt mit
\begin{align*}
  \hat{b}(\forall x\, \exists y\, R(x,y))
  &= \inf\{\widehat{b_{x/m}}(\exists y\, R(x,y): m \in M)\} \\
  &= \inf\{\sup\{\widehat{(b_{x/m})_{y/n}}(R(x,y): n \in M\} m \in M)\} \\
  &= \inf\{\sup\{\widehat{(b_{y/m})_{x/n}}(R(y,x): n \in M\} m \in M)\} \\
  &= \hat{b}(\forall y\, \exists x\, R(y,x)),
\end{align*}
dass auch $1 = \hat{b}(\forall y\, \exists x\, R(y,x))$. \\
Für die zweite Formel setze einfach $R = \leq$.

\end{solution}

% --------------------------------------------------------------------------------


\section*{Logische Axiome, MP}

% -------------------------------------------------------------------------------- %

\begin{exercise}

Gegeben $v \in C^2(\R)$, sei $u(x,t) = v\left(x/\sqrt{t}\right)$ für $t > 0$ und $x \in \R$.
\begin{enumerate}[label = (\roman*)]
  \item Zeigen Sie:
  \begin{align*}
    u_t = u_{xx} \iff v^{\primeprime}(z) + \frac{z}{2}v^{\prime}(z) = 0.
  \end{align*}
  Berechnen Sie die allgemeine Lösung $v$ und damit $u$.
  \item Wählen Sie die Konstanten in $u$ so, dass
  \begin{align*}
    \lim_{t \to 0^+} u(x,t) = 0 \text{ für } x < 0, \quad \lim_{t \to 0^+} u(x,t) = 1
    \text{ für } x > 0.
  \end{align*}
  \item Zeigen Sie, dass für $\varphi \in \mathcal{D}(\R)$ die Funktion
  $f(x,t) = (\partial_x u(\cdot,t)\ast \varphi)(x)$ (Faltung in der $x$-Variablen)
  folgendes Anfangswertproblem für die Wärmeleitungsgleichung löst:
  \begin{align*}
    f_t - f_{xx} &= 0 \\
    \lim_{t \to 0^+} f(t,x) &= \varphi(x)
  \end{align*}
\end{enumerate}
\end{exercise}

% -------------------------------------------------------------------------------- %

\begin{solution}
\phantom{}
\begin{enumerate}[label = (\roman*)]
  \item Da $t > 0$ gilt
  \begin{align*}
    0 &= u_t - u_{xx} = \partial_t v(x/\sqrt{t}) - \partial_{xx}v(x/\sqrt{t})
    = -\frac{x}{2\sqrt{t}^3}v^{\prime}(x/\sqrt{t}) - \frac{1}{t}v^{\primeprime}(x/\sqrt{t})
    = - \frac{z}{2t}v^{\prime}(z) - \frac{1}{t}v^{\primeprime}(z) \\
    &\iff \frac{z}{2}v^{\prime}(z) + v^{\primeprime}(z) = 0.
  \end{align*}
  Nun lösen wir die gewöhnliche Differentialgleichung für $w := v^{\prime}$
  \begin{align*}
    w^{\prime}(z) + \frac{z}{2}w(z) = 0 &\iff \frac{w^{\prime}(z)}{w(z)} = -\frac{z}{2}
    \iff \ln(w(z))^{\prime} = -\frac{z}{2} \iff \ln(w(z)) = -\frac{z^2}{4} + C_0 \\
    &\iff w(z) = C_1\exp\left(-z^2/4\right).
  \end{align*}
  Also erhalten wir $v(z) = C_1\int_0^z \exp(-s^2/4) ds + C_2$ und
  $u(x,t) = v(x/\sqrt{t}) = C_1\int_0^{x/\sqrt{t}} \exp(-s^2/4) ds + C_2$.
  \item
  \begin{align*}
    \lim_{t \to 0^+} u(t,x) = \lim_{t \to 0^+} C_1\int_0^{x/\sqrt{t}} \exp(-s^2/4) ds + C_2
    = \sgn(x)\sqrt{\pi}C_1 + C_2.
  \end{align*}
  Wir erhalten also die Gleichungen
  \begin{align*}
    -\sqrt{\pi}C_1 + C_2 &= 0   \iff C_2 = C_1\sqrt{\pi}\\
    \sqrt{\pi}C_1 + C_2 &= 1 \iff 2C_1\sqrt{\pi} = 1 \iff C_1 = \frac{1}{2\sqrt{\pi}}
    \iff C_2 = \frac{1}{2}.
  \end{align*}
  Unter diesen zusätzlichen Bedingungen lautet unsere Lösung nun
  \begin{align*}
    u(x,t) = \frac{1}{2\sqrt{\pi}}\int_0^{x/\sqrt{t}}\exp(-s^2/4)ds + \frac{1}{2}.
  \end{align*}
  \item
  Da $u(\cdot,t)$ und $\partial_x u(\cdot,1)$ stetig sind, sind sie insbesondere
  auch lokal integrierbar und es folgt mit Lemma 3.14
  \begin{align*}
    f_t - f_{xx} &= \partial_t((\partial_x u(\cdot,t)\ast \varphi)(x))
    - \partial_{xx}(\partial_x u(\cdot,t)\ast \varphi)(x) \\
    &= (\partial_{tx} u(\cdot,t) \ast \varphi)(x) - (\partial_{xxx} u(\cdot,t) \ast \varphi)(x) \\
    &= ((\partial_{tx} u(\cdot,t) - \partial_{xxx} u(\cdot,t)) \ast \varphi)(x) \\
    &= ((\partial_{x} (\partial_t u(\cdot,t) - \partial_{xx} u(\cdot,t))) \ast \varphi)(x) = 0.
  \end{align*}

  \begin{align*}
    \lim_{t \to 0^+} f(t,x) &= \lim_{t \to 0^+} (\partial_x u(\cdot,t)\ast \varphi)(x)
    = \lim_{t \to 0^+} (u(\cdot,t)\ast \varphi^{\prime})(x)
    = \lim_{t \to 0^+} \int_{\R}u(x-y,t) \varphi^{\prime}(y) dy \\
    &= \int_{\R}\lim_{t \to 0^+} u(x-y,t) \varphi^{\prime}(y) dy
    = \int_{-\infty}^x \lim_{t \to 0^+} u(x-y,t) \varphi^{\prime}(y) dy
    + \int_{x}^{\infty} \lim_{t \to 0^+} u(x-y,t) \varphi^{\prime}(y) dy \\
    &= \int_{-\infty}^x \varphi^{\prime}(y) dy = \varphi(x).
  \end{align*}
  Die Vertauschung von Grenzwert und Integral gelingt mittels dominierter Konvergenz
  wegen der Abschätzung
  \begin{align*}
    |u(x-y,t)| = \frac{1}{2\sqrt{\pi}}\int_0^{(x-y)/\sqrt{t}}\exp(-s^2/4)ds + \frac{1}{2}
    \leq \frac{1}{2\sqrt{\pi}}\int_0^{\infty}\exp(-s^2/4)ds + \frac{1}{2}
  \end{align*} \\
\end{enumerate}

\end{solution}

% -------------------------------------------------------------------------------- %


\section*{Modelle von Formeln, $\vDash$}

\begin{algebraUE}{151}
Seien $G,H$ Gruppen, $\sim$ eine Äquivalenzrelation auf $G$ und $f: G \rightarrow H$
eine Abbildung. Dann gilt: \\
Ist $\sim$ eine Kongruenzrelation auf $G$ bezüglich der binären Operation, dann
sogar bezüglich der Gruppenstruktur.
\end{algebraUE}
\begin{solution}
Es gilt also:
\begin{align*}
  \forall a,b,c,d \in G: a_1 \sim b_1, a_2 \sim b_2 \implies a_1a_2 \sim b_1b_2
\end{align*}
Für die 0-stellige Operation des neutralen Elements wird die Kongruenzbedingung
aufgrund der Reflexivität von jeder Äquivalenzrelation erfüllt.
Gelte nun $a \sim b$, dann folgt aufgrund $a^{-1} \sim a^{-1}, b^{-1} \sim b^{-1}$
\begin{align*}
  a \sim b, a^{-1} \sim a^{-1} \implies aa^{-1} = e \sim ba^{-1} \\
  b^{-1} \sim b^{-1}, e \sim ba^{-1} \implies b^{-1} =  b^{-1}e \sim bb^{-1}a^{-1} = a^{-1}
\end{align*}
Damit ist $\sim$ sogar eine Kongruenzrelation bezüglich der Gruppenstruktur.
\end{solution}


\section*{Formale Beweise, $\vdash$}

% --------------------------------------------------------------------------------

\begin{exercise}

Bestimmen Sie die formal adjungierten Operatoren von
\begin{enumerate}[label = (\roman*)]
  \item $L\phi = a(x,y)\phi_x + b(x,y)\phi_y + c(x,y)\phi, \quad \phi \in \mathcal{D}(\R^2)$,
  \item $L\phi = x^2\phi^{\primeprime} + \phi^{\prime} - 3x^2\phi, \quad \phi \in \mathcal{D}(\R)$,
  \item $L\phi = \Delta\phi + v(x)\nabla \phi, \quad \phi \in \mathcal{D}(\R^n)$,
\end{enumerate}
mit $a,b,c \in C^{\infty}(\R^2)$ und $v \in C^{\infty}(\R^n;\R^n)$.

\end{exercise}

% --------------------------------------------------------------------------------

\begin{solution}

\phantom{}

\end{solution}

% --------------------------------------------------------------------------------

% -------------------------------------------------------------------------------- %

\begin{exercise}

ToDo!

\end{exercise}

% -------------------------------------------------------------------------------- %

\begin{solution}

ToDo!

\end{solution}

% -------------------------------------------------------------------------------- %


\section*{Deduktionstheorem, halbformale Beweise}

Beachten Sie in den folgenden Aufgaben, dass es immer um formale Beweise (oder Ableitungen)
geht, und wir den Vollständigkeitssatz noch nicht bewiesn haben. Sie können also
nicht mit der \enquote{Wahrheit} oder \enquote{Gültigkeit} von Formeln argumentieren,
sondern nur mit dem (unseren) formalen Beweisbegriff, also mit Axiomen und Modus Ponens.

% -------------------------------------------------------------------------------- %

\begin{exercise}

ToDo!

\end{exercise}

% -------------------------------------------------------------------------------- %

\begin{solution}

ToDo!

\end{solution}

% -------------------------------------------------------------------------------- %


\end{document}
