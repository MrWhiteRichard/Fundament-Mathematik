% --------------------------------------------------------------------------------

\begin{exercise}[96]

Sei $\Sigma$ eine Menge von Formeln. Die folgenden Aussagen sind äquivalent:

\begin{enumerate}[label = \alph*)]
  \item $\Sigma \vdash \bot$.
  \item Für alle $\varphi$ gilt $\Sigma \vdash \varphi$.
  \item Für alle $\varphi$ gibt es eine endliche Teilmenge $\Sigma' \subseteq \Sigma$ mit $\Sigma' \vdash \varphi$.
\end{enumerate}
\end{exercise}

% --------------------------------------------------------------------------------

\begin{solution}

Wir zeigen zuerst $a) \implies b)$: \\
Sei $\varphi_1, \dots \varphi_n = \bot$ ein formaler Beweis aus $\Sigma$.
Um daraus einen Beweis für $\varphi$ zu konstruieren, wählen wir
$\varphi_{n + 1} = \bot \rightarrow \varphi$ als Tautologie und $\varphi_{n + 2} = \varphi$
als Modus Ponens der beiden vorhergehenden Formeln. \\
Für $b) \implies c)$ betrachte wieder $\varphi$ beliebig mit ihrer formalen Ableitung
 $\varphi_1, \dots \varphi_n = \bot$ aus $\Sigma$. Dann ist
 \begin{align*}
   \Sigma' := \Sigma \cap \{\varphi_1,\dots,\varphi_n\}
 \end{align*}
 eine endliche Teilmenge mit $\Sigma' \vdash \varphi$ (selber Beweis). \\
 Für $c) \implies a)$ gelte also $\Sigma' \vdash \varphi$ und
 $\Sigma^{''} \vdash \neg \varphi$. Schreiben wir die beiden Beweise hintereinander
 erhalten wir mit der Tautologie $\varphi \rightarrow \neg \varphi \rightarrow \bot$
 und zweimaligen Anwenden von Modus Ponens, dass $(\Sigma' \cup \Sigma^{''}) \vdash \bot$
 und somit insbesondere $\Sigma \vdash \bot$.

\end{solution}

% --------------------------------------------------------------------------------
