% -------------------------------------------------------------------------------- %

\begin{exercise}[128]

Wir betrachten die Sprache der Gruppentheorie (ein zweistelliges Funktionssymbol $*$,
ein einstelliges $^{-1}$, ein Konstantensymbol $1$). Lösen Sie möglichst viele
der folgenden Probleme (bzw. zeigen Sie gegebenenfalls, dass sie unlösbar sind):
\begin{itemize}
  \item Finden Sie eine Formel $\sigma_G$, sodass $\Mod(\sigma_G)$ die Klasse
  aller Gruppen ist.
  \item Finden Sie eine Formel $\sigma_{eG}$, sodass $\Mod(\sigma_{eG})$ die
  Klasse aller endlichen Gruppen ist.
  \item Finden Sie eine Formel $\sigma_{uG}$, sodass $\Mod(\sigma_{uG})$ die Klasse
  aller unendlichen Gruppen ist.
  \item Finden Sie eine Formelmenge $\Sigma_{eG}$, sodass $\Mod(\Sigma_{eG})$ die
  Klasse aller endlichen Gruppen ist.
  \item Finden Sie eine Formelmenge $\Sigma_{uG}$, dass $\Mod(\Sigma_{uG})$ die
  Klasse aller unendlichen Gruppen ist.
\end{itemize}
\textit{Hinweis:} Verwenden Sie für den Nachweis der Unlösbarkeit die Formeln
$\exists^{\geq n}$ und den Kompaktheitssatz.
\end{exercise}

% -------------------------------------------------------------------------------- %

\begin{solution}
Wir wissen, dass zu jedem positiven $n \in \N$ eine (zyklische) Gruppe der Ordnung $n$
existiert, sowie Gruppen von unendlicher Ordnung.
\begin{itemize}
  \item
  \begin{align*}
    \sigma_G = \forall x \forall y \forall z (x*(y*z) = (x*y)*z)\ \land \\
    \exists e \forall x (e*x = x = x*e)\ \land \\
    \forall x (x * x^{-1} = x^{-1} * x = e)
  \end{align*}
  \item Angenommen, es existiert eine Formel $\sigma_{eG}$, sodass $\Mod(\sigma_{eG})$
  die Klasse aller endlichen Gruppen ist. Dann ist $\Sigma := \{\exists^{\geq n}: n \in \N\} \cup \{\sigma_{eG}\}$ erfüllbar, da jede endliche Teilmenge erfüllbar ist.
  $\Sigma$ selbst ist aber klarerweise unerfüllbar. Widerspruch!
  \item Angenommen, es existiert eine Formel $\sigma_{uG}$, sodass $\Mod(\sigma_{uG})$ die Klasse aller unendlichen Gruppen ist. Dann ist $\Sigma = \{\neg \sigma_{uG}, \sigma_G\}$ eine
  Formelmenge, die alle endlichen Gruppen beschreibt, was allerdings laut dem
  nächsten Punkt nicht möglich ist.
  \item Angenommen, es existiert eine Formelmenge $\Sigma_{eG}$, sodass $\Mod(\Sigma_{eG})$
  die Klasse aller endlichen Gruppen ist.
  Betrachte mal wieder $\Sigma := \{\exists^{\geq n}: n \in \N\}\, \cup\, \Sigma_{eG}$.
  Wieder ist jede endliche Teilmenge davon erfüllbar, aber um $\Sigma$ zu erfüllen,
  müssten wir eine endliche Gruppe finden, deren Ordnung jedes $n \in \N$ übersteigt.
  Widerspruch!
  \item $\Sigma_{uG} = \{\exists^{\geq n}: n \in \N\} \cup \{\sigma_G\}$
\end{itemize}

\end{solution}

% -------------------------------------------------------------------------------- %
