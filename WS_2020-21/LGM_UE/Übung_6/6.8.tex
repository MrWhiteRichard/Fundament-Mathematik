% -------------------------------------------------------------------------------- %

\begin{exercise}[132]

Sei $M$ eine nichtleere Menge, $\mathscr{L}$ eine prädikatenlogische Sprache, und
sei $X$ die Menge aller $\mathscr{L}$-Strukturen mit Universum $M$. Zeigen Sie:
\begin{enumerate}[label = \alph*.]
  \item Die Familie $\{\Mod(\sigma): \sigma \text{ geschlossene Formel }\}$
  bildet Basis einer $0$-dimensionalen Topologie $\tau$ auf $X$.
  Eine Topologie heißt $0$-dimensional, wenn es eine Basis gibt, die unter
  Komplementsen abgeschlossen ist, oder äquivalent: Wenn es eine Basis gibt,
  deren Mengen alle clopen, also offen und abgeschlossen, sind. Das klassische
  Beispiel eines $0$-dimensionalen, aber nicht diskreten Raums ist die Cantormenge.
  \item Im allgemeinen ist $(X, \tau)$ kein Hausdorffraum. Geben Sie eine sinnvolle
  Äquivalenzrelation $\sim$ auf $X$ und eine sinnvolle Topologie $\tau_{\sim}$
  auf $X/\sim$ an, sodass $X/\sim$ Hausdorffraum ist.
  \item $(X,\tau)$ ist kompakt, ebenso $(X/\sim,\tau_\sim)$, also jede offene
  Überdeckung hat eine endliche Teilüberdeckung.
\end{enumerate}

\end{exercise}

% -------------------------------------------------------------------------------- %

\begin{solution}

\phantom{}

\end{solution}

% -------------------------------------------------------------------------------- %
