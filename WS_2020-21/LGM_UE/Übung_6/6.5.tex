% --------------------------------------------------------------------------------

\begin{exercise}[133]

Zeigen Sie (mit Hilfe des Kompaktheitssatzes, aber ohne Auswahlaxiom, Lemma von Zorn
oder den Hausdorffschen Kettensatz zu verwenden), dass es auf jeder Menge eine
lineare Ordnung gibt. \\
Wenn das zu einfach ist: Zu jeder partiellen Ordnung $(P,\leq)$ gibt es eine
lineare Ordnung $(P,\leq^{\prime})$ mit $\forall x \forall y: (x \leq y \implies x \leq^{\prime} y)$. \\
\textit{Hinweis:} Betrachten Sie eine Sprache, die ein zweistelliges Relationssymbol
enthält, sowie für jedes Element Ihrer Menge eine neue Konstante.

\end{exercise}

% --------------------------------------------------------------------------------

\begin{solution}

\phantom{}

\end{solution}

% --------------------------------------------------------------------------------
