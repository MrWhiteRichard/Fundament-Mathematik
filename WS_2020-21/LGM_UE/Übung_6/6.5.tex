% --------------------------------------------------------------------------------

\begin{exercise}[133]

Zeigen Sie (mit Hilfe des Kompaktheitssatzes, aber ohne Auswahlaxiom, Lemma von Zorn
oder den Hausdorffschen Kettensatz zu verwenden), dass es auf jeder Menge eine
lineare Ordnung gibt. \\
Wenn das zu einfach ist: Zu jeder partiellen Ordnung $(P,\leq)$ gibt es eine
lineare Ordnung $(P,\leq^{\prime})$ mit $\forall x \forall y: (x \leq y \implies x \leq^{\prime} y)$. \\
\textit{Hinweis:} Betrachten Sie eine Sprache, die ein zweistelliges Relationssymbol
enthält, sowie für jedes Element Ihrer Menge eine neue Konstante.

\end{exercise}

% --------------------------------------------------------------------------------

\begin{solution}

Wir bewegen uns in einer Sprache $\mathscr{L}$ mit einem zweistelligen Relationssymbol $\leq$
und einer Konstantenmenge $C = \{c_p: p \in P \}$. Definiere
\begin{align*}
  \Sigma := \{(c \leq c) \land
  (c \leq d \rightarrow d \leq c \rightarrow c = d) \land
  (c \leq d \lor d \leq c) \land
  (c \leq d \rightarrow d \leq e \rightarrow c \leq e): c,d,e \in C\}
\end{align*}
Jede endliche Teilmenge von $\Sigma$ ist erfüllbar. Dafür betrachte die endliche Menge $M$
an Konstantensymbole die in $\Sigma$ vorkommen. Diese Menge können wir nun mit
einer linearen Ordnung versehen, z.B. betrachten wir die nach der Definition der
Endlichkeit existierende Bijektion $f: M \to \{1,\dots,n\}$ für ein $n \in \N$ und
definieren $m_1 \leq m_2:\iff f(m_1) \leq_\N f(m_2)$ für $m_1,m_2 \in M$.
Nach dem Kompaktheitssatz ist dann auch $\Sigma$ erfüllbar.
Also existiert eine Struktur $\mathscr{L}$, welche $\Sigma$ erfüllt. Nun definiere
\begin{align*}
  p \leq q: \iff c_p^{\mathscr{M}} \leq^{\mathscr{M}} c_q^{\mathscr{M}}, \quad p,q \in P.
\end{align*}
Diese Vorschrift ist wohldefiniert, da wir unsere Konstantensymbolmenge so
gewählt haben, dass es zu jedem $p \in P$ genau ein wohlunterschiedenes Konstantensymbol $c_p$ gibt.
Nun gilt es noch nachzuprüfen, dass dadurch tatsächlich eine lineare Ordnung definiert wird:
Reflexivität, Transivität und Totalität übertragen sich direkt aufgrund den Gesetzen in $\Sigma$.
Die Antisymmetrie folgt ebenso aus den Gesetzen, sowie der Voraussetzung, dass $c_p, c_q$
für $p \neq q$ unterschiedliche Konstantensymbole bezeichnen.
\end{solution}

% --------------------------------------------------------------------------------
