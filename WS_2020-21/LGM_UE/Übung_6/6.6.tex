% --------------------------------------------------------------------------------

\begin{exercise}[134]

Wir betrachten eine Theorie $\Sigma$ in der Sprache mit einem zweistelligen Relationssymbol $E$.
Die Theorie verlangt, dass $E$ (genauer: $E^{\mathscr{M}}$ für jedes $\mathscr{M}$,
welches $\Sigma$ erfüllt) eine Äquivalenzrelation ist, in der jede Klasse unendlich
viele Elemente enthält. Weiters definieren wir
$\Sigma_2 := \Sigma \cup \{\forall x \forall y \forall z (xEy \lor yEz \lor xEz)\}$.
\begin{enumerate}[label = \alph*.]
  \item Geben Sie die Formeln in $\Sigma$ an.
  \item Wie viele abzählbar unendliche Modelle hat $\Sigma_2$ bis auf Isomorphie?
  Geben Sie alle an.
  \item Geben Sie mindestens 3 (nichtisomorphe) Modelle von $\Sigma_2$ an, deren
  Universum $\R$ ist.
  \item Finden Sie alle vollständigen konsistenten Theorien $\Sigma^{\prime} \supseteq \Sigma_2$
  (wobei wir zwei Theorien als gleich betrachten, wenn sie dieselben Konsequenzen haben). \\
  \textit{Hinweis:} Verwenden Sie den Vollständigkeitssatz sowie Ihr Resultat aus Punkt b.
\end{enumerate}

\end{exercise}

% --------------------------------------------------------------------------------

\begin{solution}

\phantom{}
\begin{enumerate}[label = \alph*.]
\item Definiere dafür die Formel
\begin{align*}
  &\exists^{\geq n} = \forall x \exists x_1 \cdots \exists x_n\, (x_1 \neq x_2 \land x_1 \neq x_3
  \land \cdots \land x_{n-1} \neq x_n \land xEx_1 \land \cdots \land xEx_n) \\
  &\Sigma = \{\forall x\, (xEx), \forall x \forall y\, (xEy \rightarrow yEx),
  \forall x \forall y \forall z\, (xEy \rightarrow yEz \rightarrow xEz)\} \cup
  \{\exists^{\geq n}: n \in \N\}
\end{align*}
\item Wir behaupten dass es bis auf Isomorphie nur zwei unterschiedliche Modelle von $\Sigma_2$ gibt.
Zuerst bemerken wir dafür, dass die Gesetze in $\Sigma_2$ bereits erzwingen, dass es in allen Modellen $\mathscr{M}$ von $\Sigma_2$ maximal zwei Äquivalenzklassen
geben kann. Gäbe es nämlich mindestens drei, so könnten wir $x,y,z \in M$ aus jeweils unterschiedlichen
Äquivalenzklassen wählen, welche dann die Formel $\forall x \forall y \forall z (xEy \lor yEz \lor xEz)$
verletzen würden. \\
Jetzt gilt es also noch zu zeigen, dass alle Modelle von $\Sigma_2$ mit der gleichen
Anzahl an Äquivalenzklassen paarweise isomorph sind. Für Modelle mit einer einzigen
Äquivalenzklasse ist die Interpretation von $E$ bereits zwangsläufig die Allrelation
und somit stellt jede Bijektion zwischen den Universen bereits eine mit den Interpretationen
von $E$ verträgliche Abbildung, also eine Isomorphie dar. So eine Bijektion exisitiert
immer, da wir hier nur von abzählbaren Modellen sprechen. \\
Für Modelle $\mathscr{M}_1,\mathscr{M}_2$ mit jeweils zwei Äquivalenzklassen müssen
wir also eine Bijektion finden, welche mit der von den Interpretationen von $E$
induzierten Partitionen $M_1 = P_1 \cup Q_1, M_2 = P_2 \cup Q_2$ verträglich ist,
also $f(P_1) = P_2, f(Q_1) = Q_2$ beispielsweise. Dies ist immer möglich, da die Modelle
$\mathscr{P_1}: P_1, E|_{P_1 \times P_1}$, usw. allesamt Modelle von $\Sigma_2$
mit nur einer einzigen Äquivalenzklasse sind, wo denen wir bereits wissen,
dass sie paarweise isomoprh sind. Wir erhalten also Isomorphien $f_p: P_1 \to P_2, p_q: Q_1 \to Q_2$.
Schließlich ist $f = f_p \cup f_q$ die gesuchte Bijektion zwischen $\mathscr{M}_1$
und $\mathscr{M}_2$.
\item Wir geben drei Partitionen auf $\R$ an:
\begin{enumerate}[label = \arabic*)]
  \item $\{\R\}$
  \item $\{\R\backslash\Q,\Q\}$
  \item $\{\R^+\cup\{0\},\R^-\}$
\end{enumerate}
Klarerweise ist die induzierte Äquivalenzrelation von 1) nichtisomorph zu 2) und 3).
Die Äquivalenzrelationen von 2) und 3) sind nicht isomorph, weil es keine Bijektion von
$\Q$ nach $\R^+\cup\{0\}$ oder $\R^-$ gibt.
\item Wir wissen, dass wir alle konsistenten Theorien vervollständigen können,
also existiert zumindest eine vollständige Theorie $\Sigma^{\prime} \subseteq \Sigma_2$.
In einer vollständigen Theorie wird auf jeden Fall die Formel
$\sigma^{\prime} = \forall x \forall y (xEy)$ entschieden. Damit wird auf jeden Fall die Anzahl
der Äquivalenzklassen festgelegt. Wir können zu $\Sigma_2$ sowohl $\sigma^{\prime}$,
als auch $\neg \sigma^{\prime}$ hinzufügen und bleiben konsistent.
Damit gibt es genau zwei mögliche Vervollständigungen von $\Sigma_2$:
Eine lässt nur eine unendliche Äquivalenzklasse zu, die andere erlaubt nur Modelle mit
zwei unendlichen Äquivalenzklassen. \\
Was ist mit abzählbar / nicht abzählbar unendlichen Äquivalenzklassen?
Kommen da noch mehr vollständige Theorien dazu?
Was heißt ``dieselben Konsequenzen''  genau?
\end{enumerate}

\end{solution}

% --------------------------------------------------------------------------------
