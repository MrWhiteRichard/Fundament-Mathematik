\documentclass{article}

% Hier befinden sich Pakete, die wir beinahe immer benutzen ...

\usepackage[utf8]{inputenc}

% Sprach-Paket:
\usepackage[ngerman]{babel}

% damit's nicht so, wie beim Grill aussieht:
\usepackage{fullpage}

% Mathematik:
\usepackage{amsmath, amssymb, amsfonts, amsthm}
\usepackage{bbm, mathrsfs, stmaryrd}
\usepackage{mathtools, mathdots}

% Makros mit mehereren Default-Argumenten:
\usepackage{twoopt}

% Anführungszeichen (Makro \Quote{}):
\usepackage{babel}

% if's für Makros:
\usepackage{xifthen}
\usepackage{etoolbox}

% tikz ist kein Zeichenprogramm (doch!):
\usepackage{tikz}

% bessere Aufzählungen:
\usepackage{enumitem}

% (bessere) Umgebung für Bilder:
\usepackage{graphicx, subfig, float}

% Umgebung für Code:
\usepackage{listings}

% Farben:
\usepackage{xcolor}

% Umgebung für "plain text":
\usepackage{verbatim}

% Umgebung für mehrerer Spalten:
\usepackage{multicol}

% "nette" Brüche
\usepackage{nicefrac}

% Spaltentypen verschiedener Dicke
\usepackage{tabularx}
\usepackage{makecell}

% Für Vektoren
\usepackage{esvect}

% (Web-)Links
\usepackage{hyperref}

% Zitieren & Literatur-Verzeichnis
\usepackage[style = authoryear]{biblatex}
\usepackage{csquotes}

% so ähnlich wie mathbb
%\usepackage{mathds}

% Keine Ahnung, was das macht ...
\usepackage{booktabs}
\usepackage{ngerman}
\usepackage{placeins}

% special letters:

\newcommand{\N}{\mathbb{N}}
\newcommand{\Z}{\mathbb{Z}}
\newcommand{\Q}{\mathbb{Q}}
\newcommand{\R}{\mathbb{R}}
\newcommand{\C}{\mathbb{C}}
\newcommand{\K}{\mathbb{K}}
\newcommand{\T}{\mathbb{T}}
\newcommand{\E}{\mathbb{E}}
\newcommand{\V}{\mathbb{V}}
\renewcommand{\P}{\mathbb{P}}
\newcommand{\1}{\mathbbm{1}}

\newcommand  {\B}{\mathfrak{B}}
\renewcommand{\S}{\mathfrak{S}}

% quantors:

\newcommand{\Forall}{\forall \,}
\newcommand{\Exists}{\exists \,}
\newcommand{\ExistsOnlyOne}{\exists! \,}
\newcommand{\nExists}{\nexists \,}

% MISC symbols:

\newcommand{\landau}[1]
{
  {\scriptstyle \mathcal{O}}
  \pbraces{#1}
}

\newcommand{\Landau}[1]
{
  \mathcal{O}
  \pbraces{#1}
}

\newcommand{\eps}{\mathrm{eps}}

% graphics in a box:

\newcommandtwoopt
{\includegraphicsboxed}[3][][]
{
  \begin{figure}[!h]
    \begin{boxedin}
      \ifthenelse{\isempty{#2}}
      {
        \begin{center}
          \includegraphics[width = 0.75 \textwidth]{#3}
          \label{fig:#1}
        \end{center}
      }{
        \begin{center}
          \includegraphics[width = 0.75 \textwidth]{#3}
          \caption{#2}
          \label{fig:#1}
        \end{center}
      }
    \end{boxedin}
  \end{figure}
}

% braces:

\newcommand{\pbraces}[1]{{\left  ( #1 \right  )}}
\newcommand{\bbraces}[1]{{\left  [ #1 \right  ]}}
\newcommand{\Bbraces}[1]{{\left \{ #1 \right \}}}
\newcommand{\vbraces}[1]{{\left  | #1 \right  |}}
\newcommand{\Vbraces}[1]{{\left \| #1 \right \|}}
\newcommand{\abraces}[1]{{\left \langle #1 \right \rangle}}
\newcommand{\round}[1]{\bbraces{#1}}

\newcommand
{\floor}[1]
{{\left \lfloor #1 \right \rfloor}}

\newcommand
{\ceil} [1]
{{\left \lceil  #1 \right \rceil }}

% special functions:

\newcommand{\norm}  [2][]{\Vbraces{#2}_{#1}}
\newcommand{\diag}  [1]{\mathrm{diag} \: #1}
\newcommand{\dist}  [1]{\mathrm{dist} \: #1}
\newcommand{\mean}  [1]{\mathrm{mean} \: #1}
\newcommand{\erf}   [1]{\mathrm{erf} \: #1}
\newcommand{\id}    [1]{\mathrm{id} \: #1}
\newcommand{\sgn}   [1]{\mathrm{sgn} \: #1}
\newcommand{\supp}  [1]{\mathrm{supp} \: #1}
\newcommand{\arsinh}[1]{\mathrm{arsinh} \: #1}
\newcommand{\arcosh}[1]{\mathrm{arcosh} \: #1}
\newcommand{\artanh}[1]{\mathrm{artanh} \: #1}
\newcommand{\card}  [1]{\mathrm{card} \: #1}
\newcommand{\Span}  [1]{\mathrm{span} \: #1}
\newcommand{\Aut}   [1]{\mathrm{Aut} \: #1}
\newcommand{\End}   [1]{\mathrm{End} \: #1}
\newcommand{\ggT}   [1]{\mathrm{ggT} \: #1}
\newcommand{\kgV}   [1]{\mathrm{kgV} \: #1}
\newcommand{\ord}   [1]{\mathrm{ord} \: #1}
\newcommand{\grad}  [1]{\mathrm{grad} \: #1}
\newcommand{\ran}   [1]{\mathrm{ran} \: #1}
\newcommand{\graph} [1]{\mathrm{graph} \: #1}
\newcommand{\Inv}   [1]{\mathrm{Inv} \: #1}
\newcommand{\pv}    [1]{\mathrm{pv} \: #1}
\newcommand{\Mod}{\: \mathrm{mod} \:}
\newcommand{\Char}{\mathrm{char}}
\newcommand{\At}{\mathrm{At}}
\newcommand{\Ob}{\mathrm{Ob}}
\newcommand{\Hom}{\mathrm{Hom}}
\newcommand{\orthogonal}[3][]{#2 ~\bot_{#1}~ #3}
\newcommand{\Rang}{\mathrm{Rang}}

\newcommand
{\GL}[2][]
{\mathrm{GL}_{#1} \pbraces{#2}}

% fractions:

\newcommand{\Frac}[2]{\frac{1}{#1} \pbraces{#2}}
\newcommand{\nfrac}[2]{\nicefrac{#1}{#2}}

% derivatives & integrals:

\newcommandtwoopt
{\Int}[4][][]
{\int_{#1}^{#2} #3 ~\mathrm{d} #4}

\newcommandtwoopt
{\derivative}[3][][]
{
  \frac
  {\mathrm{d}^{#1} #2}
  {\mathrm{d} #3^{#1}}
}

\newcommandtwoopt
{\pderivative}[3][][]
{
  \frac
  {\partial^{#1} #2}
  {\partial #3^{#1}}
}

\newcommand
{\primeprime}
{{\prime \prime}}

\newcommand
{\primeprimeprime}
{{\prime \prime \prime}}

% Text:

\newcommand{\Quote}[1]{\glqq #1\grqq{}}
\newcommand{\Text}[1]{{\text{#1}}}
\newcommand{\fastueberall}{\text{f.ü.}}
\newcommand{\fastsicher}{\text{f.s.}}

% -------------------------------- %
% amsthm-stuff:

\theoremstyle{definition}

% numbered theorems
\newtheorem{theorem}    {Satz}   [section]
\newtheorem{lemma}      [theorem]{Lemma}
\newtheorem{corollary}  [theorem]{Korollar}
\newtheorem{proposition}[theorem]{Proposition}
\newtheorem{remark}     [theorem]{Bemerkung}
\newtheorem{definition} [theorem]{Definition}
\newtheorem{example}    [theorem]{Beispiel}

% unnumbered theorems
\newtheorem*{theorem*}    {Satz}
\newtheorem*{lemma*}      {Lemma}
\newtheorem*{corollary*}  {Korollar}
\newtheorem*{proposition*}{Proposition}
\newtheorem*{remark*}     {Bemerkung}
\newtheorem*{definition*} {Definition}
\newtheorem*{example*}    {Beispiel}

% Please define this stuff in project ("main.tex"):

% \def \lastexercisenumber {...}
% This will be 0 by default

% \setcounter{section}{...}
% This will be 0 by default
% and hence, completely ignored

\ifnum \thesection = 0
{
  \newtheorem{exercise}{Aufgabe}
}
\else
{
  \newtheorem{exercise}{Aufgabe}[section]
}
\fi

\ifdef
{\lastexercisenumber}
{\setcounter{exercise}{\lastexercisenumber}}

\newenvironment{solution}
{
  \begin{proof}[Lösung]
}{
  \end{proof}
}

\renewcommand{\proofname}{Beweis}

% -------------------------------- %
% environment zum einkasteln:

% dickere vertical lines
\newcolumntype
{x}
[1]
{
  !{
    \centering
    \arraybackslash
    \vrule
    width #1}
}

% environment selbst (the big cheese)
\newenvironment
{boxedin}
{
  \begin{tabular}
  {
    x{1 pt}
    p{\textwidth}
    x{1 pt}
  }
  \Xhline
  {2 \arrayrulewidth}
}
{
  \\
  \Xhline{2 \arrayrulewidth}
  \end{tabular}
}

% -------------------------------- %
% MISC "Ein-Deutschungen"

\renewcommand{\figurename}{Abbildung}
\renewcommand{\tablename} {Tabelle}

% -------------------------------- %

\input{../../../Fundament-LaTeX/listings.tex}

\parskip 0pt
\parindent 0pt

\title
{
  Logik und Grundlagen der Mathematik \\
  \vspace{4pt}
  \normalsize
  \textit{6. Übung am 12.11.2020}
}
\author
{
  Richard Weiss
  \and
  Florian Schager
  \and
  Fabian Zehetgruber
}
\date{}

\begin{document}

\maketitle
\section*{Vollständige Theorien}

\begin{exercise}
  Gegeben sei die skalare ODE

  \begin{align*}
    x'' + q(t)x = 0
  \end{align*}

  mit einer stetigen Funktion $q: \R \rightarrow \R$

  Es seien $t \mapsto x(t)$ und $t \mapsto y(t)$ zwei Lösunger der ODE.
  Ihre Wronskideterminante ist durch

  \begin{align*}
    W(t) := x(t)y'(t)-x'(t)y(t)
  \end{align*}

  definiert. Die Lösungen $x(t)$ und $y(t)$ sind l.u. wenn
  $W(t) \neq 0, \forall t \in \R$ gilt.

  \begin{itemize}
    \item[a)] Zeigen sie, dass $W(t)$ konstant ist.
    \item[b)] Zeigen sie unter Verwendung von a), dass für l.u. Lösungen
    $x(t)$ und $y(t)$ gilt:
    \begin{itemize}
      \item[(i)] Aus $x(t_1) = 0$ folgt $x'(t_1) \neq 0$ und $y(t_1) \neq 0$
      \item[(ii)] Falls $x(t_1)=x(t_2)=0$ gilt und $x(t) \neq 0$ für
      $t \in (t_1 , t_2 )$ dann hat $y(t)$ in $(t_1 , t_2 )$ genau eine Nullstelle.
    \end{itemize}
  \end{itemize}

\end{exercise}

\begin{solution}
Beweisen wir diese Aussagen also.
\begin{itemize}
  \item[a)] Um das zu zeigen, berechnen wir einfach die Ableitung:
  \begin{align*}
    W'(t)= x'(t)y'(t) + x(t)y''(t) - x''(t)y(t) - x'(t)y'(t) =
    x'(t)y'(t) - x(t)q(t)y(t) + x(t)q(t)y(t) - x'(t)y'(t) = 0
  \end{align*}
  also ist die Wronskideterminante wirklich konstant.

  \item[b)] Ad (i): Da $x$ und $y$ unabhängige Lösungen sind wissen wir durch ihre
  Wronskideterminante:

  \begin{align*}
    x(t)y'(t)-x'(t)y(t) = c && c \in \R \backslash \{0\}
  \end{align*}

  Gilt also an einem Punkt $t_1: x(t_1 ) = 0$ sehen wir, dass dort

  \begin{align*}
    -x'(t_1 )y(t_1 ) = c
  \end{align*}

  Da $c \neq 0$ können wir daraus die Behauptung sofort schließen.

  Ad (ii): Nach (i) wissen wir: $y(t_1 ) \neq 0, y(t_2 ) \neq 0$. Wir können auch

  \begin{align} \label{Vorzeichen}
    -x'(t_1 )y(t_1 ) = c = -x'(t_2 )y(t_2 )
    \Rightarrow x'(t_1 )y(t_1 ) = x'(t_2 )y(t_2 )
  \end{align}

  schließen. Nun zeigen wir noch $\sgn(x'(t_1 )) = - \sgn(x'(t_2 ))$. Dazu bemerken wir
  zuerst, dass $x'(t_ i ) \neq 0, i=1,2$ da sonst $W(t_i) = 0$ im Widerspruch zu
  unserer Voraussetzung. Wenn nun $\sgn(x'(t_1 )) = 1$ gilt heißt das auch
  $x(t)>0 \, \forall t \in (t_1, t_2)$ damit muss klarerweise $\sgn(x'(t_2 ))=-1$.
  (Formaler: man schaut sich den einseitigen Grenzwert im Differenzialquotienten an.)

  Gemeinsam mit \eqref{Vorzeichen} sehen wir, dass $\sgn(y(t_1 )) = -\sgn(y(t_2 ))$.
  Dazu muss $y$ also einen Vorzeichenwechsel in $(t_1 , t_2 )$ und folglich auch
  zumindest eine Nullstelle haben.

  Um die Eindeutigkeit jener zeigen wir mit einem Widerspruchsbeweis: Sei $\tilde{t}$
  die kleinste Nullstelle von $y$ und $\hat{t}$ die nächste weitere Nullstelle in
  $(t_1 , t_2 )$. Es gilt also:

  \begin{align*}
    x( \tilde{t} ) y'( \tilde{t} ) = x( \hat{t} ) y'( \hat{t} )
  \end{align*}

  Da nun aber $x$ keine Vorzeichenwechsel in dem Intervall hat folgt daraus
  $\sgn(y(\tilde{t})) = \sgn(y(\hat{t}))$. Wir haben aber auch
  $\forall t \in (\tilde{t},\hat{t}): y(t) \neq 0 $ vorausgesetzt. Mit den selben Überlegungen
  wie bei $x$ weiter oben ist dies jedoch ein Widerspruch.
\end{itemize}
\end{solution}

\begin{exercise}
Bestimmen Sie die allgemeine Lösung der folgenden ODEs mit der
Ansatzmethode:
\begin{enumerate}[label = \textbf{\alph*)}]
\item \begin{align*}
  y^{\primeprime} + y &= \sin(t) + \sin(3t),
\end{align*}
\item \begin{align*}
  y^{\primeprime} + y = t\exp(-2t)\cos(t),
\end{align*}
\item \begin{align*}
  y^{\primeprime} - y = t\exp(-t).
\end{align*}
\end{enumerate}
Untersuchen Sie, ob die Lösungen dieser ODEs stabil, beziehungsweise
asymptotisch stabil sind.
\end{exercise}
\begin{solution}
\leavevmode \\
\begin{enumerate}[label = \textbf{\alph*)}]
  \item Das charakteristische Polynom des homogenen Systems lautet
  \begin{align*}
    \chi(\lambda) = \lambda^2 + 1
  \end{align*}
  mit den komplexen Nullstellen
  \begin{align*}
    \lambda_1 = i, \qquad \lambda_2 = -i.
  \end{align*}
  Also erhalten wir mit Satz 2.18 mit
  \begin{align*}
    y_1(t) = \exp(it), \qquad y_2(t) = \exp(-it)
  \end{align*}
  ein Fundamentalsystem für die homogene Gleichung.
  Eine beliebige Lösung $\widetilde{y}$ des homogenen Systems hat daher die Form
  \begin{align*}
    \widetilde{y}(t) = a_1\exp(it) + a_2\exp(-it)
  \end{align*}
  Eine allgemeine Lösung des inhomogenen Systems, lässt sich durch
  \begin{align*}
    y(t) = \widetilde{y}(t) + y_p(t)
  \end{align*}
  darstellen, wobei $y_p$ eine Partikulärlösung des inhomogenen System ist.
  Um diese nun zu berechnen, schreiben wir zuerst die Differentialgleichung folgendermaßen um:
  \begin{align*}
    y^{\primeprime} + y &= \sin(t) + \sin(3t)
    = \frac{-i}{2}(\exp(it) + \exp(3it) - \exp(-3it) - \exp(-it)).
  \end{align*}
  Jetzt können wir für $b_{1,2} = \exp(\pm it), b_{3,4} = \exp(\pm 3it)$
  jeweils seperat Teil-Partikulärlösungen berechnen und diese anschließend zur finalen Partikulärlösung
  aufsummieren. Einen Ansatz dafür liefert uns Satz 3.20. Da $\chi(\pm i) = 0$ sind wir im 2.Fall und
  setzen
  \begin{align*}
    y_{p_1}(t) = ct\exp(it)
  \end{align*}
  an. Einsetzen in die Differentialgleichung liefert
  \begin{align*}
    &y_{p_1}^{\primeprime}(t) + y_{p_1}(t) = c(2i - t + t)\exp(it)\stackrel{!}{=}
    \frac{-i}{2}\exp(it) \\
    &\iff c = \frac{-1}{4}.
  \end{align*}
  Damit erhalten wir die erste Teil-Partikulärlösung $y_{p_1}(t) = \frac{-t}{4}\exp(it)$. \\
  Analog berechnen wir $y_{p_2}(t) = \frac{-t}{4}\exp(-it)$. \\
  Da $\chi(\pm 3i) \neq 0$, sind wir diesmal im 1.Fall des Satzes.
  Unser Ansatz lautet daher
  \begin{align*}
    y_{p_3}(t) = c\exp(3it).
  \end{align*}
  Einsetzen in die Differentialgleichung liefert
  \begin{align*}
    &c(1 - i)\exp(3it) = \frac{-i}{2}\exp(3it) \\
    &\iff c = \frac{-i}{2(1-i)} = \frac{-i(1+i)}{2(1-i)(1+i)} = \frac{1-i}{4}
  \end{align*}
  Die dritte Teil-Partikulärlösung lautet daher $y_{p_3}(t) = \frac{1-i}{4}\exp(3it)$.
  Wiederum analog berechnet man die letzte verbleibende Teil-Partikulärlösung
  und erhält $y_{p_4}(t) = \frac{i - 1}{4}\exp(-3it)$.
  Insgesamt haben wir also mit
  \begin{align*}
    y_p(t) = \frac{1}{4}\left(-t\exp(it) - t\exp(-it) + (1-i)\exp(3it) + (i - 1)\exp(-3it)\right)
  \end{align*}
  eine Gesamt-Partikulärlösung gefunden und jubilieren.
  Die allgemeine Form der Lösung lautet daher
  \begin{align*}
    y = y_p + \widetilde{y} = \frac{1}{4}\left[-t\exp(it) - t\exp(-it) + (1-i)\exp(3it) +
    (i - 1)\exp(-3it)\right] + a_1\exp(it) + a_2\exp(-it)
  \end{align*}
  \item Das charakteristische Polynom lautet gleich zu vorhin
  \begin{align*}
    \chi(\lambda) = \lambda^2 + 1
  \end{align*}
  mit den selben Nullstellen
  \begin{align*}
    \lambda_1 = i, \qquad \lambda_2 = -i.
  \end{align*}
  Wieder formen wir die Differentialgleichung um
  \begin{align*}
    y^{\primeprime} + y = t\exp(-2t)\cos(t) = \frac{t}{2}\exp(-2t)(\exp(it) + \exp(-it))
    = \frac{t}{2}(\exp((i-2)t) + \exp(-(i+2)t))
  \end{align*}
  und wieder teilen wir die Partikulärlösungen auf. Für $b_{1} = \frac{t}{2}\exp((i-2)t)$
  gilt $\chi(i - 2) \neq 0$ und somit sind wir im Fall 1. Der Ansatz lautet
  \begin{align*}
    y_{p_1}(t) = (c_1t + c_0)\exp((i-2)t)
  \end{align*}
  Was passiert jetzt bloß? Wir setzen in die Differentialgleichung ein...
  \begin{align*}
    &y_{p_1}^{\primeprime}(t) + y_{p_1}(t) = [c_1(2i-4 + (i - 2)^2t + t) + c_0((i-2)^2 + 1)]\exp((i-2)t)
    \stackrel{!}{=} \frac{t}{2}(\exp((i-2)t) \\
    &\iff (4c_1(1-i))t + c_1(2i-4) + 4c_0(1 - i) = \frac{t}{2} \\
  \end{align*}
  Wir machen nun den Ansatz im Ansatz
  \begin{align*}
    &4c_1(1-i)t = \frac{t}{2} \\
    &\iff c_1 = \frac{1}{8(1-i)},
  \end{align*}
  setzen zurück ein
  \begin{align*}
    &\frac{2i-4}{8(1-i)} + 4c_0(1-i) = 0 \\
    &\iff c_0 = - \frac{i-2}{(4(1-i))^2} = - \frac{(i-2)}{16(1-2i-1)} =
    = \frac{(i-2)}{32i} = \frac{1 + 2i}{32}
  \end{align*}
  und erhalten ansatzweise eine Lösung.
  Wie könnte es jetzt bloß weitergehen? \\
  Für $b_2 = \frac{t}{2}(\exp(-(i+2)t)$ landen wir dank $\chi(-(i+2)) \neq 0$
  wieder im zweiten Fall. Ob man dafür jetzt dankbar ist, ist eine andere Frage.
  \begin{align*}
    y_{p_2}(t) &= (c_1t + c_0)\exp(-(i+2)t) \\
    y_{p_2}^{\primeprime}(t) + y_{p_2}(t) &=
    [c_1(-(2i+4) + (i + 2)^2t + t) + c_0((i + 2)^2 + 1)]\exp(-(i+2)t)
    \stackrel{!}{=} \frac{t}{2}(\exp(-(i+2)t) \\\
    &\iff [c_1(-(4 + 2i) + (i + 2)^2t + t) + c_0((i + 2)^2 + 1)] = \frac{t}{2}
  \end{align*}
  Überraschenderweise wählen wir den Ansatz
  \begin{align*}
    &c_1((i + 2)^2+1)t = \frac{t}{2} \\
    &\iff c_1 = \frac{1}{2(4 + 4i)} = \frac{3 - 4i}{64}
  \end{align*}
  und erhalten
  \begin{align*}
    &c_0((i + 2)^2 + 1) = \frac{(3 - 4i)(4+2i)}{64} \\
    &\iff c_0 = \frac{(3 - 4i)(4+2i)}{64(4 + 4i)} = \frac{5 - 15i}{256}
  \end{align*}
  wenig überraschend endlich die finale Partikulärlösung
  \begin{align*}
    y_p(t) = \left(\frac{1}{8(1-i)}t + \frac{1 + 2i}{32}\right)\exp((i-2)t) +
    \left(\frac{3 - 4i}{64}t + \frac{5 - 15i}{256}\right)\exp(-(i+2)t),
  \end{align*}
  womit wir leicht die allgemeine Lösung
  \begin{align*}
  y(t) = \left(\frac{1}{8(1-i)}t + \frac{1 + 2i}{32}\right)\exp((i-2)t) +
  \left(\frac{3 - 4i}{64}t + \frac{5 - 15i}{256}\right)\exp(-(i+2)t) +
  a_1\exp(it) + a_2\exp(-it)
  \end{align*}
  angeben können.
  \item Analog zu b)
\end{enumerate}
So viel analog, um Glück schreib ich zum Ausgleich das Ganze digital.
Lol, krasse Analogie.
\end{solution}


\section*{Kompaktheitssatz}
Der Kompaktheitssatz der Prädikatenlogik besagt: Wenn $\Sigma \vDash \varphi$,
dann gibt es eine endliche Menge $\Sigma^{\prime} \subseteq \Sigma$ mit $\Sigma^{\prime} \vDash \varphi$. \\
Sei $n \geq 2$. Mit $\exists^n$ (oder genauer $\exists^{\geq n}$) kürzen wir die Formel
$\forall x_2\cdots\forall x_n \exists y\, (y \neq x_2 \land \cdots \land y \neq x_n)$ ab. \\
(Oder: Die dazu beweisbar äquivalente Formel $\exists x_1 \cdots \exists x_n\, (x_1 \neq x_2 \land x_1 \neq x_3
\land \cdots \land x_{n-1} \neq x_n)$.)

% --------------------------------------------------------------------------------

\begin{exercise}[\textbf{Point estimator statistics: Comparison}]

Let $X_1,\dots,X_n$ be i.i.d. uniform $(0,\theta)$, with unknown parameter $\theta > 0$.

\begin{enumerate}[label = (\alph*)]
  \item Show that the method of moments estimator of $\theta$ is $2\bar{X}$ and the MLE
  of $\theta$ is $X_{(n)} = \max_{1 \leq i \leq n} X_i$.
  \item Compare the mean square errors of the two estimators. Which of the estimators
  should be preferred if any? Explain your reasoning.
\end{enumerate}

\end{exercise}

% --------------------------------------------------------------------------------

\begin{solution}

\phantom{}

\begin{enumerate}[label = (\alph*)]
  \item For the method of moments we solve
  \begin{align}
    \bar{X} = \mu(\theta) = \int_0^\theta \frac{x}{\theta} dx = \frac{\theta}{2}
    \iff \theta = w\bar{X}.
  \end{align}
  For the MLE we consider the likelihood function
  \begin{align*}
    L_n(\theta) = \prod_{i=1}^n \1_{[0, \theta]}(x_i)
    = \begin{cases}
      0, & \theta < \max_{i=1}^n x_i \\
      \frac{1}{\theta^n}, & \theta \geq \max_{i=1}^n x_i.
    \end{cases}
  \end{align*}

  Clearly, the maximum is attained at $\hat{\theta} = \max_{i=1}^n x_i.$

  \item

  \begin{align*}
    \E_\theta[(\hat{\theta}_1 - \theta)^2] = \E_\theta[(2\bar{X} - \theta)^2]
    = \sqrt{2}\E_\theta[(\bar{X} - \mu)^2] = \sqrt{2}\V(\bar{X}) = \sqrt{2}\frac{\theta^2}{12n}
    = \theta^2 \Landau(n^{-1})
  \end{align*}

  For the second estimator we first calculate the corresponding pdf for $x \leq \theta^2$

  \begin{align*}
    \P((\max_{1 \leq i \leq n} X_i - \theta)^2 \leq x) &=
    \P(\theta - \sqrt{x} \leq \max_{1 \leq i \leq n} X_i \leq \theta + \sqrt{x}) \\
    &= 1 - \P(\max_{1\leq i \leq n} X_i \leq \theta - \sqrt{x}) \\
    &= 1 - \prod_{i=1}^n \frac{\theta - \sqrt{x}}{\theta} \\
    &= 1 - \left(\frac{\theta - \sqrt{x}}{\theta}\right)^n
  \end{align*}

  Therefore
  \begin{align*}
    \E_\theta[(\hat{\theta}_2 - \theta)^2] &= \E_\theta[(\max_{1 \leq i \leq n} X_i - \theta)^2] \\
    &= \int_0^{\theta^2} x\frac{n-1}{2\theta\sqrt{x}}\left(\frac{\theta - \sqrt{x}}{\theta}\right)^{n-1} dx \\
    &= \frac{n-1}{\theta^n}\int_0^{\theta} y^2(\theta - y)^{n-1} dy \\
    &= -\frac{n-1}{\theta^n}\int_\theta^{0} (\theta - u)^2u^{n-1} du \\
    &= -\frac{n-1}{\theta^n}\int_\theta^{0} (\theta^2 - 2\theta u + u^2)u^{n-1} du \\
    &= \frac{n-1}{\theta^n}\left(\frac{\theta^{n+2}}{n} -
    \frac{2\theta^{n+2}}{n+1} + \frac{\theta^{n+2}}{n+2}\right) \\
    &= \theta^2(n-1)\left(\frac{(n+1)(n+2) - 2n(n+2) + n(n+1)}{n(n+1)(n+2)}\right) \\
    &= \theta^2\left(\frac{2(n-1)}{n(n+1)(n+2)}\right) = \theta^2 \Landau(n^{-2}).
  \end{align*}

  Asymptotically for $n \to \infty$ the second estimator should be preferred, since
  it converges faster.
\end{enumerate}

\end{solution}

% --------------------------------------------------------------------------------

% --------------------------------------------------------------------------------

\begin{exercise}[130]

Sei $\mathscr{M}$ eine unendliche abzählbare Struktur. Zeigen Sie, dass es in
$\Mod(\Th(\mathscr{M}))$ eine überabzählbare Struktur gibt. \\
\textit{Hinweis:} Vollständigkeitssatz und Kompaktheitssatz gelten auch für
überabzählbare Sprachen. Verwenden Sie überabzählbar viele Konstantensymbole.

\end{exercise}

% --------------------------------------------------------------------------------

\begin{solution}
Mit $\Mod(\Sigma)$ bezeichnen wir die Klasse aller Strukturen $\mathscr{M}$
(einer vorgegebenen Sprache), die $\mathscr{M} \vDash \Sigma$ erfüllen.
Sei $\mathfrak{M}$ eine Klasse von Strukturen. Mit $\Th(\mathfrak{M})$ bezeichnen
wir die Theorie von $\mathfrak{M}$, also die Menge aller geschlossenen Formeln $\varphi$,
die in jedem $\mathscr{M} \in \mathfrak{M}$ erfüllt sind. \\
In unserem Fall gilt $\Th(\mathscr{M}) = \{\sigma: \mathscr{M} \vDash \sigma\}$.
Bezeichne $\mathscr{L}$ die Sprache von $\mathscr{M}$ und sei $C$ eine überabzählbare
Menge von neuen Konstantensymbolen. Definiere $\mathscr{L}^{\prime} := \mathscr{L} + C$, sowie
$\Sigma := \Th(\mathscr{M})\, \cup\, \{c \neq d: c,d \in C, c \neq d \}$.
Jede endliche Teilmenge davon ist erfüllbar, daher muss mit der Kontraposition
des Kompaktheitssatzes auch $\Sigma$ erfüllbar sein. Also existiert ein $\mathscr{M}^{\prime}$
mit $\mathscr{M}^{\prime} \vDash \Th(\mathscr{M})\, \cup\, \{c \neq d: c,d \in C, c \neq d \}$,
insbesondere ist $\mathscr{M}^{\prime} \in \Mod(\Th(\mathscr{M}))$.
Die hinzugefügten Formeln garantieren, dass verschiedene Konstantensymbole der
neuen überabzählbaren Konstantenmenge $C$ auch unterschiedlich interpretiert werden.
Damit muss $\mathscr{M}^{\prime}$ ebenso eine überabzählbare Struktur sein.

\end{solution}

% --------------------------------------------------------------------------------

% --------------------------------------------------------------------------------

\begin{exercise}[Rayleigh distribution]

Let $X_1, \dots, X_n$ be a random sample with Rayleigh distribution

\begin{align*}
    f(x \mid \theta)
    =
    \begin{cases}
        \frac{x}{\theta^2} \exp*{-\frac{x^2}{2 \theta^2}},
        & x \geq 0 \\
        0,
        & x < 0
    \end{cases},
\end{align*}

where $\theta > 0$ is unknown.

\begin{enumerate}[label = (\alph*)]
    \item Find the method of moments estimator of $\theta$.
    \item Find teh MLE of $\theta$ and its asymptotic variance.
\end{enumerate}

\textit{Hint}:
Show that the first two moments are $\E X = \theta \sqrt \frac{\pi}{2}$ and $\E X^2 = 2 \theta^2$.

\end{exercise}

% --------------------------------------------------------------------------------

\begin{solution}

ToDo!

\end{solution}

% --------------------------------------------------------------------------------

% -------------------------------------------------------------------------------- %

\begin{exercise}

Zeigen Sie:
Ist $w$ harmonisch auf einer offenen Menge $\Omega \subseteq \R^n$, $r > 0$, $x \in \Omega$ sodass $\overline{B_r(x)} \subseteq \Omega$, dann gilt für alle $i = 1, \dots, n:$

\begin{align*}
    |\partial_i w|
    \leq
    \frac{n}{r}
    \norm[L^\infty(B_r(x))]{w}.
\end{align*}

Zeigen Sie weiters, dass für jeden Multiindex $\alpha$ mit $|\alpha| = k$ gilt:

\begin{align*}
    |\partial^\alpha w(x)|
    \leq
    \pbraces
    {
        \frac{kn}{r}
    }^k
    \norm[L^\infty(B_r(x))]{w}.
\end{align*}

\end{exercise}

% -------------------------------------------------------------------------------- %

\begin{solution}

Mit dem Satz von Schwarz erhalten wir, dass auch $\partial_i w$ harmonisch ist.
Somit folgt mit der Mittelwerteigenschaft, sowie dem Satz vor dem Satz von Gauß, dass

\begin{align*}
  |\partial_iw(x)|
  & =
  \left|\frac{n}{S_nr^n}\int_{B_{r}(x)}\partial_iw(y) dy\right| \\
  & =
  \left|\frac{n}{S_nr^n}\int_{\partial B_{r}(x)}w(y)\nu_i(y) dS(y)\right| \\
  & =
  \left|\frac{n}{S_nr^n}\int_{\partial B_{r}(x)}w(y)\frac{y_i}{|y|} dS(y)\right| \\
  & \leq
  \frac{n}{S_nr^n}S_nr^{n-1}\|w\|_{L^\infty(\partial B_{r}(x))} \\
  & =
  \frac{n}{r}\|w\|_{L^\infty(\partial B_{r}(x))} \leq \frac{n}{r}\|w\|_{L^\infty(B_{r}(x))}.
\end{align*}

Sei $\alpha$ nun ein beliebiger Multiindex der Ordnung $k$.
Wähle $i$ und $\beta$, sodass $\partial^\alpha w = \partial_i(\partial^\beta w)$.
($\beta$ hat Ordnung $k-1$.)
Da $\partial^\alpha w$ wieder harmonisch ist, können wir die obere Abschätzung anwenden.

\begin{align*}
  \implies
  |\partial^\alpha w(x)|
  =
  |\partial_i (\partial^\beta w)(x)|
  \leq
  \frac{kn}{r}
  \norm
  [
    L^\infty( B_{r/k}(x))
  ]
  {\partial^\beta w}
\end{align*}

Jetzt gilt es die Prozedur zu iterieren.
Für $y \in B_{r/k}(x)$ gilt $B_{r/k}(y) \subset B_{2r/k}(x) \subset \Omega$.
Analog zu obiger Rechnunge erhalten wir also Folgendes.
($\gamma$ sei dabei ein Multiindex mit Ordnung $k-2$.)

\begin{align*}
  \implies
  |\partial^\beta w(y)|
  =
  \cdots
  \leq
  \frac{kn}{r}
  \norm
  [
    L^\infty(B_{r/k}(y))
  ]
  {\partial^\gamma w}
  \leq
  \frac{kn}{r}
  \norm
  [
    L^\infty(B_{2r/k}(x))
  ]
  {\partial^\gamma w}
\end{align*}

Weil $y$ beliebig war, gilt die Abschätzung auch mit linker Seite $\norm[L^\infty( B_{r/k}(x))]{\partial^\beta w}$.
Nach der $k$-ten Iteration erhalten wir damit die Behauptung:

\begin{align*}
  |\partial^\alpha w(x)|
  \leq
  \pbraces
  {
      \frac{kn}{r}
  }^k
  \norm[L^\infty(B_{kr/k}(x))]{w}
\end{align*}

\end{solution}

% -------------------------------------------------------------------------------- %


\section*{Henkintheorien}

% --------------------------------------------------------------------------------

\begin{exercise}[138]

Geben Sie explizit eine konsistente Henkintheorie an, deren Sprache nur ein
einziges Konstantensymbol (möglicherweise aber weitere Prädikaten- und/oder
Funktionssymbole) enthält. \\
Wenn das zu leicht ist: Geben Sie zwei derartige Theorien an, wobei die eine
vollständig und die andere unvollständig sein soll.

\end{exercise}

% --------------------------------------------------------------------------------

\begin{solution}

\phantom{}

\end{solution}

% --------------------------------------------------------------------------------


\section*{Außer Konkurrenz}

% -------------------------------------------------------------------------------- %

\begin{exercise}[132]

Sei $M$ eine nichtleere Menge, $\mathscr{L}$ eine prädikatenlogische Sprache, und
sei $X$ die Menge aller $\mathscr{L}$-Strukturen mit Universum $M$. Zeigen Sie:
\begin{enumerate}[label = \alph*.]
  \item Die Familie $\{\Mod(\sigma): \sigma \text{ geschlossene Formel }\}$
  bildet Basis einer $0$-dimensionalen Topologie $\tau$ auf $X$.
  Eine Topologie heißt $0$-dimensional, wenn es eine Basis gibt, die unter
  Komplementsen abgeschlossen ist, oder äquivalent: Wenn es eine Basis gibt,
  deren Mengen alle clopen, also offen und abgeschlossen, sind. Das klassische
  Beispiel eines $0$-dimensionalen, aber nicht diskreten Raums ist die Cantormenge.
  \item Im allgemeinen ist $(X, \tau)$ kein Hausdorffraum. Geben Sie eine sinnvolle
  Äquivalenzrelation $\sim$ auf $X$ und eine sinnvolle Topologie $\tau_{\sim}$
  auf $X/\sim$ an, sodass $X/\sim$ Hausdorffraum ist.
  \item $(X,\tau)$ ist kompakt, ebenso $(X/\sim,\tau_\sim)$, also jede offene
  Überdeckung hat eine endliche Teilüberdeckung.
\end{enumerate}

\end{exercise}

% -------------------------------------------------------------------------------- %

\begin{solution}

\phantom{}

\end{solution}

% -------------------------------------------------------------------------------- %

% --------------------------------------------------------------------------------

\begin{exercise}[137]

Sei $\mathcal{L}$ eine prädikatenlogische Sprache, und sei $\Sigma$ eine
konsistente Henkin-Theorie in der Sprache $\mathcal{L}$ mit der folgenden
Eigenschaft:
\begin{align*}
  \text{Für alle Konstantensymbole } c,d \text{ in } \mathcal{L}
  \text{ gilt entweder } \Sigma \vdash c = d \text{ oder } \Sigma \vdash c \neq d.
\end{align*}
Weiters habe $\Sigma$ die Eigenschaft, dass es zwei Konstantensymbole $0, 1$
mit $\Sigma \vdash 0 \neq 1$ gibt. Zeigen Sie, dass $\Sigma$ vollständig sein muss.

\end{exercise}

% --------------------------------------------------------------------------------

\begin{solution}

\phantom{}

\end{solution}

% --------------------------------------------------------------------------------

% --------------------------------------------------------------------------------

\begin{exercise}[139]

Geben Sie eine Sprache $\mathscr{L}$ und zwei $\mathscr{L}$-Strukturen $\mathscr{M}_1$
und $\mathscr{M}_2$ an, sodass $\Th(\mathscr{M}_1)$ eine Henkin-Theorie ist,
nicht aber $\Th(\mathscr{M}_2)$.

\end{exercise}

% --------------------------------------------------------------------------------

\begin{solution}

\phantom{}

\end{solution}

% --------------------------------------------------------------------------------

% -------------------------------------------------------------------------------- %

\begin{exercise}[140]

Jede Theorie mit der schwachen Henkin-Eigenschaft hat auch die starke Henkin-Eigenschaft. \\
\textit{Hinweis:} Verwenden Sie die Formel $\exists x (\exists y \psi(y) \implies \psi(x))$,
siehe Aufgabe 100.

\end{exercise}

% -------------------------------------------------------------------------------- %

\begin{solution}

\phantom{}

\end{solution}

% -------------------------------------------------------------------------------- %



\end{document}
