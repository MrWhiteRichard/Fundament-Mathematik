% --------------------------------------------------------------------------------

\begin{exercise}[281]

Sei $A$ eine endliche Menge mit $n$ Elementen. Sei $W(A)$ die Menge aller $(X,R)$,
sodass $X \subseteq A$ ist und $R \subseteq X \times X$ eine lineare Ordnung von $X$
ist. Sei $\simeq$ die Isomorphierelation auf $W(A)$ und $H(A)$ die Menge aller
Äquivalenzklassen.

\begin{enumerate}[label = \alph*.]
  \item Für $A = \{1,2,3\}$: Wie viele Elemente hat $W(A)$? Geben Sie alle an.
  Wie viele Elemente hat $H(A)$? Geben Sie alle an.
  \item Für beliebiges $n$: Wie viele Elemente hat $H(A)$?
\end{enumerate}

\end{exercise}

% --------------------------------------------------------------------------------

\begin{solution}

\phantom{}

\begin{enumerate}[label = \alph*.]
  \item
  \begin{align*}
    W(A)
    \sim \{ 1 < 2 < 3; 1 < 3 < 2; 2 < 1 < 3; 2 < 3 < 1;
    3 < 1 < 2; 3 < 2 < 1; \\
    1 < 2; 2 < 1; 1 < 3; 3 < 1; 2 < 3;
    3 < 2; 1; 2 ; 3 ; \emptyset\}
  \end{align*}
  Es gilt $|W(A)| = 16$ und $|H(A)| = 4$.

  \begin{align*}
    H(A) \sim \{\{1 < 2 < 3; 1 < 3 < 2; 2 < 1 < 3; 2 < 3 < 1;
    3 < 1 < 2; 3 < 2 < 1\}; \\
    \{1 < 2; 2 < 1; 1 < 3; 3 < 1; 2 < 3;
    3 < 2\};\{1; 2 ; 3\};\{\emptyset\}\}
  \end{align*}
  \item $|H(A)| = n + 1$.
\end{enumerate}

\end{solution}

% --------------------------------------------------------------------------------
