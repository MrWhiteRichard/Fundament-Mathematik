\documentclass{article}

% Hier befinden sich Pakete, die wir beinahe immer benutzen ...

\usepackage[utf8]{inputenc}

% Sprach-Paket:
\usepackage[ngerman]{babel}

% damit's nicht so, wie beim Grill aussieht:
\usepackage{fullpage}

% Mathematik:
\usepackage{amsmath, amssymb, amsfonts, amsthm}
\usepackage{bbm, mathrsfs, stmaryrd}
\usepackage{mathtools, mathdots}

% Makros mit mehereren Default-Argumenten:
\usepackage{twoopt}

% Anführungszeichen (Makro \Quote{}):
\usepackage{babel}

% if's für Makros:
\usepackage{xifthen}
\usepackage{etoolbox}

% tikz ist kein Zeichenprogramm (doch!):
\usepackage{tikz}

% bessere Aufzählungen:
\usepackage{enumitem}

% (bessere) Umgebung für Bilder:
\usepackage{graphicx, subfig, float}

% Umgebung für Code:
\usepackage{listings}

% Farben:
\usepackage{xcolor}

% Umgebung für "plain text":
\usepackage{verbatim}

% Umgebung für mehrerer Spalten:
\usepackage{multicol}

% "nette" Brüche
\usepackage{nicefrac}

% Spaltentypen verschiedener Dicke
\usepackage{tabularx}
\usepackage{makecell}

% Für Vektoren
\usepackage{esvect}

% (Web-)Links
\usepackage{hyperref}

% Zitieren & Literatur-Verzeichnis
\usepackage[style = authoryear]{biblatex}
\usepackage{csquotes}

% so ähnlich wie mathbb
%\usepackage{mathds}

% Keine Ahnung, was das macht ...
\usepackage{booktabs}
\usepackage{ngerman}
\usepackage{placeins}

% special letters:

\newcommand{\N}{\mathbb{N}}
\newcommand{\Z}{\mathbb{Z}}
\newcommand{\Q}{\mathbb{Q}}
\newcommand{\R}{\mathbb{R}}
\newcommand{\C}{\mathbb{C}}
\newcommand{\K}{\mathbb{K}}
\newcommand{\T}{\mathbb{T}}
\newcommand{\E}{\mathbb{E}}
\newcommand{\V}{\mathbb{V}}
\renewcommand{\P}{\mathbb{P}}
\newcommand{\1}{\mathbbm{1}}

\newcommand  {\B}{\mathfrak{B}}
\renewcommand{\S}{\mathfrak{S}}

% quantors:

\newcommand{\Forall}{\forall \,}
\newcommand{\Exists}{\exists \,}
\newcommand{\ExistsOnlyOne}{\exists! \,}
\newcommand{\nExists}{\nexists \,}

% MISC symbols:

\newcommand{\landau}[1]
{
  {\scriptstyle \mathcal{O}}
  \pbraces{#1}
}

\newcommand{\Landau}[1]
{
  \mathcal{O}
  \pbraces{#1}
}

\newcommand{\eps}{\mathrm{eps}}

% graphics in a box:

\newcommandtwoopt
{\includegraphicsboxed}[3][][]
{
  \begin{figure}[!h]
    \begin{boxedin}
      \ifthenelse{\isempty{#2}}
      {
        \begin{center}
          \includegraphics[width = 0.75 \textwidth]{#3}
          \label{fig:#1}
        \end{center}
      }{
        \begin{center}
          \includegraphics[width = 0.75 \textwidth]{#3}
          \caption{#2}
          \label{fig:#1}
        \end{center}
      }
    \end{boxedin}
  \end{figure}
}

% braces:

\newcommand{\pbraces}[1]{{\left  ( #1 \right  )}}
\newcommand{\bbraces}[1]{{\left  [ #1 \right  ]}}
\newcommand{\Bbraces}[1]{{\left \{ #1 \right \}}}
\newcommand{\vbraces}[1]{{\left  | #1 \right  |}}
\newcommand{\Vbraces}[1]{{\left \| #1 \right \|}}
\newcommand{\abraces}[1]{{\left \langle #1 \right \rangle}}
\newcommand{\round}[1]{\bbraces{#1}}

\newcommand
{\floor}[1]
{{\left \lfloor #1 \right \rfloor}}

\newcommand
{\ceil} [1]
{{\left \lceil  #1 \right \rceil }}

% special functions:

\newcommand{\norm}  [2][]{\Vbraces{#2}_{#1}}
\newcommand{\diag}  [1]{\mathrm{diag} \: #1}
\newcommand{\dist}  [1]{\mathrm{dist} \: #1}
\newcommand{\mean}  [1]{\mathrm{mean} \: #1}
\newcommand{\erf}   [1]{\mathrm{erf} \: #1}
\newcommand{\id}    [1]{\mathrm{id} \: #1}
\newcommand{\sgn}   [1]{\mathrm{sgn} \: #1}
\newcommand{\supp}  [1]{\mathrm{supp} \: #1}
\newcommand{\arsinh}[1]{\mathrm{arsinh} \: #1}
\newcommand{\arcosh}[1]{\mathrm{arcosh} \: #1}
\newcommand{\artanh}[1]{\mathrm{artanh} \: #1}
\newcommand{\card}  [1]{\mathrm{card} \: #1}
\newcommand{\Span}  [1]{\mathrm{span} \: #1}
\newcommand{\Aut}   [1]{\mathrm{Aut} \: #1}
\newcommand{\End}   [1]{\mathrm{End} \: #1}
\newcommand{\ggT}   [1]{\mathrm{ggT} \: #1}
\newcommand{\kgV}   [1]{\mathrm{kgV} \: #1}
\newcommand{\ord}   [1]{\mathrm{ord} \: #1}
\newcommand{\grad}  [1]{\mathrm{grad} \: #1}
\newcommand{\ran}   [1]{\mathrm{ran} \: #1}
\newcommand{\graph} [1]{\mathrm{graph} \: #1}
\newcommand{\Inv}   [1]{\mathrm{Inv} \: #1}
\newcommand{\pv}    [1]{\mathrm{pv} \: #1}
\newcommand{\Mod}{\: \mathrm{mod} \:}
\newcommand{\Char}{\mathrm{char}}
\newcommand{\At}{\mathrm{At}}
\newcommand{\Ob}{\mathrm{Ob}}
\newcommand{\Hom}{\mathrm{Hom}}
\newcommand{\orthogonal}[3][]{#2 ~\bot_{#1}~ #3}
\newcommand{\Rang}{\mathrm{Rang}}

\newcommand
{\GL}[2][]
{\mathrm{GL}_{#1} \pbraces{#2}}

% fractions:

\newcommand{\Frac}[2]{\frac{1}{#1} \pbraces{#2}}
\newcommand{\nfrac}[2]{\nicefrac{#1}{#2}}

% derivatives & integrals:

\newcommandtwoopt
{\Int}[4][][]
{\int_{#1}^{#2} #3 ~\mathrm{d} #4}

\newcommandtwoopt
{\derivative}[3][][]
{
  \frac
  {\mathrm{d}^{#1} #2}
  {\mathrm{d} #3^{#1}}
}

\newcommandtwoopt
{\pderivative}[3][][]
{
  \frac
  {\partial^{#1} #2}
  {\partial #3^{#1}}
}

\newcommand
{\primeprime}
{{\prime \prime}}

\newcommand
{\primeprimeprime}
{{\prime \prime \prime}}

% Text:

\newcommand{\Quote}[1]{\glqq #1\grqq{}}
\newcommand{\Text}[1]{{\text{#1}}}
\newcommand{\fastueberall}{\text{f.ü.}}
\newcommand{\fastsicher}{\text{f.s.}}

% -------------------------------- %
% amsthm-stuff:

\theoremstyle{definition}

% numbered theorems
\newtheorem{theorem}    {Satz}   [section]
\newtheorem{lemma}      [theorem]{Lemma}
\newtheorem{corollary}  [theorem]{Korollar}
\newtheorem{proposition}[theorem]{Proposition}
\newtheorem{remark}     [theorem]{Bemerkung}
\newtheorem{definition} [theorem]{Definition}
\newtheorem{example}    [theorem]{Beispiel}

% unnumbered theorems
\newtheorem*{theorem*}    {Satz}
\newtheorem*{lemma*}      {Lemma}
\newtheorem*{corollary*}  {Korollar}
\newtheorem*{proposition*}{Proposition}
\newtheorem*{remark*}     {Bemerkung}
\newtheorem*{definition*} {Definition}
\newtheorem*{example*}    {Beispiel}

% Please define this stuff in project ("main.tex"):

% \def \lastexercisenumber {...}
% This will be 0 by default

% \setcounter{section}{...}
% This will be 0 by default
% and hence, completely ignored

\ifnum \thesection = 0
{
  \newtheorem{exercise}{Aufgabe}
}
\else
{
  \newtheorem{exercise}{Aufgabe}[section]
}
\fi

\ifdef
{\lastexercisenumber}
{\setcounter{exercise}{\lastexercisenumber}}

\newenvironment{solution}
{
  \begin{proof}[Lösung]
}{
  \end{proof}
}

\renewcommand{\proofname}{Beweis}

% -------------------------------- %
% environment zum einkasteln:

% dickere vertical lines
\newcolumntype
{x}
[1]
{
  !{
    \centering
    \arraybackslash
    \vrule
    width #1}
}

% environment selbst (the big cheese)
\newenvironment
{boxedin}
{
  \begin{tabular}
  {
    x{1 pt}
    p{\textwidth}
    x{1 pt}
  }
  \Xhline
  {2 \arrayrulewidth}
}
{
  \\
  \Xhline{2 \arrayrulewidth}
  \end{tabular}
}

% -------------------------------- %
% MISC "Ein-Deutschungen"

\renewcommand{\figurename}{Abbildung}
\renewcommand{\tablename} {Tabelle}

% -------------------------------- %

\input{../../../Fundament-LaTeX/listings.tex}

\graphicspath{{../../../Fundament-LaTeX/images/}}

\parskip 6 pt
\parindent 0 pt

\title
{
  Logik und Grundlagen der Mathematik \\
  \vspace{4pt}
  \normalsize
  \textit{13. Übung}
}
\author
{
  Richard Weiss
  \and
  Florian Schager
  \and
  Fabian Zehetgruber
}
\date{21.01.2021}

\begin{document}

\maketitle

\section*{Auswahlaxiom, Endlichkeit}

In diesem Abschnitt arbeiten wir mit der Theorie ZF (ohne AC). Die Existenz von
Auswahlfunktionen oder Wohlordnungen dürfen Sie also nur dann verwenden, wenn sie
explizit in der Angabe garantiert wird.

% --------------------------------------------------------------------------------

\begin{exercise}[272]

Betrachten Sie die folgenden Eigenschaften, die eine Menge $A$ haben kann:

\begin{enumerate}[label = \alph*.]
  \item Es gibt eine injektive Abbildung von $\omega$ nach $A$. (\blockquote{$\omega \leq A$})
  \item Es gibt eine fast injektive Abbildung von $\omega$ nach $A$.
  (\blockquote{fast injektiv} bedeutet, dass das Urbild jedes Bildpunktes endlich ist.)
  \item Es gibt eine injektive, aber nicht surjektive Abbildung von $A$ nach $A$.
  \item Es gibt eine fast injektive, aber nicht surjektive Abbildung von $A$ nach $A$.
  \item Für ein (alle) $x \notin A$ gibt es eine Bijektion von $A$ nach $A \cup \Bbraces{x}$.
  (\blockquote{$A = A + 1$})
  \item Es gibt eine surjektive, aber nicht injektive Abbildung von $A$ nach $A$.
  \item Es gibt eine surjektive Abbildung von $A$ auf $\omega$. (\blockquote{$\omega \leq^* A$})
  \item Es gibt eine surjektive, fast injektive Abbildung von $A$ auf $\omega$.
  \item Es gibt eine injektive Abbildung von $\omega$ nach $P(A)$. (\blockquote{$\omega \leq P(A)$})
  \item Es gibt eine injektive Abbildung von $\omega$ in die endlichen Teilmengen von $A$. (\blockquote{$\omega \leq P_\mathrm{fin}(A)$})
  \item Es gibt eine surjektive Abbildung von den endlichen Teilmengen von $A$ auf $\omega$.
  \item $A$ ist unendlich: \blockquote{$|A| = \infty$}
  \item Es gibt eine nichtleere Teilmenge von $\mathfrak{P}(A)$ ohne (bez. $\subseteq$) maximales Element.
\end{enumerate}

Geben Sie möglichst viele nichttriviale Implikationen zwischen diesen Aussagen an, die sich in ZF (also ohne Auswahlaxiom) beweisen lassen.

\end{exercise}

% --------------------------------------------------------------------------------

\begin{solution}

\phantom{}

\begin{tikzpicture}[
	> = stealth, % arrow head style
	shorten > = 1pt, % don't touch arrow head to node
	auto,
	node distance = 3cm, % distance between nodes
	semithick % line style
	]
	
	\tikzstyle{every state}=[
	draw = black,
	thick,
	fill = white,
	minimum size = 4mm
	]
	
	\node[state] (a) {$a$};
	\node[state] (b) [above of=a] {$b$};
	\node[state] (c) [right of=a] {$c$};
	\node[state] (f) [below right of=a] {$f$};
	\node[state] (h) [below of=a] {$h$};
	\node[state] (d) [right of=c] {$d$};
	\node[state] (e) [below right of=c] {$e$};
	\node[state] (g) [below left of=a] {$g$};
	\node[state] (j) [above left of=a] {$j$};
	\node[state] (i) [left of=j] {$i$};
	\node[state] (k) [left of=a] {$k$};
	\node[state] (l) [above right of=a] {$l$};
	\node[state] (m) [right of=l] {$m$};

	
	\path[->] (a) edge [bend left] node {} (b);
	\path[->] (b) edge [bend left] node {} (a);
	\path[->] (c) edge [bend left] node {} (a);
	\path[->] (a) edge node {} (c);
	\path[->] (a) edge node {} (f);
	\path[->] (a) edge [red] node {} (h);
	\path[->] (c) edge node {} (d);
	\path[->] (e) edge node {} (c);
	\path[->] (e) edge [dotted] node {} (f);
	\path[->] (a) edge node {} (g);
	\path[->] (a) edge node {} (j);
	\path[->] (j) edge node {} (i);
	\path[->] (h) edge node {} (g);
	\path[->] (j) edge node {} (k);
	\path[->] (g) edge node {} (k);
	\path[->] (l) edge node {} (m);
	\path[->] (a) edge node {} (l);
	\path[->] (l) edge [dashed, bend left] node {} (a);
\end{tikzpicture}

Hier einige Inklusionen, im Bild oben als durchgehende schwarze Pfeile eingezeichnet.

\begin{enumerate}[label = \texttt{ad}]
	\item \blockquote{a. $\implies$ b.}:
	
	Klar!
	
	\item \blockquote{b. $\implies$ a.}:
	
	\begin{enumerate}[label = \arabic*.]
		
		\item Lösung:
		
		Sei $f: \omega \to A$ fast injektiv, d.h.
		
		\begin{align*}
		\Forall x \in f[\omega]:
		|f^{-1}[\Bbraces{x}]| < \infty.
		\end{align*}
		
		Wir fassen $\omega$ und $A$ als Algebren mit leerem Typ auf.
		Auf diese können wir also den Homomorphiesatz anwenden und bekommen eine Abbildung $g$.
		
		\textbf{Achtung!}
		Möglicherweise braucht der Homomorphiesatz das Auswahlaxiom?
		Anstatt einen beliebigen Repräsentanten $u$ von $U$ zu wählen, kann man in unserem Fall den kleinsten $u := \min U$ (von endlich vielen) wählen, um $g(U) := f(u)$ zu definieren.
		Wir haben es schließlich mit Schuhen und nicht mit Socken zu tun!
		
		\phantom{}
		
		\begin{tcolorbox}[standard jigsaw, opacityback = 0]
			\centering
			\includegraphics
			[width = 0.75 \textwidth]
			{Alg/Alg - Satz 2.3.3.16.1 (Homomorphiesatz).png} \\
			\vspace{0.25 cm}
			\includegraphics
			[width = 0.75 \textwidth]
			{Alg/Alg - Satz 2.3.3.16.2 (Homomorphiesatz).png} \\
			\vspace{0.25 cm}
			\includegraphics
			[width = 0.15 \textwidth]
			{Alg/Alg - Satz 2.3.3.16.3 (Homomorphiesatz).png}
		\end{tcolorbox}
		
		\phantom{}
		
		Seien die (unendlich vielen!) Urbilder gemäß ihrem Minimum, vermöge $h$ (strikt monoton steigend), geordnet, d.h.
		
		\begin{align*}
		h := (U_n)_{n \in \omega}:
		\omega \to \omega / \sim:
		\min U_1 < \min U_2 < \cdots
		\end{align*}
		
		$g \circ h: \omega \to A$ ist, als Verkettung injektiver Funktionen, injektiv.
		
		\item Lösung:
		
		Sei $f: \omega \to A$ fast injektiv. \\
		Definiere die injektive Funktion $f': \omega \to A$ durch
		\begin{align*}
		f'(0) &:= f(0) \\
		f'(n) &:= f(\min\{k \in  \N:  f(k) \notin f[\{0,\dots,n-1\}]\}), \quad n \geq 1.
		\end{align*}
		Da das Urbild jedes Bildpunktes endlich ist, exisitiert dieses Minimum immer.
		
	\end{enumerate}

	\item \blockquote{a. $\implies$ c.}: Sei $f: \omega \to A$ injektiv.
	Definiere
	\begin{align*}
	g := \{(f(n),f(n+1)): n \in \omega\} \cup \{(a,a): a \in A \setminus f(\omega)\}
	\end{align*}
	$g$ ist sicher auf ganz $A$ definiert, injektiv und es gilt $f(0) \notin g(A)$.
	
	Diese Implikation wurde auch in Aufgabe 269 gezeigt.
	
	\item \blockquote{c. $\implies$ a.}: Siehe letzte Übung Aufgabe 270.
	
	\item \blockquote{c. $\implies$ d.}:
	
	Klar.
	
	\item \blockquote{e. $\implies$ c.}:
	
	Sei $x \not \in A$ und $f: A \to A \cup \Bbraces{x}$ bijektiv.
	$f^{-1} |_A: A \to A$ ist injektiv, trifft aber nicht $f^{-1}(x)$, ist also nicht surjektiv.
	
	\item \blockquote{a. $\implies$ f.}: Sei $f: \omega \to A$ injektiv. Definiere
	
	\begin{align*}
	g := \{(f(n+1),f(n)): n \in \omega\} \cup \{(a,a): a \in A \setminus f(\omega)\}
	\cup \{(f(0),f(0))\}
	\end{align*}
	$g$ ist surjektiv, aber nicht injektiv.
	
	\item \blockquote{h. $\implies$ g.}:
	
	Klar!
	
	\item \blockquote{a. $\implies$ g.}:
	
	Sei $f: \omega \to A$ injektiv.
	Es gibt folglich also eine (surjektive) Linksinverse $f^{-1}: f[\omega] \to \omega$.
	Wir setzen diese zu einer (surjektiven) Funktion $g: A \to \omega$ beliebig fort.
	
	\item \blockquote{g. $\implies$ k.}:
	
	Sei $f: A \to \omega$ surjektiv.
	
	\begin{gather*}
	g:
	P(A) \to \omega:
	M \mapsto \min f[M],
	\quad
	g_\mathrm{fin} := g |_{P_\mathrm{fin}(A)},
	\quad
	g_\mathrm{sing} := g_\mathrm{fin} |_{P_\mathrm{sing}(A)} \cong f \\
	\implies
	\omega
	\supseteq
	g(P(A))
	\supseteq
	g_\mathrm{fin}(P_\mathrm{fin}(A))
	\supseteq
	g_\mathrm{sing}(P_\mathrm{sing}(A))
	=
	f(A)
	=
	\omega
	\end{gather*}
	
	$g$, $g_\mathrm{fin}$, und $g_\mathrm{sing}$ sind also auch allersamt surjektiv.
	
	\item \blockquote{a. $\implies$ j.}:
	
	Sei $f: \omega \to A$ injektiv, so auch $g: \omega \to P_\mathrm{fin}(A): n \to \Bbraces{n}$.
	
	\item \blockquote{j. $\implies$ i.}:
	
	Klar!
	
	\item \blockquote{j. $\implies$ k.}:
	
	Verwende \blockquote{a. $\implies$ g.} für $P_\mathrm{fin}(A)$ anstelle von $A$.
	
	\item \blockquote{a. $\implies$ l.}:
	
	Wir führen einen Widerspruchsbeweis. Sei also $A$ endlich mit $n \in \omega$ und $f: A \to n$ bijektiv und $g: \omega \to A$ injektiv. Dann ist $h:= g \circ g: \omega \to n$ ebenfalls injektiv. Das kann aber nicht sein, denn wir sehen induktiv, dass
	\begin{align*}
		\forall k \in \omega: \forall \alpha: \omega \to k: \alpha \text{ nicht injektiv. }
	\end{align*}
	Für $k = 0 = \emptyset$ gibt es gar keine Abbildung $\alpha: \omega \to k$, also ist die Aussage klar.
	
	Nehmen wir nun also an, die Aussage gilt bereits für $k \in \omega$. Weiters nehmen wir an, es git eine injektive Funktion $\alpha: \omega \to k + 1$. Wegen der Injektivität gibt es genau ein $l \in \omega$ mit $\alpha(l) = \{k\}$. Nun finden wir die injektive Funktion 
	\begin{align*}
		u: \omega \to \omega: x \mapsto
		\begin{cases}
			x &, \text{falls } x < l \\
			x + 1 &, \text{falls } x \geq l
		\end{cases}.
	\end{align*}
	Nun ist $\beta: \omega \to n: x \mapsto \alpha(u(x))$ eine injektive Funktion. Ein Widerspruch zur Induktionsvoraussetzung. 
	
	\item \blockquote{l. $\implies$ m.}:
	
	Sei also $A$ eine undenliche Menge. Definiere
	\begin{align*}
	\mathfrak{M}:=\Bbraces{X \subseteq A \mid X \text{ endlich }}
	\end{align*}
	und sei $X \in \mathfrak{M}$ beliebig. Die Menge $X$ ist endlich, also gibt es ein $n \in \omega$ und eine Bijektion $f:n \to X$. Da $A$ unendlich ist, kann $f:n \to A$ nicht surjektiv sein also gibt es $y \in A \setminus f[n]$. Nun definiere $Z := X \cup \{y\}$ und
	\begin{align*}
	g:n + 1 \to Z: m \mapsto
	\begin{cases}
	y & ,\text{falls } m = n \\
	f(m) & ,\text{sonst }
	\end{cases}.
	\end{align*}
	Wir sehen, dass $g$ bijektiv ist also $Z$ endlich aber $Z \supsetneq X$, also ist $X$ nicht maximal.
\end{enumerate}




Hier Implikationen die sicher nicht gelten, im Bild oben als rote Pfeile eingezeichnet.

\begin{enumerate}[label = \texttt{ad}]
	\item \blockquote{a. $\implies$ h.}: Wenn $f: A \to \omega$ eine surjektive und fast injektive Abbildung ist, dann können wir $A$ schreiben als
	\begin{align*}
		A = f^{-1}(\omega) = f^{-1}\pbraces{\bigcup_{n \in \omega} \Bbraces{n}} = \bigcup_{n \in \omega} f^{-1}\pbraces{\{n\}}
	\end{align*}
	und da $f^{-1}\pbraces{{n}}$ für jedes $n$ endlich ist wissen wir, dass $A$ höchstens abzählbar unendlich sein kann. Da wir aber schon überabzählbar unendliche Mengen kennen kann diese Inklusion im Allgemeinen nicht gelten. 
\end{enumerate}


Hier Implikationen, für die man (zumindest laut Skriptum) das Auswahlaxiom braucht. Im Graphen oben sind das die strichlierten Pfeile.
\begin{enumerate}[label = \texttt{ad}]
	\item \blockquote{l. $\implies$ a.}:
	
	Laut Satz VI.5.1 im Skriptum braucht man hier das Auswahlaxiom.
\end{enumerate}


Hier noch zusätzliche Implikationen, die sich eigentlich schon aus anderen ergeben.


\begin{enumerate}[label = \texttt{ad}]

  \item \blockquote{e. $\implies$ f.}:

  Sei $x \not \in A$ und $f: A \to A \cup \Bbraces{x}$ bijektiv.
  Sei weiters $y \in A$.
  $g: A \to A$ ist surjektiv, $y$ hat aber die beiden verschiedenen Urbilder $f^{-1}(y) \neq f^{-1}(x)$, weil $x \neq y$, ist also nicht injektiv.

  \begin{align*}
    g:
    A \to A:
    z
    \mapsto
    \begin{cases}
      f(z), & f(z) \in A, \\
      y,    & f(z) = x
    \end{cases}
  \end{align*}
\end{enumerate}



Schließlich noch alte Argumentationen, die womöglich nicht stimmen.

\begin{enumerate}[label = \texttt{ad}]
	\item \blockquote{a. $\implies$ h.}: Sei $f: \omega \to A$ injektiv. Definiere
	\begin{align*}
	g = \{(f(n), n): n \in \omega\} \cup \{(a,0): a \in A \setminus f(\omega)\}.
	\end{align*}
\end{enumerate}

\end{solution}

% --------------------------------------------------------------------------------


\section*{Zorn, Hausdorff, etc.}

Auch hier arbeiten wir in ZF (ohne AC).

\begin{exercise}
Bestimmen Sie die Momente und die Momentenerzeugende für die Laplaceverteilung
mit der Dichte $f(x) = \frac{1}{2}e^{-|x|}$.
\end{exercise}

\begin{solution}

Trivial

\end{solution}

\begin{exercise}
$(X_n)$ sei eine Folge von unabhängigen exponentialverteilten Zufallsvariablen
mit Parameter 1.

\begin{itemize}
  \item[(a)] Bestimmen Sie die Verteilung von $Y_n = \max(X_1,\dots,X_n)$
  \item[(b)] Zeigen Sie, dass $Y_n - \log(n)$ in Verteilung konvergiert und bestimmen Sie die Grenzverteilung (Gumbelverteilung, doppelte Exponentialverteilung).
  \item[(c)] Bestimmen Sie die momentenerzeugende Funktion dieser Verteilung.
\end{itemize}

\end{exercise}

\begin{solution}
  Hier könnte Ihre Werbung stehen!

  \begin{itemize}
    \item[(a)] Wir wählen $x \in \R$ beliebig und berechnen die Verteilungsfunktion in diesem Punkt.

    \begin{multline*}
      F_{Y_n}(x)
      = \P \pbraces{Y_n \leq x}
      = \P \pbraces{X_1, \ldots, X_n \leq x}
      = \P \pbraces{\bigcap_{i=1}^n [X_i \leq x]} \\
      \stackrel{\text{Unabh.}}{=}
        \prod_{i=1}^n \P \pbraces{X_i \leq x}
      = \prod_{i=1}^n F_X(x)
      = \pbraces{1 - \exp(-x)}^n 1_{]0, \infty[}(x).
    \end{multline*}

    \item[(b)] Wir nehmen ein beliebiges $x \in \R$, hoffen, dass es ein Stetigkeitspunkt der Grenzverteilungsfunktion unserer Folge von Verteilungsfunktionen ist und rechnen

    \begin{multline*}
      F_{Y_n - \log(n)}(x)
      = \P \pbraces{Y_n - \log(n) \leq x}
        \P \pbraces{Y_n \leq \log(n) + x}
        \pbraces{1 - \exp(-\pbraces{\log(n) + x})}^n
      1_{]0,\infty[}(\log(n) + x) \\
      = \pbraces{1 + \frac{-\exp(-x)}{n}}^n
        \1_{]0, \infty[}(\log(n) + x)
      \xrightarrow{n \to \infty}
        \exp \pbraces{-\exp(-x)} =: F(x).
    \end{multline*}

    Also gilt

    \begin{align*}
      Y_n - \log(n) \xrightarrow{\text{schwach}} Y \sim F.
    \end{align*}

    \item[(c)] Zuerst sei bemerkt, dass die Dichtefunktion von $Y$ gegeben ist durch die Ableitung der Verteilungsfunktion $F$, also die Form

    \begin{align*}
      f:\R \to \R: x \mapsto \exp(-x) \exp \pbraces{-\exp(-x)}
    \end{align*}

    hat. Wie wir in der Rechnung sehen werden gilt $\Forall t < 1:$

    \begin{align*}
      \E(\exp(Yt))
      & \stackrel{\text{TRAFO}}{=}
        \Int[\R]{\exp(xt)}{\P Y^{-1}(x)}
      = \Int[\R]{\exp(xt) f(x)}{\lambda(x)}
      = \Int[\R]{\exp \pbraces{-\exp(-x)} \exp(-x) \exp(xt)}{\lambda(x)} \\
      & \stackrel{u = \exp(-x)}{=}
        - \Int[\infty][0]{\exp(-u) \exp \pbraces{-\log(u)t}}{\lambda(u)}
      = \Int[\R^+]{\exp(-u) u^{-t}}{\lambda(u)} \\
      & =
        \Int[\R^+]{\exp(-u) u^{(1 - t) - 1}}{\lambda(u)}
      = \Gamma(1 - t).
    \end{align*}
  \end{itemize}
\end{solution}

% --------------------------------------------------------------------------------

\begin{exercise}[280]

Seien $A$ und $B$ Mengen, eventuell mit einer zusätzlichen Struktur
(Vektorraum bzw. partielle Ordnung). Welche der folgenden Familien haben endlichen
Charakter? (Achtung: Bei manchen der folgenden Punkte hängt die Antwotz davon ab,
ob $A$ und/oder $B$ endlich oder gar leer sind.)

\begin{itemize}
  \item Alle Teilmengen von $A$.
  \item Alle unendlichen Teilmengen von $A$.
  \item Alle endlichen Teilmengen von $A$.
  \item Alle partiellen Funktionen von $A$ nach $B$.
  \item Alle partiellen injektiven Funktionen von $A$ nach $B$.
  \item Alle partiellen surjektiven Funktionen von $A$ nach $B$.
  \item Alle partiellen nichtsurjektiven Funktionen von $A$ nach $B$.
  \item Die linear unabhängigen Teilmengen von $A$.
  \item Die linear abhängigen Teilmengen von $A$.
  \item Alle partiellen ordnungserhaltenden Abbildungen von $A$ nach $B$.
  (Das heißt, wenn $f(a), f(a')$ definiert sind, und $a \leq a'$ gilt, dann
  auch $f(a) \leq f(a')$.)
\end{itemize}

\end{exercise}

% --------------------------------------------------------------------------------

\begin{solution}

\phantom{}

\end{solution}

% --------------------------------------------------------------------------------

% --------------------------------------------------------------------------------

\begin{exercise}[281]

Sei $A$ eine endliche Menge mit $n$ Elementen. Sei $W(A)$ die Menge aller $(X, R)$, sodass $X \subseteq A$ ist und $R \subseteq X \times X$ eine lineare Ordnung von $X$ ist.
Sei $\simeq$ die Isomorphierelation auf $W(A)$ und $H(A)$ die Menge aller Äquivalenzklassen.

\begin{enumerate}[label = \alph*.]

  \item Für $A = \Bbraces{1, 2, 3}$:
  Wie viele Elemente hat $W(A)$?
  Geben Sie alle an.
  Wie viele Elemente hat $H(A)$?
  Geben Sie alle an.

  \item Für beliebiges $n$:
  Wie viele Elemente hat $H(A)$?

\end{enumerate}

\end{exercise}

% --------------------------------------------------------------------------------

\begin{solution}

\phantom{}

\begin{comment}

  \begin{enumerate}[label = \alph*.]
    \item
    \begin{align*}
      W(A)
      = \{ 1 \leq 2 \leq 3; 1 \leq 3 \leq 2; 2 \leq 1 \leq 3; 2 \leq 3 \leq 1;
      3 \leq 1 \leq 2; 3 \leq 2 \leq 1\}
    \end{align*}
    Wir können die Elemente aus $W(A)$ in kanonischer Weise mit den
    Permutationen auf der Menge $A = \{1, 2, 3\}$ identifizieren.
    Dabei sind alle linearen Ordnungen vermöge der jeweiligen Permutation
    isomorph,  also hat $H(A)$ genau ein Element.
    \item Ist nicht für alle $n$ $H(A)$ einfach ein-elementig?
  \end{enumerate}

\end{comment}

\begin{enumerate}[label = \alph*.]

  \item $W(A)$ und $H(A)$ sind der unteren Darstellung zu entnehmen.

  \begin{align*}
    \begin{array}{c|c|ccc|ccc|c}
      X & \emptyset & \Bbraces{1} & \Bbraces{2} & \Bbraces{3} & \Bbraces{1, 2} & \Bbraces{1, 3} & \Bbraces{2, 3} & \Bbraces{1, 2, 3} \\
        &           &             &             &             &                &                &                &                   \\
      R &           & 1           & 2           & 3           & 1 2            & 1 3            & 2 3            & 1 2 3             \\
        &           &             &             &             & 2 1            & 3 1            & 3 2            & 1 3 2             \\
        &           &             &             &             &                &                &                & 2 1 3             \\
        &           &             &             &             &                &                &                & 2 3 1             \\
        &           &             &             &             &                &                &                & 3 1 2             \\
        &           &             &             &             &                &                &                & 3 2 1             \\
    \end{array}
  \end{align*}

  \item $|H(A)| = n + 1$

\end{enumerate}

\end{solution}

% --------------------------------------------------------------------------------


\section*{Wohlordnungen (2)}

In diesem Abschnitt arbeiten wir mit der Theorie ZF. Die Existenz von Auswahlfunktionen
oder Wohlordnungen dürfen Sie also nur dann verwenden, wenn sie explizit in der Angabe
garantiert wird. \\
Für jede lineare Ordnung $(L,<)$ und $x \in L$ sei $L_x := \{y \in L: y <_L x\}$.
Wir schreiben $A^{<L}$ für die Menge aller Funktionen, deren Definitionsbereich eine
Menge der Form $L_x$ ist, und deren Wertemenge Teilmenge von $A$ ist. \\
Für Wohlordnungen $(X, <)$ und $(Y,<)$ schreiben wir $(X,<) \sqsubset (Y,<)$
(\glqq $X$ ist kürzer als $Y$\grqq), wenn es ein $y_0 \in Y$ gibt mit
$(X, <) \simeq (Y_{<y_0},<)$. Wir schreiben $(X,<) \sqsubseteq$ für
\glqq$(X,<) \sqsubset (Y,<)$ oder $(X,<) \simeq (Y,<)$\grqq.

\begin{exercise}

Die kumulantenerzeugende Funktion einer Zufallsvariable $X$ ist

\begin{align*}
  K_X(t) = \log(M_X(t)).
\end{align*}

Wenn $M_X$ in einer Umgebung von 0 existiert, dann kann man $K_X$ als Potenzreihe schreiben:

\begin{align*}
  K_X(t) = \sum_n\frac{\kappa_nt^n}{n!}.
\end{align*}
Die Koeffizienten $\kappa_n$ heißen die Kumulanten von $X$. Drücken Sie $\kappa_n, n = 2, \ldots, 5$ durch die zentralen Momente

\begin{align*}
  m_n = \mathbb{E}((X-\mathbb{E}(X))^n)
\end{align*}

von $X - \mathbb{E}(X)$ aus. (mit $\mu = \mathbb{E}(X)$ und $Y = X - \mu$ gilt
$K_X(t) = \mu t + K_Y(t)$; diese Darstellung der Kumulanten ist einfacher als die durch die gewöhnlichen Momente, die allerdings im Internet leichter zu finden ist).

\end{exercise}

\begin{solution}

Wir berechnen vorerst die Momenterzeugende von $Y$.

\begin{align*}
  M_Y(t)
  = \E(e^{tY})
  = \E(e^{t (x - \mu)})
  = e^{-\mu t} \E(e^{tX})
  = e^{-\mu t} M_X(t)
\end{align*}

Damit erhalten wir $M_X(t) = e^{\mu t} M_Y(t)$ und mit dem Satz von Taylor, dass

\begin{align*}
  K_X(t)
  = \log(e^{\mu t} M_Y(t))
  = \mu t + \log(M_Y(t))
  = \sum_{n=0}^\infty
    \underbrace
    {
      \frac{d^n}{dt^n}
      \pbraces{\mu t + \log(M_Y(t))} \Bigg|_{t=0}
    }_{=: \kappa_n}
    \frac{t^n}{n!}.
\end{align*}

Wir brauchen noch folgende Darstellung.

\begin{align*}
  M_Y(t)
  = \E(e^{tY})
  = \E
    \pbraces
    {
      \sum_{n=0}^\infty
      \frac{(tY)^n}{n!}
    }
  = \sum_{n=0}^\infty \frac{t^n}{n!} \E(Y^n)
  = \sum_{n=0}^\infty \frac{t^n}{n!} m_n
\end{align*}

Wir berechnen die erste Ableitung.

\begin{align*}
  \frac{d}{dt}
  \pbraces{\mu t + \log \pbraces{\sum_{n=0}^\infty \frac{t^n}{n!} m_n}}
  = \mu + \Frac
    {\sum_{n=0}^\infty \frac{t^n}{n!} m_n}
    {\sum_{n=1}^\infty \frac{t^{n-1}}{(n-1)!} m_n}
  = \mu + \Frac
    {\sum_{n=0}^\infty \frac{t^n}{n!} m_n}
    {\sum_{n=0}^\infty \frac{t^n}{n!} m_{n+1}}
\end{align*}

Wir brauchen vorerst noch

\begin{align*}
  m_0 & = \E((X - \E(X))^0) = \E(1) = 1, \\
  m_1 & = \E((X - \E(X))^1) = \E(X) - \E(\E(X)) = \E(X) (1 - \E(1)) = 0.
\end{align*}

Daraus foltgt der erste Kumulant.

\begin{align*}
  \kappa_1 = \mu + \frac{m_1}{m_0} = \mu
\end{align*}

Wir berechnen die zweite Ableitung.

\begin{align*}
  \frac{d^2}{dt^2}
  \pbraces{\mu t + \log \pbraces{\sum_{n=0}^\infty \frac{t^n}{n!} m_n}}
  = \Frac
    {\pbraces{\sum_{n=0}^\infty \frac{t^n}{n!} m_n}^2}
    {
      \pbraces{\sum_{n=0}^\infty \frac{t^n}{n!} m_{n+2}}
      \pbraces{\sum_{n=0}^\infty \frac{t^n}{n!} m_n} -
      \pbraces{\sum_{n=0}^\infty \frac{t^n}{n!} m_{n+1}}^2
    }
\end{align*}

Daraus foltgt der zweite Kumulant.

\begin{align*}
  \kappa_2 = \Frac{m_0}{m_2 m_0 - m_1^2} = m_2
\end{align*}

Der Rest folgt analog.

\begin{align*}
  \kappa_3 & =
  \Frac{m_0^3}{m_0^2 m_3 + 2 m_1^3 - 3 m_0 m_1 m_2} = m_3 \\
  \kappa_4 & =
  \Frac{m_0^4}{- 6 m_1^4 + m_0^2 (m_0 m_4 - 3 m_2^2) - 4 m_0^2 m_3 m_1 + 12 m_0 m_1^2 m_2} = m_4 - 3 m_2^2 \\
  \kappa_5 & =
  \Frac{m_0^5}{m_0^4 m_5 + 24 m_1^5 - 5 m_0^3 m_4 m_1 - 10 m_0^3 m_3 m_2 + 20 m_0^2 m_3 m_1^2 - 60 m_0 m_1^3 m_2 + 30 m_0^2 m_1 m_2^2} = m_5 - 10 m_3 m_2
\end{align*}

\end{solution}

% --------------------------------------------------------------------------------

\begin{exercise}[287]

Zeigen Sie (in ZF, ohne AC): Wenn $(A,<)$ eine Wohlordnung ist, dann gibt es eine
lineare Ordnung auf der Potenzmenge von $A$.
\end{exercise}

% --------------------------------------------------------------------------------

\begin{solution}
\textbf{Gescheiterter} erster Versuch: \\
Für alle $B \subseteq A$: Definiere die partielle Funktion $f_B: \omega \to B$ durch
\begin{align*}
  f_B(n) = \begin{cases}
    \min(B \setminus f[\{0,\dots,n-1\}]) & \text{falls }
    B \setminus f[\{0,\dots,n-1\}] \neq \emptyset \\
    \min(A) & \text{sonst}
  \end{cases}
\end{align*}

Wir definieren auf $\mathfrak{P}(A)$
\begin{align*}
  B <_\mathfrak{P} C \iff [\exists n \in \omega: (f_B(n) < f_C(n)) \land \forall k < n: f_B(k) = f_C(k)] \lor (B = \emptyset \land C \neq \emptyset)
\end{align*}

\begin{itemize}
  \item[Irreflexivität:] Klar.
  \item[Transitivität:] Gelte $B <_\mathfrak{P} C$ und $C <_\mathfrak{P} D$. \\
  Dann existieren $n$ und $m$, sodass
  $f_B(n) < f_C(n)$ und $f_C(m) < f_D(m)$. Für $l := \min(n,m)$
  folgt $f_B(l) \leq f_C(l) \leq f_D(l)$, wobei bei mindestens einer Ungleichung
  echt ungleich gelten muss, also $f_B(l) < f_D(l)$ und für alle $k < l: f_B(k) = f_D(k)$.

  \item[Trichotomie:] Gelte $B \nless_\mathfrak{P} C$ und $C \nless_\mathfrak{P} B$
  und $B, C \neq \emptyset$. Also gilt für alle $k \in \N: f_B(k) = f_C(k)$.
  Für endliche Mengen folgt daraus sicher bereits Gleichheit, bei
  unendlichen Mengen könnten vielleicht noch Tricks versteckt sein. \\
\end{itemize}
Problem: Betrachte die Wohlordnung $(\N \cup +\infty, <)$, wobei wir zu den
natürlichen Zahlen ein maximales Element hinzugefügt haben. Dann wären die Mengen
$\N$ und $\N \cup +\infty$ mit unserer linearen Ordnung nicht vergleichbar.




\textbf{Erfolgreicher} zweiter Versuch: \\
Wir definieren $\overline{A} = A \cup \{-\infty\}$ mit $\forall x \in A: -\infty < x$
und weiters $\min(\emptyset) := -\infty$. \\
Auf der Potenzmenge definieren wir

\begin{align*}
  B <_\mathfrak{P} C :\min(B \setminus C) < \min(C \setminus B)
\end{align*}


\begin{center}
\def\r{1}
\def\R{1.3}
\begin{tikzpicture}[scale=1.5]
  \foreach\ang/\name in {-150/a,-30/b,90/c} \draw (\ang:\r) circle (\R) coordinate (\name) node {\name};
  \path (a) -- (b) node [midway] (ab) {ab} ;
  \path (b) -- (c) node [midway] (bc) {bc} ;
  \path (c) -- (a) node [midway] (ca) {ca} ;
  \path (c) -- (ab) node [pos=.667] (abc) {abc};
\end{tikzpicture}
\end{center}

\begin{itemize}
  \item[Irreflexivität:] Klar.
  \item[Transitivität:] Gelte $A <_\mathfrak{P} B$ und $B <_\mathfrak{P} C$. \\
  Wir verwenden ab nun die Kurznotationen aus der Skizze: \\
  (Man beachte, dass die Abkürzungen auch für die leere Menge stehen können)\\
  Laut Voraussetzung gilt also $\min(a \cup ca) < \min(b \cup bc)$ und
  $\min(b \cup ab) < \min(c \cup ca)$. Damit folgt
  \begin{align*}
    \min(a \cup ab \cup ca) = \min(a \cup ab \cup ca \cup b \cup bc)
    = \min(a \cup ab \cup ca \cup b \cup bc \cup c) \leq \min(c \cup bc)
  \end{align*}
  Angenommen, es gelte $\min(ca) < \min(a \cup ab)$, dann erhalten wir mit
  \begin{align*}
  \min(ca \cup c) &> \min(b \cup ab) \geq \min(b \cup ab \cup bc) \geq \min(a \cup ab \cup ca)
  = \min(ca)
  \end{align*}
  einen Widerspruch! Also gilt
  \begin{align*}
    \min(a \cup ab) = \min(a \cup ab \cup ca) \leq \min(c \cup bc)
  \end{align*}
  und aufgrund der Disjunktheit gilt sogar die strikte Ungleichung und somit $A <_\mathfrak{P} C$.

  \item[Trichotomie:] Gelte $A \nless_\mathfrak{P} B$ und $B \nless_\mathfrak{P} A$, also
  $\min(A \setminus B) = \min(B \setminus A)$. Dies ist aufgrund der Disjunktheit nur
  möglich, wenn $A \setminus B = B \setminus A = \emptyset$, also $A = B$.
\end{itemize}

\end{solution}

% --------------------------------------------------------------------------------


\section*{Außer Konkurrenz}

\subsection*{288 + 289}

% --------------------------------------------------------------------------------

\begin{exercise}[288]

Geben Sei eine explizite Wohlordnung von $V_{\omega}$ an. \\
\textit{Hinweis:} Geben Sie eine explizite Wohlordnung von $V_{n+1} \setminus V_n$ an.
\end{exercise}

% --------------------------------------------------------------------------------

\begin{solution}

\phantom{}

\end{solution}

% --------------------------------------------------------------------------------

\begin{exercise}[289]

Mit PWW bezeichnen wir den Satz: \glqq Für jede wohlgeordnete Menge $A$ gilt,
dass es auf $\mathfrak{P}(A)$ eine Wohlordnung gibt.\grqq\
Schließen Sie aus ZF+PWW dass es auf $V_{\omega + \omega}$ eine Wohlordnung gibt. \\
\textit{Achtung}: Das ist eine schwierige Aufgabe. Bitte kontrollieren Sie sorgfältig,
ob Sie nicht versehentlich das Auswahlaxiom verwendet haben. Wenn Sie eine einfachen
Beweis gefunden haben, dann ist er ziemlich sicher falsch, weil er nämlich in
versteckter Weise das Auswahlaxiom verwendet.

\end{exercise}

% --------------------------------------------------------------------------------

\begin{solution}

\phantom{}

\end{solution}

% --------------------------------------------------------------------------------


\end{document}
