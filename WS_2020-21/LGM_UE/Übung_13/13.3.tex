% -------------------------------------------------------------------------------- %

\begin{exercise}[279]

Geben Sie ein Beispiel einer nichtleeren Familie von abzählbarem Charakter an,
die kein maximales Element hat.

\end{exercise}

% -------------------------------------------------------------------------------- %

\begin{solution}

Als Beispiel einer Mengenfamilie mit abzählbaren, die kein maximales
Element hat betrachte die Menge $\mathcal{E}$ aller endlichen Teilmengen von $\omega$. \\
Aus $A \in \mathcal{E}$ folgt, dass $A$ endlich ist, und somit alle Teilmengen
von $A$ bereits in $\mathcal{E}$ liegen müssen. \\
Sei nun $B$ eine Menge, deren abzählbare Teilmengen allesamt in $\mathcal{E}$
liegen. Daraus folgt bereits $B \subseteq \omega$ und da $B$ damit selbst abzählbar ist,
muss bereits $B \in \mathcal{E}$ gelten. Also hat die Menge $\mathcal{E}$ abzählbaren
Charakter. \\
Da wir zu jeder endlichen Teilmenge $A$ von $\omega$ ein $n \in \omega \setminus A$
hinzufügen können, ohne die Endlichkeit zu verlieren, kann diese Menge kein
maximales Element haben.

\end{solution}

% -------------------------------------------------------------------------------- %
