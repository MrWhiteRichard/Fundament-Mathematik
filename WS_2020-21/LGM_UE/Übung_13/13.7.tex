% --------------------------------------------------------------------------------

\begin{exercise}[287]

Zeigen Sie (in ZF, ohne AC): Wenn $(A,<)$ eine Wohlordnung ist, dann gibt es eine
lineare Ordnung auf der Potenzmenge von $A$.
\end{exercise}

% --------------------------------------------------------------------------------

\begin{solution}
Für alle $B \subseteq A$: Definiere die partielle Funktion $f_B: \omega \to B$ durch
\begin{align*}
  f_B(n) = \begin{cases}
    \min(B \setminus f[\{0,\dots,n-1\}]) & \text{falls }
    B \setminus f[\{0,\dots,n-1\}] \neq \emptyset \\
    \min(A) & \text{sonst}
  \end{cases}
\end{align*}

Wir definieren auf $\mathfrak{P}(A)$
\begin{align*}
  B <_\mathfrak{P} C \iff [\exists n \in \omega: (f_B(n) < f_C(n)) \land \forall k < n: f_B(k) = f_C(k)] \lor (B = \emptyset \land C \neq \emptyset)
\end{align*}

\begin{itemize}
  \item[Irreflexivität:] Klar.
  \item[Transitivität:] Gelte $B <_\mathfrak{P} C$ und $C <_\mathfrak{P} D$. \\
  Dann existieren $n$ und $m$, sodass
  $f_B(n) < f_C(n)$ und $f_C(m) < f_D(m)$. Für $l := \min(n,m)$
  folgt $f_B(l) \leq f_C(l) \leq f_D(l)$, wobei bei mindestens einer Ungleichung
  echt ungleich gelten muss, also $f_B(l) < f_D(l)$ und für alle $k < l: f_B(k) = f_D(k)$.

  \item[Trichotomie:] Gelte $B \nless_\mathfrak{P} C$ und $C \nless_\mathfrak{P} B$
  und $B, C \neq \emptyset$. Also gilt für alle $k \in \N: f_B(k) = f_C(k)$.
  Für endliche Mengen folgt daraus sicher bereits Gleichheit, bei
  unendlichen Mengen könnten vielleicht noch Tricks versteckt sein. \\
\end{itemize}
Problem: Betrachte die Wohlordnung $(\N \cup +\infty, <)$, wobei wir zu den
natürlichen Zahlen ein maximales Element hinzugefügt haben. Dann wären die Mengen
$\N$ und $\N \cup +\infty$ mit unserer linearen Ordnung nicht vergleichbar.

\end{solution}

% --------------------------------------------------------------------------------
