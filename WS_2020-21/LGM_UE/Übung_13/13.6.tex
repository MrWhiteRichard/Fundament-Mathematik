% --------------------------------------------------------------------------------

\begin{exercise}[286]

Beweisen Sie (in ZFW, informell): Eine lineare Ordnung $(L,<)$ ist genau dann
KEINE Wohlordnung, wenn es eine Funktion $f: \omega \to L$ gibt mit
$\forall n \in \omega: f(n+1) < f(n)$. An welchen Stellen Ihres Beweises
(wenn überhaupt) verwenden Sie das Auswahlaxiom?

\end{exercise}

% --------------------------------------------------------------------------------

\begin{solution}

\phantom{}

\end{solution}

% --------------------------------------------------------------------------------
