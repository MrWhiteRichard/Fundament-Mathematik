% --------------------------------------------------------------------------------

\begin{exercise}[286]

Beweisen Sie (in ZFC, informell): Eine lineare Ordnung $(L,<)$ ist genau dann
KEINE Wohlordnung, wenn es eine Funktion $f: \omega \to L$ gibt mit
$\forall n \in \omega: f(n+1) < f(n)$. An welchen Stellen Ihres Beweises
(wenn überhaupt) verwenden Sie das Auswahlaxiom?

\end{exercise}

% --------------------------------------------------------------------------------

\begin{solution}

Beginnen wir mit der Rückrichtung: $f(\omega) \subseteq L$ hat kein kleinstes
Element, also kann $L$ keine Wohlordnung sein. \\
Hinrichtung: Sei $\emptyset \neq K \subseteq L$ eine Teilmenge ohne kleinstes Element.
Also existiert für jedes $x \in K$ ein $x' \in K$ mit $x > x'$.
Mit dem Auswahlaxiom finden wir dann ein $f$ aus der Angabe. Wie genau wenden
wir hier das Auswahlaxiom an?

\end{solution}

% --------------------------------------------------------------------------------
