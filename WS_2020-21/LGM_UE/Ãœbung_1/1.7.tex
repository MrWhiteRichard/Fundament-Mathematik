% -------------------------------------------------------------------------------- %

\begin{exercise}[21]

Welche der folgenden Formeln sind Tautologien?

\begin{enumerate}[label = \alph*.]
    \item $(p_1 \to p_2) \to (p_3 \to p_4) \to ((p_1 \lor p_3) \to (p_2 \lor p_4))$. \\
    (Implikationen werden von rechts nach links geklammert;
    $A \to B \to C$ ist als Abkürzung für $(A \to (B \to C))$ zu lesen, NICHT als $((A \to B) \to C)$, und auch NICHT als $(A \to B) \land (B \to C)$.
    \item $(p_1 \to p_3) \to (p_2 \to p_3) \to ((p_1 \lor p_2) \to p_3)$.
    \item $(p_1 \to p_3) \to (p_2 \to p_3) \to ((p_1 \land p_2) \to p_3)$.
    \item $(p_1 \to p_2 \to p_3) \to ((p_1 \land p_2) \to p_3)$
    \item $(p_1 \to p_2) \lor (p_2 \to p_1)$.
\end{enumerate}

\end{exercise}

% -------------------------------------------------------------------------------- %

\begin{solution}

\phantom{}

\begin{enumerate}[label = \alph*.]

  \item
  
  \begin{align*}
    & (p_1 \to p_2) \to ((p_3 \to p_4) \to ((p_1 \lor p_3) \to (p_2 \lor p_4))) \\
    & \iff
    \neg (p_1 \to p_2) \lor ((p_3 \to p_4) \to ((p_1 \lor p_3) \to (p_2 \lor p_4))) \\
    & \iff
    (p_1 \land \neg p_2) \lor \neg (p_3 \to p_4) \lor ((p_1 \lor p_3) \to (p_2 \lor p_4)) \\
    & \iff
    (p_1 \land \neg p_2) \lor (p_3 \land \neg p_4) \lor \neg (p_1 \lor p_3) \lor p_2 \lor p_4 \\
    & \iff
    (p_1 \land \neg p_2) \lor (p_3 \land \neg p_4) \lor (\neg p_1 \land \neg p_3) \lor p_2 \lor p_4 \\
    & \stackrel{!}{\iff} \bot
  \end{align*}

  Dann sind aber alle Konjunktionen $\iff \bot$.
  Insbesondere, sind die letzten beiden $\iff \bot$, ihre Negationen also $\iff \top$.
  Weil aber die ersten beiden $\iff \bot$, müssen $p_1 \iff p_2 \iff \bot$, ihre Negationen aber $\iff \top$.
  Dann wäre die mittlere aber $\iff \top$.
  Widerspruch!

  \item

  \begin{align*}
    & (p_1 \to p_3) \to ((p_2 \to p_3) \to ((p_1 \lor p_2) \to p_3)) \\
    & \iff
    \neg (p_1 \to p_3) \lor ((p_2 \to p_3) \to ((p_1 \lor p_2) \to p_3)) \\
    & \iff
    (p_1 \land \neg p_3) \lor \neg (p_2 \to p_3) \lor ((p_1 \lor p_2) \to p_3) \\
    & \iff
    (p_1 \land \neg p_3) \lor (p_2 \land \neg p_3) \lor \neg (p_1 \lor p_2) \lor p_3 \\
    & \iff
    (p_1 \land \neg p_3) \lor (p_2 \land \neg p_3) \lor (\neg p_1 \land \neg p_2) \lor p3 \\
    & \stackrel{!}{\iff} \bot
  \end{align*}

  Dann sind aber alle Konjunktionen $\iff \bot$.
  Insbesondere, ist die letzte $\iff \bot$, ihre Negation also $\iff \top$.
  Die ersten beiden sind ebenfalls $\iff \bot$, also $p_1 \iff p_2 \iff \bot$, bzw. $\neg p_1 \iff \neg p_2 \iff \top$.
  Dann wäre aber die vorletzte $\iff \top$.
  Widerspruch!

  \item

  \begin{align*}
    & (p_1 \to p_3) \to ((p_2 \to p_3) \to ((p_1 \lor p_2) \to p_3)) \\
    & \stackrel{\text{b.}}{\iff}
    (p_1 \land \neg p_3) \lor (p_2 \land \neg p_3) \lor \neg p_1 \lor \neg p_2 \lor p_3 \\
    & \stackrel{!}{\iff} \bot
  \end{align*}

  Dann sind aber alle Konjunktionen $\iff \bot$.
  Insbesondere, sind die letzten drei $\iff \bot$, ihre Negationen also $\iff \top$.
  Dann wären aber auch die ersten beiden $\iff \top$.
  Widerspruch!

  \item

  \begin{align*}
    & (p_1 \to (p_2 \to p_3)) \to ((p_1 \land p_2) \to p_3) \\
    & \iff
    \neg (p_1 \to (p_2 \to p_3)) \lor ((p_1 \land p_2) \to p_3) \\
    & \iff
    (p_1 \land \neg (p_2 \to p_3)) \lor \neg (p_1 \land p_2) \lor p_3 \\
    & \iff
    (p_1 \land p_2 \land \neg p_3) \lor \neg p_1 \lor \neg p_2 \lor p_3 \\
    & \stackrel{!}{\iff} \bot
  \end{align*}

  Dann sind aber alle Disjunktionen $\iff \bot$.
  Insbesondere, sind die letzten drei $\iff \bot$, ihre Negationen also $\iff \top$.
  Dann wäre aber auch die erste $\iff \top$.
  Widerspruch!

  \item

  \begin{align*}
    & (p_1 \to p_2) \lor (p_2 \to p_1) \\
    & \iff
    \neg p_1 \lor p_2 \lor \neg p_2 \lor p_1 \\
    & \iff
    (p_1 \lor \neg p_1) \lor (p_2 \lor \neg p_2)
  \end{align*}

  Offensichtlich liegt eine Tautologie vor.

\end{enumerate}

\end{solution}

% -------------------------------------------------------------------------------- %

\begin{solution}

\phantom{}

\begin{enumerate}[label = \alph*.]

  \item Betrachte die Negation:

  \begin{align*}
    \neg ((p_1 \to p_2) \to (p_3 \to p_4) \to ((p_1 \lor p_3) \to (p_2 \lor p_4))) &\iff
    (p_1 \to p_2) \land \neg ((p_3 \to p_4) \to ((p_1 \lor p_3) \to (p_2 \lor p_4))) \\
    &\iff (p_1 \to p_2) \land (p_3 \to p_4) \land \neg ((p_1 \lor p_3) \to (p_2 \lor p_4)) \\
    &\iff (p_1 \to p_2) \land (p_3 \to p_4) \land (p_1 \lor p_3) \land \neg (p_2 \lor p_4) \\
    &\iff (\neg p_1 \lor p_2) \land (\neg p_3 \lor p_4) \land (p_1 \lor p_3) \land (\neg p_2 \land \neg p_4) \\
    &\iff (\neg p_2 \land \neg p_4) \land (\neg p_1 \lor p_2) \land (\neg p_3 \lor p_4) \land (p_1 \lor p_3)
  \end{align*}

  Aus der ersten und der zweiten Konjuktion erhalten wir $(\neg p_1 \land \neg p_2 \land \neg \neg p_4)$.
  Mit der dritten erhalten wir auch $(\neg p_4)$ und damit kann $(p_1 \lor p_3)$ nicht mehr erfüllt werden.
  Also ist die Negation der Aussage eine Kontradiktion und die originale Aussage eine Tautologie.

  \item Definiere $A := (p_1 \lor p_2) \to p_3, \quad B := p_2 \to p_3, \quad C:= p_1 \to p_3$. \\

  \begin{tabular}{|c|c|c|c|c|c|c|c|}
    \hline
    $p_1$ & $p_2$ & $p_3$ & $A$ & $B$ & $B \to A$
    & $C$ & $C \to B \to A$\\
    \hline
    $1$ & $1$ & $1$ & $1$ & $1$ & $1$ & $1$ & $1$\\
    \hline
    $1$ & $1$ & $0$ & $0$ & $0$ & $1$ & $0$ & $1$\\
    \hline
    $1$ & $0$ & $1$ & $1$ & $1$ & $1$ & $1$ & $1$\\
    \hline
    $1$ & $0$ & $0$ & $0$ & $1$ & $0$ & $0$ & $1$\\
    \hline
    $0$ & $1$ & $1$ & $1$ & $1$ & $1$ & $1$ & $1$\\
    \hline
    $0$ & $0$ & $1$ & $1$ & $1$ & $1$ & $1$ & $1$\\
    \hline
    $0$ & $1$ & $0$ & $0$ & $0$ & $1$ & $1$ & $1$\\
    \hline
    $0$ & $0$ & $0$ & $1$ & $1$ & $1$ & $1$ & $1$\\
    \hline
  \end{tabular} \\

  Es liegt also eine Tautologie vor.

  \item Definiere $A := (p_1 \land p_2) \to p_3, \quad B := p_2 \to p_3, \quad C:= p_1 \to p_3$. \\

  \begin{tabular}{|c|c|c|c|c|c|c|c|}
    \hline
    $p_1$ & $p_2$ & $p_3$ & $A$ & $B$ & $B \to A$
    & $C$ & $C \to B \to A$\\
    \hline
    $1$ & $1$ & $1$ & $1$ & $1$ & $1$ & $1$ & $1$\\
    \hline
    $1$ & $1$ & $0$ & $0$ & $0$ & $1$ & $0$ & $0$\\
    \hline
    $1$ & $0$ & $1$ & $1$ & $1$ & $1$ & $1$ & $1$\\
    \hline
    $1$ & $0$ & $0$ & $1$ & $1$ & $1$ & $0$ & $0$\\
    \hline
    $0$ & $1$ & $1$ & $1$ & $1$ & $1$ & $1$ & $1$\\
    \hline
    $0$ & $0$ & $1$ & $1$ & $1$ & $1$ & $1$ & $1$\\
    \hline
    $0$ & $1$ & $0$ & $1$ & $0$ & $0$ & $1$ & $1$\\
    \hline
    $0$ & $0$ & $0$ & $1$ & $1$ & $1$ & $1$ & $1$\\
    \hline
  \end{tabular} \\

  Es liegt also keine Tautologie vor.

  \item Definiere $A:= (p_1 \land p_2) \to p_3, \quad B := p_1 \to p_2 \to p_3$. \\

  \begin{tabular}{|c|c|c|c|c|c|}
    \hline
    $p_1$ & $p_2$ & $p_3$ & $A$ & $B$ & $B \to A$\\
    \hline
    $1$ & $1$ & $1$ & $1$ & $1$ & $1$\\
    \hline
    $1$ & $1$ & $0$ & $0$ & $0$ & $1$\\
    \hline
    $1$ & $0$ & $1$ & $1$ & $1$ & $1$\\
    \hline
    $1$ & $0$ & $0$ & $1$ & $1$ & $1$\\
    \hline
    $0$ & $1$ & $1$ & $1$ & $1$ & $1$\\
    \hline
    $0$ & $0$ & $1$ & $1$ & $1$ & $1$\\
    \hline
    $0$ & $1$ & $0$ & $1$ & $1$ & $1$\\
    \hline
    $0$ & $0$ & $0$ & $1$ & $1$ & $1$\\
    \hline
  \end{tabular} \\

  Es liegt also eine Tautologie vor.

  \item

  \begin{align*}
    (p_1 \to p_2) \lor (p_2 \to p_1) \iff (\neg p_1 \lor p_2) \lor (\neg p_2 \lor p_1) \iff (\neg p_1 \lor p_1) \lor (\neg p_2 \lor p_2).
  \end{align*}

  und es liegt eine Tautologie vor.

\end{enumerate}

\end{solution}

% -------------------------------------------------------------------------------- %
