% --------------------------------------------------------------------------------

\begin{exercise}[2]

Von der Eigenschaft $E$ wissen wir bereits, dass sie auf alle Singletons (= einelementige Mengen) zutrifft.
Nehmen wir an, dass $E$ immer dann auf eine Menge $A \cup \Bbraces{b}$ zutrifft, wenn $E$ auf $A$ zutrifft (und $b$ beliebig ist).
Können wir daraus schließen,

\begin{itemize}
    \item ... dass $E$ für alle endlichen nichtleeren Mengen gilt?
    \item ... dass $E$ für alle nichtleeren Mengen gilt?
    \item ... dass $E$ für alle höchstens abzählbaren nichtleeren Mengen gilt?
\end{itemize}

\end{exercise}

% --------------------------------------------------------------------------------

\begin{solution}

ToDo!

\end{solution}

% --------------------------------------------------------------------------------
