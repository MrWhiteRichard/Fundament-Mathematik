% --------------------------------------------------------------------------------

\begin{exercise}[2]

Von der Eigenschaft $E$ wissen wir bereits, dass sie auf alle Singletons (= einelementige Mengen) zutrifft.
Nehmen wir an, dass $E$ immer dann auf eine Menge $A \cup \Bbraces{b}$ zutrifft, wenn $E$ auf $A$ zutrifft (und $b$ beliebig ist).
Können wir daraus schließen,

\begin{itemize}
    \item ... dass $E$ für alle endlichen nichtleeren Mengen gilt?
    \item ... dass $E$ für alle nichtleeren Mengen gilt?
    \item ... dass $E$ für alle höchstens abzählbaren nichtleeren Mengen gilt?
\end{itemize}

\end{exercise}

% --------------------------------------------------------------------------------

\begin{solution}
\phantom{}
\begin{itemize}
    \item Ja! Vollständige Induktion nach der Mächtigkeit der Menge.
    Unsere Induktionsbehauptung lautet: Für alle Mengen $B$ mit $|B| = n$ gilt
    \begin{align*}
      E(B) \implies \forall b: E(B \cup \{b\})
    \end{align*}
    Den Induktionsanfang für $n = 1$ erhalten wir aus der Voraussetzung.
    Gelte die Eigenschaft nun für alle Mengen $A$ mit $|A| = n$ und sei $B$
    mit $|B| = n + 1$ beliebig. Wähle ein beliebiges $x_0  \in B$. Dann gilt
    \begin{align*}
      B = \left(B\setminus\{x_0\}\right) \cup \{x_0\}
    \end{align*}
    und aufgrund $|B\setminus\{x_0\}| = n$ gilt nach Induktionsvoraussetzung $E(B)$
    \item Nein! Gegenbeispiel: $E(A)$ sei die Eigenschaft $|A| < \infty$.
    Klarerweise erfüllen alle Singletons $E$.
    Gelte nun $E(A)$, also $|A| \leq \infty$. Also existiert ein $n \in \N$
    mit $|A| = n$ und somit folgt für alle
    \begin{align*}
      b: |A \cup \{b\}| \leq n + 1
    \end{align*} und daher gilt auch $E(A \cup \{b\})$. \\
    Aber bereits abzählbar unendliche Mengen erfüllen die Eigenschaft nicht mehr.
\end{itemize}

\end{solution}

% --------------------------------------------------------------------------------
