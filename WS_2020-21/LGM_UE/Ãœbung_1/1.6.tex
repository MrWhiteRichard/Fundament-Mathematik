% -------------------------------------------------------------------------------- %

\begin{exercise}[19]

Zeigen Sie:
$(p \land q) \lor (\neg p \land \neg q) \Leftrightarrow (p \to q) \land (q \to p)$.

\end{exercise}

% -------------------------------------------------------------------------------- %

\begin{solution}

Wir rechnen die Formel nach.
Dazu benutzen wir zweimal das Distributiv-Gesetz.

\begin{align*}
    \text{rhs}~
    & \iff
    (\neg p \lor q) \land (\neg q \lor p) \\
    & \iff
    (\neg p \land (\neg q \lor p)) \lor (q \land (\neg q \lor p)) \\
    & \iff
    ((\neg p \land \neg q) \lor (\neg p \land p)) \lor ((q \land \neg q) \lor (q \land p)) \\
    & \iff
    ~\text{lhs}
\end{align*}

\end{solution}

% -------------------------------------------------------------------------------- %

\begin{solution}

Die folgende Wahrheitstafel zeigt, dass die \enquote{rhs} und \enquote{lhs} äquivalent sind. \\

\begin{tabular}{|c|c|c|c|c|c|c|}
    \hline
    $p$ & $q$ & $p \land q$ & $\neg p \land \neg q$ & $(p \land q) \lor (\neg p \land \neg q)$ & $(p \to q) \land (q \to p)$ \\
    \hline
    $1$ & $1$ & $1$ & $0$ & $1$ & $1$ \\
    \hline
    $1$ & $0$ & $0$ & $0$ & $0$ & $0$ \\
    \hline
    $0$ & $1$ & $0$ & $0$ & $0$ & $0$ \\
    \hline
    $0$ & $0$ & $0$ & $1$ & $1$ & $1$ \\
    \hline
\end{tabular}

\end{solution}

% -------------------------------------------------------------------------------- %
