% --------------------------------------------------------------------------------

\begin{exercise}[18]

Zeigen Sie:

\begin{enumerate}[label = \alph*]
    \item $\neg (p_1 \land p_2) \Leftrightarrow \neg p_1 \lor \neg p_2$.
    \item Für alle Formeln $A$ und $B$ gilt $\neg (A \land B) \Leftrightarrow \neg A \lor \neg B$.
\end{enumerate}

\end{exercise}

% --------------------------------------------------------------------------------

\begin{solution}

\phantom{}

\begin{enumerate}[label = \alph*]

    \item \phantom{} \\

    \begin{tabular}{|c|c|c|c|c|}
        \hline
        $p_1$ & $p_2$ & $\neg (p_1 \land p_2)$ & $\neg p_1 \lor \neg p_2$ & $\neg (p_1 \land p_2) \leftrightarrow \neg p_1 \lor \neg p_2$ \\
        \hline
        $1$ & $1$ & $0$ & $0$ & $1$\\
        \hline
        $1$ & $0$ & $1$ & $1$ & $1$\\
        \hline
        $0$ & $1$ & $1$ & $1$ & $1$\\
        \hline
        $0$ & $0$ & $1$ & $1$ & $1$\\
        \hline
    \end{tabular} \\

    Wir wissen also, dass die letzte Aussage eine Tautologie ist.
    Nach Satz II.1.9 aus dem Skriptum gilt die zu zeigende Äquivalenz.

    \item \phantom{} \\
    
    \begin{tabular}{|c|c|c|c|c|}
        \hline
        $A$ & $B$ & $\neg (A \land B)$ & $\neg A \lor \neg B$ & $\neg (A \land B) \Leftrightarrow \neg A \lor \neg B$ \\
        \hline
        $1$ & $1$ & $0$ & $0$ & $1$\\
        \hline
        $1$ & $0$ & $1$ & $1$ & $1$\\
        \hline
        $0$ & $1$ & $1$ & $1$ & $1$\\
        \hline
        $0$ & $0$ & $1$ & $1$ & $1$\\
        \hline
    \end{tabular}        

\end{enumerate}

\end{solution}

% --------------------------------------------------------------------------------

\begin{solution}

\textbf{Eine weitere Lösung für (a)} \\
Eine (vielleicht) etwas übersichtlichere Wahrheitstafel.

\begin{align*}
    \begin{array}{cccccccccc}
        \neg & (p_1 & \land & p_2) & \iff & \neg & p_1 & \lor & \neg & p_2 \\
        \hline
        2 & 0 & 1 & 0 & 3 & 1 & 0 & 2 & 1 & 0 \\
        \hline
        F & W & W & W & W & F & & F & F & \\
        W & W & F & F & W & F & & W & W & \\
        W & F & F & W & W & W & & W & F & \\
        W & F & F & F & W & W & & W & W &
    \end{array}
\end{align*}

\end{solution}

% --------------------------------------------------------------------------------

\begin{solution}

\textbf{Eine weitere Lösung für (b)} \\
Diese orientiert sich stark an Abschnitt II.1.C im Skriptum.
Wir wählen zwei beliebige aussagenlogische Formeln $A$ und $B$. Sei nun $V$ eine Variablenmenge so, dass $A \in \mathcal{F}(V)$ und $B \in \mathcal{F}(V)$ und $\{p_1, p_2\} \subseteq V$ und $p_1 \neq p_2$. Hier ist $\mathcal{F}(V)$ die Menge aller Formeln über der Variablenmenge $V$.
Wir definieren

\begin{align*}
    g: V \to \mathcal{F}(V): x \mapsto
    \begin{cases}
        A & , x = p_1 \\
        B & , x \neq p_1
    \end{cases}.
\end{align*}

Nach Bemerkung II.1.15 im Skriptum lässt sich $g$ eindeutig zu einem Formelhomomorphismus $f:\mathcal{F}(V) \to \mathcal{F}(V)$ fortsetzen. Aus Punkt (a) wissen wir, dass $\varphi := \neg (p_1 \land p_2) \leftrightarrow \neg p_1 \lor \neg p_2$ eine Tautologie ist, mit Satz II.1.16 ist auch 

\begin{align*}
    f(\varphi) = \neg (A \land B) \leftrightarrow \neg A \lor \neg B
\end{align*}

eine Tautologie und mit Satz II.1.9 schließlich gilt die zu zeigende Äquivalenz.

\end{solution}

% --------------------------------------------------------------------------------
