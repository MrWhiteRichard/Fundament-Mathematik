% --------------------------------------------------------------------------------

\begin{exercise}[22]

Sei $n \geq 2$.
Wieviele Belegungen (der Variablen $p_1,\dots,p_n$) erfüllen die Formel
\begin{align*}
  A := (p_1 \to p_2) \land (p_2 \to p_3) \land \dots \land (p_{n-1} \to p_n)?
\end{align*}

\end{exercise}

% --------------------------------------------------------------------------------

\begin{solution}

	Sei $b$ eine Belegung und wähle den kleinsten Index $i$, sodass $b(p_i) = 1$
(falls so ein Index nicht existiert, setze $i := n + 1$).
Soll $b$ nun die Formel $A$ erfüllen muss jedenfalls für $j > i: b(p_j) = 1$ gelten (Induktion).
Damit gibt es maximal $n+1$ Belegungen, die die Formel erfüllen können.
Definiere also

\begin{align*}
  b_i: p_j \mapsto \begin{cases}
    0 & j < i \\
    1 & j \geq i
  \end{cases}, i = 1,\dots,n+1.
\end{align*}

Fall 1: $j < i:$

\begin{align*}
  b_i(p_{j-1}) = 0, \quad b_i(p_j) = 0 \implies \hat{b_i}(p_{j-1} \to p_j) = 1
\end{align*}

Fall 2: $j = i$

\begin{align*}
  b_i(p_{j-1}) = 0, \quad b_i(p_j) = 1 \implies \hat{b_i}(p_{j-1} \to p_j) = 1
\end{align*}

Fall 3: $j > i:$

\begin{align*}
  b_i(p_{j-1}) = 1, \quad b_i(p_j) = 1 \implies \hat{b_i}(p_{j-1} \to p_j) = 1
\end{align*}

Damit erfüllen alle $b_i$ tatsächlich die Formel $A$ und wir erhalten insgesamt $n+1$ mögliche Belegungen.

\end{solution}

% --------------------------------------------------------------------------------

\begin{solution}

\begin{align*}
	A_n & := (p_1 \to p_2) \land (p_2 \to p_3) \land \dots \land (p_{n-1} \to p_n), \\
	W_n & := \Bbraces{b \mid b ~\text{ist Belegung und}~ \hat{b}(A_n) = 1}, \\
	c_n & := \vbraces{W_n}
\end{align*}

Wir zeigen mit Induktion nach $n$, dass $c_n = n + 1$, für $n \geq 2$ gilt.
Der Induktionsanfang ist trivial.
Für den Induktionsschritt sei $n \geq 2$ und $c_n = n + 1$.
Für $b \in W_{n + 1}$ gibt es zwei Fälle.

\begin{enumerate}[label = Fall \arabic*:]

	\item $b(p_{n + 1}) = 1$.
	Unabhängig von $b(p_n)$, ist $\hat{b}(p_n \rightarrow p_{n+1}) = 1$.
	Damit, ist aber $A_{n+1} \iff A_n$.
	Insgesamt erhalten wir durch die Induktionsvoraussetzung $c_n = n + 1$ Belegungen für welche die Formel $A_{n+1}$ wahr ist.

	\item $b(p_{n + 1}) = 0$.
	Damit $\hat{b}(A_{n+1}) = 1$, muss $\Forall k = 1, \ldots, n: b(p_k) = 0$.
	Sonst ist für das größte solche $k$:
	
	\begin{align*}
		b(p_k) = 1, b(p_{k+1}) = 0
		\implies
		\hat{b}(p_k \to p_{k+1}) = 0
		\implies
		\hat{b}(A_{n+1}) = 0.
	\end{align*}

	Umgekehrt gilt für die triviale Belegung $b = 0$ auch $\hat{b}(A_{n+1}) = 1$.
	Also, erhalten wir in dem Fall noch $1$ Belegung (die triviale) für welche die Formel wahr ist.

\end{enumerate}

Summieren wir auf so ergibt sich $c_{n + 1} = c_n + 1 = n + 2$.

\end{solution}

% --------------------------------------------------------------------------------
