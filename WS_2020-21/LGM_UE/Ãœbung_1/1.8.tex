% --------------------------------------------------------------------------------

\begin{exercise}[22]

Sei $n \geq 2$.
Wieviele Belegungen (der Variablen $p_1,\dots,p_n$) erfüllen die Formel
\begin{align*}
  A := (p_1 \to p_2) \land (p_2 \to p_3) \land \dots \land (p_{n-1} \to p_n)?
\end{align*}

\end{exercise}

% --------------------------------------------------------------------------------

\begin{solution}
Sei $b$ eine Belegung und wähle den kleinsten Index $i$, sodass $b(p_i) = 1$
(falls so ein Index nicht existiert, setze $i := n + 1$).
Soll $b$ nun die Formel $A$ erfüllen muss jedenfalls für $j > i: b(p_j) = 1$ gelten (Induktion).
Damit gibt es maximal $n+1$ Belegungen, die die Formel erfüllen können.
Definiere also
\begin{align*}
  b_i: p_j \mapsto \begin{cases}
    0 & j < i \\
    1 & j \geq i
  \end{cases}, i = 1,\dots,n+1.
\end{align*}
Fall 1: $j < i:$
\begin{align*}
  b_i(p_{j-1}) = 0, \quad b_i(p_j) = 0 \implies \hat{b_i}(p_{j-1} \to p_j) = 1
\end{align*}
Fall 2: $j = i$
\begin{align*}
  b_i(p_{j-1}) = 0, \quad b_i(p_j) = 1 \implies \hat{b_i}(p_{j-1} \to p_j) = 1
\end{align*}
Fall 3: $j > i:$
\begin{align*}
  b_i(p_{j-1}) = 1, \quad b_i(p_j) = 1 \implies \hat{b_i}(p_{j-1} \to p_j) = 1
\end{align*}
Damit erfüllen alle $b_i$ tatsächlich die Formel $A$ und wir erhalten insgesamt $n+1$ mögliche Belegungen.
\end{solution}

\textbf{zweite Lösung mit Induktion:}

\begin{solution}
	Wir definieren 
	\begin{align*}
	A_n := (p_1 \to p_2) \land (p_2 \to p_3) \land \dots \land (p_{n-1} \to p_n) \quad \text{und} \quad W_n := \{b \mid b \text{ ist Belegung und } \hat{b}(A_n) = 1¸\}
	\end{align*}
	sowie $c_n := \vbraces{W_n}$. Wir Behaupten, dass $c_n = n + 1$ gilt. Da $c_2 = 3$ gilt stimmt die Aussage für $n = 2$. Sei nun also $n \geq 2$ und $c_n = n + 1$. Wir wählen ein beliebiges $b \in W_{n + 1}$ und betrachten zwei Fälle.
	\begin{enumerate}[label = Fall \arabic*:]
		\item $b(p_{n + 1}) = 1$. Dann gilt mit Sicherheit, also unabhängig vom Wert $b(p_n)$, die Gleichheit $b(p_n \rightarrow p_{n+1}) = 1$. Wegen $\hat{b}(A_{n+1}) = 1$ ist außerdem $\hat{b}(A_{n}) = 1$. Umgekehrt gilt für jede Belegung $b$ mit $b(p_{n+1}) = 1$ und $\hat{b}(A_n) = 1$ auch $\hat{b}(A_{n+1}) = 1$. Insgesamt erhalten wir so $c_n$ Belegungen für welche die Formel wahr ist.
		\item $b(p_{n + 1}) = 1$. Dann muss schon für alle $k$ die Gleichheit $b(p_k) = 0$ gelten. Umgekehrt gilt für $b = 0$ auch $\hat{b}(A_{n + 1}) = 1$ also erhalten wir in dem Fall noch $1$ Belegung für welche die Formel wahr ist. 
	\end{enumerate}
	Summieren wir auf so ergibt sich $c_{n + 1} = c_n + 1 = n + 2$.
\end{solution}

% --------------------------------------------------------------------------------
