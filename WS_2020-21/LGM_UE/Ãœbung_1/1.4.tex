% -------------------------------------------------------------------------------- %

\begin{exercise}[16]

Geben Sie alle zweistelligen Operationen auf der $2$-elementigen Menge $M := \Bbraces{\texttt{wahr}, \texttt{falsch}}$ an (das heißt: alle Funktionen $f: M \times M \to M$, und finden Sie treffende Namen für jede dieser Abbildungen.
(Die Abbildung, die dem Paar $(\texttt{wahr}, \texttt{wahr})$ den Wert \texttt{wahr} zuordnet, den drei anderen Paaren der Wert \texttt{falsch}, könnte man zum Beispiel \enquote{Konjunktion}, oder \enquote{und-Verknüpfung}, oder \enquote{beide}, oder \enquote{Serienschaltung} nennen.)

\end{exercise}

% -------------------------------------------------------------------------------- %

\begin{solution}
\phantom{}
\begin{itemize}
  \item $f_1$:\quad\begin{tabular}{|c||c|c|}
\hline
 & wahr & falsch \\
\hline
\hline
wahr & wahr & wahr \\
\hline
falsch & wahr & wahr \\
\hline
\end{tabular} nennen wir Wahr.
  \item $f_2$:\quad\begin{tabular}{|c||c|c|}
\hline
& wahr & falsch \\
\hline
\hline
wahr & wahr & wahr \\
\hline
falsch & wahr & falsch \\
\hline
\end{tabular} nennen wir Disjunktion
\item $f_3$:\quad\begin{tabular}{|c||c|c|}
\hline
& wahr & falsch \\
\hline
\hline
wahr & wahr & wahr \\
\hline
falsch & falsch & wahr \\
\hline
\end{tabular} und
$f_4$:\quad\begin{tabular}{|c||c|c|}
\hline
& wahr & falsch \\
\hline
\hline
wahr & wahr & falsch \\
\hline
falsch & wahr & wahr \\
\hline
\end{tabular}
nennen wir Implikation.
\item $f_5$:\quad\begin{tabular}{|c||c|c|}
\hline
& wahr & falsch \\
\hline
\hline
wahr & wahr & wahr \\
\hline
falsch & falsch & falsch \\
\hline
\end{tabular} und
$f_6$:\quad\begin{tabular}{|c||c|c|}
\hline
& wahr & falsch \\
\hline
\hline
wahr & wahr & falsch \\
\hline
falsch & wahr & falsch\\
\hline
\end{tabular} nennen wir Projektion.
\item
$f_7$:\quad\begin{tabular}{|c||c|c|}
\hline
& wahr & falsch \\
\hline
\hline
wahr & falsch & wahr \\
\hline
falsch & falsch & wahr\\
\hline
\end{tabular} und
$f_8$:\quad\begin{tabular}{|c||c|c|}
\hline
& wahr & falsch \\
\hline
\hline
wahr & falsch & falsch \\
\hline
falsch & wahr & wahr\\
\hline
\end{tabular} nennen wir Negation der Projektion.
 \item
 $f_9$:\quad\begin{tabular}{|c||c|c|}
 \hline
 & wahr & falsch \\
 \hline
 \hline
 wahr & wahr & falsch \\
 \hline
 falsch & falsch & wahr\\
 \hline
 \end{tabular} nennen wir Äquivalenz.
 \item
 $f_{10}$:\quad\begin{tabular}{|c||c|c|}
 \hline
 & wahr & falsch \\
 \hline
 \hline
 wahr & falsch & wahr \\
 \hline
 falsch & wahr & falsch\\
 \hline
 \end{tabular} nennen wir Nicht-Äquivalenz.
 \item
 $f_{11}$:\quad\begin{tabular}{|c||c|c|}
 \hline
 & wahr & falsch \\
 \hline
 \hline
 wahr & wahr & falsch \\
 \hline
 falsch & falsch & falsch\\
 \hline
 \end{tabular} nennen wir Konjunktion.
 \item
 $f_{12}$:\quad\begin{tabular}{|c||c|c|}
 \hline
 & wahr & falsch \\
 \hline
 \hline
 wahr & falsch & falsch \\
 \hline
 falsch & falsch & wahr\\
 \hline
 \end{tabular} nennen wir Nicht-Disjunktion.
 \item
 $f_{13}$:\quad\begin{tabular}{|c||c|c|}
 \hline
 & wahr & falsch \\
 \hline
 \hline
 wahr & falsch & wahr \\
 \hline
 falsch & wahr & wahr\\
 \hline
 \end{tabular} nennen wir Nicht-Konjunktion.
 \item
 $f_{14}$:\quad\begin{tabular}{|c||c|c|}
 \hline
 & wahr & falsch \\
 \hline
 \hline
 wahr & falsch & wahr \\
 \hline
 falsch & falsch & falsch\\
 \hline
 \end{tabular} und
 $f_{15}$:\quad\begin{tabular}{|c||c|c|}
 \hline
 & wahr & falsch \\
 \hline
 \hline
 wahr & falsch & falsch \\
 \hline
 falsch & wahr & falsch\\
 \hline
 \end{tabular} nennen wir Nicht-Implikation.
  \item
  $f_{16}$:\quad\begin{tabular}{|c||c|c|}
  \hline
  & wahr & falsch \\
  \hline
  \hline
  wahr & falsch & falsch \\
  \hline
  falsch & falsch & falsch\\
  \hline
  \end{tabular} nennen wir Falsch.
\end{itemize}

\end{solution}

% -------------------------------------------------------------------------------- %
