% --------------------------------------------------------------------------------

\begin{exercise}[24]

Sei $n$ riesig, $k \leq 2^n$. Geben Sie eine Formel in den Variablen $p_1,\dots,p_n$ an,
die von genau $k$ Belegungen erfüllt wird. Versuchen Sie, eine möglichst kleine Formel zu finden.

\end{exercise}

% --------------------------------------------------------------------------------

\begin{solution}

Wir stellen zuerst $k \in \{0,\dots,2^n-1\}$ binär dar:
\begin{align*}
  k = \sum_{i=0}^{n-1}a_i2^i, \quad a_i \in \{0,1\}.
\end{align*}
Als nächstes definieren wir für $i \in \{0,\dots,n-1\}$ disjunkte Mengen $B_i$
von Belegungen mit $|B_i| = 2^i$ wie folgt:
\begin{align*}
  B_{n-1} &:= \{b: b(p_1) = 0\} \\
  B_{n-2} &:= \{b: b(p_1) = 1, b(p_2) = 0\} \\
  &\vdots \\
  B_{1} &:= \{b: b(p_1) = 1, \dots, b(p_{n-2}) = 1, b(p_{n-1}) = 0\} \\
  B_{0} &:= \{b: b(p_1) = 1, \dots, b(p_{n-1}) = 1, b(p_{n}) = 0\}.
\end{align*}
Jetzt müssen wir noch Formeln $A_i$ finden, sodass die Menge aller Belegungen die $A_i$
erfüllen genau $B_i$ ist. Das funktioniert so:
\begin{align*}
  A_{n-1} &:= \neg p_1 \\
  A_{n-2} &:= p_1 \land \neg p_2 \\
  &\vdots \\
  A_{1} &:= p_1 \land \dots \land p_{n-2} \land \neg p_{n-1}\\
  A_{0} &:= p_1 \land \dots \land p_{n-1} \land \neg p_{n}\\
\end{align*}
Die Disjunktion aller $A_i$ liefert dann schließlich eine Formel, die von der
Vereinigung aller $B_i$ erfüllt wird.
Da die Mengen $B_i$ paarweise disjunkt sind, ist die Mächtigkeit von der Vereinigung
gleich der Summe der einzelnen Mächtigkeiten.

Für vorgegebenes $k$ betrachten wir also die Disjunktion aller $A_i$ mit $i \in \{i \in \{0,\dots,n-1\}: a_i \neq 0 \}$. \\
Wir verwenden also maximal $\sum_{i=0}^{n-1}2(n-i) + n - 1 = n^2 + 1 = O(n^2)$ Symbole. Es geht also anscheinend noch besser.

Zur Vollständigkeit halber sei noch gesagt, dass für $k = 2^n$ einfach die Formel $T$ gewählt werden kann.
\end{solution}

% --------------------------------------------------------------------------------
