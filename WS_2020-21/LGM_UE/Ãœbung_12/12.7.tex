% --------------------------------------------------------------------------------

\begin{exercise}[270]

Sei $g: M \to M$ injektiv, aber nicht surjektiv. Geben Sie (in Abhängigkeit von $g$)
explizit eine injektive Abbildung $f: \N \to M$ an.
(Genauer: Geben Sie eine explizite Familie $(f_a: a \in I)$ von solchen Abbildungen
an, mit $I \neq \emptyset$.)

\end{exercise}

% --------------------------------------------------------------------------------

\begin{solution}

$M \setminus g[M] \neq \emptyset$. Wir finden also auch ohne Auswahlaxiom ein $m \in M \setminus g[M]$. Nun definieren wir die Funktion
\begin{align*}
	f: \N \to M: n \mapsto
	\begin{cases}
		m & \text{falls } n = 0 \\
		g(f(n - 1)) & \text{falls } n > 0 
	\end{cases}.
\end{align*}
Wir Behaupten nun
\begin{align*}
	\forall n \forall k (f(k) = f(n) \Rightarrow k = n)
\end{align*}
Wir beweisen das induktiv.

Sei $n = 0$. Dann gilt für alle $k > 0$
\begin{align*}
	m = f(0) = f(k) = g(f(k - 1))
\end{align*}
und daraus folgt der Widerspruch $m \in g[M]$. Also gilt $k = 0$. 

Nehmen wir nun an es gilt für $n \in \N$ bereits 
\begin{align*}
	\forall k \in \N: (f(n) = f(k) \Rightarrow k = n)
\end{align*}
und es gilt $f(n + 1) = f(k)$ für ein $k \in \N$. Wie im Induktionsanfang schließen wir $k \neq 0$. Daher gilt
\begin{align*}
	g(f(n)) = f(n + 1) = g(k) = g(f(k - 1))
\end{align*}
und wegen der Injektivität von $g$ daher $f(n) = f(k - 1)$. Nach Induktionsvoraussetzung gilt $n = k - 1$ und daher $n + 1 = k$. 

\end{solution}
