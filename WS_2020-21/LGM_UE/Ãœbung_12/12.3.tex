% --------------------------------------------------------------------------------

\begin{exercise}[259]

Zeigen Sie (halbwegs formal), dass aus dem schwachen Paarmengenaxiom zusammen mit
einer geeigneten Instanz des Aussonderungsaxiom des Paarmengenaxiom folgt.

\end{exercise}

% --------------------------------------------------------------------------------

\begin{solution}

\phantom{}

\begin{algorithmic}[1]
	\State $\vdash \pbraces{z = x \rightarrow (x \in A \leftrightarrow z \in A)} \rightarrow \pbraces{z = y \rightarrow (y \in A \leftrightarrow z \in A)} \rightarrow \pbraces{z \in B \leftrightarrow \pbraces{ z \in A \land \pbraces{z = x \lor z = y}}} \rightarrow \pbraces{x \in A \land y \in A} \rightarrow \pbraces{z \in B \leftrightarrow z = x \lor z=y}$ \Comment Tautologie
	\State $\vdash z = x \rightarrow (x \in A \leftrightarrow z \in A)$ \Comment Leibniz Axiom
	\State $\vdash \pbraces{z \in B \leftrightarrow \pbraces{ z \in A \land \pbraces{z = x \lor z = y}}} \rightarrow \pbraces{x \in A \land y \in A} \rightarrow \pbraces{z \in B \leftrightarrow z = x \lor z=y}$ \Comment Modus Ponens
	\State $\vdash  \forall z \pbraces{z \in B \leftrightarrow \pbraces{ z \in A \land \pbraces{z = x \lor z = y}}} \rightarrow \pbraces{x \in A \land y \in A} \rightarrow \pbraces{z \in B \leftrightarrow z = x \lor z=y}$ \Comment schwache $\forall$-Einführung
	\State $\vdash \forall z \pbraces{z \in B \leftrightarrow \pbraces{ z \in A \land \pbraces{z = x \lor z = y}}} \rightarrow \exists B\pbraces{\pbraces{x \in A \land y \in A} \rightarrow \pbraces{z \in B \leftrightarrow z = x \lor z=y}}$ \Comment schwache $\exists$-Einführung
	\State $\vdash \forall z \pbraces{z \in B \leftrightarrow \pbraces{ z \in A \land \pbraces{z = x \lor z = y}}} \rightarrow \pbraces{x \in A \land y \in A} \rightarrow \exists B \pbraces{z \in B \leftrightarrow z = x \lor z=y}$ \Comment Pränex rückwärts (Satz IV.5.11)
	\State $\vdash \exists B \forall z \pbraces{z \in B \leftrightarrow \pbraces{ z \in A \land \pbraces{z = x \lor z = y}}} \rightarrow \pbraces{x \in A \land y \in A} \rightarrow \exists B \pbraces{z \in B \leftrightarrow z = x \lor z=y}$ \Comment starke $\exists$-Einführung
	\State $\vdash \forall x \forall y \exists B \forall z \pbraces{z \in B \leftrightarrow \pbraces{ z \in A \land \pbraces{z = x \lor z = y}}} \rightarrow \pbraces{x \in A \land y \in A} \rightarrow \exists B \pbraces{z \in B \leftrightarrow z = x \lor z=y}$ \Comment zwei Mal schwache $\forall$-Einführung
	\State $\vdash \forall x \forall y \exists B \forall z \pbraces{z \in B \leftrightarrow \pbraces{ z \in A \land \pbraces{z = x \lor z = y}}} \rightarrow \forall A\pbraces{\pbraces{x \in A \land y \in A} \rightarrow \exists B \pbraces{z \in B \leftrightarrow z = x \lor z=y}}$ \Comment starke $\forall$-Einführung
	\State $\vdash \forall x \forall y \exists B \forall z \pbraces{z \in B \leftrightarrow \pbraces{ z \in A \land \pbraces{z = x \lor z = y}}} \rightarrow \exists A \pbraces{x \in A \land y \in A} \rightarrow \exists B \pbraces{z \in B \leftrightarrow z = x \lor z=y}$ \Comment Pränex rückwärts (Satz IV.5.11)
	\State $\vdash \forall x \forall y \exists B \forall z \pbraces{z \in B \leftrightarrow \pbraces{ z \in A \land \pbraces{z = x \lor z = y}}} \rightarrow \forall x \forall y\pbraces{\exists A \pbraces{x \in A \land y \in A} \rightarrow \exists B \pbraces{z \in B \leftrightarrow z = x \lor z=y}}$ \Comment zwei Mal starke $\forall$-Einführung
	\State $\vdash \forall x \forall y \exists B \forall z \pbraces{z \in B \leftrightarrow \pbraces{ z \in A \land \pbraces{z = x \lor z = y}}} \rightarrow \forall x \forall y \exists A \pbraces{x \in A \land y \in A} \rightarrow \forall x \forall y \exists B \pbraces{z \in B \leftrightarrow z = x \lor z=y}$ \Comment Distributivitätsaxiom
\end{algorithmic}

\end{solution}
