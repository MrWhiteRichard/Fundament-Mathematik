% --------------------------------------------------------------------------------

\begin{exercise}[258]

Zeigen Sie (halbwegs formal), dass aus dem Vereinigungsmengenaxiom und dem
Paarmengenaxiom das kleine Vereinigungsmengenaxiom folgt. Geben Sie eine Struktur
an, in der das Vereinigungsmengenaxiom gilt, aber das kleine Vereinigungsmengenaxiom
verletzt ist.

\end{exercise}

% --------------------------------------------------------------------------------

\begin{solution}
$\Gamma := \{$Paarmengenaxiom, Vereinigungsmengenaxiom$\}$. \\
Paarmengenaxiom: $\forall x\, \forall y\, \exists P \, \forall z
(z \varepsilon P \leftrightarrow z = x \lor z = y)$. \\
Vereinigungsmengenaxiom:
$\forall \mathcal{A}\, \exists S\, \forall z (z \varepsilon S \leftrightarrow
\exists B (z \varepsilon B \land B \varepsilon \mathcal{A}))$. \\
Kleines Vereinigungsmengenaxiom:
$\forall A\, \forall B\, \exists S\, \forall z (z \varepsilon S \leftrightarrow
z \varepsilon A \lor z \varepsilon B)$. \\

Seien $A, B$ beliebige Mengen.
Das Paarmengenaxiom liefert uns eine Menge $C = \{A, B\}$ und das
Vereinigungsmengenaxiom liefert dann $S = \bigcup C = A \cup B$.

\begin{algorithmic}[1]
\State $\Gamma \vdash \forall x\, \forall y\, \exists P \, \forall z
(z \varepsilon P \leftrightarrow z = x \lor z = y)$
\State $\Gamma \vdash \exists P \, \forall z
(z \varepsilon P \leftrightarrow z = A \lor z = B)$
\Comment Substitutionsaxiom + 2x MP
\State $\Gamma \vdash \forall \mathcal{A}\, \exists S\, \forall z (z \varepsilon S \leftrightarrow
\exists C (z \varepsilon C \land C \varepsilon \mathcal{A}))$
\State $\Gamma \vdash \exists S\, \forall z (z \varepsilon S \leftrightarrow
\exists C (z \varepsilon C \land C \varepsilon P))$
\State ?
\State $\Gamma \vdash \exists S \forall z (z \varepsilon S \leftrightarrow z \varepsilon A \lor z \varepsilon B)$
\State $\Gamma \vdash \forall A\, \forall B\, \exists S \forall z (z \varepsilon S \leftrightarrow z \varepsilon A \lor z \varepsilon B)$

\end{algorithmic}

\begin{algorithmic}[1]
  \State $(B \varepsilon \mathcal{A} \leftrightarrow B = x \lor B = y),
  (z \varepsilon S \leftrightarrow (z \varepsilon B \land B \varepsilon \mathcal{A})),
  (B = x)
  \vdash z \varepsilon x \lor z \varepsilon y$
\end{algorithmic}

\end{solution}
