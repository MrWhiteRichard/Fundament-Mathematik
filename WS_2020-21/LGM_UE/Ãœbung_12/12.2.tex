% -------------------------------------------------------------------------------- %

\begin{exercise}[258]

Zeigen Sie (halbwegs formal), dass aus dem Vereinigungsmengenaxiom und dem
Paarmengenaxiom das kleine Vereinigungsmengenaxiom folgt. Geben Sie eine Struktur
an, in der das Vereinigungsmengenaxiom gilt, aber das kleine Vereinigungsmengenaxiom
verletzt ist.

\end{exercise}

% -------------------------------------------------------------------------------- %

\begin{solution}
$\Gamma := \{$Paarmengenaxiom, Vereinigungsmengenaxiom$\}$. \\
Paarmengenaxiom: $\forall x\, \forall y\, \exists P \, \forall z
(z \varepsilon P \leftrightarrow z = x \lor z = y)$. \\
Vereinigungsmengenaxiom:
$\forall \mathcal{A}\, \exists S\, \forall z (z \varepsilon S \leftrightarrow
\exists B (z \varepsilon B \land B \varepsilon \mathcal{A}))$. \\
Kleines Vereinigungsmengenaxiom:
$\forall A\, \forall B\, \exists S\, \forall z (z \varepsilon S \leftrightarrow
z \varepsilon A \lor z \varepsilon B)$. \\

Seien $A, B$ beliebige Mengen.
Das Paarmengenaxiom liefert uns eine Menge $C = \{A, B\}$ und das
Vereinigungsmengenaxiom liefert dann $S = \bigcup C = A \cup B$.

\begin{algorithmic}[1]
\State $\Gamma \vdash \forall x\, \forall y\, \exists P \, \forall z
(z \varepsilon P \leftrightarrow z = x \lor z = y)$
\State $\Gamma \vdash \exists P \, \forall z
(z \varepsilon P \leftrightarrow z = A \lor z = B)$
\Comment Substitutionsaxiom + 2x MP
\State $\Gamma \vdash \forall \mathcal{A}\, \exists S\, \forall z (z \varepsilon S \leftrightarrow
\exists C (z \varepsilon C \land C \varepsilon \mathcal{A}))$
\State $\Gamma \vdash \exists S\, \forall z (z \varepsilon S \leftrightarrow
\exists C (z \varepsilon C \land C \varepsilon P))$
\State ?
\State $\Gamma \vdash \exists S \forall z (z \varepsilon S \leftrightarrow z \varepsilon A \lor z \varepsilon B)$
\State $\Gamma \vdash \forall A\, \forall B\, \exists S \forall z (z \varepsilon S \leftrightarrow z \varepsilon A \lor z \varepsilon B)$

\end{algorithmic}

\begin{algorithmic}[1]
  \State $(B \varepsilon \mathcal{A} \leftrightarrow B = x \lor B = y),
  (z \varepsilon S \leftrightarrow (z \varepsilon B \land B \varepsilon \mathcal{A})),
  (B = x)
  \vdash z \varepsilon x \lor z \varepsilon y$
\end{algorithmic}

Fast die richtige Aussage:

\begin{algorithmic}[1]
	\State $\vdash \pbraces{B = C\rightarrow \pbraces{x \in B \leftrightarrow x \in C}} \rightarrow \pbraces{B = D\rightarrow \pbraces{x \in B  \leftrightarrow x \in D}} \rightarrow$
	\State $\pbraces{\pbraces{x \in S \rightarrow \pbraces{x \in B \land B \in \mathcal{A}}} \land \pbraces{\pbraces{\pbraces{x \in C \land C \in \mathcal{A}} \lor \pbraces{x \in D \land D \in \mathcal{A}}}\rightarrow x \in S}} \rightarrow$
	\State $\pbraces{\pbraces{B \in \mathcal{A} \rightarrow \pbraces{B = C \lor B = D}} \land C \in \mathcal{A} \land D \in \mathcal{A}} \rightarrow \pbraces{x \in S \leftrightarrow \pbraces{x \in C \lor x \in D}}$ \Comment{Tautologie}
	\State $\pbraces{\pbraces{x \in S \rightarrow \pbraces{x \in B \land B \in \mathcal{A}}} \land \pbraces{\pbraces{\pbraces{x \in C \land C \in \mathcal{A}} \lor \pbraces{x \in D \land D \in \mathcal{A}}}\rightarrow x \in S}} \rightarrow$
	\State $\pbraces{\pbraces{B \in \mathcal{A} \rightarrow \pbraces{B = C \lor B = D}} \land C \in \mathcal{A} \land D \in \mathcal{A}} \rightarrow \pbraces{x \in S \leftrightarrow \pbraces{x \in C \lor x \in D}}$ \Comment{Leibniz Axiome und Modus Ponens}
	\State $\pbraces{\pbraces{x \in S \rightarrow \pbraces{x \in B \land B \in \mathcal{A}}} \land \pbraces{\pbraces{\pbraces{x \in C \land C \in \mathcal{A}} \lor \pbraces{x \in D \land D \in \mathcal{A}}}\rightarrow x \in S}} \rightarrow$
	\State $\forall B \pbraces{\pbraces{B \in \mathcal{A} \rightarrow \pbraces{B = C \lor B = D}} \land C \in \mathcal{A} \land D \in \mathcal{A}} \rightarrow \pbraces{x \in S \leftrightarrow \pbraces{x \in C \lor x \in D}}$ \Comment{schwache $\forall$-Einführung}
	\State $\pbraces{\pbraces{x \in S \rightarrow \exists B \pbraces{x \in B \land B \in \mathcal{A}}} \land \pbraces{\pbraces{\pbraces{x \in C \land C \in \mathcal{A}} \lor \pbraces{x \in D \land D \in \mathcal{A}}}\rightarrow x \in S}} \rightarrow$
	\State $\forall B \pbraces{\pbraces{B \in \mathcal{A} \rightarrow \pbraces{B = C \lor B = D}} \land C \in \mathcal{A} \land D \in \mathcal{A}} \rightarrow \pbraces{x \in S \leftrightarrow \pbraces{x \in C \lor x \in D}}$ \Comment{schwache $\exists$-Einführung}
	\State $\pbraces{\pbraces{x \in S \rightarrow \exists B \pbraces{x \in B \land B \in \mathcal{A}}} \land \pbraces{\exists C\pbraces{x \in C \land C \in \mathcal{A}} \rightarrow x \in S}} \rightarrow$
	\State $\forall B \pbraces{\pbraces{B \in \mathcal{A} \rightarrow \pbraces{B = C \lor B = D}} \land C \in \mathcal{A} \land D \in \mathcal{A}} \rightarrow \pbraces{x \in S \leftrightarrow \pbraces{x \in C \lor x \in D}}$ \Comment{schwache $\forall$-Einführung und zwei Mal Pränex Rückwärts und Redundanz weglassen}
	\State $\pbraces{x \in S \leftrightarrow \exists B \pbraces{x \in B \land B \in \mathcal{A}}}  \rightarrow \forall B \pbraces{\pbraces{B \in \mathcal{A} \rightarrow \pbraces{B = C \lor B = D}} \land C \in \mathcal{A} \land D \in \mathcal{A}} \rightarrow$
	\State$ \pbraces{x \in S \leftrightarrow \pbraces{x \in C \lor x \in D}}$ \Comment{kompakter aufschreiben}
	\State $\forall x \pbraces{x \in S \leftrightarrow \exists B \pbraces{x \in B \land B \in \mathcal{A}}}  \rightarrow \forall C \forall D \exists \mathcal{A} \forall B \pbraces{\pbraces{B \in \mathcal{A} \rightarrow \pbraces{B = C \lor B = D}} \land C \in \mathcal{A} \land D \in \mathcal{A}} \rightarrow$
	\State$ \pbraces{x \in S \leftrightarrow \pbraces{x \in C \lor x \in D}}$ \Comment{schwache $\exists$- und $\forall$-Einführungen}
	\State$ \pbraces{x \in S \leftrightarrow \pbraces{x \in C \lor x \in D}}$ \Comment{kompakter aufschreiben}
	\State $\forall x \pbraces{x \in S \leftrightarrow \exists B \pbraces{x \in B \land B \in \mathcal{A}}}  \rightarrow \forall C \forall D \exists \mathcal{A} \forall B \pbraces{\pbraces{B \in \mathcal{A} \rightarrow \pbraces{B = C \lor B = D}} \land C \in \mathcal{A} \land D \in \mathcal{A}} \rightarrow$
	\State$ \exists S \forall x\pbraces{x \in S \leftrightarrow \pbraces{x \in C \lor x \in D}}$ \Comment{starke $\forall$-Einführung und schwache $\exists$-Einführung}
	\State $\forall \mathcal{A} \exists \forall x \pbraces{x \in S \leftrightarrow \exists B \pbraces{x \in B \land B \in \mathcal{A}}}  \rightarrow \forall C \forall D \exists \mathcal{A} \forall B \pbraces{\pbraces{B \in \mathcal{A} \rightarrow \pbraces{B = C \lor B = D}} \land C \in \mathcal{A} \land D \in \mathcal{A}} \rightarrow$
	\State$ \forall C \forall D\exists S \forall x\pbraces{x \in S \leftrightarrow \pbraces{x \in C \lor x \in D}}$ \Comment{restliche $\forall$- und $\exists$-Einführungen}
\end{algorithmic}

\end{solution}
