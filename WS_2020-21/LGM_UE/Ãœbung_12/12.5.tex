% --------------------------------------------------------------------------------

\begin{exercise}[268]

Zeigen Sie halbwegs formal, dass es in jeder endlichen nichtleeren partiellen
Ordnung ein maximales Element gibt. \\
\textit{Hinweis:} Definieren Sie eine geeignete Menge $M \subseteq \omega$, von
der Sie dann zeigen, dass sie induktiv ist.

\end{exercise}

% --------------------------------------------------------------------------------

\begin{solution}
\begin{align*}
  M := \{n \varepsilon \omega \mid \forall A[ (\exists f: A \to n \text{ bijektiv})
  \implies (\exists x \varepsilon A: \forall y \in A: y \geq x \implies  y = x)]\}
  \cup \{\emptyset\}
\end{align*}
Laut dem Aussonderungsaxiom existiert eine solche Menge.
Wir wollen nun zeigen, dass die Menge sogar induktiv ist:
Per Definition ist $\emptyset \varepsilon M$.
\begin{align*}
  \forall x (x \varepsilon M \rightarrow (x + 1) \varepsilon M):
\end{align*}
Sei $A$ eine Menge mit $f: A \to x + 1$ bijektiv. Betrachte die Menge
\begin{align*}
  x + 1 &= \{0,\dots,x\} \\
  A' &= \{z \varepsilon A: f(z) \neq x\} = A \setminus \{f^{-1}(x)\} \\
  f' &:= f \cap (A' \times x) = f|_{A'}
\end{align*}
$f'$ ist eine Bijektion von $A'$ nach $x$, also existiert ein maximales Element $x_0$ von $A'$.
Jetzt vergleiche $x_0$ mit $f^{-1}(x)$: \\
Fall 1: $x_0 \leq f^{-1}(x)$: Dann ist $f^{-1}(x)$ ein maximales Element von $A$ (Transitivität). \\
Fall 2: Sonst: $x_0$ ist ein maximales Element von $A$.
\end{solution}
