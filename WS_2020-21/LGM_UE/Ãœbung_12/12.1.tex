\subsection*{253/254}

% --------------------------------------------------------------------------------

\begin{exercise}[253]

Zeigen Sie halbwegs formal, dass das schwache Singletonaxiom $\forall x\, \exists S(x \varepsilon S)$
zur Formel $\forall x\, \exists S \, \forall z (z = x \rightarrow z \varepsilon S)$
äquivalent ist.

\end{exercise}

% --------------------------------------------------------------------------------

\begin{solution}
Hinrichtung:
\begin{algorithmic}[1]
    \State $x \varepsilon S, z = x \vdash z = x \rightarrow
    (x \varepsilon S \rightarrow z \varepsilon S)$
    \Comment Leibniz-Axiom
    \State $x \varepsilon S, z = x \vdash z = x$
    \State $x \varepsilon S, z = x \vdash x \varepsilon S \rightarrow z \varepsilon S$
    \Comment Modus Ponens
    \State $x \varepsilon S, z = x \vdash x \varepsilon S$
    \State $x \varepsilon S, z = x \vdash z \varepsilon S$
    \Comment Modus Ponens
    \State $x \varepsilon S \vdash z = x \rightarrow z \varepsilon S$
    \Comment Deduktionstheorem
    \State $x \varepsilon S \vdash \forall z (z = x \rightarrow z \varepsilon S)$
    \Comment Generalisierungstheorem
    \State $\vdash x \varepsilon S \rightarrow \forall z (z = x \rightarrow z \varepsilon S)$
    \Comment Deduktionstheorem
    \State $\vdash x \varepsilon S \rightarrow \exists S\, \forall z (z = x \rightarrow z \varepsilon S)$
    \Comment Schwache $\exists$-Einführung
    \State $\vdash \exists S (x \varepsilon S) \rightarrow \exists S\, \forall z (z = x \rightarrow z \varepsilon S)$
    \Comment Starke $\exists$-Einführung
    \State $\vdash \forall x \exists S (x \varepsilon S) \rightarrow \exists S\, \forall z (z = x \rightarrow z \varepsilon S)$
    \Comment Schwache $\forall$-Einführung
    \State $\vdash \forall x\, \exists S (x \varepsilon S) \rightarrow \forall x\, \exists S\, \forall z (z = x \rightarrow z \varepsilon S)$
    \Comment Starke $\forall$-Einführung
\end{algorithmic}
\pagebreak
Rückrichtung:
\begin{algorithmic}[1]
    \State $\forall z (z = x \rightarrow z \varepsilon S) \vdash
    \forall z (z = x \rightarrow z \varepsilon S)
    \rightarrow (x = x \rightarrow x \varepsilon S)$
    \Comment Substitutionsaxiom
    \State $\forall z(z = x \rightarrow z \varepsilon S) \vdash \forall z (z = x \rightarrow z \varepsilon S)$
    \State $\forall z(z = x \rightarrow z \varepsilon S) \vdash (x = x \rightarrow x \varepsilon S)$
    \Comment Modus Ponens
    \State $\forall z(z = x \rightarrow z \varepsilon S) \vdash x = x$
    \Comment Leibniz-Axiom
    \State $\forall z(z = x \rightarrow z \varepsilon S) \vdash x \varepsilon S$
    \Comment Modus Ponens
    \State $\vdash \forall z (z = x \rightarrow z \varepsilon S) \rightarrow x \varepsilon S$
    \Comment Deduktionstheorem
    \State $\vdash \forall z (z = x \rightarrow z \varepsilon S) \rightarrow \exists S (x \varepsilon S)$
    \Comment Schwache $\exists$-Einführung
    \State $\vdash \exists S\, \forall z (z = x \rightarrow z \varepsilon S) \rightarrow \exists S (x \varepsilon S)$
    \Comment Starke $\exists$-Einführung
    \State $\vdash \forall x\, \exists S\, \forall z (z = x \rightarrow z \varepsilon S) \rightarrow \exists S (x \varepsilon S)$
    \Comment Schwache $\forall$-Einführung
    \State $\vdash \forall x\, \exists S\, \forall z (z = x \rightarrow z \varepsilon S)
    \rightarrow \forall x\, \exists S (x \varepsilon S)$
    \Comment Starke $\forall$-Einführung
\end{algorithmic}
\end{solution}

% --------------------------------------------------------------------------------

\begin{exercise}[254]

Zeigen Sie (mit einem halbwegs formalen Beweis), dass das Singletonaxiom
aus dem Paarmengenaxiom folgt.

\end{exercise}

% --------------------------------------------------------------------------------

\begin{solution}

\phantom{}

\end{solution}
