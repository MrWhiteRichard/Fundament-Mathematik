% -------------------------------------------------------------------------------- %

\begin{exercise}[269]

Sei $f: \N \to M$ injektiv.
Geben Sie (in Abhängigkeit von $f$) explizit eine injektie aber nicht surjektive Funktion $g: M \to M$ an.
(Welche Instanz(en) des Aussonderungsaxioms oder Ersetzungsaxioms verwenden Sie in einem formalen Beweis dieser Aussage?)

\end{exercise}

% -------------------------------------------------------------------------------- %

\begin{solution}

Weil $f$ injektiv ist, hat es eine (injektive) Links-Inverse $f^{-1}$.
Sei $h: \N \to \N$ injektiv, aber nicht surjektiv (z.B. die Nachfolger-Funktion).

\begin{align*}
    g(m)
    :=
    \begin{cases}
        m,                           & m \in M \setminus f[\N], \\
        (f \circ h \circ f^{-1})(m), & m \in f[\N],
    \end{cases}
    \quad
    m \in M
\end{align*}

Aussonderungsaxiom: Sei $\varphi(z,p_1,\dots,p_k)$ eine Formel, $M$ eine neue Variable.
\begin{align*}
  \forall p_1\, \cdots \forall p_k \forall A\, \exists M\, \forall z (z \varepsilon M
  \leftrightarrow (\varphi(z,p_1,\dots,p_k) \land z \varepsilon A))
\end{align*}

Das Aussonderungsaxiom verwenden wir für die Wohldefiniertheit von

\begin{multline*}
    f[\N] := \Bbraces{m \in M: \Exists n \in \N: f(n) = m}, \\
    \text{d.h.}
    \quad
    \Exists f[\N] (m \in f[\N] \leftrightarrow (m \in M \land \varphi(m))),
    \quad
    \varphi(m): \Exists n (n \in \N \land f(n) = m).
\end{multline*}

\end{solution}
