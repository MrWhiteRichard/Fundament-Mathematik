\documentclass{article}

% Hier befinden sich Pakete, die wir beinahe immer benutzen ...

\usepackage[utf8]{inputenc}

% Sprach-Paket:
\usepackage[ngerman]{babel}

% damit's nicht so, wie beim Grill aussieht:
\usepackage{fullpage}

% Mathematik:
\usepackage{amsmath, amssymb, amsfonts, amsthm}
\usepackage{bbm, mathrsfs, stmaryrd}
\usepackage{mathtools, mathdots}

% Makros mit mehereren Default-Argumenten:
\usepackage{twoopt}

% Anführungszeichen (Makro \Quote{}):
\usepackage{babel}

% if's für Makros:
\usepackage{xifthen}
\usepackage{etoolbox}

% tikz ist kein Zeichenprogramm (doch!):
\usepackage{tikz}

% bessere Aufzählungen:
\usepackage{enumitem}

% (bessere) Umgebung für Bilder:
\usepackage{graphicx, subfig, float}

% Umgebung für Code:
\usepackage{listings}

% Farben:
\usepackage{xcolor}

% Umgebung für "plain text":
\usepackage{verbatim}

% Umgebung für mehrerer Spalten:
\usepackage{multicol}

% "nette" Brüche
\usepackage{nicefrac}

% Spaltentypen verschiedener Dicke
\usepackage{tabularx}
\usepackage{makecell}

% Für Vektoren
\usepackage{esvect}

% (Web-)Links
\usepackage{hyperref}

% Zitieren & Literatur-Verzeichnis
\usepackage[style = authoryear]{biblatex}
\usepackage{csquotes}

% so ähnlich wie mathbb
%\usepackage{mathds}

% Keine Ahnung, was das macht ...
\usepackage{booktabs}
\usepackage{ngerman}
\usepackage{placeins}

% special letters:

\newcommand{\N}{\mathbb{N}}
\newcommand{\Z}{\mathbb{Z}}
\newcommand{\Q}{\mathbb{Q}}
\newcommand{\R}{\mathbb{R}}
\newcommand{\C}{\mathbb{C}}
\newcommand{\K}{\mathbb{K}}
\newcommand{\T}{\mathbb{T}}
\newcommand{\E}{\mathbb{E}}
\newcommand{\V}{\mathbb{V}}
\renewcommand{\P}{\mathbb{P}}
\newcommand{\1}{\mathbbm{1}}

\newcommand  {\B}{\mathfrak{B}}
\renewcommand{\S}{\mathfrak{S}}

% quantors:

\newcommand{\Forall}{\forall \,}
\newcommand{\Exists}{\exists \,}
\newcommand{\ExistsOnlyOne}{\exists! \,}
\newcommand{\nExists}{\nexists \,}

% MISC symbols:

\newcommand{\landau}[1]
{
  {\scriptstyle \mathcal{O}}
  \pbraces{#1}
}

\newcommand{\Landau}[1]
{
  \mathcal{O}
  \pbraces{#1}
}

\newcommand{\eps}{\mathrm{eps}}

% graphics in a box:

\newcommandtwoopt
{\includegraphicsboxed}[3][][]
{
  \begin{figure}[!h]
    \begin{boxedin}
      \ifthenelse{\isempty{#2}}
      {
        \begin{center}
          \includegraphics[width = 0.75 \textwidth]{#3}
          \label{fig:#1}
        \end{center}
      }{
        \begin{center}
          \includegraphics[width = 0.75 \textwidth]{#3}
          \caption{#2}
          \label{fig:#1}
        \end{center}
      }
    \end{boxedin}
  \end{figure}
}

% braces:

\newcommand{\pbraces}[1]{{\left  ( #1 \right  )}}
\newcommand{\bbraces}[1]{{\left  [ #1 \right  ]}}
\newcommand{\Bbraces}[1]{{\left \{ #1 \right \}}}
\newcommand{\vbraces}[1]{{\left  | #1 \right  |}}
\newcommand{\Vbraces}[1]{{\left \| #1 \right \|}}
\newcommand{\abraces}[1]{{\left \langle #1 \right \rangle}}
\newcommand{\round}[1]{\bbraces{#1}}

\newcommand
{\floor}[1]
{{\left \lfloor #1 \right \rfloor}}

\newcommand
{\ceil} [1]
{{\left \lceil  #1 \right \rceil }}

% special functions:

\newcommand{\norm}  [2][]{\Vbraces{#2}_{#1}}
\newcommand{\diag}  [1]{\mathrm{diag} \: #1}
\newcommand{\dist}  [1]{\mathrm{dist} \: #1}
\newcommand{\mean}  [1]{\mathrm{mean} \: #1}
\newcommand{\erf}   [1]{\mathrm{erf} \: #1}
\newcommand{\id}    [1]{\mathrm{id} \: #1}
\newcommand{\sgn}   [1]{\mathrm{sgn} \: #1}
\newcommand{\supp}  [1]{\mathrm{supp} \: #1}
\newcommand{\arsinh}[1]{\mathrm{arsinh} \: #1}
\newcommand{\arcosh}[1]{\mathrm{arcosh} \: #1}
\newcommand{\artanh}[1]{\mathrm{artanh} \: #1}
\newcommand{\card}  [1]{\mathrm{card} \: #1}
\newcommand{\Span}  [1]{\mathrm{span} \: #1}
\newcommand{\Aut}   [1]{\mathrm{Aut} \: #1}
\newcommand{\End}   [1]{\mathrm{End} \: #1}
\newcommand{\ggT}   [1]{\mathrm{ggT} \: #1}
\newcommand{\kgV}   [1]{\mathrm{kgV} \: #1}
\newcommand{\ord}   [1]{\mathrm{ord} \: #1}
\newcommand{\grad}  [1]{\mathrm{grad} \: #1}
\newcommand{\ran}   [1]{\mathrm{ran} \: #1}
\newcommand{\graph} [1]{\mathrm{graph} \: #1}
\newcommand{\Inv}   [1]{\mathrm{Inv} \: #1}
\newcommand{\pv}    [1]{\mathrm{pv} \: #1}
\newcommand{\Mod}{\: \mathrm{mod} \:}
\newcommand{\Char}{\mathrm{char}}
\newcommand{\At}{\mathrm{At}}
\newcommand{\Ob}{\mathrm{Ob}}
\newcommand{\Hom}{\mathrm{Hom}}
\newcommand{\orthogonal}[3][]{#2 ~\bot_{#1}~ #3}
\newcommand{\Rang}{\mathrm{Rang}}

\newcommand
{\GL}[2][]
{\mathrm{GL}_{#1} \pbraces{#2}}

% fractions:

\newcommand{\Frac}[2]{\frac{1}{#1} \pbraces{#2}}
\newcommand{\nfrac}[2]{\nicefrac{#1}{#2}}

% derivatives & integrals:

\newcommandtwoopt
{\Int}[4][][]
{\int_{#1}^{#2} #3 ~\mathrm{d} #4}

\newcommandtwoopt
{\derivative}[3][][]
{
  \frac
  {\mathrm{d}^{#1} #2}
  {\mathrm{d} #3^{#1}}
}

\newcommandtwoopt
{\pderivative}[3][][]
{
  \frac
  {\partial^{#1} #2}
  {\partial #3^{#1}}
}

\newcommand
{\primeprime}
{{\prime \prime}}

\newcommand
{\primeprimeprime}
{{\prime \prime \prime}}

% Text:

\newcommand{\Quote}[1]{\glqq #1\grqq{}}
\newcommand{\Text}[1]{{\text{#1}}}
\newcommand{\fastueberall}{\text{f.ü.}}
\newcommand{\fastsicher}{\text{f.s.}}

% -------------------------------- %
% amsthm-stuff:

\theoremstyle{definition}

% numbered theorems
\newtheorem{theorem}    {Satz}   [section]
\newtheorem{lemma}      [theorem]{Lemma}
\newtheorem{corollary}  [theorem]{Korollar}
\newtheorem{proposition}[theorem]{Proposition}
\newtheorem{remark}     [theorem]{Bemerkung}
\newtheorem{definition} [theorem]{Definition}
\newtheorem{example}    [theorem]{Beispiel}

% unnumbered theorems
\newtheorem*{theorem*}    {Satz}
\newtheorem*{lemma*}      {Lemma}
\newtheorem*{corollary*}  {Korollar}
\newtheorem*{proposition*}{Proposition}
\newtheorem*{remark*}     {Bemerkung}
\newtheorem*{definition*} {Definition}
\newtheorem*{example*}    {Beispiel}

% Please define this stuff in project ("main.tex"):

% \def \lastexercisenumber {...}
% This will be 0 by default

% \setcounter{section}{...}
% This will be 0 by default
% and hence, completely ignored

\ifnum \thesection = 0
{
  \newtheorem{exercise}{Aufgabe}
}
\else
{
  \newtheorem{exercise}{Aufgabe}[section]
}
\fi

\ifdef
{\lastexercisenumber}
{\setcounter{exercise}{\lastexercisenumber}}

\newenvironment{solution}
{
  \begin{proof}[Lösung]
}{
  \end{proof}
}

\renewcommand{\proofname}{Beweis}

% -------------------------------- %
% environment zum einkasteln:

% dickere vertical lines
\newcolumntype
{x}
[1]
{
  !{
    \centering
    \arraybackslash
    \vrule
    width #1}
}

% environment selbst (the big cheese)
\newenvironment
{boxedin}
{
  \begin{tabular}
  {
    x{1 pt}
    p{\textwidth}
    x{1 pt}
  }
  \Xhline
  {2 \arrayrulewidth}
}
{
  \\
  \Xhline{2 \arrayrulewidth}
  \end{tabular}
}

% -------------------------------- %
% MISC "Ein-Deutschungen"

\renewcommand{\figurename}{Abbildung}
\renewcommand{\tablename} {Tabelle}

% -------------------------------- %


\parindent 0pt

\title
{
  Logik und Grundlagen der Mathematik \\
  \vspace{4pt}
  \normalsize
  \textit{9. Übung am 03.12.2020}
}
\author
{
  Richard Weiss
  \and
  Florian Schager
  \and
  Fabian Zehetgruber
}
\date{}

\begin{document}

\maketitle

\section*{Primitiv rekursive Funktionen}
Wir nennen eine Relation $R \subseteq \N^n$ primitiv rekursiv (genauer: \glqq
primitiv rekursive Relation\grqq\, oder \glqq primitiv rekursiv als Relation\grqq), wenn
ihre charakteristische Funktion $\chi_R$ primitiv rekursiv ist. \\
ACHTUNG: Es gibt Funktionen, die zwar nicht primitiv rekursiv sind, die aber
in diesem Sinn eine primitiv rekursive Relation sind. \\
In jeder der folgenden Aufgaben dürfen Sie die jeweils vorigen Aufgaben verwenden.


% --------------------------------------------------------------------------------

\begin{exercise}

Sei $u(x,t)$ eine beschränkte Lösung des Cauchyproblems für die Wärmeleitungsgleichung
\begin{align*}
  \begin{cases}
    u_t = a^2u_xx & \text{für } t > 0 \text{ und } x \in \R \\
    u(x,0) = \varphi(x) & \text{für } x \in \R,
  \end{cases}
\end{align*}
woebi $a > 0$ ist und $\varphi \in C(\R)$
\begin{align*}
  \lim_{x \to \infty} \varphi(x) = b, \quad \lim_{x \to -\infty} \varphi(x) = c
\end{align*}
erfüllt. Berechnen Sie den Grenzwert von $u(x,t),\ x \in \R$ für $t \to \infty$.
\end{exercise}

% --------------------------------------------------------------------------------

\begin{solution}

\phantom{}

\end{solution}

% --------------------------------------------------------------------------------

\begin{exercise}
Verallgemeinern Sie den Beweis von Theorem 5.35 des Skriptums für den allgemeinen
Fall $n \in \N$. Dazu müssen Sie im Wesentlichen folgendes zeigen:
\begin{enumerate}[label = \textbf{\alph*)}]
  \item Mit den angepassten Definitionen aus dem Skript gilt
  \begin{align}
    E_{\ell + 1} = \left(A_{\rho}^{\top} \otimes I\right)E_{\ell} + F_{\ell},
  \end{align}
  wobei $I \in \R^{n \times n}$ die Einheitsmatrix und $A \otimes B$ das Kroneckerprodukt
  zweier Matrizen $A \in \R^{k \times k}$ und $B \in \R^{n \times n}$ ist, also
  \begin{align}
    A \otimes B := \begin{pmatrix}
      A_{11}B & \hdots & A_{1k}B \\
      \vdots & & \vdots \\
      A_{k1}B & \hdots & A_{kk}B
    \end{pmatrix}
    \in \R^{kn \times kn}.
  \end{align}
  \item Aus der Wurzelbedingung folgt
  \begin{align}
    \sup_{k \in \N_0} \left\| (A_{\rho}^{\top} \otimes I)^k\right\|_{\infty} < \infty.
  \end{align}
\end{enumerate}
Sie müssen auch erklären können, warum dies die beiden wesentlichen Änderungen
zum skalaren Fall sind!
\end{exercise}
\begin{solution}
Wir schreiben wieder den Fehler $e_{\ell} = y(t_{\ell}) - y_{\ell} \in \R^n, \ell = 0,\dots,N$ um als
\begin{align*}
  E_{\ell} := \begin{pmatrix}
    e_{\ell - k + 1} \\
    \vdots \\
    e_{\ell}
  \end{pmatrix}
  \in R^{kn}
\end{align*}
Analog zum Beweis im Skript erhalten wir mit
$\delta_{\ell} := \Phi(t_{\ell},y(t_{\ell + 1},\dots,y(t_{\ell+1-k}),h) -
\Phi(t_{\ell},y_{\ell + 1},\dots,y_{\ell+1-k},h)$
\begin{align*}
  e_{\ell + 1} = h\delta_{\ell} - \eta_{\ell}(y,h) - \sum_{j = 1}^k \alpha_{k - j}e_{\ell + 1 - j} \in \R^n.
\end{align*}
Also gilt mit $F_{\ell} := (0,\dots,0,h\delta_{\ell} - \eta_{\ell}(y,h)) \in \R^{nk}$
\begin{align*}
  E_{\ell + 1} = \begin{pmatrix}
    0_n & I_n & 0_n & \hdots & 0_n \\
    \vdots & \ddots & I_n & \hdots & 0_n\\
    \vdots &  & \ddots & \ddots & 0_n \\
    0_n & \hdots & \hdots & 0_n & I_n \\
    -\alpha_0I_n & \hdots & \hdots & \hdots & -\alpha_{k-1}I_n
  \end{pmatrix}E_{\ell} + F_{\ell}.
\end{align*}
Wieder analog zum Skript folgt
\begin{align*}
  E_{\ell + k - 1} =  \left(A_{\rho}^{\top} \otimes I\right)^{\ell}E_{k-1}
  + \sum_{j = 0}^{\ell - 1}\left(A_{\rho}^{\top} \otimes I\right)^{\ell - j - 1}F_{j + k - 1},
  \qquad \ell = 0,\dots,N-k+1.
\end{align*}
Die Schranke für $\|F_{\ell + k - 1}\|_{\infty}$ geht im mehr-dimensionalen
genauso wie im Skript durch.
Im dritten Schritt müssen wir wieder etwas anpassen. Wir definieren
\begin{align*}
  M := \sup_{k \in \N_0} \left\| (A_{\rho}^{\top} \otimes I)^k\right\|_{\infty}
\end{align*}
Laut Lemma 5.32. ist $M < \infty$ äquivalent zu $\rho((A_{\rho}^{\top} \otimes I) \leq 1$
und alle $\lambda \in \sigma(A)$ mit $|\lambda| = 1$ sind halbeinfach.
Wir wissen, dass
\begin{align*}
  \sigma(A_{\rho}^{\top} \otimes I) = \sigma(A_{\rho}^{\top} \otimes I)^{\top}
  = \sigma(A_{\rho} \otimes I)
\end{align*}
Man kann zeigen (wenn man denn kann), dass das zugehörige charakteristische Polynom
\begin{align*}
  \left(x^k + \sum_{i = 0}^{k-1}\alpha_ix^i\right)^n
\end{align*}
ist. Induktion nach $n$ ($k$ beliebig fest): $n = 1$.
Hier haben wir den Fall aus dem Skript. \\
$n-1 \to n:$
\begin{align*}
  A_n = \begin{pmatrix}
    0_n & \hdots & \hdots & 0_n & -\alpha_0I_n \\
    I_n & \ddots & & \vdots & -\alpha_1I_n \\
    0_n & I_n & \ddots & \vdots & \vdots \\
    \vdots & \ddots & \ddots & 0_n & -\alpha_{k-2}I_n \\
    0_n & \hdots & 0_n & I_n & -\alpha_{k-1}I_n
  \end{pmatrix}
\end{align*}
Jetzt müssen wir nach den letzten $n$ Zeilen entwickeln?
Wie man leicht erkennt sind die Eigenwerte also genau die Nullstellen von
\begin{align*}
  x^k + \sum_{i = 0}^{k-1}\alpha_ix^i
\end{align*}
mit $n$-facher Vielfachheit. Aus der Wurzelbedingung folgt damit direkt, dass
$\sigma(A_{\rho} \otimes I) \leq 1$. Alle Nullstellen mit $|\lambda| = 1$
sind jetzt aber nicht mehr einfach, sondern haben Vielfachheit $n$.
Wenn wir jetzt noch zeigen können, kann diese Eigenwerte halbeinfach sein müssen,
sind wir fertig.
Mit $M < \infty$ können wir jetzt Schritt 3 aus dem Skript analog nachrechnen.
Im letzten Schritt müssen wir nur noch $\|e_{\ell}\|$ abschätzen.
Das geht wieder analog zum Skript, wenn wir $|e_{\ell + k  - 1}|$
durch $\|e_{\ell + k  - 1}\|_{\infty}$ ersetzen.
\end{solution}


\section*{Programmierung}
Sei $p: \N^3 \to \N$ eine injektive primitiv rekursive Funktion, die außerdem
$p(a,b,c) > \max(a,b,c)$ für alle $a,b,c$ erfüllt.
Wir definieren eine Folge $f_k$ von primitiv rekursiven Funktionen so: \\
$f_0$ ist die $0$-stellige konstante Funktion $0$, $f_1$ ist die $1$-stellige
konstante Funktion $0$. \\
Für $n = p(a,b,c)$: Wenn $a = 0$, dann ist $f_n$ die konstante $b$-stellige Funktion $0$.
Wenn $a = 1$, dann ist $f_n$ die Nachfolgerfunktion $S: \N \to \N$. Wenn $a = 2$
und $b \geq c \geq 1$, dann
ist $f_n = \pi_c^b$. Wenn $a = 3$, und $f_b: \N^k \to \N^l,\ f_c: \N^k \to \N$, dann
ist $f_n: \N^k \to \N^l \times \N = \N^{l+1}$ als
$f_n(\overline{x}) = (f_b(\overline{x}),f_c(\overline{x}))$.
Wenn $a = 4,\ f_b: \N^k \to \N^l,\ f_c: \N^l \to \N^m$, dann ist $f_n := f_c \circ f_b$.
Wenn $a = 5$ und $f_b$ $(k+2)$-stellig, $f_c$ $k$-stellig,
dann ist $f_n: \N^{k+1} \to \N$ durch $f_n = \mathrm{PR}(f_b,f_c)$ definiert.
Sonst sei $f_n$ die konstante einstellige Nullfunktion.

% --------------------------------------------------------------------------------

\begin{exercise}

Betrachten Sie das Anfangswertproblem für die \textit{freie Schrödingergleichung}
\begin{align*}
  \begin{cases}
  i\frac{\partial u}{\partial t} = -\Delta u & \text{in } \Omega = \R^n \times [0,\infty), \\
  u(x,0) = f(x) & \text{für } x \in \R^n,
  \end{cases}
\end{align*}
wobei $f \in L^1(\R^n) \cap L^2(\R^n)$ und $i$ die imaginäre Einheit ist.
\begin{enumerate}[label = (\roman*)]
  \item Bestimmen Sie für eine Lösung $u$ des AWP mit $u(t) \in L^1(\R^n)$
  für alle $t \geq 0$ eine Darstellung mittels Fourier-Transformation.
  \item Zeigen Sie, dass $\|u(\cdot,t)\|_{L^2(\Omega)} = \|f\|_{L^2(\Omega)}$ für $t > 0$.
\end{enumerate}

\end{exercise}

% --------------------------------------------------------------------------------

\begin{solution}
	\phantom{}
	\begin{enumerate}[label = (\roman*)]
		\item Wir betrachten das Fouriertransformierte Anfangswertproblem
			\begin{align*}
			i \widehat{u}_t = \widehat{- \Delta u} = - \sum_{i = 1}^n \widehat{\pderivative[2][]{x_i}u} = \sum_{i = 1}^n x_i^2 \widehat{u} = |x|^2 \widehat{u} \\
			\widehat{u}(k,0) = \widehat{f}(x).
			\end{align*}
			Eine Lösung davon ist gegeben durch
			\begin{align*}
			\widehat{u}(k,t) = \widehat{f}(k) \mathrm{e}^{-i|k|^2t}
			\end{align*}
			und mit der Definition $\widehat{w}(k,t) := \mathrm{e}^{-i|k|^2t}$ erhalten wir
			\begin{align*}
			u(x,t) = (2\pi)^{-n} \Int[\R][]{\widehat{f}(k) \widehat{w}(k,t) \mathrm{e}^{i x \cdot k}}{k} = (2\pi)^{-n} \Int[\R][]{\widehat{f \ast w}(k, t) \mathrm{e}^{i x \cdot k}}{k} = (f \ast w)(x,t)
			\end{align*}
			Nun ist die Frage, was ist $w$?
		\item
	\end{enumerate}

\end{solution}

% --------------------------------------------------------------------------------

% --------------------------------------------------------------------------------

\begin{exercise}
Betrachten Sie die eindimensionale Wärmeleitungsgleichung mit gemischten Randbedingungen
\begin{align*}
  \begin{cases}
    u_t - u_{xx} = 0 & \text{für } x \in (0,\pi),\ t > 0, \\
    u(0,t) = 0 & \text{für } t > 0, \\
    u_x(\pi,t) = 0 & \text{für } t > 0, \\
    u(x,0) = u_0(x) & \text{für } x \in (0,\pi),
  \end{cases}
\end{align*}
wobei $u_0 \in L^2(0,\pi)$.
\begin{enumerate}[label = (\roman*)]
  \item Bestimmen Sie ein vollständiges Orthonormalsystem $(\phi_n)_{n \in \N} \subset L^2(0,\pi)$
  mit $\phi^{\primeprime}_n = \lambda_n\phi_n$ in $(0,\pi)$ mit Randbedingungen
  $\phi_n(0) = \phi_n^{\prime}(\pi) = 0.$
  \item Konstruieren Sie aus $(\phi_n)_{n \in \N}$ eine Lösungsformel für das
  obige parabolische Problem.
  \item Welche Abklingrate (für $t \to \infty$) hat die Wärmeenergie
  $E(t) := \int_0^{\pi} u(x,t) dx$ für eine Lösung $u$?
\end{enumerate}



\end{exercise}

% --------------------------------------------------------------------------------

\begin{solution}
\begin{enumerate}[label = (\roman*)]
  \item Wir betrachten zunächst Lösungen der Differentialgleichung

  \begin{align*}
    \phi_n
    =
    c_1 e^{\sqrt{\lambda_n}x} + c_2 e^{-\sqrt{\lambda_n}x}
  \end{align*}

  Fall 1: $\lambda_n > 0$: \\
  Dann hat die Lösung folgende Form:

  \begin{align*}
    \phi_n(x)
    =
    c_1 \cosh(\sqrt{\lambda_n}x) + c_2 \sinh(\sqrt{\lambda_n}x)
  \end{align*}

  Diese kann die Randbedingungen jedoch nur im trivialen Fall $c_1, c_2 = 0$ erfüllen, da

  \begin{align*}
    &\phi_n(0) = c_1 \stackrel{!}{=} 0\\
    \implies
    &\phi^\prime(\pi) = c_2 \sqrt{\lambda_n} \cosh({\sqrt{\lambda_n}x}) \stackrel{!}{=} 0
  \end{align*}

  Nun ist $\cosh$ jedoch eine strikt positive Funktion, also muss auch $c_2 = 0$.

  Fall 2: $\lambda_n = 0$: \\
  Hier ist unsere Differentialgleichung nur $\phi_n^\primeprime = 0$, hat also die Form $c_1 x + c_2$. Auch hier erhalten wir $\phi_n \equiv 0$.

  Fall 3: $\lambda_n < 0$: \\
  Lösungen sind gegeben durch

  \begin{align*}
    \phi_n(x) = c_1 \sin(\sqrt{|\lambda_n|}x) + c_2 \cos(\sqrt{|\lambda_n|}x)
  \end{align*}

  Für die RB setzen wir ein
  \begin{align*}
    &\phi_n(0) = c_2 \stackrel{!}{=}0 \\
    \implies
    &\phi_n^\prime(\pi) = c_2 \sqrt{|\lambda_n|} \cos({\sqrt{|\lambda_n|}\pi}) \stackrel{!}{=} 0
    \iff
    \sqrt{|\lambda_n|}\pi = \frac{\pi}{2} + k\pi, \quad k \in \Z
  \end{align*}

  Wir haben also zunächst mit der Wahl $k := n-1$ eine Darstellung von $\phi_n$, mit einer noch zu bestimmenden Konstante c, als

  \begin{align*}
    \phi_n(x) = c \sin\left(\frac{2n-1}{2}x\right)
  \end{align*}

  Zweimaligens differenzieren liefert

  \begin{align*}
    \phi_n^\primeprime(x) = \underbrace{-\left(\frac{2n-1}{2}\right)^2}_{\lambda_n} c\sin\left(\frac{2n-1}{2}x\right)
  \end{align*}

  Da $\norm[L^2((0,\pi))]{\sin(\frac{2n-1}{2}x)} = \sqrt{\frac{\pi}{2}}$ wählen wir $c = \sqrt{\frac{2}{\pi}}$ um $\norm[L^2((0,\pi))]{\phi_n} = 1$ zu erhalten. \\

  Wir haben also ein Orthonormalsystem gefunden, von dem wir noch zeigen wollen, dass es auch vollständig ist.
  Aus der Analysis wissen wir, dass $\{\frac{1}{\sqrt{\pi}} \cos(nx): n \in \N\} \cup \{\frac{1}{\sqrt{\pi}} \sin(nx): n \in \N^{+}\}$ eine vollständiges Orthonormalsystem vom $L^2(-\pi, \pi)$ ist.
  Betrachten wir also eine beliebige Funktion $f \in L^2(0, \pi)$ und setzen wir diese ungerade fort auf $(- \pi, \pi)$, wissen wir, dass wir das neue $\tilde{f}$ darstellen können als

  \begin{align*}
    \tilde{f} = \sum_{n=0}^{\infty} (\tilde{f}, \frac{1}{\sqrt{\pi}} \cos(nx))_{L^2(- \pi,\pi)} \frac{1}{\sqrt{\pi}} \cos(nx) + \sum_{n=1}^{\infty} (\tilde{f}, \frac{1}{\sqrt{\pi}} \sin(nx))_{L^2(-\pi, \pi)} \frac{1}{\sqrt{\pi}} \sin(nx)
  \end{align*}

  Die cosinus-Teile fallen weg, da $\tilde{f}$ ungerade ist.

  Aus der oberen Gleichheit folgt also, dass für unser $f \in L^2(0, \pi)$

  \begin{align*}
    f = \sum_{n=1}^{\infty} ({f}, \sqrt{\frac{2}{\pi}} \sin(nx))_{L^2(0, \pi)} \sqrt{\frac{2}{\pi}} \sin(nx)
  \end{align*}

  und somit $\{\sqrt{\frac{2}{\pi}} \sin(nx): n \in \N^{+}\}$ eine Orthonormalbasis ist. Daraus kann man vermutlich zeigen, dass auch unser Orthonormalsystem $\{\sqrt{\frac{2}{\pi}} \sin(\frac{2n-1}{2}x): n \in \N^{+}\}$ vollständig ist - darauf müssen wir an dieser Stelle leider verzichten.

  \item Da wir ein vollständiges Orthonormalsystem haben, konvergiert die folgende Reihe unbedingt und damit absolut.

  \begin{align*}
    \sum_{n=1}^\infty (u_0, \phi_n)_{L^2} \phi_n = u_0
  \end{align*}

  Folgende Funktion ist also wohldefiniert.

  \begin{align*}
    u(x,t)
    :=
    \sum_{n=1}^\infty e^{\lambda_n t} (u_0, \phi_n)_{L^2} \phi_n
    \leq
    \sum_{n=1}^\infty
    \underbrace
    {
      \abs{e^{\lambda_n t}}
    }_{\leq 1}
    \abs
    {
      (u_0, \phi_n)_{L^2} \phi_n
    }
    < \infty
  \end{align*}

  Diese Funktion löst die PDE, dazu setzen wir ein

  \begin{align*}
    u_t - u_{xx} &=\sum_{n=1}^\infty \lambda_n e^{\lambda_n t} (u_0, \phi_n)_{L^2} \phi_n - \sum_{n=1}^\infty \lambda_n e^{\lambda_n t} (u_0, \phi_n)_{L^2} \phi_n= 0 \\
    u(0,t) &= \sum_{n=1}^\infty e^{\lambda_n t} (u_0, \phi_n)_{L^2} \sqrt{\frac{2}{\pi}} \sin(\sqrt{-\lambda_k} 0) = 0 \\
    u_x(\pi, t) &= \sum_{n=1}^\infty \lambda_n e^{\lambda_n t} (u_0, \phi_n)_{L^2} \sqrt{\frac{2}{\pi}} \cos(\sqrt{-\lambda_k} \pi)= 0 \\
    u(x, 0) &= \sum_{n=1}^\infty (u_0, \phi_n)_{L^2} \phi_n(x) = u_0(x)
  \end{align*}

  \item Unter der Annahme, dass mit der Abklingrate eine Ähnliche Ungleichung wie in $(6.15)$ im Skript gemeint ist, schätzen wir ab (wobei $\lambda_1$ der betragsmäßig kleinste (d.h. größte) EW ist)

  \begin{align*}
    \norm[L^2]{u(\cdot,t)}^2
    & =
    \norm[L^2]
    {
      \sum_{n=1}^\infty e^{\lambda_n t} (u_0, \phi_n)_{L^2} \phi_n
    }^2 \\
    & \stackrel{\text{Pythagoras}}{=}
    \sum_{n=1}^\infty
    \norm[L^2]
    {
      e^{\lambda_n t} (u_0, \phi_n)_{L^2} \phi_n
    }^2 \\
    & =
    \Int[0][\pi]{\sum_{n=1}^\infty e^{2\lambda_n t} (u_0, \phi_n)_{L^2}^2 \phi_n^2}{x} \\
    & \leq
    e^{2\lambda_1 t} \sum_{n=1}^\infty (u_0, \phi_n)_{L^2}^2 \underbrace{\norm[L^2(0, \pi)]{\phi_n}^2}_1
    \stackrel
    {
      \mathrm{Parceval}
    }{=}
    e^{2\lambda_1 t} \norm[L^2]{u_0}^2
  \end{align*}

  Insgesamt erhalten wir eine Abschätzung

  \begin{align*}
    |E(t)|
    \leq
    \Int[0][\pi]{|u(x,t)|}{x}
    \leq
    \sqrt{\pi} \norm[L^2]{u(\cdot,t)}
    \leq
    \sqrt{\pi} e^{\lambda_1 t}\norm[L^2]{u_0}
  \end{align*}
  \end{enumerate}
\end{solution}


\section*{Die Ackermannfunktion}
Eine totale Funktion $a: \N \times \N \to \N$ heißt Ackermannfunktion, wenn sie
für alle $x,y \in \N$ die folgenden Bedingungen erfüllt:
\begin{flalign*}
  &A.1\ a(0,y) = y + 1 & \\
  &A.2\ a(x+1,0) = a(x,1) & \\
  &A.3\ a(x+1,y+1) = a(x,a(x+1,y)). &
\end{flalign*}

% -------------------------------------------------------------------------------- %

\begin{exercise}[\textbf{Most powerful test for the normal variance - $\mu$ is unknown}]

    Let $X_1,\dots,X_n$ be i.i.d. $\mathcal{N}(\mu,\sigma^2)$, where $\mu$ is unknown.
    
    
    \begin{enumerate}[label = (\alph*)]
      \item Is there an MP test at level $\alpha$ for testing
      
      \begin{align*}
          H_0: \sigma^2 = \sigma_0^2 \quad \text{vs.} \quad H_1: \sigma^2 = \sigma_1^2, \sigma_1 > \sigma_0?
      \end{align*}

      If not, find the corresponding GLRT.

      \item Is the above generalized likelihood ratio (GLR) test also a GLRT for testing the
      one-sided hypothesis

      \begin{align*}
          H_0: \sigma^2 \leq \sigma_0^2 \quad \text{vs.} \quad H_1: \sigma^2 > \sigma_0^2.
      \end{align*}

      \item Find the GLRT at level $\alpha$ for testing
      
      \begin{align*}
          H_0: \sigma^2 \geq \sigma_0^2 \quad \text{vs.} \quad H_1: \sigma^2 < \sigma_0^2.
      \end{align*}
    \end{enumerate}
    
    \end{exercise}
    
    % -------------------------------------------------------------------------------- %
    
    \begin{solution}
    
    We consider the composite hypotheses $\Theta_0 := \{ (\mu, \sigma_0^2): \mu \in \R\}$
    and $\Theta_1 := \{ (\mu, \sigma_1^2): \mu \in \R \}$.

    \begin{enumerate}[label = (\alph*)]
        \item We already know the general MLE for $(\mu,\sigma^2)$:
        
        \begin{align*}
            \hat{\mu} &= \bar{X} \\
            \hat{\sigma}^2 &= \frac{1}{n}\sum_{i=1}^n (X_i - \bar{X})^2
        \end{align*}

        For the MLE under $H_0$ we simply replace $\hat{\sigma}^2$ by $\sigma_0^2$.

        For the MLE under $H_1$ we simply replace $\hat{\sigma}^2$ by $\sigma_1^2$.

        Now we can write down the GLRT:

        \begin{align*}
            \lambda(\textbf{x}) 
            &= \frac{L(\hat{\mu}, \sigma_1^2 | \textbf{x})}{L(\hat{\mu}, \sigma_0^2 | \textbf{x})} \\
            &= \left(\frac{\sigma_0^2}{\sigma_1^2}\right)^{n/2}
            \exp\left(-\left( \frac{1}{2\sigma_1^2} - \frac{1}{2\sigma_0^2}\right) \sum_{i=1}^n (X_i - \bar{X})^2\right) \\
        \end{align*}
  
        We simplify our test statistic to $T(\textbf{X}) = \sum_{i=1}^n (X_i - \bar{X})^2$.
        Finally we find the corresponding critical value for $T(\textbf{X})$ by solving

        \begin{align*}
            \alpha &\stackrel{!}{=} \sup_{\theta \in \Theta_0} \P(T(\textbf{X}) \geq C) \\
            &= \sup_{\mu \in \R} \P(T(\textbf{X}) \geq C) \\
            &= \sup_{\mu \in \R} \P\left(\sum_{i=1}^n \frac{(X_i - \bar{X})^2}{\sigma_0^2/n} \geq \frac{n}{\sigma_0^2}C\right) \\
            &= 1 - F_{\chi^2(n)}\left(\frac{n}{\sigma_0^2}C\right) \\
            \iff C &= \frac{\sigma_0^2}{n}\chi_{1 - \alpha}^2(n).
        \end{align*}

        \item Our new MLE under $H_0$ for calculating the GLRT for the one-sided
        hypothesis now reads

        \begin{align*}
            \hat{\mu} &= \bar{X} \\
            \hat{\sigma}^2 &= \begin{cases}
                \frac{1}{n}\sum_{i=1}^n (X_i - \bar{X})^2, & \frac{1}{n}\sum_{i=1}^n (X_i - \bar{X})^2 \leq \sigma_0^2 \\
                \sigma_0^2, & \text{otherwise}.
            \end{cases}
        \end{align*}

        Therefore we obtain the new GLR

        \begin{align*}
            \lambda(\textbf{x}) = \begin{cases}
                \left(\frac{\sigma_0^2}{\hat{\sigma}^2}\right)^{n/2}
            \exp\left(-\frac{n}{2}\left(1 - \frac{\hat{\sigma}^2}{\sigma_0^2}\right)\right), & \frac{1}{n}\sum_{i=1}^n (X_i - \bar{X})^2 > \sigma_0^2 \\
            1, & \text{otherwise}.
            \end{cases}
        \end{align*}

        Now we can calculate the critical $C$ by solving

        \begin{align*}
            \alpha \stackrel{!}{=} \sup_{\theta \in \Theta_0}\P(T(\textbf{X}) \geq C)
        \end{align*}
      \end{enumerate}
    
    \end{solution}
    
    % -------------------------------------------------------------------------------- %
    
% --------------------------------------------------------------------------------

\begin{exercise}[208]

Zeigen Sie, dass $A$ in beiden Argumenten streng monoton ist; außerdem, dass
$A(x,y+1) \leq A(x+1,y)$ für alle $x,y$ gilt.

\end{exercise}

% --------------------------------------------------------------------------------

\begin{solution}

	\phantom{}

\end{solution}


Für $c \in \N$ schreiben wir $A_c$ für die Funktion $A_c: \N \to \N, A_c(y) = A(c,y)$. \\
Für jede totale Funktion $f: \N^k \to \N$ schreiben wir $f < A_c$, wenn für alle
$\vv{x} \in \N^k$ die Ungleichung $f(\vv{x}) < A_c(\max \vv{x})$ gilt.

% -------------------------------------------------------------------------------- %

\begin{exercise}[\hl{Implementation Task: Reinforce with Baseline}]

\phantom{}

Implement the ``Reinforce with Baseline'' algorithm (p. 329) and benchmark
with an environment of your choice (i.e. from the textbook or a pre-build
environment from https://gym.openai.com). Any approximation may be used and
the use of existing software libraries is also allowed for the baseline.

\end{exercise}

% -------------------------------------------------------------------------------- %

\begin{solution}

\phantom{}

\end{solution}

% -------------------------------------------------------------------------------- %

% --------------------------------------------------------------------------------

\begin{exercise}[211]

Schließen Sie aus den vorigen Aufgaben:
\begin{enumerate}[label = \alph*.]
  \item Für jede primitiv rekursive Funktion $f$ gibt es ein $c$ mit $f < A_c$.
  \item Die Funktion $x \mapsto A(x,x)$ ist nicht primitiv rekursiv (aber berechenbar).
\end{enumerate}

\end{exercise}

% --------------------------------------------------------------------------------

\begin{solution}
	\phantom{}
	\begin{enumerate}[label = \alph*.]
		\item
			\begin{enumerate}[label = \arabic*.]
				\item Für die konstante Nullfunktion $0: \N^k \to \N: x \mapsto 0$ gilt für beliebiges $x \in \N^k$ die Ungleichung 
					\begin{align*}
					0(x) = 0 < x + 1 = A(0, x) = A_0(x)
					\end{align*}
				
				\item Für die Nachfolgerfunktion $S: \N \to \N: x \mapsto x + 1$ gilt für alle $x \in \N$ wegen der in Aufgabe 208 gezeigten strikten Monotonie von $A$ im ersten Argument
					\begin{align*}
					S(x) = x + 1 = A(0, x) < A(1, x) = A_1(x)
					\end{align*} 
				
				\item Für eine Projektion $\Pi_k^n: \N^n \to \N: x \mapsto x_k$ gilt für beliebiges $x \in \N^n$
				\begin{align*}
				\Pi_k^n(x) = x_k < \max(x) + 1 = A(0, \max(x)) = A_0(x)
				\end{align*}
				
				\item Die Abgeschlossenheit bezüglich der Verknüpfung folgt direkt aus Aufgabe 209
				
				\item Die Abgeschlossenheit bezüglich der primitiven Rekursion folgt direkt aus Aufgabe 210
			\end{enumerate} 
		\item Sei $c \in \N$ beliebig. Wir nennen unsere Abbildung $f: \N \to \N: x \mapsto A(x,x)$. Wegen der in Aufgabe 208 gezeigten strengen Monotonie gilt 
		\begin{align*}
		A_c(c + 1) = A(c, c + 1) < A(c + 1, c + 1) = f(c + 1)
		\end{align*}
		nach Punkt (a) kann daher $f$ nicht primitiv rekursiv sein.
	\end{enumerate}
	
\end{solution}


\end{document}
