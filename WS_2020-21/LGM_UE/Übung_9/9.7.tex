% -------------------------------------------------------------------------------- %
\subsection*{209/210}

\begin{exercise}[209]

Seien $f_1,\dots,f_k: \N^n \to \N$ und $h: \N^k \to \N$ mit $f_1,\dots,f_k, h < A_c$.
Wenn\footnote{$g := h(f_1,\dots,f_k)$ ist Abkürzung für $\forall \vv{x} = (x_1,\dots,x_n):
g(\vv{x}) = h(f_1(\vv{x}),\dots,f_k(\vv{x}))$} $g := h(f_1,\dots,f_k)$, dann gibt
es ein $c^{\prime}$ mit $g < A_{c^{\prime}}$.

\end{exercise}

% -------------------------------------------------------------------------------- %

\begin{solution}

	Wir definieren $c^\prime := c + 2$ und berechnen für beliebiges $x \in \N^n \setminus \{0\}$
	\begin{align*}
	g(x) &= h(f_1(x), \dots, f_k(x)) < A_c\pbraces{\max\Bbraces{f_i(x) \mid i \in \{1, \dots, k\} }} < A_c(A_c(\max x)) \\
	&= A(c, A(c, \max x)) < A(c, A(c + 1, \max x)) = A(c + 1, \max x + 1) \leq A(c + 2, \max x) = A_{c^\prime}(\max x).
	\end{align*}
\end{solution}

\begin{exercise}[210]

Wenn $f$ durch primitive Rekursion aus $h$ und $g$ entsteht, und $h > A_c,\ g < A_c$
gilt, dann gibt es ein $c^{\prime}$ mit $f < A_{c^{\prime}}$.

\end{exercise}

% -------------------------------------------------------------------------------- %

\begin{solution}

\phantom{}

\end{solution}
