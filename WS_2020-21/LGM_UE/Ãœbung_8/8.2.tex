% --------------------------------------------------------------------------------

\begin{exercise}[186]

Gegeben ist die Klauselmenge $M_n = \{\{P(a)\};\{\neg P(x),P(f(x))\};\{\neg P(f^{2^n}(a))\}\}$,
wobei $f^k(a)$ wie folgt definiert ist: $f^0(a)$ ist $a$ und $f^{k+1}(a)$ ist $f(f^k(a))$.
Geben Sie eine Resolutionswiderlegung von $M_n$ mit $\Landau(n)$ Schritten an.
Zeigen Sie, dasss jede AL-Resolutionswiderlegung von $G(M_n)$ mindestens $2^n$
Schritte hat.

\end{exercise}

% --------------------------------------------------------------------------------

\begin{solution}

	Wir zeigen mittels Induktion
	\begin{align*}
	\forall k \in \N: \text{mit } 2k \text{ Schritten erreichen wir } \Bbraces{\neg P(x), P\pbraces{f^{2^k}(x)}}
	\end{align*}
	Für $k = 0$ brauchen wir tatsächlich $0$ Schritte, da $\{\neg P(x),P(f(x))\} \in M_n$.
	Nehmen wir für $k \in \N$ nun an, dass in $2k$ Schritten die Menge
	$\Bbraces{\neg P(x), P\pbraces{f^{2^k}(x)}}$ erreicht werden kann,
	so können wir mit $x / f^{2^k}(x)$ die Menge
	$\Bbraces{\neg P\pbraces{f^{2^k}(x)}, P\pbraces{f^{2^{k + 1}}(x)}}$
	erreichen und dann resolvieren um $\Bbraces{\neg P(x), P\pbraces{f^{2^{k + 1}}(x)}}$ zu erhalten.
	Wenn wir das in $k$ Schritten erreicht haben, müssen wir nur noch zweimal resolvieren,
	um zuerst $\{P(f^{2^k(a)})\}$ und anschließend $\emptyset$ zu erhalten. \\
	\begin{align*}
		G(M_n) = \{\{P(a)\};\{\neg P(f^{2^n}(a))\}\} \cup
		\{\{\neg P(x/t), P(f(x/t))\}: t \text{ variablenfreier Term}\}
	\end{align*}
	Nun gibt es drei sinnvolle Typen von Resolutionen:
	\begin{enumerate}
		\item
		\begin{align*}
			C_{j_1} &= \{P(f^k(a))\} \\
			C_{j_2} &= \{\neg P(f^k(a));P(f^{k+1}(a))\} \\
			\mathrm{Res}(C_{j_1},C_{j_2}) &= \{P(f^{k+1}(a))\}
		\end{align*}
		\item
		\begin{align*}
		C_{j_1} &= \{\neg P(f^k(a))\} \\
		C_{j_2} &= \{\neg P(f^{k-1}(a));P(f^{k}(a))\} \\
		\mathrm{Res}(C_{j_1},C_{j_2}) &= \{\neg P(f^{k-1}(a))\}
		\end{align*}
		\item
		\begin{align*}
		C_{j_1} &= \{\neg P(f^{k-1}(a));P(f^{k}(a))\} \\
		C_{j_2} &= \{\neg P(f^{k}(a));P(f^{k+1}(a))\} \\
		\mathrm{Res}(C_{j_1},C_{j_2}) &= \{\neg P(\neg f^{k-1}(a));P(f^{k+1}(a))\}
		\end{align*}
	\end{enumerate}
	Iterierte Anwendung einer beliebigen Kombination dieser Strategien führt mit $\geq 2^n$
	Schritten zum Ziel, da in jedem Schritt nur eine $f$-Potenz \glqq gewonnen\grqq\ wird.
\end{solution}
