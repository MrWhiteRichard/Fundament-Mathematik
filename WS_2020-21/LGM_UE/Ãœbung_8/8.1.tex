% -------------------------------------------------------------------------------- %

\begin{exercise}[182]

Zeigen Sie mit Hilfe der Resolutionsmethode, dass die Formeln
\begin{enumerate}
  \item $\exists x\, \exists y\, \forall z\, ((P(x) \lor P(y)) \land \neg P(z))$
  \item $\forall z\, \exists x\, \exists y\, ((P(x) \lor P(y)) \land \neg P(z))$
\end{enumerate}
unerfüllbar sind.
\end{exercise}

% -------------------------------------------------------------------------------- %

\begin{solution}
\phantom{}
\begin{enumerate}
  \item Im ersten Schritt schreiben wir die Formel zur erfüllungsäquivalenten Formel
  von letzter Woche um:
  \begin{align*}
    \forall z\,((P(c) \lor P(d)) \land \neg P(z))
  \end{align*}
  Also betrachten wir folgende Klauselmenge:
  \begin{align*}
    \{\{P(c), P(d)\};\{\neg P(z)\}\}
  \end{align*}
  Resolutionsableitung:
  \begin{flalign*}
    C_1 &= \{P(c), P(d)\} &\\
    C_2 &= \{\neg P(z)\} &\\
    C_3 &= \mathrm{Res}(C_1,C_2) = \{P(c)\} & (\sigma: z \mapsto d)\\
    C_4 &= \mathrm{Res}(C_2,C_3) = \emptyset  & (\sigma: z \mapsto c)
  \end{flalign*}
  \item Im ersten Schritt schreiben wir die Formel zur erfüllungsäquivalenten Formel
  von letzter Woche um:
  \begin{align*}
    \forall z\,((P(g(z)) \lor P(f(z))) \land \neg P(z))
  \end{align*}
  Also betrachten wir folgende variablendisjunkte Klauselmenge:
  \begin{align*}
    \{\{P(g(z)), P(f(z))\};\{\neg P(x)\}\}
  \end{align*}
  Resolutionsableitung:
  \begin{flalign*}
    C_1 &= \{P(g(z)), P(f(z))\} &\\
    C_2 &= \{\neg P(x)\} &\\
    C_3 &= \mathrm{Res}(C_1,C_2) = \{P(g(z))\} & (\sigma: x \mapsto f(z))\\
    C_4 &= \mathrm{Res}(C_2,C_3) = \emptyset  & (\sigma: x \mapsto g(z))
  \end{flalign*}

\end{enumerate}



\end{solution}
