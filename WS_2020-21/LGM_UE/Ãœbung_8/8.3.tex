% --------------------------------------------------------------------------------

\begin{exercise}[194 + 195]

Zeigen Sie, dass mehrere der folgenden Funktionen primitiv rekursiv sind:
(Manche dieser Funktionen sind gelegentlich undefiniert, z.B. an der Stelle $0$.
Setzen Sie die Funktion so zu einer totalen Funktion fort, dass Sie von der
Fortsetzung zeigen können, dass sie primitiv rekursiv ist.) \\

\textbf{Geben Sie explizit die primitiv rekursiven Funktionen $h$ und $g$ an,
die Sie im Schema der primitiven Rekursion
\footnote{Gemeint ist eine Definition der Form $f(\vv{x},0) = h(\vv{x}),\
f(\vv{x}, y + 1) = g(\vv{x}, f(\vv{x},y), y)$. Aus notationellen Gründen
treten oft die als trivial geltenden Projektionen $\Pi_k^n$ auf.}
verwenden.}

\begin{itemize}
	\item Addition, Multiplikation, $n!$, modulo-Funktion, $\binom{n}{k}$
	\item Die durch
	\begin{align*}
		f(x) = \begin{cases}
			g(x) & \text{wenn } x \in A \\
			h(x) & \text{sonst}
		\end{cases}
	\end{align*}
	definierte Funktion, wenn $g,h$ primitiv rekursive Funktionen sind und $A$
	eine primitiv rekursive Menge.
	\item Die charakteristische Funktion jeder endlichen Menge.
	\item $(n,k) \mapsto (q,r)$ mit $qk + r = n,\ 0 \leq r < k$.
	\item $(n,k) \mapsto \lfloor \frac{n}{k} \rfloor$. (Gaußklammer)
	\item $(n,k) \mapsto \max(0, \lfloor \frac{p(n)}{q(k)}\rfloor)$,
	wobei $p$ und $q$ beliebige Polynome mit ganzzahligen (möglicherweise
	negativen) Koeffizienten sind.
	\item Die Funktion $(n,k) \mapsto k$-te Dezimalstelle von $n$.
	\item $n \mapsto \lfloor \sqrt{2}*10^n\rfloor$.
	\item Eine geeignete (von Ihnen zu wählende) injektive (bijektive?) Abbildung
	$\N \times \N \to \N$.
	\item Die Fibonacci-Folge $f(0) = f(1) = 1,\ f(n+2) = f(n+1) + f(n)$.\\
	\textit{Hinweis:} Betrachten Sie zunächst die Funktion $n \mapsto 2^{f(n)}\cdot 3^{f(n+1)}$.
	\item Ihre Lieblingsfunktion. (Möglichst nichttrivial.)

\end{itemize}

\end{exercise}

% --------------------------------------------------------------------------------

\begin{solution}
	\phantom{}
	\begin{enumerate}
		\item Addition: 
			\begin{align*}
			\forall x \in \N : x + 0 = x = \pi_1^1(x) \\
			\forall x, y \in \N : x + (y + 1) = (x + y) + 1
			\end{align*}
		\item Multiplikation:
			\begin{align*}
			\forall x \in \N : x0 = 0 \\
			\forall x,y \in \N: x(y+ 1) = xy + x
			\end{align*}
		\item Vorgänger:
			\begin{align*}
			 \forall x \in \N^0: 0 - 1 = 0, \quad \forall x \in \N^0 \forall y \in \N:  y + 1 - 1 = y
			\end{align*}
		\item Subtraktion:
			\begin{align*}
			\forall x \in \N: x - 0 = x \\
			\forall x \in \N \forall y \in \N: x - (y + 1) = (x - y) - 1
			\end{align*}
		\item 
			\begin{align*}
			f(x) = \begin{cases}
			g(x) & \text{wenn } x \in A \\
			h(x) & \text{sonst}
			\end{cases},
			\end{align*}
			wobei $f,g, \chi_A$ primitiv rekursive Funktionen sind.
			\begin{align*}
			f(x) = \chi_A(x) g(x) + (1 - \chi_A(x)) h(x)
			\end{align*}
		\item $\chi_{\{0\}}$
			\begin{align*}
			\chi_{\{0\}}(0) = 0, \quad \forall y \in \N: \chi_{\{0\}}(y + 1) = 0 
			\end{align*}
		\item $\chi_{\{a\}}$ für $a \in \N$
			\begin{align*}
			\chi_{\{a\}}(x) = \chi_{\{0\}}(a - x) \chi_{\{0\}}(x - a)
			\end{align*}
		\item $\chi_A$ für eine endliche Menge $A = \{a_1, \dots, a_l\} \subseteq \N$
			\begin{align*}
			\chi_A = \chi_{\{a_1\}} + \dots + \chi_{\{a_l\}}
			\end{align*}
		\item Faktorielle
			\begin{align*}
			0! = 1, \quad \forall y \in \N: (y + 1)! = y! (y + 1)
			\end{align*}
	\end{enumerate}

\end{solution}
