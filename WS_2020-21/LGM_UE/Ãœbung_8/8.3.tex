% --------------------------------------------------------------------------------

\begin{exercise}[194 + 195]

Zeigen Sie, dass mehrere der folgenden Funktionen primitiv rekursiv sind:
(Manche dieser Funktionen sind gelegentlich undefiniert, z.B. an der Stelle $0$.
Setzen Sie die Funktion so zu einer totalen Funktion fort, dass Sie von der
Fortsetzung zeigen können, dass sie primitiv rekursiv ist.) \\

\textbf{Geben Sie explizit die primitiv rekursiven Funktionen $h$ und $g$ an,
die Sie im Schema der primitiven Rekursion
\footnote{Gemeint ist eine Definition der Form $f(\vv{x},0) = h(\vv{x}),\
f(\vv{x}, y + 1) = g(\vv{x}, f(\vv{x},y), y)$. Aus notationellen Gründen
treten oft die als trivial geltenden Projektionen $\Pi_k^n$ auf.}
verwenden.}

\begin{enumerate}
	\item Addition, Multiplikation, $n!$, modulo-Funktion, $\binom{n}{k}$
	\item Die durch
	\begin{align*}
		f(x) = \begin{cases}
			g(x) & \text{wenn } x \in A \\
			h(x) & \text{sonst}
		\end{cases}
	\end{align*}
	definierte Funktion, wenn $g,h$ primitiv rekursive Funktionen sind und $A$
	eine primitiv rekursive Menge.
	\item Die charakteristische Funktion jeder endlichen Menge.
	\item $(n,k) \mapsto (q,r)$ mit $qk + r = n,\ 0 \leq r < k$.
	\item $(n,k) \mapsto \lfloor \frac{n}{k} \rfloor$. (Gaußklammer)
	\item $(n,k) \mapsto \max(0, \lfloor \frac{p(n)}{q(k)}\rfloor)$,
	wobei $p$ und $q$ beliebige Polynome mit ganzzahligen (möglicherweise
	negativen) Koeffizienten sind.
	\item Die Funktion $(n,k) \mapsto k$-te Dezimalstelle von $n$.
	\item $n \mapsto \lfloor \sqrt{2}*10^n\rfloor$.
	\item Eine geeignete (von Ihnen zu wählende) injektive (bijektive?) Abbildung
	$\N \times \N \to \N$.
	\item Die Fibonacci-Folge $f(0) = f(1) = 1,\ f(n+2) = f(n+1) + f(n)$.\\
	\textit{Hinweis:} Betrachten Sie zunächst die Funktion $n \mapsto 2^{f(n)}\cdot 3^{f(n+1)}$.
	\item Ihre Lieblingsfunktion. (Möglichst nichttrivial.)

\end{enumerate}

\end{exercise}

% --------------------------------------------------------------------------------

\begin{solution}
	\phantom{}
	\begin{enumerate}
		\item Addition $ + = \mathrm{PR}(\pi_1^1, s\circ\pi_3^3)$:
			\begin{align*}
			\forall x \in \N : +(x,0) &= x + 0 = x = \pi_1^1(x) \\
			\forall x, y \in \N : +(x,y+1) &= x + (y + 1) = (x + y) + 1 = s\circ \pi_3^3(x,y,+(x,y))
			\end{align*}
			Multiplikation $\cdot =\mathrm{PR}(0,+(\pi_1^3,\pi_3^3))$:
			\begin{align*}
			\forall x \in \N : \cdot(x,0) &= x \cdot 0 = 0 \\
			\forall x,y \in \N: \cdot(x,y+1) &= x(y+ 1) = xy + x = +(\pi_1^3,\pi_3^3)(x,y,\cdot(x,y))
			\end{align*}
			Vorgänger $v = \mathrm{PR}(0,s\circ\pi_1^2)$:
			\begin{align*}
			 \forall x \in \N^0: v(,0) &= 0 \\
			 \quad \forall x \in \N^0\, \forall y \in \N:v(,y+1) &= y + 1 - 1 = y = s\circ \pi_1^2(,y,v(,y))
			\end{align*}
			Subtraktion $- = \mathrm{PR}(\pi_1^1,v\circ\pi_3^3)$:
			\begin{align*}
			\forall x \in \N: -(x,0) &= x - 0 = x = \pi_1^1(x)\\
			\forall x \in \N\, \forall y \in \N: -(x,y+1) &= x - (y + 1) = (x - y) - 1 = v\circ\pi_3^3(x,y,-(x,y))
			\end{align*}
			Faktorielle $! = \mathrm{PR}(s\circ0,\cdot(s\circ\pi_1^2,\pi_2^2))$:
			\begin{align*}
				\forall x \in \N^0: 0! &= 1 = s\circ 0(,)\\
				\forall x \in \N^0\, \forall y \in \N: (y + 1)! &= y! (y + 1) = \cdot(s\circ\pi_1^2,\pi_2^2)(,y,y!)
			\end{align*}
		\item
			\begin{align*}
			f(x) = \begin{cases}
			g(x) & \text{wenn } x \in A \\
			h(x) & \text{sonst}
			\end{cases},
			\end{align*}
			wobei $f,g, \chi_A$ primitiv rekursive Funktionen sind.
			\begin{align*}
			f(x) &= \chi_A(x) g(x) + (1 - \chi_A(x)) h(x) \\
			&= +(\cdot(\chi_A,g), \cdot(-(1,\chi_A),h))(x)
			\end{align*}
		\item $\chi_{\{0\}}$
			\begin{align*}
			\chi_{\{0\}}(0) &= 1 = s\circ 0 \\
			\forall y \in \N: \chi_{\{0\}}(y + 1) &= 0
			\end{align*}
		$\chi_{\{a\}}$ für $a \in \N$
			\begin{align*}
			\chi_{\{a\}}(x) = \chi_{\{0\}}(a - x) \chi_{\{0\}}(x - a)
			\end{align*}
		$\chi_A$ für eine endliche Menge $A = \{a_1, \dots, a_l\} \subseteq \N$
			\begin{align*}
			\chi_A = \chi_{\{a_1\}} + \dots + \chi_{\{a_l\}}
			\end{align*}
		\item
			Modulo: $\mathrm{mod}(x,y) := y\, \mathrm{mod}\, x$
			\begin{align*}
				\forall x \in \N: \mathrm{mod}(x,0) &= 0 \\
				\forall x \in \N\, \forall y \in \N: \mathrm{mod}(x,y+1) &=
				\begin{cases}
					0 = 0\circ \pi_1^3(x,y,\mathrm{mod}(x,y)), & y+1 = x \\
					y+1 = s\circ\pi_2^3(x,y,\mathrm{mod}(x,y)), & \text{sonst}
				\end{cases}
			\end{align*}
			Definiere $A := \{(x,x+1): x \in \N\}$
			\begin{align*}
				\forall x \in \N: \chi_A(x,0) &= 0 \\
				\forall x \in \N\,\forall y \in \N: \chi_A(x,y+1) &= \chi_{\{0\}}(x-y)\chi_{\{0\}}(y-x)
				= \cdot(\chi_{\{0\}}(-(\pi_1^3,\pi_2^3)),\chi_{\{0\}}(-(\pi_2^3,\pi_1^3)))(x,y)
			\end{align*}
			Binomialkoeffizient: (Wir setzen $\binom{x}{y} = 0$ für $y > x$)
			\begin{align*}
				\forall x \in \N: \binom{0}{x} = \chi_{\{0\}}(x) \\
				\forall x \in \N\,\forall y \in \N: \binom{y+1}{x} = \begin{cases}
					\chi_{\{0\}}(x-y-1)\chi_{\{0\}}(y+1-x), & y+1 \leq x \\
					\frac{(y+1)\binom{y}{x}}{y+1-x}, & \text{sonst}
				\end{cases}
			\end{align*}
		\item konstante Funktion als Verknüpfung von konstanter Nullfunktion und Nachfolgerfunktion
			\begin{align*}
			k = 0 + \sum_{i = 1}^k 1
			\end{align*}
		\item Division
		\item bijektive Abbildung zwischen $\N \times \N \to \N$: gemäß dem Cantorschen Diagonalverfahren summieren wir die Diagonalen jeweils von links unten nach rechts oben auf
			\begin{align*}
			f(n,k) := k + \sum_{i = 1}^{n + k} i = k + \frac{(n + k) (n + k + 1)}{2}
			\end{align*}
	\end{enumerate}

\end{solution}
