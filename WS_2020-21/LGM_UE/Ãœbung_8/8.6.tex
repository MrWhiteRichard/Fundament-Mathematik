% --------------------------------------------------------------------------------

\begin{exercise}[204]

Wir betrachten eine Sprache $\mathscr{L}$, die keine Konstanten und keine
Funktionssymbole enthält, und als einziges Relationssymbol die Gleichheit.
Seien $M_1 \subseteq M_2$ unendliche Mengen; wir fassen sie als $\mathscr{L}$-Strukturen
$\mathscr{M}_1$, bzw. $\mathscr{M}_2$ auf. Verwenden Sie die vorige Aufgabe,
um $\mathscr{M}_1 \preccurlyeq \mathscr{M}_2$ zu zeigen.

\end{exercise}

% --------------------------------------------------------------------------------

\begin{solution}

Wir müssen hier nur noch für alle endliche Mengen $E \subseteq M_1$ und alle $b \in M_2$
einen Automorphismus $\pi$ finden, der $\pi|_E = \id_E$ und $\pi(b) \in M_1$ erfüllt. \\
Seien dazu $E \subseteq M_1, b \in M_2$ beliebig.
\begin{itemize}
  \item Fall 1: $b \in M_1$: \\
  Wir können $\pi = \id_{M_2}$ wählen.
  \item Fall 2: $b \notin M_1$: \\
  Da $E \subset M_1$ eine endliche Menge ist, finden wir ein $m_0 \in M\backslash E$ und
  wir definieren
  \begin{align*}
    \pi(m) = \begin{cases}
      m_0, & m = b \\
      b, & m = m_0 \\
      m, & \text{sonst}
    \end{cases}
  \end{align*}
  Klarerweise gilt $\pi|_E = id_E$ und $\pi$ ist auch bijektiv.
  Mit Aufgabe 203 erhalten wir damit bereits schon $\mathscr{M}_1 \preccurlyeq \mathscr{M}_2$.
\end{itemize}
\end{solution}
