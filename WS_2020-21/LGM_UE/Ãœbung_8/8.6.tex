% --------------------------------------------------------------------------------

\begin{exercise}[204]

Wir betrachten eine Sprache $\mathscr{L}$, die keine Konstanten und keine
Funktionssymbole enthält, und als einziges Relationssymbol die Gleichheit.
Seien $M_1 \subseteq M_2$ unendliche Mengen; wir fassen sie als $\mathscr{L}$-Strukturen
$\mathscr{M}_1$, bzw. $\mathscr{M}_2$ auf. Verwenden Sie die vorige Aufgabe,
um $\mathscr{M}_1 \preccurlyeq \mathscr{M}_2$ zu zeigen.

\end{exercise}

% --------------------------------------------------------------------------------

\begin{solution}

  \phantom{}

\end{solution}
