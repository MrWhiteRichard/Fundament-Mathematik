% -------------------------------------------------------------------------------- %

\begin{exercise}[196]

Sei $f: \N \to \N$ primitiv rekursiv. Dann ist auch die durch
$(\Sigma f)(n) := \sum_{i < n} f(i)$ definierte Funktion primitiv rekursiv,
und überdies (schwach) monoton wachsend.
\\
\\
Sei umgekehrt $g:\N \to \N$ schwach monoton wachsend. Wenn $g$ primitiv rekursiv ist, dann ist auch die durch $(\Delta g)(n) := g(n + 1) - g(n)$ definierte Funktion primitiv rekursiv.

\end{exercise}

% -------------------------------------------------------------------------------- %

\begin{solution}

	\begin{align*}
	(\Sigma f)(0) = \Sigma_\emptyset = 0. \\
	\forall n \in \N: (\Sigma f)(n + 1) = (\Sigma f)(n) + f(n)
	\end{align*}
	und $\Delta g = -(g\circ s, g)$ ist als Verknüpfung von primitiv rekursiven Funktionen wieder primitiv rekursiv.

\end{solution}
