\documentclass{article}

% Hier befinden sich Pakete, die wir beinahe immer benutzen ...

\usepackage[utf8]{inputenc}

% Sprach-Paket:
\usepackage[ngerman]{babel}

% damit's nicht so, wie beim Grill aussieht:
\usepackage{fullpage}

% Mathematik:
\usepackage{amsmath, amssymb, amsfonts, amsthm}
\usepackage{bbm}
\usepackage{mathtools, mathdots}

% Makros mit mehereren Default-Argumenten:
\usepackage{twoopt}

% Anführungszeichen (Makro \Quote{}):
\usepackage{babel}

% if's für Makros:
\usepackage{xifthen}
\usepackage{etoolbox}

% tikz ist kein Zeichenprogramm (doch!):
\usepackage{tikz}

% bessere Aufzählungen:
\usepackage{enumitem}

% (bessere) Umgebung für Bilder:
\usepackage{graphicx, subfig, float}

% Umgebung für Code:
\usepackage{listings}

% Farben:
\usepackage{xcolor}

% Umgebung für "plain text":
\usepackage{verbatim}

% Umgebung für mehrerer Spalten:
\usepackage{multicol}

% "nette" Brüche
\usepackage{nicefrac}

% Spaltentypen verschiedener Dicke
\usepackage{tabularx}
\usepackage{makecell}

% Für Vektoren
\usepackage{esvect}

% (Web-)Links
\usepackage{hyperref}

% Zitieren & Literatur-Verzeichnis
\usepackage[style = authoryear]{biblatex}
\usepackage{csquotes}

% so ähnlich wie mathbb
%\usepackage{mathds}

% Keine Ahnung, was das macht ...
\usepackage{booktabs}
\usepackage{ngerman}
\usepackage{placeins}

% special letters:

\newcommand{\N}{\mathbb{N}}
\newcommand{\Z}{\mathbb{Z}}
\newcommand{\Q}{\mathbb{Q}}
\newcommand{\R}{\mathbb{R}}
\newcommand{\C}{\mathbb{C}}
\newcommand{\K}{\mathbb{K}}
\newcommand{\T}{\mathbb{T}}
\newcommand{\E}{\mathbb{E}}
\newcommand{\V}{\mathbb{V}}
\renewcommand{\S}{\mathbb{S}}
\renewcommand{\P}{\mathbb{P}}
\newcommand{\1}{\mathbbm{1}}

% quantors:

\newcommand{\Forall}{\forall \,}
\newcommand{\Exists}{\exists \,}
\newcommand{\ExistsOnlyOne}{\exists! \,}
\newcommand{\nExists}{\nexists \,}
\newcommand{\ForAlmostAll}{\forall^\infty \,}

% MISC symbols:

\newcommand{\landau}{{\scriptstyle \mathcal{O}}}
\newcommand{\Landau}{\mathcal{O}}


\newcommand{\eps}{\mathrm{eps}}

% graphics in a box:

\newcommandtwoopt
{\includegraphicsboxed}[3][][]
{
  \begin{figure}[!h]
    \begin{boxedin}
      \ifthenelse{\isempty{#1}}
      {
        \begin{center}
          \includegraphics[width = 0.75 \textwidth]{#3}
          \label{fig:#2}
        \end{center}
      }{
        \begin{center}
          \includegraphics[width = 0.75 \textwidth]{#3}
          \caption{#1}
          \label{fig:#2}
        \end{center}
      }
    \end{boxedin}
  \end{figure}
}

% braces:

\newcommand{\pbraces}[1]{{\left  ( #1 \right  )}}
\newcommand{\bbraces}[1]{{\left  [ #1 \right  ]}}
\newcommand{\Bbraces}[1]{{\left \{ #1 \right \}}}
\newcommand{\vbraces}[1]{{\left  | #1 \right  |}}
\newcommand{\Vbraces}[1]{{\left \| #1 \right \|}}
\newcommand{\abraces}[1]{{\left \langle #1 \right \rangle}}
\newcommand{\round}[1]{\bbraces{#1}}

\newcommand
{\floorbraces}[1]
{{\left \lfloor #1 \right \rfloor}}

\newcommand
{\ceilbraces} [1]
{{\left \lceil  #1 \right \rceil }}

% special functions:

\newcommand{\norm}  [2][]{\Vbraces{#2}_{#1}}
\newcommand{\diam}  [2][]{\mathrm{diam}_{#1} \: #2}
\newcommand{\diag}  [1]{\mathrm{diag} \: #1}
\newcommand{\dist}  [1]{\mathrm{dist} \: #1}
\newcommand{\mean}  [1]{\mathrm{mean} \: #1}
\newcommand{\erf}   [1]{\mathrm{erf} \: #1}
\newcommand{\id}    [1]{\mathrm{id} \: #1}
\newcommand{\sgn}   [1]{\mathrm{sgn} \: #1}
\newcommand{\supp}  [1]{\mathrm{supp} \: #1}
\newcommand{\arsinh}[1]{\mathrm{arsinh} \: #1}
\newcommand{\arcosh}[1]{\mathrm{arcosh} \: #1}
\newcommand{\artanh}[1]{\mathrm{artanh} \: #1}
\newcommand{\card}  [1]{\mathrm{card} \: #1}
\newcommand{\Span}  [1]{\mathrm{span} \: #1}
\newcommand{\Aut}   [1]{\mathrm{Aut} \: #1}
\newcommand{\End}   [1]{\mathrm{End} \: #1}
\newcommand{\ggT}   [1]{\mathrm{ggT} \: #1}
\newcommand{\kgV}   [1]{\mathrm{kgV} \: #1}
\newcommand{\ord}   [1]{\mathrm{ord} \: #1}
\newcommand{\grad}  [1]{\mathrm{grad} \: #1}
\newcommand{\ran}   [1]{\mathrm{ran} \: #1}
\newcommand{\graph} [1]{\mathrm{graph} \: #1}
\newcommand{\Inv}   [1]{\mathrm{Inv} \: #1}
\newcommand{\pv}    [1]{\mathrm{pv} \: #1}
\newcommand{\GL}    [1]{\mathrm{GL} \: #1}
\newcommand{\Mod}{\mathrm{Mod} \:}
\newcommand{\Th}{\mathrm{Th} \:}
\newcommand{\Char}{\mathrm{char}}
\newcommand{\At}{\mathrm{At}}
\newcommand{\Ob}{\mathrm{Ob}}
\newcommand{\Hom}{\mathrm{Hom}}
\newcommand{\orthogonal}[3][]{#2 ~\bot_{#1}~ #3}
\newcommand{\Rang}{\mathrm{Rang}}
\newcommand{\NIL}{\mathrm{NIL}}
\newcommand{\Res}{\mathrm{Res}}
\newcommand{\lxor}{\dot \lor}
\newcommand{\Div}{\mathrm{div} \:}
\newcommand{\meas}{\mathrm{meas} \:}

% fractions:

\newcommand{\Frac}[2]{\frac{1}{#1} \pbraces{#2}}
\newcommand{\nfrac}[2]{\nicefrac{#1}{#2}}

% derivatives & integrals:

\newcommandtwoopt
{\Int}[4][][]
{\int_{#1}^{#2} #3 ~\mathrm{d} #4}

\newcommandtwoopt
{\derivative}[3][][]
{
  \frac
  {\mathrm{d}^{#1} #2}
  {\mathrm{d} #3^{#1}}
}

\newcommandtwoopt
{\pderivative}[3][][]
{
  \frac
  {\partial^{#1} #2}
  {\partial #3^{#1}}
}

\newcommand
{\primeprime}
{{\prime \prime}}

\newcommand
{\primeprimeprime}
{{\prime \prime \prime}}

% Text:

\newcommand{\Quote}[1]{\glqq #1\grqq{}}
\newcommand{\Text}[1]{{\text{#1}}}
\newcommand{\fastueberall}{\text{f.ü.}}
\newcommand{\fastsicher}{\text{f.s.}}

% -------------------------------- %
% amsthm-stuff:

\theoremstyle{definition}

% numbered theorems
\newtheorem{theorem}{Satz}
\newtheorem{lemma}{Lemma}
\newtheorem{corollary}{Korollar}
\newtheorem{proposition}{Proposition}
\newtheorem{remark}{Bemerkung}
\newtheorem{definition}{Definition}
\newtheorem{example}{Beispiel}

% unnumbered theorems
\newtheorem*{theorem*}{Satz}
\newtheorem*{lemma*}{Lemma}
\newtheorem*{corollary*}{Korollar}
\newtheorem*{proposition*}{Proposition}
\newtheorem*{remark*}{Bemerkung}
\newtheorem*{definition*}{Definition}
\newtheorem*{example*}{Beispiel}

% Please define this stuff in project ("main.tex"):

% \def \lastexercisenumber {...}
% This will be 0 by default

% \setcounter{section}{...}
% This will be 0 by default
% and hence, completely ignored

\ifnum \thesection = 0
{\newtheorem{exercise}{Aufgabe}}
\else
{\newtheorem{exercise}{Aufgabe}[section]}
\fi

\ifdef
{\lastexercisenumber}
{\setcounter{exercise}{\lastexercisenumber}}

\newcommand{\solution}
{
    \renewcommand{\proofname}{Lösung}
    \renewcommand{\qedsymbol}{}
    \proof
}

\renewcommand{\proofname}{Beweis}

% -------------------------------- %
% environment zum einkasteln:

% dickere vertical lines
\newcolumntype
{x}
[1]
{!{\centering\arraybackslash\vrule width #1}}

% environment selbst (the big cheese)
\newenvironment
{boxedin}
{
  \begin{tabular}
  {
    x{1 pt}
    p{\textwidth}
    x{1 pt}
  }
  \Xhline
  {2 \arrayrulewidth}
}
{
  \\
  \Xhline{2 \arrayrulewidth}
  \end{tabular}
}

% -------------------------------- %
% MISC "Ein-Deutschungen"

\renewcommand
{\figurename}
{Abbildung}

\renewcommand
{\tablename}
{Tabelle}

% -------------------------------- %

% ---------------------------------------------------------------- %
% https://www.overleaf.com/learn/latex/Code_listing

\definecolor{codegreen} {rgb}{0, 0.6, 0}
\definecolor{codegray}    {rgb}{0.5, 0.5, 0.5}
\definecolor{codepurple}{rgb}{0.58, 0, 0.82}
\definecolor{backcolour}{rgb}{0.95, 0.95, 0.92}

\lstdefinestyle{overleaf}
{
    backgroundcolor = \color{backcolour},
    commentstyle = \color{codegreen},
    keywordstyle = \color{magenta},
    numberstyle = \tiny\color{codegray},
    stringstyle = \color{codepurple},
    basicstyle = \ttfamily \footnotesize,
    breakatwhitespace = false,
    breaklines = true,
    captionpos = b,
    keepspaces = true,
    numbers = left,
    numbersep = 5pt,
    showspaces = false,
    showstringspaces = false,
    showtabs = false,
    tabsize = 2
}

% ---------------------------------------------------------------- %
% https://en.wikibooks.org/wiki/LaTeX/Source_Code_Listings

\lstdefinestyle{customc}
{
    belowcaptionskip = 1 \baselineskip,
    breaklines = true,
    frame = L,
    xleftmargin = \parindent,
    language = C,
    showstringspaces = false,
    basicstyle = \footnotesize \ttfamily,
    keywordstyle = \bfseries \color{green!40!black},
    commentstyle = \itshape \color{purple!40!black},
    identifierstyle = \color{blue},
    stringstyle = \color{orange},
}

\lstdefinestyle{customasm}
{
    belowcaptionskip = 1 \baselineskip,
    frame = L,
    xleftmargin = \parindent,
    language = [x86masm] Assembler,
    basicstyle = \footnotesize\ttfamily,
    commentstyle = \itshape\color{purple!40!black},
}

% ---------------------------------------------------------------- %
% https://tex.stackexchange.com/questions/235731/listings-syntax-for-literate

\definecolor{maroon}        {cmyk}{0, 0.87, 0.68, 0.32}
\definecolor{halfgray}      {gray}{0.55}
\definecolor{ipython_frame} {RGB}{207, 207, 207}
\definecolor{ipython_bg}    {RGB}{247, 247, 247}
\definecolor{ipython_red}   {RGB}{186, 33, 33}
\definecolor{ipython_green} {RGB}{0, 128, 0}
\definecolor{ipython_cyan}  {RGB}{64, 128, 128}
\definecolor{ipython_purple}{RGB}{170, 34, 255}

\lstdefinestyle{stackexchangePython}
{
    breaklines = true,
    %
    extendedchars = true,
    literate =
    {á}{{\' a}} 1 {é}{{\' e}} 1 {í}{{\' i}} 1 {ó}{{\' o}} 1 {ú}{{\' u}} 1
    {Á}{{\' A}} 1 {É}{{\' E}} 1 {Í}{{\' I}} 1 {Ó}{{\' O}} 1 {Ú}{{\' U}} 1
    {à}{{\` a}} 1 {è}{{\` e}} 1 {ì}{{\` i}} 1 {ò}{{\` o}} 1 {ù}{{\` u}} 1
    {À}{{\` A}} 1 {È}{{\' E}} 1 {Ì}{{\` I}} 1 {Ò}{{\` O}} 1 {Ù}{{\` U}} 1
    {ä}{{\" a}} 1 {ë}{{\" e}} 1 {ï}{{\" i}} 1 {ö}{{\" o}} 1 {ü}{{\" u}} 1
    {Ä}{{\" A}} 1 {Ë}{{\" E}} 1 {Ï}{{\" I}} 1 {Ö}{{\" O}} 1 {Ü}{{\" U}} 1
    {â}{{\^ a}} 1 {ê}{{\^ e}} 1 {î}{{\^ i}} 1 {ô}{{\^ o}} 1 {û}{{\^ u}} 1
    {Â}{{\^ A}} 1 {Ê}{{\^ E}} 1 {Î}{{\^ I}} 1 {Ô}{{\^ O}} 1 {Û}{{\^ U}} 1
    {œ}{{\oe}}  1 {Œ}{{\OE}}  1 {æ}{{\ae}}  1 {Æ}{{\AE}}  1 {ß}{{\ss}}  1
    {ç}{{\c c}} 1 {Ç}{{\c C}} 1 {ø}{{\o}} 1 {å}{{\r a}} 1 {Å}{{\r A}} 1
    {€}{{\EUR}} 1 {£}{{\pounds}} 1
}


% Python definition (c) 1998 Michael Weber
% Additional definitions (2013) Alexis Dimitriadis
% modified by me (should not have empty lines)

\lstdefinelanguage{iPython}{
    morekeywords = {access, and, break, class, continue, def, del, elif, else, except, exec, finally, for, from, global, if, import, in, is, lambda, not, or, pass, print, raise, return, try, while}, %
    %
    % Built-ins
    morekeywords = [2]{abs, all, any, basestring, bin, bool, bytearray, callable, chr, classmethod, cmp, compile, complex, delattr, dict, dir, divmod, enumerate, eval, execfile, file, filter, float, format, frozenset, getattr, globals, hasattr, hash, help, hex, id, input, int, isinstance, issubclass, iter, len, list, locals, long, map, max, memoryview, min, next, object, oct, open, ord, pow, property, range, raw_input, reduce, reload, repr, reversed, round, set, setattr, slice, sorted, staticmethod, str, sum, super, tuple, type, unichr, unicode, vars, xrange, zip, apply, buffer, coerce, intern}, %
    %
    sensitive = true, %
    morecomment = [l] \#, %
    morestring = [b]', %
    morestring = [b]", %
    %
    morestring = [s]{'''}{'''}, % used for documentation text (mulitiline strings)
    morestring = [s]{"""}{"""}, % added by Philipp Matthias Hahn
    %
    morestring = [s]{r'}{'},     % `raw' strings
    morestring = [s]{r"}{"},     %
    morestring = [s]{r'''}{'''}, %
    morestring = [s]{r"""}{"""}, %
    morestring = [s]{u'}{'},     % unicode strings
    morestring = [s]{u"}{"},     %
    morestring = [s]{u'''}{'''}, %
    morestring = [s]{u"""}{"""}, %
    %
    % {replace}{replacement}{lenght of replace}
    % *{-}{-}{1} will not replace in comments and so on
    literate = 
    {á}{{\' a}} 1 {é}{{\' e}} 1 {í}{{\' i}} 1 {ó}{{\' o}} 1 {ú}{{\' u}} 1
    {Á}{{\' A}} 1 {É}{{\' E}} 1 {Í}{{\' I}} 1 {Ó}{{\' O}} 1 {Ú}{{\' U}} 1
    {à}{{\` a}} 1 {è}{{\` e}} 1 {ì}{{\` i}} 1 {ò}{{\` o}} 1 {ù}{{\` u}} 1
    {À}{{\` A}} 1 {È}{{\' E}} 1 {Ì}{{\` I}} 1 {Ò}{{\` O}} 1 {Ù}{{\` U}} 1
    {ä}{{\" a}} 1 {ë}{{\" e}} 1 {ï}{{\" i}} 1 {ö}{{\" o}} 1 {ü}{{\" u}} 1
    {Ä}{{\" A}} 1 {Ë}{{\" E}} 1 {Ï}{{\" I}} 1 {Ö}{{\" O}} 1 {Ü}{{\" U}} 1
    {â}{{\^ a}} 1 {ê}{{\^ e}} 1 {î}{{\^ i}} 1 {ô}{{\^ o}} 1 {û}{{\^ u}} 1
    {Â}{{\^ A}} 1 {Ê}{{\^ E}} 1 {Î}{{\^ I}} 1 {Ô}{{\^ O}} 1 {Û}{{\^ U}} 1
    {œ}{{\oe}}  1 {Œ}{{\OE}}  1 {æ}{{\ae}}  1 {Æ}{{\AE}}  1 {ß}{{\ss}}  1
    {ç}{{\c c}} 1 {Ç}{{\c C}} 1 {ø}{{\o}} 1 {å}{{\r a}} 1 {Å}{{\r A}} 1
    {€}{{\EUR}} 1 {£}{{\pounds}} 1
    %
    {^}{{{\color{ipython_purple}\^ {}}}} 1
    { = }{{{\color{ipython_purple} = }}} 1
    %
    {+}{{{\color{ipython_purple}+}}} 1
    {*}{{{\color{ipython_purple}$^\ast$}}} 1
    {/}{{{\color{ipython_purple}/}}} 1
    %
    {+=}{{{+=}}} 1
    {-=}{{{-=}}} 1
    {*=}{{{$^\ast$ = }}} 1
    {/=}{{{/=}}} 1,
    literate = 
    *{-}{{{\color{ipython_purple} -}}} 1
     {?}{{{\color{ipython_purple} ?}}} 1,
    %
    identifierstyle = \color{black}\ttfamily,
    commentstyle = \color{ipython_cyan}\ttfamily,
    stringstyle = \color{ipython_red}\ttfamily,
    keepspaces = true,
    showspaces = false,
    showstringspaces = false,
    %
    rulecolor = \color{ipython_frame},
    frame = single,
    frameround = {t}{t}{t}{t},
    framexleftmargin = 6mm,
    numbers = left,
    numberstyle = \tiny\color{halfgray},
    %
    %
    backgroundcolor = \color{ipython_bg},
    % extendedchars = true,
    basicstyle = \scriptsize,
    keywordstyle = \color{ipython_green}\ttfamily,
}

% ---------------------------------------------------------------- %
% https://tex.stackexchange.com/questions/417884/colour-r-code-to-match-knitr-theme-using-listings-minted-or-other

\geometry{verbose, tmargin = 2.5cm, bmargin = 2.5cm, lmargin = 2.5cm, rmargin = 2.5cm}

\definecolor{backgroundCol}  {rgb}{.97, .97, .97}
\definecolor{commentstyleCol}{rgb}{0.678, 0.584, 0.686}
\definecolor{keywordstyleCol}{rgb}{0.737, 0.353, 0.396}
\definecolor{stringstyleCol} {rgb}{0.192, 0.494, 0.8}
\definecolor{NumCol}         {rgb}{0.686, 0.059, 0.569}
\definecolor{basicstyleCol}  {rgb}{0.345, 0.345, 0.345}

\lstdefinestyle{stackexchangeR}
{
    language = R,                                        % the language of the code
    basicstyle = \small \ttfamily \color{basicstyleCol}, % the size of the fonts that are used for the code
    % numbers = left,                                      % where to put the line-numbers
    numberstyle = \color{green},                         % the style that is used for the line-numbers
    stepnumber = 1,                                      % the step between two line-numbers. If it is 1, each line will be numbered
    numbersep = 5pt,                                     % how far the line-numbers are from the code
    backgroundcolor = \color{backgroundCol},             % choose the background color. You must add \usepackage{color}
    showspaces = false,                                  % show spaces adding particular underscores
    showstringspaces = false,                            % underline spaces within strings
    showtabs = false,                                    % show tabs within strings adding particular underscores
    % frame = single,                                      % adds a frame around the code
    % rulecolor = \color{white},                           % if not set, the frame-color may be changed on line-breaks within not-black text (e.g. commens (green here))
    tabsize = 2,                                         % sets default tabsize to 2 spaces
    captionpos = b,                                      % sets the caption-position to bottom
    breaklines = true,                                   % sets automatic line breaking
    breakatwhitespace = false,                           % sets if automatic breaks should only happen at whitespace
    keywordstyle = \color{keywordstyleCol},              % keyword style
    commentstyle = \color{commentstyleCol},              % comment style
    stringstyle = \color{stringstyleCol},                % string literal style
    literate = %
    *{0}{{{\color{NumCol} 0}}} 1
     {1}{{{\color{NumCol} 1}}} 1
     {2}{{{\color{NumCol} 2}}} 1
     {3}{{{\color{NumCol} 3}}} 1
     {4}{{{\color{NumCol} 4}}} 1
     {5}{{{\color{NumCol} 5}}} 1
     {6}{{{\color{NumCol} 6}}} 1
     {7}{{{\color{NumCol} 7}}} 1
     {8}{{{\color{NumCol} 8}}} 1
     {9}{{{\color{NumCol} 9}}} 1
}

% ---------------------------------------------------------------- %
% Fundament Mathematik

\lstdefinestyle{fundament}{basicstyle = \ttfamily}

% ---------------------------------------------------------------- %


\parskip 0pt
\parindent 0pt

\title
{
  Logik und Grundlagen der Mathematik \\
  \vspace{4pt}
  \normalsize
  \textit{8. Übung am 26.11.2020}
}
\author
{
  Richard Weiss
  \and
  Florian Schager
  \and
  Fabian Zehetgruber
}
\date{}

\begin{document}

\maketitle

\section*{Resolution}
Unter einer \glqq Instanz\grqq\ einer Formel $\varphi$ (die meist als Klausel gegeben ist),
verstehen wir jede Formel $\varphi[x_1/t_1,\dots,x_n/t_n]$, die man durch (beliebig viele)
sinnvolle Substitutionen erhält. Unter einer Grundinstanz verstehen wir eine Instanz
ohne freie Variablen.

\begin{exercise}
Sei $A \in \R^{d \times d}$ und $\omega > s(A) := \max \{\Re(\lambda): \lambda \in \sigma(A)\}$.
Zeigen Sie: Es gibt ein $M \geq 1$, sodass
\begin{align*}
  |\exp(tA)| \leq M\exp(\omega t), \qquad t \geq 0.
\end{align*}
Warum gilt diese Aussage nicht, wenn lediglich $\omega \geq s(A)$ gefordert wird?
\end{exercise}
\begin{solution}
Anmerkung: Wir verwenden für das Beispiel die Zeilensummennorm. Die Aussage ist
aufgrund der Äquivalenz aller Normen im endlich-dimensionalen für beliebige Normen gültig. \\
Sei also $A = VJV{-1}$ mit der zugehörigen Jordan-Normalform $J$.
\begin{align*}
  \|\exp(tA)\| = \|V\exp(tJ)V^{-1}\| \leq \underbrace{\|V\|\|V^{-1}\|}_{=:\widetilde{M}}\|\exp(tJ)\|
\end{align*}
Da die Exponentialfunktion einer Block-Diagonalmatrix wieder eine Block-Diagonalmatrix ist,
können wir für jeden Eigenwert $\lambda$ die zugehörigen Jordan-Kästchen betrachten.
Für
\begin{align*}
  \widetilde{J} = \begin{pmatrix}
    \lambda & 1 & & \\
    & \ddots & \ddots & \\
    & & \ddots & 1 \\
    & & & \lambda
  \end{pmatrix}
\end{align*}
ist
\begin{align*}
  \exp(t\widetilde{J}) = \exp(\lambda t)\begin{pmatrix}
    1 & t & \dots & \frac{t^{r-1}}{(r-1)!} \\
    & 1 & \dots & \frac{t^{r-2}}{(r-2)!} \\
    & & \ddots & \vdots \\
    & & & 1
  \end{pmatrix}
\end{align*}
Damit ist $\|\exp(t\widetilde{J})\| \leq \exp(\lambda t)\sum_{i=0}^{d-1} t^d$.
Da $|\exp(\lambda)| = \exp(\Re(\lambda))$ brauchen wir für $\|\exp(tJ)\|$
aufgrund der Block-Diagonalform nur den Eigenwert
mit größtem Realteil betrachten. Das größte Jordan-Kästchen davon hat maximal Dimension $d$
und es folgt
\begin{align*}
  \|\exp(tA)\| \leq \widetilde{M}\|\exp(s(A)t)\|\sum_{i=0}^{d-1} t^i \stackrel{!}{\leq} M \exp(\omega t).
\end{align*}
Wir formen um und erhalten für $M$
\begin{align*}
  M\geq \widetilde{M}\exp(\underbrace{t(s(A) - \omega)}_{< 0})\sum_{i=0}^{d-1} t^d =: f(t)
\end{align*}
Wie man leicht sieht, ist
\begin{align*}
  \lim_{t \to \infty}f(t) = 0
\end{align*}
und somit können wir $M := \max_{t \geq 0} f(t)$ wählen. \\
Als Gegenbeispiel dafür, dass die Aussage für $\omega \geq s(A)$ nicht mehr stimmt,
betrachte
\begin{align*}
  A = \begin{pmatrix}
    1 & 1 \\ 0 & 1
  \end{pmatrix}
\end{align*}
mit dem einzigen Eigenwert $\lambda = 1$. Wähle also $\omega = s(A) = 1$ und berechne
\begin{align*}
  \|\exp(tA)\| = \exp(1)\left\|\begin{pmatrix}
    1 & t \\ 0 & 1
  \end{pmatrix}\right\|
  = \exp(1)(t+1)
\end{align*}
Es gibt also kein $M > 0$, sodass die Gleichung
\begin{align*}
  \exp(1)(t+1) \leq M \exp(1)
\end{align*}
für alle $t \geq 0$ erfüllt wird.
\end{solution}

\begin{exercise}
Betrachten Sie eine skalare ODE $y^{\prime} = f(t,y)$ mit $f: \R \times \R \to \R$
stetig und lokal Lipschitz im 2.Argument. Sei $t \mapsto y(t), t \geq t_0$
die Lösung des AWP $y(t_0) = y_0$. Es seien $t \mapsto y_1(t)$ und $t \mapsto y_2(t)$
zwei differenzierbare Funktionen $\R \to \R$, für die gilt
\begin{align*}
  y_1(t_0) &\leq y_0, \qquad y_1^{\prime} \leq f(t,y_1), \qquad t \geq t_0 \\
  y_2(t_0) &\geq y_0, \qquad y_2^{\prime} \geq f(t,y_2), \qquad t \geq t_0.
\end{align*}
\begin{enumerate}[label = \textbf{\alph*)}]
  \item Zeigen Sie, dass für $t \geq t_0$ gilt
  \begin{align*}
    y_1(t) \leq y(t) \leq y_2(t).
  \end{align*}
  \textit{Hinweis:} Verwenden Sie Aufgabe 3.3 und die stetige Abhängigkeit von
  AWPs von der rechten Seite $f$.
  \item Zeigen Sie damit, dass für die Lösung des AWP
  \begin{align*}
    y^{\prime} = -y^3 + \sin(t), \qquad y(0) = y_0, \qquad -2 \leq y_0 \leq 2
  \end{align*}
  gilt $-2 \leq y(t) \leq 2$ für $t \geq 0$.
  \item Zeigen Sie, dass diese ODE eine $2\pi$-periodische Lösung hat. \\
  \textit{Hinweis:} Brouwerscher Fixpunktsatz.
\end{enumerate}
\end{exercise}
\begin{solution}
\leavevmode \\
\begin{enumerate}[label = \textbf{\alph*)}]
  \item Sei $\epsilon \in \R^+$ beliebig,
  und $|y_2 - z_2| < \frac{\epsilon}{L}$, wobei
  $L$ die Lipschitz-Konstante von $f$ in $y$ bezeichnet.
  Dann gilt
  \begin{align*}
    |f(t,y_2) - f(t,z_2)| \leq L |y_2 - z_2| < \epsilon.
  \end{align*}
  Wähle $\delta := \frac{\epsilon}{2L}, z_1(t) := y_1(t) - \delta, z_2(t) := y_2(t) + \delta$, dann folgt
  \begin{align*}
    z_1^{\prime}(t) &= y_1^{\prime}(t) = f(t,y_1) > f(t,z_1) - \epsilon \\
    z_2^{\prime}(t) &= y_2^{\prime}(t) = f(t,y_2) > f(t,z_2) - \epsilon
  \end{align*}
  Sei nun $z_{\epsilon}$ eine Lösung von $z_{\epsilon}^{\prime} = f(t,z) - \epsilon, z_{\epsilon}(t_0) = y_0$.
  Es gilt
  \begin{align*}
    z_1(t_0) = y_1(t_0) - \delta < y_0 = z_{\epsilon}(t_0) < y_2(t_0) + \delta = z_2(t_0)
  \end{align*}
  und nach Aufgabe 3.3 folgt
  \begin{align*}
    \forall t \geq t_0: z_1(t) < z_{\epsilon}(t) < z_2(t).
  \end{align*}
  Mit Satz 4.1 folgt aufgrund $|z_{\epsilon}^{\prime}(t) - f(t,z)| = |f(t,z) - \epsilon - f(t,z)| \leq \epsilon$
  \begin{align*}
  |y(t) - z_{\epsilon}(t)| \leq \underbrace{|y(t_0) - z_{\epsilon}(t_0)|}_{=0}\exp(L|t - t_0|) + \frac{\epsilon}{L}\left(\exp(L|t-t_0|) - 1\right)
  = \frac{\epsilon}{L}\left(\exp(L|t-t_0|) - 1\right)
  \stackrel{\epsilon \to 0}{\longrightarrow} 0.
  \end{align*}
  Wir erhalten $\lim_{\epsilon \to 0} z_{\epsilon}(t) = y(t)$. Insgesamt gilt
  \begin{align*}
    y_1(t) = \lim_{\epsilon \to 0}y_1(t) - \frac{\epsilon}{2L} \leq \underbrace{\lim_{\epsilon \to 0} z_{\epsilon}(t)}_{= y(t)} \leq
    \lim_{\epsilon \to 0}y_2(t) + \frac{\epsilon}{2L} = y_2(t).
  \end{align*}
  \item Definiere $y_1(t) = -2, y_2(t) = 2$. Dann gilt
  \begin{align*}
    -2 = y_1(0) \leq y(0) \leq y_2(0) \leq 2
  \end{align*}
  und
  \begin{align*}
    0 &= y_1^{\prime}(t) \leq f(t,y_1) = -(-2)^3 + \sin(t) \leq 7 \\
    0 &= y_2^{\prime}(t) \geq f(t,y_2) = -2^3 + \sin(t) \geq -7.
  \end{align*}
  Damit folgt nach Punkt a)
  \begin{align*}
    -2 \leq y(t) \leq 2.
  \end{align*}
  \item Der Fixpunktsatz von Brouwer besagt, dass jede stetige Abbildung $f: K \to K$
  von einer kompakten und konvexen Teilmenge $K \subset \R^n$ in sich einen Fixpunkt hat.
  Betrachte also die Funktion
  \begin{align*}
    \varphi: \begin{cases}
      [-2,2] &\to [-2,2] \\
      y_0 &\mapsto y_{0,y_0}(2\pi)
    \end{cases},
  \end{align*}
  welche aufgrund der stetigen Abhänigkeit von den Daten stetig ist.
  Aufgrund Punkt b) ist diese Funktion wohldefiniert und mit dem Brouwerschen Fixpunktsatz
  existiert ein $y_1 \in [-2,2]$, sodass
  \begin{align*}
    y_{0,y_1}(0) = y_1 = y_{0,y_1}(2\pi).
  \end{align*}
  $\widetilde{y}(t) = y_{0,y_1}(2\pi + t)$ erfüllt klarerweise das Anfangswertproblem $\widetilde{y}^{\prime}(t) =
  \widetilde{f}(t,\widetilde{y}(t)) = f(2\pi+t,\widetilde{y}(t)),
  \widetilde{y}(0) = y_1$. Es gilt
  \begin{align*}
    f(2\pi + t,y) = -y^3 + \sin(2\pi + t) = f(t,y)
  \end{align*}
  Jetzt gilt aufgrund der Eindeutigkeit von Anfangswertproblemen
  $y_{0,y_1}(2\pi + t) = \widetilde{y}(t) = y_{0,y_1}(t)$.
\end{enumerate}
\end{solution}


\section*{Primitiv rekursive Funktionen}
Wir nennen eine Relation $R \subseteq \N^n$ primitiv rekursiv (genauer: \glqq
primitiv rekursive Relation\grqq\, oder \glqq primitiv rekursiv als Relation\grqq), wenn
ihre charakteristische Funktion $\chi_R$ primitiv rekursiv ist. \\
ACHTUNG: Es gibt Funktionen, die zwar nicht primitiv rekursiv sind, die aber
in diesem Sinn eine primitiv rekursive Relation sind. \\
In jeder der folgenden Aufgaben dürfen Sie die jeweils vorigen Aufgaben verwenden.

% --------------------------------------------------------------------------------

\begin{exercise}

Sei $\Omega = (0,1)$. Lösen Sie das folgende Problem: Gesucht ist ein $u \in H_0^1(\Omega)$ mit
\begin{align*}
  \int_\Omega a(x)u^{\prime}(x)v^{\prime}(x) - v(x) dx = 0 \text{ für alle } v \in H_0^1(\Omega),
\end{align*}
wobei $a(x) = \1_{(0,\frac{1}{2}]}(x) + 2\cdot\1_{(\frac{1}{2},1)}(x)$.
\end{exercise}

% --------------------------------------------------------------------------------

\begin{solution}
Motivation: Für $u \in C^1(\Omega) \cap H^2(\Omega)$ und $v \in H_0^1(\Omega)$ gilt
mit partieller Integration für Sobolevfunktionen, sowie der Tatsache, dass aufgrund
des Sobolevschen Einbettungssatzes $v \in C(\overline{\Omega})$ ist und somit
$Tv = v|\partial\Omega, Tu = u|\partial\Omega$:
\begin{align*}
  \int_\Omega a(x)u^{\prime}(x)v^{\prime}(x) &= \int_\Omega v(x) dx \\
  \iff \int_0^{\frac{1}{2}} u^{\prime}(x)v^{\prime}(x)dx + 2\int_{\frac{1}{2}}^1 u^{\prime}(x)v^{\prime}(x) dx &= \int_0^1 v(x) dx \\
  \iff u^{\prime}\pbraces{\frac{1}{2}}v\pbraces{\frac{1}{2}} - u^{\prime}\pbraces{\frac{1}{2}}v\pbraces{\frac{1}{2}} -
  2\int_{\frac{1}{2}}^1 u^{\primeprime}(x)v(x) dx \int_0^{\frac{1}{2}}u^{\primeprime}(x)v(x) &= \int_0^1 v(x) dx \\
  \iff - 2\int_{\frac{1}{2}}^1 u^{\primeprime}(x)v(x) dx - \int_0^{\frac{1}{2}}u^{\primeprime}(x)v(x) &= \int_0^1 v(x) dx \\
  \iff \int_0^1 v(x)(a(x)u^{\primeprime}(x) + 1) dx &= 0
\end{align*}
Ansatz: Suche $u \in C^1(\Omega) \subset H_1(\Omega)$, sodass $u^{\primeprime}(x)a(x) + 1 = 0$ punktweise, sowie $u(0) = u(1) = 0$.
\begin{align*}
  u(x) = -\int\int \frac{1}{a(z)} dz dy  &=
  -\int\int \1_{(0,\frac{1}{2}]}(x) + \frac{1}{2}\cdot\1_{(\frac{1}{2},1)}(x) dz dy  \\
  &= -\int y - \frac{2y-1}{4}\1_{(\frac{1}{2},1)}(y) + C dy \\
  &= -\frac{x^2}{2} + \frac{4x^2-4x+1}{16}\1_{(\frac{1}{2},1)}(x) + Cx + D
\end{align*}
Damit $u \in H_0^1(\Omega)$, muss noch $u(0) = u(1) = 0$ erreicht werden
\begin{align*}
  0 &\stackrel{!}{=} u(1) = -\frac{1}{2} + \frac{1-1 + 1}{4} + C + D \iff C + D = \frac{1}{4} \\
  0 &\stackrel{!}{=} u(0) = D \iff D = 0 \iff C = \frac{1}{4} \\
\end{align*}
Unser Lösungskandidat $u \in C^1(\Omega)$ ist also
\begin{align*}
  u(x) &= -\frac{1}{16}(8x^2-4x - (4x^2-4x+1)\1_{(\frac{1}{2},1)}(x)) \\
  u^{\prime}(x) &= -\frac{1}{16}(16x - 4 - (8x-4)\1_{(\frac{1}{2},1)}(x))
\end{align*}
Berechne nun die distributionelle Ableitung von $u^{\prime}(x)$:
\begin{align*}
  \langle u^{\primeprime}, \phi \rangle &= \langle u^{\prime}, -\phi^{\prime}\rangle
  = -\int_0^1 u^{\prime}(x)\phi^{\prime}(x) dx =
  \frac{1}{16}\left(\int_0^{\frac{1}{2}}(16x-4)\phi^{\prime}(x) dx +
  \int_{\frac{1}{2}}^1(16x-4-8x+4)\phi^{\prime}(x) dx\right) \\
  &= \frac{1}{16}\left([(16x-4)\phi(x)]_{x=0}^{\frac{1}{2}} + [8x\phi(x)]_{x=1/2}^{1} -
  16\int_0^{\frac{1}{2}}\phi(x) - 8\int_{\frac{1}{2}}^1\phi(x)dx\right) \\
  &= -\int_0^1 \frac{1}{a(x)}\phi(x) dx.
\end{align*}
Also ist $u \in H_0^2(\Omega) \cap C^1(\Omega)$, es gilt $a(x)u^{\primeprime}(x) + 1 = 0$
fast überall. Damit können wir alle Schritte aus der Motivation genauso wieder
zurückgehen und erhalten
\begin{align*}
  \int_0^1 v(x)(a(x)u^{\primeprime}(x) + 1) dx &= 0 \\
  \iff - \int_0^{\frac{1}{2}}u^{\primeprime}(x)v(x) -2\int_{\frac{1}{2}}^1 u^{\primeprime}(x)v(x) dx
  &= \int_0^1 v(x) dx \\
   \iff \underbrace{u^{\prime}\pbraces{\frac{1}{2}}v\pbraces{\frac{1}{2}} - u^{\prime}\pbraces{\frac{1}{2}}v\pbraces{\frac{1}{2}}}_{=0} - \int_0^{\frac{1}{2}}u^{\primeprime}(x)v(x)
  -2 \int_{\frac{1}{2}}^1 u^{\primeprime}(x)v(x) dx  &=
  \int_0^1 v(x) dx \\
  \stackrel{PI}{\iff} \int_0^{\frac{1}{2}} u^{\prime}(x)v^{\prime}(x)dx +
  2\int_{\frac{1}{2}}^1 u^{\prime}(x)v^{\prime}(x) dx &= \int_0^1 v(x) dx \\
  \iff \int_\Omega a(x)u^{\prime}(x)v^{\prime}(x) &= \int_\Omega v(x) dx
\end{align*}
\end{solution}

% --------------------------------------------------------------------------------

\begin{solution}

Wir stellen zunächst einen Ansatz für $u$ auf, indem wir partiell integrieren (streng genommen nach Satz 5.8 (Gauß für Sobolev-Funktionen)).

\begin{align*}
  0
  & \stackrel{!}{=}
  \Int[\Omega]{a(x) u^\prime(x) v^\prime(x) - v(x)}{x}
  =
  \Int[0][\frac{1}{2}]{u^\prime(x) v^\prime(x)}{x}
  +
  \Int[\frac{1}{2}][1]{2 u^\prime(x) v^\prime(x)}{x}
  -
  \Int[\Omega]{v(x)}{x} \\
  & \stackrel
  {
    \mathrm{PI}
  }{=}
  u^\prime(x) v(x) \Big |_{x=0}^{\frac{1}{2}}
  -
  \Int[0][\frac{1}{2}]{u^\primeprime(x) v(x)}{x}
  +
  2 u^\prime(x) v(x) \Big |_{x=\frac{1}{2}}^1
  -
  \Int[\frac{1}{2}][1]{2 u^\primeprime(x) v(x)}{x}
  -
  \Int[\Omega]{v(x)}{x} \\
  & =
  u^\prime \pbraces{\frac{1}{2} +} v\pbraces{\frac{1}{2} -}
  -
  \Int[0][\frac{1}{2}]{u^\primeprime(x) v(x)}{x}
  -
  2 u^\prime \pbraces{\frac{1}{2} -} v\pbraces{\frac{1}{2} +}
  -
  \Int[\frac{1}{2}][1]{2 u^\primeprime(x) v(x)}{x}
  -
  \Int[\Omega]{v(x)}{x}
\end{align*}

\begin{align*}
  \implies
  u^\prime \pbraces{\frac{1}{2} +} v\pbraces{\frac{1}{2} -}
  -
  \Int[0][\frac{1}{2}]{u^\primeprime(x) v(x)}{x}
  -
  2 u^\prime \pbraces{\frac{1}{2} -} v\pbraces{\frac{1}{2} +}
  -
  \Int[\frac{1}{2}][1]{2 u^\primeprime(x) v(x)}{x}
  \stackrel{!}{=}
  \Int[\Omega]{v(x)}{x}
\end{align*}

Laut Satz 5.9 (Einbettungssatz von Sobolev), ist $v$ stetig.

\begin{align*}
  \implies
  v \in C(\Omega)
\end{align*}

\includegraphicsunboxed{PDEs/PDEs_-_Satz_5-9_(Einbettungssatz_von_Sobolev).png}

Laut Satz 5.6 (Spur von Sobolevfunktionen) und Satz 5.7 (Charakterisierung von $H_0^1$-Funktionen), muss $v \in H_0^1(\Omega)$ am Rand $\partial \Omega = \Bbraces{0, 1}$ verschwinden.

\begin{align*}
  \implies
  v(0) = v(1) = 0
\end{align*}

\includegraphicsunboxed{PDEs/PDEs_-_Satz_5-6_(Spur_von_Sobolevfunktionen).png}
\includegraphicsunboxed{PDEs/PDEs_-_Satz_5-7_(Charakterisierung_von_H_0^1-Funktionen).png}

Das müsste auch für $u$ gelten.
$u$ setzen wir daher an als quadratischen Spline mit Stützstellen $0, \frac{1}{2}, 1$ an.

\begin{align*}
  u(x)
  & :=
  \begin{cases}
    c_1 x^2 + c_2 x + c_3, & 0 \leq x < \frac{1}{2}, \\
    c_4 x^2 + c_5 x + c_6, & \frac{1}{2} < x \leq 1
  \end{cases} \\
  \implies
  u^\prime(x)
  & =
  \begin{cases}
    2 c_1 x + c_2, & 0 \leq x < \frac{1}{2}, \\
    2 c_4 x + c_5, & \frac{1}{2} < x \leq 1
  \end{cases} \\
  \implies
  u^\primeprime(x)
  & =
  \begin{cases}
    2 c_1, & 0 \leq x < \frac{1}{2}, \\
    2 c_4, & \frac{1}{2} < x \leq 1
  \end{cases}
\end{align*}

Aus unserer oberern Umformulierung können wir nun folgende Bedingungen ziehen.

\begin{align*}
  \implies
  \begin{cases}
    u^\prime \pbraces{\frac{1}{2} -} = 2 u^\prime \pbraces{\frac{1}{2} +}
    & \implies 2 (2 c_1 \pbraces{\frac{1}{2}} + c_2) = 2 c_4 \pbraces{\frac{1}{2}} + c_5 \\
    u^\primeprime(x)
    =
    \begin{cases}
      -1,           & 0 \leq x < \frac{1}{2}, \\
      -\frac{1}{2}, & \frac{1}{2} < x \leq 1
    \end{cases}
    & \implies 2 c_1 = -1, \quad 2 c_4 = -\frac{1}{2} \\
    u\pbraces{\frac{1}{2} -} = u\pbraces{\frac{1}{2} +}
    & \implies c_1 \pbraces{\frac{1}{2}}^2 + c_2 \pbraces{\frac{1}{2}} + c_3 = c_4 \pbraces{\frac{1}{2}}^2 + c_5 \pbraces{\frac{1}{2}} + c_6 \\
    u(0) = 0
    & \implies c_3 = 0 \\
    u(1) = 0
    & \implies c_4 + c_5 + c_6 = 0
    \end{cases}
\end{align*}

Das können wir auch schön als Lineares Gleichungssystem schreiben und lösen.

\begin{align*}
  \begin{pmatrix}
    2 & 2 & 0 & -1 & -1 & 0 \\
    2 & 0 & 0 &  0 &  0 & 0 \\
    0 & 0 & 0 &  2 &  0 & 0 \\
    \frac{1}{4} & \frac{1}{2} & 1 & -\frac{1}{4} & -\frac{1}{2} & -1 \\
    0 & 0 & 1 &  0 &  0 & 0 \\
    0 & 0 & 0 &  1 &  1 & 1
  \end{pmatrix}
  c
  =
  \begin{pmatrix}
    0 \\ -1 \\ -\frac{1}{2} \\ 0 \\ 0 \\ 0
  \end{pmatrix}
  \implies
  c
  =
  \begin{pmatrix}
    -\frac{1}{2} \\ \frac{11}{24} \\ 0 \\ -\frac{1}{4} \\ \frac{1}{6} \\ \frac{1}{12}
  \end{pmatrix}
\end{align*}

Damit erhalten wir unseren Kandidaten für $u$.

\begin{align*}
  u(x)
  =
  \begin{cases}
    -\frac{1}{2} x^2 + \frac{11}{24} x,              & 0 \leq x < \frac{1}{2}, \\
    -\frac{1}{4} x^2 + \frac{1}{6} x + \frac{1}{12}, & \frac{1}{2} < x \leq 1
  \end{cases}
\end{align*}

Von diesem müssen wir noch zeigen, dass $u \in H_0^1(\Omega)$.

\begin{enumerate}[label = \arabic*.]

  \item Schritt (\Quote{$H^1$}):

  \begin{enumerate}[label = (\roman*)]

    \item $u$ ist stetig, also auch $u^2$.
    $\overline{\Omega}$ ist kompakt.
    Daher ist $u^2$ integrierbar, d.h. $u$ quadratisch integrierbar auf $\Omega$.

    \begin{align*}
      u \in C(\Omega) & \implies u^2 \in C(\Omega) \\
      \overline{\Omega} ~\text{kompakt}~ & \implies u^2 \in L^1(\Omega) \implies u \in L^2(\Omega)
    \end{align*}

    \item Wir zeigen zunächst, dass die distributionelle Ableitung $u^\prime$ von $u$ wie folgt aussieht.

    \begin{align*}
      u^\prime(x)
      \stackrel{!}{=}
      \begin{cases}
        -x + \frac{11}{24},           & 0 \leq x < \frac{1}{2}, \\
        -\frac{1}{2} x + \frac{1}{6}, & \frac{1}{2} < x \leq 1
      \end{cases}
    \end{align*}

    Dazu rechnen wir, via partieller Integration, $\Forall \varphi \in \mathcal{D}(\Omega):$

    \begin{align*}
      \abraces{u^\prime, \varphi}
      & :=
      -\abraces{u, \varphi^\prime} \\
      & =
      -\Int[0][\frac{1}{2}]{\pbraces{-\frac{1}{2} x^2 + \frac{11}{24} x} \varphi^\prime(x)}{x} \\
      & \quad
      -\Int[\frac{1}{2}][1]{\pbraces{-\frac{1}{4} x^2 + \frac{1}{6} x + \frac{1}{12}} \varphi^\prime(x)}{x} \\
      & \stackrel
      {
        \mathrm{PI}
      }{=}
      \Int[0][\frac{1}{2}]{\pbraces{-x + \frac{11}{24}} \varphi(x)}{x}
      +
      \frac{1}{2} x^2 \varphi(x) \Big |_{x=0}^{\frac{1}{2}}
      -
      \frac{11}{24} x \varphi(x) \Big |_{x=0}^{\frac{1}{2}} \\
      & \quad
      \Int[\frac{1}{2}][1]{\pbraces{-\frac{1}{2} x + \frac{1}{6}} \varphi(x)}{x}
      +
      \frac{1}{4} x \varphi(x) \Big |_{x=\frac{1}{2}}^1
      -
      \frac{1}{6} x \varphi(x) \Big |_{x=\frac{1}{2}}^1
      -
      \frac{1}{12} \varphi(x) \Big |_{x=\frac{1}{2}}^1 \\
      & =
      \frac{1}{2} \pbraces{\frac{1}{2}}^2 \varphi \pbraces{\frac{1}{2}} - \frac{11}{24} \pbraces{\frac{1}{2}} \varphi \pbraces{\frac{1}{2}}
      -
      \frac{1}{2} \pbraces{\frac{1}{2}}^2 \varphi \pbraces{\frac{1}{2}} + \frac{1}{6} \pbraces{\frac{1}{2}} \varphi \pbraces{\frac{1}{2}} + \frac{1}{12} \varphi \pbraces{\frac{1}{2}} \\
      & \quad
      +
      \Int[0][1]{u^\prime(x) \varphi(x)}{x} \\
      & =
      \underbrace
      {
        \pbraces
        {
          \frac{1}{8} - \frac{11}{48} - \frac{1}{16} + \frac{1}{12} + \frac{1}{12}
        }
      }_0
      \varphi \pbraces{\frac{1}{2}}
      +
      \Int[0][1]{u^\prime(x) \varphi(x)}{x} \\
      & =
      \Int[0][1]{u^\prime(x) \varphi(x)}{x}
    \end{align*}

    $u^\prime$ hat nur eine Unstetigkeitsstelle (in $\frac{1}{2}$).
    Daher können wir analog zu (ii) argumentieren, dass $u^\prime \in L^2(\Omega)$.

    \begin{align*}
      \implies
      u \in H^1(\Omega)
    \end{align*}

  \end{enumerate}

  \item Schritt (\Quote{$H_0$}):

  $u$ ist stetig und verschwindet am Rand.
  Laut Satz 5.7 (Charakterisierung von $H_0^1$-Funktionen), sind wir fertig.

  \begin{align*}
    u \in C(\Omega)
    \implies
    T(u) = u |_{\partial \Omega} = 0
    \implies
    u \in H_0^1(\Omega)
  \end{align*}

\end{enumerate}

\end{solution}

% --------------------------------------------------------------------------------

\begin{exercise}
Sei
\begin{align}
  \rho(\lambda) := (\lambda - \lambda_1)(\lambda - \lambda_2)^2(\lambda - \lambda_3)^3
\end{align}
mit paarweise verschiedenen $\lambda_1,\lambda_2,\lambda_3$.
\begin{enumerate}[label = \textbf{\alph*)}]
  \item Geben Sie alle Lösungen der zugehörigen linearen Differenzengleichung explizit an.
  \item Zeigen Sie, dass die in a) gefundenen Lösungen wirklich alle Lösungen
  der Differenzengleichung sind. Führen Sie dazu die Schritte im Beweis von
  Theorem 5.25 des Vorlesungsskriptes explizit für dieses Problem durch.
\end{enumerate}
\end{exercise}
\begin{solution}
Durch Ausmultiplizieren kann man das charakteristische Polynom auf die Form $\rho(\lambda) = \sum_{j=0}^{6}\alpha_j \lambda^j$ bringen, somit lautet die zugehörige Lineare Differenzengleichung
\begin{align}\label{diffeq}
  \sum_{j=0}^{6} \alpha_j y_{l+j}, \qquad \ell \in \N_0
\end{align}
\begin{enumerate}[label = \textbf{\alph*)}]
  \item Da die Nullstellen des charakteristischen Polynoms genau $\lambda_1,\lambda_2,\lambda_3$
  mit Vielfachheiten $1,2,3$ sind, sind
  alle Lösungen der Differenzengleichung gegeben durch Linearkombinationen der Folgen
  \begin{align*}
    (\lambda_{l}^{(n,m)})_{l \in \N_0} \text{~~mit~~} n = 1,2,3, \qquad m = 0,...,n-1
  \end{align*}
  die wie folgt definiert sind:
  \begin{align*}
    \lambda_{l}^{(1,0)} =& \lambda_1^l \\
    \lambda_{l}^{(2,0)} =& \lambda_2^l \\
    \lambda_{l}^{(3,0)} =& \lambda_3^l \\
    \lambda_{l}^{(2,1)} =& \begin{cases}
0 & l = 0 \\
l \lambda_2^{l-1} & \, l > 0
\end{cases} \\
    \lambda_{l}^{(3,1)} =& \begin{cases}
0 & l = 0 \\
l \lambda_3^{l-1} & \, l > 0
\end{cases} \\
    \lambda_{l}^{(3,2)} =& \begin{cases}
0 & l \leq 1 \\
l(l-1) \lambda_3^{l-2} & \, l > 1
\end{cases} \\
  \end{align*}

  \item Wir zeigen zunächst, dass es sich bei den Folgen tatsächlich um Lösungen der linearen Differenzengleichung handelt, indem wir sie in \eqref{diffeq} einsetzen. Das machen wir für jedes $m=1,...,3$ jeweils nur einmal exemplarisch.

  \begin{align*}
    \sum_{j=0}^{k} \alpha_j \lambda_{l+j}^{(1,0)} =& \sum_{j=0}^{k} \alpha_j \lambda_1^{l+j} = \lambda_1^l \rho(\lambda_1) = 0 \\
    \sum_{j=0}^{k} \alpha_j \lambda_{l+j}^{(2,1)} =& \sum_{j=0}^{k} \alpha_j (l+j)\lambda_2^{l+j-1} = \lambda_2^l\rho^{(1)}(\lambda_2) = 0 \\
    \sum_{j=0}^{k} \alpha_j \lambda_{l+j}^{(3,2)} =& \sum_{j=0}^{k} \alpha_j (l+j)(l+j-1)\lambda_3^{l+j-2} = \lambda_3^l\rho^{(2)}(\lambda_3) = 0.
  \end{align*}


  Nachdem wir gezeigt haben, dass es sich um Lösungen handelt, stellen wir fest, dass es $k=6$ verschiedene Lösungen sind (den Fundamentalsatz brauchen wir hier also gar nicht). Nun wollen wir noch zeigen, dass die Lösungen linear unabhängig sind und somit tatsächlich den $k$-dimensionalen Lösungsraum aufspannen.

  Lineare Unabhängigkeit zeigen wir, indem wir annehmen
  \begin{align*}
    \Forall \ell \in \N_0: \sum_{n=1}^3 \sum_{m=0}^{n-1} \alpha_{n,m} \lambda_{\ell}^{(n,m)} = 0 .
  \end{align*}
  Für $\ell=0,...,k-1$ führt das zu folgendem linearen Gleichungssystem
  \begin{align*}
    \underbrace{\begin{pmatrix}
      \lambda_1^0 & \lambda_2^0 & 0 & \lambda_3^0 & 0 & 0\\
      \lambda_1^1 & \lambda_2^1 & \lambda_2^0 & \lambda_3^1 & \lambda_3^0 & 0\\
      \lambda_1^2 & \lambda_2^2 & 2\lambda_2^1 & \lambda_3^2 & 2\lambda_3^1 & 2\lambda_3^0\\
      \lambda_1^3 & \lambda_2^3 & 3\lambda_2^2 & \lambda_3^3 & 3\lambda_3^2 & 6\lambda_3^1\\
      \lambda_1^4 & \lambda_2^4 & 4\lambda_2^3 & \lambda_3^4 & 4\lambda_3^3 & 12\lambda_3^2\\
      \lambda_1^5 & \lambda_2^5 & 5\lambda_2^4 & \lambda_3^5 & 5\lambda_3^4 & 20\lambda_3^3
    \end{pmatrix}}_{:=A}
    \begin{pmatrix}
      \alpha_{1,0} \\ \alpha_{2,0} \\ \alpha_{2,1} \\ \alpha_{3,0}  \\ \alpha_{3,1}  \\ \alpha_{3,2}
    \end{pmatrix} =
    \begin{pmatrix}
      0 \\ 0 \\ 0 \\ 0 \\ 0 \\ 0
    \end{pmatrix}.
  \end{align*}
  Die Lösung für das gesuchte Polynom $p \in \P_5$ ($p(x) = \sum_{j=0}^5 a_j x^j$) einer Hermite-Interpolation mit den Stützstellen $\lambda_1, \lambda_2, \lambda_2, \lambda_3, \lambda_3, \lambda_3$ lässt sich durch folgendes Gleichungssystem bestimmen
  \begin{align*}
    A^T  \begin{pmatrix}
      a_0 \\ a_1 \\ a_2 \\ a_3 \\ a_4 \\ a_5
    \end{pmatrix} =
    \begin{pmatrix}
      p(\lambda_1) \\ p(\lambda_2) \\ p^{(1)}(\lambda_2) \\ p(\lambda_3) \\ p^{(1)}(\lambda_3) \\ p^{(2)}(\lambda_3)
    \end{pmatrix}
  \end{align*}

  Da dieses Problem eindeutig lösbar ist, ist $A^T$ regulär und somit auch $A$ und es müssen alle Koeffizienten $\alpha_{n,m} = 0$ sein. Also bekommen wir lineare Unabhängigkeit.
\end{enumerate}
\end{solution}


\section*{Nachtrag: Elementare Untermodelle}

% -------------------------------------------------------------------------------- %

\begin{exercise}[Sufficiency, bias, Rao-Blackwell theorem]

Let $X_1, \dots, X_n$ be i.i.d. $\mathit{Poi}(\lambda)$, with unknown $\lambda > 0$.

\begin{enumerate}[label = (\alph*)]

    \item Show that $Y = \sum_{i=n}^n X_i$ is a sufficient statistic for $\lambda$.

    \item Find an unbiased estimator of $p_r = P(X = r)$, which depends only on $X_1$.
    Find $P(X_1 = r \mid Y = k)$ both for $k \geq r$ and $k < r$.
    Hence use the Rao-Blackwell theorem to improve your estimator of $p_r$.

\end{enumerate}

\end{exercise}

% -------------------------------------------------------------------------------- %

\begin{solution}

\phantom{}

\begin{enumerate}[label = (\alph*)]

    \item This follows directly from Homework 7, Exercise 5, and the theorem from \cite[lecture, 8, slide 48]{EStat}.
    
    \item If we choose $W := \mathbf 1_{X_1 = r}$, then
    
    \begin{align*}
        \E W
        & =
        \sum_{k \in \Z}
            \mathbf 1_{k = r}
            P(X = k) \\
        & =
        P(X = r) \\
        & =
        p_r.
    \end{align*}

    \begin{enumerate}[label = \arabic*.]

        \item Case ($k < r$):

        We get $X_1 = r$ only if there is an $i = 2, \dots, n$, such that $X_i < 0$.
        But this has probability $P(X_i < 0) = 0$, so

        \begin{align*}
            P(X_1 = r \mid Y = k) = 0.
        \end{align*}

        \item Case ($k \geq r$):
        
        \begin{align*}
            P(X_1 = r \mid Y = k)
            & =
            \frac
            {
                P(X_1 = r, Y = k)
            }{
                P(Y = k)
            } \\
            & =
            \frac
            {
                P
                \pbraces
                {
                    X_1 = r,
                    \sum_{i=2}^n X_i = k - r
                }
            }{
                P(Y = k)
            } \\
            & =
            \frac
            {
                P(X_1 = r)
                P \pbraces{\sum_{i=2}^n X_i = k - r}
            }{
                P(Y = k)
            } \\
            & =
            \frac
            {
                \frac{\lambda^r}{r!} \mathrm e^{-\lambda}
                \frac
                {
                    ((n - 1) \lambda)^{k - r}
                }{
                    (k - r)!
                }
                \mathrm e^{-(n-1) \lambda}
            }{
                \frac{(n \lambda)^k}{k!} \mathrm e^{-n \lambda}
            } \\
            & =
            \frac{(n - 1)^{k - r}}{n^k}
            \binom{k}{r}
        \end{align*}

    \end{enumerate}

    We could even go with $\binom{k}{r} = 0$ for $k < r$ and unify both cases, by virtue of the previous formula.

    Now, $W$ is an unbiased estimator of $\tau(\lambda) := \frac{\lambda^r}{r!} \mathrm e^{-\lambda} = P(X = r) = p_r$ and $Y$ a sufficient static for $\lambda$.
    The, by virtue of the Rao-Blackwell theorem on \cite[lecture 8, slide 37]{EStat}, improved estimator is

    \begin{align*}
        \E(W \mid Y = k))
        & =
        \E(\mathbf 1_{X_1 = r} \mid Y = k) \\
        & =
        \sum_{s \in \Z}
            \mathbf 1_{s = r}
            P(X_1 = s \mid Y = k) \\
        & =
        P(X_1 = r \mid Y = k),
    \end{align*}

    for $k \in \Z$.

\end{enumerate}

\end{solution}

% -------------------------------------------------------------------------------- %

\begin{algebraUE}{312}
Zeigen Sie, dass Polynomalgebren $A[X]$ für gegebenes $A$ und $X$ bis auf Äquivalenz
in einer geeigneten Kategorie eindeutig bestimmt sind. In Varietäten sind
Polynomalgebren bis auf Isomorphie eindeutig bestimmt.
\end{algebraUE}
\begin{solution}
Beweis.
\end{solution}



\end{document}
