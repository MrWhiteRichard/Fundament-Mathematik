% --------------------------------------------------------------------------------

\begin{exercise}[207]

Zeigen Sie, dass es
\begin{enumerate}[label = (\alph*)]
	\item höchstens eine
	\item mindestens eine
\end{enumerate}
Ackermannfunktion gibt. (Sobald wir dies bewiesen haben, geben wir der nun eindeutig
bestimmten Ackermannfunktion den Namen $A$.)


Ihr Beweis enthält vermutlich mehrere Hilfssätze, die Sie jeweils mit vollständiger
Induktion beweisen. Geben Sie immer explizit an, von welcher Menge $M$ sie die
Eigenschaften $0 \in M$ und $x \in M \rightarrow x + 1 \in M$ zeigen.
\begin{itemize}
	\item[(c)] Berechnen (oder beschreiben) Sie $A(4,2020)$.
	\item[(d)] Schreiben Sie ein Programm (bzw: Skizzieren Sie einen Algorithmus)
	ohne rekursive Funktionsaufrufe, welches bei Eingabe $x,y$ den Wert von $A(x,y)$
	berechnet. (Sie dürfen davon ausgehen, dass Sie beliebig viel Speicherplatz zur
	Verfügung haben, etwa in Form von mehrdimensionalen Arrays.)
\end{itemize}
\end{exercise}

% --------------------------------------------------------------------------------

\begin{solution}
\phantom{}
\begin{enumerate}[label = (\alph*)]
	\item Seien $a_1,a_2$ zwei beliebige Ackermannfunktionen. \\
	Wir wenden vollständige Induktion zuerst auf die Menge
	\begin{align*}
		M = \{n \in \N: \forall y \in \N: a_1(n,y) = a_2(n,y)\}
	\end{align*}
	Es gilt $0 \in M$, da
	\begin{align*}
		\forall y \in \N: a_1(0,y) = y + 1 = a_2(0,y).
	\end{align*}
	Den Induktionsschritt $(n - 1 \rightarrow n)$ selbst zeigen mir mit einer weiteren Induktion:
	\begin{align*}
		M_n = \{y \in \N: a_1(n,y) = a_2(n,y)\}
	\end{align*}
	Es gilt $0 \in M_n$, da mit der Induktionsvoraussetzung der äußeren Induktion gilt:
	\begin{align*}
		a_1(n,0) = a_1(n-1,1) = a_2(n-1,1) = a_2(n,0).
	\end{align*}
	Gelte nun $y-1 \in M_n$:
	\begin{align*}
		a_1(n,y) = a_1(n-1,a_1(n,y-1)) = a_1(n-1,a_2(n,y-1)) = a_2(n-1,a_2(n,y-1)) = a_2(n,y).
	\end{align*}
	Damit gilt $y \in M_n$ und mit der inneren Induktion erhalten wir $M_n = \N$
	was uns wiederum $n \in M$ liefert, was schließlich $M = \N$ beweist.
	\item Für die Existenz definieren wir die Bedinungen
	\begin{flalign*}
	  &A_n.1\ a(0,y) = y + 1 & \\
	  &A_n.2\ \forall k < n: a(k+1,0) = a(k,1) & \\
	  &A_n.3\ \forall k < n: a(k+1,y+1) = a(k,a(k+1,y)). &
	\end{flalign*}
	und zeigen mit Induktion
	\begin{align*}
		M = \{n \in \N: \exists a_n: \{0,\dots,n\} \times \N \to \N \text{ sodass } A_n.1,A_n.2,A_n.3 \text{ gilt}\}
		\stackrel{!}{=} \N.
	\end{align*}
	Es gilt $0 \in M$, da
	\begin{align*}
		a_0(0,y) := y + 1
	\end{align*}
	die Bedingungen $A_0.1,A_0.2,A_0.3$ erfüllt. \\
	Für den Induktionsschritt $n \rightarrow n+1$ definieren wir
	\begin{align*}
		a_{n+1}(x,y) := \begin{cases}
			a_n(x,y), & x < k \\
			a_n(n,1), & x = n + 1, y = 0 \\
			a_n(n,a_{n+1}(n+1, y - 1)), & x = n + 1, y \neq 0
		\end{cases}
	\end{align*}
	Bedingung $A_{n+1}.1$ wird aufgrund der Induktionsvoraussetzung an das $a_n$ erfüllt.
	Bedingugen $A_{n+1}.2, A_{n+1}.3$ werden aufgrund der Induktionsvoraussetzung
	jedenfalls von $k < n$ erfüllt und aufgrund der Definition von $a_{n+1}$
	auch von $k = n$. \\
	(Formal bräuchte man noch wie im ersten Schritt eine innere Induktion, die
	zeigt, dass durch die Vorschrift im dritten Fall wirklich eine Funktion
	wohldefiniert ist, aber man siehts denk ich.) \\
	Schließlich erfüllt die Funktion $A = \bigcup_{n \in \N}a_n$ die originalen
	Bedingungen $A.1,A.2,A.3$.
	\item Wir induzieren $A(1,n) = n + 2$:
	\begin{align*}
		A(1,0) &= A(0,1) = 1 + 1 = 0 + 2 \\
		A(1,n) &= A(0,A(1,n-1)) = A(0,n+1) = n + 1 + 1 = n + 2.
	\end{align*}
	Wir induzieren $A(2,n) = 2n + 3$:
	\begin{align*}
		A(2,0) &= A(1,1) = 1 + 2 = 0 + 3 \\
		A(2,n) &= A(1,A(2,n-1)) = A(1,2(n-1) + 3) = A(1, 2n + 1) = 2n + 1 + 2 = 2n + 3.
	\end{align*}
	Für $A(3,n)$ stellen wir eine Rekursiongleichung auf:
	\begin{align*}
		A(3,0) &= A(2,1) = 5 \\
		A(3,n) &= A(2,A(3,n-1)) = 2A(3,n-1) + 3.
	\end{align*}
	Die Lösung davon ist gegeben durch
	\begin{align*}
		A(3,n) = 5\prod_{j=1}^n2 + \sum_{i=1}^n3\prod_{j=i+1}^n2 = 5\cdot2^n + 3\sum_{i=1}^n 2^{n-1}
		= 5\cdot2^n + 3(2^n - 1).
	\end{align*}
	Schließlich erhalten wir für $A(4,n)$ die Rekursiongleichung
	\begin{align*}
		A(4,0) &= A(3,1) = 13 \\
		A(4,n) &= A(3,A(4,n-1)) = 5\cdot2^{A(4,n-1)} + 3(2^{A(4,n-1)} - 1).
	\end{align*}
	Ja, und viel Spaß das jetzt noch explizit auszurechnen.
	Wär da nicht die Summe in der Rekursionsgleichung könnte mans noch mit der
	Pfeilnotation darstellen.
	\begin{align*}
		a_0 &= b \\
		a_n &= c\cdot d \uparrow a_{n-1}
	\end{align*}
	hat als Lösungsdarstellung für $n \geq 2$
	\begin{align*}
		a_n = c\cdot (d^c\uparrow\uparrow (n-1))\uparrow d^b.
	\end{align*}
	mit $x\uparrow y := x^y$ und $x\uparrow\uparrow y := \underbrace{x^{x^{\cdots^{x}}}}_{y-\text{mal}}$
	\item
	\begin{flalign*}
	&\textsc{A}(x,y): & \\
	&\text{Wir benötigen ein zweidimensionales erweiterbares Array} A & \\
	& \text{und einen Vektor } S \text{ der Länge } x + 1 \text{ mit Anfangswert überall } -1 & \\
	& A(0,0) := 1 & \\
	& S(0) := 0 & \\
	& \textbf{while } S(x) < y & \\
	& \quad A(0,S(0) + 1) := S(0) + 2 & \\
	& \quad S(0) := S(0) + 1 & \\
	& \quad \textbf{for } i = 1,\dots,x \textbf{ do}& \\
	& \quad \quad \textbf{if } S(i) = -1 \textbf{ do}&\\
	& \quad \quad \quad \textbf{if } S(i-1) \geq 1 \textbf{ do}&\\
	& \quad \quad \quad \quad A(i,0) := A(i - 1, 1) &\\
	& \quad \quad \quad \quad S(i) := 0 &\\
	& \quad \quad \quad \textbf{else} &\\
	& \quad \quad \quad \quad \textbf{break} &\\
	& \quad \quad \quad \textbf{end if} & \\
	& \quad \quad \textbf{else if } S(i-1) \geq A(i, S(i)) \textbf{ do} &\\
	& \quad \quad \quad A(i,S(i)+1) := A(i - 1, A(i, S(i))) &\\
	& \quad \quad \textbf{else} & \\
	& \quad \quad \quad \textbf{break} &\\
	& \quad \quad \textbf{end if} &\\
	& \quad \textbf{end for} &\\
	& \textbf{end while} & \\
	& \textbf{return } A(x,y) & \\
	\end{flalign*}
	Ein Beispiel, damit man sich das besser vorstellen kann: $A(3,0)$
	\begin{flalign*}
		1&: A(0,0) = 1, S(0) = 0 & \\
		2&: A(0,1) = 2, S(0) = 1 & \\
		3&: i = 1, S(1) = -1, S(0) = 1 \geq 1 \implies A(1,0) = A(0,1) = 2, S(1) = 0 & \\
		4&: i = 2, S(2) = -1, S(1) = 0 < 1 \implies \textbf{ break} & \\
		5&: A(0,2) = 3, S(0) = 2 & \\
		6&: i = 1, S(1) = 0, S(0) = 2 \geq A(1,S(1)) = A(1,0) = 2
		\implies A(1,1) = A(0,A(1,0)) = A(0,2) = 3, S(1) = 1 & \\
		7&: i = 2, S(2) = -1, S(1) = 1 \geq 1 \implies A(2,0) = A(1,1) = 3, S(2) = 0 & \\
		8&: i = 3, S(3) = -1, S(2) = 0 < 1 \implies \textbf{ break} & \\
		9&: A(0,3) = 4, S(0) = 3 & \\
		10&: i = 1, S(1) = 1, S(0) = 3 \geq A(1,S(1)) = A(1,1) = 3
		\implies A(1,2) = A(0,A(1,1)) = A(0,3) = 4, S(1) = 2 & \\
		11&: i = 2, S(2) = 0, S(1) = 2 < A(2,S(2)) = A(2,0) = 3
		\implies \textbf{ break} & \\
		12&: A(0,4) = 5, S(0) = 4 & \\
		13&: i = 1, S(1) = 2, S(0) = 4 \geq A(1,S(1)) = A(1,2) = 4
		\implies A(1,3) = A(0,A(1,2)) = A(0,4) = 5, S(1) = 3& \\
		14&: i = 2, S(2) = 0, S(1) = 3 \geq A(2,S(2)) = A(2,0) = 3
		\implies A(2,1) = A(1,A(2,0)) = A(1,3) = 5, S(2) = 1 & \\
		15&: i = 3, S(3) = -1, S(2) = 1 \geq 1 \implies A(3,0) = A(2,1) = 5, S(3) = 0 & \\
		16&: i = 4, S(4) = -1, S(3) = 0 < 1 \implies \textbf{break} & \\
		17&: S(x) = S(3) = 0 = y \implies \textbf{return } A(3,0) = 5 &
	\end{flalign*}
\end{enumerate}

\end{solution}
