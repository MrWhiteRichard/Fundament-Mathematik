% --------------------------------------------------------------------------------

\begin{exercise}[207]

Zeigen Sie, dass es
\begin{enumerate}[label = (\alph*)]
	\item höchstens eine
	\item mindestens eine
\end{enumerate}
Ackermannfunktion gibt. (Sobald wir dies bewiesen haben, geben wir der nun eindeutig
bestimmten Ackermannfunktion den Namen $A$.)


Ihr Beweis enthält vermutlich mehrere Hilfssätze, die Sie jeweils mit vollständiger
Induktion beweisen. Geben Sie immer explizit an, von welcher Menge $M$ sie die
Eigenschaften $0 \in M$ und $x \in M \rightarrow x + 1 \in M$ zeigen.
\begin{itemize}
	\item[(c)] Berechnen (oder beschreiben) Sie $A(4,2020)$.
	\item[(d)] Schreiben Sie ein Programm (bzw: Skizzieren Sie einen Algorithmus)
	ohne rekursive Funktionsaufrufe, welches bei Eingabe $x,y$ den Wert von $A(x,y)$
	berechnet. (Sie dürfen davon ausgehen, dass Sie beliebig viel Speicherplatz zur
	Verfügung haben, etwa in Form von mehrdimensionalen Arrays.)
\end{itemize}
\end{exercise}

% --------------------------------------------------------------------------------

\begin{solution}

\phantom{}

\end{solution}
