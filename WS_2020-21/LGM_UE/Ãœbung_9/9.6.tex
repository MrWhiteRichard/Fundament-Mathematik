% --------------------------------------------------------------------------------

\begin{exercise}[208]

Zeigen Sie, dass $A$ in beiden Argumenten streng monoton ist; außerdem, dass
$A(x,y+1) \leq A(x+1,y)$ für alle $x,y$ gilt.

\end{exercise}

% --------------------------------------------------------------------------------

\begin{solution}
\phantom{}
\begin{itemize}
  \item Wir zeigen zuerst induktiv Monotonie in $y$:
  \begin{align}\label{ind_1.1}
    M = \{n \in \N: \forall y_1 > y_2: a(n,y_1) > a(n,y_2) \land a(n,1) > 1\} \stackrel{!}{=} \N.
  \end{align}
  Es gilt $0 \in M$, da für $y_1 > y_2$
  \begin{align*}
    A(0,y_1) &= y_1 + 1 > y_2 + 1 = A(0,y_2) \\
    A(0,1) &= 2 > 1
  \end{align*}
  Im Induktionsschritt $(n-1) \rightarrow n$ führen wir eine weitere Induktion mit
  \begin{align}\label{ind_1.2}
    M_n = \{m \in \N: a(n,m+1) > a(n,m)\} \stackrel{!}{=} \N.
  \end{align}
  Es gilt $0 \in M_n$, da $n-1 \in M \stackrel{\eqref{ind_1.1}}{\implies}
  a(n-1,1) > 1 \stackrel{\eqref{ind_1.1}}{\implies} a(n-1,a(n-1,1)) > a(n-1,1)$ und daher
  \begin{align*}
    a(n,1) = a(n-1,a(n,0)) = a(n-1,a(n-1,1)) \stackrel{\eqref{ind_1.1}}{>} a(n-1,1) = a(n,0).
  \end{align*}
  Für den Induktionsschritt $(m-1) \in M_n \rightarrow m \in M_n $ rechnen wir
  \begin{align*}
    a(n,m+1) = a(n-1,a(n,m)) \stackrel{\eqref{ind_1.2},\eqref{ind_1.1}}{>} a(n-1,a(n,m-1)) = a(n,m).
  \end{align*}
  Damit gilt $M_n = \N$ und zusammen mit $a(n,1) > a(n,0) = a(n-1,1) \stackrel{\eqref{ind_1.1}}{>} 1$ folgt daraus
  $n \in M$ und damit $M = \N$.
  \item Monotonie in $x$: \\
  Wir zeigen zuerst induktiv
  \begin{align}\label{hilf1}
    M = \{n \in \N: \forall x > 0: a(x,n) > n + 1\} \stackrel{!}{=} \N
  \end{align}
  mit Hilfe von
  \begin{align*}
    M_0 =  \{m \in \N: m = 0 \lor a(m,0) > 1\}
  \end{align*}
  Es gilt klarerweise $0 \in M_0$ und
  \begin{align*}
    a(1,0) &= a(0,1) = 2 > 1, \\
    a(m-1,0) > 1 &\implies a(m,0) = a(m-1,1) \stackrel{\eqref{ind_1.1}}{>} a(m-1,0) > 1.
  \end{align*}
  Damit erhalten wir $M_0 = \N$ und $0 \in M$. \\
  Induktionsschritt $(n-1) \in M \rightarrow n \in M$:
  \begin{align}\label{hilf2}
    M_n =  \{m \in \N: m = 0 \lor a(m,n) > n + 1\} \stackrel{!}{=} \N
  \end{align}
  Es gilt $1 \in M_n$, da
  \begin{align*}
    a(1,n) = a(0,a(1,n-1)) \stackrel{\eqref{hilf1},\eqref{ind_1.1}}{>} a(0,n) = n + 1.
  \end{align*}
  Induktionsschritt: $(m-1) \in M_n \rightarrow m \in M_n$:
  \begin{align*}
    a(m,n) = a(m-1,a(m,n-1)) \stackrel{\eqref{hilf1},\eqref{ind_1.1}}{>}
    a(m-1,n) \stackrel{\eqref{hilf2}}{>} n + 1.
  \end{align*}
  Also folgt $M_n = \N$, damit $n \in M$ und insgesamt $M = \N$. \\
  Jetzt können wir die Monotonie in $x$ zeigen:
  \begin{align}\label{ind2}
    M = \{n \in \N: \forall x_1 > x_2: a(x_1,n) > a(x_2,n)\} \stackrel{!}{=} \N
  \end{align}
  Es gilt $0 \in M$, da
  \begin{align*}
    a(x,0) = a(x-1,1) \stackrel{\eqref{ind_1.1}}{>} a(x-1,0).
  \end{align*}
  Für den Induktionsschritt $(n-1) \rightarrow n$ rechnen wir:
  \begin{align*}
    a(x,n) = a(x-1,a(x,n-1)) \stackrel{\eqref{hilf1},\eqref{ind_1.1}}{>} a(x-1,n)
  \end{align*}
  \item $A(x+1,y) \geq A(x,y+1)$:
  \begin{align}\label{ind3}
    M = \{n \in \N: \forall x \in \N: A(x+1,n) \geq A(x,n+1)\}
  \end{align}
  Es gilt $0 \in M$, da
  \begin{align*}
    \forall x \in \N: A(x+1,0) &= A(x,1).
  \end{align*}
  Induktionsschritt: $(n-1) \rightarrow n$:
  \begin{align*}
    A(x+1,n) = A(x,A(x+1,n-1)) \stackrel{\eqref{ind3},\eqref{ind_1.1}}{\geq} A(x,A(x,n))
    \stackrel{\eqref{ind2}}{\geq} A(x-1,A(x,n)) = A(x,n+1).
  \end{align*}
\end{itemize}

\end{solution}
