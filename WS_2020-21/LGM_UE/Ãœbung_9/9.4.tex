% -------------------------------------------------------------------------------- %

\begin{exercise}[206]

Sei $F(n,k) := f_n(k)$, falls $f_n: \N \to \N$, und $F(n,k) = 0$, falls $f_n$
nicht einstellig ist. Skizzieren Sie ein Programm $P$, das $F$ berechnet
(also bei Eingabe von $n$ und $k$ den Wert $F(n,k)$ ausgibt). \\


Verwenden Sie Ihr Programm $P$, um ein neues Programm $Q$ zu definieren, sodass
die von $Q$ berechnete Funktion $f_Q$ total ist, aber mit keiner Funktion $f_n$
übereinstimmt. (Oder sogar: $\forall n\,\forall k > n: f_Q(k) > f_n(k)$ erfüllt.)
\end{exercise}

% -------------------------------------------------------------------------------- %

\begin{solution}
\begin{flalign*}
  &\textsc{FindeKoeffizienten}(n): & \\
  &\textbf{for } a = 0,\dots,5 \textbf{ do}& \\
	&\quad \textbf{for } b = 0,\dots,n-1 \textbf{ do}& \\
	&\quad \quad \textbf{for } c = 0,\dots,n-1 \textbf{ do}& \\
  &\quad \quad \quad \textbf{if } p(a,b,c) = n \textbf{ then}:& \\
	&\quad \quad \quad \quad \textbf{return} (a,b,c) & \\
	&\quad \quad \quad \textbf{end if}& \\
	&\quad \quad \textbf{end for }& \\
	&\quad \textbf{end for }& \\
	&\textbf{end for }& \\
	&\textbf{return} (0,1,0) & \\
\end{flalign*}
Wir führen eine Hilfsvariable $m$ ein, die die Stelligkeit der Funktion bezeichnet,
da bei der Entstehung durch primitive Rekursion auch Funktionen von höherer
Stelligkeit auftreten werden.
\begin{flalign*}
	&\textsc{P}(n,k): & \\
  &m = \dim(k)& \\
	&(a,b,c) = \textsc{FindeKoeffizienten}(n) & \\
	&\textbf{if } a = 0 \textbf{ then}& \\
	&\quad \textbf{return } 0 & \\
	&\textbf{else if } a = 1 \land m = 1\textbf{ then}& \\
	&\quad \textbf{return } k[1] + 1 & \\
	&\textbf{else if } a = 2 \textbf{ then}& \\
	&\quad \textbf{if } m = b \geq c \geq 1 \textbf{ then}& \\
	&\quad \quad \textbf{return } k[c] & \\
	&\quad \textbf{else}& \\
	&\quad \quad \textbf{return } 0 & \\
	&\quad \textbf{end if}& \\
	&\textbf{else if } a = 3 \textbf{ then}& \\
	&\quad \textbf{return } (P(b,k),P(c,k)) & \\
	&\textbf{else if } a = 4 \textbf{ then}& \\
	&\quad \textbf{return } P(c,F(b,k)) & \\
	&\textbf{else if } a = 5 \textbf{ then}& \\
	&\quad \textbf{if } k[-1] = 0 \textbf{ then}& \\
	&\quad \quad \textbf{return } P(c,k[:-1]) & \\
	&\quad \textbf{else }& \\
	&\quad \quad \textbf{return } P(b,[k-1,P(b,k-1)]) & \\
	&\quad \textbf{end if}& \\
  &\textbf{else }& \\
  &\quad \textbf{return } 0& \\
	&\textbf{end if}&
\end{flalign*}

\begin{flalign*}
	&\textsc{Q}(k): & \\
	& x = 1 & \\
	&\textbf{for } i = 0,\dots,k \textbf{ do} & \\
	&\quad x = x + P(i,k) & \\
	&\textbf{end for}& \\
	&\textbf{return } x
\end{flalign*}
Damit gilt $\forall k \in \N\, \forall n \leq k: Q(k) > P(n,k) = f_n(k)$. \\
Insbesondere existiert zu jedem $n \in \N$ ein $k > n$ mit $Q(k) > f_n(k)$.
Daher kann $Q$ kein Element von $\{f_n: n \in \N\}$ sein.
\end{solution}
