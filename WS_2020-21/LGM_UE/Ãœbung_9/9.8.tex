% -------------------------------------------------------------------------------- %

\begin{exercise}[211]

Schließen Sie aus den vorigen Aufgaben:
\begin{enumerate}[label = \alph*.]
  \item Für jede primitiv rekursive Funktion $f$ gibt es ein $c$ mit $f < A_c$.
  \item Die Funktion $x \mapsto A(x,x)$ ist nicht primitiv rekursiv (aber berechenbar).
\end{enumerate}

\end{exercise}

% -------------------------------------------------------------------------------- %

\begin{solution}
	\phantom{}
	\begin{enumerate}[label = \alph*.]
		\item
			\begin{enumerate}[label = \arabic*.]
				\item Für die konstante Nullfunktion $0: \N^k \to \N: \vv{x} \mapsto 0$
        gilt für beliebiges $\vv{x} \in \N^k$ die Ungleichung
					\begin{align*}
					0(\vv{x}) = 0 < \max \vv{x} + 1 = A(0, \max \vv{x}) = A_0(\max \vv{x})
					\end{align*}

				\item Für die Nachfolgerfunktion $S: \N \to \N: x \mapsto x + 1$ gilt für alle $x \in \N$ wegen der in Aufgabe 208 gezeigten strikten Monotonie von $A$ im ersten Argument
					\begin{align*}
					S(x) = x + 1 = A(0, x) < A(1, x) = A_1(x)
					\end{align*}

				\item Für eine Projektion $\Pi_k^n: \N^n \to \N: \vv{x} \mapsto x_k$
        gilt für beliebiges $\vv{x} \in \N^n$
				\begin{align*}
				\Pi_k^n(\vv{x}) = x_k < \max(\vv{x}) + 1 = A(0, \max(\vv{x})) = A_0(\max \vv{x})
				\end{align*}

				\item Die Abgeschlossenheit bezüglich der Verknüpfung folgt direkt aus Aufgabe 209

				\item Die Abgeschlossenheit bezüglich der primitiven Rekursion folgt direkt aus Aufgabe 210
			\end{enumerate}
		\item Sei $c \in \N$ beliebig. Wir nennen unsere Abbildung $f: \N \to \N: x \mapsto A(x,x)$. Wegen der in Aufgabe 208 gezeigten strengen Monotonie gilt
		\begin{align*}
		A_c(c + 1) = A(c, c + 1) < A(c + 1, c + 1) = f(c + 1)
		\end{align*}
		nach Punkt (a) kann daher $f$ nicht primitiv rekursiv sein.
	\end{enumerate}

\end{solution}
