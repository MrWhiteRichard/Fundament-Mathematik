% --------------------------------------------------------------------------------

\begin{exercise}[200]

Es gibt eine nicht berechenbare totale Funktion $g: \N \times \N \to \N$, sodass
die durch $f_n(k) := g(n,k)$ und $h_k(n) := g(n,k)$ definierten Funktionen
$f_0,f_1,\dots,h_0,h_1,\dots: \N \to \N$ alle berechenbar (sogar primitiv rekursiv) sind. \\
\textit{Hinweis:} Fassen Sie die Funktion aus der vorigen Aufgabe als Teilmenge von
$\N \times \N$ auf.

\end{exercise}

% --------------------------------------------------------------------------------

\begin{solution}

	Wir setzten voraus, dass wir bereits eine nicht berechenbare Funktion $\beta: \N \to \N$ kennen. Nun betrachten wir 
	\begin{align*}
	\alpha: \N \to \N: n \mapsto \sum_{i = 0}^n (\beta(i) + 1)
	\end{align*}
	Diese Funktion ist strikt monoton wachsend, daher auch injektiv und nicht berechenbar, denn sonst wäre auch $\Delta\alpha(n) = \alpha(n + 1) - \alpha(n) = \beta(n+1)$ berechenbar und damit auch $\beta(n) = \beta(n + 1 - 1)$. Nun definieren wir 
	\begin{align*}
	g: \N \times \N \to \N : (n, k) \mapsto 
	\begin{cases}
	0 &, \text{falls }\alpha(n) = k \\
	1 &, \text{sonst}
	\end{cases}
	\end{align*}
	Nun erhalten wir für beliebiges $n \in \N$ 
	\begin{align*}
	f_n: \N \to \N : k \mapsto g(n,k) =  
	\begin{cases}
	0 &, \text{falls } \alpha(n) = k \\
	1 &, \text{sonst}
	\end{cases}
	\end{align*}
	Es gilt $f_n = 1 - \chi_{\{\alpha(n)\}}$ und ist damit berechenbar. Weiters definieren wir für beliebiges $k \in \N$ die Funktion
	\begin{align*}
	h_k: \N \to \N : k \mapsto g(n,k) =  
	\begin{cases}
	0 &, \text{falls } \alpha(n) = k \\
	1 &, \text{sonst}
	\end{cases}
	\end{align*}
	Wegen der Injektivität von $\alpha$ gibt es höchstens ein $l \in \N$ mit $\alpha(l) = k$ und daher ist $h_k$ entweder die konstante Einsfunktion oder an einer Stelle $0$ und sonst $1$, also berechenbar.
	Wäre allerdings $g$ berechenbar, so auch
	\begin{align*}
	\mu g:  \N \to \N: n \mapsto \min\{k \in \N \mid g(n,k) = 0\} = \min\{k \in \N \mid f_n(k) = 0\} = \alpha(n)
	\end{align*}
	was nicht sein kann.
\end{solution}
