\documentclass{article}

% Hier befinden sich Pakete, die wir beinahe immer benutzen ...

\usepackage[utf8]{inputenc}

% Sprach-Paket:
\usepackage[ngerman]{babel}

% damit's nicht so, wie beim Grill aussieht:
\usepackage{fullpage}

% Mathematik:
\usepackage{amsmath, amssymb, amsfonts, amsthm}
\usepackage{bbm}
\usepackage{mathtools, mathdots}

% Makros mit mehereren Default-Argumenten:
\usepackage{twoopt}

% Anführungszeichen (Makro \Quote{}):
\usepackage{babel}

% if's für Makros:
\usepackage{xifthen}
\usepackage{etoolbox}

% tikz ist kein Zeichenprogramm (doch!):
\usepackage{tikz}

% bessere Aufzählungen:
\usepackage{enumitem}

% (bessere) Umgebung für Bilder:
\usepackage{graphicx, subfig, float}

% Umgebung für Code:
\usepackage{listings}

% Farben:
\usepackage{xcolor}

% Umgebung für "plain text":
\usepackage{verbatim}

% Umgebung für mehrerer Spalten:
\usepackage{multicol}

% "nette" Brüche
\usepackage{nicefrac}

% Spaltentypen verschiedener Dicke
\usepackage{tabularx}
\usepackage{makecell}

% Für Vektoren
\usepackage{esvect}

% (Web-)Links
\usepackage{hyperref}

% Zitieren & Literatur-Verzeichnis
\usepackage[style = authoryear]{biblatex}
\usepackage{csquotes}

% so ähnlich wie mathbb
%\usepackage{mathds}

% Keine Ahnung, was das macht ...
\usepackage{booktabs}
\usepackage{ngerman}
\usepackage{placeins}

% special letters:

\newcommand{\N}{\mathbb{N}}
\newcommand{\Z}{\mathbb{Z}}
\newcommand{\Q}{\mathbb{Q}}
\newcommand{\R}{\mathbb{R}}
\newcommand{\C}{\mathbb{C}}
\newcommand{\K}{\mathbb{K}}
\newcommand{\T}{\mathbb{T}}
\newcommand{\E}{\mathbb{E}}
\newcommand{\V}{\mathbb{V}}
\renewcommand{\S}{\mathbb{S}}
\renewcommand{\P}{\mathbb{P}}
\newcommand{\1}{\mathbbm{1}}

% quantors:

\newcommand{\Forall}{\forall \,}
\newcommand{\Exists}{\exists \,}
\newcommand{\ExistsOnlyOne}{\exists! \,}
\newcommand{\nExists}{\nexists \,}
\newcommand{\ForAlmostAll}{\forall^\infty \,}

% MISC symbols:

\newcommand{\landau}{{\scriptstyle \mathcal{O}}}
\newcommand{\Landau}{\mathcal{O}}


\newcommand{\eps}{\mathrm{eps}}

% graphics in a box:

\newcommandtwoopt
{\includegraphicsboxed}[3][][]
{
  \begin{figure}[!h]
    \begin{boxedin}
      \ifthenelse{\isempty{#1}}
      {
        \begin{center}
          \includegraphics[width = 0.75 \textwidth]{#3}
          \label{fig:#2}
        \end{center}
      }{
        \begin{center}
          \includegraphics[width = 0.75 \textwidth]{#3}
          \caption{#1}
          \label{fig:#2}
        \end{center}
      }
    \end{boxedin}
  \end{figure}
}

% braces:

\newcommand{\pbraces}[1]{{\left  ( #1 \right  )}}
\newcommand{\bbraces}[1]{{\left  [ #1 \right  ]}}
\newcommand{\Bbraces}[1]{{\left \{ #1 \right \}}}
\newcommand{\vbraces}[1]{{\left  | #1 \right  |}}
\newcommand{\Vbraces}[1]{{\left \| #1 \right \|}}
\newcommand{\abraces}[1]{{\left \langle #1 \right \rangle}}
\newcommand{\round}[1]{\bbraces{#1}}

\newcommand
{\floorbraces}[1]
{{\left \lfloor #1 \right \rfloor}}

\newcommand
{\ceilbraces} [1]
{{\left \lceil  #1 \right \rceil }}

% special functions:

\newcommand{\norm}  [2][]{\Vbraces{#2}_{#1}}
\newcommand{\diam}  [2][]{\mathrm{diam}_{#1} \: #2}
\newcommand{\diag}  [1]{\mathrm{diag} \: #1}
\newcommand{\dist}  [1]{\mathrm{dist} \: #1}
\newcommand{\mean}  [1]{\mathrm{mean} \: #1}
\newcommand{\erf}   [1]{\mathrm{erf} \: #1}
\newcommand{\id}    [1]{\mathrm{id} \: #1}
\newcommand{\sgn}   [1]{\mathrm{sgn} \: #1}
\newcommand{\supp}  [1]{\mathrm{supp} \: #1}
\newcommand{\arsinh}[1]{\mathrm{arsinh} \: #1}
\newcommand{\arcosh}[1]{\mathrm{arcosh} \: #1}
\newcommand{\artanh}[1]{\mathrm{artanh} \: #1}
\newcommand{\card}  [1]{\mathrm{card} \: #1}
\newcommand{\Span}  [1]{\mathrm{span} \: #1}
\newcommand{\Aut}   [1]{\mathrm{Aut} \: #1}
\newcommand{\End}   [1]{\mathrm{End} \: #1}
\newcommand{\ggT}   [1]{\mathrm{ggT} \: #1}
\newcommand{\kgV}   [1]{\mathrm{kgV} \: #1}
\newcommand{\ord}   [1]{\mathrm{ord} \: #1}
\newcommand{\grad}  [1]{\mathrm{grad} \: #1}
\newcommand{\ran}   [1]{\mathrm{ran} \: #1}
\newcommand{\graph} [1]{\mathrm{graph} \: #1}
\newcommand{\Inv}   [1]{\mathrm{Inv} \: #1}
\newcommand{\pv}    [1]{\mathrm{pv} \: #1}
\newcommand{\GL}    [1]{\mathrm{GL} \: #1}
\newcommand{\Mod}{\mathrm{Mod} \:}
\newcommand{\Th}{\mathrm{Th} \:}
\newcommand{\Char}{\mathrm{char}}
\newcommand{\At}{\mathrm{At}}
\newcommand{\Ob}{\mathrm{Ob}}
\newcommand{\Hom}{\mathrm{Hom}}
\newcommand{\orthogonal}[3][]{#2 ~\bot_{#1}~ #3}
\newcommand{\Rang}{\mathrm{Rang}}
\newcommand{\NIL}{\mathrm{NIL}}
\newcommand{\Res}{\mathrm{Res}}
\newcommand{\lxor}{\dot \lor}
\newcommand{\Div}{\mathrm{div} \:}
\newcommand{\meas}{\mathrm{meas} \:}

% fractions:

\newcommand{\Frac}[2]{\frac{1}{#1} \pbraces{#2}}
\newcommand{\nfrac}[2]{\nicefrac{#1}{#2}}

% derivatives & integrals:

\newcommandtwoopt
{\Int}[4][][]
{\int_{#1}^{#2} #3 ~\mathrm{d} #4}

\newcommandtwoopt
{\derivative}[3][][]
{
  \frac
  {\mathrm{d}^{#1} #2}
  {\mathrm{d} #3^{#1}}
}

\newcommandtwoopt
{\pderivative}[3][][]
{
  \frac
  {\partial^{#1} #2}
  {\partial #3^{#1}}
}

\newcommand
{\primeprime}
{{\prime \prime}}

\newcommand
{\primeprimeprime}
{{\prime \prime \prime}}

% Text:

\newcommand{\Quote}[1]{\glqq #1\grqq{}}
\newcommand{\Text}[1]{{\text{#1}}}
\newcommand{\fastueberall}{\text{f.ü.}}
\newcommand{\fastsicher}{\text{f.s.}}

% -------------------------------- %
% amsthm-stuff:

\theoremstyle{definition}

% numbered theorems
\newtheorem{theorem}{Satz}
\newtheorem{lemma}{Lemma}
\newtheorem{corollary}{Korollar}
\newtheorem{proposition}{Proposition}
\newtheorem{remark}{Bemerkung}
\newtheorem{definition}{Definition}
\newtheorem{example}{Beispiel}

% unnumbered theorems
\newtheorem*{theorem*}{Satz}
\newtheorem*{lemma*}{Lemma}
\newtheorem*{corollary*}{Korollar}
\newtheorem*{proposition*}{Proposition}
\newtheorem*{remark*}{Bemerkung}
\newtheorem*{definition*}{Definition}
\newtheorem*{example*}{Beispiel}

% Please define this stuff in project ("main.tex"):

% \def \lastexercisenumber {...}
% This will be 0 by default

% \setcounter{section}{...}
% This will be 0 by default
% and hence, completely ignored

\ifnum \thesection = 0
{\newtheorem{exercise}{Aufgabe}}
\else
{\newtheorem{exercise}{Aufgabe}[section]}
\fi

\ifdef
{\lastexercisenumber}
{\setcounter{exercise}{\lastexercisenumber}}

\newcommand{\solution}
{
    \renewcommand{\proofname}{Lösung}
    \renewcommand{\qedsymbol}{}
    \proof
}

\renewcommand{\proofname}{Beweis}

% -------------------------------- %
% environment zum einkasteln:

% dickere vertical lines
\newcolumntype
{x}
[1]
{!{\centering\arraybackslash\vrule width #1}}

% environment selbst (the big cheese)
\newenvironment
{boxedin}
{
  \begin{tabular}
  {
    x{1 pt}
    p{\textwidth}
    x{1 pt}
  }
  \Xhline
  {2 \arrayrulewidth}
}
{
  \\
  \Xhline{2 \arrayrulewidth}
  \end{tabular}
}

% -------------------------------- %
% MISC "Ein-Deutschungen"

\renewcommand
{\figurename}
{Abbildung}

\renewcommand
{\tablename}
{Tabelle}

% -------------------------------- %


\parindent 0pt

\title
{
  Logik und Grundlagen der Mathematik \\
  \vspace{4pt}
  \normalsize
  \textit{8. Übung am 26.11.2020}
}
\author
{
  Richard Weiss
  \and
  Florian Schager
  \and
  Fabian Zehetgruber
}
\date{}

\begin{document}

\maketitle

\section*{Primitiv rekursive Funktionen}
Wir nennen eine Relation $R \subseteq \N^n$ primitiv rekursiv (genauer: \glqq
primitiv rekursive Relation\grqq\, oder \glqq primitiv rekursiv als Relation\grqq), wenn
ihre charakteristische Funktion $\chi_R$ primitiv rekursiv ist. \\
ACHTUNG: Es gibt Funktionen, die zwar nicht primitiv rekursiv sind, die aber
in diesem Sinn eine primitiv rekursive Relation sind. \\
In jeder der folgenden Aufgaben dürfen Sie die jeweils vorigen Aufgaben verwenden.


\begin{exercise}
Ziel dieser Aufgabe ist eine Beziehung zwischen der linearisierten Stabilität und
dem Konzept der Ljapunovfunktion. Genauer: Wir zeigen die Äquivalenz der folgenden
beiden Aussagen für eine Matrix $A \in R^{d \times d}$:
\begin{enumerate}
  \item Die Ruhelage $y^* = 0$ der ODE $y^{\prime} = Ay$ ist asymptotisch stabil.
  \item Es gibt eine symmetrisch positiv definite Lösung der Matrixgleichung
  (\glqq Ljapunovgleichung\grqq)
  \begin{align} \label{ljapunov}
    A^{\top}Q + QA = -I.
  \end{align}
\end{enumerate}
\begin{enumerate}[label = \textbf{\alph*)}]
  \item Sei die Ruhelage $y^* = 0$ asymptotisch stabil. Definieren Sie die Matrix
  \begin{align*}
    Q := \int_{t = 0}^{\infty} \exp(tA)^{\top}\exp(tA) dt
  \end{align*}
  Zeigen Sie: $Q$ ist symmetrisch positiv definit (insbesondere also $(Qx,x)_2 > 0$
  für $x \neq 0$) und erfüllt \eqref{ljapunov}. Wie Teilaufgabe b) zeigen wird,
  ist die Funktion $V(x) := (Qx,x)_2$ eine strikte Ljapunovfunktion für die obige ODE.
  \item Sei $Q$ eine symmetrisch positiv definite Lösung von \eqref{ljapunov}.
  Zeigen Sie: $V(x) := (Qx,x)_2$ ist eine strikte Ljapunovfunktion für die obige ODE.
  Zeigen Sie: Die Ruhelage $y^* = 0$ ist asymptotisch stabil.
\end{enumerate}
\end{exercise}
\begin{solution}
\begin{enumerate}[label = \textbf{\alph*)}]
  \item Wir zeigen zuerst die Symmetrie:
  \begin{align*}
    Q^{\top} = \left(\int_{t = 0}^{\infty} \exp(tA)^{\top}\exp(tA) dt\right)^{\top}
    = \int_{t = 0}^{\infty} \left(\exp(tA)^{\top}\exp(tA)\right)^{\top} dt
    = \int_{t = 0}^{\infty} \exp(tA)^{\top}\exp(tA) dt = Q.
  \end{align*}
  Für $x \neq 0$ folgt
  \begin{align*}
    x^{\top}Qx &= x^{\top}\int_{t = 0}^{\infty} \exp(tA)^{\top}\exp(tA) dt~x
    = \int_{t = 0}^{\infty} x^{\top}\exp(tA)^{\top}\exp(tA) x~ dt
    = \int_{t = 0}^{\infty} (\exp(tA)x)^{\top}(\exp(tA)x) dt \\
    &= \int_{t = 0}^{\infty} (\exp(tA)x,\exp(tA)x)_2 dt > 0
  \end{align*}
  auch die positive Definitheit. \\
  Aber die entscheidende Frage fehlt noch! Warum existiert das Integral?
  Es wird wohl damit zusammenhängen, dass aufgrund der asymptotischen Stabilität
  von $y^* = 0$ alle Eigenwerte von $A$ negativen Realteil haben müssen.
  Bezeichne also mit $\lambda_1$ den Eigenwert mit größtem Realteil.
  Weiters sei $\omega := \frac{\Re(\lambda_1)}{2} > s(A) := \max \{\Re(\lambda): \lambda \in \sigma(A)\}$
  Daher gilt mit Aufgabe 8.1 die Abschätzung:
  \begin{align*}
    \|\exp(tA)\|_{\infty} &\leq  M\exp(\omega t)
  \end{align*}
  mit einer Konstante $M \in \R.$
  Analoges gilt für $\|\exp(tA)^{\top}\|$, da die Eigenwerte gleich bleiben. Insgesamt erhalten wir also
  \begin{align*}
    \|\exp(tA)^{\top}\exp(tA)\|_{\infty} \leq \widetilde{M}\exp(\Re(\lambda_1) t)
  \end{align*}
  und damit die Existenz des Integrals, da die Abschätzung der Matrixnorm auch
  eine Abschätzung jeder Komponente ist.
  \begin{align*}
    A^{\top}Q + QA &= \int_{t = 0}^{\infty} A^{\top}\exp(tA)^{\top}\exp(A) + \exp(tA)^{\top}\exp(tA)Adt \\
    &= \int_{t = 0}^{\infty} \frac{d}{dt}(\exp(tA)^{\top}\exp(tA)) dt
    = \lim_{t \to \infty} \exp(tA)^{\top}\exp(tA) - I
  \end{align*}
  Aus Satz 5.6 folgt aufgrund der Tatsache, dass $\exp(tA)$ ein Fundamentalsystem
  und der Attraktivität der Ruhelage $y^* = 0$ für $y^{\prime} = Ay$
  ist, dass $\lim_{t \to \infty}\|\exp(tA)\| = 0$ und damit
  \begin{align*}
    A^{\top}Q + QA = \lim_{t \to \infty} \exp(tA)^{\top}\exp(tA) - I = -I.
  \end{align*}
  \item Wir rechnen das hinreichende Kriterium für eine strikte Ljapunovfunktion nach:
  \begin{align*}
    \nabla V(y) \cdot f(y) &= \nabla (Qy,y)_2 Ay = 2y^{\top}Q Ay =
    y^{\top}Q^{\top} Ay + y^{\top}Q Ay = y^{\top}(Q^{\top}A + QA)y \\
    &\stackrel{?????}{=} y^{\top}(A^{\top}Q + QA)y = - y^{\top}y = -(y,y)_2 < 0
  \end{align*}
  Aufgrund der positiven Definitheit gilt für alle $y \neq 0$
  \begin{align*}
    V(y) = (Qy,y)_2 > 0,
  \end{align*}
  also ist $y^* = 0$ ein striktes Minimum von $V$. Wenn jetzt noch zusätzlich $y^*$
  eine isolierte Ruhelage ist, folgt daraus die asymptotische Stabilität.
  Dies ist genau dann der Fall, wenn $A$ regulär ist.
\end{enumerate}
\end{solution}

% -------------------------------------------------------------------------------- %

\begin{exercise}[Exercise 9.1]

Show that tabular methods such as presented in Part I of this book are
a special case of linear function approximation. What would the feature
vectors be?

\end{exercise}

% -------------------------------------------------------------------------------- %

\begin{solution}

\phantom{}

\end{solution}

% -------------------------------------------------------------------------------- %


\section*{Programmierung}
Sei $p: \N^3 \to \N$ eine injektive primitiv rekursive Funktion, die außerdem
$p(a,b,c) > \max(a,b,c)$ für alle $a,b,c$ erfüllt.
Wir definieren eine Folge $f_k$ von primitiv rekursiven Funktionen so: \\
$f_0$ ist die $0$-stellige konstante Funktion $0$, $f_1$ ist die $1$-stellige
konstante Funktion $0$. \\
Für $n = p(a,b,c)$: Wenn $a = 0$, dann ist $f_n$ die konstante $b$-stellige Funktion $0$.
Wenn $a = 1$, dann ist $f_n$ die Nachfolgerfunktion $S: \N \to \N$. Wenn $a = 2$
und $b \geq c \geq 1$, dann
ist $f_n = \pi_c^b$. Wenn $a = 3$, und $f_b: \N^k \to \N^l,\ f_c: \N^k \to \N$, dann
ist $f_n: \N^k \to \N^l \times \N = \N^{l+1}$ als
$f_n(\overline{x}) = (f_b(\overline{x}),f_c(\overline{x}))$.
Wenn $a = 4,\ f_b: \N^k \to \N^l,\ f_c: \N^l \to \N^m$, dann ist $f_n := f_c \circ f_b$.
Wenn $a = 5$ und $f_b$ $(k+2)$-stellig, $f_c$ $k$-stellig,
dann ist $f_n: \N^{k+1} \to \N$ durch $f_n = \mathrm{PR}(f_b,f_c)$ definiert.
Sonst sei $f_n$ die konstante einstellige Nullfunktion.

\begin{exercise}
Betrachten Sie für $\lambda, \mu, \gamma, a > 0$ das System
\begin{align*}
  x^{\prime} &= - \lambda xy - \mu x + \mu a \\
  y^{\prime} &= \lambda xy - \mu y + \gamma y \\
  z^{\prime} &= \gamma y - \mu z.
\end{align*}
Zeigen Sie, dass das System im Fall $a\lambda > \mu - \gamma > 0$ genau eine
nichttriviale Ruhelage $(x^*,y^*,z^*) \in (0,\infty)^3$ hat. Zeigen Sie, dass
die Funktion
\begin{align*}
  V(x,y,z) = x - x^*\ln(x) + y - y^*\ln(y)
\end{align*}
eine Ljapunovfunktion ist. Was können Sie über die Stabilität der Ruhelage $(x^*,y^*,z^*)$
sagen?
\end{exercise}

\begin{solution}
Wir setzen $f$ gleich Null und erhalten drei Gleichungen
\begin{align}
  \mu (a-x) &= \lambda xy \label{1}\\
  y(\lambda x + \gamma - \mu) &= 0 \implies \lambda x  = \mu - \gamma \lor y = 0\\
  \mu z &= \gamma y \label{3}
\end{align}
Fall 1: $y = 0$: \\
Es folgt unmittelbar $z = 0, x = a$ \\
Fall 2: $y \neq 0$: \\
Es folgt $x = \frac{\mu - \gamma}{\lambda} > 0$. Setzen wir in die erste Gleichung ein
erhalten wir
\begin{align*}
  y = \frac{\mu a}{\mu - \gamma} - \frac{\mu}{\lambda}
  \iff y = \frac{\mu(\lambda a - (\mu - \gamma))}{(\mu - \gamma)\lambda} > 0
\end{align*}
Schließlich erhalten wir aus der letzten Gleichung
\begin{align*}
  z = \frac{\gamma}{\mu} \frac{\mu(\lambda a - (\mu - \gamma))}{(\mu - \gamma)\lambda} > 0.
\end{align*}
Damit lautet unsere eindeutige nichttriviale Ruhelage
\begin{align*}
  (x^*,y^*,z^*) = \left(\frac{\mu - \gamma}{\lambda},\frac{\mu(\lambda a - (\mu - \gamma))}{(\mu - \gamma)\lambda},
    \frac{\gamma}{\mu} \frac{\mu(\lambda a - (\mu - \gamma))}{(\mu - \gamma)\lambda}\right).
\end{align*}
Wieder ziehen wir das hinreichende Kriterium für eine strikte Ljapunovfunktion zu Rate
\begin{align*}
  \nabla V(x,y,z)f(x,y,z) &= \left(1 - \frac{x^*}{x}, 1 - \frac{y^*}{y}, 0\right)^{\top}
  \left(- \lambda xy - \mu x + \mu a,\lambda xy - \mu y + \gamma y, \gamma y - \mu z\right) \\
  &= \left(1 - \frac{x^*}{x})(- \lambda xy + \mu (a - x)\right) +
  \left(\left(1 - \frac{y^*}{y}\right)\left(\lambda xy - \mu y + \gamma y\right)\right) \\
  &= - \lambda xy + \mu (a - x) + \lambda x^*y - \mu \frac{x^*(a-x)}{x} + \lambda xy - \mu y + \gamma y
  + y^*(\mu - \gamma - \lambda x)  \\
  &= \mu \left(a - x - \frac{x^*(a-x)}{x}\right) + y(\gamma - \mu + \lambda x^*) + y^*(\mu - \gamma - \lambda x) \\
  &= \mu \left(a - x - \frac{(\mu - \gamma)(a-x)}{\lambda x}\right) + y(\gamma - \mu + \mu - \gamma) +
  y^*(\mu - \gamma - \lambda) \\
  &=\mu \frac{(a-x)(\gamma - \mu + \lambda x)}{\lambda x} + y^*(\mu - \gamma - \lambda x) \\
  &= \mu \frac{(a-x(1+\frac{\lambda}{\mu} y^*))(\gamma - \mu + \lambda x)}{\lambda x} \\
\end{align*}
Der Zähler hat also genau die Nullstellen $x^*, \frac{a}{1 + \frac{\lambda}{\mu}y^*}$.
Aufgrund \eqref{1} folgt
\begin{align*}
  \lambda x^*y + \mu x* = \mu a \iff x^* = \frac{\mu a}{1 + \lambda y*} = \frac{a}{1 + \frac{\lambda}{\mu}y^*}.
\end{align*}
Also hat der Zähler genau eine doppelte Nullstelle und
es kann daher im Zähler keinen Vorzeichenwechsel geben. Da der linke Faktor für $x \to \infty$ sicher negativ wird
und der rechte Faktor für $x \to \infty$ positiv, muss der Nenner insgesamt immer negativ sein.
Da die Ljapunovfunktion für für positive $x$ definiert ist, ist der Nenner immer positiv
und wir haben also eine strikte Ljapunovfunktion gegeben. \\
Setzen wir nun die Ruhelage in die Ljapunovfunktion ein:
\begin{align*}
  \nabla V(x^*,y^*,z^*) = \left(1 - \frac{x^*}{x^*}, 1 - \frac{y^*}{y^*}, 0\right) = (0,0,0).
\end{align*}
Betrachten wir nun die Hesse-Matrix
\begin{align*}
  D^2 V(x^*,y^*,z^*) = \begin{pmatrix}
    1 + \frac{x^*}{(x^*)^2} & 0 & 0 \\
    0 & 1 + \frac{y^*}{(y^*)^2} & 0 \\
    0 & 0 & 0
  \end{pmatrix}
  = \begin{pmatrix}
    1 + \frac{1}{x^*} & 0 & 0\\
    0 & 1 + \frac{1}{y^*} & 0 \\
    0 & 0 & 0 \\
  \end{pmatrix},
\end{align*}
welche positiv semidefinit ist. Wie man leicht erkennt, kann $(x^*,y^*,z^*)$ kein
striktes Minimum von $V$ sein, da $V$ gar nicht von $z$ abhängt. Damit können wir
die direkte Methode von Ljapunov nicht anwenden und somit nichts über die Stabilität
der Ruhelage aussagen. Ist das ein legitimes Resultat?
\end{solution}

% -------------------------------------------------------------------------------- %

\begin{exercise}[Most powerful test for the normal variance - $\mu$ is known]

Let $X_1, \dots, X_n$ be iid $\mathcal N(\mu, \sigma^2)$, where $\mu$ is known.

\begin{enumerate}[label = (\alph*)]

    \item Find an MP test at level $\alpha$ for testing two simple hypoheses
    
    \begin{align*}
        H_0: \sigma^2 = \sigma_0^2
        \quad
        \textit{vs}
        \quad
        H_1: \sigma^2 = \sigma_1^2, \sigma_1 > \sigma_0.
    \end{align*}

    \item Show that the MP teset is a UMP teset for testing
    
    \begin{align*}
        H_0: \sigma^2 \leq \sigma_0^2
        \quad
        \textit{vs}
        \quad
        H_1: \sigma^2 > \sigma_0^2.
    \end{align*}

    \textit{Hint:}
    $\sum_i (X_i - \mu)^2 \sim \sigma^2 \chi^2(n)$.

\end{enumerate}

\end{exercise}

% -------------------------------------------------------------------------------- %

\begin{solution}

First, we show the hint.

\begin{align*}
    \pbraces{\frac{X_i - \mu}{\sigma}}^2 \sim \chi^2(1)
    & \implies
    (X_i - \mu)^2 \sim \sigma^2 \chi^2(1) \\
    & \implies
    T(\mathbf X) := \sum_{i=1}^n (X_i - \mu)^2 \sim \sigma^2 \chi^2(n)
\end{align*}

\begin{enumerate}[label = (\alph*)]

    \item We proceed analogously to the previous exercise.
    
    \begin{align*}
        \lambda(\mathbf x)
        & =
        \frac
        {
            L(\sigma_1 \mid \mathbf x)
        }{
            L(\sigma_0 \mid \mathbf x)
        } \\
        & =
        \frac
        {
            \prod_{i=1}^n
                \frac{1}{\sqrt{2 \pi \sigma_1^2}}
                \mathrm e^{-\frac{(x_i - \mu)^2}{2 \sigma_1^2}}
        }{
            \prod_{i=1}^n
                \frac{1}{\sqrt{2 \pi \sigma_0^2}}
                \mathrm e^{-\frac{(x_i - \mu)^2}{2 \sigma_0^2}}
        } \\
        & =
        \pbraces{\frac{\sigma_0}{\sigma_1}}^2
        \exp \sum_{i=1}^n \frac{(x_i - \mu)^2}{2 \sigma_0} - \frac{(x_i - \mu)^2}{2 \sigma_1} \\
        & =
        \pbraces{\frac{\sigma_0}{\sigma_1}}^2
        \exp \pbraces{\frac{1}{2} \frac{\sigma_1 - \sigma_0}{\sigma_0 \sigma_1} T(\mathbf x)}
    \end{align*}

    \begin{align*}
        \Omega_1
        :=
        \Bbraces{\mathbf x: \lambda(\mathbf x) \geq C}
    \end{align*}

    \begin{align*}
        \lambda(\mathbf x) \geq C
        \iff
        T(\mathbf x) \geq 2 \frac{\sigma_0 \sigma_1}{\sigma_1 - \sigma_0} \ln \pbraces{\pbraces{\frac{\sigma_1}{\sigma_0}}^2 C} =: C^\star
    \end{align*}

    \begin{align*}
        \implies &
        \alpha = P_{\sigma_0}(\lambda(\mathbf X) \geq C) = P_{\sigma_0}(T(\mathbf X) \geq C^\star) = 1 - P_{\sigma_0}(T(\mathbf X) \leq C^\star) \\
        \implies &
        1 - \alpha = P_{\sigma_0}(T(\mathbf X) / \sigma_0^2 \leq C^\star / \sigma_0^2) = F_{\chi^2(n)}(C^\star / \sigma_0^2) \\
        \implies &
        C^\star / \sigma_0^2 = \chi_{1 - \alpha}^2(n) \\
        \implies &
        C^\star = \sigma_0^2 \chi_{1 - \alpha}^2(n)
    \end{align*}

    \item We proceed analogously to the previous exercise.

\end{enumerate}

\end{solution}

% -------------------------------------------------------------------------------- %


\section*{Die Ackermannfunktion}
Eine totale Funktion $a: \N \times \N \to \N$ heißt Ackermannfunktion, wenn sie
für alle $x,y \in \N$ die folgenden Bedingungen erfüllt:
\begin{flalign*}
  &A.1\ a(0,y) = y + 1 & \\
  &A.2\ a(x+1,0) = a(x,1) & \\
  &A.3\ a(x+1,y+1) = a(x,a(x+1,y)). &
\end{flalign*}

% --------------------------------------------------------------------------------

\begin{exercise}[207]

Zeigen Sie, dass es
\begin{enumerate}[label = (\alph*)]
	\item höchstens eine
	\item mindestens eine
\end{enumerate}
Ackermannfunktion gibt. (Sobald wir dies bewiesen haben, geben wir der nun eindeutig
bestimmten Ackermannfunktion den Namen $A$.)


Ihr Beweis enthält vermutlich mehrere Hilfssätze, die Sie jeweils mit vollständiger
Induktion beweisen. Geben Sie immer explizit an, von welcher Menge $M$ sie die
Eigenschaften $0 \in M$ und $x \in M \rightarrow x + 1 \in M$ zeigen.
\begin{itemize}
	\item[(c)] Berechnen (oder beschreiben) Sie $A(4,2020)$.
	\item[(d)] Schreiben Sie ein Programm (bzw: Skizzieren Sie einen Algorithmus)
	ohne rekursive Funktionsaufrufe, welches bei Eingabe $x,y$ den Wert von $A(x,y)$
	berechnet. (Sie dürfen davon ausgehen, dass Sie beliebig viel Speicherplatz zur
	Verfügung haben, etwa in Form von mehrdimensionalen Arrays.)
\end{itemize}
\end{exercise}

% --------------------------------------------------------------------------------

\begin{solution}
\phantom{}
\begin{enumerate}[label = (\alph*)]
	\item Seien $a_1,a_2$ zwei beliebige Ackermannfunktionen. \\
	Wir wenden vollständige Induktion zuerst auf die Menge
	\begin{align*}
		M = \{n \in \N: \forall y \in \N: a_1(n,y) = a_2(n,y)\}
	\end{align*}
	Es gilt $0 \in M$, da
	\begin{align*}
		\forall y \in \N: a_1(0,y) = y + 1 = a_2(0,y).
	\end{align*}
	Den Induktionsschritt $(n - 1 \rightarrow n)$ selbst zeigen mir mit einer weiteren Induktion:
	\begin{align*}
		M_n = \{y \in \N: a_1(n,y) = a_2(n,y)\}
	\end{align*}
	Es gilt $0 \in M_n$, da mit der Induktionsvoraussetzung der äußeren Induktion gilt:
	\begin{align*}
		a_1(n,0) = a_1(n-1,1) = a_2(n-1,1) = a_2(n,0).
	\end{align*}
	Gelte nun $y-1 \in M_n$:
	\begin{align*}
		a_1(n,y) = a_1(n-1,a_1(n,y-1)) = a_1(n-1,a_2(n,y-1)) = a_2(n-1,a_2(n,y-1)) = a_2(n,y).
	\end{align*}
	Damit gilt $y \in M_n$ und mit der inneren Induktion erhalten wir $M_n = \N$
	was uns wiederum $n \in M$ liefert, was schließlich $M = \N$ beweist.
	\item Für die Existenz definieren wir die Bedinungen
	\begin{flalign*}
	  &A_n.1\ a(0,y) = y + 1 & \\
	  &A_n.2\ \forall k < n: a(k+1,0) = a(k,1) & \\
	  &A_n.3\ \forall k < n: a(k+1,y+1) = a(k,a(k+1,y)). &
	\end{flalign*}
	und zeigen mit Induktion
	\begin{align*}
		M = \{n \in \N: \exists a_n: \{0,\dots,n\} \times \N \to \N \text{ sodass } A_n.1,A_n.2,A_n.3 \text{ gilt}\}
		\stackrel{!}{=} \N.
	\end{align*}
	Es gilt $0 \in M$, da
	\begin{align*}
		a_0(0,y) := y + 1
	\end{align*}
	die Bedingungen $A_0.1,A_0.2,A_0.3$ erfüllt. \\
	Für den Induktionsschritt $n \rightarrow n+1$ definieren wir
	\begin{align*}
		a_{n+1}(x,y) := \begin{cases}
			a_n(x,y), & x < k \\
			a_n(n,1), & x = n + 1, y = 0 \\
			a_n(n,a_{n+1}(n+1, y - 1)), & x = n + 1, y \neq 0
		\end{cases}
	\end{align*}
	Bedingung $A_{n+1}.1$ wird aufgrund der Induktionsvoraussetzung an das $a_n$ erfüllt.
	Bedingugen $A_{n+1}.2, A_{n+1}.3$ werden aufgrund der Induktionsvoraussetzung
	jedenfalls von $k < n$ erfüllt und aufgrund der Definition von $a_{n+1}$
	auch von $k = n$. \\
	(Formal bräuchte man noch wie im ersten Schritt eine innere Induktion, die
	zeigt, dass durch die Vorschrift im dritten Fall wirklich eine Funktion
	wohldefiniert ist, aber man siehts denk ich.) \\
	Schließlich erfüllt die Funktion $A = \bigcup_{n \in \N}a_n$ die originalen
	Bedingungen $A.1,A.2,A.3$.
	\item Wir induzieren $A(1,n) = n + 2$:
	\begin{align*}
		A(1,0) &= A(0,1) = 1 + 1 = 0 + 2 \\
		A(1,n) &= A(0,A(1,n-1)) = A(0,n+1) = n + 1 + 1 = n + 2.
	\end{align*}
	Wir induzieren $A(2,n) = 2n + 3$:
	\begin{align*}
		A(2,0) &= A(1,1) = 1 + 2 = 0 + 3 \\
		A(2,n) &= A(1,A(2,n-1)) = A(1,2(n-1) + 3) = A(1, 2n + 1) = 2n + 1 + 2 = 2n + 3.
	\end{align*}
	Für $A(3,n)$ stellen wir eine Rekursiongleichung auf:
	\begin{align*}
		A(3,0) &= A(2,1) = 5 \\
		A(3,n) &= A(2,A(3,n-1)) = 2A(3,n-1) + 3.
	\end{align*}
	Die Lösung davon ist gegeben durch
	\begin{align*}
		A(3,n) = 5\prod_{j=1}^n2 + \sum_{i=1}^n3\prod_{j=i+1}^n2 = 5\cdot2^n + 3\sum_{i=1}^n 2^{n-1}
		= 5\cdot2^n + 3(2^n - 1).
	\end{align*}
	Schließlich erhalten wir für $A(4,n)$ die Rekursiongleichung
	\begin{align*}
		A(4,0) &= A(3,1) = 13 \\
		A(4,n) &= A(3,A(4,n-1)) = 5\cdot2^{A(4,n-1)} + 3(2^{A(4,n-1)} - 1).
	\end{align*}
	Ja, und viel Spaß das jetzt noch explizit auszurechnen.
	Wär da nicht die Summe in der Rekursionsgleichung könnte mans noch mit der
	Pfeilnotation darstellen.
	\begin{align*}
		a_0 &= b \\
		a_n &= c\cdot d \uparrow a_{n-1}
	\end{align*}
	hat als Lösungsdarstellung für $n \geq 2$
	\begin{align*}
		a_n = c\cdot (d^c\uparrow\uparrow (n-1))\uparrow d^b.
	\end{align*}
	mit $x\uparrow y := x^y$ und $x\uparrow\uparrow y := \underbrace{x^{x^{\cdots^{x}}}}_{y-\text{mal}}$
	\item
	\begin{flalign*}
	&\textsc{A}(x,y): & \\
	&\text{Wir benötigen ein zweidimensionales erweiterbares Array} A & \\
	& \text{und einen Vektor } S \text{ der Länge } x + 1 \text{ mit Anfangswert überall } -1 & \\
	& A(0,0) := 1, \quad S(0) := 0 & \\
	& \textbf{while } S(x) < y & \\
	& \quad i := 0, \quad j := S(0) & \\
	& \quad \textbf{while } i = 0 \text{ oder } j = 0  \text{ oder } (j \neq 0 \land A(i, j - 1) < S(i - 1)) & \\
	& \quad \quad \textbf{if } i = 0 & \\
	& \quad \quad \quad A(i,j) := j + 1&\\
	& \quad \quad \textbf{else if } j = 0 &\\
	& \quad \quad \quad A(i,j) := A(i - 1, 1) &\\
	& \quad \quad \textbf{else if } j \neq 0 &\\
	& \quad \quad \quad A(i,j) := A(i - 1, A(i, j - 1)) &\\
	& \quad \quad \textbf{end if} &\\
	& \quad \quad S(i) := S(i) + 1, \quad i := i + 1, \quad j := j - 1 & \\
	& \quad \textbf{end while} &\\
	& \textbf{end while}
	\end{flalign*}
	Ein Beispiel, damit man sich das besser vorstellen kann.
	\begin{align*}
		A =
		\begin{pmatrix}
			(0,0) & (0,1) & (0,2) & (0,3) & \cdots \\
			(1,0) & (1,1) & (1,2) & \cdots & \\
			(2,0) & (2,1) & \cdots & & \\
			(3,0) & \cdots &&&
		\end{pmatrix}
		S =
		\begin{pmatrix}
			3 \\
			0 \\
			-1 \\
			-1 \\
		\end{pmatrix}
	\end{align*}
\end{enumerate}

\end{solution}

\begin{exercise}
Betrachten Sie die folgende Verallgemeinerung eines mathematischen Pendels (ohne Reibung):
\begin{align*}
  y^{\primeprime} + g(y) = 0,
\end{align*}
wobei die Funktion $g$ auf $(-a,a)$ definiert ist und $g(0) = 0, g(x) > 0$ für $x > 0$
und $g(x) < 0$ für $x < 0$ erfüllt ($xg(x) > 0$ für $x \neq 0$). Überführen Sie
diese ODE 2.Ordnung in ein System erster Ordnung. Geben Sie eine Ljapunovfunktion
an. Zeigen Sie, dass $(y,y^{\prime}) = (0,0)$ eine stabile Ruhelage ist.
\end{exercise}
\begin{solution}
System erster Ordnung:
\begin{align*}
  y^{\prime} &=  z\\
  z^{\prime} &= -g(y)
\end{align*}
Unsere Ljapunovfunktion lautet
\begin{align*}
  V(y,z) = \frac{z^2}{2} + \int_0^y g(x) dx
\end{align*}
mit
\begin{align*}
  \nabla V(y,z)f(y,z) = (g(y),z)(z,-g(y)) = 0
\end{align*}
Nun betrachte
\begin{align*}
  \nabla V(0,0) = 0
\end{align*}
Weiters gilt $V(y,z) > 0$ für $(y,z) \neq (0,0)$, da
\begin{align*}
  \int_0^y g(x) dx > 0
\end{align*}
für $y \neq 0$.
Damit ist $(0,0)$ ein striktes Minimum von $V$ und damit nach der direkten Methode
von Ljapunov eine stabile Ruhelage.
\end{solution}

\begin{exercise}
Betrachten Sie das System
\begin{align*}
  x^{\prime} &= -y + x(1 - x^2 - y^2) \\
  y^{\prime} &= x + y(1 - x^2 - y^2)
\end{align*}
Zeigen Sie: Die Mengen $M_1 = \{(0,0)\}, M_2 = \{(x,y): x^2 + y^2 = 1\}$
und $M_3 = \{(x,y): x^2 + y^2 < 1\}$ sind invariante Mengen.
\end{exercise}
\begin{solution}
Betrachte
\begin{align*}
  f(0,0) = (0,0)
\end{align*}
Damit ist $(0,0)$ eine Ruhelage und $M_1$ klarerweise eine invariante Menge. \\
Umrechnen in Polarkoordinaten:
\begin{align}\label{eq1}
  x = r\cos(\phi) \implies x^{\prime} = r^\prime\cos(\phi)  - r\sin(\phi)\phi^{\prime}
\end{align}
\begin{align*}
  y = r\sin(\phi) \implies y^{\prime} = r^{\prime}\sin(\phi) + r\cos(\phi)\phi^{\prime}
\end{align*}
\begin{align*}
  r^2 &= x^2 + y^2 \implies
  2rr^{\prime} = 2xx^{\prime} + 2yy^{\prime} = 2x(-y + x(1 - x^2 - y^2)) + 2y(x + y(1 - x^2 - y^2)) \\
  &= 2x^2(1 - x^2 - y^2) + 2y^2(1 - x^2 - y^2)
  = 2r^2(1 - r^2)
\end{align*}
Also erhalten wir
\begin{align*}
  r^{\prime} = r(1 - r^2).
\end{align*}
Weiters gilt aufgrund \eqref{eq1}
\begin{align*}
  r \sin(\phi)\phi^{\prime} &= -x^{\prime} + r^{\prime}\cos(\phi) = y + (r^2 - 1)x - r(r^2 - 1)\cos(\phi) \\
  &= r\sin(\phi) + (r^2 - 1)(x-\cos(\phi)) =
  r\sin(\phi) \implies \phi^{\prime} = 1.
\end{align*}
In dieser Darstellung lassen sich die letzten beiden Aussagen direkt aus der Differentialgleichung ablesen:
Für $x^2 + y^2 = r^2 = 1$ gilt
\begin{align*}
  r^{\prime} = 1(1-1)= 0
\end{align*}
und damit für alle folgt $\forall t > 0: x_{0,x_0}(t)^2 + y_{0,y_0}(t)^2 = 1$.
Da sich aufgrund der Eindeutigkeit von Anfangswertproblemen Lösungen nicht schneiden
können, muss auch $M_3$ eine invariante Menge sein.
\end{solution}

% --------------------------------------------------------------------------------

\begin{exercise}[211]

Schließen Sie aus den vorigen Aufgaben:
\begin{enumerate}[label = \alph*.]
  \item Für jede primitiv rekursive Funktion $f$ gibt es ein $c$ mit $f < A_c$.
  \item Die Funktion $x \mapsto A(x,x)$ ist nicht primitiv rekursiv (aber berechenbar).
\end{enumerate}

\end{exercise}

% --------------------------------------------------------------------------------

\begin{solution}
	\phantom{}
	\begin{enumerate}[label = \alph*.]
		\item
			\begin{enumerate}[label = \arabic*.]
				\item Für die konstante Nullfunktion $0: \N^k \to \N: x \mapsto 0$ gilt für beliebiges $x \in \N^k$ die Ungleichung 
					\begin{align*}
					0(x) = 0 < x + 1 = A(0, x) = A_0(x)
					\end{align*}
				
				\item Für die Nachfolgerfunktion $S: \N \to \N: x \mapsto x + 1$ gilt für alle $x \in \N$ wegen der in Aufgabe 208 gezeigten strikten Monotonie von $A$ im ersten Argument
					\begin{align*}
					S(x) = x + 1 = A(0, x) < A(1, x) = A_1(x)
					\end{align*} 
				
				\item Für eine Projektion $\Pi_k^n: \N^n \to \N: x \mapsto x_k$ gilt für beliebiges $x \in \N^n$
				\begin{align*}
				\Pi_k^n(x) = x_k < \max(x) + 1 = A(0, \max(x)) = A_0(x)
				\end{align*}
				
				\item Die Abgeschlossenheit bezüglich der Verknüpfung folgt direkt aus Aufgabe 209
				
				\item Die Abgeschlossenheit bezüglich der primitiven Rekursion folgt direkt aus Aufgabe 210
			\end{enumerate} 
	\end{enumerate}
	
\end{solution}


\end{document}
