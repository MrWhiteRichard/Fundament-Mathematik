\documentclass{article}

\def \lastexercisenumber {0}

\input{../../../Fundament-LaTeX/packages_de.tex}
\input{../../../Fundament-LaTeX/macros_de.tex}
% ---------------------------------------------------------------- %
% amsthm-environments:

\theoremstyle{definition}

% numbered theorems
\newtheorem{theorem}             {Satz}[section]
\newtheorem{lemma}      [theorem]{Lemma}
\newtheorem{corollary}  [theorem]{Korollar}
\newtheorem{proposition}[theorem]{Proposition}
\newtheorem{remark}     [theorem]{Bemerkung}
\newtheorem{definition} [theorem]{Definition}
\newtheorem{example}    [theorem]{Beispiel}
\newtheorem{heuristics} [theorem]{Heuristik}

% unnumbered theorems
\newtheorem*{theorem*}    {Satz}
\newtheorem*{lemma*}      {Lemma}
\newtheorem*{corollary*}  {Korollar}
\newtheorem*{proposition*}{Proposition}
\newtheorem*{remark*}     {Bemerkung}
\newtheorem*{definition*} {Definition}
\newtheorem*{example*}    {Beispiel}
\newtheorem*{heuristics*} {Heuristik}

% ---------------------------------------------------------------- %
% exercise- and solution-environments:

\newtheorem{exercise}{Aufgabe}

% if the exercise counter should start at a given exercise number please include the following in the main.tex document
% \setcounter{exercise}{<last exercise number>}

\newenvironment{solution}
{
  \begin{proof}[Lösung]
}{
  \end{proof}
}

% ---------------------------------------------------------------- %
% a tcolorbox-preset designed to mimic the text boxes typically used by Prof. Stefan Hetzl

% starting template:
% https://tex.stackexchange.com/a/527829

% provide box title as optional argument
\newtcolorbox[auto counter]{hetzlbox}[1][]{%
    colback = white,
    coltitle = black,
    fonttitle = \bfseries,
    sharp corners,
    detach title,
    width = 12cm,
    #1,
    code = {\ifdefempty{\tcbtitletext}{}{\tcbset{before upper = {{\centering \tcbtitle \par} \medskip}}}},
    boxrule = 0.5pt
}

% ---------------------------------------------------------------- %
% MISC translations for environment-names

\renewcommand{\proofname} {Beweis}
\renewcommand{\figurename}{Abbildung}
\renewcommand{\tablename} {Tabelle}

% ---------------------------------------------------------------- %

\input{../../../Fundament-LaTeX/listings.tex}

\usepackage{algorithm}

\parskip 0pt
\parindent 0pt

\title
{
  Logik und Grundlagen der Mathematik \\
  \vspace{4pt}
  \normalsize
  \textit{11. Übung am 17.12.2020}
}
\author
{
  Richard Weiss
  \and
  Florian Schager
  \and
  Fabian Zehetgruber
}
\date{}

\begin{document}

\maketitle

\section*{Berechenbare Funktionen und (semi-)entscheidbare Mengen}

Die Menge der $\mu$-rekursiven Funktionen ist die kleinste Menge von (möglicherweise partiellen)
Funktionen, die alle primitiv rekursiven Funktionen enthält, unter Komposition und
primitiver Rekursion abgeschlossen ist, und außerdem Folgendes erfüllt:

\begin{adjustwidth}{1cm}{}
Wenn $f: \N^k \times \N$ total und $\mu$-rekursiv ist, \\
dann ist die partielle Funktion $\vv{x} \mapsto \min\{y: f(\vv{x},y) = 0\}$ auch
$\mu$-rekursiv.
\end{adjustwidth}

% ---------------------------------------------------------------- %

\begin{exercise}[Test power in the $z$-test]

Let $X_1, \dots, X_n$ be i.i.d. random variables with $X_1 \sim N(\mu, \sigma^2)$, and $H_0: \mu = \mu_0$.

\begin{enumerate}[label = (\alph*)]

    \item Compute the teset power of the left-sided $z$-test.
    Express it through cdf of the $N(0, 1)$-distribution, depending on $\mu_0$, $\mu$, $\sigma$, $n$ and the significance level $\alpha$.

    \item Comment on the impact of $\mu_0$, $\mu$, $\sigma$, $n$ and $\alpha$ on the test power.

\end{enumerate}

\end{exercise}

% ---------------------------------------------------------------- %

\begin{solution}

\phantom{}

\begin{enumerate}[label = (\alph*)]

    \item The test reads

    \begin{align*}
        H_0: \mu = \mu_0
        \quad
        \textit{vs}
        \quad
        H_1: \mu > \mu_0,
    \end{align*}
    
    with test statistic
    
    \begin{align*}
        T = \frac{\bar X_n - \mu}{\sigma / \sqrt n} \sim N(0, 1).
    \end{align*}
    
    According to \cite[lecture 9, slide 27]{EStat}, the
    
    \begin{align*}
        \text{$p$-value}
        =
        P(Z \geq T)
        =
        1 - P(Z \leq T)
        =
        1 - \Phi(T).
    \end{align*}
    
    Thus, we get the power
    
    \begin{align*}
        \pi(\mu_0, \mu, \sigma, n, \alpha) 
        & =
        P(\text{accept}~ H_1 \mid H_1 ~\text{is true}) \\
        & =
        P(\text{$p$-value} < \alpha \mid H_1 ~\text{is true}) \\
        & =
        P(1 - \alpha < \Phi(T) \mid H_1 ~\text{is true}) \\
        & =
        P_\mu(T > z_\alpha) \\
        & \stackrel{!}{\approx}
        P_{\mu_0}\pbraces{T > z_\alpha - \frac{\mu - \mu_0}{\sigma/ \sqrt n}} \\
        & =
        1 - \Phi\pbraces{z_\alpha - \frac{\mu - \mu_0}{\sigma/ \sqrt n}} \\
        & =
        \Phi\pbraces{\frac{\mu - \mu_0}{\sigma/ \sqrt n} - z_\alpha}.
    \end{align*}
    
    For \enquote ! we used \cite[lecture 9, slide 46]{EStat}.

    \item Because $\Phi$ is a cdf, it is monotonically increasing.
    The same goes for the \enquote{inverse-cdf} (i.e. quantile function) and thus $\alpha \mapsto z_\alpha = \Phi^{-1}(1 - \alpha)$ is monotonically decreasing.
    Overall, $\pi$

    \begin{itemize}
        \item increases in $\mu$, $n$ and $\alpha$, and
        \item decreases in $\mu_0$ and $\sigma$.
    \end{itemize}

    In order to make the test more powerful, one should increase the

    \begin{itemize}
        \item actual effect size $\frac{\mu - \mu_0}{\sigma}$,
        \item sample size $n$, and
        \item significance level $\alpha$.
    \end{itemize}

\end{enumerate}

\end{solution}

% ---------------------------------------------------------------- %
\begin{exercise}
  Betrachten Sie die Hamilton-Funktion $H(p,q) = \frac{1}{2}p^2 + \frac{1}{2}q^2$.
  Zeigen Sie, dass das symplektische Euler-Verfahren
  \begin{align}
    \begin{pmatrix}
      p_{\ell + 1} \\ q_{\ell + 1}
    \end{pmatrix}
    = \begin{pmatrix}
      p_{\ell} \\ q_{\ell}
    \end{pmatrix} +
    h\begin{pmatrix}
      -\nabla_q H(p_{\ell + 1},q_{\ell}) \\
      \nabla_p H(p_{\ell + 1},q_{\ell})
    \end{pmatrix}
  \end{align}
  im Allgemeinen \textit{nicht} die Energie $H(p,q)$ erhält. Zeigen Sie dazu,
  dass es Anfangsbedingungen $p_0,q_0$ gibt, sodass $H(p_{\ell},q_{\ell})
  \neq H(p_0,q_0)$ für die numerischen Lösungen $p_{\ell},q_{\ell}$ des
  symplektischen Eulerverfahren gilt. \\
  Betrachten Sie weiter die gestörte Hamilton-Funktion
  \begin{align}
    H_h(p,q) := \frac{1}{2}\begin{pmatrix}
      p \\ q
    \end{pmatrix}^{\top}
    \begin{pmatrix}
      1 & - \nicefrac{h}{2} \\ -\nicefrac{h}{2} & 1
    \end{pmatrix}
    \begin{pmatrix}
      p \\ q
    \end{pmatrix}.
  \end{align}
  Zeigen Sie, dass für alle $p,q$ mit $|p|,|q| \leq R \in \R$ gilt, dass
  $\|H(p,q) - H_h(p,q)\| = \Landau{h}$ und dass das symplektische Eulerverfahren
  $H_h$ erhält, also zeigen Sie $H_h(p_{\ell},q_{\ell}) = H_h(p_0,q_0)$.
\end{exercise}

\begin{solution}
Symplektischer Euler in unserem Fall macht
\begin{align*}
\begin{pmatrix}
  p_{\ell + 1} \\ q_{\ell + 1}
\end{pmatrix}
= \begin{pmatrix}
  p_{\ell} \\ q_{\ell}
\end{pmatrix} +
h\begin{pmatrix}
  -q_{\ell} \\
  p_{\ell + 1}
\end{pmatrix},
\end{align*}
also $p_{\ell + 1} = p_{\ell} - hq_{\ell}$ und $q_{\ell + 1} = q_{\ell} + h(p_{\ell} - hq_{\ell})$. \\
Jetzt wähle einfach ganz stumpf $p_0 = q_0 = 1$ mit $H(p_0,q_0) = 1$.
\begin{align*}
  p_1 &= 1 - h, \quad q_1 = 1 + h(1-h) = 1 + h - h^2 \\
  H(p_1,q_1) &= \frac{(1-h)^2+ (1+h-h^2)^2}{2} = \frac{1+h^2-2h+h^4-2h^3-h^2+2h+1}{2}
  = \frac{2+h^4-2h^3}{2} < 1 \text{ für } 0 < h < 2.
\end{align*}
\begin{align*}
  &2H_h(p_{\ell + 1},q_{\ell + 1})
  = \begin{pmatrix}
    p_{\ell + 1} \\ q_{\ell + 1}
  \end{pmatrix}^{\top}
  \begin{pmatrix}
    1 & - \nicefrac{h}{2} \\ -\nicefrac{h}{2} & 1
  \end{pmatrix}
  \begin{pmatrix}
    p_{\ell + 1} \\ q_{\ell + 1}
  \end{pmatrix}
  = p_{\ell + 1}\left(p_{\ell + 1} - \frac{h}{2}q_{\ell + 1}\right) +
  q_{\ell + 1}\left(q_{\ell + 1} - \frac{h}{2}p_{\ell + 1}\right) \\
  &= (p_{\ell} - hq_{\ell})
  \left((p_{\ell} - hq_{\ell}) - \frac{h}{2}(q_{\ell} + h(p_{\ell} - hq_{\ell}))\right) +
  (q_{\ell} + h(p_{\ell} - hq_{\ell}))
  \left((q_{\ell} + h(p_{\ell} - hq_{\ell})) - \frac{h}{2}(p_{\ell} - hq_{\ell})\right) \\
  &= (p_{\ell} - hq_{\ell})
  \left((p_{\ell} - hq_{\ell})\left(1-\frac{h^2}{2}\right) - \frac{h}{2}q_{\ell}\right) +
  (q_{\ell} + h(p_{\ell} - hq_{\ell}))
  \left(q_{\ell} + \frac{h}{2}(p_{\ell} - hq_{\ell})\right) \\
  &= p_{\ell}
  \left(p_{\ell}- \frac{h}{2}q_{\ell}\right)
  + p_{\ell}\left(-\frac{h^2}{2}p_{\ell} - hq_{\ell}\left(1-\frac{h^2}{2}\right)\right)
  -hq_{\ell}\left((p_{\ell} - hq_{\ell})\left(1-\frac{h^2}{2}\right)- \frac{h}{2}q_{\ell}\right) \\
    &+ q_{\ell}\left(q_{\ell} + \frac{h}{2}p_{\ell}\right)
  - \frac{h^2}{2}q_{\ell}^2
  + h(p_{\ell} - hq_{\ell})\left(q_{\ell} + \frac{h}{2}(p_{\ell} - hq_{\ell})\right) \\
  &= 2H_h(p_{\ell},q_{\ell}) +
  p_{\ell}\left(-\frac{h^2}{2}p_{\ell} - hq_{\ell}\left(1-\frac{h^2}{2}\right)\right)
  -q_{\ell}\left(h(p_{\ell} - hq_{\ell})\left(1-\frac{h^2}{2}\right)- \frac{h^2}{2}q_{\ell}\right) \\
  &+ h(p_{\ell} - hq_{\ell})\left(q_{\ell} + \frac{h}{2}(p_{\ell} - hq_{\ell})\right)
  - \frac{h^2}{2}q_{\ell}^2 \\
  &= 2H_h(p_{\ell},q_{\ell}) +
  p_{\ell}\left(-\frac{h^2}{2}p_{\ell} - hq_{\ell}\left(1-\frac{h^2}{2}\right)\right)
  + h(p_{\ell} - hq_{\ell})\left(q_{\ell} + \frac{h}{2}(p_{\ell} - hq_{\ell})-
  q_{\ell}\left(1-\frac{h^2}{2}\right)\right) \\
  &= 2H_h(p_{\ell},q_{\ell}) +
  p_{\ell}\left(-\frac{h^2}{2}p_{\ell} - hq_{\ell}\left(1-\frac{h^2}{2}\right)\right)
  + p_{\ell}\frac{h^2}{2}(p_{\ell} - hq_{\ell}) \\
  &= 2H_h(p_{\ell},q_{\ell}) - p_{\ell}hq_{\ell}
\end{align*}
Wer da noch durchblickt, kann gerne suchen, wo der letzte Term noch wegfällt.
\end{solution}


\section*{Berechenbare Funktionen auf Strings}

Im Folgenden sei $S$ die Menge aller Strings über einem festen Alphabet $A$.
Für $x \in S$ sei $|x|$ die Länge von $x$. \\
Um zu zeigen, dass eine Menge $A$ von Strings entscheidbar oder semi-entscheidbar
ist, geben Sie (informell) einen Algorithmus an, der $\chi_A$ bzw. $\tilde{\chi}_A$
berechnet.

\begin{algebraUE}{361}
Gib das Minimialpolynom von $\sqrt{2} + \sqrt{3}$ und $\sqrt{3} + i$ über $\Q$ an.
\end{algebraUE}

\begin{solution}
  Gib das Minimialpolynom von $\sqrt{2} + \sqrt{3}$ und $\sqrt{3} + i$ über $\Q$ an.

  \begin{itemize}
  \item Folgendes Polynome hat $\sqrt{2} + \sqrt{3}$ als Nullstelle:

  $m_1(x) := x^4 - 10x^2 + 1 = (x + \sqrt{5+\sqrt{24}})(x + \sqrt{5-\sqrt{24}})(x - \sqrt{5+\sqrt{24}})(x - \sqrt{5-\sqrt{24}}).$
  
  \item Folgendes Polynom hat $\sqrt{3} + i$ als Nullstelle:

  $m_2(x) := x^4 - 4x^2 + 16 = (x - \sqrt{3} + i)(x - \sqrt{3} - i)(x + \sqrt{3} + i)(x + \sqrt{3} - i).$

  Hier können wir jeweils die beiden Faktoren, deren Nullstellen sich nur durch Konjugation unterscheiden, zu einem quadratischen Polynom über $\R$ zusammenfassen, welches dann klarerweise irreduzibel ist. In beiden Fällen erhält man Polynome, die nichtrationale Koeffizienten haben. Daher sind beide Polynome irreduzibel über $\Q$.
  \end{itemize}
\end{solution}


\section*{Unentscheidbare Mengen; universelle Mengen}

\begin{algebraUE}{363}
  Seien $\alpha, \beta, \gamma \in \C$ die Nullstellen von $f(x) = x^3 - 2.$ Man bestimme den Grad des Körpers $\Q(\alpha, \beta, \gamma)$ (des Zerfällungskörpers, siehe 6.2.1) über $\Q$.
\end{algebraUE}

\begin{solution}
Die Nullstellen lauten
\begin{align*}
  \alpha = \sqrt[3]{2}, \quad \beta = \sqrt[3]{2}\exp\left(\frac{2\pi i}{3}\right), \quad \gamma = \sqrt[3]{2}\exp\left(\frac{4\pi i}{3}\right).
\end{align*}
Also lässt sich $f$ über $\R$ in irreduzible Polynome über $\R$ faktorisieren:
\begin{align*}
  f(x) = (x - \sqrt[3]{2})(x^2 - 2\sqrt[3]{2}\cos\left(\frac{2\pi}{3}\right) + 2\sqrt[3]{4}) = (x - \sqrt[3]{2})(x^2 + \sqrt[3]{2} x + 2\sqrt[3]{4}).
\end{align*}
Diese Faktoren liegen nicht in $\Q$ und somit ist $f(x)$ über $\Q$ irreduzibel.
Daher ist $f(x)$ das Minimalpolynom von $\alpha$ über $\Q$ und es gilt nach Satz 6.1.3.4
$[\Q(\alpha): Q] = \grad(f) = 3$. \\
Da $\alpha \in \R$ gilt sicher auch $\Q(\alpha) \subset \R$ und damit ist
\begin{align*}
  g(x) = (x^2 + \sqrt[3]{2} x + 2\sqrt[3]{4})
\end{align*}
irreduzibel über $\Q(\alpha)$ und somit das Minimalpolynom von $\beta$ über $\Q(\alpha)$.
Wieder folgt $[\Q(\alpha)(\beta): \Q(\alpha)] = \grad(g) = 2$. \\
Für $\gamma$ bemerken wir, dass $\alpha + \beta + \gamma = 0 \iff \gamma = -\alpha - \beta \in \Q(\alpha)(\beta)$,
also $\Q(\alpha)(\beta)(\gamma) = \Q(\alpha)(\beta)$. \\
Fassen wir zusammen:
\begin{align*}
  [\Q(\alpha,\beta,\gamma): \Q] =  [\Q(\alpha)(\beta)(\gamma): \Q]
  = [\Q(\alpha)(\beta): \Q] = [\Q(\alpha)(\beta): \Q(\alpha)]\cdot[\Q(\alpha): \Q] = 3\cdot2= 6.
\end{align*}
Dabei verwenden wir den Gradsatz und die Tatsache, dass $\Q(\alpha,\beta,\gamma) = \Q(\alpha)(\beta)(\gamma)$,
welche aus der Definition klar ist.
\end{solution}


\section*{Wohlordnungen}

Eine strikte lineare Ordnung $(A, <)$ (mit der zugehörigen reflexiven Ordnung $\leq$)
heißt Wohlordnung, wenn jede nicht-leere Teilmenge von $A$ ein kleinstes Element hat:
$\forall B \subseteq A: (B \neq \emptyset \Rightarrow \Exists b \in B\, \Forall x \in B: b \leq x)$

\begin{algebraUE}{372'}
Zeigen Sie, dass das regelmäßige Siebeneck mit Radius 1 nicht konstruierbar ist.
\end{algebraUE}

\begin{solution}
  Wir definieren eine komplexe Zahl $z \in \C$ sei genau dann konstruierbar, wenn ihr
  Real- und ihr Imaginärteil es sind. Dann behaupten wir, dass jedes
  aus einer Menge von Punkten $A \supseteq \{(0,0),(0,1)\}$ konstruierbare $z := a + ib \in \C$,
  in einer Quadratwurzelerweiterung von $\Q(B)$ liegt, wobei
  $B$ die Menge der Koordinaten aus $A$ bezeichnet.
  Mit Satz 6.1.6.8 erhalten wir, dass $a,b$ jeweils in einer Quadratwurzelerweiterung $L_a,L_b$ von $\Q(B)$ liegen.
  Jetzt können wir die Quadratwurzelerweiterungsschritte einfach aneinanderreihen
  und erhalten eine Quadratwurzelerweiterung $L$, welche $a$ und $b$ erhält.
  Diese können wir dann gegebenfalls noch mit $i$ erweitern. Da $i^2 = -1$
  sicher in $L$ liegt, ist auch $L(i)$ eine Quadratwurzelerweiterung. \\
  Wenn wir das regelmäßige Siebeneck konstruieren können, können wir es auch um
  den Mittelpunkt $(0,0)$ konstruieren. In diesem Fall stimmen die Koordinaten der Eckpunkte genau
  mit den Real- und Imaginärteilen siebten komplexen Einheitswurzeln $\zeta_7^i, i = 1,\dots,7$ überein.
  Wir zeigen, dass die primitive siebte Einheitswurzel $\zeta_7$ nicht in einer
  Quadratwurzelerweiterung liegen kann und mit der Kontraposition der vorigen Aussage
  nicht konstruierbar sein kann. Nach Definition ist $\zeta_7$ eine Nullstelle des Polynoms $x^7-1.$ Durch Polynomdivision erhält man die Faktorisierung
  \begin{align}
      x^7-1 = (x-1)\underbrace{(x^6+x^5+x^4+x^3+x^2+x+1)}_{=: m(x)}
  \end{align}
  und erkennt, dass $\zeta_7$ Nullstelle des zweiten Faktors sein muss.

  Wäre $m(x)$ reduzibel, so wäre auch $m(x+1)$ reduzibel. Allerdings gilt
  \begin{align}
      m(x+1) = x^6+7x^5+21x^4+35x^3+35x^2+21x+7,
  \end{align}
  mit $7\nmid a_6, 7 | a_i, i = 0,\dots,5, 7^2 \nmid a_0$,
  was nach dem Eisensteinkriterium irreduzibel in $\Z[x]$ ist.
  Laut Proposition 5.3.2.9 ist $f$ dann auch irreduzibel über $\Q[x]$, da
  $\Q$ der Quotientenkörper von $\Z$ ist.
  $m(x)$ ist also ebenfalls irreduzibel über $\Q$ und daher das Minimalpolynom von $\zeta_7$ über $\Q$. 

  Damit gilt $[\mathbb{Q}(\zeta_7):\mathbb{Q}] = \mathrm{grad}(m(x)) = 6$; der Grad der Körpererweiterung ist also nicht Teiler einer Zweierpotenz und $\zeta_7$ folglich nicht konstruierbar. Laut Definition ist folglich Real- oder Imaginärteil nicht konstruierbar und auf jeden Fall
  ist ein Eckpunkt unseres regelmäßigen Siebenecks nicht konstruierbar.


  *** Alternative***

  Wir zeigen, dass wir $\sin(\frac{2 \pi }{7})$ nicht konstruieren können,
  und damit auch nicht das regelmäßige Siebeneck mit Radius 1.

  Es gilt:

  \begin{align*}
    \cos(2 \pi) + i \sin(2 \pi) = (\cos(\frac{2 \pi }{7}) + i \sin(\frac{2 \pi }{7}))^{7}
  \end{align*}

  Ausmultiplizieren und Vergleichen der Imaginärteile (mit Python) führt zu

  \begin{align*}
    0 = \sin(2 \pi) = \sin(\frac{2 \pi }{7}) (-64 \sin(\frac{2 \pi }{7})^{6} + 112 \sin(\frac{2 \pi }{7})^4 - 56 \sin(\frac{2 \pi }{7})^2) + 7).
  \end{align*}

  Es ist also $\sin(\frac{2 \pi }{7})$ Nullstelle vom Polynom

  \begin{align*}
    p(x) = x\underbrace{(-64x^6 + 112x^4 - 56x^2 + 7)}_{=:\tilde{p}(x)}
  \end{align*}

  Das Polynom $\tilde{p}(x)$ ist nach dem Eisensteinkriterium (mit der Primzahl $7$) irreduzibel und somit Minimalpolynom von $\sin(\frac{2 \pi }{7})$.

  Damit gilt $[\mathbb{Q}(\sin(\frac{2 \pi }{7})):\mathbb{Q}] = \mathrm{grad}(\tilde{p}(x)) = 6$; der Grad der Körpererweiterung ist also nicht Teiler einer Zweierpotenz und $\sin(\frac{2 \pi }{7})$ folglich nicht konstruierbar.

\end{solution}

% --------------------------------------------------------------------------------

\begin{exercise}[239]

\phantom{}

\end{exercise}

% --------------------------------------------------------------------------------

\begin{solution}

\phantom{}

\end{solution}

% --------------------------------------------------------------------------------

\begin{exercise}[240]

Definieren Sie die lexikographische Ordnung auf $\{0,1\}^{\N}$. Gibt es ein kleinstes Element?
Zeigen Sie, dass diese Ordnung eine lineare Ordnung aber keine Wohlordnung ist.

\end{exercise}

% --------------------------------------------------------------------------------

\begin{solution}

\begin{align*}
  (x_n)_{n\in\N} <_{\mathrm{lex}} (y_n)_{n\in\N}
  :\iff \exists n_0 \in \N: x_{n_0} < y_{n_0} \land \forall n < n_0: x_n = y_n
\end{align*}
Das kleinste Element ist $x^* :=(0,0,\dots)$.
Die Teilmenge $\{0,1\}^{\N} \setminus \{x^*\}$ hat aber kein kleinstes Element.
Angenommen sie hätte ein kleinstes Element $x^{\prime}$, dann folgt aus
$x^{\prime} \neq x^*$
\begin{align*}
  \exists n_0 \in \N: x^{\prime}_{n_0} = 1
\end{align*}
und wir finden mit $(x^{\primeprime}_n)_{n \in \N} = (\delta_{n_0+1,n})_{n \in \N}$ ein echt kleineres
Element.

\end{solution}


\section*{Bijektionen}

Beachten Sie, dass wir $f(x)$ für den Funktionswert von $f$ an der Stelle $x$ schreiben.
Für $U \subseteq \dom(f)$ nennen wir die Menge $\{f(u): u \in U\}$ nicht $f(U)$
sondern $f[U]$.

% --------------------------------------------------------------------------------

\begin{exercise}[243]

Seien $f: A \to B$ und $g: B \to A$ injektiv. Der Einfachheit halber seien
$A$ und $B$ disjunkt. Zeigen Sie, dass es eine Bijektion $h: A \to B$ gibt,
indem Sie den folgenden Beweis vervollständigen: \\
Wir definieren $A_0 := A, B_0 := B, A_{n+1} := g[B_n], B_{n+1} := f[A_n]$. \\
Sei $X_1 := \bigcup_{k\in\N}A_{2k}\setminus A_{2k+1},
X_2 := \bigcup_{k\in\N}A_{2k+1}\setminus A_{2k+2}, X_3 := \bigcap_{k \in \N} A_k$ \\
und $Y_1 := \bigcup_{k\in\N}B_{2k}\setminus B_{2k+1},
Y_2 := \bigcup_{k\in\N}B_{2k+1}\setminus B_{2k+2}, Y_3 := \bigcap_{k \in \N} B_k$.

\begin{enumerate}[label = \alph*.]
  \item Zeigen Sie, dass $\{X_1,X_2,X_3\}$ eine Partition von $A$ ist.
  \item Definieren Sie $h: A \to B$ mit einer Fallunterscheidung:
  Für $x \in X_1$ verwenden Sie $f$, um $h(x)$ zu definieren, für $x \in X_2$
  hingegen $g$. Und für $x \in X_3?$
  \item Zeigen Sie, dass die so definierte Funktion wohldefiniert ist,
  und überdies eine Bijektion.
\end{enumerate}

\end{exercise}

% --------------------------------------------------------------------------------

\begin{solution}

\phantom{}
\begin{enumerate}[label = \alph*.]
  \item Wir zeigen zuerst mit Induktion $\forall k \in \N: A_{k+2} \subseteq A_{k}$.
  \begin{align*}
    k = 0&: \quad A_2 = g[B_1] = g[f[A_0]] = g[f[A]] \subseteq g[B] \subseteq A = A_0 \\
    (k-1) \rightsquigarrow k&: \quad
    A_{k+2} = g[f[A_{k}]] \subseteq g[f[A_{k-2}]] = A_{k}.
  \end{align*}

  Als nächstes zeigen wir, dass $X_1,X_2,X_3$ paarweise disjunkt sind:
  Sei $x \in X_1 \cap X_2$: \\
  Dann existieren $k_1, k_2$, sodass $x \in A_{2k_1}\setminus A_{2k_1+1}$, sowie
  $x \in A_{2k_2 + 1}\setminus A_{2k_2+2}$. \\
  Fall 1: $k_1 \leq k_2$: Dann ist $x \in A_{2k_2 + 1} \subseteq A_{2k_1 + 1}$. Widerspruch! \\
  Fall 2: $k_1 > k_2$: Dann ist $x \in A_{2k_1} \subseteq A_{2k_2 + 2}$. Widerspruch! \\
  Die Disjunktheit von $X_3$ mit $X_1$ und $X_2$ ist offensichtlich. \\
  Ganz analog sieht man auch, dass $\{Y_1,Y_2,Y_3\}$ eine Partition von $B$ sein muss.

  \item
  Für Fall $x \in X_3$ können wir uns aussuchen, ob wir $f$ oder $g^{-1}$
  zur Definition hernehmen, wir entscheiden uns mal für $f$.
  \begin{align*}
    h(x) = \begin{cases}
      f(x) & x \in X_1 \\
      g^{-1}(x) & x \in X_2 \\
      f(x) & x \in X_3
    \end{cases}
  \end{align*}

  \item Die Wohldefiniertheit ist nur für $x \in X_2$ auf den ersten Blick fraglich,
  da aber $X_2 \subset A_1 = g[B]$ gilt, folgt sie sofort aus der Injektivität von $g$. \\
  Weiters gilt aufgrund der Injektivität von $f$
  \begin{align*}
    f(X_1) &= f\left[\bigcup_{k\in\N}A_{2k}\setminus A_{2k+1}\right]
    = \bigcup_{k\in\N}f\left[A_{2k}\setminus A_{2k+1}\right]
    = \bigcup_{k\in\N}f\left[A_{2k}\right]\setminus f[A_{2k+1}]
    = \bigcup_{k\in\N}B_{2k+1}\setminus B_{2k+2} = Y_2, \\
    g(Y_1) &= g\left[\bigcup_{k\in\N}B_{2k}\setminus B_{2k+1}\right]
    = \bigcup_{k\in\N}g\left[B_{2k}\setminus B_{2k+1}\right]
    = \bigcup_{k\in\N}g\left[B_{2k}] \setminus g[B_{2k+1}\right]
    = \bigcup_{k\in\N}A_{2k+1}\setminus A_{2k+2} = X_2, \\
    f(X_3) &= f\left[\bigcap_{k \in \N} A_k\right]
    = \bigcap_{k \in \N}f\left[A_k\right]
    = \bigcap_{k \in \N}B_{k+1} = Y_3, \\
    g(Y_3) &= g\left[\bigcap_{k \in \N} B_k\right]
    = \bigcap_{k \in \N}g\left[A_k\right]
    = \bigcap_{k \in \N}A_{k+1} = X_3.
  \end{align*}
  Also sehen wir, dass $f: X_1 \to Y_2$ und $g: X_2 \to Y_1$ Bijektionen sind,
  sowie $f: X_3 \to Y_3$ und $g: Y_3 \to X_3$, also können wir $h$ für $x \in X_3$
  tatsächlich auf beide Arten definieren und erhalten in jedem Fall eine Bijektion
  von $A$ nach $B$.
\end{enumerate}


\end{solution}


\end{document}
