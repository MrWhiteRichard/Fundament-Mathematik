% -------------------------------------------------------------------------------- %
\subsection*{230/231}

Für jede Menge $\Sigma$ von geschlossenen Formeln sei $cl(\Sigma)$
die Menge aller aus $\Sigma$ ableitbaren geschlossenen Formeln.
\begin{exercise}[230]

Wenn $\Sigma$ semi-entscheidbar ist, dann auch $cl(\Sigma)$.

\end{exercise}

% -------------------------------------------------------------------------------- %

\begin{solution}
Unter Verwendung von Aufgabe 229: Sei $P$ ein Programm, welches die
charakteristische Funktion die Menge der logischen Axiome berechnet,
und $Q$ ein Programm, das $\tilde{\chi}_\Sigma$ berechnet.

\begin{algorithm}
    \caption{Programm, welches $\tilde{\chi}_{cl(\Sigma)}$ berechnet}
    \begin{algorithmic}[1]
    \Procedure{$\tilde{\chi}_{cl(\Sigma)}$ }{$\sigma$}
        \If{$P(\sigma) = 1$}
        \State \Return 1
        \EndIf
    \EndProcedure
    \end{algorithmic}
\end{algorithm}

\end{solution}

\begin{exercise}[231]

Sei $\Sigma$ semi-entscheidbar. Dann gibt es eine entscheidbare Menge $\Sigma^{\prime}$
mit $cl(\Sigma^{\prime}) = cl(\Sigma)$. \\
\textit{Hinweis:} $\varphi \land \varphi$

\end{exercise}
