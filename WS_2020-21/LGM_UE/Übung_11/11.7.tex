% --------------------------------------------------------------------------------

\begin{exercise}[240]

Definieren Sie die lexikographische Ordnung auf $\{0,1\}^{\N}$. Gibt es ein kleinstes Element?
Zeigen Sie, dass diese Ordnung eine lineare Ordnung aber keine Wohlordnung ist.

\end{exercise}

% --------------------------------------------------------------------------------

\begin{solution}

\begin{align*}
  (x_n)_{n\in\N} <_{\mathrm{lex}} (y_n)_{n\in\N}
  :\iff \exists n_0 \in \N: x_{n_0} < y_{n_0} \land \forall n < n_0: x_n = y_n
\end{align*}
Das kleinste Element ist $x^* :=(0,0,\dots)$.
Die Teilmenge $\{0,1\}^{\N} \setminus \{x^*\}$ hat aber kein kleinstes Element.
Angenommen sie hätte ein kleinstes Element $x^{\prime}$, dann folgt aus
$x^{\prime} \neq x^*$
\begin{align*}
  \exists n_0 \in \N: x^{\prime}_{n_0} = 1
\end{align*}
und wir finden mit $(x^{\primeprime}_n)_{n \in \N} = (\delta_{n_0+1,n})_{n \in \N}$ ein echt kleineres
Element.

\end{solution}
