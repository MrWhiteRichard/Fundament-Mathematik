% -------------------------------------------------------------------------------- %

\begin{exercise}[25]

Sei $\Sigma \cup \Bbraces{A}$ eine Menge von aussagenlogischen Formeln.
Zeigen Sie:

\begin{enumerate}
    \item $\Sigma$ ist genau dann erfüllbar, wenn zumindest eine der Mengen $\Sigma \cup \Bbraces{A}$, $\Sigma \cup \Bbraces{\neg A}$ erfüllbar ist.
    \item $\Sigma$ ist genau dann $\ast$erfüllbar, wenn zumindest eine der Mengen $\Sigma \cup \Bbraces{A}$, $\Sigma \cup \Bbraces{\neg A}$ $\ast$erfüllbar ist.
\end{enumerate}

\end{exercise}

% -------------------------------------------------------------------------------- %

\begin{solution}
\phantom{}
\begin{enumerate}
	\item
	\begin{enumerate}
		\item[`$\Rightarrow$']  Sei $\Sigma$ erfüllbar und $b$ eine Belegung aller in $\Sigma \cup \{A\}$ vorkommenden Variablen, sodass $\hat{b}(\Sigma) = 1$ gilt.
		\begin{enumerate}[label = Fall \arabic*:]
			\item $\hat{b}(A) = 1$. Dann ist $\hat{b}(\Sigma \cup \{A\}) =1$.
			\item $\hat{b}(A) = 0$. Dann ist $\hat{b}(\neg A) = 1$ und damit $\hat{b}(\Sigma \cup \{\neg A\}) =1$.
		\end{enumerate}
		\item[`$\Leftarrow$'] Sei $\Sigma \cup \{A\}$ oder $\Sigma \cup \{\neg A\}$ erfüllbar. In jedem Fall ist natürlich $\Sigma$ als Teilmenge ebenfalls erfüllbar.
	\end{enumerate}
	\item Wir bemerken, dass aus der Erfüllbarkeit auch die *Erfüllbarkeit folgt. Nun werfen wir einen Blick auf die Lösung von Aufgabe 26 und erkennen, dass auch die umgekehrte Implikation gilt. Die Menge $\Sigma$ ist also *erfüllbar genau dann wenn sie erfüllbar ist. Nach dem vorherigen Punkt ist das genau dann der Fall, wenn $\Sigma \cup \{A\}$ oder $\Sigma \cup \{\neg A\}$ erfüllbar ist. \\
  --------------------------------------------------------------------------------\\
  Alternative:
  Wir führen einen Widerspruchsbeweis:
  Angenommen, $\Sigma \cup \Bbraces{A}$, $\Sigma \cup \Bbraces{\neg A}$ sind nicht *erfüllbar.
  Also existieren endliche Teilmengen $E_1, E_2 \subset \Sigma$, sodass für alle Belegungen $b$ gilt
  \begin{align*}
    &\exists \sigma \in E_1 \cup \Bbraces{A}: \hat{b}(\sigma) = 0 \\
    &\exists \sigma \in E_2 \cup \Bbraces{\neg A}: \hat{b}(\sigma) = 0.
  \end{align*}
  Nun wähle eine Belegung $b$, die für alle $\sigma \in E_1 \cup  E_2: \hat{b}(\sigma) = 1$ erfüllt.
  Dann muss aber auch gelten, dass
  \begin{align*}
    &\forall \sigma \in E_1 \cup \Bbraces{A}: \hat{b}(\sigma) = 1 \\
    &\text{oder}\\
    &\forall \sigma \in E_2 \cup \Bbraces{\neg A}: \hat{b}(\sigma) = 1.
  \end{align*}
  Widerspruch!
\end{enumerate}

\end{solution}

% -------------------------------------------------------------------------------- %
