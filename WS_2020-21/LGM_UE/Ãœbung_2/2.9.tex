% --------------------------------------------------------------------------------

\begin{exercise}[48]

Seien $A = \Bbraces{a_1, \dots, a_n}$ und $B = \Bbraces{b_1, \dots, b_k}$ endliche Mengen.
Sei $P = \Bbraces{p_{i, j} \mid 1 \leq i \leq n, 1 \leq j \leq k}$ eine Menge aussagenlogischer Variablen.
Jede Belegung $b$ von $P$ induziert eine Relation $R_b \subseteq A \times B$ durch $(a_i, b_j) \in R_b$ gdw $b(p_{i, j}) = 1$.
Finden Sie aussagenlogische Formeln $\varphi_1, \varphi_2, \varphi_3, \varphi_4$ so dass:

\begin{enumerate}[label = \arabic*.]
    \item $\hat{b}(\varphi_1) = 1$ gdw $R_b$ ist eine Funktion
    \item $\hat{b}(\varphi_2) = 1$ gdw $R_b$ ist eine injektive Funktion
    \item $\hat{b}(\varphi_3) = 1$ gdw $R_b$ ist eine surjektive Funktion
    \item $\hat{b}(\varphi_4) = 1$ gdw $R_b$ ist eine bijektive Funktion
\end{enumerate}

Für welche $(n, k) \in \N \times \N$ ist $\varphi_2$ unerfüllbar? \\

(
    Anmerkung:
    Die Größe der Formeln hängt von $n$ und $k$ ab.
)

\end{exercise}

% --------------------------------------------------------------------------------

\begin{solution}
\phantom{}
\begin{enumerate}[label = \arabic*.]
	\item $\varphi_1 = \bigwedge_{i=1}^n \bigvee_{j=1}^k\left(p_{i,j} \land \bigwedge_{l=1,l\neq j}^{k}\neg p_{i,l}\right)$
	\item $\varphi_2 = \bigwedge_{i=1}^n \bigvee_{j=1}^k\left(p_{i,j} \land \bigwedge_{l=1,l\neq j}^{k}\neg p_{i,l}\right)
  \land  \left(\bigwedge_{j=1}^k \bigvee_{i=1}^n\left(p_{i,j} \land \bigwedge_{l=1,l\neq j}^{n}\neg p_{l,j}\right)\right)$
	\item $\varphi_3 = \bigwedge_{i=1}^n \bigvee_{j=1}^k\left(p_{i,j} \land \bigwedge_{l=1,l\neq j}^{k}\neg p_{i,l}\right) \land
  \left(\bigwedge_{j=1}^k \bigvee_{i=1}^np_{i,j}\right)$
	\item $\varphi_4 = \bigwedge_{i=1}^n \bigvee_{j=1}^k\left(p_{i,j} \land \bigwedge_{l=1,l\neq j}^{k}\neg p_{i,l}\right)
  \land  \left(\bigwedge_{j=1}^k \bigvee_{i=1}^n\left(p_{i,j} \land \bigwedge_{l=1,l\neq j}^{n}\neg p_{l,j}\right)\right) \land
  \left(\bigwedge_{j=1}^k \bigvee_{i=1}^np_{i,j}\right)$
\end{enumerate}
Für $k < n$ gibt es keine injektive Funktion von $A$ nach $B$, also ist $\varphi_2$ unerfüllbar, für $k \geq n$ gibt es schon eine, also ist $\varphi_2$ dann erfüllbar.

\end{solution}

% --------------------------------------------------------------------------------
