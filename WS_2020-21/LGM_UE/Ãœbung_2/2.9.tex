% -------------------------------------------------------------------------------- %

\begin{exercise}[48]

Seien $A = \Bbraces{a_1, \dots, a_n}$ und $B = \Bbraces{b_1, \dots, b_k}$ endliche Mengen.
Sei $P = \Bbraces{p_{i, j} \mid 1 \leq i \leq n, 1 \leq j \leq k}$ eine Menge aussagenlogischer Variablen.
Jede Belegung $b$ von $P$ induziert eine Relation $R_b \subseteq A \times B$ durch $(a_i, b_j) \in R_b$ gdw $b(p_{i, j}) = 1$.
Finden Sie aussagenlogische Formeln $\varphi_1, \varphi_2, \varphi_3, \varphi_4$ so dass:

\begin{enumerate}[label = \arabic*.]
    \item $\hat{b}(\varphi_1) = 1$ gdw $R_b$ ist eine Funktion
    \item $\hat{b}(\varphi_2) = 1$ gdw $R_b$ ist eine injektive Funktion
    \item $\hat{b}(\varphi_3) = 1$ gdw $R_b$ ist eine surjektive Funktion
    \item $\hat{b}(\varphi_4) = 1$ gdw $R_b$ ist eine bijektive Funktion
\end{enumerate}

Für welche $(n, k) \in \N \times \N$ ist $\varphi_2$ unerfüllbar? \\

(
    Anmerkung:
    Die Größe der Formeln hängt von $n$ und $k$ ab.
)

\end{exercise}

% -------------------------------------------------------------------------------- %

\begin{solution}

Nachdem wir Formeln angeben müssen, ist es sicher keine schlechte Idee, die Angabe auch in Formeln aufzuschreiben.

\begin{align*}
  P =
  \begin{Bmatrix}
    p_{11}, & \cdots & p_{1k}, \\
    \vdots  & \ddots & \vdots \\
    p_{n1}, & \cdots & p_{nk}
  \end{Bmatrix},
  \quad
  \begin{matrix}
    A = \Bbraces{a_1, \dots, a_n}, \\
    B = \Bbraces{b_1, \dots, b_k}
  \end{matrix}
\end{align*}

\begin{align*}
  b: p \to \Bbraces{0, 1}
  \rightsquigarrow
  R_b
  =
  \Bbraces
  {
    (a_i, b_j) \in A \times B:
    \:
    \begin{matrix}
      i = 1, \dots, n, \\
      j = 1, \dots, k,
    \end{matrix}
    \quad
    b(p_{ij}) = 1
  }
\end{align*}

\begin{enumerate}[label = \arabic*.]

  \item

  \begin{align*}
    R_b ~\text{Funktion}
    :\iff
    & \Forall a_i \in A:
    \ExistsOnlyOne b_j \in B:
    (a_i, b_j) \in R_b \\
    \iff
    & \Forall a_i \in A:
    \Exists b_j \in B:
    (a_i, b_j) \in R_b
    ~ \land \\
    & \Forall a_i \in A:
    \Forall b_{j_1}, b_{j_2} \in B:
    (a_i, b_{j_1}), (a_i, b_{j_2}) \in R_b
    \implies
    b_{j_1} = b_{j_2} \\
    \iff
    & \Forall a_i \in A:
    \Exists b_j \in B:
    (a_i, b_j) \in R_b
    ~ \land \\
    & \Forall a_i \in A:
    \Forall b_{j_1} \in B:
    (a_i, b_{j_1}) \in R_b
    \implies
    \Forall b_{j_2} \in B \setminus \Bbraces{b_{j_1}}:
    (a_i, b_{j_2}) \not \in R_b
  \end{align*}

  Wir brauchen für unsere Formel also $2$ Bauteile.

  \begin{align*}
    \phi_1
    & :=
    \bigwedge_{i=1}^n \bigvee_{j=1}^kp_{i,j} \\
    \phi_2
    & :=
    \bigwedge_{i=1}^n \bigwedge_{j_1=1}^kp_{i,j_1}
    \to
    \bigwedge_{\substack{j_2=1 \\ j_2\neq j_1}}^{k}\neg p_{i,j_2}
  \end{align*}

  \begin{align*}
    \rightsquigarrow
    \varphi_1 := \phi_1 \land \phi_2
  \end{align*}

  \item Sei $R_b$ eine Funktion.

  \begin{align*}
    R_b ~\text{injektiv}
    :\iff
    & \Forall a_{i_1}, a_{i_2} \in A:
    \Forall b_j \in B:
    (a_{i_1}, b_j), (a_{i_1}, b_j) \in R_b
    \implies
    a_{i_1} = a_{i_2} \\
    \iff
    & \Forall b_j \in B:
    \Forall a_{i_1} \in A:
    (a_{i_1}, b_j) \in R_b
    \implies
    \Forall a_{i_2} \in A \setminus \Bbraces{a_{i_1}}:
    (a_{i_2}, b_j) \not \in R_b
  \end{align*}

  Wir brauchen also noch folgendes, zu $\phi_2$ analoges, Bauteil.

  \begin{align*}
    \phi_3
    :=
    \bigwedge_{j=1}^k \bigwedge_{i_1=1}^np_{i_1,j}
    \to
    \bigwedge_{\substack{i_2=1 \\ i_2\neq i_1}}^{n}\neg p_{i_2,j}
  \end{align*}

  \begin{align*}
    \rightsquigarrow
    \varphi_2
    :=
    \varphi_1 \land \phi_3
  \end{align*}

  % $\varphi_2 = \bigwedge_{i=1}^n \bigvee_{j=1}^k\left(p_{i,j} \land \bigwedge_{l=1,l\neq j}^{k}\neg p_{i,l}\right)
  % \land  \left(\bigwedge_{j=1}^k \bigvee_{i=1}^n\left(p_{i,j} \land \bigwedge_{l=1,l\neq j}^{n}\neg p_{l,j}\right)\right)$

  \item Sei $R_b$ eine Funktion.

  \begin{align*}
    R_b ~\text{surjektiv}
    :\iff
    & \Forall b_j \in B:
    \Exists a_i \in A:
    (a_i, b_j) \in R_b
  \end{align*}

  Wir brauchen also noch folgendes, zu $\phi_1$ analoges, Bauteil.

  \begin{align*}
    \phi_4
    :=
    \bigwedge_{j=1}^k \bigvee_{i=1}^np_{i,j}
  \end{align*}

  \begin{align*}
    \rightsquigarrow
    \varphi_3
    :=
    \varphi_1 \land \phi_4
  \end{align*}

  % $\varphi_3 = \bigwedge_{i=1}^n \bigvee_{j=1}^k\left(p_{i,j} \land \bigwedge_{l=1,l\neq j}^{k}\neg p_{i,l}\right) \land
  % \left(\bigwedge_{j=1}^k \bigvee_{i=1}^np_{i,j}\right)$

  \item Sei $R_b$ eine Funktion.

  \begin{align*}
    R_b ~\text{bijektiv}
    :\iff
    R_b ~\text{injektiv}
    \land
    R_b ~\text{surjektiv}
  \end{align*}

  \begin{align*}
    \rightsquigarrow
    \varphi_4
    :=
    \varphi_2 \land \varphi_3
    =
    \varphi_1 \land \phi_3 \land \phi_4
    =
    \phi_1 \land \phi_2 \land \phi_3 \land \phi_4
  \end{align*}

  % $\varphi_4 = \bigwedge_{i=1}^n \bigvee_{j=1}^k\left(p_{i,j} \land \bigwedge_{l=1,l\neq j}^{k}\neg p_{i,l}\right)
  % \land  \left(\bigwedge_{j=1}^k \bigvee_{i=1}^n\left(p_{i,j} \land \bigwedge_{l=1,l\neq j}^{n}\neg p_{l,j}\right)\right) \land
  % \left(\bigwedge_{j=1}^k \bigvee_{i=1}^np_{i,j}\right)$
\end{enumerate}

Für $k < n$ gibt es keine injektive Funktion von $A$ nach $B$, also ist $\varphi_2$ unerfüllbar, für $k \geq n$ gibt es schon eine, also ist $\varphi_2$ dann erfüllbar.
Das kann man sich auf anschaulich mit einem klassischen Funktions-Diagramm skizzieren:
Zwei getrennte Kreise (Ellipsen), die Punkte beinhalten, die jeweils zwischen den Kreisen verbunden sind.

\end{solution}

% -------------------------------------------------------------------------------- %

\begin{solution}

\phantom{}

\begin{enumerate}[label = \arabic*.]

  \item $\varphi_1 = \bigwedge_{i=1}^n \bigvee_{j=1}^k\left(p_{i,j} \land \bigwedge_{l=1,l\neq j}^{k}\neg p_{i,l}\right)$

  \item $\varphi_2 = \bigwedge_{i=1}^n \bigvee_{j=1}^k\left(p_{i,j} \land \bigwedge_{l=1,l\neq j}^{k}\neg p_{i,l}\right)
  \land  \left(\bigwedge_{j=1}^k \pbraces{\pbraces{\bigvee_{i=1}^n p_{i,j}} \rightarrow \pbraces{\bigvee_{i=1}^n\left(p_{i,j} \land \bigwedge_{l=1,l\neq j}^{n}\neg p_{l,j}\right)}}\right)$

  \item $\varphi_3 = \bigwedge_{i=1}^n \bigvee_{j=1}^k\left(p_{i,j} \land \bigwedge_{l=1,l\neq j}^{k}\neg p_{i,l}\right) \land
  \left(\bigwedge_{j=1}^k \bigvee_{i=1}^np_{i,j}\right)$

  \item $\varphi_4 = \bigwedge_{i=1}^n \bigvee_{j=1}^k\left(p_{i,j} \land \bigwedge_{l=1,l\neq j}^{k}\neg p_{i,l}\right)
  \land  \left(\bigwedge_{j=1}^k \bigvee_{i=1}^n\left(p_{i,j} \land \bigwedge_{l=1,l\neq j}^{n}\neg p_{l,j}\right)\right)$

\end{enumerate}

\end{solution}

% -------------------------------------------------------------------------------- %
