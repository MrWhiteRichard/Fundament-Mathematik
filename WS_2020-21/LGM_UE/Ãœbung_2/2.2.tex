% --------------------------------------------------------------------------------

\begin{exercise}[26]

Geben Sie eine $\ast$-erfüllbare Menge an, die nicht erfüllbar ist.

\end{exercise}

% --------------------------------------------------------------------------------

\begin{solution}
Nach Korollar II.3.14 kann es solch eine Menge nicht geben. Der Beweis folgt.

Sei $\Sigma$ eine unerfüllbare Menge. Für eine beliebige Formel $\varphi \in \Sigma$ definieren wir mit $M_\varphi$ eine äquivalente KNF. Weiters sei für eine ganze Teilmenge $\Theta \subseteq \Sigma$ die Menge $M_\Theta := \bigcup_{\varphi \in \Theta} M_\varphi$. Es sei bemerkt, dass $M_\Theta$ erfüllbar ist, genau dann wenn es eine Belegung gibt, sodass alle $M_\varphi$ erfüllbar sind und das ist genau dann der Fall, wenn es eine Belegung gibt, sodass $\Theta$ erfüllbar ist. Da $\Sigma$ unerfüllbar ist, gilt das also auch für $M_\Sigma$. Daher gibt es eine Resolutionswiderlegung $C_1, \dots, C_n$ von $M_\Sigma$. Sei die endlichen Menge $M_0 \subseteq M_\Sigma$ die Menge aller von der Resolutionswiderlegung verwendeten Klauseln. Da es für $M_0$ eine Resolutionswiderlegung gibt, ist $M_0$ unerfüllbar. Durch `ergänzen der zerrissenen Formeln'` finden wir eine endliche Menge $\Sigma_0 \subseteq \Sigma$ mit $M_0 \subseteq M_{\Sigma_0}$. Da $M_0$ unerfüllbar ist gilt das klarerweise auch für $M_{\Sigma_0}$ und damit auch für die endliche Menge $\Sigma_0$. Daher ist $\Sigma$ nicht *-erfüllbar.

\end{solution}

% --------------------------------------------------------------------------------
