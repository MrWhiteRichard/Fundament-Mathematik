% -------------------------------------------------------------------------------- %

\begin{exercise}[40]

Zeigen Sie, dass die leere Klausel mit (mehrfach ausgeführter) Resolution aus den Klauseln $M = \Bbraces{\Bbraces{\neg p, q}; \Bbraces{\neg r, s}; \Bbraces{p, r}; \Bbraces{\neg q}; \Bbraces{\neg s}}$ herleitbar ist.
Was hat dies mit Aufgabe 21 zu tun?

\end{exercise}

% -------------------------------------------------------------------------------- %

\begin{solution}

\phantom{}

\includegraphicsboxed{../../../Fundament-LaTeX/images/LGM/LGM - Definition II.3.2 (Resolvente).png}
\includegraphicsboxed{../../../Fundament-LaTeX/images/LGM/LGM - Definition II.3.3.png}

Wir zeigen, dass es eine Resolutionswiderlegung gibt.

\begin{align*}
    \Bbraces{q, r}
    & =
    \mathrm{Res}_p(\Bbraces{p, r}, \Bbraces{\neg p, q}) \\
    \Bbraces{q, s}
    & =
    \mathrm{Res}_r(\Bbraces{q, r}, \Bbraces{\neg r, s}) \\
    \Bbraces{s}
    & =
    \mathrm{Res}_q(\Bbraces{q, s}, \Bbraces{\neg q}) \\
    \emptyset
    & =
    \mathrm{Res}_s(\Bbraces{s}, \Bbraces{\neg s})
\end{align*}

\includegraphicsboxed{../../../Fundament-LaTeX/images/LGM/LGM - Satz II.2.6.png}

Laut Satz II.2.6, können wir die Aussagen von Aufgabe 21 in eine DNF bringen.
Wenn wir für eine Formel in DNF aus Aufgabe 21, so wie hier, auf eine Resolutionswiderlegung kommen, dann ist diese keine Tautologie.

\end{solution}

% -------------------------------------------------------------------------------- %

\begin{solution}

Definiere
\begin{align*}
    C_1 &:= \Bbraces{\neg p, q} \\
    C_2 &:= \Bbraces{\neg r, s} \\
    C_3 &:= \Bbraces{p, r} \\
    C_4 &:= \Bbraces{\neg q} \\
    C_5 &:= \Bbraces{\neg s} \\
    C_6 &:= \mathrm{Res}_p(C_3,C_1) = \{q, r\} \\
    C_7 &:= \mathrm{Res}_r(C_6,C_2) = \{q, s\} \\
    C_8 &:= \mathrm{Res}_q(C_7,C_4) = \{s\} \\
    C_9 &:= \mathrm{Res}_s(C_8,C_5) = \emptyset.
\end{align*}
Damit ist $(C_1,\dots,C_9)$ eine Resolutionswiderlegung von $M$. \\
Bei Aufgabe 21 galt es zu entscheiden, ob gewisse Formeln Tautologien sind oder nicht.
Dazu kann man natürlich auch zuerst die Formeln auf KNF bringen und darauf den
Resolutionsalgorithmus anwenden, welcher in endlich vielen Schritten eine
unter Resolution abgeschlossene Menge bringt. Enthält diese die leere Menge,
so ist die Formel widerlegbar, also keine Tautologie, anderenfalls schon.

\end{solution}

% -------------------------------------------------------------------------------- %
