% --------------------------------------------------------------------------------

\begin{exercise}[30]

Welche der folgenden Formeln sind in CNF, welche in DNF?

\begin{align*}
    \neg p_1,
    \neg p_1 \lor p_2,
    \neg (p_1 \lor p_3),
    \neg p_1 \land p_4,
    \neg p_1 \to p_5,
    (\neg p_1 \lor p_6) \land p_7,
    (((p_1 \land p_2) \lor p_3) \land p_4)
\end{align*}

Anmerkung:
\Quote{$\neg$} bindet stärker als die anderen Junktoren;
Daher:
$\neg p_1 \lor p_2 := ((\neg p_1) \lor p_2)$.

\end{exercise}

% --------------------------------------------------------------------------------

\begin{solution}

\phantom{}

\includegraphicsboxed{Definition II.2.1 (Literal).png}
\includegraphicsboxed{Definition II.2.2 (Klausel).png}
\includegraphicsboxed{Definition II.2.4 (Konjunktive Normalform).png}
\includegraphicsboxed{Definition II.2.5 (Disjunktive Normalform).png}

\begin{align*}
    \neg p_1 &: \text{KNF, DNF} \\
    \neg p_1 \lor p_2 &: \text{KNF, DNF} \\
    \neg (p_1 \lor p_3) &: \text{nix} \\
    \neg p_1 \land p_4 &: \text{KNF, DNF} \\
    \neg p_1 \to p_5 &: \text{nix} \\
    (\neg p_1 \lor p_6) \land p_7 &: \text{KNF} \\
    (((p_1 \land p_2) \lor p_3) \land p_4) &: \text{nix}
\end{align*}

\end{solution}

% --------------------------------------------------------------------------------
