% --------------------------------------------------------------------------------

\begin{exercise}[49]

Ein Sudoku ist eine Matrix $S = (s_{i, j)} \in \Bbraces{\lambda, 1, \dots, 0})^{9 \times 9}$ wobei das Symbol $\lambda$ für \Quote{leer} stehen soll.
Eine Lösung von $S$ ist eine Matrix $L = (l_{i, j}) \in \Bbraces{1, \dots, 9}^{9 \times 9}$ so dass gilt:

\begin{enumerate}[label = \arabic*.]

    \item $s_{i, j} \neq \lambda$ impliziert $l_{i, j} = s_{i, j}$, und
    
    \item Für die folgenden $K \subseteq \Bbraces{1, \dots, 9} \times \Bbraces{1, \dots, 9}$ gilt:
    
    \begin{align*}
        (i_1, j_1), (i_2, j_2) \in K,
        (i_1, j_2) \neq (i_2, j_2)
        ~\text{impliziert}~
        l_{i_1, j_1} \neq l_{i_2, j_2}
    \end{align*}

    \begin{enumerate}
        \item Für jede Zeile,
        \item Für jede Spalte,
        \item Für jede 3x3-Matrix mit Startkoordinaten kongruent $1$ modulo $3$
    \end{enumerate}

\end{enumerate}

Finden SIe, ähnlich wie in Aufgabe 48, eine Menge $P$ aussagenlogischer Variablen, eine Bijektion von Belegungen von $P$ mit Relationen über $\Bbraces{1, \dots, 9} \times \Bbraces{1, \dots, 9} \times \Bbraces{1, \dots, 9}$ sowie eine Formel $\varphi_S$ so dass $\hat{b}(\varphi_S) = 1$ gdw $b$ eine Lösung von $S$ induziert.

\end{exercise}

% --------------------------------------------------------------------------------

\begin{solution}

NoToDo!
Wir haben schon Aufgabe 48.

\end{solution}

% --------------------------------------------------------------------------------
