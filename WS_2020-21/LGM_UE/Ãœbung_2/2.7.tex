% --------------------------------------------------------------------------------

\begin{exercise}[45]

Sei $m$ eine Klauselmenge und $M^\prime = \Bbraces{C \in M \mid \text{es gibt keine Variable}~ p ~\text{mit}~ p \in C ~\text{und}~ \neg p \in C}$.
Zeigen Sie dass $M$ und $M^\prime$ äquivalent sind.
(
    Das heißt:
    Jede Belegung $b$, die $M$ erfüllt, erfüllt auch $M^\prime$, und umgekehrt.
)

\end{exercise}

% --------------------------------------------------------------------------------

\begin{solution}
\phantom{}
\begin{enumerate}
	\item[`$\Rightarrow$'] Sei $b$ eine Belegung mit $b(M) = 1$. Wegen $M^\prime \subseteq M$ gilt klarerweise auch $b(M^\prime) = 1$.
	\item[`$\Leftarrow$'] Angenommen es gibt eine Belegung $b$ mit $b(M^\prime) = 1$ und $b(C) = 0$ für ein $C \in M$. Natürlich kann dann nicht $C \in M^\prime$ gelten, also gibt es eine Varibale $q$ mit $q \in C$ und $\neg q \in C$. Die Klausel $C$ hat also die Form $ C = q \lor \neg q \lor C^\prime$. Wir sehen dass in jedem Fall, egal ob $b(q) = 1$ oder $b(q) = 0$ gilt, die Gleichheit $b(C) = 1$ gilt. Widerspruch!
\end{enumerate}
\end{solution}

% --------------------------------------------------------------------------------
