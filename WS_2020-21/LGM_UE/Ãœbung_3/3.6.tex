% --------------------------------------------------------------------------------

\begin{exercise}[58]

Geben Sie ein möglichst einfaches Ableitungssystem an, in dem eine
Zeichenfolge der Form $1^n$ genau dann ableitbar ist, wenn $n > 1$ und keine
Primzahl ist.

\end{exercise}

% --------------------------------------------------------------------------------

\begin{solution}
Wir wählen als Axiome für $n \geq 2$
\begin{align*}
  \frac{\emptyset}{1^n|1^{2n}}
\end{align*}
und als Regeln für $n, k \in \N$
\begin{align*}
  \frac{1^n|1^k}{1^n|1^{k+n}} \quad \frac{1^n|1^k}{1^k}.
\end{align*}
Wir behaupten nun, dass die Zeichenfolge $1^n$ genau dann ableitbar ist,
wenn $n$ keine Primzahl ist.

Dazu zeigen wir, dass die Menge der ableitbaren Zeichenfolgen genau aus
den Zeichenfolgen der Form $1^n$ mit $n$ nicht prim und $1^n|1^k$ mit $n$
echter Teiler von $k$ besteht.

Diese Menge enthält sicher die Axiome und ist unter den Regeln abgeschlossen,
da mit der ersten Regel, wenn $n$ echter Teiler von $k$, dann auch $n$
echter Teiler von $k + n$ und mit Regel 2, wenn $n$ echter Teiler von $k$,
dann ist $k$ keine Primzahl.

Wir zeigen nun noch, dass alle diese Zeichenfolgen tatsächlich herleitbar sind.
Sei $n$ keine Primzahl, also gibt es $p, q \neq n$ mit $n = pq$.
\begin{algorithmic}
  \State $\frac{\emptyset}{1^p|1^{2p}}$ \Comment Axiom
  \State $\frac{1^p|1^{2p}}{1^p|1^{3p}}$ \Comment Regel 1
  \State $\vdots$ \Comment Regel 1 iterieren
  \State $\frac{1^p|1^{(q-1)p}}{1^p|1^{(q-1)p + p}}$
  \State $\frac{1^p|1^{qp}}{1^n}$
\end{algorithmic}
Sei nun $n$ echter Teiler von $k$, also $k = nm$.
Dann erhalten wir durch iteriertes Anwenden der ersten Regel $1^n|1^k$ wie oben.
\end{solution}

% --------------------------------------------------------------------------------
