% --------------------------------------------------------------------------------

\begin{exercise}[54]

Geben Sie ein Kriterium an, das entscheidet, ob eine vorgegebene Zeichenfolge ableitbar ist.

\end{exercise}

% --------------------------------------------------------------------------------

\begin{solution}
Eine Zeichenfolge ist genau dann ableitbar wenn:
\begin{enumerate}[label = \alph*)]
  \item Die Zeichenfolge hat die Form $x + y = z$, wobei $x,y,z$
  endliche Folgen von Einsern bezeichnen.
  \item Die Anzahl der Einser auf beiden Seiten des Gleichheitszeichen stimmen überein.
\end{enumerate}

Wir zeigen nun, dass die Menge $N$ an Zeichenfolgen, die unser Kriterium erfüllt
alle Axiome enthält, unter den vorgegebenen Regeln abgeschlossen ist und
die kleinste Menge mit dieser Eigenschaft ist.

\begin{itemize}
  \item Enthält Axiome: Klar.
  \item Unter Regeln abgeschlossen: \\
  Sei $A = x + y = z \in N$. Dann ist $1A1: 1x + y = z1$ ebenfalls in $N$
  und $x + y1 = 1z$ ebenso.
  \item Kleinste solche Menge: Sei $x + y = z \in N$ beliebig mit
  $|x| = n, |y| = m$ und daher $|z| = n + m$. Durch $(n-1)$-malige
  Anwendung der ersten und $(m-1)$-malige Anwendung der zweiten Regel
  können wir $x + y = z$ herleiten. Also enthält $N$ genau die
  ableitbaren Zeichenfolgen.
\end{itemize}
\end{solution}

% --------------------------------------------------------------------------------
