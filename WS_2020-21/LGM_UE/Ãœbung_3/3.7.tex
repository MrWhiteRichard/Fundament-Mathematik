% -------------------------------------------------------------------------------- %

\begin{exercise}[83]

Sei $h$ ein Homomorphismus von den aussagenlogischen in die prädikatenlogischen
Formeln einer festen Sprache $\mathscr{L}$, also
\begin{align*}
  h(\top) = \top, \quad h(\bot) = \bot, \quad h(\varphi \land \psi) =
  h(\varphi) \land h(\psi) \quad \cdots.
\end{align*}
Sei $\mathscr{U}$ eine $\mathscr{L}$-Struktur, und sei $b$ eine Belegung
im prädikatenlogischen Sinn. \\
Dann gibt es eine aussagenlogische Belegung $b'$, die
$\hat{b'}(A) = \hat{b}(h(A))$ für alle aussagenlogischen Formeln $A$ erfüllt.
\end{exercise}

% -------------------------------------------------------------------------------- %

\begin{solution}

Definiere für alle aussagenlogischen Variablen $p$

\begin{align*}
  b'(p) := \hat{b}(h(p)).
\end{align*}

Wir führen einen Induktionsbeweis über die Menge aller aussagenlogischen Formeln.

Für aussagenlogische Variablen folgt die Aussage aus der Definition von $b'$.
Die Abgeschlossenheit unter den Junktoren folgt aus der Homomorphie-Eigenschaft von $h$, hier exemplarisch für die Konjunktion:
\begin{align*}
  \hat{b'}(A \land B) = \hat{b'}(A) \land_{Bool} \hat{b'}(B)
  = \hat{b}(h(A)) \land_{Bool} \hat{b}(h(B)) = \hat{b}(h(A) \land h(B))
  = \hat{b}(h(A \land B))
\end{align*}

\end{solution}

% -------------------------------------------------------------------------------- %

% -------------------------------------------------------------------------------- %

\begin{exercise}[84]

Schließen Sie aus der vorigen Aufgabe: Wenn $A$ aussagenlogische Tautologie ist,
$h$ ein Homomorphismus, dann ist $h(A)$ allgemeingültig.
\end{exercise}

% -------------------------------------------------------------------------------- %

\begin{solution}

Es gilt also für alle aussagenlogischen Belegungen $\hat{b}(A) = 1$. \\
Sei nun $b$ beliebige prädikatenlogische Belegung und $b'$ definiert wie
in der Aufgabe davor. Dann gilt

\begin{align*}
  \hat{b}(h(A)) = \hat{b'}(A) = 1
\end{align*}
und $h(A)$ ist somit prädikatenlogisch allgemeingültig.

\end{solution}
