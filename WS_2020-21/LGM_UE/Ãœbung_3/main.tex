\documentclass{article}

% ---------------------------------------------------------------- %
% short package descriptions are copied from
% https://ctan.org/

% ---------------------------------------------------------------- %

% Accept different input encodings
\usepackage[utf8]{inputenc}

% Standard package for selecting font encodings
\usepackage[T1]{fontenc}

% ---------------------------------------------------------------- %

% Multilingual support for Plain TEX or LATEX
\usepackage[ngerman]{babel}

% ---------------------------------------------------------------- %

% Set all page margins to 1.5cm
\usepackage{fullpage}

% Margin adjustment and detection of odd/even pages
\usepackage{changepage}

% Flexible and complete interface to document dimensions
\usepackage{geometry}

% ---------------------------------------------------------------- %
% mathematics

\usepackage{amsmath}  % AMS mathematical facilities for LATEX
\usepackage{amssymb}
\usepackage{amsfonts} % TEX fonts from the American Mathematical Society
\usepackage{amsthm}   % Typesetting theorems (AMS style)

% Mathematical tools to use with amsmath
\usepackage{mathtools}

% Support for using RSFS fonts in maths
\usepackage{mathrsfs}

% Commands to produce dots in math that respect font size
\usepackage{mathdots}

% "Blackboard-style" cm fonts
\usepackage{bbm}

% Typeset in-line fractions in a "nice" way
\usepackage{nicefrac}

% Typeset quotient structures with LATEX
\usepackage{faktor}

% Vector arrows
\usepackage{esvect}

% St Mary Road symbols for theoretical computer science
\usepackage{stmaryrd}

% Three series of mathematical symbols
\usepackage{mathabx}

% ---------------------------------------------------------------- %
% algorithms

% Package for typesetting pseudocode
\usepackage{algpseudocode}

% Typeset source code listings using LATEX
\usepackage{listings}

% Reimplementation of and extensions to LATEX verbatim
\usepackage{verbatim}

% If necessary, please use the following 2 packages locally, but never both.
% This is because the algorithm environment gets defined in both packages, which leads to name conflicts.
% \usepackage{algorithm2e}
% \usepackage{algorithm}

% ---------------------------------------------------------------- %
% utilities

% A generic document command parser
\usepackage{xparse}

% Extended conditional commands
\usepackage{xifthen}

% e-TEX tools for LATEX
\usepackage{etoolbox}

% Define commands with suffixes
\usepackage{suffix}

% Extensive support for hypertext in LATEX
\usepackage{hyperref}

% Driver-independent color extensions for LATEX and pdfLATEX
\usepackage{xcolor}

% ---------------------------------------------------------------- %
% graphics

% -------------------------------- %

\usepackage{tikz}

% MISC
\usetikzlibrary{patterns}
\usetikzlibrary{decorations.markings}
\usetikzlibrary{positioning}
\usetikzlibrary{arrows}
\usetikzlibrary{arrows.meta}
\usetikzlibrary{overlay-beamer-styles}

% finite state machines
\usetikzlibrary{automata}

% turing machines
\usetikzlibrary{calc}
\usetikzlibrary{chains}
\usetikzlibrary{decorations.pathmorphing}

% -------------------------------- %

% Draw tree structures
\usepackage[noeepic]{qtree}

% Enhanced support for graphics
\usepackage{graphicx}

% Figures broken into subfigures
\usepackage{subfig}

% Improved interface for floating objects
\usepackage{float}

% Control float placement
\usepackage{placeins}

% Include PDF documents in LATEX
\usepackage{pdfpages}

% ---------------------------------------------------------------- %

% Control layout of itemize, enumerate, description
\usepackage[inline]{enumitem}

% Intermix single and multiple columns
\usepackage{multicol}
\setlength{\columnsep}{1cm}

% Coloured boxes, for LATEX examples and theorems, etc
\usepackage{tcolorbox}

% ---------------------------------------------------------------- %
% tables

% Tabulars with adjustable-width columns
\usepackage{tabularx}

% Tabular column heads and multilined cells
\usepackage{makecell}

% Publication quality tables in LATEX
\usepackage{booktabs}

% ---------------------------------------------------------------- %
% bibliography and quoting

% Sophisticated Bibliographies in LATEX
\usepackage[backend = biber, style = alphabetic]{biblatex}

% Context sensitive quotation facilities
\usepackage{csquotes}

% ---------------------------------------------------------------- %

% ---------------------------------------------------------------- %
% special letters

\newcommand{\N}{\mathbb N}
\newcommand{\Z}{\mathbb Z}
\newcommand{\Q}{\mathbb Q}
\newcommand{\R}{\mathbb R}
\newcommand{\C}{\mathbb C}
\newcommand{\K}{\mathbb K}
\newcommand{\T}{\mathbb T}
\newcommand{\E}{\mathbb E}
\newcommand{\V}{\mathbb V}
\renewcommand{\S}{\mathbb S}
\renewcommand{\P}{\mathbb P}
\newcommand{\1}{\mathbbm 1}
\newcommand{\G}{\mathbb G}

\newcommand{\iu}{\mathrm i}

% ---------------------------------------------------------------- %
% quantors

\newcommand{\Forall}        {\forall ~}
\newcommand{\Exists}        {\exists ~}
\newcommand{\nExists}       {\nexists ~}
\newcommand{\ExistsOnlyOne} {\exists! ~}
\newcommand{\nExistsOnlyOne}{\nexists! ~}
\newcommand{\ForAlmostAll}  {\forall^\infty ~}

% ---------------------------------------------------------------- %
% graphics boxed

\newcommand
{\includegraphicsboxed}
[2][0.75]
{
    \begin{center}
        \begin{tcolorbox}[standard jigsaw, opacityback = 0]

            \centering
            \includegraphics[width = #1 \textwidth]{#2}

        \end{tcolorbox}
    \end{center}
}

\newcommand
{\includegraphicsunboxed}
[2][0.75]
{
    \begin{center}
        \includegraphics[width = #1 \textwidth]{#2}
    \end{center}
}

\NewDocumentCommand
{\includegraphicsgraphicsboxed}
{ O{0.75} O{0.25} m m}
{
    \begin{center}
        \begin{tcolorbox}[standard jigsaw, opacityback = 0]

            \centering
            \includegraphics[width = #1 \textwidth]{#3} \\
            \vspace{#2 cm}
            \includegraphics[width = #1 \textwidth]{#4}

        \end{tcolorbox}
    \end{center}
}

\NewDocumentCommand
{\includegraphicsgraphicsunboxed}
{ O{0.75} O{0.25} m m}
{
    \begin{center}

        \centering
        \includegraphics[width = #1 \textwidth]{#3} \\
        \vspace{#2 cm}
        \includegraphics[width = #1 \textwidth]{#4}

    \end{center}
}

% ---------------------------------------------------------------- %
% braces

\newcommand{\pbraces}[1]{{\left  ( #1 \right  )}}
\newcommand{\bbraces}[1]{{\left  [ #1 \right  ]}}
\newcommand{\Bbraces}[1]{{\left \{ #1 \right \}}}
\newcommand{\vbraces}[1]{{\left  | #1 \right  |}}
\newcommand{\Vbraces}[1]{{\left \| #1 \right \|}}

\newcommand{\abraces}[1]{{\left \langle #1 \right \rangle}}

\newcommand{\floorbraces}[1]{{\left \lfloor #1 \right \rfloor}}
\newcommand{\ceilbraces} [1]{{\left \lceil  #1 \right \rceil }}

\newcommand{\dbbraces}    [1]{{\llbracket     #1 \rrbracket}}
\newcommand{\dpbraces}    [1]{{\llparenthesis #1 \rrparenthesis}}
\newcommand{\dfloorbraces}[1]{{\llfloor       #1 \rrfloor}}
\newcommand{\dceilbraces} [1]{{\llceil        #1 \rrceil}}

\newcommand{\dabraces}[1]{{\left \langle \left \langle #1 \right \rangle \right \rangle}}

\newcommand{\abs}  [1]{\vbraces{#1}}
\newcommand{\round}[1]{\bbraces{#1}}
\newcommand{\floor}[1]{\floorbraces{#1}}
\newcommand{\ceil} [1]{\ceilbraces{#1}}

% ---------------------------------------------------------------- %

% MISC

% metric spaces
\newcommand{\norm}[2][]{\Vbraces{#2}_{#1}}
\DeclareMathOperator{\metric}{d}
\DeclareMathOperator{\dist}  {dist}
\DeclareMathOperator{\diam}  {diam}

% O-notation
\newcommand{\landau}{{\scriptstyle \mathcal{O}}}
\newcommand{\Landau}{\mathcal{O}}

% ---------------------------------------------------------------- %

% math operators

% hyperbolic trigonometric function inverses
\DeclareMathOperator{\areasinh}{areasinh}
\DeclareMathOperator{\areacosh}{areacosh}
\DeclareMathOperator{\areatanh}{areatanh}

% special functions
\DeclareMathOperator{\id} {id}
\DeclareMathOperator{\sgn}{sgn}
\DeclareMathOperator{\Inv}{Inv}
\DeclareMathOperator{\erf}{erf}
\DeclareMathOperator{\pv} {pv}

% exponential function as power
\WithSuffix \newcommand \exp* [1]{\mathrm{e}^{#1}}

% operations on sets
\DeclareMathOperator{\meas}{meas}
\DeclareMathOperator{\card}{card}
\DeclareMathOperator{\Span}{span}
\DeclareMathOperator{\conv}{conv}
\DeclareMathOperator{\cof}{cof}
\DeclareMathOperator{\mean}{mean}
\DeclareMathOperator{\avg}{avg}
\DeclareMathOperator*{\argmax}{argmax}
\DeclareMathOperator*{\argsmax}{argsmax}

% number theory stuff
\DeclareMathOperator{\ggT}{ggT}
\DeclareMathOperator{\kgV}{kgV}
\DeclareMathOperator{\modulo}{mod}

% polynomial stuff
\DeclareMathOperator{\ord}{ord}
\DeclareMathOperator{\grad}{grad}

% function properties
\DeclareMathOperator{\ran}{ran}
\DeclareMathOperator{\supp}{supp}
\DeclareMathOperator{\graph}{graph}
\DeclareMathOperator{\dom}{dom}
\DeclareMathOperator{\Def}{def}
\DeclareMathOperator{\rg}{rg}

% matrix stuff
\DeclareMathOperator{\GL}{GL}
\DeclareMathOperator{\SL}{SL}
\DeclareMathOperator{\U}{U}
\DeclareMathOperator{\SU}{SU}
\DeclareMathOperator{\PSU}{PSU}
% \DeclareMathOperator{\O}{O}
% \DeclareMathOperator{\PO}{PO}
% \DeclareMathOperator{\PSO}{PSO}
\DeclareMathOperator{\diag}{diag}

% algebra stuff
\DeclareMathOperator{\At}{At}
\DeclareMathOperator{\Ob}{Ob}
\DeclareMathOperator{\Hom}{Hom}
\DeclareMathOperator{\End}{End}
\DeclareMathOperator{\Aut}{Aut}
\DeclareMathOperator{\Lin}{L}

% other function classes
\DeclareMathOperator{\Lip}{Lip}
\DeclareMathOperator{\Mod}{Mod}
\DeclareMathOperator{\Dil}{Dil}

% constants
\DeclareMathOperator{\NIL}{NIL}
\DeclareMathOperator{\eps}{eps}

% ---------------------------------------------------------------- %
% doubble & tripple powers

\newcommand
{\primeprime}
{{\prime \prime}}

\newcommand
{\primeprimeprime}
{{\prime \prime \prime}}

\newcommand
{\astast}
{{\ast \ast}}

\newcommand
{\astastast}
{{\ast \ast \ast}}

% ---------------------------------------------------------------- %
% derivatives

\NewDocumentCommand
{\derivative}
{ O{} O{} m m}
{
    \frac
    {\mathrm d^{#2} {#1}}
    {\mathrm d {#3}^{#2}}
}

\NewDocumentCommand
{\pderivative}
{ O{} O{} m m}
{
    \frac
    {\partial^{#2} {#1}}
    {\partial {#3}^{#2}}
}

\DeclareMathOperator{\Div}{div}
\DeclareMathOperator{\rot}{rot}

% ---------------------------------------------------------------- %
% integrals

\NewDocumentCommand
{\Int}
{ O{} O{} m m}
{\int_{#1}^{#2} #3 ~ \mathrm d #4}

\NewDocumentCommand
{\Iint}
{ O{} O{} m m m}
{\iint_{#1}^{#2} #3 ~ \mathrm d #4 ~ \mathrm d #5}

\NewDocumentCommand
{\Iiint}
{ O{} O{} m m m m}
{\iiint_{#1}^{#2} #3 ~ \mathrm d #4 ~ \mathrm d #5 ~ \mathrm d #6}

\NewDocumentCommand
{\Iiiint}
{ O{} O{} m m m m m}
{\iiiint_{#1}^{#2} #3 ~ \mathrm d #4 ~ \mathrm d #5 ~ \mathrm d #6 ~ \mathrm d #7}

\NewDocumentCommand
{\Idotsint}
{ O{} O{} m m m}
{\idotsint_{#1}^{#2} #3 ~ \mathrm d #4 \dots ~ \mathrm d #5}

\NewDocumentCommand
{\Oint}
{ O{} O{} m m}
{\oint_{#1}^{#2} #3 ~ \mathrm d #4}

% ---------------------------------------------------------------- %

% source:
% https://tex.stackexchange.com/questions/203257/tikz-chains-with-one-side-of-the-leftmost-node-thickbold

% #1 (optional): current state, e.g. $q_0$
% #2: cursor position, e.g. 1
% #3: number of displayed cells, e.g. 5
% #4: contents of cells, e.g. {$\triangleright$, $x_1$, \dots, $x_n$, \textvisiblespace}

\newcommand{\turingtape}[4][]
{
    \begin{tikzpicture}

        \tikzset{tape/.style={minimum size=.7cm, draw}}

        \begin{scope}[start chain=0 going right, node distance=0mm]
            \foreach \x [count=\i] in #4
            {
                \ifnum\i=#3 % if last node reset outer sep to 0pt
                    \node [on chain=0, tape, outer sep=0pt] (n\i) {\x};
                    \draw (n\i.north east) -- ++(.1,0) decorate [decoration={zigzag, segment length=.12cm, amplitude=.02cm}] {-- ($(n\i.south east)+(+.1,0)$)} -- (n\i.south east) -- cycle;
                \else
                    \node [on chain=0, tape] (n\i) {\x};
                \fi

                \ifnum\i=1 % if first node draw a thick line at the left
                    \draw [line width=.1cm] (n\i.north west) -- (n\i.south west);
                \fi
            }
 
            \node [right=.25cm of n#3] {$\cdots$};
            \node [tape, above left=.25cm and 1cm of n1] (q) {#1};
            \draw [>=latex, ->] (q) -| (n#2);

        \end{scope}

    \end{tikzpicture}
}

% ---------------------------------------------------------------- %

% ---------------------------------------------------------------- %
% amsthm-environments:

\theoremstyle{definition}

% numbered theorems
\newtheorem{theorem}             {Satz}[section]
\newtheorem{lemma}      [theorem]{Lemma}
\newtheorem{corollary}  [theorem]{Korollar}
\newtheorem{proposition}[theorem]{Proposition}
\newtheorem{remark}     [theorem]{Bemerkung}
\newtheorem{definition} [theorem]{Definition}
\newtheorem{example}    [theorem]{Beispiel}
\newtheorem{heuristics} [theorem]{Heuristik}

% unnumbered theorems
\newtheorem*{theorem*}    {Satz}
\newtheorem*{lemma*}      {Lemma}
\newtheorem*{corollary*}  {Korollar}
\newtheorem*{proposition*}{Proposition}
\newtheorem*{remark*}     {Bemerkung}
\newtheorem*{definition*} {Definition}
\newtheorem*{example*}    {Beispiel}
\newtheorem*{heuristics*} {Heuristik}

% ---------------------------------------------------------------- %
% exercise- and solution-environments:

% Please define this stuff in project ("main.tex"):
% \def \lastexercisenumber {...}

\newtheorem{exercise}{Aufgabe}
\setcounter{exercise}{\lastexercisenumber}

\newenvironment{solution}
{
  \begin{proof}[Lösung]
}{
  \end{proof}
}

% ---------------------------------------------------------------- %
% MISC translations for environment-names

\renewcommand{\proofname} {Beweis}
\renewcommand{\figurename}{Abbildung}
\renewcommand{\tablename} {Tabelle}

% ---------------------------------------------------------------- %

% ---------------------------------------------------------------- %
% https://www.overleaf.com/learn/latex/Code_listing

\definecolor{codegreen} {rgb}{0, 0.6, 0}
\definecolor{codegray}    {rgb}{0.5, 0.5, 0.5}
\definecolor{codepurple}{rgb}{0.58, 0, 0.82}
\definecolor{backcolour}{rgb}{0.95, 0.95, 0.92}

\lstdefinestyle{overleaf}
{
    backgroundcolor = \color{backcolour},
    commentstyle = \color{codegreen},
    keywordstyle = \color{magenta},
    numberstyle = \tiny\color{codegray},
    stringstyle = \color{codepurple},
    basicstyle = \ttfamily \footnotesize,
    breakatwhitespace = false,
    breaklines = true,
    captionpos = b,
    keepspaces = true,
    numbers = left,
    numbersep = 5pt,
    showspaces = false,
    showstringspaces = false,
    showtabs = false,
    tabsize = 2
}

% ---------------------------------------------------------------- %
% https://en.wikibooks.org/wiki/LaTeX/Source_Code_Listings

\lstdefinestyle{customc}
{
    belowcaptionskip = 1 \baselineskip,
    breaklines = true,
    frame = L,
    xleftmargin = \parindent,
    language = C,
    showstringspaces = false,
    basicstyle = \footnotesize \ttfamily,
    keywordstyle = \bfseries \color{green!40!black},
    commentstyle = \itshape \color{purple!40!black},
    identifierstyle = \color{blue},
    stringstyle = \color{orange},
}

\lstdefinestyle{customasm}
{
    belowcaptionskip = 1 \baselineskip,
    frame = L,
    xleftmargin = \parindent,
    language = [x86masm] Assembler,
    basicstyle = \footnotesize\ttfamily,
    commentstyle = \itshape\color{purple!40!black},
}

% ---------------------------------------------------------------- %
% https://tex.stackexchange.com/questions/235731/listings-syntax-for-literate

\definecolor{maroon}        {cmyk}{0, 0.87, 0.68, 0.32}
\definecolor{halfgray}      {gray}{0.55}
\definecolor{ipython_frame} {RGB}{207, 207, 207}
\definecolor{ipython_bg}    {RGB}{247, 247, 247}
\definecolor{ipython_red}   {RGB}{186, 33, 33}
\definecolor{ipython_green} {RGB}{0, 128, 0}
\definecolor{ipython_cyan}  {RGB}{64, 128, 128}
\definecolor{ipython_purple}{RGB}{170, 34, 255}

\lstdefinestyle{stackexchangePython}
{
    breaklines = true,
    %
    extendedchars = true,
    literate =
    {á}{{\' a}} 1 {é}{{\' e}} 1 {í}{{\' i}} 1 {ó}{{\' o}} 1 {ú}{{\' u}} 1
    {Á}{{\' A}} 1 {É}{{\' E}} 1 {Í}{{\' I}} 1 {Ó}{{\' O}} 1 {Ú}{{\' U}} 1
    {à}{{\` a}} 1 {è}{{\` e}} 1 {ì}{{\` i}} 1 {ò}{{\` o}} 1 {ù}{{\` u}} 1
    {À}{{\` A}} 1 {È}{{\' E}} 1 {Ì}{{\` I}} 1 {Ò}{{\` O}} 1 {Ù}{{\` U}} 1
    {ä}{{\" a}} 1 {ë}{{\" e}} 1 {ï}{{\" i}} 1 {ö}{{\" o}} 1 {ü}{{\" u}} 1
    {Ä}{{\" A}} 1 {Ë}{{\" E}} 1 {Ï}{{\" I}} 1 {Ö}{{\" O}} 1 {Ü}{{\" U}} 1
    {â}{{\^ a}} 1 {ê}{{\^ e}} 1 {î}{{\^ i}} 1 {ô}{{\^ o}} 1 {û}{{\^ u}} 1
    {Â}{{\^ A}} 1 {Ê}{{\^ E}} 1 {Î}{{\^ I}} 1 {Ô}{{\^ O}} 1 {Û}{{\^ U}} 1
    {œ}{{\oe}}  1 {Œ}{{\OE}}  1 {æ}{{\ae}}  1 {Æ}{{\AE}}  1 {ß}{{\ss}}  1
    {ç}{{\c c}} 1 {Ç}{{\c C}} 1 {ø}{{\o}} 1 {å}{{\r a}} 1 {Å}{{\r A}} 1
    {€}{{\EUR}} 1 {£}{{\pounds}} 1
}


% Python definition (c) 1998 Michael Weber
% Additional definitions (2013) Alexis Dimitriadis
% modified by me (should not have empty lines)

\lstdefinelanguage{iPython}{
    morekeywords = {access, and, break, class, continue, def, del, elif, else, except, exec, finally, for, from, global, if, import, in, is, lambda, not, or, pass, print, raise, return, try, while}, %
    %
    % Built-ins
    morekeywords = [2]{abs, all, any, basestring, bin, bool, bytearray, callable, chr, classmethod, cmp, compile, complex, delattr, dict, dir, divmod, enumerate, eval, execfile, file, filter, float, format, frozenset, getattr, globals, hasattr, hash, help, hex, id, input, int, isinstance, issubclass, iter, len, list, locals, long, map, max, memoryview, min, next, object, oct, open, ord, pow, property, range, raw_input, reduce, reload, repr, reversed, round, set, setattr, slice, sorted, staticmethod, str, sum, super, tuple, type, unichr, unicode, vars, xrange, zip, apply, buffer, coerce, intern}, %
    %
    sensitive = true, %
    morecomment = [l] \#, %
    morestring = [b]', %
    morestring = [b]", %
    %
    morestring = [s]{'''}{'''}, % used for documentation text (mulitiline strings)
    morestring = [s]{"""}{"""}, % added by Philipp Matthias Hahn
    %
    morestring = [s]{r'}{'},     % `raw' strings
    morestring = [s]{r"}{"},     %
    morestring = [s]{r'''}{'''}, %
    morestring = [s]{r"""}{"""}, %
    morestring = [s]{u'}{'},     % unicode strings
    morestring = [s]{u"}{"},     %
    morestring = [s]{u'''}{'''}, %
    morestring = [s]{u"""}{"""}, %
    %
    % {replace}{replacement}{lenght of replace}
    % *{-}{-}{1} will not replace in comments and so on
    literate = 
    {á}{{\' a}} 1 {é}{{\' e}} 1 {í}{{\' i}} 1 {ó}{{\' o}} 1 {ú}{{\' u}} 1
    {Á}{{\' A}} 1 {É}{{\' E}} 1 {Í}{{\' I}} 1 {Ó}{{\' O}} 1 {Ú}{{\' U}} 1
    {à}{{\` a}} 1 {è}{{\` e}} 1 {ì}{{\` i}} 1 {ò}{{\` o}} 1 {ù}{{\` u}} 1
    {À}{{\` A}} 1 {È}{{\' E}} 1 {Ì}{{\` I}} 1 {Ò}{{\` O}} 1 {Ù}{{\` U}} 1
    {ä}{{\" a}} 1 {ë}{{\" e}} 1 {ï}{{\" i}} 1 {ö}{{\" o}} 1 {ü}{{\" u}} 1
    {Ä}{{\" A}} 1 {Ë}{{\" E}} 1 {Ï}{{\" I}} 1 {Ö}{{\" O}} 1 {Ü}{{\" U}} 1
    {â}{{\^ a}} 1 {ê}{{\^ e}} 1 {î}{{\^ i}} 1 {ô}{{\^ o}} 1 {û}{{\^ u}} 1
    {Â}{{\^ A}} 1 {Ê}{{\^ E}} 1 {Î}{{\^ I}} 1 {Ô}{{\^ O}} 1 {Û}{{\^ U}} 1
    {œ}{{\oe}}  1 {Œ}{{\OE}}  1 {æ}{{\ae}}  1 {Æ}{{\AE}}  1 {ß}{{\ss}}  1
    {ç}{{\c c}} 1 {Ç}{{\c C}} 1 {ø}{{\o}} 1 {å}{{\r a}} 1 {Å}{{\r A}} 1
    {€}{{\EUR}} 1 {£}{{\pounds}} 1
    %
    {^}{{{\color{ipython_purple}\^ {}}}} 1
    { = }{{{\color{ipython_purple} = }}} 1
    %
    {+}{{{\color{ipython_purple}+}}} 1
    {*}{{{\color{ipython_purple}$^\ast$}}} 1
    {/}{{{\color{ipython_purple}/}}} 1
    %
    {+=}{{{+=}}} 1
    {-=}{{{-=}}} 1
    {*=}{{{$^\ast$ = }}} 1
    {/=}{{{/=}}} 1,
    literate = 
    *{-}{{{\color{ipython_purple} -}}} 1
     {?}{{{\color{ipython_purple} ?}}} 1,
    %
    identifierstyle = \color{black}\ttfamily,
    commentstyle = \color{ipython_cyan}\ttfamily,
    stringstyle = \color{ipython_red}\ttfamily,
    keepspaces = true,
    showspaces = false,
    showstringspaces = false,
    %
    rulecolor = \color{ipython_frame},
    frame = single,
    frameround = {t}{t}{t}{t},
    framexleftmargin = 6mm,
    numbers = left,
    numberstyle = \tiny\color{halfgray},
    %
    %
    backgroundcolor = \color{ipython_bg},
    % extendedchars = true,
    basicstyle = \scriptsize,
    keywordstyle = \color{ipython_green}\ttfamily,
}

% ---------------------------------------------------------------- %
% https://tex.stackexchange.com/questions/417884/colour-r-code-to-match-knitr-theme-using-listings-minted-or-other

\geometry{verbose, tmargin = 2.5cm, bmargin = 2.5cm, lmargin = 2.5cm, rmargin = 2.5cm}

\definecolor{backgroundCol}  {rgb}{.97, .97, .97}
\definecolor{commentstyleCol}{rgb}{0.678, 0.584, 0.686}
\definecolor{keywordstyleCol}{rgb}{0.737, 0.353, 0.396}
\definecolor{stringstyleCol} {rgb}{0.192, 0.494, 0.8}
\definecolor{NumCol}         {rgb}{0.686, 0.059, 0.569}
\definecolor{basicstyleCol}  {rgb}{0.345, 0.345, 0.345}

\lstdefinestyle{stackexchangeR}
{
    language = R,                                        % the language of the code
    basicstyle = \small \ttfamily \color{basicstyleCol}, % the size of the fonts that are used for the code
    % numbers = left,                                      % where to put the line-numbers
    numberstyle = \color{green},                         % the style that is used for the line-numbers
    stepnumber = 1,                                      % the step between two line-numbers. If it is 1, each line will be numbered
    numbersep = 5pt,                                     % how far the line-numbers are from the code
    backgroundcolor = \color{backgroundCol},             % choose the background color. You must add \usepackage{color}
    showspaces = false,                                  % show spaces adding particular underscores
    showstringspaces = false,                            % underline spaces within strings
    showtabs = false,                                    % show tabs within strings adding particular underscores
    % frame = single,                                      % adds a frame around the code
    % rulecolor = \color{white},                           % if not set, the frame-color may be changed on line-breaks within not-black text (e.g. commens (green here))
    tabsize = 2,                                         % sets default tabsize to 2 spaces
    captionpos = b,                                      % sets the caption-position to bottom
    breaklines = true,                                   % sets automatic line breaking
    breakatwhitespace = false,                           % sets if automatic breaks should only happen at whitespace
    keywordstyle = \color{keywordstyleCol},              % keyword style
    commentstyle = \color{commentstyleCol},              % comment style
    stringstyle = \color{stringstyleCol},                % string literal style
    literate = %
    *{0}{{{\color{NumCol} 0}}} 1
     {1}{{{\color{NumCol} 1}}} 1
     {2}{{{\color{NumCol} 2}}} 1
     {3}{{{\color{NumCol} 3}}} 1
     {4}{{{\color{NumCol} 4}}} 1
     {5}{{{\color{NumCol} 5}}} 1
     {6}{{{\color{NumCol} 6}}} 1
     {7}{{{\color{NumCol} 7}}} 1
     {8}{{{\color{NumCol} 8}}} 1
     {9}{{{\color{NumCol} 9}}} 1
}

% ---------------------------------------------------------------- %
% Fundament Mathematik

\lstdefinestyle{fundament}{basicstyle = \ttfamily}

% ---------------------------------------------------------------- %


\parskip 0pt
\parindent 0pt

\title
{
  Logik und Grundlagen der Mathematik \\
  \vspace{4pt}
  \normalsize
  \textit{3. Übung am 22.10.2020}
}
\author
{
  Richard Weiss
  \and
  Florian Schager
  \and
  Fabian Zehetgruber
}
\date{}

\begin{document}

\maketitle

\section*{Substitution}

% --------------------------------------------------------------------------------

\begin{exercise}

Zeigen Sie:

\begin{enumerate}[label = (\roman*)]

    \item Ist $f \in C^\infty(\R^n)$ und $y \in \R^n$, dann gibt es Funktionen $f_i \in C^\infty(\R^n)$ sodass

    \begin{align*}
        f(x) = f(y) + \sum_{i=1}^n (x_i - y_i) f_i(x).
    \end{align*}

    \item Gilt $xT = 0$ für eine Distribution $T \in \mathcal{D}^\prime(\R)$, so ist $T = c \delta$ für eine Konstante $c$.
    \item $u \in \mathcal{D}^\prime(\R)$ mit $u^\prime = 0$ impliziert $u = ~\text{const}$.
    \item Für jedes $f \in C^\infty(\R)$ existieren Konstanten $c_0, c_1$ sodass

    \begin{align*}
        f \delta^\prime
        =
        c_0 \delta
        +
        c_1 \delta^\prime.
    \end{align*}

\end{enumerate}

\end{exercise}

% --------------------------------------------------------------------------------

\begin{solution}
\phantom{}
\begin{enumerate}[label = (\roman*)]
	\item Sei also $f \in C^\infty(\R^n)$ und $y \in \R^n$. Wir verwenden die Taylorsche Formel, siehe dazu Kaltenbäck Satz 10.2.10, und erhalten
	\begin{align*}
	f(x) = f(y) + \int_0^1 df((1-t)y + tx)(x - y) dt = f(y) + \sum_{i = 1}^n (x_i - y_i) \int_0^1 \pderivative[][f]{y_i} ((1-t)y + tx) dt.
	\end{align*}
	Mit der Definition
	\begin{align*}
	f_i(x) := \int_0^1  \pderivative[][f]{y_i} ((1-t)y + tx) dt
	\end{align*}
	erhalten wir die gewünschte Gestalt, die $f_i$ sind auch $C^\infty$ weil man eine beliebige partielle Ableitung des Integranden auf dem Intervall $[0,1]$ mit dem Supremum majorisieren kann also den Differentialoperator mit dem Integral vertauschen darf. Vergleiche dazu Kusolitsch Korollar 9.37.
	\item Sei $T \in \mathcal{D}^\prime(\R)$ mit $xT = 0$ und $\varphi \in \mathcal{D}(\R)$. Gemäß Hinweis wählen wir ein $\chi \in \mathcal{D}(\R)$ mit $\forall x \in \supp{\varphi}: \chi(x) = 1$ (gibt es sowas?
  Ja, betrachte die Faltung von $\1_A$ mit $A \subset \R$ kompakt, sodass $\dist(\supp \varphi, A^C) > \delta$ mit einer bekannten Testfunktion $g$ mit $\supp g \subset [-\delta, \delta]$.
  Da $f \in L^1(\R)$ und $g \in C^{\infty}(\R)$ ist auch $f*g \in C^{\infty}(\R)$ und es gilt $f*g \equiv 1$ auf $\supp \varphi$.) Wir berechnen
	\begin{align*}
	\abraces{T,\varphi} = \abraces{T, \chi \varphi} \stackrel{(i)}{=} \abraces{T, \chi (\varphi(0) + x\varphi_1)} = \varphi(0) \abraces{T, \chi} + \underbrace{\abraces{xT, \chi \varphi_1}}_{=0}.
	\end{align*}
	Nun hängt das $\chi$ allerdings noch von $\varphi$ ab, wir erhalten obige Gleichheit allerdings für alle $\chi \in \mathcal{D}(\R)$ mit $\forall x \in \supp{\varphi}: \chi(x) = 1$ Für ein $\psi \in \mathcal{D}(\R)$ mit $\psi(0) \neq 0$ gilt also für alle entsprechenden $\chi$ die Gleichheit $c := \frac{\abraces{T,\psi}}{\psi(0)} = \abraces{T, \chi}$. Für eine beliebige weitere Funktion $\hat{\psi}$ können wir nun das $\chi$ so wählen, dass es nicht nur am Träger von $\hat{\psi}$ sondern auch am Träger von $\psi$ den Wert $1$ annimmt. Für solch ein $\chi$ kennen wir aber schon den Wert $\abraces{T, \chi} = c$. 
	\item Sei $u \in \mathcal{D}^\prime(\R)$ mit $u^\prime = 0$. Wir erinnern uns an den Beweis von Blümlingers Prop. 6.14. Sei $\Psi_0$ eine Testfunktion die $\int_\R \Psi_0 d\lambda = 1$ erfüllt. Wir definieren für beliebiges $\phi \in \mathcal{D}(\R)$ die Funktion $\zeta := \phi - \int_\R \phi d\lambda \Psi_0$, welche $\int_\R \zeta d\lambda = 0$ erfüllt. Die Funktion $\theta(x) := \int_{-\infty}^{x} \zeta d\lambda$ ist aus $\mathcal{D}(\R)$ und es gilt $\theta^\prime = \zeta$. So erhalten wir die Darstellung $\phi = \theta^\prime + \int_\R \phi d\lambda \Psi_0$. So können wir berechnen
	\begin{align*}
	\abraces{u, \phi} = \abraces{u, \theta^\prime} + \abraces{u, \int_\R \phi d\lambda \Psi_0} = \abraces{u^\prime, \theta} + \int_\R \phi d\lambda \abraces{u, \Psi_0} = \abraces{\abraces{u, \Psi_0}, \phi}
	\end{align*}
	\item Sei $f \in C^\infty(\R)$. Es gilt
	\begin{align*}
	\abraces{f \delta^\prime, \varphi} = \abraces{\delta^\prime, f\varphi} = - (f\varphi)^\prime (0) = -f(0) \varphi^\prime(0) - f^\prime(0) \varphi(0) = f(0) \abraces{\delta^\prime, \phi} - f^\prime(0) \abraces{\delta, \varphi}
	\end{align*}
\end{enumerate}

\end{solution}

% --------------------------------------------------------------------------------

% -------------------------------------------------------------------------------- %

\begin{exercise}[Continuous two-dimensional random variable]

The joint pdf of two random variables $X$ and $Y$ is defined by

\begin{align*}
    f(x, y)
    =
    \begin{cases}
        c (x + 2 y), & 0 < y < 1 ~\text{and}~ 0 < 2 \\
        0,           & \text{otherwise}
    \end{cases}.
\end{align*}

\begin{enumerate}[label = (\alph*)]
    \item Find the value of $c$ and the marginal distribution of $Y$.
    \item Find the joint cdf of $X$ and $Y$.
    \item Find the marginal distribution of $X$ and the pdf of $Z = \frac{9}{(X + 1)^2}$.
\end{enumerate}

\end{exercise}

% -------------------------------------------------------------------------------- %

\begin{solution}

\phantom{}

\begin{enumerate}[label = (\alph*)]

    \item

    \begin{multline*}
        1
        =
        \Int[\R^2]{f(x, y)}{(x, y)}
        =
        \Int[0][1]
        {
            \Int[0][1]
            {
                c (x + 2 y)
            }{x}
        }{y}
        =
        c
        \pbraces
        {
            \Int[0][2]{x}{x} \Int[0][1]{}{y}
            +
            2 \Int[0][2]{}{x} \Int[0][1]{y}{y}
        } \\
        =
        c
        \pbraces
        {
            \frac{2^2}{2} \cdot 1 + 2 \cdot 2 \cdot \frac{1}{2}
        }
        =
        4 c
    \end{multline*}

    \begin{align*}
        \implies c = \frac{1}{4}
    \end{align*}

    \begin{multline*}
        f_Y(y)
        =
        \Int[\R]{f(x, y)}{x}
        =
        \mathbf 1_{(0, 1)} \Int[0][2]{\frac{1}{4} (x + 2 y)}{x}
        =
        \mathbf 1_{(0, 1)}
        \frac{1}{4}
        \pbraces
        {
            \Int[0][2]{x}{x}
            +
            2 y \Int[0][2]{}{x}
        } \\
        =
        \mathbf 1_{(0, 1)}
        \frac{1}{4}
        \pbraces
        {
            \frac{2^2}{2}
            +
            2 y \cdot 2
        }
        =
        \mathbf 1_{(0, 1)}
        \pbraces{y + \frac{1}{2}}
    \end{multline*}

    \item

    \begin{align*}
        F_{X, Y}(x, y)
        & =
        \Int[-\infty][x]
        {
            \Int[-\infty][y]
            {
                f(\xi, \eta)
            }{\eta}
        }{\xi} \\
        & =
        \mathbf 1_{(0, \infty)^2}(x, y)
        \Int[0][\min \Bbraces{x, 2}]
        {
            \Int[0][\min \Bbraces{y, 1}]
            {
                \frac{1}{4}
                (\xi + 2 \eta)
            }{\eta}
        }{\xi} \\
        & =
        \mathbf 1_{(0, \infty)^2}(x, y)
        \frac{1}{4}
        \pbraces
        {
            \Int[0][\min \Bbraces{x, 2}]{\xi}{\xi}
            \Int[0][\min \Bbraces{y, 1}]{}{\eta}
            +
            2
            \Int[0][\min \Bbraces{x, 2}]{}{\xi}
            \Int[0][\min \Bbraces{y, 1}]{\eta}{\eta}
        } \\
        & =
        \mathbf 1_{(0, \infty)^2}(x, y)
        \frac{1}{4}
        \pbraces
        {
            \frac{\min \Bbraces{x, 2}^2}{2} \cdot y
            +
            2 \min \Bbraces{x, 2} \cdot \frac{\min \Bbraces{y, 1}^2}{2}
        } \\
        & =
        \mathbf 1_{(0, \infty)^2}
        \pbraces
        {
            \frac{\min \Bbraces{x, 2}^2 \min \Bbraces{y, 1}}{8}
            +
            \frac{\min \Bbraces{x, 2} \min \Bbraces{y, 1}^2}{4}
        }
    \end{align*}

    \item

    \begin{multline*}
        f_X(x)
        =
        \Int[\R]{f(x, y)}{y}
        =
        \mathbf 1_{(0, 2)}(x) \Int[0][1]{\frac{1}{4} (x + 2 y)}{y}
        =
        \mathbf 1_{(0, 2)}(x)
        \frac{1}{4}
        \pbraces
        {
            x \Int[0][1]{}{y}
            +
            2 \Int[0][1]{y}{y}
        } \\
        =
        \mathbf 1_{(0, 2)}(x) \frac{1}{4} \pbraces{x \cdot 1 + 2 \cdot \frac{1}{2}}
        =
        \mathbf 1_{(0, 2)}(x)
        \pbraces
        {
            \frac{x}{4}
            +
            \frac{1}{4}
        }
    \end{multline*}

    Recall the theorem from \cite[Lecture 3, Slide 38]{EStat}.

    \begin{align*}
        &
        g:
            (-\infty, -1) \cup (-1, \infty) \to (0, \infty), \,
            x \mapsto \frac{9}{(x + 1)^2} \stackrel{!}{=} y \\
        & \iff
        \frac{9}{y} = (x + 1)^2 \\
        & \iff
        \frac{3}{\sqrt y}
        =
        \begin{Bmatrix}
            +x + 1 \\
            -x + 1
        \end{Bmatrix}
        \iff
        \pm \pbraces{\frac{3}{\sqrt y} - 1} = x
    \end{align*}

    has two right inverses

    \begin{align*}
        h_+:
            (0, \infty) \to (-\infty, -1), \,
            y \mapsto \frac{3}{\sqrt y} - 1,
        \quad
        h_-:
            (0, \infty) \to (-1, \infty), \,
            y \mapsto -\frac{3}{\sqrt y} + 1
    \end{align*}

    with derivative(s)

    \begin{align*}
        h_\pm^\prime(y)
        =
        \mp \frac{1}{2} \frac{3}{y^{3 / 2}}
        =
        \mp \frac{3 / 2}{y^{3 / 2}}.
    \end{align*}

    \begin{align*}
        \implies
        f_Z(z)
        & =
        -
        f_X \pbraces{ \frac{3}{\sqrt z} - 1} \frac{3 / 2}{z^{3 / 2}}
        +
        f_X \pbraces{-\frac{3}{\sqrt z} + 1} \frac{3 / 2}{z^{3 / 2}} \\
        & \stackrel{!}{=}
        \pbraces
        {
            -
            \mathbf 1_{(1, 9)}(z)
            \pbraces
            {
                \frac{\frac{3}{\sqrt z} - 1}{4}
                +
                \frac{1}{4}
            }
            +
            \mathbf 1_{(9, \infty)}(z)
            \pbraces
            {
                \frac{-\frac{3}{\sqrt z} + 1}{4}
                +
                \frac{1}{4}
            }
        }
        \frac{3 / 2}{z^{3 / 2}} \\
        & =
        \pbraces
        {
            \mathbf 1_{(1, 9)}(z)
            \pbraces
            {
                \frac{-\frac{3}{\sqrt z} + 1}{4}
                -
                \frac{1}{4}
            }
            +
            \mathbf 1_{(9, \infty)}(z)
            \pbraces
            {
                \frac{-\frac{3}{\sqrt z} + 1}{4}
                +
                \frac{1}{4}
            }
        }
        \frac{3 / 2}{z^{3 / 2}} \\
        & =
        \pbraces
        {
            -
            \mathbf 1_{(1, 9)}(z)
            \frac{3 / 4}{\sqrt z}
            +
            \mathbf 1_{(9, \infty)}(z)
            \pbraces
            {
                -\frac{3 / 4}{\sqrt z}
                +
                \frac{1}{2}
            }
        }
        \frac{3 / 2}{z^{3 / 2}}
    \end{align*}

    For \enquote{$!$} we have used the following.

    \begin{align*}
        \frac{3}{\sqrt z} - 1 \in (0, 2)
        & \iff
        z \in (1, 9), \\
        -\frac{3}{\sqrt z} + 1 \in (0, 2)
        & \iff
        z \in (9, \infty)
    \end{align*}

\end{enumerate}

\end{solution}

% -------------------------------------------------------------------------------- %


\section*{Ableitungskalküle}

In den folgenden Beispielen betrachten wir Zeichenfolgen (Strings), die aus
den Zeichen $1,+,=$ zusammengesetzt sind. Auch die leere Folge gilt als
Zeichenfolge; sie hat Länge $0$. Meist wird sie mit $\epsilon$ oder mit $\Lambda$
bezeichnet. \\
Gewisse Zeichenfolgen zeichnen wir als \enquote{ableitbar} aus. \\
Gewisse Zeichenfolgen nennen wir \enquote{Axiome}. Axiome $A$ schreiben wir in
der Form

\begin{align*}
  \frac{\emptyset}{A}
\end{align*}

an. Wir lesen dies als \enquote{A ist ableitbar}. Eine \enquote{Regel}, die wir in
der Form

\begin{align*}
  \frac{A_1,\dots,A_n}{B}
\end{align*}

schreiben, lesen wir als \enquote{Wenn $A_1,\dots,A_n$ ableitbar sind, dann auch $B$}. \\
Die Menge der ableitbaren Zeichenfolgen ist die kleinste Menge $M$, die alle
Axiome enthält und unter allen Regeln abgeschlossen ist.
% --------------------------------------------------------------------------------

\begin{exercise}

\phantom{}

\begin{enumerate}[label = (\roman*)]
    \item Zeigen Sie, dass die Funktion
    
    \begin{align*}
        f(x)
        =
        \begin{cases}
            \ln{|x|} & x \neq 0 \\
            0        & x = 0
        \end{cases}
    \end{align*}

    eine reguläre Distribution definiert, die punktweise Ableitung

    \begin{align*}
        f^\prime(x)
        =
        \begin{cases}
            \frac{1}{x}        & x \neq 0 \\
            \text{undefiniert} & x = 0
        \end{cases}
    \end{align*}

    jedoch nicht.

    \item Es bezeichne $\pv{(\frac{1}{x})}$ die Distribution
    
    \begin{align*}
        \langle \pv{\pbraces{\frac{1}{x}}}, \varphi \rangle
        =
        \lim_{\varepsilon \to 0+}
        \pbraces
        {
            \Int[-\infty][-\varepsilon]
            {
                \frac{\varphi(x)}{x}
            }{x}
            +
            \Int[\varepsilon][\infty]
            {
                \frac{\varphi(x)}{x}
            }{x}
        }
        =
        \lim_{\varepsilon \to 0+}
        \Int[|x| > \varepsilon]
        {
            \frac{\varphi(x)}{x}
        }{x}.
    \end{align*}

    Zeigen Sie, dass $\abraces{\pv{(\frac{1}{x})}, \varphi} = \Int[0][\infty]{\frac{\varphi(x) - \varphi(-x)}{x}}{x}$.

    \item Überprüfen Sie, dass $(\ln{|x|})^\prime = \pv{(\frac{1}{x})}$ in $\mathcal{D}^\prime(\R)$ gilt.

\end{enumerate}

\end{exercise}

% --------------------------------------------------------------------------------

\begin{solution}
\phantom{}
\begin{enumerate}[label = (\roman*)]
	\item Gegeben sei eine beliebige kompakte Menge $K \subseteq \R$. Wir wählen $a \in [1, \infty]$ mit $K \subseteq [-a, a]$. Nun ist
	\begin{align*}
	\int_K \vbraces{f(x)} dx &\leq 2 \pbraces{\int_1^a \log(x) dx - \int_0^1 \log(x) dx} \\
	&= 2 \pbraces{(\log(a)a - a) - (\log(1)1 - 1) - \pbraces{(\log(1)1 - 1) - \lim_{\epsilon \to 0+} (\log(\epsilon)\epsilon - \epsilon )}} \\
	&= \log(a)a - a + 2 < \infty.
	\end{align*}
	Betrachte hingegen 
	\begin{align*}
	\int_0^1 \vbraces{f^\prime(x)} dx = \int_0^1 \frac{1}{x} dx = \log(1) - \lim_{\epsilon \to 0+} \log(\epsilon) = \infty.
	\end{align*}
	\item Wir berechnen
	\begin{align*}
	\abraces{\pv\pbraces{\frac{1}{x}}, \phi} &= \lim_{\varepsilon \to 0+}
	\pbraces
	{
		\Int[-\infty][-\varepsilon]
		{
			\frac{\varphi(t)}{t}
		}{t}
		+
		\Int[\varepsilon][\infty]
		{
			\frac{\varphi(x)}{x}
		}{x}
	} \\
	&= \lim_{\varepsilon \to 0+}
	\pbraces
	{
	-\Int[\varepsilon][\infty]
	{
		\frac{\varphi(-x)}{x}
	}{x}
	+
	\Int[\varepsilon][\infty]
	{
		\frac{\varphi(x)}{x}
	}{x}
	} = \Int[0][\infty]{\frac{\varphi(x) - \varphi(-x)}{x}}{x}
	\end{align*}
	\item 
	\begin{align*}
	\abraces{\pbraces{\log|x|}^\prime, \phi} &= -\abraces{\log|x|, \phi^\prime} = -\Int[\R][]{\log|x| \phi^\prime(x)}{x} = -\Int[0][\infty]{\log|x| (\phi^\prime(x) + \phi^\prime(-x))}{x} \\
	&= -\Int[0][\infty]{\log|x| (\phi(x) - \phi(-x))^\prime}{x} = \Int[0][\infty]{\frac{\varphi(x) - \varphi(-x)}{x}}{x} = \abraces{\pv\pbraces{\frac{1}{x}}, \phi}
	\end{align*}
\end{enumerate}

\end{solution}

% --------------------------------------------------------------------------------

% --------------------------------------------------------------------------------

\begin{exercise}[Random ariables on the unit disk]

Let $(X, Y)$ be uniformly distributed on the unit disk $\Bbraces{f(x; y): x^2 + y^2 \leq 1}$.
Let

\begin{align*}
    R = \sqrt{X^2 + Y^2}.
\end{align*}

Find the cdf, pdf, and teh expectation the random variable $R$.

\end{exercise}

% --------------------------------------------------------------------------------

\begin{solution}

Because the area of the unit disc is $\lambda^2(B_1(0)) = \pi$, we obtain the pdf $f_{X, Y}(x, y) = \mathbf 1_{B_1(0)}(x, y) \frac{1}{\pi}$.
$R = \sqrt{X^2 + Y^2}$ is the radius component of the polar coordinate transfomation.
Its inverse is given by

\begin{align*}
    x = r \cos \varphi,
    \quad
    y = r \sin \varphi,
    \quad
    \text{with commonly known Jacobian}
    \quad
    J(r, \varphi) = r.
\end{align*}

Recall the theorem from \cite[Lecture 3, Slide 32]{EStat}.

\begin{align*}
    f_{R, \Phi}(r, \varphi)
    \stackrel
    {
        \text{TRAFO}
    }{=}
    f_{X, Y}
    \begin{pmatrix}
        r \cos \varphi \\
        r \sin \varphi
    \end{pmatrix}
    r
    =
    \mathbf 1_{\overline B_1(0)}
    \begin{pmatrix}
        r \cos \varphi \\
        r \sin \varphi
    \end{pmatrix}
    \frac{1}{\pi}
    r
    =
    \mathbf 1_{[0, 1]}(r)
    \mathbf 1_{[0, 2 \pi)}(\varphi)
    \frac{r}{\pi}
\end{align*}

\begin{align*}
    f_R(r)
    =
    \Int[\R]{f_{R, \Phi}(r, \varphi)}{\varphi}
    =
    \mathbf 1_{[0, 1]}(r) \frac{r}{\pi} \Int[0][2 \pi]{}{\varphi}
    =
    \mathbf 1_{[0, 1]}(r) \cdot 2 r
\end{align*}

\begin{align*}
    F_R(r)
    =
    \Int[-\infty][r]{f_R(\rho)}{\rho} \,
    =
    \mathbf 1_{[0, \infty)}(r) \cdot 2 \Int[0][\min \Bbraces{r, 1}]{r}{r}
    \mathbf 1_{[0, \infty)}(r) \cdot 2 \frac{\min \Bbraces{r, 1}^2}{2}
    \mathbf 1_{[0, \infty)}(r) \min \Bbraces{r, 1}^2
\end{align*}

\begin{align*}
    E(R)
    =
    \Int[\R]{r \cdot f_R(r)}{r}
    =
    2 \Int[0][1]{r^2}{r}
    =
    \frac{2}{3}
\end{align*}

\end{solution}

% --------------------------------------------------------------------------------


In den folgenden Beispielen betrachten wir Zeichenfolgen, die aus $1,+,!,=$
zusammengesetzt sind. Unser Ableitungssystem enthält nun das einzige Axiom
$\frac{\emptyset}{1!}$ und die folgenden Regeln
\begin{align*}
  \frac{A!}{1A!} \quad \frac{A!}{A + 1 = A} \quad \frac{A + B = C}{A + B1 = CA}
\end{align*}
für beliebige Zeichenfolgen $A,B,C$.

% --------------------------------------------------------------------------------

\begin{exercise}[Transformations]

Suppose $X$ and $Y$ are independent gamma distributed random variables with $X \sim \operatorname{Gamma}(\alpha_1, \beta)$ and $Y \sim \operatorname{Gamma}(\alpha_2, \beta)$.
Consider the following two random variables

\begin{align*}
    U = X + Y
    \quad
    \text{and}
    \quad
    V = \frac{X}{X + Y}.
\end{align*}

\begin{enumerate}[label = (\alph*)]
    \item Show that $U \sim \operatorname{Gamma}(\alpha_1 + \alpha_2, \beta)$.
    \item Show that $U$ and $V$ are also independent random variables.
\end{enumerate}

\end{exercise}

% --------------------------------------------------------------------------------

\begin{solution}

We first calculate the inverse transformation $h$ \dots

\begin{align*}
    V = \frac{X}{X + Y}
    & \implies
    X = V (X + Y) = V X + V Y \\
    & \implies
    V Y = X - V X = X (1 - V ) \\
    U = X + Y
    & \implies
    \frac{X}{V} - X = X \pbraces{\frac{1}{V} - 1} = Y = U - X \\
    & \implies
    \frac{X}{V} = U \\
    & \implies
    X = U V =: h_1(U, V) \\
    U = X + Y
    & \implies
    Y = U - X = U - U V = U (1 - V) =: h_2(U, V),
\end{align*}

\dots and its Jacobian \dots

\begin{align*}
    J(u, v)
    :=
    \det \nabla h(u, v)
    =
    \det
    \begin{pmatrix}
        \partial_u h_1(u, v) & \partial_v h_1(u, v) \\
        \partial_u h_2(u, v) & \partial_v h_2(u, v)
    \end{pmatrix}
    =
    \det
    \begin{pmatrix}
            v &  u \\
        1 - v & -u
    \end{pmatrix}
    =
    -u v - u (1 - v)
    =
    -u.
\end{align*}

Because $X$ and $Y$ are independent, $f_{X, Y}(x, y) = f_X(x) f_Y(y)$.
Recall the theorem from \cite[Lecture 3, Slide 32]{EStat}.

\begin{align*}
    f_{U, V}(u, v)
    & \stackrel
    {
        \text{TRAFO}
    }{=}
    f_{X, Y}(h(u, v)) |J(u, v)| \\
    & =
    f_X(h_1(u, v)) f_Y(h_2(u, v)) |u| \\
    & =
    f_X(u v) f_Y(u (1 - v)) |u| \\
    & =
    \frac{\beta^{\alpha_1}}{\Gamma(\alpha_1)} (u v)      ^{\alpha_1 - 1} \exp*{-\beta u v}       \mathbf 1_{(0, \infty)}(u v)
    \frac{\beta^{\alpha_2}}{\Gamma(\alpha_2)} (u (1 - v))^{\alpha_2 - 1} \exp*{-\beta u (1 - v)} \mathbf 1_{(0, \infty)}(u (1 - v))
    |u| \\
    & =
    \frac{\beta^{\alpha_1 + \alpha_2}}{\Gamma(\alpha_1) \Gamma(\alpha_2)}
    u^{\alpha_1 + \alpha_2 - 2}
    v^{\alpha_1 - 1} (1 - v)^{\alpha_2 - 1}
    \exp*{-\beta u}
    \mathbf 1_{(0, \infty)}(u v)
    \mathbf 1_{(0, \infty)}(u (1 - v))
    |u| \\
    & \stackrel{!}{=}
    \frac{\beta^{\alpha_1 + \alpha_2}}{\Gamma(\alpha_1) \Gamma(\alpha_2)}
    u^{\alpha_1 + \alpha_2 - 1}
    \exp*{-\beta u}
    \mathbf 1_{(0, \infty)}(u)
    v^{\alpha_1 - 1} (1 - v)^{\alpha_2 - 1}
    \mathbf 1_{(0, 1)}(v)
\end{align*}

For \enquote{$!$} we have used the following calculation.

\begin{multline*}
    \mathbf 1_{(0, \infty)}(u v) \mathbf 1_{(0, \infty)}(u (1 - v)) = 1
    \iff
    u v > 0, \quad u - u v > 0
    \stackrel
    {
        u > u v > 0
    }{\iff}
    1 > v > 0, \quad u > 0 \\
    \iff
    \mathbf 1_{(0, 1)}(v) \mathbf 1_{(0, \infty)}(u) = 1
\end{multline*}

\begin{enumerate}[label = (\alph*)]

    \item

    \begin{multline*}
        f_U(u)
        =
        \Int[\R]{f_{U, V}(u, v)}{v}
        =
        \frac{\beta^{\alpha_1 + \alpha_2}}{\Gamma(\alpha_1) \Gamma(\alpha_2)}
        u^{\alpha_1 + \alpha_2 - 1}
        \exp*{-\beta u}
        \mathbf 1_{(0, \infty)}(u)
        \underbrace
        {
            \Int[0][1]
            {
                v^{\alpha_1 - 1}
                (1 - v)^{\alpha_2 - 1}
            }{v}
        }_{
            B(\alpha_1, \alpha_2)
        } \\
        =
        \frac{\beta^{\alpha_1 + \alpha_2}}{\Gamma(\alpha_1 + \alpha_2)}
        u^{\alpha_1 + \alpha_2 - 1}
        \exp*{-\beta u}
        \mathbf 1_{(0, \infty)}(u)
    \end{multline*}

    \item

    \begin{align*}
        f_V(v)
        & =
        \Int[\R]{f_{U, V}(u, v)}{u} \\
        & =
        \frac{\beta^{\alpha_1 + \alpha_2}}{\Gamma(\alpha_1) \Gamma(\alpha_2)}
        \Int[0][\infty]
        {
            u^{\alpha_1 + \alpha_2 - 1}
            \exp*{-\beta u}
        }{u} \,
        v^{\alpha_1 - 1}
        (1 - v)^{\alpha_2 - 1}
        \mathbf 1_{(0, 1)}(v) \\
        & \stackrel{!}{=}
        \frac{\beta^{\alpha_1 + \alpha_2}}{\Gamma(\alpha_1) \Gamma(\alpha_2)}
        \Int[0][\infty]
        {
            \frac{s^{\alpha_1 + \alpha_2 - 1}}{\beta^{\alpha_1 + \alpha_2 - 1}}
            \exp*{-s}
            \frac{1}{\beta}
        }{s} \,
        v^{\alpha_1 - 1}
        (1 - v)^{\alpha_2 - 1}
        \mathbf 1_{(0, 1)}(v) \\
        & =
        \frac{\Gamma(\alpha_1 + \alpha_2)}{\Gamma(\alpha_1) \Gamma(\alpha_2)}
        v^{\alpha_1 - 1}
        (1 - v)^{\alpha_2 - 1}
        \mathbf 1_{(0, 1)}(v)
    \end{align*}

    For \enquote{$!$} we have used the following substitution.

    \begin{align*}
        s = \beta u
        \implies
        \begin{cases}
            u = \frac{s}{\beta} \\
            \derivative[][s]{u} = \beta \implies \mathrm d u = \frac{1}{\beta} \mathrm d s
        \end{cases}
    \end{align*}

    \begin{align*}
        \implies
        f_U(u) f_V(v) = f_{U, V}(u, v)        
    \end{align*}

\end{enumerate}

\end{solution}

% --------------------------------------------------------------------------------

\begin{exercise}

$f: \R \to \R$ sei überall differenzierbar. Zeigen Sie, dass $f^\prime$ Borel-messbar ist.

\end{exercise}

% --------------------------------------------------------------------------------

\begin{solution}

\begin{align*}
  \Forall n \in \N:
  f_n: x \mapsto \frac{f(x + 1/n) - f(x)}{1/n},
  \enspace \text{messb.}
  \Rightarrow
  f^\prime = \lim_{n \to \infty} f_n
  \enspace \text{messb.}
\end{align*}

\end{solution}

% --------------------------------------------------------------------------------


\section*{Logische Axiome, MP}
% -------------------------------------------------------------------------------- %

\begin{exercise}

\begin{enumerate}[label = (\alph*)]

  \item
  Definieren Sie: messbare Funktion, Treppenfunktion, Konvergenz im Maß, Konvergenz fast überall, Konvergenz fast gleichmäßig.
  
  \item
  Formulieren und beweisen Sie den Approximationssatz für reellwertige messbare Funktionen.

\end{enumerate}

\end{exercise}

% -------------------------------------------------------------------------------- %

\begin{solution}

\phantom{}

(a) $f: (\Omega_1, \mathfrak{S}_1) \to (\Omega_2, \mathfrak{S}_2) \enspace \text{Treppenfunktion}
: \Leftrightarrow
\Exists a_1, \ldots, a_n \in \Omega_2,
\Exists A_1, \ldots, A_n \in \mathfrak{S}_1:$

\begin{align*}
  \sum_{i=1}^n A_i = \Omega_1, \enspace
  \sum_{i=1}^n a_i A_i = f
\end{align*}

Rest siehe Aufgabe 1 und 12 (a). \\

(b) Siehe Skript.

\end{solution}

% -------------------------------------------------------------------------------- %


\end{document}
