% --------------------------------------------------------------------------------

\begin{exercise}[56]

Geben Sie ein Kriterium ab, das entscheidet, ob eine vorgegebene Zeichenfolge
ableitbar ist.
\end{exercise}

% --------------------------------------------------------------------------------

\begin{solution}

Wir vereinbaren die Kurzschreibweise $n := \underbrace{1\dots 1}_{n \text{ Mal}}$
und stellen folgendes Kriterium auf:

Eine Zeichenfolge $A$ ist genau dann ableitbar, wenn sie eine der folgenden
Bedingungen erfüllt:
\begin{enumerate}[label = \alph*)]
  \item $\exists n \in \N: A = n!$
  \item $\exists n,m \in N: A = (n + m = m\cdot n)$
\end{enumerate}
Wieder enthält die Menge an Zeichenfolgen, die unser Kriterium erfüllen das Axiom
und ist unter Regeln abgeschlossen, da
\begin{itemize}
  \item Regel 1: $\frac{n!}{(n+1)!}$
  \item Regel 2: $\frac{n!}{n + 1 = n}$
  \item Regel 3: $\frac{n + m = n\cdot m}{n + (m + 1) = m\cdot n + n = n(m+1)}$.
\end{itemize}
Weiters können wir für jedes Element dieser Menge explizit eine Herleitung angeben:
Für $n!$ müssen wir nur $(n-1)$-Mal die erste Regel anwenden, beginnend beim
einzigen Axiom. Für $n + m = n\cdot m$ leiten wir zuerst $n!$ her, wenden
einmal Regel 2 an, um $n + 1 = n$ zu erhalten und schließlich nach $(m-1)$-maliger
Anwendung der dritten Regel $n + m = n \cdot m$ zu erhalten.
\end{solution}

% --------------------------------------------------------------------------------
