% --------------------------------------------------------------------------------

\begin{exercise}[22]

Sei $n \geq 2$.
Wieviele Belegungen (der Variablen $p_1,\dots,p_n$) erfüllen die Formel
\begin{align*}
  A := (p_1 \to p_2) \land (p_2 \to p_3) \land \dots \land (p_{n-1} \to p_n)?
\end{align*}

\end{exercise}

% --------------------------------------------------------------------------------

\begin{solution}
Sei $b$ eine Belegung und wähle den kleinsten Index $i$, sodass $b(p_i) = 1$
(falls so ein Index nicht existiert, setze $i := n + 1$).
Soll $b$ nun die Formel $A$ erfüllen muss jedenfalls für $j > i: b(p_j) = 1$ gelten (Induktion).
Damit gibt es maximal $n+1$ Belegungen, die die Formel erfüllen können.
Definiere also
\begin{align*}
  b_i: p_j \mapsto \begin{cases}
    0 & j < i \\
    1 & j \geq i
  \end{cases}, i = 1,\dots,n+1.
\end{align*}
Fall 1: $j < i:$
\begin{align*}
  b_i(p_{j-1}) = 0, \quad b_i(p_j) = 0 \implies \hat{b_i}(p_{j-1} \to p_j) = 1
\end{align*}
Fall 2: $j = i$
\begin{align*}
  b_i(p_{j-1}) = 0, \quad b_i(p_j) = 1 \implies \hat{b_i}(p_{j-1} \to p_j) = 1
\end{align*}
Fall 3: $j > i:$
\begin{align*}
  b_i(p_{j-1}) = 1, \quad b_i(p_j) = 1 \implies \hat{b_i}(p_{j-1} \to p_j) = 1
\end{align*}
Damit erfüllen alle $b_i$ tatsächlich die Formel $A$ und wir erhalten insgesamt $n+1$ mögliche Belegungen.
\end{solution}

% --------------------------------------------------------------------------------
