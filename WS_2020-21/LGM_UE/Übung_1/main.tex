\documentclass{article}

% Hier befinden sich Pakete, die wir beinahe immer benutzen ...

\usepackage[utf8]{inputenc}

% Sprach-Paket:
\usepackage[ngerman]{babel}

% damit's nicht so, wie beim Grill aussieht:
\usepackage{fullpage}

% Mathematik:
\usepackage{amsmath, amssymb, amsfonts, amsthm}
\usepackage{bbm, mathrsfs, stmaryrd}
\usepackage{mathtools, mathdots}

% Makros mit mehereren Default-Argumenten:
\usepackage{twoopt}

% Anführungszeichen (Makro \Quote{}):
\usepackage{babel}

% if's für Makros:
\usepackage{xifthen}
\usepackage{etoolbox}

% tikz ist kein Zeichenprogramm (doch!):
\usepackage{tikz}

% bessere Aufzählungen:
\usepackage{enumitem}

% (bessere) Umgebung für Bilder:
\usepackage{graphicx, subfig, float}

% Umgebung für Code:
\usepackage{listings}

% Farben:
\usepackage{xcolor}

% Umgebung für "plain text":
\usepackage{verbatim}

% Umgebung für mehrerer Spalten:
\usepackage{multicol}

% "nette" Brüche
\usepackage{nicefrac}

% Spaltentypen verschiedener Dicke
\usepackage{tabularx}
\usepackage{makecell}

% Für Vektoren
\usepackage{esvect}

% (Web-)Links
\usepackage{hyperref}

% Zitieren & Literatur-Verzeichnis
\usepackage[style = authoryear]{biblatex}
\usepackage{csquotes}

% so ähnlich wie mathbb
%\usepackage{mathds}

% Keine Ahnung, was das macht ...
\usepackage{booktabs}
\usepackage{ngerman}
\usepackage{placeins}

% special letters:

\newcommand{\N}{\mathbb{N}}
\newcommand{\Z}{\mathbb{Z}}
\newcommand{\Q}{\mathbb{Q}}
\newcommand{\R}{\mathbb{R}}
\newcommand{\C}{\mathbb{C}}
\newcommand{\K}{\mathbb{K}}
\newcommand{\T}{\mathbb{T}}
\newcommand{\E}{\mathbb{E}}
\newcommand{\V}{\mathbb{V}}
\renewcommand{\P}{\mathbb{P}}
\newcommand{\1}{\mathbbm{1}}

\newcommand  {\B}{\mathfrak{B}}
\renewcommand{\S}{\mathfrak{S}}

% quantors:

\newcommand{\Forall}{\forall \,}
\newcommand{\Exists}{\exists \,}
\newcommand{\ExistsOnlyOne}{\exists! \,}
\newcommand{\nExists}{\nexists \,}

% MISC symbols:

\newcommand{\landau}[1]
{
  {\scriptstyle \mathcal{O}}
  \pbraces{#1}
}

\newcommand{\Landau}[1]
{
  \mathcal{O}
  \pbraces{#1}
}

\newcommand{\eps}{\mathrm{eps}}

% graphics in a box:

\newcommandtwoopt
{\includegraphicsboxed}[3][][]
{
  \begin{figure}[!h]
    \begin{boxedin}
      \ifthenelse{\isempty{#2}}
      {
        \begin{center}
          \includegraphics[width = 0.75 \textwidth]{#3}
          \label{fig:#1}
        \end{center}
      }{
        \begin{center}
          \includegraphics[width = 0.75 \textwidth]{#3}
          \caption{#2}
          \label{fig:#1}
        \end{center}
      }
    \end{boxedin}
  \end{figure}
}

% braces:

\newcommand{\pbraces}[1]{{\left  ( #1 \right  )}}
\newcommand{\bbraces}[1]{{\left  [ #1 \right  ]}}
\newcommand{\Bbraces}[1]{{\left \{ #1 \right \}}}
\newcommand{\vbraces}[1]{{\left  | #1 \right  |}}
\newcommand{\Vbraces}[1]{{\left \| #1 \right \|}}
\newcommand{\abraces}[1]{{\left \langle #1 \right \rangle}}
\newcommand{\round}[1]{\bbraces{#1}}

\newcommand
{\floor}[1]
{{\left \lfloor #1 \right \rfloor}}

\newcommand
{\ceil} [1]
{{\left \lceil  #1 \right \rceil }}

% special functions:

\newcommand{\norm}  [2][]{\Vbraces{#2}_{#1}}
\newcommand{\diag}  [1]{\mathrm{diag} \: #1}
\newcommand{\dist}  [1]{\mathrm{dist} \: #1}
\newcommand{\mean}  [1]{\mathrm{mean} \: #1}
\newcommand{\erf}   [1]{\mathrm{erf} \: #1}
\newcommand{\id}    [1]{\mathrm{id} \: #1}
\newcommand{\sgn}   [1]{\mathrm{sgn} \: #1}
\newcommand{\supp}  [1]{\mathrm{supp} \: #1}
\newcommand{\arsinh}[1]{\mathrm{arsinh} \: #1}
\newcommand{\arcosh}[1]{\mathrm{arcosh} \: #1}
\newcommand{\artanh}[1]{\mathrm{artanh} \: #1}
\newcommand{\card}  [1]{\mathrm{card} \: #1}
\newcommand{\Span}  [1]{\mathrm{span} \: #1}
\newcommand{\Aut}   [1]{\mathrm{Aut} \: #1}
\newcommand{\End}   [1]{\mathrm{End} \: #1}
\newcommand{\ggT}   [1]{\mathrm{ggT} \: #1}
\newcommand{\kgV}   [1]{\mathrm{kgV} \: #1}
\newcommand{\ord}   [1]{\mathrm{ord} \: #1}
\newcommand{\grad}  [1]{\mathrm{grad} \: #1}
\newcommand{\ran}   [1]{\mathrm{ran} \: #1}
\newcommand{\graph} [1]{\mathrm{graph} \: #1}
\newcommand{\Inv}   [1]{\mathrm{Inv} \: #1}
\newcommand{\pv}    [1]{\mathrm{pv} \: #1}
\newcommand{\Mod}{\: \mathrm{mod} \:}
\newcommand{\Char}{\mathrm{char}}
\newcommand{\At}{\mathrm{At}}
\newcommand{\Ob}{\mathrm{Ob}}
\newcommand{\Hom}{\mathrm{Hom}}
\newcommand{\orthogonal}[3][]{#2 ~\bot_{#1}~ #3}
\newcommand{\Rang}{\mathrm{Rang}}

\newcommand
{\GL}[2][]
{\mathrm{GL}_{#1} \pbraces{#2}}

% fractions:

\newcommand{\Frac}[2]{\frac{1}{#1} \pbraces{#2}}
\newcommand{\nfrac}[2]{\nicefrac{#1}{#2}}

% derivatives & integrals:

\newcommandtwoopt
{\Int}[4][][]
{\int_{#1}^{#2} #3 ~\mathrm{d} #4}

\newcommandtwoopt
{\derivative}[3][][]
{
  \frac
  {\mathrm{d}^{#1} #2}
  {\mathrm{d} #3^{#1}}
}

\newcommandtwoopt
{\pderivative}[3][][]
{
  \frac
  {\partial^{#1} #2}
  {\partial #3^{#1}}
}

\newcommand
{\primeprime}
{{\prime \prime}}

\newcommand
{\primeprimeprime}
{{\prime \prime \prime}}

% Text:

\newcommand{\Quote}[1]{\glqq #1\grqq{}}
\newcommand{\Text}[1]{{\text{#1}}}
\newcommand{\fastueberall}{\text{f.ü.}}
\newcommand{\fastsicher}{\text{f.s.}}

% -------------------------------- %
% amsthm-stuff:

\theoremstyle{definition}

% numbered theorems
\newtheorem{theorem}    {Satz}   [section]
\newtheorem{lemma}      [theorem]{Lemma}
\newtheorem{corollary}  [theorem]{Korollar}
\newtheorem{proposition}[theorem]{Proposition}
\newtheorem{remark}     [theorem]{Bemerkung}
\newtheorem{definition} [theorem]{Definition}
\newtheorem{example}    [theorem]{Beispiel}

% unnumbered theorems
\newtheorem*{theorem*}    {Satz}
\newtheorem*{lemma*}      {Lemma}
\newtheorem*{corollary*}  {Korollar}
\newtheorem*{proposition*}{Proposition}
\newtheorem*{remark*}     {Bemerkung}
\newtheorem*{definition*} {Definition}
\newtheorem*{example*}    {Beispiel}

% Please define this stuff in project ("main.tex"):

% \def \lastexercisenumber {...}
% This will be 0 by default

% \setcounter{section}{...}
% This will be 0 by default
% and hence, completely ignored

\ifnum \thesection = 0
{
  \newtheorem{exercise}{Aufgabe}
}
\else
{
  \newtheorem{exercise}{Aufgabe}[section]
}
\fi

\ifdef
{\lastexercisenumber}
{\setcounter{exercise}{\lastexercisenumber}}

\newenvironment{solution}
{
  \begin{proof}[Lösung]
}{
  \end{proof}
}

\renewcommand{\proofname}{Beweis}

% -------------------------------- %
% environment zum einkasteln:

% dickere vertical lines
\newcolumntype
{x}
[1]
{
  !{
    \centering
    \arraybackslash
    \vrule
    width #1}
}

% environment selbst (the big cheese)
\newenvironment
{boxedin}
{
  \begin{tabular}
  {
    x{1 pt}
    p{\textwidth}
    x{1 pt}
  }
  \Xhline
  {2 \arrayrulewidth}
}
{
  \\
  \Xhline{2 \arrayrulewidth}
  \end{tabular}
}

% -------------------------------- %
% MISC "Ein-Deutschungen"

\renewcommand{\figurename}{Abbildung}
\renewcommand{\tablename} {Tabelle}

% -------------------------------- %


\parindent 0pt

\title
{
  Logik und Grundlagen der Mathematik - Übung \\
  \vspace{4pt}
  \normalsize
  \textit{1. UE am 8.10.2020}
}
\author
{
  Richard Weiss
  \and
  Florian Schager
  % \and
  % Christian Sallinger
  \and
  Fabian Zehetgruber
  % \and
  % Paul Winkler
  % \and
  % Christian Göth
}
\date{}

\begin{document}

\maketitle

Hinweis:
Manche (sehr wenige) der folgenden Beispiele sind falsch, manche enthalten offene Fragen, manche sind besonders schwierig.
Die Lösung eines falschen Beispiels besteht in einer Erklärung, was bzw. warum etwas falsch ist.
(Ein falscher Allsatz kann zB durch ein Gegenbeispiel widerlegt werden.)

% --------------------------------------------------------------------------------

\begin{exercise}[Card game]

A deck of $52$ cards has $13$ ranks ($2$, $3$, $4$, $5$, $6$, $7$, $8$, $9$, $10$, $J$, $Q$, $K$, $A$) an $4$ suits ($\textcolor{red}{\heartsuit}$, $\spadesuit$, $\textcolor{red}{\diamondsuit}$, $\clubsuit$).
Three cards are drawn randomly without replacement from a deck of $52$ cards.

\begin{enumerate}[label = (\alph*)]
    \item What ist the probability that all cards are in the same suit?
    \item What ist the probability that all cards have the same rank?
    \item What ist the probability that the three cards contain exactly one pair (a pair means two cards with the same rank from two different suits)?
\end{enumerate}

\end{exercise}

% --------------------------------------------------------------------------------

\begin{solution}

ToDo!

\end{solution}

% --------------------------------------------------------------------------------

% --------------------------------------------------------------------------------

\begin{exercise}[2]

Von der Eigenschaft $E$ wissen wir bereits, dass sie auf alle Singletons (= einelementige Mengen) zutrifft.
Nehmen wir an, dass $E$ immer dann auf eine Menge $A \cup \Bbraces{b}$ zutrifft, wenn $E$ auf $A$ zutrifft (und $b$ beliebig ist).
Können wir daraus schließen,

\begin{itemize}
    \item ... dass $E$ für alle endlichen nichtleeren Mengen gilt?
    \item ... dass $E$ für alle nichtleeren Mengen gilt?
    \item ... dass $E$ für alle höchstens abzählbaren nichtleeren Mengen gilt?
\end{itemize}

\end{exercise}

% --------------------------------------------------------------------------------

\begin{solution}
\phantom{}
\begin{itemize}
    \item Ja! Vollständige Induktion nach der Mächtigkeit der Menge.
    Unsere Induktionsbehauptung lautet: Für alle Mengen $B$ mit $|B| = n$ gilt
    \begin{align*}
      E(B) \implies \forall b: E(B \cup \{b\})
    \end{align*}
    Den Induktionsanfang für $n = 1$ erhalten wir aus der Voraussetzung.
    Gelte die Eigenschaft nun für alle Mengen $A$ mit $|A| = n$ und sei $B$
    mit $|B| = n + 1$ beliebig. Wähle ein beliebiges $b_0  \in B$. Dann gilt
    \begin{align*}
      B = B\{x_0\} \cup \{x_0\}
    \end{align*}
    und aufgrund $|B\{x_0\}| = n$ gilt nach Induktionsvoraussetzung $E(B)$
    \item Nein! Gegenbeispiel: $E(A)$ sei die Eigenschaft $|A| < \infty$.
    Klarerweise erfüllen alle Singletons $E$.
    Gelte nun $E(A)$, also $|A| \leq \infty$. Also existiert ein $n \in \N$
    mit $|A| = n$ und somit folgt für alle
    \begin{align*}
      b: |A \cup \{b\}| \leq n + 1
    \end{align*} und daher gilt auch $E(A \cup \{b\})$. \\
    Aber bereits abzählbar unendliche Mengen erfüllen die Eigenschaft nicht mehr.
\end{itemize}

\end{solution}

% --------------------------------------------------------------------------------

% -------------------------------------------------------------------------------- %

\begin{exercise}

\phantom{}

\begin{enumerate}[label = (\alph*)]
  
  \item
  Definieren Sie Ring, Semiring, monotones System, Dynkin-System, Algebra, Sigmaalgebra.

  \item
  Zeigen Sie: Wenn $\mathfrak{R}$ ein Ring ist, dann stimmt das von $\mathfrak{R}$ erzeugte monotone System mit dem erzeugten Sigmaring überein.

\end{enumerate}

\end{exercise}

% -------------------------------------------------------------------------------- %

\begin{solution}

\phantom{}

\begin{itemize}

  \item $\emptyset \neq \mathfrak{R} \subseteq 2^\Omega \enspace \text{Ring} : \Leftrightarrow \Forall A, B \in \mathfrak{R}:$
  \begin{itemize}
    \item $A \cup B \in \mathfrak{R}$
    \item $A \setminus B \in \mathfrak{R}$
  \end{itemize}

  \item $\emptyset \neq \mathfrak{T} \subseteq 2^\Omega \enspace \text{Semiring} : \Leftrightarrow \Forall A, B \in \mathfrak{T}:$
  \begin{itemize}
    \item $A \cap B \in \mathfrak{T},$
    \item $A \subseteq B \Rightarrow \Exists C_1, \ldots, C_n \in \mathfrak{T}, \Text{disj.}:
    B \setminus A = \sum_{i=1}^n C_i,$
    \item $\Forall k = 1, \ldots, n:
    A \cup \sum_{i=1}^k C_i \in \mathfrak{T}$
  \end{itemize}

  \item $\mathfrak{M} \subseteq 2^\Omega \enspace \text{monotones System} : \Leftrightarrow \Forall (A_n) \in \mathfrak{M}, \Text{mon.}: \lim_{n \to \infty} A_n \in \mathfrak{M}$

  \item $\emptyset \neq \mathfrak{D} \subseteq 2^\Omega \enspace \text{Dynkin-System} : \Leftrightarrow$
  \begin{itemize}
    \item $\Forall A, B \in \mathfrak{D}:
    A \subseteq B \Rightarrow B \setminus A \in \mathfrak{D}$
    \item $\Forall (A_n) \in \mathfrak{D}, \text{disj.}: \sum_{n \in \N} A_n \in \mathfrak{D}$
    \item $\Omega \in \mathfrak{D}$
  \end{itemize}

  \item $\emptyset \neq \mathfrak{A} \subseteq 2^\Omega \enspace \text{Algebra} : \Leftrightarrow$
  \begin{itemize}
    \item $\mathfrak{A} \enspace \text{Ring},$
    \item $\Omega \in \mathfrak{A}$
  \end{itemize}

  \item $\emptyset \neq \mathfrak{A}_\sigma \subseteq 2^\Omega \enspace \text{Sigmaalgebra} : \Leftrightarrow$
  \begin{itemize}
    \item $\Forall (A_n) \in \mathfrak{A}_\sigma, \text{disj.}: \sum_{n \in \N} A_n \in \mathfrak{A}_\sigma$
    \item $\Forall A, B \in \mathfrak{A}_\sigma: A \setminus B \in \mathfrak{A}_\sigma,$
    \item $\Omega \in \mathfrak{A}_\sigma$
  \end{itemize}

\end{itemize}

Der nächste Teil ist genau das \enquote{Monotone Class Theorem}! Siehe Skript.

\end{solution}

% -------------------------------------------------------------------------------- %

% -------------------------------------------------------------------------------- %

\begin{exercise}

Berechnen Sie die Ableitung der Funktion

\begin{align*}
    f(t)
    =
    \Int[0][1]
    {
        \frac{1}{x}
        \sin(t x)
    }{x}.
\end{align*}

Begründen Sie dabei genau alle nichtelementaren Schritte!

\end{exercise}

% -------------------------------------------------------------------------------- %

\begin{solution}

\phantom{}

\begin{align*}
    g(t, x)
    =
    \frac{1}{x}
    \sin(t x)
    \implies
    \lim_{x \to 0}
    g(t, x)
    =
    \lim_{x \to 0}
    \frac{\sin(t x) - \sin{0}}{x - 0}
    =
    \pderivative{x}
    \sin(t x)
    \Big |_{x=0}
    =
    t \cos(t 0)
    =
    t
\end{align*}

Wir können also den Integranden $g$ stetig auf ganz $[0, 1]^2$ fortsetzen.

\begin{tcolorbox}[standard jigsaw, opacityback = 0]
    \centering
    \includegraphics
    [width = 0.75 \textwidth]
    {Ana1&2/Ana1&2 - 8.7.12.1 Korollar.png} \\
    \includegraphics
    [width = 0.75 \textwidth]
    {Ana1&2/Ana1&2 - 8.7.12.2 Korollar.png}
\end{tcolorbox}

Laut Korollar 8.7.12, können wir also die Ableitung in das Parameterintegral hineinziehen.

\begin{align*}
    \implies
    f^\prime(t)
    =
    \Int[0][1]
    {
        \derivative{t}
        \frac{1}{x}
        \sin(t x)
    }{x}
    =
    \Int[0][1]
    {
        \cos(t x)
    }{x}
    \stackrel{!}{=}
    \frac{1}{t}
    \Int[0][t]{\cos u}{u}
    =
    \frac{1}{t}
    \sin{u} \Big |_{u=0}^t
    =
    \frac{\sin t}{t}
\end{align*}

Dabei haben wir folgende Substitution verwendet.

\begin{align*}
    u = t x
    \implies
    \derivative[][u]{x} = t
    \implies
    \mathrm{d}x = \mathrm{d}u \frac{1}{t}
\end{align*}

\end{solution}

% -------------------------------------------------------------------------------- %

% --------------------------------------------------------------------------------

\begin{exercise}[18]

Zeigen Sie:

\begin{enumerate}[label = \alph*]
    \item $\neg (p_1 \land p_2) \Leftrightarrow \neg p_1 \lor \neg p_2$.
    \item Für alle Formeln $A$ und $B$ gilt $\neg (A \land B) \Leftrightarrow \neg A \lor \neg B$.
\end{enumerate}

\end{exercise}

% --------------------------------------------------------------------------------

\begin{solution}

ToDo!

\end{solution}

% --------------------------------------------------------------------------------

% -------------------------------------------------------------------------------- %

\begin{exercise}

Für $a, t > 0$ sei

\begin{align*}
    u(t, a)
    =
    \Int[0][+\infty]
    {
        \frac{t}{t^2 + x^2}
        \cos{a x}
    }{x}.
\end{align*}

Zeigen Sie, dass $u: (0, +\infty) \times (0, +\infty) \to \R$ stetig ist und dass für jedes feste $a > 0$ die Funktion $t \mapsto u(t, a)$ differenzierbar ist.

Weiters berechne man $\lim_{a \to 0} u(t, a)$.

\end{exercise}

% -------------------------------------------------------------------------------- %

\begin{solution}

\phantom{}

\begin{enumerate}[label = \arabic*.]

    \item Teil ($u$ stetig):
    
    \begin{multline*}
        \text{d.h.}~
        \Forall a, t \in (0, +\infty):
        \Forall (t_n)_{n \in \N}, (a_n)_{n \in \N} \in (0, +\infty)^\N: \\
        \pbraces
        {
            t_n \xrightarrow{n \to \infty} t,
            a_n \xrightarrow{n \to \infty} a
        }
        \implies
        \lim_{n \to \infty} u(t_n, a_n) = u(t, a)
    \end{multline*}

    Weil $(a_n)_{n \in \N}$ konvergiert, finden wir ein Intervall $(\alpha, \beta) \subset (0, +\infty)$, sodass für fast alle (o.B.d.A. alle) $n \in \N$, $t_n \in (\alpha, \beta)$.

    \includegraphicsboxed{MassWHT1&2/MassWHT1&2 - Satz 5.7.png}

    Wir wollen Satz 5.7 (Satz von der dominierten Konvergenz) auf die folgende Folge von stetigen, also messbaren, Funktionen anwenden.

    \begin{align*}
        f_n(x)
        :=
        \frac{t_n}{t_n^2 + x^2}
        \cos(a_n x),
        \quad
        x \in (0, +\infty),
        \quad
        n \in \N
    \end{align*}

    \begin{align*}
        \implies
        |f_n(x)|
        =
        \abs
        {
            \frac{t_n}{t_n^2 + x^2}
        }
        \underbrace
        {
            |\cos(a_n x)|
        }_{
            \leq 1
        }
        \leq
        \frac{\beta}{\alpha^2 + x^2}
        =:
        g(x)
    \end{align*}

    Die Majorante $g$ ist auch integrierbar.

    \begin{multline*}
        \Int[0][\infty]{g(x)}{x}
        =
        \Int[0][\infty]
        {
            \frac{\beta}{\alpha^2 + x^2}
        }{x}
        =
        \frac{\beta}{\alpha^2}
        \Int[0][\infty]
        {
            \frac{1}{1 + (x / \alpha)^2}
        }{x} \\
        \stackrel{!}{=}
        \frac{\beta}{\alpha}
        \Int[0][\infty]{\frac{1}{1 + u^2}}{u}
        =
        \frac{\beta}{\alpha}
        \arctan u \Big |_{u=0}^\infty
        =
        \frac{\beta}{\alpha}
        \frac{\pi}{2}
        <
        \infty
    \end{multline*}

    Dabei haben wir folgende Substitution verwendet.

    \begin{align*}
        u = \frac{x}{\alpha}
        \implies
        \derivative[][u]{x} = \frac{1}{\alpha}
        \implies
        \mathrm{d}x = \alpha \mathrm{d}u
    \end{align*}

    \item Teil ($\Forall a > 0: u(\cdot, a)$ differenzierbar):
    
    Seien $t_0 \in (0, +\infty)$ und $\delta > 0$, sodass $B(t_0, \delta) \subset (0, +\infty)$.
    Wir wollen wieder Satz 2.1.7 (siehe oben) anwenden; diesmal auf die folgende Funktionen-Familie.

    \begin{align*}
        f_t(x)
        :=
        \frac{t}{t^2 + x^2}
        \cos(a x)
        =
        \frac{1}{t + x^2 / t}
        \cos(a x),
        \quad
        x, t \in (0, +\infty)
    \end{align*}

    Wir finden eine, von $t$ unabhängige, Majorante.

    \begin{align*}
        \abs
        {
            \pderivative[][f_t]{t}(x)
        }
        \leq
        \abs
        {
            \frac{-1}{(t + x^2 / t)^2}
            (1 - x^2 / t^2)
            \cos(a x)
        }
        =
        \abs
        {
            \frac{x^2 - t^2}{(x^2 + t^2)^2}
        }
        \underbrace{|\cos(a x)|}_{\leq 1}
        \leq
        \frac
        {
            (t_0 + \delta)^2 + x^2
        }{
            ((t_0 - \delta)^2 + x^2)^2
        }
        =:
        g(x)
    \end{align*}

    Die Majorante $g$ ist auch integrierbar.

    \begin{multline*}
        \Int[0][\infty]{g(x)}{x}
        \stackrel
        {
            \mathrm{WolframAlpha}
        }{=}
        \frac{1}{4 (t_0 - \delta)^3}
        \pi
        (
            (t_0 + \delta)^2
            +
            (t_0 - \delta)^2
        ) \\
        =
        \frac{1}{4 (t_0 - \delta)^3}
        \pi
        (
            t_0^2 + 2 t_0 \delta + \delta^2
            +
            t_0^2 - 2 t_0 \delta + \delta^2
        )
        =
        \frac{\pi}{2}
        \frac{t_0^2 + \delta^2}{(t_0 - \delta)^3}
        <
        \infty
    \end{multline*}

    \item Teil ($\lim_{a \to 0} u(t, a)$):
    
    Wir wissen ja bereits, dass $u$ stetig ist.
    Daher dürfen dir den $\lim$ in $u$ hineinziehen.

    \begin{align*}
        \implies
        \lim_{a \to 0} u(t, a)
        =
        u(t, 0)
        =
        \Int[0][\infty]
        {
            \frac{t}{t^2 + x^2}
        }{x}
        =
        \frac{1}{t}
        \Int[0][\infty]
        {
            \frac{1}{1 + (x / t)^2}
        }{x}
        \stackrel{!}{=}
        \arctan u \Big |_{u=0}^\infty
        =
        \frac{\pi}{2}
    \end{align*}

    Dabei haben wir folgende Substitution verwendet.

    \begin{align*}
        u = \frac{x}{t}
        \implies
        \derivative[][u]{x} = \frac{1}{t}
        \implies
        \mathrm{d}x = t \mathrm{d}x
    \end{align*}

\end{enumerate}

\end{solution}

% -------------------------------------------------------------------------------- %

\begin{exercise}

Hier könnte Ihre Werbung stehen!

\begin{itemize}
  \item[(a)] Definieren Sie: äußeres Maß, Messbarkeit nach Caratheodory, von einer Maßfunktion erzeugtes Maß.
  \item[(b)] Zeigen Sie: wenn $\mu^\ast_n$ äußere Maße über derselben Menge sind, dann auch $\sup_n \mu^\ast_n$
\end{itemize}

\end{exercise}

% --------------------------------------------------------------------------------

\begin{solution}

(a) Siehe Aufgabe 1 (a) und Aufgabe 4 (a). \\

(b) Die ersten drei Eigenschaften sind offensichtlich. Seien also $A, (B_k) \in 2^\Omega$, mit $A \subseteq \bigcup_{k \in \N} B_k$, dann gilt Folgendes.
\begin{align*}
  \sum_{k \in \N} \sup_n \mu^\ast_n(B_k)
  \geq
  \sup_n \sum_{k \in \N} \mu^\ast_n(B_k)
  \geq
  \sup_n \mu^\ast_n(A)
\end{align*}

\end{solution}


\phantom{}

Sei $b$ eine Belegung, $A$ eine Formel.
Statt $\hat{b}(A) = 1$ sagen wir auch \Quote{$b$ erfüllt die Formel $A$}.
In den nächsten 3 Aufgabe verstehen wir unter einer \Quote{Belegung} eine Funktion von der Menge $\Bbraces{p_1, \ldots, p_n}$ nach $\Bbraces{0, 1}$.

% --------------------------------------------------------------------------------

\begin{exercise}[Exercise 3.4]

Give a table analogous to that in Example 3.3 (textbook p. 52), but for $p(s',r|s,a)$.
It should have columns for $s, a, s', r$, and $p(s',r|s,a)$, and a row for every 4-tuple
for which $p(s',r|s,a) > 0$.

\end{exercise}

% --------------------------------------------------------------------------------

\begin{solution}
Let $p_{\texttt{search}}$ and $p_{\texttt{wait}}$ denote the probability
distributions of the number of cans picked up while searching or waiting
respectively.
\begin{center}
\begin{tabular}{ ccc|c|c }
 $s$ & $a$ & $s'$ & $r$ & $p(s',r|s,a)$\\
 \hline
 \texttt{high} & \texttt{search} & \texttt{high} & $k$ & $\alpha \cdot p_{\texttt{search}}(k)$ \\
 \texttt{high} & \texttt{search} & \texttt{low} & $k$ & $(1 -\alpha) \cdot p_{\texttt{search}}(k)$ \\
 \texttt{low} & \texttt{search} & \texttt{high} & $-3$ & $1- \beta$ \\
 \texttt{low} & \texttt{search} & \texttt{low} & $k$ & $\beta \cdot p_{\texttt{search}}(k)$ \\
 \texttt{high} & \texttt{wait} & \texttt{high} & $k$ & $p_{\texttt{wait}}(k)$ \\
 \texttt{low} & \texttt{wait} & \texttt{low} & $k$ & $p_{\texttt{wait}}(k)$ \\
 \texttt{low} & \texttt{recharge} & \texttt{high} & $0$ & $1$ \\
\end{tabular}
\end{center}


\end{solution}

% --------------------------------------------------------------------------------


\end{document}
