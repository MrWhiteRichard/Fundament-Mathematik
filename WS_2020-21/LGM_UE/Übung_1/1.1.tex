% --------------------------------------------------------------------------------

\begin{exercise}[1]

Welche der folgenden Aussagen gelten allgemein (d.h., für beliebige $x, x_1, y, \ldots$)?
Begründen Sie Ihre Antwort (Beweis oder Gegenbeispiel).

\begin{enumerate}[label = \alph*.]

    \item Wenn $\Bbraces{x} = \Bbraces{y}$, dann ist auch $x = y$.

    \item Wenn $\Bbraces{x, z} = \Bbraces{y, z}$, dann ist auch $x = y$.

    \item Wenn $\Bbraces{x_1, x_2} = \Bbraces{y_1, y_2}$, dann gilt zumindest eine der folgenden beiden Aussagen:
    
    \begin{align*}
        \text{(12)} \enspace
        x_1 = y_1 ~\text{und}~ x_2 = y_2;
        \quad
        \text{(21)} \enspace
        x_1 = y_2 ~\text{und}~ x_2 = y_1.
    \end{align*}

    \item Wenn $\Bbraces{x_1, x_2, x_3} = \Bbraces{y_1, y_2, y_3}$, dann ist zumindest eine der folgenden 6 Aussagen wahr:
    
    \begin{align*}
        \text{(123)} \enspace
        x_1 = y_1, x_2 = y_2, x_3 = y_3. \\
        \text{(132)} \enspace
        x_1 = y_1, x_2 = y_3, x_3 = y_2. \\
        \text{(213)} \enspace
        x_1 = y_2, x_2 = y_1, x_3 = y_3. \\
        \text{(231)} \enspace
        x_1 = y_2, x_2 = y_3, x_3 = y_1. \\
        \text{(312)} \enspace
        x_1 = y_3, x_2 = y_1, x_3 = y_2. \\
        \text{(321)} \enspace
        x_1 = y_3, x_2 = y_2, x_3 = y_1.
    \end{align*}
\end{enumerate}

\end{exercise}

% --------------------------------------------------------------------------------

\begin{solution}

\phantom{}

\begin{enumerate}[label = \alph*.]

    \item

    \begin{align*}
        \Bbraces{x} = \Bbraces{y}
        \iff
        \Forall z:
        \underbrace{z \in \Bbraces{x}}_{\iff z = x}
        \iff
        \underbrace{z \in \Bbraces{y}}_{\iff z = y}
        \iff
        x = y
    \end{align*}

    \item

    \begin{align*}
        \Bbraces{x, z} = \Bbraces{y, z}
        \iff
        \Forall a:
        \underbrace{a \in \Bbraces{x, z}}_{\iff a = x \lor a = z}
        \iff
        \underbrace{a \in \Bbraces{y, z}}_{\iff a = y \lor a = z}
    \end{align*}

    Damit wir die obige Formel verwenden können (der Übersicht halber), wählen wir $a := x$.

    \begin{itemize}
        
        \item
        [\Quote{$a = y$}:]
        Q.E.D

        \item
        [\Quote{$a \neq y$}:]

        \begin{align*}
            \implies
            x = a = z
            \implies
            |\Bbraces{x, z}| = 1 \neq 2 = |\Bbraces{y, z}|
        \end{align*}

    \end{itemize}

    \item

    \begin{align*}
        \implies
        x_1 \in \Bbraces{y_1, y_2}
        \iff
        x_1 = y_1 \lor x_1 = y_2
    \end{align*}

    \begin{align*}
        \text{o.B.d.A}~ x_1 = y_1
        \stackrel{\text{b.}}{\implies}
        ~\text{(12)}
    \end{align*}

    \item $X := \Bbraces{x_1, x_2, x_3}$, $Y := \Bbraces{y_1, y_2, y_3}$

    \begin{itemize}

        \item
        [\Quote{$|X| = 1$}:]
        \begin{align*}
            \Bbraces{x_1} = \Bbraces{y_1}
            \stackrel{\text{a.}}{\implies}
            ~\text{Behauptung}
        \end{align*}

        \item
        [\Quote{$|X| = 2$}:]
        \begin{align*}
            \Bbraces{x_1, x_2} = \Bbraces{y_1, y_2}
            \stackrel{\text{c.}}{\implies}
            ~\text{Behauptung}
        \end{align*}

        \item
        [\Quote{$|X| = 3$}:]
        \begin{align*}
            \implies
            & \Exists f:
            X = Y \to \Bbraces{1, 2, 3},
            ~\text{bijektiv},
            \begin{cases}
                x_1 \mapsto 1 \\
                x_2 \mapsto 2 \\
                x_3 \mapsto 3
            \end{cases} \\
            \implies
            & \Exists \pi \in S_3: (f(y_1), f(y_2), f(y_3)) = (\pi(1), \pi(2), \pi(3)) \\
            \implies
            & (f(y_{\pi^{-1}(1)}), f(y_{\pi^{-1}(2)}), f(y_{\pi^{-1}(3)})) = (\pi(\pi^{-1}(1)), \pi(\pi^{-1}(2)), \pi(\pi^{-1}(3))) = \\
            & (1, 2, 3) = (f(x_1), f(x_2), f(x_3)) \\
            \implies
            & x_1 = y_{\pi^{-1}(1)}, x_2 = y_{\pi^{-1}(2)}, x_3 = y_{\pi^{-1}(3)}
        \end{align*}

    \end{itemize}
	\textbf{Hier sei noch auf folgendes Beispiel hingewiesen:} Sei $a \neq b$ und 
	\begin{align*}
	x_1 = x_2 = y_1 = a \quad \text{sowie} \quad x_3 = y_2 = y_3 = b \quad \text{also} \quad \\
	\{x_1, x_2, x_3\} = \{a,a,b\} = \{a,b\}  = \{a,b,b\} = \{y_1, y_2, y_3\}.
	\end{align*}
	Nun gilt $x_1 \neq y_2$ und $x_1 \neq y_3$, aber $x_1 = y_1$, es bleiben also noch die Permutationen $(123)$ und $(132)$ als Möglichkeiten. Nun gilt aber weder $x_2 = y_2$ noch $x_2 = y_3$ also tritt keiner der in der Angabe genannten Fälle auf.
\end{enumerate}

\end{solution}

% --------------------------------------------------------------------------------
