% --------------------------------------------------------------------------------

\begin{exercise}[2]

Von der Eigenschaft $E$ wissen wir bereits, dass sie auf alle Singletons (= einelementige Mengen) zutrifft.
Nehmen wir an, dass $E$ immer dann auf eine Menge $A \cup \Bbraces{b}$ zutrifft, wenn $E$ auf $A$ zutrifft (und $b$ beliebig ist).
Können wir daraus schließen,

\begin{itemize}
    \item ... dass $E$ für alle endlichen nichtleeren Mengen gilt?
    \item ... dass $E$ für alle nichtleeren Mengen gilt?
    \item ... dass $E$ für alle höchstens abzählbaren nichtleeren Mengen gilt?
\end{itemize}

\end{exercise}

% --------------------------------------------------------------------------------

\begin{solution}

  \phantom{}

  \begin{itemize}

    \item Ja!
    Beweis:
    Vollständige Induktion nach der Mächtigkeit der Menge:
    Unsere Induktionsbehauptung lautet

    \begin{align*}
      \Forall n \in \N:
      \Forall A ~\text{Menge}, |A| = n:
      E(A).
    \end{align*}

    Den Induktionsanfang für $n = 1$ erhalten wir aus der Voraussetzung. \\
    Gelte $E(A)$ für alle Mengen $A$ mit $|A| = n$.
    Sei $B$ mit $|B| = n + 1$ beliebig.
    Wähle ein beliebiges $x \in B$.
    Wir setzen $A := B \setminus \Bbraces{x}$.

    \begin{align*}
      \implies
      |A| = |B \setminus \Bbraces{x}| = |B| - 1 = n
      \implies
      E(A)
    \end{align*}

    Nach Induktionsvoraussetzung erhalten wir also $E(A \cup \Bbraces{x})$.
    Damit gilt aber auch $E(B)$, weil

    \begin{align*}
      \implies
      B = (B \setminus \Bbraces{x}) \cup \Bbraces{x}
      =
      A \cup \Bbraces{x}.
    \end{align*}

    \item Nein!
    Gegenbeispiel:

    \begin{align*}
      E(A) :\iff |A| < \infty,
      \quad
      A ~\text{Menge}
    \end{align*}

    Klarerweise erfüllen alle Singletons $E$.
    Gelte nun $E(A)$, für eine Menge $A$, also $|A| < \infty$.

    \begin{align*}
      \implies
      \Forall b:
      |A \cup \Bbraces{b}| \leq |A| + 1 < \infty
    \end{align*}

    Daher gilt auch $E(A \cup \Bbraces{b})$.
    $E$ gilt aber nicht für unendliche Mengen (d.h. Mengen mit unendlicher Mächtigkeit).

    \item Nein!
    Gegenbeispiel (von vorher):
    Bereits abzählbar unendliche Mengen sind unendlich.
    Sie erfüllen daher die Eigenschaft $E$ nicht mehr.

  \end{itemize}

\end{solution}

% --------------------------------------------------------------------------------
