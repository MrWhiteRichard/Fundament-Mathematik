% --------------------------------------------------------------------------------

\begin{exercise}[75]

\phantom{}
	Zeigen Sie von möglichst vielen der folgenden Mengen, dass Sie Spektren sind: Die Menge der Primzahlen, die Menge aller zusammengesetzen Zahlen (=Nichtprimzahlen $> 1$), die Menge aller Quadratzahlen, die Menge aller Zweierpotenzen, die Menge aller Potenzen von $5$, die Menge aller Primzahlpotenzen, die Menge $\{1, 14, 141, 1414, 14142, \dots \} = \Bbraces{\floorbraces{10^n \sqrt{2}} : n = 0, 1, 2, \dots}$.

\end{exercise}

% --------------------------------------------------------------------------------

\begin{solution}
\phantom{}

	\begin{enumerate}[label = \arabic*.]
		\item Primzahlpotenzen $R$: Sei $\varphi$ eine Formel, für die $\mathrm{Mod}(\varphi)$ die Klasse aller Körper ist.
		Dann ist das Spektrum von $\varphi$ genau die Menge aller Primzahlpotenzen (Zerfällungskörper des Polynoms $x^{p^n}- x$).

		\item Primzahlen $P$: Sei $\varphi$ eine Formel, für die $\mathrm{Mod}(\varphi)$ die Klasse aller Körper ist. Weiters verlangen wir, dass es höchstens einen Automorphismus (die Identität) gibt. Vergleich dazu Algebra Skriptum Proposition 6.3.3.2.

		\item zusammengesetzte Zahlen $Y$: Nach Aufgabe 71 ist $\{1\} \in \mathscr{S}$ und mit $M := \N \setminus \{0\}$ gilt  nach Aufgabe 74 auch $A:= M \setminus \{1\} \in \mathscr{S}$ und $B := M \setminus P \in \mathscr{S}$ und schließlich gilt nach Aufgabe 71 die Aussage $Y = A \cap B \in \mathscr{S}$. \\
		\textit{Geht leider auch nicht.}

		\item Zweierpotenzen $Z$: Sei $\psi$ eine Formel, die genau die Booleschen Algebren beschreibt. Dann ist das Spektrum von $\psi$ die Menge aller Zweierpotenzen (Darstellungssatz von Stone).

		\item Fünferpotenzen $F$: Sei $\psi$ eine Formel, die genau die Körper der
		Charakteristik $5$ beschreibt.

		\item Quadratzahlen $Q$: Wenn $M$ ein Modell dann gibt es eine Teilmenge $A \subseteq M$ und eine Bijektion $f: A \times A \to M$. \\
		\textit{Ich glaub nicht, dass das so geht, in der Prädikatenlogik erster
		Stufe können wir nur über Elemente, aber nicht über Teilmengen Aussagen treffen.}
	\end{enumerate}

\end{solution}
