% --------------------------------------------------------------------------------

\begin{exercise}[75]

\phantom{}
	Zeigen Sie von möglichst vielen der folgenden Mengen, dass Sie Spektren sind: Die Menge der Primzahlen, die Menge aller zusammengesetzen Zahlen (=Nichtprimzahlen $> 1$), die Menge aller Quadratzahlen, die Menge aller Zweierpotenzen, die Menge aller Potenzen von $5$, die Menge aller Primzahlpotenzen, die Menge $\{1, 14, 141, 1414, 14142, \dots \} = \Bbraces{\floorbraces{10^n \sqrt{2}} : n = 0, 1, 2, \dots}$.

\end{exercise}

% --------------------------------------------------------------------------------

\begin{solution}
\phantom{}

	\begin{enumerate}[label = \arabic*.]
		\item Primzahlpotenzen $R$: Sei $\varphi$ eine Formel, für die $\mathrm{Mod}(\varphi)$ die Klasse aller Körper ist.
		Dann ist das Spektrum von $\varphi$ genau die Menge aller Primzahlpotenzen (Zerfällungskörper des Polynoms $x^{p^n}- x$).

		\item Primzahlen $P$: Sei $\varphi$ eine Formel, für die $\mathrm{Mod}(\varphi)$ die Klasse aller Körper ist. Weiters verlangen wir, dass es höchstens einen Automorphismus (die Identität) gibt. Vergleich dazu Algebra Skriptum Proposition 6.3.3.2.

		\item zusammengesetzte Zahlen $Y$: Nach Aufgabe 71 ist $\{1\} \in \mathscr{S}$ und mit $M := \N \setminus \{0\}$ gilt  nach Aufgabe 74 auch $A:= M \setminus \{1\} \in \mathscr{S}$ und $B := M \setminus P \in \mathscr{S}$ und schließlich gilt nach Aufgabe 71 die Aussage $Y = A \cap B \in \mathscr{S}$. \\
		\textit{Geht leider auch nicht.}

		\item Zweierpotenzen $Z$: Sei $\psi$ eine Formel, die genau die Booleschen Algebren beschreibt. Dann ist das Spektrum von $\psi$ die Menge aller Zweierpotenzen (Darstellungssatz von Stone).

		\item Fünferpotenzen $F$: Sei $\psi$ eine Formel, die genau die Körper der
		Charakteristik $5$ beschreibt.

		\item Quadratzahlen $Q$: Wenn $M$ ein Modell dann gibt es eine Teilmenge $A \subseteq M$ und eine Bijektion $f: A \times A \to M$. Wir realisieren das so, dass wir zuerst eine Totalordnung auf unserer Menge verlangen, also
		\begin{align*}
		\forall x (x \leq x), \quad \forall x \forall y ((x \leq y \land y \leq x) \rightarrow x = y),\\
		\forall x \forall y \forall z((x \leq y \land y \leq z) \rightarrow x \leq z), \quad \forall x \forall y (x \leq y \lor y \leq x)
		\end{align*}
		und dann wollen wir eine Relation auf einem Quadrat, also
		\begin{align*}
		\exists a \exists b \forall x \forall y(\exists zR(x,y,z) \leftrightarrow a \leq x \leq b \land a \leq y \leq b¸)
		\end{align*}
		Nun haben wir einmal den richtigen Definitionsbereich, wir brauchen noch, dass es sich tatsächlich um eine Funktion und eine Bijektion handelt, also
		\begin{align*}
		\forall x \forall y((\exists u R(x,y,u) \land \exists v R(x,y,v)) \rightarrow u = v) \\
		\forall z ((\exists x \exists y R(x,y,z) \land \exists u \exists v R(u,v,z)) \rightarrow u = x \land y = v) \\
		\forall z \exists x \exists y R(x,y,z)
		\end{align*}
	\end{enumerate}

\end{solution}
