% --------------------------------------------------------------------------------

\begin{exercise}[75]

\phantom{}
	Zeigen Sie von möglichst vielen der folgenden Mengen, dass Sie Spektren sind: Die Menge der Primzahlen, die Menge aller zusammengesetzen Zahlen (=Nichtprimzahlen $> 1$), die Menge aller Quadratzahlen, die Menge aller Zweierpotenzen, die Menge aller Potenzen von $5$, die Menge aller Primzahlpotenzen, die Menge $\{1, 14, 141, 1414, 14142, \dots \} = \Bbraces{\floorbraces{10^n \sqrt{2}} : n = 0, 1, 2, \dots}$.

\end{exercise}

% --------------------------------------------------------------------------------

\begin{solution}

Sei $\varphi$ eine Formel, für die $\mathrm{Mod}(\varphi)$ die Klasse aller Körper ist.
Dann ist das Spektrum von $\varphi$ genau die Menge aller Primzahlpotenzen
(Zerfällungskörper des Polynoms $x^{p^n}- x$). \\
Sei $\psi$ eine Formel, die genau die Booleschen Algebren beschreibt.
Dann ist das Spektrum von $\psi$ die Menge aller Zweierpotenzen (Darstellungssatz von Stone).

\end{solution}
