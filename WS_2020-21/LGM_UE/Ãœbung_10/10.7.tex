% --------------------------------------------------------------------------------
\subsection*{72//73//74}

\begin{exercise}[72]

\phantom{}
	Zeigen Sie, dass $\mathscr{S}$ unter Durchschnitt abgeschlossen ist. (Anleitung: Sei $A = Sp(\varphi_1)$, $B = Sp(\varphi_2)$. Erklären Sie, warum Sie ohne Beschränkung der Allgemeinheit annehmen dürfen, dass die Sprachen zu $\varphi_1$ und $\varphi_2$ disjunkt sind...)

\end{exercise}

% --------------------------------------------------------------------------------

\begin{solution}

\phantom{}
	Seien $\varphi_1$ und $\varphi_2$ Formeln in den Sprachen $\mathcal{L}_1, \mathcal{L}_2$
	respektive.
	Da $\varphi_1$ und $\varphi_2$ nur endlich viele Funktionssymbole, Relationssymbole,
	Konstantensymbole und Variable verwenden können wir ohne Beschränkung der Allgemeinheit
	annehmen, dass die Sprachen zu den beiden Formeln disjunkt sind.\newline
	Insbesondere können wir somit $\varphi_1,\varphi_2$ als Formeln in der gemeinsamen Sprache
	$\mathcal{L}_1 \cup \mathcal{L}_2$ auffassen. \\
	Sei nun $n \in A \cap B$. Wähle ein Modell $\mathfrak{M}_1 = (M_1,I_1)$ mit
	$\vbraces{M_1} = n$ und $\mathfrak{M}_1 \vDash \varphi_1$ sowie ein Modell
	$\mathfrak{M}_2 = (M_2,I_2)$
	mit $\vbraces{M_2} = n$ und $\mathfrak{M}_2 \vDash \varphi_2$.
	Wir können ohne Beschränkung der Allgemeinheit sagen, dass $M_1 = M_2$ gilt,
	da es eine Bijektion $\varphi: M_1 \to M_2$ zwischen den beiden Mengen gibt.
	Wegen der Disjunktheit der Sprachen können wir nun die Interpretationen \Quote{vereinigen}:
	\begin{align*}
	\mathfrak{M} := (M_2, (\varphi\circ I_1) \cup I_2)
	\end{align*}
	wobei es keine Probleme bei den Interpretationen gibt. Wir erhalten nun klarerweise $\mathfrak{M} \vDash \varphi_1 \land \varphi_2$.  \newline
	Betrachten wir umgekehrt $n \in Sp(\varphi_1 \land \varphi_2)$ dann gibt es ein Modell $\mathfrak{M}$ mit $\vbraces{M} = n$ und $\mathfrak{M} \vDash \varphi_1 \land \varphi_2$ und damit gilt klarerweise auch $\mathfrak{M} \vDash \varphi_1$ und $\mathfrak{M} \vDash \varphi_2$.
\end{solution}

\begin{exercise}[73]

\phantom{}
	Zeigen Sie, dass $\mathscr{S}$ unter Vereinigung abgeschlossen ist.

\end{exercise}

% --------------------------------------------------------------------------------

\begin{solution}

\phantom{}
	Wir wählen zuerst ein $n \in Sp(\varphi_1) \cup Sp(\varphi_2)$. Es gelte ohne Beschränkung der Allgemeinheit $n \in Sp(\varphi_1)$. Nun wissen wir, es gibt ein Modell $\mathfrak{M}$ mit $|M| = n$ mit $\mathfrak{M} \vDash \varphi_1$. Dieses Modell adaptieren indem wir den in $\varphi_2$ vorkommenden Konstanten, Funktions- und Relationssymbolen, welche nicht schon eine Intepretation haben beliebig interpretieren. Das neue Modell nennen wir $\tilde{\mathfrak{M}}$. Klarerweise gilt
	\begin{align*}
		\tilde{\mathfrak{M}} \vDash \varphi_1 \lor \varphi_2
	\end{align*}
	also $n \in Sp(\varphi_1 \lor \varphi_2)$. \newline
	Betrachten wir umgekehrt $n \in Sp(\varphi_1 \lor \varphi_2)$. Das bedeutet es gibt ein Modell
	 $\mathfrak{M}$ von $\varphi_1 \lor \varphi_2$ mit $|M| = n$.
	 Mit der Definition der Theorie
	 $\Sigma := \{\exists^n, \neg \exists^{n + 1}, \varphi_1 \lor \varphi_2\}$
	 gilt also $\mathfrak{M} \vDash \Sigma$, also ist $\Sigma$ erfüllbar,
	 nach dem Gödelschen Vollständigkeitssatz also auch konsistent.
	 Wir können diese Theorie vervollständigen zu einer konsistenten Theorie $\bar{\Sigma}$.
	 Benützen wir nun die andere Richtung des Vollständigkeitssatzes so ist $\bar{\Sigma}$ erfüllbar,
	 es gibt also ein Modell $\bar{\mathfrak{M}}$ mit $\bar{\mathfrak{M}} \vDash \Sigma$.
	 Außerdem gilt
	 $\bar{\Sigma} \vdash \varphi_1 \lor \varphi_2$.
	 Wegen der Vollständigkeit von $\bar{\Sigma}$ folgt mit Aufgabe 113 daraus bereits $\bar{\Sigma} \vdash \varphi_1$ oder 
	 $\bar{\Sigma} \vdash \varphi_2$ und mit dem Vollständigkeitssatz sagen wir
	 ohne Beschränkung der Allgeimeinheit $\bar{\Sigma} \vDash \varphi_1$,
	 also $\bar{\mathfrak{M}} \vDash \varphi_1$ und $|\bar{M}| = n$, also $n \in Sp(\varphi_1)$.
\end{solution}

\begin{exercise}[74]

\phantom{}
	Zeigen Sie, dass $\mathscr{S}$ unter Komplement abgeschlossen ist.

\end{exercise}

% --------------------------------------------------------------------------------

\begin{solution}

\phantom{}
	Hoffentlich stimmt das...

\end{solution}
