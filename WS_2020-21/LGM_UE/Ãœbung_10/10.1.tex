% --------------------------------------------------------------------------------

\begin{exercise}[213]

Es gibt eine berechenbare partielle\footnote{\glqq Es gibt eine partielle Funktion
$f: \N \times \N \mapsto \N \dots$\grqq\ heißt:
\glqq Es gibt eine Menge $A \subseteq \N \times \N$ und eine Funktion $f: A \to \N$
(also mit Definitionsbereich $A$) \dots\grqq\ So eine Funktion darf also durchaus
auch auf ganz $\N \times \N$ definiert sein.}
Funktion $f: \N \times \N \to \N$, sodass die partielle Funktion
$g(x) := \min\{y: f(x,y) = 0\}$ nicht berechenbar ist. \\
\textit{Hinweis:} Verwenden Sie die vorige Aufgabe: Man kann eine Funktion
$f$ mit $f(x,1) = 0$ für alle $x$ finden.

\end{exercise}

% --------------------------------------------------------------------------------

\begin{solution}

Sei $A \subseteq \N$ die semi-entscheidbare Menge aus Aufgabe 212.
Laut Definition ist dann
\begin{align*}
  \widetilde{\chi}_A(x) = \begin{cases}
    1, & x \in A \\
    \text{undefiniert} & x \notin A
  \end{cases}
\end{align*}
berechenbar und
\begin{align*}
  \chi_A(x) = \begin{cases}
    1, & x \in A \\
    0 & x \notin A
  \end{cases}
\end{align*}
nicht berechenbar. Definiere also $B := (A \times \{0\}) \cup (\N \times \{1\})$
und
\begin{align*}
  f&: B \to \N, \quad (x,y) \mapsto \begin{cases}
      0, & y = 1 \\
      \widetilde{\chi}_A(x) - 1, & y = 0
  \end{cases} \\
  f &= \chi_{\{0\}}(y)(\widetilde{\chi}_A(x) - 1) \implies f \text{ berechenbar}.
\end{align*}
Dann ist
\begin{align*}
  g(x) = \min\{y \in \N: f(x,y) = 0\} =
  \begin{cases}
      0, & \widetilde{\chi}_A(x) = 1 \\
      1, & \text{sonst}
  \end{cases}
  = 1 - \chi_A(x)
\end{align*}
nicht berechenbar und $f$ berechenbar.
\end{solution}
