% --------------------------------------------------------------------------------

\begin{exercise}[216]

\phantom{}
	Für alle $(n_1, \dots, n_k) \in \N^k$  definieren wir $\abraces{n_1, \dots, n_k} := p_1^{n_1 + 1} \cdot \cdots \cdot p_k^{n_k + 1}$, wobei $(p_1, p_2, p_3, \dots) = (2,3,5,7,\dots)$ die Folge der Primzahlen ist. (Runde Klammern für Folgen, spitze Klammern für einzelne Zahlen, die Folgen codieren.)
	\newline
	\newline
	Für jede Funktion $f:\N^k \times \N \to \N$ definieren wir $\hat{f}:\N^k \times \N \to \N$ so:
	\begin{align*}
	\hat{f}(\vec{x}, y) = \abraces{f(\vec{x}, 0), \dots, f(\vec{x}, y - 1)},
	\end{align*}
	also insbesondere $\hat{f}(\vec{x}, 0) = \langle \rangle = 1$, und $\hat{f}(\vec{x}, 1) = \langle f(\vec{x}, 0) \rangle = 2^{f(\vec{x}, 0) + 1}$.
	\newline
	\newline
	Zeigen Sie:
	\begin{enumerate}[label = (\alph*)]
		\item $f$ ist primitiv rekursiv genau dann, wenn $\hat{f}$ primitiv rekursiv ist.
		\item Wenn $f$ total ist, dann ist $f$ genau dann berechenbar, wenn $\hat{f}$ berechenbar ist.
	\end{enumerate}
\end{exercise}

% --------------------------------------------------------------------------------

\begin{solution}

\phantom{}
\begin{enumerate}[label = (\alph*)]
	\item
		\begin{enumerate}
			\item[``$\Rightarrow$''] Es sei also vorausgesetzt, dass $f$ primitiv rekursiv ist. Dann gilt
				\begin{align*}
				\hat{f}(\vec{x},0) = 1, \quad \hat{f}(\vec{x}, y + 1) = \hat{f}(\vec{x},y)p_{y+1}^{f(\vec{x}, y) + 1}
				\end{align*}
				also haben wir eine Darstellung gefunden an welcher wir ekrennen, dass $\hat{f}$ primitiv rekursiv ist.
			\item[``$\Leftarrow$''] Nun sei umgekehrt vorausgesetzt, dass $\hat{f}$ primitiv rekursiv ist. Für diese Richtung wollen wir die Funktion
				\begin{align*}
				(\cdot)_\cdot : \N \times \N \to \N : (x,y) \mapsto
				\begin{cases}
				0 &, \text{falls } x = 0 \lor y = 0 \\
				\max\{k \in \N : p_y^k \mid x \} &, \text{sonst}
				\end{cases}
				\end{align*}
				betrachten. $(x)_y$ ist also der Exponent der $y$-ten Primzahl in der Primfaktorzerlegung von $x$. Es gilt
				\begin{align*}
				f(\vec{x},y) = \pbraces{\hat{f}(\vec{x}, y + 1)}_{y + 1} - 1.
				\end{align*}
				Zu zeigen bleibt, dass obige Funktion primitiv rekursiv ist.
				Dafür zeigen wir zunächst, dass die Menge aller Primzahlen primitiv rekursiv
				ist. Wir wissen bereits, dass
				\begin{align*}
					\div: (n,k) \mapsto (q,r) \quad \text{sodass } qk + r = n, 0 \leq r < k
				\end{align*}
				primitiv rekursiv ist und somit auch
				\begin{align*}
					\mathrm{div}: (n,k) \mapsto \chi_{\{0\}}(\Pi_2^2(\div(n,k))).
				\end{align*}
				Es gilt also $\mathrm{div}(n,k) = 1$ genau dann, wenn $k | n$ und $0$ sonst.
				Damit ist auch
				\begin{align*}
					g: (n,z) \mapsto \sum_{k=1}^z \div(n,k)
				\end{align*}
				primitiv rekursiv, und wir erhalten mit $\tau(x) = g(x,x)$
				\begin{align*}
					\chi_\P(x) = \chi_{=}(\tau(x), 2)
				\end{align*}
				ist primitiv rekursiv, also sind die Primzahlen als Menge primitiv rekursiv. \\
				Definiere den beschränkten Minimierungs-Operator
				\begin{itemize}
					\item für Funktionen:
						\begin{align*}
							\mu y \leq z(f(n_1,\dots,n_k,y) = 0) :=
							\begin{cases}
								\min\{y \leq z: f(n_1,\dots,n_k,y) = 0\} & \exists y \leq z: f(n_1,\dots,n_k,y) = 0 \\
								z + 1 & \text{sonst}
							\end{cases}.
						\end{align*}
					\item für Relationen:
					\begin{align*}
						\mu y \leq z(R(n_1,\dots,n_k,y)) :=
						\begin{cases}
							\min\{y \leq z: R(n_1,\dots,n_k,y)\} & \exists y \leq z: R(n_1,\dots,n_k,y) \\
							z + 1 & \text{sonst}
						\end{cases}.
					\end{align*}

				\end{itemize}

				Jetzt verwenden wir Theorem aus dem Internet: \\
				Wenn $f$ primitiv rekursiv, bzw. $R$ primitiv rekursiv als Relation ist,
				dann auch $g(n_1,\dots,n_k,z) := \mu y \leq z(f(n_1,\dots,n_k,y) = 0)$,
				bzw $g(n_1,\dots,n_k,z) := \mu y \leq z(R(n_1,\dots,n_k,y))$. \\
				Jetzt haben wir alles gesammelt, um zu zeigen, dass die Funktion
				$\varphi_{prim}(k)$, welche $k$ auf die $k$-te Primzahl abbildet,
				primitiv rekursiv ist. Die Relation
				\begin{align*}
					R(m,y):  \chi_\P(y) = 1 \land m < y
				\end{align*}
				ist als Schnitt zweier primitiv rekursiver Relationen wieder rekursiv.
				Mit der oberen Schranke $\varphi_{prim}(n+1) \leq 2^{2^n}$ erhalten wir für
				\begin{align*}
					g(m,z) := \mu y \leq z(\chi_\P(y) = 1 \land m < y)
				\end{align*}
				$g(\varphi_{prim}(n),z) = \varphi_{prim}(n+1)$ mit $z \geq 2^{2^n}$. \\
				Damit haben wir eine primitiv rekursive Darstellung für $\varphi_{prim}$:
				\begin{align*}
					\varphi_{prim}(0) &= 1 \\
					\varphi_{prim}(n + 1) &= g(\varphi_{prim}(n),2^{2^n}).
				\end{align*}
				Jetzt können wir $(n,k) \mapsto (n)_k$ primitv rekursiv beschreiben durch
				\begin{align*}
					(n)_k = \mu y \leq n(\mathrm{div}(n,\varphi_{prim}(k)^y) - 1 = 0)
				\end{align*}
		\end{enumerate}
	\item Kann man hier den Beweis aus (a) nicht direkt übernehmen?
\end{enumerate}

\end{solution}
