% --------------------------------------------------------------------------------

\begin{exercise}[71]

\phantom{}
	Zeigen Sie, dass jede endliche Teilmenge von  $\{1,2, \dots \}$ ein Spektrum ist. (Wie definieren Sie \Quote{endlich}?) Verwenden Sie vollständige Induktion? Wenn ja, geben Sie explizit die Behauptung $B(n)$ an, von der Sie $B(0)$ und $B(n) \Rightarrow B(n + 1)$ zeigen.)

\end{exercise}

% --------------------------------------------------------------------------------

\begin{solution}

\phantom{}
	Wir behaupten
	\begin{align*}
	M := \Bbraces{n \in \N \mid \forall A \subseteq \N \setminus \{0\}: \pbraces{\vbraces{A} = n \Rightarrow  A \in \mathscr{S}}} = \N
	\end{align*}
	Für $n = 0$ gilt $A = \emptyset$ und es gilt $Sp(\exists x \exists y(x = y \land x \neq y)) = A$. \newline
	Nehmen wir nun an $n \in M$ und betrachten wir eine Menge $A \subseteq \N \setminus \{0\}$ mit $\vbraces{A} = n + 1$. Nun wählen wir ein Element $k \in A$ und definieren $C := A \setminus \{k\}$, damit gilt $\vbraces{C} = n$. Nach unserer Annahmen folgt, dass es eine geschlossene Formel $\varphi$ gibt mit $C = Sp(\varphi)$. Nun unterscheiden wir zwei Fälle:
	\begin{enumerate}[label = Fall \arabic*:]
		\item $k = 1$, dann ist
			\begin{align*}
			A = Sp(\varphi \lor (\forall x \forall y: x = y))
			\end{align*}
		\item $k > 1$, dann ist
			\begin{align*}
			A = Sp(\varphi \lor (\exists^k \land \neg \exists^{k + 1})),
			\end{align*}
			wobei $\exists^k = \exists x_1 \dots \exists x_k (x_1 \neq x_2 \land x_1 \neq x_2 \land \dots \land x_{k - 1} \neq x_k)$
	\end{enumerate}
\end{solution}
