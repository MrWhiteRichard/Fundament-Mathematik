\documentclass{article}

% ---------------------------------------------------------------- %
% short package descriptions are copied from
% https://ctan.org/

% ---------------------------------------------------------------- %

% Accept different input encodings
\usepackage[utf8]{inputenc}

% Standard package for selecting font encodings
\usepackage[T1]{fontenc}

% ---------------------------------------------------------------- %

% Multilingual support for Plain TEX or LATEX
\usepackage[ngerman]{babel}

% ---------------------------------------------------------------- %

% Set all page margins to 1.5cm
\usepackage{fullpage}

% Margin adjustment and detection of odd/even pages
\usepackage{changepage}

% Flexible and complete interface to document dimensions
\usepackage{geometry}

% ---------------------------------------------------------------- %
% mathematics

\usepackage{amsmath}  % AMS mathematical facilities for LATEX
\usepackage{amssymb}
\usepackage{amsfonts} % TEX fonts from the American Mathematical Society
\usepackage{amsthm}   % Typesetting theorems (AMS style)

% Mathematical tools to use with amsmath
\usepackage{mathtools}

% Support for using RSFS fonts in maths
\usepackage{mathrsfs}

% Commands to produce dots in math that respect font size
\usepackage{mathdots}

% "Blackboard-style" cm fonts
\usepackage{bbm}

% Typeset in-line fractions in a "nice" way
\usepackage{nicefrac}

% Typeset quotient structures with LATEX
\usepackage{faktor}

% Vector arrows
\usepackage{esvect}

% St Mary Road symbols for theoretical computer science
\usepackage{stmaryrd}

% Three series of mathematical symbols
\usepackage{mathabx}

% ---------------------------------------------------------------- %
% algorithms

% Package for typesetting pseudocode
\usepackage{algpseudocode}

% Typeset source code listings using LATEX
\usepackage{listings}

% Reimplementation of and extensions to LATEX verbatim
\usepackage{verbatim}

% If necessary, please use the following 2 packages locally, but never both.
% This is because the algorithm environment gets defined in both packages, which leads to name conflicts.
% \usepackage{algorithm2e}
% \usepackage{algorithm}

% ---------------------------------------------------------------- %
% utilities

% A generic document command parser
\usepackage{xparse}

% Extended conditional commands
\usepackage{xifthen}

% e-TEX tools for LATEX
\usepackage{etoolbox}

% Define commands with suffixes
\usepackage{suffix}

% Extensive support for hypertext in LATEX
\usepackage{hyperref}

% Driver-independent color extensions for LATEX and pdfLATEX
\usepackage{xcolor}

% ---------------------------------------------------------------- %
% graphics

% -------------------------------- %

\usepackage{tikz}

% MISC
\usetikzlibrary{patterns}
\usetikzlibrary{decorations.markings}
\usetikzlibrary{positioning}
\usetikzlibrary{arrows}
\usetikzlibrary{arrows.meta}
\usetikzlibrary{overlay-beamer-styles}

% finite state machines
\usetikzlibrary{automata}

% turing machines
\usetikzlibrary{calc}
\usetikzlibrary{chains}
\usetikzlibrary{decorations.pathmorphing}

% -------------------------------- %

% Draw tree structures
\usepackage[noeepic]{qtree}

% Enhanced support for graphics
\usepackage{graphicx}

% Figures broken into subfigures
\usepackage{subfig}

% Improved interface for floating objects
\usepackage{float}

% Control float placement
\usepackage{placeins}

% Include PDF documents in LATEX
\usepackage{pdfpages}

% ---------------------------------------------------------------- %

% Control layout of itemize, enumerate, description
\usepackage[inline]{enumitem}

% Intermix single and multiple columns
\usepackage{multicol}
\setlength{\columnsep}{1cm}

% Coloured boxes, for LATEX examples and theorems, etc
\usepackage{tcolorbox}

% ---------------------------------------------------------------- %
% tables

% Tabulars with adjustable-width columns
\usepackage{tabularx}

% Tabular column heads and multilined cells
\usepackage{makecell}

% Publication quality tables in LATEX
\usepackage{booktabs}

% ---------------------------------------------------------------- %
% bibliography and quoting

% Sophisticated Bibliographies in LATEX
\usepackage[backend = biber, style = alphabetic]{biblatex}

% Context sensitive quotation facilities
\usepackage{csquotes}

% ---------------------------------------------------------------- %

% ---------------------------------------------------------------- %
% special letters

\newcommand{\N}{\mathbb N}
\newcommand{\Z}{\mathbb Z}
\newcommand{\Q}{\mathbb Q}
\newcommand{\R}{\mathbb R}
\newcommand{\C}{\mathbb C}
\newcommand{\K}{\mathbb K}
\newcommand{\T}{\mathbb T}
\newcommand{\E}{\mathbb E}
\newcommand{\V}{\mathbb V}
\renewcommand{\S}{\mathbb S}
\renewcommand{\P}{\mathbb P}
\newcommand{\1}{\mathbbm 1}
\newcommand{\G}{\mathbb G}

\newcommand{\iu}{\mathrm i}

% ---------------------------------------------------------------- %
% quantors

\newcommand{\Forall}        {\forall ~}
\newcommand{\Exists}        {\exists ~}
\newcommand{\nExists}       {\nexists ~}
\newcommand{\ExistsOnlyOne} {\exists! ~}
\newcommand{\nExistsOnlyOne}{\nexists! ~}
\newcommand{\ForAlmostAll}  {\forall^\infty ~}

% ---------------------------------------------------------------- %
% graphics boxed

\newcommand
{\includegraphicsboxed}
[2][0.75]
{
    \begin{center}
        \begin{tcolorbox}[standard jigsaw, opacityback = 0]

            \centering
            \includegraphics[width = #1 \textwidth]{#2}

        \end{tcolorbox}
    \end{center}
}

\newcommand
{\includegraphicsunboxed}
[2][0.75]
{
    \begin{center}
        \includegraphics[width = #1 \textwidth]{#2}
    \end{center}
}

\NewDocumentCommand
{\includegraphicsgraphicsboxed}
{ O{0.75} O{0.25} m m}
{
    \begin{center}
        \begin{tcolorbox}[standard jigsaw, opacityback = 0]

            \centering
            \includegraphics[width = #1 \textwidth]{#3} \\
            \vspace{#2 cm}
            \includegraphics[width = #1 \textwidth]{#4}

        \end{tcolorbox}
    \end{center}
}

\NewDocumentCommand
{\includegraphicsgraphicsunboxed}
{ O{0.75} O{0.25} m m}
{
    \begin{center}

        \centering
        \includegraphics[width = #1 \textwidth]{#3} \\
        \vspace{#2 cm}
        \includegraphics[width = #1 \textwidth]{#4}

    \end{center}
}

% ---------------------------------------------------------------- %
% braces

\newcommand{\pbraces}[1]{{\left  ( #1 \right  )}}
\newcommand{\bbraces}[1]{{\left  [ #1 \right  ]}}
\newcommand{\Bbraces}[1]{{\left \{ #1 \right \}}}
\newcommand{\vbraces}[1]{{\left  | #1 \right  |}}
\newcommand{\Vbraces}[1]{{\left \| #1 \right \|}}

\newcommand{\abraces}[1]{{\left \langle #1 \right \rangle}}

\newcommand{\floorbraces}[1]{{\left \lfloor #1 \right \rfloor}}
\newcommand{\ceilbraces} [1]{{\left \lceil  #1 \right \rceil }}

\newcommand{\dbbraces}    [1]{{\llbracket     #1 \rrbracket}}
\newcommand{\dpbraces}    [1]{{\llparenthesis #1 \rrparenthesis}}
\newcommand{\dfloorbraces}[1]{{\llfloor       #1 \rrfloor}}
\newcommand{\dceilbraces} [1]{{\llceil        #1 \rrceil}}

\newcommand{\dabraces}[1]{{\left \langle \left \langle #1 \right \rangle \right \rangle}}

\newcommand{\abs}  [1]{\vbraces{#1}}
\newcommand{\round}[1]{\bbraces{#1}}
\newcommand{\floor}[1]{\floorbraces{#1}}
\newcommand{\ceil} [1]{\ceilbraces{#1}}

% ---------------------------------------------------------------- %

% MISC

% metric spaces
\newcommand{\norm}[2][]{\Vbraces{#2}_{#1}}
\DeclareMathOperator{\metric}{d}
\DeclareMathOperator{\dist}  {dist}
\DeclareMathOperator{\diam}  {diam}

% O-notation
\newcommand{\landau}{{\scriptstyle \mathcal{O}}}
\newcommand{\Landau}{\mathcal{O}}

% ---------------------------------------------------------------- %

% math operators

% hyperbolic trigonometric function inverses
\DeclareMathOperator{\areasinh}{areasinh}
\DeclareMathOperator{\areacosh}{areacosh}
\DeclareMathOperator{\areatanh}{areatanh}

% special functions
\DeclareMathOperator{\id} {id}
\DeclareMathOperator{\sgn}{sgn}
\DeclareMathOperator{\Inv}{Inv}
\DeclareMathOperator{\erf}{erf}
\DeclareMathOperator{\pv} {pv}

% exponential function as power
\WithSuffix \newcommand \exp* [1]{\mathrm{e}^{#1}}

% operations on sets
\DeclareMathOperator{\meas}{meas}
\DeclareMathOperator{\card}{card}
\DeclareMathOperator{\Span}{span}
\DeclareMathOperator{\conv}{conv}
\DeclareMathOperator{\cof}{cof}
\DeclareMathOperator{\mean}{mean}
\DeclareMathOperator{\avg}{avg}
\DeclareMathOperator*{\argmax}{argmax}
\DeclareMathOperator*{\argsmax}{argsmax}

% number theory stuff
\DeclareMathOperator{\ggT}{ggT}
\DeclareMathOperator{\kgV}{kgV}
\DeclareMathOperator{\modulo}{mod}

% polynomial stuff
\DeclareMathOperator{\ord}{ord}
\DeclareMathOperator{\grad}{grad}

% function properties
\DeclareMathOperator{\ran}{ran}
\DeclareMathOperator{\supp}{supp}
\DeclareMathOperator{\graph}{graph}
\DeclareMathOperator{\dom}{dom}
\DeclareMathOperator{\Def}{def}
\DeclareMathOperator{\rg}{rg}

% matrix stuff
\DeclareMathOperator{\GL}{GL}
\DeclareMathOperator{\SL}{SL}
\DeclareMathOperator{\U}{U}
\DeclareMathOperator{\SU}{SU}
\DeclareMathOperator{\PSU}{PSU}
% \DeclareMathOperator{\O}{O}
% \DeclareMathOperator{\PO}{PO}
% \DeclareMathOperator{\PSO}{PSO}
\DeclareMathOperator{\diag}{diag}

% algebra stuff
\DeclareMathOperator{\At}{At}
\DeclareMathOperator{\Ob}{Ob}
\DeclareMathOperator{\Hom}{Hom}
\DeclareMathOperator{\End}{End}
\DeclareMathOperator{\Aut}{Aut}
\DeclareMathOperator{\Lin}{L}

% other function classes
\DeclareMathOperator{\Lip}{Lip}
\DeclareMathOperator{\Mod}{Mod}
\DeclareMathOperator{\Dil}{Dil}

% constants
\DeclareMathOperator{\NIL}{NIL}
\DeclareMathOperator{\eps}{eps}

% ---------------------------------------------------------------- %
% doubble & tripple powers

\newcommand
{\primeprime}
{{\prime \prime}}

\newcommand
{\primeprimeprime}
{{\prime \prime \prime}}

\newcommand
{\astast}
{{\ast \ast}}

\newcommand
{\astastast}
{{\ast \ast \ast}}

% ---------------------------------------------------------------- %
% derivatives

\NewDocumentCommand
{\derivative}
{ O{} O{} m m}
{
    \frac
    {\mathrm d^{#2} {#1}}
    {\mathrm d {#3}^{#2}}
}

\NewDocumentCommand
{\pderivative}
{ O{} O{} m m}
{
    \frac
    {\partial^{#2} {#1}}
    {\partial {#3}^{#2}}
}

\DeclareMathOperator{\Div}{div}
\DeclareMathOperator{\rot}{rot}

% ---------------------------------------------------------------- %
% integrals

\NewDocumentCommand
{\Int}
{ O{} O{} m m}
{\int_{#1}^{#2} #3 ~ \mathrm d #4}

\NewDocumentCommand
{\Iint}
{ O{} O{} m m m}
{\iint_{#1}^{#2} #3 ~ \mathrm d #4 ~ \mathrm d #5}

\NewDocumentCommand
{\Iiint}
{ O{} O{} m m m m}
{\iiint_{#1}^{#2} #3 ~ \mathrm d #4 ~ \mathrm d #5 ~ \mathrm d #6}

\NewDocumentCommand
{\Iiiint}
{ O{} O{} m m m m m}
{\iiiint_{#1}^{#2} #3 ~ \mathrm d #4 ~ \mathrm d #5 ~ \mathrm d #6 ~ \mathrm d #7}

\NewDocumentCommand
{\Idotsint}
{ O{} O{} m m m}
{\idotsint_{#1}^{#2} #3 ~ \mathrm d #4 \dots ~ \mathrm d #5}

\NewDocumentCommand
{\Oint}
{ O{} O{} m m}
{\oint_{#1}^{#2} #3 ~ \mathrm d #4}

% ---------------------------------------------------------------- %

% source:
% https://tex.stackexchange.com/questions/203257/tikz-chains-with-one-side-of-the-leftmost-node-thickbold

% #1 (optional): current state, e.g. $q_0$
% #2: cursor position, e.g. 1
% #3: number of displayed cells, e.g. 5
% #4: contents of cells, e.g. {$\triangleright$, $x_1$, \dots, $x_n$, \textvisiblespace}

\newcommand{\turingtape}[4][]
{
    \begin{tikzpicture}

        \tikzset{tape/.style={minimum size=.7cm, draw}}

        \begin{scope}[start chain=0 going right, node distance=0mm]
            \foreach \x [count=\i] in #4
            {
                \ifnum\i=#3 % if last node reset outer sep to 0pt
                    \node [on chain=0, tape, outer sep=0pt] (n\i) {\x};
                    \draw (n\i.north east) -- ++(.1,0) decorate [decoration={zigzag, segment length=.12cm, amplitude=.02cm}] {-- ($(n\i.south east)+(+.1,0)$)} -- (n\i.south east) -- cycle;
                \else
                    \node [on chain=0, tape] (n\i) {\x};
                \fi

                \ifnum\i=1 % if first node draw a thick line at the left
                    \draw [line width=.1cm] (n\i.north west) -- (n\i.south west);
                \fi
            }
 
            \node [right=.25cm of n#3] {$\cdots$};
            \node [tape, above left=.25cm and 1cm of n1] (q) {#1};
            \draw [>=latex, ->] (q) -| (n#2);

        \end{scope}

    \end{tikzpicture}
}

% ---------------------------------------------------------------- %

% ---------------------------------------------------------------- %
% amsthm-environments:

\theoremstyle{definition}

% numbered theorems
\newtheorem{theorem}             {Satz}[section]
\newtheorem{lemma}      [theorem]{Lemma}
\newtheorem{corollary}  [theorem]{Korollar}
\newtheorem{proposition}[theorem]{Proposition}
\newtheorem{remark}     [theorem]{Bemerkung}
\newtheorem{definition} [theorem]{Definition}
\newtheorem{example}    [theorem]{Beispiel}
\newtheorem{heuristics} [theorem]{Heuristik}

% unnumbered theorems
\newtheorem*{theorem*}    {Satz}
\newtheorem*{lemma*}      {Lemma}
\newtheorem*{corollary*}  {Korollar}
\newtheorem*{proposition*}{Proposition}
\newtheorem*{remark*}     {Bemerkung}
\newtheorem*{definition*} {Definition}
\newtheorem*{example*}    {Beispiel}
\newtheorem*{heuristics*} {Heuristik}

% ---------------------------------------------------------------- %
% exercise- and solution-environments:

% Please define this stuff in project ("main.tex"):
% \def \lastexercisenumber {...}

\newtheorem{exercise}{Aufgabe}
\setcounter{exercise}{\lastexercisenumber}

\newenvironment{solution}
{
  \begin{proof}[Lösung]
}{
  \end{proof}
}

% ---------------------------------------------------------------- %
% MISC translations for environment-names

\renewcommand{\proofname} {Beweis}
\renewcommand{\figurename}{Abbildung}
\renewcommand{\tablename} {Tabelle}

% ---------------------------------------------------------------- %

% ---------------------------------------------------------------- %
% https://www.overleaf.com/learn/latex/Code_listing

\definecolor{codegreen} {rgb}{0, 0.6, 0}
\definecolor{codegray}    {rgb}{0.5, 0.5, 0.5}
\definecolor{codepurple}{rgb}{0.58, 0, 0.82}
\definecolor{backcolour}{rgb}{0.95, 0.95, 0.92}

\lstdefinestyle{overleaf}
{
    backgroundcolor = \color{backcolour},
    commentstyle = \color{codegreen},
    keywordstyle = \color{magenta},
    numberstyle = \tiny\color{codegray},
    stringstyle = \color{codepurple},
    basicstyle = \ttfamily \footnotesize,
    breakatwhitespace = false,
    breaklines = true,
    captionpos = b,
    keepspaces = true,
    numbers = left,
    numbersep = 5pt,
    showspaces = false,
    showstringspaces = false,
    showtabs = false,
    tabsize = 2
}

% ---------------------------------------------------------------- %
% https://en.wikibooks.org/wiki/LaTeX/Source_Code_Listings

\lstdefinestyle{customc}
{
    belowcaptionskip = 1 \baselineskip,
    breaklines = true,
    frame = L,
    xleftmargin = \parindent,
    language = C,
    showstringspaces = false,
    basicstyle = \footnotesize \ttfamily,
    keywordstyle = \bfseries \color{green!40!black},
    commentstyle = \itshape \color{purple!40!black},
    identifierstyle = \color{blue},
    stringstyle = \color{orange},
}

\lstdefinestyle{customasm}
{
    belowcaptionskip = 1 \baselineskip,
    frame = L,
    xleftmargin = \parindent,
    language = [x86masm] Assembler,
    basicstyle = \footnotesize\ttfamily,
    commentstyle = \itshape\color{purple!40!black},
}

% ---------------------------------------------------------------- %
% https://tex.stackexchange.com/questions/235731/listings-syntax-for-literate

\definecolor{maroon}        {cmyk}{0, 0.87, 0.68, 0.32}
\definecolor{halfgray}      {gray}{0.55}
\definecolor{ipython_frame} {RGB}{207, 207, 207}
\definecolor{ipython_bg}    {RGB}{247, 247, 247}
\definecolor{ipython_red}   {RGB}{186, 33, 33}
\definecolor{ipython_green} {RGB}{0, 128, 0}
\definecolor{ipython_cyan}  {RGB}{64, 128, 128}
\definecolor{ipython_purple}{RGB}{170, 34, 255}

\lstdefinestyle{stackexchangePython}
{
    breaklines = true,
    %
    extendedchars = true,
    literate =
    {á}{{\' a}} 1 {é}{{\' e}} 1 {í}{{\' i}} 1 {ó}{{\' o}} 1 {ú}{{\' u}} 1
    {Á}{{\' A}} 1 {É}{{\' E}} 1 {Í}{{\' I}} 1 {Ó}{{\' O}} 1 {Ú}{{\' U}} 1
    {à}{{\` a}} 1 {è}{{\` e}} 1 {ì}{{\` i}} 1 {ò}{{\` o}} 1 {ù}{{\` u}} 1
    {À}{{\` A}} 1 {È}{{\' E}} 1 {Ì}{{\` I}} 1 {Ò}{{\` O}} 1 {Ù}{{\` U}} 1
    {ä}{{\" a}} 1 {ë}{{\" e}} 1 {ï}{{\" i}} 1 {ö}{{\" o}} 1 {ü}{{\" u}} 1
    {Ä}{{\" A}} 1 {Ë}{{\" E}} 1 {Ï}{{\" I}} 1 {Ö}{{\" O}} 1 {Ü}{{\" U}} 1
    {â}{{\^ a}} 1 {ê}{{\^ e}} 1 {î}{{\^ i}} 1 {ô}{{\^ o}} 1 {û}{{\^ u}} 1
    {Â}{{\^ A}} 1 {Ê}{{\^ E}} 1 {Î}{{\^ I}} 1 {Ô}{{\^ O}} 1 {Û}{{\^ U}} 1
    {œ}{{\oe}}  1 {Œ}{{\OE}}  1 {æ}{{\ae}}  1 {Æ}{{\AE}}  1 {ß}{{\ss}}  1
    {ç}{{\c c}} 1 {Ç}{{\c C}} 1 {ø}{{\o}} 1 {å}{{\r a}} 1 {Å}{{\r A}} 1
    {€}{{\EUR}} 1 {£}{{\pounds}} 1
}


% Python definition (c) 1998 Michael Weber
% Additional definitions (2013) Alexis Dimitriadis
% modified by me (should not have empty lines)

\lstdefinelanguage{iPython}{
    morekeywords = {access, and, break, class, continue, def, del, elif, else, except, exec, finally, for, from, global, if, import, in, is, lambda, not, or, pass, print, raise, return, try, while}, %
    %
    % Built-ins
    morekeywords = [2]{abs, all, any, basestring, bin, bool, bytearray, callable, chr, classmethod, cmp, compile, complex, delattr, dict, dir, divmod, enumerate, eval, execfile, file, filter, float, format, frozenset, getattr, globals, hasattr, hash, help, hex, id, input, int, isinstance, issubclass, iter, len, list, locals, long, map, max, memoryview, min, next, object, oct, open, ord, pow, property, range, raw_input, reduce, reload, repr, reversed, round, set, setattr, slice, sorted, staticmethod, str, sum, super, tuple, type, unichr, unicode, vars, xrange, zip, apply, buffer, coerce, intern}, %
    %
    sensitive = true, %
    morecomment = [l] \#, %
    morestring = [b]', %
    morestring = [b]", %
    %
    morestring = [s]{'''}{'''}, % used for documentation text (mulitiline strings)
    morestring = [s]{"""}{"""}, % added by Philipp Matthias Hahn
    %
    morestring = [s]{r'}{'},     % `raw' strings
    morestring = [s]{r"}{"},     %
    morestring = [s]{r'''}{'''}, %
    morestring = [s]{r"""}{"""}, %
    morestring = [s]{u'}{'},     % unicode strings
    morestring = [s]{u"}{"},     %
    morestring = [s]{u'''}{'''}, %
    morestring = [s]{u"""}{"""}, %
    %
    % {replace}{replacement}{lenght of replace}
    % *{-}{-}{1} will not replace in comments and so on
    literate = 
    {á}{{\' a}} 1 {é}{{\' e}} 1 {í}{{\' i}} 1 {ó}{{\' o}} 1 {ú}{{\' u}} 1
    {Á}{{\' A}} 1 {É}{{\' E}} 1 {Í}{{\' I}} 1 {Ó}{{\' O}} 1 {Ú}{{\' U}} 1
    {à}{{\` a}} 1 {è}{{\` e}} 1 {ì}{{\` i}} 1 {ò}{{\` o}} 1 {ù}{{\` u}} 1
    {À}{{\` A}} 1 {È}{{\' E}} 1 {Ì}{{\` I}} 1 {Ò}{{\` O}} 1 {Ù}{{\` U}} 1
    {ä}{{\" a}} 1 {ë}{{\" e}} 1 {ï}{{\" i}} 1 {ö}{{\" o}} 1 {ü}{{\" u}} 1
    {Ä}{{\" A}} 1 {Ë}{{\" E}} 1 {Ï}{{\" I}} 1 {Ö}{{\" O}} 1 {Ü}{{\" U}} 1
    {â}{{\^ a}} 1 {ê}{{\^ e}} 1 {î}{{\^ i}} 1 {ô}{{\^ o}} 1 {û}{{\^ u}} 1
    {Â}{{\^ A}} 1 {Ê}{{\^ E}} 1 {Î}{{\^ I}} 1 {Ô}{{\^ O}} 1 {Û}{{\^ U}} 1
    {œ}{{\oe}}  1 {Œ}{{\OE}}  1 {æ}{{\ae}}  1 {Æ}{{\AE}}  1 {ß}{{\ss}}  1
    {ç}{{\c c}} 1 {Ç}{{\c C}} 1 {ø}{{\o}} 1 {å}{{\r a}} 1 {Å}{{\r A}} 1
    {€}{{\EUR}} 1 {£}{{\pounds}} 1
    %
    {^}{{{\color{ipython_purple}\^ {}}}} 1
    { = }{{{\color{ipython_purple} = }}} 1
    %
    {+}{{{\color{ipython_purple}+}}} 1
    {*}{{{\color{ipython_purple}$^\ast$}}} 1
    {/}{{{\color{ipython_purple}/}}} 1
    %
    {+=}{{{+=}}} 1
    {-=}{{{-=}}} 1
    {*=}{{{$^\ast$ = }}} 1
    {/=}{{{/=}}} 1,
    literate = 
    *{-}{{{\color{ipython_purple} -}}} 1
     {?}{{{\color{ipython_purple} ?}}} 1,
    %
    identifierstyle = \color{black}\ttfamily,
    commentstyle = \color{ipython_cyan}\ttfamily,
    stringstyle = \color{ipython_red}\ttfamily,
    keepspaces = true,
    showspaces = false,
    showstringspaces = false,
    %
    rulecolor = \color{ipython_frame},
    frame = single,
    frameround = {t}{t}{t}{t},
    framexleftmargin = 6mm,
    numbers = left,
    numberstyle = \tiny\color{halfgray},
    %
    %
    backgroundcolor = \color{ipython_bg},
    % extendedchars = true,
    basicstyle = \scriptsize,
    keywordstyle = \color{ipython_green}\ttfamily,
}

% ---------------------------------------------------------------- %
% https://tex.stackexchange.com/questions/417884/colour-r-code-to-match-knitr-theme-using-listings-minted-or-other

\geometry{verbose, tmargin = 2.5cm, bmargin = 2.5cm, lmargin = 2.5cm, rmargin = 2.5cm}

\definecolor{backgroundCol}  {rgb}{.97, .97, .97}
\definecolor{commentstyleCol}{rgb}{0.678, 0.584, 0.686}
\definecolor{keywordstyleCol}{rgb}{0.737, 0.353, 0.396}
\definecolor{stringstyleCol} {rgb}{0.192, 0.494, 0.8}
\definecolor{NumCol}         {rgb}{0.686, 0.059, 0.569}
\definecolor{basicstyleCol}  {rgb}{0.345, 0.345, 0.345}

\lstdefinestyle{stackexchangeR}
{
    language = R,                                        % the language of the code
    basicstyle = \small \ttfamily \color{basicstyleCol}, % the size of the fonts that are used for the code
    % numbers = left,                                      % where to put the line-numbers
    numberstyle = \color{green},                         % the style that is used for the line-numbers
    stepnumber = 1,                                      % the step between two line-numbers. If it is 1, each line will be numbered
    numbersep = 5pt,                                     % how far the line-numbers are from the code
    backgroundcolor = \color{backgroundCol},             % choose the background color. You must add \usepackage{color}
    showspaces = false,                                  % show spaces adding particular underscores
    showstringspaces = false,                            % underline spaces within strings
    showtabs = false,                                    % show tabs within strings adding particular underscores
    % frame = single,                                      % adds a frame around the code
    % rulecolor = \color{white},                           % if not set, the frame-color may be changed on line-breaks within not-black text (e.g. commens (green here))
    tabsize = 2,                                         % sets default tabsize to 2 spaces
    captionpos = b,                                      % sets the caption-position to bottom
    breaklines = true,                                   % sets automatic line breaking
    breakatwhitespace = false,                           % sets if automatic breaks should only happen at whitespace
    keywordstyle = \color{keywordstyleCol},              % keyword style
    commentstyle = \color{commentstyleCol},              % comment style
    stringstyle = \color{stringstyleCol},                % string literal style
    literate = %
    *{0}{{{\color{NumCol} 0}}} 1
     {1}{{{\color{NumCol} 1}}} 1
     {2}{{{\color{NumCol} 2}}} 1
     {3}{{{\color{NumCol} 3}}} 1
     {4}{{{\color{NumCol} 4}}} 1
     {5}{{{\color{NumCol} 5}}} 1
     {6}{{{\color{NumCol} 6}}} 1
     {7}{{{\color{NumCol} 7}}} 1
     {8}{{{\color{NumCol} 8}}} 1
     {9}{{{\color{NumCol} 9}}} 1
}

% ---------------------------------------------------------------- %
% Fundament Mathematik

\lstdefinestyle{fundament}{basicstyle = \ttfamily}

% ---------------------------------------------------------------- %


\parskip 0pt
\parindent 0pt

\title
{
  Logik und Grundlagen der Mathematik \\
  \vspace{4pt}
  \normalsize
  \textit{10. Übung am 10.12.2020}
}
\author
{
  Richard Weiss
  \and
  Florian Schager
  \and
  Fabian Zehetgruber
}
\date{}

\begin{document}

\maketitle

\section*{Berechenbare Funktionen und (semi-)entscheidbare Mengen}

Die Menge der $\mu$-rekursiven Funktionen ist die kleinste Menge von (möglicherweise partiellen)
Funktionen, die alle primitiv rekursiven Funktionen enthält, unter Komposition und
primitiver Rekursion abgeschlossen ist, und außerdem Folgendes erfüllt:

\begin{adjustwidth}{1cm}{}
Wenn $f: \N^k \times \N$ total und $\mu$-rekursiv ist, \\
dann ist die partielle Funktion $\vv{x} \mapsto \min\{y: f(\vv{x},y) = 0\}$ auch
$\mu$-rekursiv.
\end{adjustwidth}

% -------------------------------------------------------------------------------- %

\begin{exercise}

\phantom{}
\textit{(Inhomogeneous boundary data)} Consider the initial value problem (IVP)
\begin{align*}
  \left\{\begin{array}{ll}
  u_{t}=u_{x x}-u+\sin (3 \pi x)+x & \text { for } x \in(0,1), \quad t>0, \\
  u(0, t)=0 & \text { for } t>0, \\
  u(1, t)=1 & \text { for } t>0, \\
  u(x, 0)=\sin (\pi x)+x & \text { for } x \in(0,1).
  \end{array}\right.
\end{align*}

\begin{itemize}
  \item[(i)] Transform the IVP into a problem with homogeneous boundary data.
  \item[(ii)] Discuss the existence of a solution for the transformed problem.
  \item[(iii)] Find a complete orthonormal system consisting of eigenfunctions of the operator $L u=-u_{x x}+u$ with suitable boundary conditions.
  \item[(iv)] Find a solution of the IVP using the orthonormal system from (iii).
\end{itemize}

\end{exercise}

% -------------------------------------------------------------------------------- %

\begin{solution}

\phantom{} \begin{itemize}
    \item[(i)] Mit der Transformation $v := u - x$ erhalten wir folgendes Randwertproblem mit homogenen Randbedingungen:
    \begin{align}\label{rwp}
        \begin{cases}
            v_t - v_{xx} + v = \sin(3\pi x) & \text{für~} x \in (0, 1),~ t > 0,\\
            v(0, t) = 0 & \text{für~} t > 0,\\
            v(1, t) = 0 & \text{für~} t > 0,\\
            v(x, 0) = \sin(\pi x) & \text{für~} x \in (0, 1).
        \end{cases}
    \end{align}

    \item[(ii)] Wir wollen Satz 6.19 (Existenz von Lösungen inhomogener Probleme) anwenden; dafür werden wir nun die Voraussetzungen überprüfen.

    Die allgemeine parabolische Differentialgleichung aus der Vorlesung hat die Form
    \begin{align*}
        u_t \underbrace{- \mathrm{div}(A \nabla u) + cu}_{=~ L(u)} = f(\cdot, t).
    \end{align*}
    Um den Satz anwenden zu können, müssen $A$, $|\nabla A|$ und $c \geq 0$ in $\Omega$ beschränkt und $A$ symmetrisch und gleichmäßig positiv definit sein.
    Im betrachteten eindimensionalen Fall entartet die Matrix $A$ zu einer Konstanten, nämlich $A \equiv 1.$ Ebenso gilt $|\nabla A| \equiv 0$ sowie $c \equiv 1 \geq 0$ und $\Omega = (0,1)$
    ist offen und beschränkt mit $\partial\Omega \in C_1$, womit der Differentialoperator $L$ die richtige Form hat.

    Wegen $v_0 = \sin(\pi x) \in L^2(\Omega)$ und $f(\cdot,t) = \sin(3\pi x) \in C^0([0, \infty); L^2(\Omega))$ ($f$ ist in unserem Beispiel konstant in $t$) sind alle Voraussetzungen erfüllt.

    Also können wir Satz 6.19 anwenden und erhalten eine Lösung $v$ in der dort postulierten Form.

    \item[(iii)] Wir wollen Funktionen $\varphi_k$ und Zahlen $\lambda_k \in \R$ mit
    \begin{align}\label{98}
        L\varphi_k = - \varphi_k^{\prime\prime} + \varphi_k = \lambda_k\varphi_k
    \end{align}
    finden, wobei $\varphi_k(0) = \varphi_k(1) = 0$ gelten soll.
    Wir formen \eqref{98} um zu $\varphi_k^{\prime\prime} = (1 - \lambda_k) \varphi_k$.
    Die allgemeine Lösung dieser Differentialgleichung ist gegeben durch
    \begin{align*}
        \varphi_k(x) = c_1 \sin(\sqrt{\lambda_k - 1}~ x) + c_2 \cos(\sqrt{\lambda_k - 1}~ x).
    \end{align*}
    Aus der Randbedingung für $x = 0$ folgt sofort $c_2 = 0.$ Aus jener bei $x = 1$ folgt $\sqrt{\lambda_k - 1} = k\pi$ und somit $\lambda_k = 1 + k^2\pi^2, k \in \N.$
    Damit hat unser Orthonormalsystem die Form
    \begin{align*}
        \varphi_k(x) = c_1 \sin(k\pi x),~ k \in \N.
    \end{align*}
    Wegen
    \begin{align*}
        \int_0^1 \sin(n\pi x) \sin(m\pi x) \mathrm{~d}x =
        \begin{cases}
            0, & n \neq m\\
            \frac{1}{2}, & n = m
        \end{cases}
    \end{align*}
    wählen wir noch $c_1 = \sqrt 2$ und erhalten unsere finale Lösung:
    \begin{align*}
        \varphi_k(x) &= \sqrt 2 \sin(k\pi x),\\
        \lambda_k &= 1 + (k\pi)^2, k \in \N.
    \end{align*}
    Die Dichtheit in $L^2(\Omega)$ folgt mit demselben Argument wie letzte Woche.
    \item[(iv)] Mithilfe von Satz 6.19 können wir nun die konkrete Lösung von \eqref{rwp} bestimmen:
    \begin{align*}
        v(t) &= \mathrm{e}^{-Lt}v_0 + \int_0^t \mathrm{e}^{-L(t-s)} f(s) \mathrm{~d} s\\
        &= \sum_{k = 1}^\infty \mathrm{e}^{-\lambda_k t}(v_0, \varphi_k) \varphi_k + \int_0^t \sum_{k = 1}^\infty \mathrm{e}^{-\lambda_k (t-s)}(f(s), \varphi_k) \varphi_k \mathrm{~d} s\\
        &= \frac{1}{\sqrt{2}}\sum_{k = 1}^\infty \mathrm{e}^{-\lambda_k t}(\varphi_1, \varphi_k) \varphi_k +
        \frac{1}{\sqrt{2}}\int_0^t \sum_{k = 1}^\infty
        \mathrm{e}^{-\lambda_k (t-s)}(\varphi_3, \varphi_k) \varphi_k \mathrm{~d} s\\
        &= \mathrm{e}^{-\lambda_1 t} \sin(\pi x) + \int_0^t \mathrm{e}^{-\lambda_3(t-s)} \sin(3\pi x) \mathrm{~d} s \\
        &= \mathrm{e}^{-\lambda_1 t} \sin(\pi x) + \frac{1}{\lambda_3}
        (\mathrm{e}^{-\lambda_3 t} - 1) \sin(3\pi x) \\
        &= \mathrm{e}^{-(1+\pi^2)t} \sin(\pi x) + \frac{1}{1+9\pi^2}
        (\mathrm{e}^{-(1+9\pi^2) t} - 1) \sin(3\pi x). \\
    \end{align*}
    Das ursprüngliche Problem wird also gelöst von
    \begin{align*}
        u = v + x = \mathrm{e}^{-(1+\pi^2)t} \sin(\pi x) + \frac{1}{1+9\pi^2}
        (\mathrm{e}^{-(1+9\pi^2) t} - 1) \sin(3\pi x) + x.
    \end{align*}
    \end{itemize}


\end{solution}

% -------------------------------------------------------------------------------- %

\pagebreak
% --------------------------------------------------------------------------------

\begin{exercise}
\textit{(Periodic Sobolev Spaces)} Let $\Omega=(0,2 \pi)$ and consider the
complete orthonormal system of $L^{2}(\Omega)$ given by
\begin{align*}
  \left\{C_{0}=\frac{1}{\sqrt{2 \pi}}, C_{n}(x)=\frac{1}{\sqrt{\pi}} \cos (n x), S_{n}(x)=\frac{1}{\sqrt{\pi}} \sin (n x) \,\Bigg|\, n \in \mathbb{N}\right\}.
\end{align*}
\begin{enumerate}[label = (\roman*)]
  \item Show that for $k \in \mathbb{N}$ the space
  \begin{align*}
      H_{per}^{k}(\Omega):=\left\{f \in H^{k}(\Omega) \mid f^{(j)}(0)=f^{(j)}(2 \pi) \text { for } j=0, \ldots, k-1\right\}
  \end{align*}
  is a well-defined Hilbert space.
  \item Show that $f \in H_{\text {per}}^{1}(\Omega)$ if and only if
  \begin{align*}
      f=\sum_{m=1}^{\infty} a_{m} S_{m}+\sum_{m=0}^{\infty} b_{m} C_{m} \quad \text { with } \quad
      \sum_{m=1}^{\infty} m^{2}\left(\left|a_{m}\right|^{2}+\left|b_{m}\right|^{2}\right)<\infty.
  \end{align*}
  In this case, $f$ can be differentiated \glqq term-wise\grqq.
  \item For $n \in \mathbb{N}$ consider the projection

  \begin{align*}
    P_{n}&: H_{p e r}^{k}(\Omega) \rightarrow H_{p e r}^{k}(\Omega) \\
    f&=\sum_{m=1}^{\infty} a_{m} S_{m}+\sum_{m=0}^{\infty} b_{m} C_{m}
    \mapsto P_{n} f=\sum_{m=1}^{n} a_{m} S_{m}+\sum_{m=0}^{n} b_{m} C_{m}.
  \end{align*}

  Show that for $f \in H_{\text {per}}^{k}(\Omega)$ it holds that
  \begin{align*}
  \left\|f-P_{n} f\right\|_{L^{2}(\Omega)} \leq \frac{1}{(n+1)^{k}}\left\|f^{(k)}\right\|_{L^{2}(\Omega)}.
  \end{align*}

\end{enumerate}
\end{exercise}

% --------------------------------------------------------------------------------

\begin{solution}

\phantom{}

\end{solution}

% --------------------------------------------------------------------------------

% --------------------------------------------------------------------------------

\begin{exercise}[216]

\phantom{}
	Für alle $(n_1, \dots, n_k) \in \N^k$  definieren wir $\abraces{n_1, \dots, n_k} := p_1^{n_1 + 1} \cdot \cdots \cdot p_k^{n_k + 1}$, wobei $(p_1, p_2, p_3, \dots) = (2,3,5,7,\dots)$ die Folge der Primzahlen ist. (Runde Klammern für Folgen, spitze Klammern für einzelne Zahlen, die Folgen codieren.)
	\newline
	\newline
	Für jede Funktion $f:\N^k \times \N \to \N$ definieren wir $\hat{f}:\N^k \times \N \to \N$ so:
	\begin{align*}
	\hat{f}(\vec{x}, y) = \abraces{f(\vec{x}, 0), \dots, f(\vec{x}, y - 1)},
	\end{align*}
	also insbesondere $\hat{f}(\vec{x}, 0) = \langle \rangle = 1$, und $\hat{f}(\vec{x}, 1) = \langle f(\vec{x}, 0) \rangle = 2^{f(\vec{x}, 0) + 1}$.
	\newline
	\newline
	Zeigen Sie:
	\begin{enumerate}[label = (\alph*)]
		\item $f$ ist primitiv rekursiv genau dann, wenn $\hat{f}$ primitiv rekursiv ist.
		\item Wenn $f$ total ist, dann ist $f$ genau dann berechenbar, wenn $\hat{f}$ berechenbar ist.
	\end{enumerate}
\end{exercise}

% --------------------------------------------------------------------------------

\begin{solution}

\phantom{}
\begin{enumerate}[label = (\alph*)]
	\item
		\begin{enumerate}
			\item[``$\Rightarrow$''] Es sei also vorausgesetzt, dass $f$ primitiv rekursiv ist. Dann gilt
				\begin{align*}
				\hat{f}(\vec{x},0) = 1, \quad \hat{f}(\vec{x}, y + 1) = \hat{f}(\vec{x},y)p_{y+1}^{f(\vec{x}, y) + 1}
				\end{align*}
				also haben wir eine Darstellung gefunden an welcher wir ekrennen, dass $\hat{f}$ primitiv rekursiv ist.
			\item[``$\Leftarrow$''] Nun sei umgekehrt vorausgesetzt, dass $\hat{f}$ primitiv rekursiv ist. Für diese Richtung wollen wir die Funktion
				\begin{align*}
				(\cdot)_\cdot : \N \times \N \to \N : (x,y) \mapsto
				\begin{cases}
				0 &, \text{falls } x = 0 \lor y = 0 \\
				\max\{k \in \N : p_y^k \mid x \} &, \text{sonst}
				\end{cases}
				\end{align*}
				betrachten. $(x)_y$ ist also der $y$-te Exponent der Primzahl in der Primfaktorzerlegung von $x$. Es gilt
				\begin{align*}
				f(\vec{x},y) = \pbraces{\hat{f}(\vec{x}, y + 1)}_{y + 1} - 1.
				\end{align*}
				Zu zeigen bleibt, dass obige Funktion primitiv rekursiv ist. Haben wir das in der VO gemacht, oder wurde das für die Übung ausgelassen?
		\end{enumerate}
	\item Kann man hier den Beweis aus (a) nicht direkt übernehmen?
\end{enumerate}

\end{solution}

\begin{exercise}
Betrachten Sie das RWP
\begin{align*}
  Ly &:= -(py^{\prime})^{\prime} + qy = f \text{ auf } (a,b), \\
  R_1y &:= \alpha_1y(a) + \alpha_2p(a)y^{\prime}(a) = \rho_1, \\
  R_2y &:= \beta_1y(b) + \beta_2p(b)y^{\prime}(b) = \rho_2,
\end{align*}
wobei $p,q$ hinreichend glatt sind, $p > 0$ auf $[a,b]$. Sei $(\alpha_1,\alpha_2) \neq (0,0)$
und $(\beta_1,\beta_2) \neq (0,0)$. Nehmen Sie an, dass das RWP für $f = 0, \rho_1,\rho_2 = 0$
nur trivial lösbar sei. Seien weiters $y_1,y_2$ zwei linear unabhängige Lösungen
von $Ly = 0$ mit $R_1y_1 = 0$ und $R_2y_2 = 0$. Dann gilt:
\begin{align*}
  \kappa(x) &:= p(x)(y_1^{\prime}y_2 - y_1y_2^{\prime}) \equiv \text{const} \neq 0 \\
  G(x,t) &= \frac{1}{\kappa}\Bigg\{\begin{matrix}
    y_1(t)y_2(x), & a \leq t \leq x \leq b \\
    y_1(x)y_2(t), & a \leq x \leq t \leq b
  \end{matrix}
\end{align*}
\end{exercise}

\begin{solution}
Berechne
\begin{align*}
  \kappa^{\prime}(x) &= p^{\prime}(x)(y_1^{\prime}y_2 - y_1y_2^{\prime})
  + p(x)(y_1^{\prime}y_2^{\prime} + y_1^{\primeprime}y_2 - y_1^{\prime}y_2^{\prime} - y_1y_2^{\primeprime})
  = p^{\prime}(x)(y_1^{\prime}y_2 - y_1y_2^{\prime})
  + p(x)(y_1^{\primeprime}y_2  - y_1y_2^{\primeprime}) \\
  &= y_2(p^{\prime}(x)y_1^{\prime} + p(x)y_1^{\primeprime}) -
  y_1(p^{\prime}(x)y_2^{\prime} + p(x)y_2^{\primeprime})
  = y_2(py_1^{\prime})^{\prime} - y_1(py_2^{\prime})^{\prime} \\
  &= y_2qy_1 - y_1qy_2 = 0.
\end{align*}
Oder auch mit der Lagrange-Identität. Sei also angenommen, dass $y_1,y_2 \in C_2([a,b])$. Dann gilt
\begin{align*}
  \kappa^{\prime}(x) = (p(x)(y_1^{\prime}y_2 - y_1y_2^{\prime}))^{\prime} = y_1Ly_2 - y_2Ly_1 = 0.
\end{align*}
Angenommen $\kappa(x) \equiv 0$, dann folgt aufgrund $p(x) > 0$
\begin{align*}
  y_1^{\prime}y_2 = y_1y_2^{\prime} \iff
  \frac{y_1^{\prime}}{y_1} = \frac{y_2^{\prime}}{y_2} \iff
  (\ln(y_1))^{\prime} = (\ln(y_2))^{\prime} \iff
  \ln(y_1) = \ln(y_2) + C \iff
  y_1 = y_2\exp(C)
\end{align*}
im Widerspruch zur linearen Unabhängigkeit von $y_1,y_2$. \\
Sei $H$ nun eine beliebige Funktion, welche die Bedingungen aus Satz 6.7.
erfüllt. Wir zeigen, dass daraus bereits $H = G$ folgt.
\begin{itemize}
  \item $LH(\cdot,t) = 0$ auf $(a,t) \cup (t,b)$: \\
  Es folgt für alle $t \in T:$
  \begin{align*}
    \forall x < t:& \exists \gamma_{1,t}, \gamma_{2,t} \in \R:
    H(x,t) = \gamma_{1,t}y_1(x) + \gamma_{2,t}y_2(x) \\
    \forall x > t:& \exists \delta_{1,t}, \delta_{2,t} \in \R:
    H(x,t) = \gamma_{1,t}y_1(x) + \delta_{2,t}y_2(x)
  \end{align*}
  und
  \begin{align*}
    H(x,t) = \begin{cases}
      \gamma_1(t)y_1(x) + \gamma_2(t)y_2(x), & x < t \\
      \delta_1(t)y_1(x) + \delta_2(t)y_2(x), & x > t.
    \end{cases}
  \end{align*}
  \item $R_1H(\cdot,t) = 0$:
  \begin{align*}
    R_1H(\cdot,t) = R_1(\gamma_1(t)y_1 + \gamma_2(t)y_2)
    = \gamma_1(t)R_1(y_1) + \gamma_2(t)R_1(y_2) = \gamma_2(t)R_1(y_2) \stackrel{!}{=} 0.
  \end{align*}
  Da das Problem für $\rho_1 = \rho_2 = 0$ nur trivial lösbar ist, folgt $R_1(y_2) \neq 0$
  und damit $\gamma_2(t) = 0$.
  \item $R_2H(\cdot,t) = 0$: \\
  Analog sieht man ein, dass $\delta_1(t) = 0$.
  \item $H(\cdot,t)$ ist stetig bei $x = t$:
  \begin{align*}
    \gamma_1(t)y_1(t) = \delta_2(t)y_2(t)
    \iff
    \gamma_1(t) = \frac{\delta_2(t)y_2(t)}{y_1(t)}
  \end{align*}
  \item $\partial_x H(t^+,t)- \partial_x H(t^-,t) = - \frac{1}{p(t)}$:
  \begin{align*}
    \partial_x H(t^+,t) - \partial_x H(t^-,t) &= \delta_2(t)y_2^{\prime}(t)- \gamma_1(t)y_1^{\prime}(t)
    = \delta_2(t)y_2^{\prime}(t)- \frac{\delta_2(t)y_2(t)}{y_1(t)}y_1^{\prime}(t) \\
    &= \delta_2(t)\left(y_2^{\prime}(t)-  \frac{y_2(t)}{y_1(t)}y_1^{\prime}(t)\right)
    \stackrel{!}{=} -\frac{1}{p(t)}.
  \end{align*}
  Wir erhalten
  \begin{align*}
    \delta_2(t) &= -\left(p(t)\left(y_2^{\prime}(t)-  \frac{y_2(t)}{y_1(t)}y_1^{\prime}(t)\right)\right)^{-1}
    = -\left(p(t)\frac{y_2^{\prime}y_1(t) - y_2(t)y_1^{\prime}(t)}{y_1(t)}\right)^{-1} \\
    &= \left(\frac{\kappa}{y_1(t)}\right)^{-1} = \frac{y_1(t)}{\kappa}
  \end{align*}
  und
  \begin{align*}
    \gamma_1(t) = \frac{y_2(t)}{\kappa}
  \end{align*}
\end{itemize}
Insgesamt gilt also
\begin{align*}
  H(x,t) = \begin{Bmatrix}
    \frac{y_2(t)y_1(x)}{\kappa}, & t \leq x \\
    \frac{y_1(t)y_2(x)}{\kappa}, & t \geq x
  \end{Bmatrix} = G(x,t).
\end{align*}
\end{solution}

% --------------------------------------------------------------------------------

\begin{exercise}[220]

\phantom{}
	Wenn $f: \N^k \to \N$ eine totale Funktion ist, die (als Relation) eine $\Sigma_1$-Menge ist, dann ist $f$ auch $\Delta_1$ (d.h., die Menge $(\N^k \times \N) \setminus f$ ist auch $\Sigma_1$).

\end{exercise}

% --------------------------------------------------------------------------------

\begin{solution}

\begin{align*}
	(x_1,\dots,x_k,y) \in (\N^k \times \N) \setminus f \iff
	\exists x^{\prime}_1\, \cdots \exists x^{\prime}_k\, \exists  y^{\prime}\,
	(x^{\prime}_1,\dots,x^{\prime}_k,y^{\prime}) \in f \land x_1 = x^{\prime}_1
	\land \dots x_k = x^{\prime}_k \land y \neq y^{\prime}.
\end{align*}

\end{solution}


\section*{Spektren}
Für jede geschlossene Formel $\varphi$ (das heißt: $\varphi$ hat keine freien Variable)
definieren wir das \textit{Spektrum} $Sp(\varphi)$ als die Menge aller natürlichen Zahlen $n$,
sodass es ein endliches Modell von $\varphi$ mit genau $n$ Elementen gibt. \newline
(Wir sagen, dass $\mathcal{M}$ ein Modell von $\varphi$ ist, wenn für alle Belegungen $
b$ die Gleichung $\hat{b}(\varphi) = 1$ gilt.) \newline
\newline
Wir schreiben $\mathscr{S}$ für die Menge aller Spektren (für beliebige prädikatenlogische Sprachen).
$\mathscr{S}$ ist eine Untermenge der Potenzmenge von $\N \setminus \{0\}$.

\begin{exercise}
Betrachten Sie die Schwingung einer einseitig eingespannten Saite, welche die
Differentialgleichung
\begin{align*}
  \frac{1}{c^2}\frac{\partial^2}{\partial t ^2}y(x,t) = \frac{\partial^2}{\partial x^2}
  y(x,t), \qquad x \in (0,1), \qquad t > 0
\end{align*}
mit den Randbedingungen
\begin{align*}
  y(0,t) = 0, \qquad \frac{\partial}{\partial x}y(1,t) = 0, \qquad t > 0,
\end{align*}
erfüllt. Die Anfangsbedingungen seien $y(\cdot,0) = y_0(\cdot)$ und
$\frac{\partial}{\partial t}y(\cdot,0) = y_1(\cdot)$. Formulieren und lösen Sie
das Sturm-Liouville Eigenwertproblem, welches durch den Ansatz der Separation
der Variablen entsteht. Geben Sie eine (formale) Lösung als Reihe an.
\end{exercise}
\begin{solution}
Machen wir den Ansatz mit Separation der Variablen. Das bedeutet wir setzen

\begin{align*}
  y(x,t)
  =
  v(x)w(t)
\end{align*}

Setzen wir nun in unsere Differentialgleichung ein erhalten wir

\begin{align*}
  \frac{1}{c^2}w^\primeprime(t)v(x)
  \stackrel{!}{=}
  v^\primeprime(x)w(t) \\
  \implies
  \frac{1}{c^2}\frac{w^\primeprime(t)}{w(t)}
  =
  \frac{v^\primeprime(x)}{v(x)}
\end{align*}

Dabei hängt die linke Seite nur von $t$ und die rechte Seite nur von $x$ ab. Deswegen
muss es also eine Konstante $\lambda$ geben, die folgendes erfüllt:

\begin{align*}
  \frac{1}{c^2}\frac{w^\primeprime(t)}{w(t)}
  =
  -\lambda
  =
  \frac{v^\primeprime(x)}{v(x)}
  \implies
  \begin{cases}
    -\lambda v - v^\primeprime = 0 & \text{auf } (0,1), \quad v(0)=0=v^\prime(1) \\
    -\lambda c^2 w - w^\primeprime = 0 & \text{auf }(0,\infty)
  \end{cases}
\end{align*}

Aus der Vorlesung kennen wir die allgemeinen Lösungen für $v$ und  $w$:

\begin{align*}
  v(x) = c_1 \sin(\sqrt{\lambda}x) + c_2 \cos(\sqrt{\lambda}x) \\
  w(t) = c_3 \sin(c\sqrt{\lambda}t) + c_4 \cos(c\sqrt{\lambda}t)
\end{align*}

Um nun die Randbedingungen zu erfüllen, setzen wir zuerst $c_2 = 0$ und erhalten
\begin{align*}
  y(0,t) = v(0)w(t) = c_1\sin(0)w(t) = 0.
\end{align*}
Sehen wir uns nun die erste Ableitung
von $v$ an (o.B.d.A $c_1 \neq 0$)
\begin{align*}
  v^\prime(1)
  =
  c_1 \sqrt{\lambda}\cos(\sqrt{\lambda})
  \stackrel{!}{=}
  0 \\
  \implies
  \lambda = 0
  \lor
  \lambda
  =
  \pi^2\left(n - \frac{1}{2}\right)^2 \quad \text{mit } n\in \Z \backslash\{0\}
\end{align*}

Um eine nicht-triviale Lösung zu erhalten setzen wir $\lambda = \pi^2\left(n - \frac{1}{2}\right)^2$ für beliebiges $n \in \N$ und haben nun mit

\begin{align*}
  v(x)w(t)
  =
  c_1\sin\left(\pi\left(n - \frac{1}{2}\right) x\right)
  \left(c_3 \sin\left(c\pi\left(n - \frac{1}{2}\right)t\right) +
  c_4 \cos\left(c\pi\left(n - \frac{1}{2}\right)t\right)\right)
\end{align*}

eine Lösung unserer Differentialgleichung die auch die Randbedingungen erfüllt.
Machen wir nun den Ansatz für die Lösung mit dem Superpositionsprinzip

\begin{align*}
  y(x,t)
  =
  \sum_{n=1}^{\infty} \sin\left(\pi\left(n - \frac{1}{2}\right) x\right)
  \left(c_{1,n}\sin\left(c\pi\left(n - \frac{1}{2}\right) t\right)+c_{2,n}\cos\left(c\pi\left(n - \frac{1}{2}\right) t\right)\right)
\end{align*}

Um nun die Koeffizienten $c_{1,n},c_{2,n}$ zu erhalten, sehen wir uns die
Anfangsbedingungen an.

\begin{align*}
  y_0(x)
  &\stackrel{!}{=} y(x,0) = \sum_{n=1}^{\infty} \sin\left(\pi\left(n - \frac{1}{2}\right) x\right)
  \left(c_{1,n}\sin(0)+c_{2,n}\cos(0)\right) \\
  &= \sum_{n=1}^\infty c_{2,n}\sin\left(\pi\left(n - \frac{1}{2}\right) x\right),       \quad x \in (0,1) \\
  y_1(x)
  &\stackrel{!}{=}
  \frac{\partial}{\partial t}y(x,0) = \sum_{n=1}^{\infty} \sin\left(\pi\left(n - \frac{1}{2}\right) x\right)
  \left(c_{1,n}c\pi\left(n - \frac{1}{2}\right)\cos(0)- c_{2,n}c\pi\left(n - \frac{1}{2}\right)\sin(0)\right) \\
  &= \sum_{n=1}^\infty c_{1,n}c\pi\left(n - \frac{1}{2}\right)\sin\left(\pi\left(n - \frac{1}{2}\right) x\right),  \quad x \in (0,1)
\end{align*}

Falls man nun $y_0$ und $y_1$ symmetrisch auf $(-1,1)$
fortsetzt und dann die Sinusreihen bildet (falls diese dann auch gegen die entsprechenden Funktionen
konvergieren) erhält man ebenso die Darstellung:

\begin{align*}
  y_0(x)
  &=
  \sum_{n=1}^\infty a_n \sin(\pi nx) \\
  y_1(x)
  &=
  \sum_{n=1}^\infty b_n \sin(n \pi x), \quad \text{mit} \\
  a_n
  &=
  \int_0^2 y_0(x)\sin(n\pi x) dx \\
  b_n
  &=
  \int_0^2 y_1(x)\sin(n\pi x) dx \\
\end{align*}
\end{solution}

% --------------------------------------------------------------------------------
\subsection*{72//73//74}

\begin{exercise}[72]

\phantom{}
	Zeigen Sie, dass $\mathscr{S}$ unter Durchschnitt abgeschlossen ist. (Anleitung: Sei $A = Sp(\varphi_1)$, $B = Sp(\varphi_2)$. Erklären Sie, warum Sie ohne Beschränkung der Allgemeinheit annehmen dürfen, dass die Sprachen zu $\varphi_1$ und $\varphi_2$ disjunkt sind...)

\end{exercise}

% --------------------------------------------------------------------------------

\begin{solution}

\phantom{}
	Seien $\varphi_1$ und $\varphi_2$ Formeln in den Sprachen $\mathcal{L}_1, \mathcal{L}_2$
	respektive.
	Da $\varphi_1$ und $\varphi_2$ nur endlich viele Funktionssymbole, Relationssymbole,
	Konstantensymbole und Variable verwenden können wir ohne Beschränkung der Allgemeinheit
	annehmen, dass die Sprachen zu den beiden Formeln disjunkt sind.\newline
	Insbesondere können wir somit $\varphi_1,\varphi_2$ als Formeln in der gemeinsamen Sprache
	$\mathcal{L}_1 \cup \mathcal{L}_2$ auffassen. \\
	Sei nun $n \in A \cap B$. Wähle ein Modell $\mathfrak{M}_1 = (M_1,I_1)$ mit
	$\vbraces{M_1} = n$ und $\mathfrak{M}_1 \vDash \varphi_1$ sowie ein Modell
	$\mathfrak{M}_2 = (M_2,I_2)$
	mit $\vbraces{M_2} = n$ und $\mathfrak{M}_2 \vDash \varphi_2$.
	Wir können ohne Beschränkung der Allgemeinheit sagen, dass $M_1 = M_2$ gilt,
	da es eine Bijektion $\varphi: M_1 \to M_2$ zwischen den beiden Mengen gibt.
	Wegen der Disjunktheit der Sprachen können wir nun die Interpretationen \enquote{vereinigen}:
	\begin{align*}
	\mathfrak{M} := (M_2, (\varphi\circ I_1) \cup I_2)
	\end{align*}
	wobei es keine Probleme bei den Interpretationen gibt. Wir erhalten nun klarerweise $\mathfrak{M} \vDash \varphi_1 \land \varphi_2$.  \newline
	Betrachten wir umgekehrt $n \in Sp(\varphi_1 \land \varphi_2)$ dann gibt es ein Modell $\mathfrak{M}$ mit $\vbraces{M} = n$ und $\mathfrak{M} \vDash \varphi_1 \land \varphi_2$ und damit gilt klarerweise auch $\mathfrak{M} \vDash \varphi_1$ und $\mathfrak{M} \vDash \varphi_2$.
\end{solution}

\begin{exercise}[73]

\phantom{}
	Zeigen Sie, dass $\mathscr{S}$ unter Vereinigung abgeschlossen ist.

\end{exercise}

% --------------------------------------------------------------------------------

\begin{solution}

\phantom{}
	Wir wählen zuerst ein $n \in Sp(\varphi_1) \cup Sp(\varphi_2)$. Es gelte ohne Beschränkung der Allgemeinheit $n \in Sp(\varphi_1)$. Nun wissen wir, es gibt ein Modell $\mathfrak{M}$ mit $|M| = n$ mit $\mathfrak{M} \vDash \varphi_1$. Dieses Modell adaptieren indem wir den in $\varphi_2$ vorkommenden Konstanten, Funktions- und Relationssymbolen, welche nicht schon eine Intepretation haben beliebig interpretieren. Das neue Modell nennen wir $\tilde{\mathfrak{M}}$. Klarerweise gilt
	\begin{align*}
		\tilde{\mathfrak{M}} \vDash \varphi_1 \lor \varphi_2
	\end{align*}
	also $n \in Sp(\varphi_1 \lor \varphi_2)$. \newline
	Betrachten wir umgekehrt $n \in Sp(\varphi_1 \lor \varphi_2)$. Das bedeutet es gibt ein Modell
	 $\mathfrak{M}$ von $\varphi_1 \lor \varphi_2$ mit $|M| = n$.
	 Mit der Definition der Theorie
	 $\Sigma := \{\exists^n, \neg \exists^{n + 1}, \varphi_1 \lor \varphi_2\}$
	 gilt also $\mathfrak{M} \vDash \Sigma$, also ist $\Sigma$ erfüllbar,
	 nach dem Gödelschen Vollständigkeitssatz also auch konsistent.
	 Wir können diese Theorie vervollständigen zu einer konsistenten Theorie $\bar{\Sigma}$.
	 Benützen wir nun die andere Richtung des Vollständigkeitssatzes so ist $\bar{\Sigma}$ erfüllbar,
	 es gibt also ein Modell $\bar{\mathfrak{M}}$ mit $\bar{\mathfrak{M}} \vDash \Sigma$.
	 Außerdem gilt
	 $\bar{\Sigma} \vdash \varphi_1 \lor \varphi_2$.
	 Wegen der Vollständigkeit von $\bar{\Sigma}$ folgt mit Aufgabe 113 daraus bereits $\bar{\Sigma} \vdash \varphi_1$ oder 
	 $\bar{\Sigma} \vdash \varphi_2$ und mit dem Vollständigkeitssatz sagen wir
	 ohne Beschränkung der Allgeimeinheit $\bar{\Sigma} \vDash \varphi_1$,
	 also $\bar{\mathfrak{M}} \vDash \varphi_1$ und $|\bar{M}| = n$, also $n \in Sp(\varphi_1)$.
\end{solution}

\begin{exercise}[74]

\phantom{}
	Zeigen Sie, dass $\mathscr{S}$ unter Komplement abgeschlossen ist.

\end{exercise}

% --------------------------------------------------------------------------------

\begin{solution}

\phantom{}
	Hoffentlich stimmt das...

\end{solution}

% --------------------------------------------------------------------------------

\begin{exercise}[75]

\phantom{}
	Zeigen Sie von möglichst vielen der folgenden Mengen, dass Sie Spektren sind: Die Menge der Primzahlen, die Menge aller zusammengesetzen Zahlen (=Nichtprimzahlen $> 1$), die Menge aller Quadratzahlen, die Menge aller Zweierpotenzen, die Menge aller Potenzen von $5$, die Menge aller Primzahlpotenzen, die Menge $\{1, 14, 141, 1414, 14142, \dots \} = \Bbraces{\floorbraces{10^n \sqrt{2}} : n = 0, 1, 2, \dots}$.

\end{exercise}

% --------------------------------------------------------------------------------

\begin{solution}
\phantom{}

	\begin{enumerate}[label = \arabic*.]
		\item Primzahlpotenzen $R$: Sei $\varphi$ eine Formel, für die $\mathrm{Mod}(\varphi)$ die Klasse aller Körper ist.
		Dann ist das Spektrum von $\varphi$ genau die Menge aller Primzahlpotenzen (Zerfällungskörper des Polynoms $x^{p^n}- x$).

		\item Primzahlen $P$: Sei $\varphi$ eine Formel, für die $\mathrm{Mod}(\varphi)$ die Klasse aller Körper ist. Weiters verlangen wir, dass es höchstens einen Automorphismus (die Identität) gibt. Vergleich dazu Algebra Skriptum Proposition 6.3.3.2.

		\item zusammengesetzte Zahlen $Y$: Nach Aufgabe 71 ist $\{1\} \in \mathscr{S}$ und mit $M := \N \setminus \{0\}$ gilt  nach Aufgabe 74 auch $A:= M \setminus \{1\} \in \mathscr{S}$ und $B := M \setminus P \in \mathscr{S}$ und schließlich gilt nach Aufgabe 71 die Aussage $Y = A \cap B \in \mathscr{S}$.

		\item Zweierpotenzen $Z$: Sei $\psi$ eine Formel, die genau die Booleschen Algebren beschreibt. Dann ist das Spektrum von $\psi$ die Menge aller Zweierpotenzen (Darstellungssatz von Stone).

		\item Fünferpotenzen $F$: Sei $\psi$ eine Formel, die genau die Körper der
		Charakteristik $5$ beschreibt.
	\end{enumerate}

\end{solution}


\end{document}
