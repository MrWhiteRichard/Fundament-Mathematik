% -------------------------------------------------------------------------------- %

\begin{exercise}[203]

Seien $\mathscr{M}_1 \leq \mathscr{M}_2$ mit folgender Eigenschaft: Für alle $k$
und endliche Mengen $E \subseteq M_1$ der Größe $k$ und für alle $b \in M_2$
gibt es einen Automorphismus $\pi: M_2 \to M_2$, der einerseits alle Elemente von
$E$ auf sich selbst abbildet, aber andererseits $\pi(b) \in M_1$ erfüllt. \\
Zeigen Sie, dass dann $\mathscr{M}_1 \preccurlyeq \mathscr{M}_2$ gilt. \\
\textit{Hinweis:} Induktion nach Aufbau von $\varphi$.
\end{exercise}

% -------------------------------------------------------------------------------- %

\begin{solution}

	Wir zeigen zuerst mit Induktion nach dem Formelaufbau
	\begin{align*}
	\forall \varphi \pbraces{\varphi \text{ Formel } \Rightarrow \forall b \forall \pi: (b \text{ Belegung mit Werten in } M_2, \pi \text{ Automorphismus } \Rightarrow \widehat{\pi \circ b}(\varphi) = \widehat{b}(\varphi))}
	\end{align*}
	Für eine beliebige Variable $x$ und eine beliebige Konstante $c$ gilt
	\begin{align*}
	\overline{\pi \circ b}(x) = \pi(b(x)) = \pi\pbraces{\overline{b}(x)}, \quad \overline{\pi \circ b}(c) = c^{\mathscr{M}_2} = \pi\pbraces{c^{\mathscr{M}_2}}= \pi\pbraces{\overline{b}(c)}
	\end{align*}
	und für Terme $t_1, \dots, t_k$ mit $\forall i \in \{1, \dots, k\}: \overline{\pi \circ b}(t_i) = \pi\pbraces{\overline{b}(t_i)}$ und ein beliebiges $k$-stelliges Funktionssymbol $f$ gilt
	\begin{align*}
	\overline{\pi \circ b}\pbraces{f(t_1, \dots, t_k)} &= f^{\mathscr{M}_2}\pbraces{\overline{\pi \circ b}(t_1), \dots , \overline{\pi \circ b}(t_k)} = f^{\mathscr{M}_2}\pbraces{\pi\pbraces{\overline{b}(t_1)}, \dots, \pi\pbraces{\overline{b}(t_k)}} \\
	&= \pi \pbraces{f^{\mathscr{M}_2}\pbraces{\overline{b}(t_1), \dots, \overline{b}(t_k)}} = \pi \pbraces{\overline{b} \pbraces{f(t_1, \dots, t_k)}}
	\end{align*}
	Also gilt für alle Terme $t$ die Gleichheit $\overline{\pi \circ b}(t) = \pi\pbraces{\overline{b}(t)}$.

	Nun schauen wir uns die Atomformeln an und betrachten dafür Terme $t_1, \dots, t_k$ und ein $k$-stelliges Relationssymbol $R$. Es gilt
	\begin{align*}
	\widehat{\pi \circ b}\pbraces{R(t_1, \dots, t_k)} = 1 \Leftrightarrow \pbraces{\overline{\pi \circ b}(t_1), \dots, \overline{\pi \circ b}(t_k)} \in R^{\mathscr{M}_2} \Leftrightarrow \pbraces{\pi\pbraces{\overline{b}(t_1)}, \dots, \pi \pbraces{\overline{b}(t_k)}} \in R^{\mathscr{M}_2} \\
	\Leftrightarrow \pbraces{\overline{b}(t_1), \dots, \overline{b}(t_k)} \in R^{\mathscr{M}_2} \Leftrightarrow \widehat{b}\pbraces{R(t_1, \dots, t_k)} = 1
	\end{align*}
	Für zwei Formeln $\varphi, \psi$ mit $\widehat{\pi \circ b}(\varphi) = \widehat{b}(\varphi)$ und $\widehat{\pi \circ b}(\psi) = \widehat{b}(\psi)$ gilt exemplarisch für alle Junktoren
	\begin{align*}
	\widehat{\pi \circ b}(\varphi \land \psi) = \widehat{\pi \circ b}(\varphi) \land_B \widehat{\pi \circ b}(\psi) = \widehat{b}(\varphi) \land_B \widehat{b}(\psi) = \widehat{b}(\varphi \land \psi)
	\end{align*}
	Zuletzt zeigen wir noch für den Allquantor
	\begin{align*}
	\widehat{\pi \circ b}(\forall \varphi) &= \inf\Bbraces{\widehat{(\pi \circ b)_{x \to m}}(\varphi) \mid m \in M_2} = \inf\Bbraces{\widehat{\pi \circ b_{x \to \pi^{-1}(m)}}(\varphi) \mid m \in M_2} \\
	&= \inf\Bbraces{\widehat{b_{x \to \pi^{-1}(m)}}(\varphi) \mid m \in M_2} = \inf\Bbraces{\widehat{b_{x \to m}}(\varphi) \mid m \in M_2} = \widehat{b}(\forall \varphi)
	\end{align*}
	Für den Existenzquantor geht es analog.

	Nun wollen wir nocheinmal eine Induktion nach dem Formelaufbau durchführen und so für alle Belegungen $b$ mit Werten in $M_1$ zeigen, dass
	\begin{align*}
	\mathscr{M}_1 \vDash \varphi[b] \iff \mathscr{M}_2 \vDash \varphi[b]
	\end{align*}
	gilt.

	Notation: Sei $b$ Belegung mit Werten in $M_1 \subseteq M_2$:
	\begin{align*}
	\overline{b}_i &:= \overline{b}_{\mathscr{M}_i}, \quad i = 1,2 \\
	\hat{b}_i &:= \hat{b}_{\mathscr{M}_i}, \quad i = 1,2. \\
	\end{align*}
	Zuerst schauen wir uns die Atomformeln an und beschäftigen uns dafür zuerst mit den Termen.
	\begin{align*}
	\overline{b}_1(c) &= c^{\mathscr{M}_1} = c^{\mathscr{M}_2} = \overline{b}_2(c), \\
	\overline{b}_1(x) &= b(x) = \overline{b}_2(x), \\
	\overline{b}_1(f(t_1, \dots, t_k)) &= f^{\mathscr{M}_1}(\overline{b}_1(t_1), \dots, \overline{b}_1(t_k))
	= f^{\mathscr{M}_1}\underbrace{(\overline{b}_2(t_1), \dots, \overline{b}_2(t_k))}_{\in M_1^k} =
	f^{\mathscr{M}_2}(\overline{b}_2(t_1), \dots, \overline{b}_2(t_k))
	= \overline{b}_2(f(t_1, \dots, t_k))
	\end{align*}
	Also wissen wir jetzt $\overline{b}_1 = \overline{b}_2$ und schreiben stets einfach $\overline{b}$. Insbesondere hat $\overline{b}$ nur Werte in $M_1$.

	Nun berechnen wir
	\begin{align*}
	\widehat{b}_1(R(t_1, \dots, t_k)) = 1 \Leftrightarrow (\overline{b}(t_1), \dots, \overline{b}(t_k)) \in R^{\mathscr{M}_1} \Leftrightarrow  (\overline{b}(t_1), \dots, \overline{b}(t_k)) \in R^{\mathscr{M}_2} \Leftrightarrow \widehat{b}_2(R(t_1, \dots, t_k)) = 1.
	\end{align*}
	Nun ist die gewünschte Aussage für Atomformeln also gezeigt. Für Verträglichkeit mit den Junktoren berechnen wir exemplarisch
	\begin{align*}
	\widehat{b}_1(\psi_1 \land \psi_2) = \widehat{b}_1(\psi_1) \land_B \widehat{b}_1 (\psi_2) = \widehat{b}_2(\psi_1) \land_B \widehat{b}_2(\psi_2) = \widehat{b}_2(\psi_1 \land \psi_2).
	\end{align*}
	Zu guter Letzt müssen wir uns um die Quantoren kümmern. Wir betrachten eine Formel $\varphi$
  für welche alle Belegungen $a$ mit Werten in $M_1$ die Gleichheit
  $\hat{a}_1(\varphi) = \hat{a}_2(\varphi)$ gilt.
  \begin{align*}
    E := \{b(y): y \in \mathrm{Var}(\varphi)\backslash\{x\} \}
  \end{align*}
  Für jedes $m \in M_2$ sei nun $\pi_m$ ein Automorphismus, der $E$ festhält und $m$ nach $M_1$ schickt.
  Es gilt
  \begin{align*}
  \widehat{b}_1(\forall x\, \varphi) &=
  \inf\Bbraces{\pbraces{\widehat{b_{x \to  m}}}_1(\varphi) \mid m \in M_1} \\
  &\geq \inf\Bbraces{\pbraces{\widehat{b_{x \to  m}}}_1(\varphi) \mid m \in M_2}
  = \widehat{b}_2(\forall x\, \varphi)
  \end{align*}
  Aufgrund $\pi_m(m) \in M_1$,
  $\widehat{\pi_m \circ (b_{x \to  m})} = \widehat{\pi_m \circ (b)}_{x \to \pi_m(m)}$
  und $b_{x \to m}(y) = \pi_m \circ b_{x \to \pi_m(m)}(y)$ für alle $y \in \mathrm{Var}(\varphi)$
  und $m \in M_2$ gilt sogar
	\begin{align*}
	\widehat{b}_1(\forall x\, \varphi) &=
  \inf\Bbraces{\pbraces{\widehat{b_{x \to  m}}}_1(\varphi) \mid m \in M_1}
  = \inf\Bbraces{\pbraces{\widehat{b_{x \to  m}}}_2(\varphi) \mid m \in M_1} \\
	&\leq \inf\Bbraces{\pbraces{\widehat{\pi_m \circ (b)}_{x \to \pi_m(m)}}_2(\varphi) \mid m \in M_2}
  = \inf\Bbraces{\pbraces{\widehat{\pi_m \circ (b_{x \to  m})}}_2(\varphi) \mid m \in M_2}\\
	&= \inf\Bbraces{\pbraces{\widehat{b_{x \to  m}}}_2(\varphi) \mid m \in M_2}
  = \widehat{b}_2(\forall x\, \varphi)
	\end{align*}
	und damit sind wir fertig!
\end{solution}
