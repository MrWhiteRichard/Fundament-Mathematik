\documentclass{article}

\input{../../../Fundament-LaTeX/packages_de.tex}
\input{../../../Fundament-LaTeX/macros_de.tex}
% ---------------------------------------------------------------- %
% amsthm-environments:

\theoremstyle{definition}

% numbered theorems
\newtheorem{theorem}             {Satz}[section]
\newtheorem{lemma}      [theorem]{Lemma}
\newtheorem{corollary}  [theorem]{Korollar}
\newtheorem{proposition}[theorem]{Proposition}
\newtheorem{remark}     [theorem]{Bemerkung}
\newtheorem{definition} [theorem]{Definition}
\newtheorem{example}    [theorem]{Beispiel}
\newtheorem{heuristics} [theorem]{Heuristik}

% unnumbered theorems
\newtheorem*{theorem*}    {Satz}
\newtheorem*{lemma*}      {Lemma}
\newtheorem*{corollary*}  {Korollar}
\newtheorem*{proposition*}{Proposition}
\newtheorem*{remark*}     {Bemerkung}
\newtheorem*{definition*} {Definition}
\newtheorem*{example*}    {Beispiel}
\newtheorem*{heuristics*} {Heuristik}

% ---------------------------------------------------------------- %
% exercise- and solution-environments:

\newtheorem{exercise}{Aufgabe}

% if the exercise counter should start at a given exercise number please include the following in the main.tex document
% \setcounter{exercise}{<last exercise number>}

\newenvironment{solution}
{
  \begin{proof}[Lösung]
}{
  \end{proof}
}

% ---------------------------------------------------------------- %
% a tcolorbox-preset designed to mimic the text boxes typically used by Prof. Stefan Hetzl

% starting template:
% https://tex.stackexchange.com/a/527829

% provide box title as optional argument
\newtcolorbox[auto counter]{hetzlbox}[1][]{%
    colback = white,
    coltitle = black,
    fonttitle = \bfseries,
    sharp corners,
    detach title,
    width = 12cm,
    #1,
    code = {\ifdefempty{\tcbtitletext}{}{\tcbset{before upper = {{\centering \tcbtitle \par} \medskip}}}},
    boxrule = 0.5pt
}

% ---------------------------------------------------------------- %
% MISC translations for environment-names

\renewcommand{\proofname} {Beweis}
\renewcommand{\figurename}{Abbildung}
\renewcommand{\tablename} {Tabelle}

% ---------------------------------------------------------------- %

\input{../../../Fundament-LaTeX/listings.tex}

\parskip 0pt
\parindent 0pt

\title
{
  Logik und Grundlagen der Mathematik \\
  \vspace{4pt}
  \normalsize
  \textit{8. Übung am 26.11.2020}
}
\author
{
  Richard Weiss
  \and
  Florian Schager
  \and
  Fabian Zehetgruber
}
\date{}

\begin{document}

\maketitle

\section*{Resolution}
Unter einer \glqq Instanz\grqq\ einer Formel $\varphi$ (die meist als Klausel gegeben ist),
verstehen wir jede Formel $\varphi[x_1/t_1,\dots,x_n/t_n]$, die man durch (beliebig viele)
sinnvolle Substitutionen erhält. Unter einer Grundinstanz verstehen wir eine Instanz
ohne freie Variablen.

\begin{exercise}
Konstruieren Sie ein Adamsverfahren mit $k = 2, s = 1$ und $r = 0$ und variablen
Schrittweiten $h_j := t_j - t_{j-1}$ der Form
\begin{align}
  y_{i+1} = y_i + \sum_{j=0}^2 \beta_{i,j}(h_{i-1},h_i,h_{i+1})f_{i-j},
  \qquad f_{i-j} := f(t_{i-j},y_{i-j}).
\end{align}
Zur Bestimmung der Konstanten $\beta_{i,j}$ können Sie ein Computeralgebrasystem
verwenden. Vergleichen Sie Ihr Verfahren mit den Werten aus dem Vorlesungsskript
für äquidistante Schrittweiten.
\end{exercise}
\begin{solution}
Lösung.
\end{solution}

% --------------------------------------------------------------------------------

\begin{exercise}

Sei $\Omega \subset \R^n$ ein beschränktes Gebiet mit $\partial\Omega \in C^2$.
Zeigen Sie, dass die \textit{biharmonische Gleichung}

\begin{align*}
  \Delta^2 u = f ~\text{in}~ \Omega,
  \quad
  u = \nabla u \cdot \nu = 0 ~\text{auf}~ \partial \Omega
\end{align*}

eine schwache Lösung besitzt, falls $f \in L^2(\Omega)$.
Hierbei ist $\nu$ der äußere Normaleneinheitsvektor auf $\partial\Omega$.
Anleitung:

\begin{enumerate}[label = (\alph*)]

  \item Zeigen Sie zunächst, dass die schwache Formulierung des obigen Randwertproblems

  \begin{align*}
    \Int[\Omega]{\Delta u \Delta v}{x} = \Int[\Omega]{f v}{x}
    ~\text{für alle}~ v \in H_0^2(\Omega)
  \end{align*}

  lautet.

  \item Zeigen Sie, dass $a(u, v) = \Int[\Omega]{\Delta u \Delta v}{x}$ eine stetige und koerzive Bilinearform ist.
  Verwenden Sie hierbei folgende Ungleichung:
  Für alle $u \in H_0^2(\Omega)$ gilt $\norm[H^2(\Omega)]{u} \leq C \norm[L^2(\Omega)]{\Delta u}$.

  \item Zeigen Sie die Existenz einer schwachen Lösung des Randwertproblems.

\end{enumerate}

\end{exercise}

% --------------------------------------------------------------------------------

\begin{solution}

\phantom{}

\begin{enumerate}[label = (\alph*)]

  \item Laut Satz 5.4 (Charakterisierng von Sobolevfunktionen) folgt aus $v \in H_0^2(\Omega)$, dass $\nabla v \in H_0^1(\Omega)$.
  
  \begin{align*}
    v \in H_0^2(\Omega)
    \implies
    \nabla v \in H_0^1(\Omega)
  \end{align*}

  Wir multiplizieren die Differentialgleichung mit $v \in H^1(\Omega)$, integrieren über $\Omega$ und integrieren zweimal partiell:

  \begin{align*}
    \Delta^2 u = f
    \implies
    (\Delta u) v = f v
  \end{align*}

  \begin{align*}
    \implies
    F(v)
    & :=
    \Int[\Omega]{f v}{x}
    =
    \Int[\Omega]{(\Delta^2 u) v}{x}
    =
    \Int[\Omega]{(\Div \nabla \Delta u) v}{x} \\
    & \stackrel
    {
      \mathrm{Gauß}
    }{=}
    -\Int[\Omega]{\nabla \Delta u \cdot \nabla v}{x}
    +
    \Int[\partial \Omega]{\underbrace{(\nabla u \cdot \nu)}_0 v}{x} \\
    & \stackrel
    {
      \mathrm{Gauß}
    }{=}
    \Int[\Omega]{\Delta u \Div \nabla v}{x}
    -
    \Int[\partial \Omega]{(\underbrace{\nabla v}_0 \cdot \nu) \Delta u}{s}
    =
    \Int[\Omega]{\Delta u \Delta v}{x}
    =:
    a(u, v)
  \end{align*}

  \item

  \begin{itemize}
    \item Stetigkeit:
    
    \begin{align*}
      \abs{\Delta v}^2
      =
      \abs
      {
        \sum_{i=1}^n
        \pderivative[2]{x_i} u
      }^2
      \stackrel
      {
        \mathrm{CSB}
      }{\leq}
      \pbraces
      {
        \sum_{i=1}^n
        1^2
      }
      \pbraces
      {
        \sum_{i=1}^n
        \abs{\pderivative[2]{x_i} u}^2
      }
      \leq
      n
      \sum_{i,j=1}^n
      \abs
      {
        \frac{\partial^2}{\partial x_i \partial x_j} u
      }^2
      =
      n \abs{\Hess u}^2
    \end{align*}
    
    \begin{align*}
      \implies
      \abs{a(u, v)}
      & =
      \abs{\Int[\Omega]{\Delta u \Delta v}{x}}
      \leq
      \Int[\Omega]{\abs{\Delta u \Delta v}}{x}
      \stackrel
      {
        \mathrm{CSB}
      }{\leq}
      \pbraces
      {
        \Int[\Omega]{\abs{\Delta u}^2}{x}
      }^{1/2}
      \pbraces
      {
        \Int[\Omega]{\abs{\Delta v}^2}{x}
      }^{1/2} \\
      & \leq
      \pbraces
      {
        \Int[\Omega]{n \abs{\Hess u}^2}{x}
      }^{1/2}
      \pbraces
      {
        \Int[\Omega]{n \abs{\Hess v}^2}{x}
      }^{1/2} \\
      & =
      n
      \pbraces
      {
        \Int[\Omega]{\abs{\Hess u}^2}{x}
      }^{1/2}
      \pbraces
      {
        \Int[\Omega]{\abs{\Hess v}^2}{x}
      }^{1/2} \\
      & =
      n \norm[L^2(\Omega)]{\Hess u} \norm[L^2(\Omega)]{\Hess v}
      \leq
      n \norm[H^1(\Omega)]{u} \norm[H^1(\Omega)]{v}
    \end{align*}

    \item Koerzivität:
    
    \begin{align*}
      a(u, u)
      =
      \Int[\Omega]{\Delta u \Delta u}{x}
      =
      \Int[\Omega]{\abs{\Delta u}^2}{x}
      =
      \norm[L^2(\Omega)]{\Delta u}^2
      \geq
      C^{-2} \norm[H^2(\Omega)]{u}^2
    \end{align*}

  \end{itemize}

  \item Laut Konstruktion, ist $H_0^2$ ist mit $(\cdot, \cdot)_{H^2(\Omega)}$ ein Hilberbraum.
  In (b) haben wir gezeigt, dass die Bilinearform $a$ stetig und koerziv ist.
  $F \in H^{-1}(\Omega)$ ist ein lineares und stetiges Funktional.

  \begin{align*}
    \abs{F(v)}
    =
    \abs{\Int[\Omega]{f v}{x}}
    \leq
    \Int[\Omega]{\abs{f v}}{x}
    =
    (\norm[L^1(\Omega)]{f v})
    \stackrel
    {
      \mathrm{CSB}
    }{\leq}
    \norm[L^2(\Omega)]{f} \norm[L^2(\Omega)]{v}
    \leq
    \norm[L^2(\Omega)]{f} \norm[H^2(\Omega)]{v}
  \end{align*}

  Laut dem Lemma von Lax-Milgram existiert nun genau ein $u \in H_0^2(\Omega)$, sodass $a(u, v) = F(v)$ für alle $v \in H_0^2(\Omega)$.

  \begin{align*}
      \implies
      \ExistsOnlyOne u \in H_0^2(\Omega):
      \Forall v \in H_0^2(\Omega):
      a(u, v) = F(v)
  \end{align*}

\end{enumerate}

\end{solution}

% --------------------------------------------------------------------------------


\section*{Primitiv rekursive Funktionen}
Wir nennen eine Relation $R \subseteq \N^n$ primitiv rekursiv (genauer: \glqq
primitiv rekursive Relation\grqq\, oder \glqq primitiv rekursiv als Relation\grqq), wenn
ihre charakteristische Funktion $\chi_R$ primitiv rekursiv ist. \\
ACHTUNG: Es gibt Funktionen, die zwar nicht primitiv rekursiv sind, die aber
in diesem Sinn eine primitiv rekursive Relation sind. \\
In jeder der folgenden Aufgaben dürfen Sie die jeweils vorigen Aufgaben verwenden.

% --------------------------------------------------------------------------------

\begin{exercise}[194 + 195]

Zeigen Sie, dass mehrere der folgenden Funktionen primitiv rekursiv sind:
(Manche dieser Funktionen sind gelegentlich undefiniert, z.B. an der Stelle $0$.
Setzen Sie die Funktion so zu einer totalen Funktion fort, dass Sie von der
Fortsetzung zeigen können, dass sie primitiv rekursiv ist.) \\

\textbf{Geben Sie explizit die primitiv rekursiven Funktionen $h$ und $g$ an,
die Sie im Schema der primitiven Rekursion
\footnote{Gemeint ist eine Definition der Form $f(\vv{x},0) = h(\vv{x}),\
f(\vv{x}, y + 1) = g(\vv{x}, f(\vv{x},y), y)$. Aus notationellen Gründen
treten oft die als trivial geltenden Projektionen $\Pi_k^n$ auf.}
verwenden.}

\begin{itemize}
	\item Addition, Multiplikation, $n!$, modulo-Funktion, $\binom{n}{k}$
	\item Die durch
	\begin{align*}
		f(x) = \begin{cases}
			g(x) & \text{wenn } x \in A \\
			h(x) & \text{sonst}
		\end{cases}
	\end{align*}
	definierte Funktion, wenn $g,h$ primitiv rekursive Funktionen sind und $A$
	eine primitiv rekursive Menge.
	\item Die charakteristische Funktion jeder endlichen Menge.
	\item $(n,k) \mapsto (q,r)$ mit $qk + r = n,\ 0 \leq r < k$.
	\item $(n,k) \mapsto \lfloor \frac{n}{k} \rfloor$. (Gaußklammer)
	\item $(n,k) \mapsto \max(0, \lfloor \frac{p(n)}{q(k)}\rfloor)$,
	wobei $p$ und $q$ beliebige Polynome mit ganzzahligen (möglicherweise
	negativen) Koeffizienten sind.
	\item Die Funktion $(n,k) \mapsto k$-te Dezimalstelle von $n$.
	\item $n \mapsto \lfloor \sqrt{2}*10^n\rfloor$.
	\item Eine geeignete (von Ihnen zu wählende) injektive (bijektive?) Abbildung
	$\N \times \N \to \N$.
	\item Die Fibonacci-Folge $f(0) = f(1) = 1,\ f(n+2) = f(n+1) + f(n)$.\\
	\textit{Hinweis:} Betrachten Sie zunächst die Funktion $n \mapsto 2^{f(n)}\cdot 3^{f(n+1)}$.
	\item Ihre Lieblingsfunktion. (Möglichst nichttrivial.)

\end{itemize}

\end{exercise}

% --------------------------------------------------------------------------------

\begin{solution}

  \phantom{}

\end{solution}

\begin{algebraUE}{291}
Seien $t(x_1,\dots,x_n)$ und $t^{\prime}(x_1,\dots,x_n)$ Terme (in einer festen
Sprache $L$), in denen jeweils nur die Variablen $x_1,\dots,x_n$ (oder Teilmengen
davon) vorkommen. Sei $\mathcal{V}$ eine Varietät (zur Sprache $L$). Für $C \in \mathcal{V}$
schreiben wir $C \vDash t \approx t^{\prime}$ (gelesen: ``Das Gesetz $t = t^{\prime}$
gilt in $C$'') als Abkürzung für
\begin{align*}
  \forall c_1,\dots,c_n \in C: t(c_1,\dots,c_n) = t^{\prime}(c_1,\dots,c_n).
\end{align*}
Sei $F \in \mathcal{V}$ frei über der $n$-elementigen Menge $\{b_1,\dots,b_n\}$ in $\mathcal{V}$.
Zeigen Sie, dass die folgenden Aussagen äquivalent sind, und schließen Sie daraus, dass 4.1.3.4. gilt:
\begin{enumerate}[label = (\alph*)]
  \item In $F$ gilt $t(b_1,\dots,b_n) = t^{\prime}(b_1,\dots,b_n)$.
  \item Für alle $C \in \mathcal{V}$ gilt $C \vDash t \approx t^{\prime}$.
  \item Es gilt $F \vDash t \approx t^{\prime}$.
\end{enumerate}
\end{algebraUE}
\begin{solution}
Beweis.
\end{solution}


\section*{Nachtrag: Elementare Untermodelle}

% --------------------------------------------------------------------------------

\begin{exercise}[\textbf{Sufficieny, bias, Rao-Blackwell theorem}]

Let $X_1,\dots,X_n$ be i.i.d $\Poi(\lambda)$, with unknown $\lambda > 0$.

\begin{enumerate}[label = (\alph*)]
    \item Show that $Y = \sum_{i=1}^n X_i$ is a sufficient statistic for $\lambda$.
    \item Find an unbiased estimator of $p_r = \P(X = r)$, which depends only
    on $X_1$.

    Find $\P(X_1 = r | Y = k)$ both for $k \geq r$ and $k < r$.

    Hence use the Rao-Blackwell theorem to improve your estimator of $p_r$.
\end{enumerate}

\end{exercise}

% --------------------------------------------------------------------------------

\begin{solution}

\phantom{}

\begin{enumerate}[label = (\alph*)]
    \item The joint distribution reads
    
    \begin{align*}
        f(\textbf{x}|\theta) &= \exp(-\lambda n)\prod_{i=1}^n{\frac{\lambda^{x_i}}{x_i!}} \\
        &= \underbrace{\exp\left(-\lambda n + \log(\lambda)\sum_{i=1}^n x_i\right)}_{:= g(T(x)|\theta)}
        \underbrace{\prod_{i=1}^n{\frac{1}{x_i!}}}_{:=h(x)}
    \end{align*}

    which implies that $Y = \sum_{i=1}^n X_i$ is indeed a sufficient statistic for $\lambda$.

    \item We simply use
    \begin{align*}
        \hat{p_r} := \begin{cases}
            1, & X_1 = r \\
            0, & \text{otherwise}
        \end{cases},
    \end{align*}

    which obviously fulfils $\E[\hat{p_r}(X)] = \P(X = r)$.

    Next we calculate for $k \geq r$

    \begin{align*}
        \P(X_1 = r | Y = k) 
        &= \frac{\P(X_1 = r \cap Y = k)}{\P(Y = k)} \\
        &= \frac{\P(X_1 = r)\P(\sum_{i=2}^n X_i = k - r)}{\P(\sum_{i=1}^n X_i = k)} \\
        &= \frac{\exp(-\lambda)\frac{\lambda^r}{r!}\exp(-(n-1)\lambda)\frac{(n-1)^{k-r}\lambda^{k-r}}{(k-r)!}}
        {\exp(-n\lambda)\frac{n^k\lambda^k}{k!}} \\
        &= \frac{k!}{(k-r)!r!}\frac{(n-1)^{k-r}}{n^k} = \binom{k}{r}\frac{(n-1)^{k-r}}{n^k}.
    \end{align*}

    For $k < r$ the probability is clearly $0$.
    

    Now, according to the Rao-Blackwell theorem we define our new estimator
    as

    \begin{align*}
        \Phi(Y) &:= \E[\hat{p_r}(X_1)|Y] 
        = \E\left[\1_{\{r\}}(X_1)| Y\right]
        = \P(X_1 = r | Y) \\
        &= \begin{cases}
            \binom{Y}{r}\frac{(n-1)^{Y-r}}{n^Y}, & Y \geq r \\
            0, & Y < r
        \end{cases}.
    \end{align*}

    with $\E_\lambda[\Phi(Y)] = p_r$ and $\V_\lambda(\Phi(Y)) \leq \V_\lambda(\hat{p_r}(X_1))$.
    
\end{enumerate}

\end{solution}

% --------------------------------------------------------------------------------

% --------------------------------------------------------------------------------

\begin{exercise}[204]

Wir betrachten eine Sprache $\mathscr{L}$, die keine Konstanten und keine
Funktionssymbole enthält, und als einziges Relationssymbol die Gleichheit.
Seien $M_1 \subseteq M_2$ unendliche Mengen; wir fassen sie als $\mathscr{L}$-Strukturen
$\mathscr{M}_1$, bzw. $\mathscr{M}_2$ auf. Verwenden Sie die vorige Aufgabe,
um $\mathscr{M}_1 \preccurlyeq \mathscr{M}_2$ zu zeigen.

\end{exercise}

% --------------------------------------------------------------------------------

\begin{solution}

Wir müssen hier nur noch für alle endliche Mengen $E \subseteq M_1$ und alle $b \in M_2$
einen Automorphismus $\pi$ finden, der $\pi|_E = \id_E$ und $\pi(b) \in M_1$ erfüllt. \\
Seien dazu $E \subseteq M_1, b \in M_2$ beliebig.
\begin{itemize}
  \item Fall 1: $b \in M_1$: \\
  Wir können $\pi = \id_{M_2}$ wählen.
  \item Fall 2: $b \notin M_1$: \\
  Da $E \subset M_1$ eine endliche Menge ist, finden wir ein $m_0 \in M\backslash E$ und
  wir definieren
  \begin{align*}
    \pi(m) = \begin{cases}
      m_0, & m = b \\
      b, & m = m_0 \\
      m, & \text{sonst}
    \end{cases}
  \end{align*}
  Klarerweise gilt $\pi|_E = id_E$ und $\pi$ ist auch bijektiv.
  Mit Aufgabe 203 erhalten wir damit bereits schon $\mathscr{M}_1 \preccurlyeq \mathscr{M}_2$.
\end{itemize}
\end{solution}



\end{document}
