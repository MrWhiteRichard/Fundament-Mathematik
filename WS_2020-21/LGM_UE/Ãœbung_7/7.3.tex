% -------------------------------------------------------------------------------- %

\begin{exercise}[144]
Geben Sie ein explizites Beispiel einer konsistenten Skolemtheorie an.
\end{exercise}

% -------------------------------------------------------------------------------- %

\begin{solution}
	Wir betrachten die Theorie
	\begin{align*}
	\Sigma := \{\forall x \forall y (x = y)\}.
	\end{align*}
	$\Sigma$ wird von jedem ein-elementigen Modell erfüllt und ist somit konsistent.
	Wählen wir nun eine beliebige Formel der Form $\exists y \psi$
	mit den freien Variablen $x_1, \dots, x_n$, wobei $n \in \N$.
	Aus der Vorlesung wissen wir bereits
	\begin{align*}
	\Sigma \vdash \exists y \psi \rightarrow \psi[y/f(\overline{x})] \Leftrightarrow \Sigma \vDash \exists y \psi \rightarrow \psi[y/f(\overline{x})]
	\end{align*}
	also reicht es wenn wir zeigen, dass der rechte Ausdruck wahr ist. Nehmen wir also ein Modell $\mathscr{M}$ in dem $\Sigma$ gilt. Die Theorie sagt uns, dass unser Modell genau ein Element $m$ hat. Nun berechnen wir für eine Belegung $b$ mit $\widehat{b}(\exists \psi) = 1$ den Wert
	\begin{align*}
	\widehat b(\psi[y/f(\overline{x})]) = \widehat{b_{y \to \overline{b}(f(\overline{x}))}}(\psi) = \widehat{b_{y \to m}}(\psi) = \sup\{\widehat{b_{x \to k}}(\psi) \mid k \in M\} = \widehat{b}(\exists \psi) = 1
	\end{align*}
	wobei unsere Sprache ein $n$-stelliges Funktionssymbol $f$ enthalten soll.
	Damit gilt für jede beliebige Belegung $b$, jede Formel $\psi$ mit freien Variablen
	$\overline{x}$ und jedes Modell $\mathscr{M}$ mit $\mathscr{M} \vDash \Sigma$
	\begin{align*}
		\hat{b}(\exists y \psi \rightarrow \psi[y/f(\overline{x})]) = 1
		\implies \mathscr{M} \vDash \exists y \psi \rightarrow \psi[y/f(\overline{x})]
		\implies \Sigma \vDash \exists y \psi \rightarrow \psi[y/f(\overline{x})].
	\end{align*}
	Also ist unsere Theorie eine Skolemtheorie.
\end{solution}

% -------------------------------------------------------------------------------- %
