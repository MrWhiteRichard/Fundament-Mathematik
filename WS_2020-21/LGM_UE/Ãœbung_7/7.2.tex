% --------------------------------------------------------------------------------

\begin{exercise}[143]
Finden Sie ein Beispiel $\mathscr{L},\mathscr{M}_1,\mathscr{M}_2$, sodass
$\mathscr{M}_1 \preccurlyeq \mathscr{M}_2$ und $M_1 \neq M_2$.
\end{exercise}

\begin{solution}
	Da wir das Gleichheitssymbol ohnehin in unserer Sprache haben wollen wir uns nicht weiter belasten und nicht mehr in unsere Sprache aufnehmen, also keine Konstantensymbole, Funktionssymbole oder weitere Relationssymbole. Wäre $M_1$ endlich mit $n \in \N$ Elementen, dann gelten wegen $ |M_2| > |M_1| = n$ die Aussagen
	\begin{align*}
	\mathscr{M}_1 \nvDash \exists^{\geq n + 1} \ [b], \quad \mathscr{M}_2 \vDash \exists^{\geq n + 1} \ [b]
	\end{align*}
	 Also müssen $M_1$ und $M_2$ mindestens abzählbar undendlich viele Elemente haben. Wir wählen $\mathscr{M}_1 = \N \setminus \{0\}$ sowie $\mathscr{M}_2 = \N$. Angenommen es gibt eine Formel $\varphi$ und eine Belegung $b$ mit Werten in $M_1$ mit
	 \begin{align*}
	 (\mathscr{M}_1 \vDash \varphi \ [b] \land \mathscr{M}_2 \nvDash \varphi \ [b]) \lor (\mathscr{M}_1 \nvDash \varphi \ [b] \land \mathscr{M}_2 \vDash \varphi \ [b])
	 \end{align*}
	 Wir betrachten den Linken Teil als ersten Fall. Nach Satz IV.5.11 gibt es eine Formel $\varphi^\prime$ in Pränexform mit $\vDash \varphi \leftrightarrow \varphi^\prime$. Also gilt $\mathscr{M}_1 \vDash \varphi^\prime \ [b]$ und $\mathscr{M}_2 \nvDash \varphi^\prime \ [b]$. Die Formel $\varphi^\prime$ muss mindestens einen Allquantor enthalten, weil für Existenzquantoren gilt 
	 \begin{align*}
	 \sup\{\widehat{b_{x \to m}} \mid m \in M_1 \} \leq \sup\{\widehat{b_{x \to m}} \mid m \in M_2 \}
	 \end{align*}
	 Also muss es irgendwo eine Teilforlme $\psi$ von $\varphi^\prime$ geben mit
	 \begin{align*}
	 \inf\{\widehat{b_{x \to m}}(\psi) \mid m \in M_2\} = 0, \quad \inf\{\widehat{b_{x \to m}}(\psi) \mid m \in M_1\} = 1
	 \end{align*}
	 also $\widehat{b_{x \to 0}}(\psi) = 0$. Die Formel $\psi$ muss die Variable $x$ enthalten, also einen Teil der Form $x = y$. Wegen $\widehat{b_{x \to m}}(\psi) = 1$ für alle $m \in M_1$ muss die Fomel in Wirklichkeit schon die Form $x = x$ haben und damit gilt aber $\widehat{b_{x \to 0}}(\psi) = 1$
\end{solution}
