% -------------------------------------------------------------------------------- %

\begin{exercise}[143]
Finden Sie ein Beispiel $\mathscr{L},\mathscr{M}_1,\mathscr{M}_2$, sodass
$\mathscr{M}_1 \preccurlyeq \mathscr{M}_2$ und $M_1 \neq M_2$.
\end{exercise}

\begin{solution}
	Da wir das Gleichheitssymbol ohnehin in unserer Sprache haben wollen wir uns nicht weiter belasten und nicht mehr in unsere Sprache aufnehmen, also keine Konstantensymbole, Funktionssymbole oder weitere Relationssymbole. Wäre $M_1$ endlich mit $n \in \N$ Elementen, dann gelten wegen $ |M_2| > |M_1| = n$ die Aussagen
	\begin{align*}
	\mathscr{M}_1 \nvDash \exists^{\geq n + 1} \ [b], \quad \mathscr{M}_2 \vDash \exists^{\geq n + 1} \ [b]
	\end{align*}
	 Also müssen $M_1$ und $M_2$ mindestens abzählbar undendlich viele Elemente haben. Wir wählen $\mathscr{M}_1 = \N \setminus \{0\}$ sowie $\mathscr{M}_2 = \N$. Wir zeigen zuerst mittels Induktion nach Formelaufbau
	 \begin{align*}
	 \forall \varphi \forall x \forall b \exists k \in \N \setminus  \{0\} \forall l \geq k (x \text{ Variable }, \varphi \text{ Formel }, b \text{ Belegung mit Werten in } M_1 \Rightarrow \widehat{b_{x \to 0}}(\varphi) = \widehat{b_{x \to l}}(\varphi))
	 \end{align*}
	 Für eine Atomformel der Form $x = x$ ist die Aussage Trivial, für eine der Form $x = y$ wählen wir $k > b(y)$. Exemplarisch rechnen wir
	 \begin{align*}
	 \widehat{b_{x \to 0}}(\varphi_1 \land \varphi_2) = \widehat{b_{x \to 0}}(\varphi_1) \land_B \widehat{b_{x \to 0}}(\varphi_2) = \widehat{b_{x \to k_1}}(\varphi_1) \land_B  \widehat{b_{x \to k_2}}(\varphi_2) = \widehat{b_{x \to \max\{k_1, k_2\}}}(\varphi_1 \land \varphi_2).
	 \end{align*}
	 Weiters gilt 
	 \begin{align*}
	 \widehat{b_{x \to 0}}(\forall \varphi) = \inf\{\widehat{(b_{x \to 0})_{y \to m}}(\varphi) \mid m \in M_1\} =  \inf\{\widehat{(b_{y \to m})_{x \to 0}}(\varphi) \mid m \in M_1\} = \inf\{\widehat{(b_{y \to m})_{x \to k_m}}(\varphi) \mid m \in M_1\}
	 \end{align*}
	 Hier weiß ich nicht mehr weiter...
\end{solution}
