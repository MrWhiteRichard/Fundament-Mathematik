% --------------------------------------------------------------------------------

\begin{exercise}[143]
Finden Sie ein Beispiel $\mathscr{L},\mathscr{M}_1,\mathscr{M}_2$, sodass
$\mathscr{M}_1 \preccurlyeq \mathscr{M}_2$ und $M_1 \neq M_2$.
\end{exercise}

\begin{solution}
	Da wir das Gleichheitssymbol ohnehin in unserer Sprache haben wollen wir uns nicht weiter belasten und nicht mehr in unsere Sprache aufnehmen, also keine Konstantensymbole, Funktionssymbole oder weitere Relationssymbole. Wäre $M_1$ endlich mit $n \in \N$ Elementen, dann gelten wegen $ |M_2| > |M_1| = n$ die Aussagen
	\begin{align*}
	\mathscr{M}_1 \nvDash \exists^{\geq n + 1} \ [b], \quad \mathscr{M}_2 \vDash \exists^{\geq n + 1} \ [b]
	\end{align*}
	 Also müssen $M_1$ und $M_2$ mindestens abzählbar undendlich viele Elemente haben. Wir wählen $\mathscr{M}_1 = \N \setminus \{0\}$ sowie $\mathscr{M}_2 = \N$.
\end{solution}
