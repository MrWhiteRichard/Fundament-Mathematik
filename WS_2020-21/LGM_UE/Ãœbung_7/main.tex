\documentclass{article}

% Hier befinden sich Pakete, die wir beinahe immer benutzen ...

\usepackage[utf8]{inputenc}

% Sprach-Paket:
\usepackage[ngerman]{babel}

% damit's nicht so, wie beim Grill aussieht:
\usepackage{fullpage}

% Mathematik:
\usepackage{amsmath, amssymb, amsfonts, amsthm}
\usepackage{bbm}
\usepackage{mathtools, mathdots}

% Makros mit mehereren Default-Argumenten:
\usepackage{twoopt}

% Anführungszeichen (Makro \Quote{}):
\usepackage{babel}

% if's für Makros:
\usepackage{xifthen}
\usepackage{etoolbox}

% tikz ist kein Zeichenprogramm (doch!):
\usepackage{tikz}

% bessere Aufzählungen:
\usepackage{enumitem}

% (bessere) Umgebung für Bilder:
\usepackage{graphicx, subfig, float}

% Umgebung für Code:
\usepackage{listings}

% Farben:
\usepackage{xcolor}

% Umgebung für "plain text":
\usepackage{verbatim}

% Umgebung für mehrerer Spalten:
\usepackage{multicol}

% "nette" Brüche
\usepackage{nicefrac}

% Spaltentypen verschiedener Dicke
\usepackage{tabularx}
\usepackage{makecell}

% Für Vektoren
\usepackage{esvect}

% (Web-)Links
\usepackage{hyperref}

% Zitieren & Literatur-Verzeichnis
\usepackage[style = authoryear]{biblatex}
\usepackage{csquotes}

% so ähnlich wie mathbb
%\usepackage{mathds}

% Keine Ahnung, was das macht ...
\usepackage{booktabs}
\usepackage{ngerman}
\usepackage{placeins}

% special letters:

\newcommand{\N}{\mathbb{N}}
\newcommand{\Z}{\mathbb{Z}}
\newcommand{\Q}{\mathbb{Q}}
\newcommand{\R}{\mathbb{R}}
\newcommand{\C}{\mathbb{C}}
\newcommand{\K}{\mathbb{K}}
\newcommand{\T}{\mathbb{T}}
\newcommand{\E}{\mathbb{E}}
\newcommand{\V}{\mathbb{V}}
\renewcommand{\S}{\mathbb{S}}
\renewcommand{\P}{\mathbb{P}}
\newcommand{\1}{\mathbbm{1}}

% quantors:

\newcommand{\Forall}{\forall \,}
\newcommand{\Exists}{\exists \,}
\newcommand{\ExistsOnlyOne}{\exists! \,}
\newcommand{\nExists}{\nexists \,}
\newcommand{\ForAlmostAll}{\forall^\infty \,}

% MISC symbols:

\newcommand{\landau}{{\scriptstyle \mathcal{O}}}
\newcommand{\Landau}{\mathcal{O}}


\newcommand{\eps}{\mathrm{eps}}

% graphics in a box:

\newcommandtwoopt
{\includegraphicsboxed}[3][][]
{
  \begin{figure}[!h]
    \begin{boxedin}
      \ifthenelse{\isempty{#1}}
      {
        \begin{center}
          \includegraphics[width = 0.75 \textwidth]{#3}
          \label{fig:#2}
        \end{center}
      }{
        \begin{center}
          \includegraphics[width = 0.75 \textwidth]{#3}
          \caption{#1}
          \label{fig:#2}
        \end{center}
      }
    \end{boxedin}
  \end{figure}
}

% braces:

\newcommand{\pbraces}[1]{{\left  ( #1 \right  )}}
\newcommand{\bbraces}[1]{{\left  [ #1 \right  ]}}
\newcommand{\Bbraces}[1]{{\left \{ #1 \right \}}}
\newcommand{\vbraces}[1]{{\left  | #1 \right  |}}
\newcommand{\Vbraces}[1]{{\left \| #1 \right \|}}
\newcommand{\abraces}[1]{{\left \langle #1 \right \rangle}}
\newcommand{\round}[1]{\bbraces{#1}}

\newcommand
{\floorbraces}[1]
{{\left \lfloor #1 \right \rfloor}}

\newcommand
{\ceilbraces} [1]
{{\left \lceil  #1 \right \rceil }}

% special functions:

\newcommand{\norm}  [2][]{\Vbraces{#2}_{#1}}
\newcommand{\diam}  [2][]{\mathrm{diam}_{#1} \: #2}
\newcommand{\diag}  [1]{\mathrm{diag} \: #1}
\newcommand{\dist}  [1]{\mathrm{dist} \: #1}
\newcommand{\mean}  [1]{\mathrm{mean} \: #1}
\newcommand{\erf}   [1]{\mathrm{erf} \: #1}
\newcommand{\id}    [1]{\mathrm{id} \: #1}
\newcommand{\sgn}   [1]{\mathrm{sgn} \: #1}
\newcommand{\supp}  [1]{\mathrm{supp} \: #1}
\newcommand{\arsinh}[1]{\mathrm{arsinh} \: #1}
\newcommand{\arcosh}[1]{\mathrm{arcosh} \: #1}
\newcommand{\artanh}[1]{\mathrm{artanh} \: #1}
\newcommand{\card}  [1]{\mathrm{card} \: #1}
\newcommand{\Span}  [1]{\mathrm{span} \: #1}
\newcommand{\Aut}   [1]{\mathrm{Aut} \: #1}
\newcommand{\End}   [1]{\mathrm{End} \: #1}
\newcommand{\ggT}   [1]{\mathrm{ggT} \: #1}
\newcommand{\kgV}   [1]{\mathrm{kgV} \: #1}
\newcommand{\ord}   [1]{\mathrm{ord} \: #1}
\newcommand{\grad}  [1]{\mathrm{grad} \: #1}
\newcommand{\ran}   [1]{\mathrm{ran} \: #1}
\newcommand{\graph} [1]{\mathrm{graph} \: #1}
\newcommand{\Inv}   [1]{\mathrm{Inv} \: #1}
\newcommand{\pv}    [1]{\mathrm{pv} \: #1}
\newcommand{\GL}    [1]{\mathrm{GL} \: #1}
\newcommand{\Mod}{\mathrm{Mod} \:}
\newcommand{\Th}{\mathrm{Th} \:}
\newcommand{\Char}{\mathrm{char}}
\newcommand{\At}{\mathrm{At}}
\newcommand{\Ob}{\mathrm{Ob}}
\newcommand{\Hom}{\mathrm{Hom}}
\newcommand{\orthogonal}[3][]{#2 ~\bot_{#1}~ #3}
\newcommand{\Rang}{\mathrm{Rang}}
\newcommand{\NIL}{\mathrm{NIL}}
\newcommand{\Res}{\mathrm{Res}}
\newcommand{\lxor}{\dot \lor}
\newcommand{\Div}{\mathrm{div} \:}
\newcommand{\meas}{\mathrm{meas} \:}

% fractions:

\newcommand{\Frac}[2]{\frac{1}{#1} \pbraces{#2}}
\newcommand{\nfrac}[2]{\nicefrac{#1}{#2}}

% derivatives & integrals:

\newcommandtwoopt
{\Int}[4][][]
{\int_{#1}^{#2} #3 ~\mathrm{d} #4}

\newcommandtwoopt
{\derivative}[3][][]
{
  \frac
  {\mathrm{d}^{#1} #2}
  {\mathrm{d} #3^{#1}}
}

\newcommandtwoopt
{\pderivative}[3][][]
{
  \frac
  {\partial^{#1} #2}
  {\partial #3^{#1}}
}

\newcommand
{\primeprime}
{{\prime \prime}}

\newcommand
{\primeprimeprime}
{{\prime \prime \prime}}

% Text:

\newcommand{\Quote}[1]{\glqq #1\grqq{}}
\newcommand{\Text}[1]{{\text{#1}}}
\newcommand{\fastueberall}{\text{f.ü.}}
\newcommand{\fastsicher}{\text{f.s.}}

% -------------------------------- %
% amsthm-stuff:

\theoremstyle{definition}

% numbered theorems
\newtheorem{theorem}{Satz}
\newtheorem{lemma}{Lemma}
\newtheorem{corollary}{Korollar}
\newtheorem{proposition}{Proposition}
\newtheorem{remark}{Bemerkung}
\newtheorem{definition}{Definition}
\newtheorem{example}{Beispiel}

% unnumbered theorems
\newtheorem*{theorem*}{Satz}
\newtheorem*{lemma*}{Lemma}
\newtheorem*{corollary*}{Korollar}
\newtheorem*{proposition*}{Proposition}
\newtheorem*{remark*}{Bemerkung}
\newtheorem*{definition*}{Definition}
\newtheorem*{example*}{Beispiel}

% Please define this stuff in project ("main.tex"):

% \def \lastexercisenumber {...}
% This will be 0 by default

% \setcounter{section}{...}
% This will be 0 by default
% and hence, completely ignored

\ifnum \thesection = 0
{\newtheorem{exercise}{Aufgabe}}
\else
{\newtheorem{exercise}{Aufgabe}[section]}
\fi

\ifdef
{\lastexercisenumber}
{\setcounter{exercise}{\lastexercisenumber}}

\newcommand{\solution}
{
    \renewcommand{\proofname}{Lösung}
    \renewcommand{\qedsymbol}{}
    \proof
}

\renewcommand{\proofname}{Beweis}

% -------------------------------- %
% environment zum einkasteln:

% dickere vertical lines
\newcolumntype
{x}
[1]
{!{\centering\arraybackslash\vrule width #1}}

% environment selbst (the big cheese)
\newenvironment
{boxedin}
{
  \begin{tabular}
  {
    x{1 pt}
    p{\textwidth}
    x{1 pt}
  }
  \Xhline
  {2 \arrayrulewidth}
}
{
  \\
  \Xhline{2 \arrayrulewidth}
  \end{tabular}
}

% -------------------------------- %
% MISC "Ein-Deutschungen"

\renewcommand
{\figurename}
{Abbildung}

\renewcommand
{\tablename}
{Tabelle}

% -------------------------------- %


\parindent 0pt

\title
{
  Logik und Grundlagen der Mathematik \\
  \vspace{4pt}
  \normalsize
  \textit{5. Übung am 05.11.2020}
}
\author
{
  Richard Weiss
  \and
  Florian Schager
  \and
  Fabian Zehetgruber
}
\date{}

\begin{document}

\maketitle
\section*{Skolemtheorien}
Sei $\Sigma$ eine Theorie in der Sprache $\mathscr{L}$. Wir nennen $\Sigma$ eine
Skolem-Theorie, wenn es für alle $n \geq 0$ und alle Formeln der Form $\exists y \psi$
mit den freien Variablen $x_1,\dots,x_n$ ein $n$-stelliges Funktionssymbol $f$ gibt,
sodass $\sigma \vdash \exists y \psi \rightarrow \psi[y/f(\overline{x})]$ (wobei wir
$f(\overline{x})$ statt $f(x_1,\dots,x_n)$ schreiben). \\
Für zwei $\mathscr{L}$-Strukturen $\mathscr{M}_1$ und $\mathscr{M}_2$ mit $M_1 \subseteq M_2$
sagen wir $\mathscr{M}_1 \leq \mathscr{M}_2$ (``$\mathscr{M}_1$ ist Unterstruktur von $\mathscr{M}_2$''),
wenn für alle Funktionssymbole $f$ in $\mathscr{L}$ gilt, dass
$f^{\mathscr{M}_1} = f^{\mathscr{M}_2}\upharpoonright M_1$ (genauer:
$f^{\mathscr{M}_1} = f^{\mathscr{M}_2}\upharpoonright M_1^k$) wenn $k$ die Stelligkeit von $f$ ist),
sowie $c^{\mathscr{M}_1} = c^{\mathscr{M}_2}$ für alle Konstantensymbole, und
$R^{\mathscr{M}_1} = R^{\mathscr{M}_2}\cap M^k$ für alle ($k$-stelligen) Relationssymbole $R$. \\
Wir schreiben $\mathscr{M}_1 \preccurlyeq \mathscr{M}_2$, wenn erstens $\mathscr{M}_1 \leq \mathscr{M}_2$
gilt, und überdies für jede Formel $\varphi$ und jede Belegung $b$ mit Werten in
$M_1: \mathscr{M}_1 \vDash \varphi[b] \iff \mathscr{M}_2 \vDash \varphi[b]$.
% --------------------------------------------------------------------------------

\begin{exercise}[Uniform distribution]

Let $X_1, \dots, X_n$ be a random sample from uniform $(\theta, 1)$ distribution, where $\theta < 1$ is an unknown parameter.

\begin{enumerate}[label = (\alph*)]

    \item Find the MLE $\hat \theta$ of $\theta$.

    \item Is $\hat \theta$ asymptotically normal?
    If yes, find the asymptotic mean and variance.
    Otherwise, find a sequence $r_n$ and $a_n$ suh that $r_n (\hat \theta - a_n)$ converges in distribution to a non-degenerate (not pointmass) distribution.

\end{enumerate}

\end{exercise}

% --------------------------------------------------------------------------------

\begin{solution}

\phantom{}

\begin{enumerate}[label = (\alph*)]

    \item

    \begin{align*}
        L_n(\theta \mid x)
        & =
        f_{X_1, \dots, X_n}(x \mid \theta) \\
        & =
        \prod_{i=1}^n
            f_{X_i}(x_i \mid \theta) \\
        & =
        \prod_{i=1}^n
            \frac{1}{1 - \theta} \mathbf 1_{(\theta, 1)}(x_i) \\
        & =
        \frac{1}{(1 - \theta)^n} \mathbf 1_{(\theta, 1)^n}(x) \\
        & =
        \frac{1}{(1 - \theta)^n}
        \mathbf 1_{(\theta, \infty)}
        \pbraces
        {
            \min_{i=1}^n x_i
        }
        \mathbf 1_{(-\infty, 1)}
        \pbraces
        {
            \max_{i=1}^n x_i
        }
    \end{align*}

    We go back to the definition of the maximum liklihood estimator, on \cite*[lecture 6, slide 47]{EStat}.
    The liklihood gien $x$ is zero for $\theta > \min_{i=1}^n x_i$ and is an increasing function of $\theta$ for $\theta \leq \min_{i=1}^n x_i$.
    As a consequence, the maximum liklihood estimator of $\theta$ is

    \begin{align*}
        \hat \theta
        & =
        \min_{i=1}^n X_i \\
        & =
        X_{(1)},
    \end{align*}

    the smallest order statistic.

    \item The cdf of $\hat \theta$ is

    \begin{align*}
        F_{\hat \theta}(x)
        & =
        \P \pbraces{\min_{i=1}^n X_i \leq x} \\
        & =
        1 - \P \pbraces{\min_{i=1}^n X_i > x} \\
        & =
        1 - \prod_{i=1}^n \P(X_i > x) \\
        & =
        1 - \P(X_1 > x)^n \\
        & =
        1 - (1 - \P(X_1 \leq x))^2 \\
        & =
        1 - (1 - F_{X_1}(x))^n \\
        & =
        \begin{cases}
            0,
            & x \leq \theta, \\
            1 - \pbraces{1 - \frac{x - \theta}{1 - \theta}}^n,
            & \theta < x < 1, \\
            1,
            & 1 \leq x.
        \end{cases}
    \end{align*}

    The asymptotic cdf is $\mathbf 1_{(0, \infty)}(x)$;
    the asymptotic pdf is the $\theta$-shifted delta distribution $\delta_\theta$, because

    \begin{align*}
        \Forall \varphi \in C_0^\infty:
            \abraces{\mathbf 1_{(\theta, \infty)}^\prime, \varphi}
            & =
            - \abraces{\mathbf 1_{(\theta, \infty)}, \varphi^\prime} \\
            & =
            -\int_{-\infty}^\infty
                \mathbf 1_{(\theta, \infty)}(x) \varphi^\prime(x)
                ~ \mathrm d x \\
            & =
            -\int_\theta^\infty
                \varphi^\prime(x)
                ~ \mathrm d x \\
            & =
            -\varphi(x) \Big |_{x=\theta}^\infty \\
            & =
            \varphi(\theta) \\
            & =
            \abraces{\delta_\theta, \varphi}.
    \end{align*}

    Let us calculate the cdf of the affine transformed $\hat \theta$.

    \begin{align*}
        F_{r_n (\hat \theta - a_n)}(x)
        & =
        \P \pbraces{r_n \pbraces{\min_{i=1}^n X_i - a_n} \leq x} \\
        & =
        \P \pbraces{\min_{i=1}^n X_i \leq \frac{x}{r_n} + a_n} \\
        & ~~ \vdots \\
        & =
        \begin{cases}
            0,
            & x / r_n + a_n \leq \theta
            \iff
            x \leq r_n (\theta - a_n), \\
            1 - \pbraces{1 - \frac{x / r_n + a_n - \theta}{1 - \theta}}^n,
            & \theta < x / r_n + a_n < 1
            \iff
            r_n (\theta - a_n) \leq x \leq r_n (1 - a_n), \\
            1,
            & 1 \leq x / r_n + a_n
            \iff
            r_n (1 - a_n) \leq x
        \end{cases}
    \end{align*}

    Now, we can see that this expression screams \enquote{Euler}.
    Thus, we set

    \begin{align*}
        a_n := \theta,
        \quad
        \text{and}
        \quad
        r_n := \frac{n}{\theta - 1},
        \quad
        \text{for}
        \quad
        n \in \N.
    \end{align*}

    Plugging in, we get

    \begin{align*}
        \cdots
        & =
        \begin{cases}
            0,
            & x \leq 0, \\
            1 - \pbraces{1 + \frac{x}{n}}^n,
            & 0 < x < n, \\
            1,
            & n \leq x
        \end{cases} \\
        & \xrightarrow{n \to \infty}
        (1 - \mathrm e^x) \mathbf 1_{0, \infty}(x).
    \end{align*}

    Finally,

    \begin{align*}
        r_n (\hat \theta - a_n)
        \xrightarrow[n \to \infty]{\text d}
        \exp(1).
    \end{align*}

\end{enumerate}

\end{solution}

% --------------------------------------------------------------------------------

% --------------------------------------------------------------------------------

\begin{exercise}[143]
Finden Sie ein Beispiel $\mathscr{L},\mathscr{M}_1,\mathscr{M}_2$, sodass
$\mathscr{M}_1 \preccurlyeq \mathscr{M}_2$ und $M_1 \neq M_2$.
\end{exercise}

% --------------------------------------------------------------------------------

\begin{solution}
Wir betrachten eine Sprache ohne Gleichheitsrelation und ohne Konstantensymbole, allerdings
mit einer einstelligen Relation $R$.
\begin{align*}
  \mathscr{M}_1 &= (\{0\}, R^{\mathscr{M}_1}= \emptyset) \\
  \mathscr{M}_2 &= (\{0,1\}, R^{\mathscr{M}_2}= \emptyset) \\
\end{align*}
Jede Atomformel hat also die Form $R(x)$ und ist sowohl in $\mathscr{M}_1$,
als auch in $\mathscr{M}_2$ unter jeder Belegung $b$ falsch. \\
Wir zeigen mittels Induktion nach dem Formelaufbau, dass jede Formel
entweder unter jeder Belegung in $\mathscr{M}_1$ und $\mathscr{M}_1$ wahr oder
unter jeder Belegung $\mathscr{M}_1$ und $\mathscr{M}_1$ falsch ist. \\
Induktiv über den Formelaufbau zeigt man dann, dass das bereits für alle Formeln gelten muss.
\end{solution}

% --------------------------------------------------------------------------------

\begin{solution}
	Ein weiterer Vorschlag ist $\mathscr{L} = \{\leq\}$ und $\mathscr{M}_1 = 2\N$ sowie $\mathscr{M}_2 = \N$. Auch hier ist mir noch kein sauberer Beweis gelungen.
\end{solution}

% --------------------------------------------------------------------------------

\begin{exercise}

  Sei $\Omega = \{(x, y) \in \R^2: 0 < x < 1, x^{1/5} < y < 1\}.$
  \begin{itemize}
      \item[(a)] Finden Sie eine Funktion $u \in H^2(\Omega),$ sodass $u \not\in C^0(\overline{\Omega}).$ \textit{Hinweis:} $u(x, y) = y^\alpha.$
      \item[(b)] Warum ist das kein Widerspruch zur stetigen Einbettung von $H^2(\Omega)$ in $C^0(\Omega)$ in zweidimensionalen Gebieten?
  \end{itemize}

\end{exercise}

% --------------------------------------------------------------------------------

\begin{solution}
\phantom{}
\begin{itemize}
    \item[(a)] Wähle $\alpha = -\frac{1}{2}$, dann ist $u(x,y) = y^{\alpha}$ auf der $y$-Achse
    sicher nicht stetig bis zum Rand und somit nicht in $C^0(\overline{\Omega})$.
    \begin{align*}
      \|u\|_{L^2(\Omega)} &= \int_0^1\int_{x^{1/5}}^1 (y^{-1/2})^2 dy dx
      = -\int_0^1 \ln(x^{1/5}) dx = -\frac{1}{5}[x\ln(x) - x]_0^1 = \frac{1}{5} \\
      \left\|\frac{\partial}{\partial y}u\right\|_{L^2(\Omega)} &=
      \int_0^1\int_{x^{1/5}}^1 \left(-\frac{1}{2}y^{-3/2}\right)^2 dy dx
      = \frac{1}{4}\int_0^1\left[-\frac{1}{2}y^{-2}\right]_{x^{1/5}}^1 dx \\
      &= -\frac{1}{8}\int_0^1 1 - x^{-2/5} dx = -\frac{1}{8}\left(1 - \frac{5}{3}\right) = \frac{1}{12} \\
      \left\|\frac{\partial^2}{\partial y^2}u(x,y)\right\| &=
      \int_0^1\int_{x^{1/5}}^1 \left(\frac{3}{4}y^{-5/2}\right)^2 dy dx
      = \frac{9}{16}\int_0^1 \left[-\frac{1}{4}y^{-4}\right]_{x^{1/5}}^1 dx \\
      &= -\frac{9}{64}\int_0^1 1 - x^{-4/5} dx = -\frac{9}{64}(1 - 5) = \frac{9}{16}
    \end{align*}
    Da die $x$-Ableitungen wegfallen, ist $u \in H^2(\Omega)$.
    \item[b)]
    \includegraphicsboxed{PDEs - Satz 5-9 (Einbettungssatz von Sobolev).png}
    In unserem Fall ist die Bedingung $k - n/2 = 2 - 1 > 0 = m$ erfüllt, also
    muss die Bedingung an $\partial \Omega$ verletzt sein.
    In der Tat kann man zeigen, dass der Rand sich in keiner Umgebung von $(0,0)$
    durch eine Lipschitz-stetige Funktion darstellen lassen kann, da
    \begin{align*}
      \lim_{x \to 0^+}\frac{x^{1/5}}{x} = x^{-4/5} = +\infty.
    \end{align*}
\end{itemize}


\end{solution}

% --------------------------------------------------------------------------------

\begin{algebraUE}{208}
Beweisen Sie folgende Variante des Hauptsatzes 3.4.5.2.: \\
Jede endliche abelsche Gruppe $A$ ist direkte Summe zyklischer Gruppen $(C_{m_i})_{i=1}^n$,
deren Ordnungen $m_i > 1$ eine Teilerkette $m_1 | m_2 | \dots | m_n$ bilden.
Die $m_i$ sind durch $A$ eindeutig bestimmt. \\
\textit{Hinweis:} Aus Lemma 3.4.4.1. folgt leicht, dass direkte Summen zyklischer
Gruppen mit teilerfremden Ordnungen wieder zyklisch sind. Damit lässt sich die
hier zu beweisende Variante ohne große Mühe aus dem Hauptsatz in der Version
von 3.4.5.2. ableiten.
\end{algebraUE}
\begin{solution}
Außergewöhnliche Argumentation.
\end{solution}

\section*{Skolemisierung}
\begin{exercise}
Betrachten Sie ein $m$-stufiges Kollokationsverfahren mit Kollokationspunkten
$c_1,\dots,c_m$. Wir definieren das Polynom
\begin{align*}
  M(x) := \frac{1}{m!}\prod_{i = 1}^m (x - c_i).
\end{align*}
\begin{enumerate}[label = \textbf{\alph*)}]
  \item Zeigen Sie, dass sich für dieses Verfahren die Stabilitätsfunktion $R(z)$
  mit $z = \lambda h$ schreiben lässt als das rationale Polynom $R(z) = P(z)/Q(z)$,
  wobei $P,Q \in \Pi_m$ gegeben sind durch
  \begin{align*}
    P(z) &= M^{(m)}(1) + M^{(m-1)}(1)z + \dots + M(1)z^m, \\
    Q(z) &= M^{(m)}(0) + M^{(m-1)}(0)z + \dots + M(0)z^m.
  \end{align*}
  \item Verwenden Sie diese explizite Darstellung von $R(z)$ um zu zeigen, dass
  Gauß-Verfahren nicht L-stabil sind.
\end{enumerate}
\textit{Hinweis (zu a):} Um die Darstellung von $R(z)$ zu erhalten, betrachten Sie
das übliche Modellproblem mit $h = 1$(dies impliziert $z = \lambda$). Aus der
Definition der Kollokationspolynome $q \in \Pi_m$ schließen Sie nun, dass
\begin{align}\label{a}
  q^{\prime}(x) - zq(x) = KM(x)
\end{align}
für eine Konstante $K \neq 0$. Leiten Sie Gleichung \eqref{a} $s=0,\dots,m$
mal ab, um einen Ausdruck für $q(x)$ zu erhalten. Schließlich gilt
$R(z) = \nicefrac{q(1)}{q(0)}$.
\end{exercise}
\begin{solution}
Die Kollokationspolynome $q_l \in \Pi_m$ werden durch
\begin{align*}
  q_l(t_l) &= y_l, \\
  q_l^{\prime}(t_l + c_jh_l) &= f(t_l + c_jh_l, q_l(t_l + c_jh_l)), \qquad j = 1,\dots,m \\
  y_{l+1} &:= q_l(t_l + h_l)
\end{align*}
eindeutig festgelegt. Wir wählen nun $f(y) = \lambda y, t_0 = 0, h_0 = 1$
und erhalten also
\begin{align*}
  q(0) &= y_0 \\
  q^{\prime}(c_j) &= \lambda q(c_j), \qquad j = 1,\dots,m \\
  y_1 := q(1).
\end{align*}
Aus der Definition der Stabilitätsfunktion folgt
\begin{align*}
  y_1 = R(\lambda)y_0 \implies  R(z) = R(\lambda) = \frac{y_1}{y_0} = \frac{q(1)}{q(0)}.
\end{align*}
Als nächstes zeigen wir
\begin{align*}
  q^{\prime}(x) - zq(x) - KM(x) \equiv 0
\end{align*}
für geeignetes $K$. Dafür bemerken wir, dass für $c_j, j=1,\dots,m$
\begin{align*}
  q^{\prime}(c_j) - zq(c_j) - KM(c_j) = zq(c_j) - zq(c_j) = 0
\end{align*}
gilt. Nun wähle $x_0: q^{\prime}(x_0) - zq(x_0) \neq 0$ beliebig. Dies ist sicher möglich,
da anderenfalls $\forall x: q^{\prime}(x) = zq(x)$ und somit $q(x) = C\exp(zx)$ folgen würde.
Nun setze
\begin{align*}
  K := \frac{q^{\prime}(x_0) - zq(x_0)}{M(x_0)}
\end{align*}
und es folgt
\begin{align*}
  q^{\prime}(x_0) - zq(x_0) - KM(x_0) = 0.
\end{align*}
Damit hat das Polynom $q^{\prime} - zq - KM \in \Pi_m$ insgesamt $m+1$ Nullstellen
und muss damit das Nullpolynom sein.
Jetzt können wir $KM(x), s = 1,\dots,m$ Mal ableiten
und erhalten für $q(x) = \sum_{k=1}^m a_kx^k$
\begin{align*}
  \partial_x^s KM(x) &= \partial_x^s (q^{\prime}(x) + zq(x))
  = \partial_x^{s+1}q(x) - z\partial_x^s q(x)
  = \sum_{k=0}^{m-s-1}a_{k+s+1}\frac{(k+s+1)!}{k!}x^k - z\sum_{k=0}^{m-s}a_{k+s}\frac{(k+s)!}{k!}x^k.
\end{align*}
Nun setzen wir für $x = 1$ ein.
\begin{align*}
  KM^{(s)}(1) &= \sum_{k=0}^{m-s-1}a_{k+s+1}\frac{(k+s+1)!}{k!}
  - z\underbrace{\sum_{k=0}^{m-s}a_{k+s}\frac{(k+s)!}{k!}}_{=: b_s} \\
  KP(z) &= \sum_{s=0}^m KM^{(s)}(1)z^{m-s} = \sum_{s=0}^m (b_{s+1} - zb_s)z^{m-s} \\
  &= \sum_{s=0}^m z^{m-s}b_{s+1} - z^{m-s+1}b_s
  = -z^{m+1}b_0 + b_{m+1}
  = -z^{m+1}\sum_{i=1}^m a_i
  = -z^{m+1}q(1)
\end{align*}
Für $x=0$ erhalten wir wiederum
\begin{align*}
  KM^{(s)}(0) &= \begin{cases}
    a_{s+1}(s+1)! - za_ss!, & s < m \\
    -za_mm!, & s = m
  \end{cases} \\
  KQ(z) &= K\sum_{s=0}^m M^{(s)}(0)z^{m-s} = \sum_{s=0}^{m-1}(b_{s+1}-zb_s)z^{m-s} - zb_m \\
  &= \sum_{s=0}^{m-1}z^{m-s}b_{s+1} - z^{m-s+1}b_s - zb_m
  = -z^{m+1}b_0
  = -z^{m+1}a_0
  = -z^{m+1}q(0)
\end{align*}
Daraus folgt
\begin{align*}
  \frac{P(z)}{Q(z)}=\frac{KP(z)}{KQ(z)} = \frac{-z^{m+1}q(1)}{-z^{m+1}q(0)} = \frac{q(1)}{q(0)} = R(z).
\end{align*}
\item
Aus Aufgabe 27 von letzter Woche wissen wir, dass die Kollokationspunkte einer
Gauß-Kollokation für die konstante Gewichtsfunktion $\omega(x) \equiv 1$
symmetrisch um $\nicefrac{1}{2}$ liegen, also $c_j = 1 - c_{m+1-j}, j = 1,\dots,m$.
Daraus folgt
\begin{align*}
  M\left(\frac{1}{2}+x\right) &= \frac{1}{m!}\prod_{j=1}^m\left(\frac{1}{2}+x-c_j\right)
  = \frac{1}{m!}\prod_{j=1}^m\left(\frac{1}{2}+x - (1 - c_{m+1-j})\right) \\
  &= (-1)^m\frac{1}{m!}\prod_{j=1}^m\left(\frac{1}{2} - x - c_{m+1-j}\right)
  = (-1)^m M\left(\frac{1}{2}-x\right).
\end{align*}
und wir erhalten
\begin{align*}
  |M(0)| = |M(1)|.
\end{align*}
Da $R$ ein rationales Polynom ist, ist der Grenzwert für $z \rightarrow \infty$
bereits durch die Führungskoeffizienten bestimmt:
\begin{align*}
  \lim_{\Re(z) \rightarrow -\infty} |R(z)| = \left|\frac{M(0)}{M(1)}\right| = 1.
\end{align*}
\end{solution}

% --------------------------------------------------------------------------------

\begin{exercise}[176]
Finden Sie eine Formel $\forall x\, \exists y\, \varphi$, die das Funktionssymbol $f$
enthält, sodass $\forall x\, \exists y\, \varphi$ nicht erfüllbarkeitsäquivalent ist zu
$\forall x\, \varphi(y/f(x))$.
\end{exercise}

% --------------------------------------------------------------------------------

\begin{solution}
	Wir können die Struktur $M = \{0, 1\}$ mit der gewöhnlichen $\leq$ Relation betrachten.
  Weiters interpretieren wir $f(0) = 1$ und $f(1) = 0$.
  So gilt zwar für $b(y) = 0$ die Formel $\forall x \exists y (f(x) \leq f(y))$
  allerdings gilt nicht $\forall x (f(x) \leq f(f(x)))$,
  wie man an $f(0) = 1 > 0 = f(f(0))$ sieht. \\


  Da bin ich mir noch nicht sicher. Kann ich nicht einfach ein anderes Modell
  finden, in dem $\forall x\, \varphi(y/f(x))$ erfüllt wird?
\end{solution}

% --------------------------------------------------------------------------------




\end{document}
