% --------------------------------------------------------------------------------

\begin{exercise}[92]

Geben Sie Formeln $\varphi$ und $\psi$ (in einer geeigneten Sprache $\mathscr{L}$)
an, sodass zwar (a) aber nicht (b) gilt:
\begin{enumerate}[label = \alph*)]
  \item Für alle $\mathscr{L}$-Strukturen $\mathscr{M}$ gilt:
  Wenn $\mathscr{M} \vDash \varphi$, dann $\mathscr{M} \vDash \psi$.
  \item Für alle $\mathscr{L}$-Strukturen $\mathscr{M}$ gilt:
  $\mathscr{M} \vDash \varphi \rightarrow \psi$.
\end{enumerate}

\end{exercise}

% --------------------------------------------------------------------------------

\begin{solution}

Wir wählen unsere Formeln $\varphi = x < y, \psi = y < x$. \\
Es gilt (a), da aus
\begin{align*}
  \forall b: \hat{b}(\varphi) = 1
\end{align*}
folgt, dass
\begin{align*}
  \forall b: \hat{b}(\psi) = \widehat{(b_{x/b(y)})_{y/b(x)}}(\varphi) = 1.
\end{align*}
Damit (b) nicht gilt, wählen wir das Modell $\mathscr{M} = (\{0,1\}, <)$ und sehen
mit der Belegung $b: x \mapsto 0, y \mapsto 1$

\begin{align*}
  \hat{b}(\varphi) &= \hat{b}(x < y) = 1 \\
  \hat{b}(\psi) &= \hat{b}(y < x) = 0,
\end{align*}

also $\mathscr{M} \nvDash \varphi \rightarrow \psi$.
\end{solution}

% --------------------------------------------------------------------------------
