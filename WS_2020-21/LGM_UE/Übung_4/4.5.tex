% -------------------------------------------------------------------------------- %

\begin{exercise}[94]

\phantom{}

\begin{enumerate}[label = \alph*)]
  \item Geben Sie einen formalen Beweis für eine der Formeln in 77-82 an.
  \item Seien $\varphi$ und $\psi$ beliebige Formeln. \\
  Geben Sie einen formalen Beweis für die Formel $(\forall x\, \varphi) \rightarrow
  \forall x \, (\varphi \lor \psi)$
  an.
\end{enumerate}

\end{exercise}

% -------------------------------------------------------------------------------- %

\begin{solution}

\phantom{}

\begin{enumerate}
  \item Wir beweisen 82, also
  \begin{align*}
    \forall x\, (Px \rightarrow Qx) \rightarrow ((\forall x\, Px) \rightarrow (\forall x\, Qx)).
  \end{align*}
  Genau genommen sind wir damit schon fertig, denn 82 ist einfach ein Distributivitätsaxiom.
  \item
  \begin{algorithmic}[1]
    \State $\vdash \forall x\, (\varphi \rightarrow \varphi \lor \psi)$
    \Comment Allquantor vor Tautologie: Axiom
    \State $\vdash \forall x\, (\varphi \rightarrow (\varphi \lor \psi))
    \rightarrow ((\forall x\, \varphi) \rightarrow (\forall x\, (\varphi \lor \psi)))$
    \Comment Distributivitätsaxiom
    \State $\vdash (\forall x \varphi) \rightarrow \forall x (\varphi \rightarrow \psi)$
    \Comment Modus Ponens
  \end{algorithmic}
\end{enumerate}

\end{solution}

% -------------------------------------------------------------------------------- %
