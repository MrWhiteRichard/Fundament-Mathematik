% ------------------------------------------------------------------------------

\begin{exercise}[225]

Eine Menge $A \subseteq \N$ heißt $\Pi_1$-Menge, wenn $\N \setminus A$ eine
$\Sigma_1$-Menge ist. \\
Eine Menge $B \subseteq \N$ heißt $\Delta_1$-Menge, wenn $B$ sowohl $\Sigma_1$-
als auch $\Pi_1$-Menge ist. \\
\begin{enumerate}[label = \alph*.]
	\item Seien $S_0,S_1 \subseteq \N\ \Sigma_1$-Mengen mit $S_0 \cup S_1 = \N$.
	Dann gibt es eine $\Delta_1$-Menge $E$, sodass $S_0 \setminus S_1 \subseteq E \subseteq S_0$
	gilt. \\
	\textit{Hinweis:} Für $n \in S_0 \cap S_1$ vergleiche die jeweils kleinsten
	Zeugen für $n \in S_0$ und $n \in S_1$.
	\item Seien $Q_0,Q_1 \in \N\ \Pi_1$-Mengen mit $Q_0 \cap Q_1 = \emptyset$.
	Dann gibt es eine $\Delta_1$-Menge $E$, die $Q_0$ und $Q_1$ trennt, also
	$Q_0 \subseteq E, Q_1 \subseteq \N \setminus E$.
\end{enumerate}

\end{exercise}

% ------------------------------------------------------------------------------

\begin{solution}
Da $S_0,S_1 \ \Sigma_1$-Mengen sind, gibt es $\Sigma_0$-Formeln $\psi_0,\psi_1$, sodass
\begin{align*}
	x \in S_0 &\iff \exists y\, \psi_0(x,y) \\
	x \in S_1 &\iff \exists y\, \psi_1(x,y) \\
\end{align*}
Wir kreieren nun eine neue $\Sigma_1$-Formel $\varphi$:
\begin{align*}
	\exists y_0\, (\psi_0(x,y_0) \land \forall y_1 < y_0\, \neg \psi_1(x,y_1))
\end{align*}
und definieren $E := \{x \in \N: \N \vDash \varphi(x)\}$.
Es gilt
\begin{align*}
	x \in S_0 \setminus S_1 &\iff \exists y_0\, \psi_0(x,y_0) \land
	\neg \exists y_1\, \psi_1(x,y_1) \iff
	\exists y_0\, \psi_0(x,y_0) \land
	 \forall y_1\, \neg \psi_1(x,y_1) \\
	 &\implies (\exists y_0\, \psi_0(x,y_0) \land
 	 \forall y_1 < y_0\, \neg \psi_1(x,y_1)) \iff x \in E \\
	 &\iff \exists y_0\, (\psi_0(x,y_0) \land \forall y_1 < y_0\, \neg \psi_1(x,y_1))
	 \implies \exists y_0\, \psi_0(x,y_0) \iff x \in S_0.
\end{align*}
Es bleibt also noch zu zeigen, dass $\N \setminus E$ ebenso eine $\Sigma_1$-Menge ist:
Aus $S_0 \cup S_1 = \N$ folgt
\begin{align*}
	\N \vDash \exists y_0\, (\psi_0(x,y_0) \land \forall y_1 < y_0\, \neg \psi_1(x,y_1))
	\lor \exists y_1\, (\psi_1(x,y_1) \land \forall y_0 \leq y_1\, \neg \psi_0(x,y_0))
\end{align*}
und damit
\begin{align*}
	x \in \N \setminus E &\iff \exists y_1\, (\psi_1(x,y_1) \land \forall y_0 \leq y_1\, \neg \psi_0(x,y_0)).
\end{align*}
\end{solution}
