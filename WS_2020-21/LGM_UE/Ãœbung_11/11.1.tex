% -------------------------------------------------------------------------------- %

\begin{exercise}[224]
	Für $i = 0,1$ sei $P_i$ die Menge aller Programme $p$, die bei Eingabe $p$ halten und $i$ ausgeben. Es gilt offenbar $P_0 \cap P_1 = \emptyset$. Zeigen Sie, dass die Mengen $P_0$ und $P_1$ semi-entscheidbar sind. (indem Sie Programme skizzieren, die die jeweiligen partiellen charakteristischen $\tilde{\chi}_{P_i}$ berechnen), es aber keine entscheidbare Menge $E$ mit $P_0 \subseteq E$, $P_1 \subseteq \N \setminus E$ gibt.

\end{exercise}

% -------------------------------------------------------------------------------- %

\begin{solution}

\phantom{}

	\begin{algorithm}
		\caption{charakteristische Funktion von $P_i$ für $i \in \{0,1\}$}
		\begin{algorithmic}[1]
			\Procedure{$\tilde{\chi}_{P_i}$}{$p$}
				\If{$p(p) = i$}
					\State \Return 1
				\EndIf
			\EndProcedure
		\end{algorithmic}
	\end{algorithm}

	Für den zweiten Teil identifizieren wir die Menge aller Programme mit einer
	beliebigen Abzählung und nehmen an, es gibt eine entscheidbare Menge
	$E$ mit $P_0 \subseteq E$ und $P_1 \subseteq \N \setminus E$. \\
	Dann können wir ein Programm $q$ schreiben.

	\begin{algorithm}
		\caption{Programm $q$}
		\begin{algorithmic}[1]
			\Procedure{$q$}{$p$}
				\If{$\chi_E(p) = 1$}
					\State \Return $1$
				\Else
					\State \Return$0$
				\EndIf
			\EndProcedure
		\end{algorithmic}
	\end{algorithm}

	Nun betrachten wir $q(q)$.
	\begin{enumerate}[label = Fall \arabic*:]
		\item $q(q) = 1$. Dann folgt $q \in E$ also $q \notin P_1$ und daher $q(q) \neq 1$. Ein Widerspruch!

		\item $q(q) = 0$. Dann folgt $q \notin E$ also $q \notin P_0$ und damit $q(q) \neq 0$. Ein Widerspruch!
	\end{enumerate}
\end{solution}
