% --------------------------------------------------------------------------------

\begin{exercise}[236]

\phantom{}
	\begin{enumerate}[label = (\alph*)]
		\item Sei $A \subseteq \N^2$ eine $\Sigma_1$-Menge. Dann sind die Mengen
		$B := \{y \in \N: (5,y) \in A\}$ und \\
		$C := \{x \in \N: (x,x) \in A\}$ auch $\Sigma_1$-Mengen.

		\item Es gibt eine $\Sigma_1$-Menge $U \subseteq \N$, die keine $\Pi_1$-Menge ist. (Das heißt: $\N \setminus U$ ist keine $\Sigma_1$-Menge.)    Hinweis: Verwenden Sie (a) sowie die beiden vorigen Aufgaben.
	\end{enumerate}

\end{exercise}

% --------------------------------------------------------------------------------

\begin{solution}

\phantom{}
	\begin{enumerate}[label = (\alph*)]
		\item
		Wir können $B$ und $C$ als Projektionen von Schnitten von $\Sigma_1$-Mengen darstellen:
		\begin{align*}
			B^{\prime} &:= \{(x,y) \in \N^2: x = 5\} \\
			B &= \{y \in \N: (x,y) \in A \cap B^{\prime}\} \\
			C^{\prime} &:= \{(x,y) \in \N^2: x = y\} \\
			C &= \{x \in \N: (x,y) \in A \cap C^{\prime}\}
		\end{align*}
		\item Aus der vorigen Aufgabe holen wir uns eine $\Sigma_1$-Menge $A \subseteq \N \times \N$
		mit der Eigenschaft, dass es für jede $\Sigma_1$-Menge $B \subseteq \N$
		ein $k \in \N$ mit $B = \{y \in \N \mid (k,y) \in A\}$ gibt.
		Wegen der vorvorigen Aufgabe können wir weiters eine Bijektion
		$p_2: \N^2 \to \N$ finden, sodass für alle $C \subseteq \N^2$ gilt,
		dass $C$ genau dann eine $\Sigma_1$-Menge ist, wenn $p_2[C]$ eine $\Sigma_1$-Menge ist.

		Wir definieren $U := p_2[A]$, damit ist $U$ eine $\Sigma_1$-Menge. Nun nehmen wir an $\N \setminus U$ ist ebenfalls eine $\Sigma_1$-Menge. Damit ist auch
		\begin{align*}
			p_2^{-1}[\N \setminus U] = p_2^{-1}[\N] \setminus p_2^{-1}[U] = \N^2 \setminus A
		\end{align*}
		eine $\Sigma_1$-Menge. Mit Punkt (a) folgt, dass auch $B := \{x \in \N \mid (x,x) \in \N^2 \setminus A\}$ eine $\Sigma_1$-Menge ist. Wir unterscheiden für beliebiges $k \in \N$ nun zwei Fälle.
		\begin{enumerate}[label = Fall \arabic*:]
			\item $k \in B$. Dann ist $(k,k) \in \N^2 \setminus A$ und daher $k \notin \{x \in \N \mid (k,x) \in A\}$. Und damit $B \neq \{x \in \N \mid (k,x) \in A\}$.
			\item $k \notin B$. Dann ist $(k,k) \notin \N^2 \setminus A$ und daher $(k,k) \in A$. Damit ist  $k \in \{x \in \N \mid (k,x) \in A\}$ und damit $B \neq \{x \in \N \mid (k,x) \in A\}$
		\end{enumerate}
	In jedem Fall gilt also
	\begin{align*}
		B \neq \{x \in \N \mid (k,x) \in A\}
	\end{align*}
	und da das für alle $k \in \N$ gilt ist das ein Widerspruch zur Eigenschaft von $A$.
	\end{enumerate}

\end{solution}
