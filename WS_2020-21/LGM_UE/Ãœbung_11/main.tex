\documentclass{article}

% ---------------------------------------------------------------- %
% short package descriptions are copied from
% https://ctan.org/

% ---------------------------------------------------------------- %

% Accept different input encodings
\usepackage[utf8]{inputenc}

% Standard package for selecting font encodings
\usepackage[T1]{fontenc}

% ---------------------------------------------------------------- %

% Multilingual support for Plain TEX or LATEX
\usepackage[ngerman]{babel}

% ---------------------------------------------------------------- %

% Set all page margins to 1.5cm
\usepackage{fullpage}

% Margin adjustment and detection of odd/even pages
\usepackage{changepage}

% Flexible and complete interface to document dimensions
\usepackage{geometry}

% ---------------------------------------------------------------- %
% mathematics

\usepackage{amsmath}  % AMS mathematical facilities for LATEX
\usepackage{amssymb}
\usepackage{amsfonts} % TEX fonts from the American Mathematical Society
\usepackage{amsthm}   % Typesetting theorems (AMS style)

% Mathematical tools to use with amsmath
\usepackage{mathtools}

% Support for using RSFS fonts in maths
\usepackage{mathrsfs}

% Commands to produce dots in math that respect font size
\usepackage{mathdots}

% "Blackboard-style" cm fonts
\usepackage{bbm}

% Typeset in-line fractions in a "nice" way
\usepackage{nicefrac}

% Typeset quotient structures with LATEX
\usepackage{faktor}

% Vector arrows
\usepackage{esvect}

% St Mary Road symbols for theoretical computer science
\usepackage{stmaryrd}

% Three series of mathematical symbols
\usepackage{mathabx}

% ---------------------------------------------------------------- %
% algorithms

% Package for typesetting pseudocode
\usepackage{algpseudocode}

% Typeset source code listings using LATEX
\usepackage{listings}

% Reimplementation of and extensions to LATEX verbatim
\usepackage{verbatim}

% If necessary, please use the following 2 packages locally, but never both.
% This is because the algorithm environment gets defined in both packages, which leads to name conflicts.
% \usepackage{algorithm2e}
% \usepackage{algorithm}

% ---------------------------------------------------------------- %
% utilities

% A generic document command parser
\usepackage{xparse}

% Extended conditional commands
\usepackage{xifthen}

% e-TEX tools for LATEX
\usepackage{etoolbox}

% Define commands with suffixes
\usepackage{suffix}

% Extensive support for hypertext in LATEX
\usepackage{hyperref}

% Driver-independent color extensions for LATEX and pdfLATEX
\usepackage{xcolor}

% ---------------------------------------------------------------- %
% graphics

% -------------------------------- %

\usepackage{tikz}

% MISC
\usetikzlibrary{patterns}
\usetikzlibrary{decorations.markings}
\usetikzlibrary{positioning}
\usetikzlibrary{arrows}
\usetikzlibrary{arrows.meta}
\usetikzlibrary{overlay-beamer-styles}

% finite state machines
\usetikzlibrary{automata}

% turing machines
\usetikzlibrary{calc}
\usetikzlibrary{chains}
\usetikzlibrary{decorations.pathmorphing}

% -------------------------------- %

% Draw tree structures
\usepackage[noeepic]{qtree}

% Enhanced support for graphics
\usepackage{graphicx}

% Figures broken into subfigures
\usepackage{subfig}

% Improved interface for floating objects
\usepackage{float}

% Control float placement
\usepackage{placeins}

% Include PDF documents in LATEX
\usepackage{pdfpages}

% ---------------------------------------------------------------- %

% Control layout of itemize, enumerate, description
\usepackage[inline]{enumitem}

% Intermix single and multiple columns
\usepackage{multicol}
\setlength{\columnsep}{1cm}

% Coloured boxes, for LATEX examples and theorems, etc
\usepackage{tcolorbox}

% ---------------------------------------------------------------- %
% tables

% Tabulars with adjustable-width columns
\usepackage{tabularx}

% Tabular column heads and multilined cells
\usepackage{makecell}

% Publication quality tables in LATEX
\usepackage{booktabs}

% ---------------------------------------------------------------- %
% bibliography and quoting

% Sophisticated Bibliographies in LATEX
\usepackage[backend = biber, style = alphabetic]{biblatex}

% Context sensitive quotation facilities
\usepackage{csquotes}

% ---------------------------------------------------------------- %

% ---------------------------------------------------------------- %
% special letters

\newcommand{\N}{\mathbb N}
\newcommand{\Z}{\mathbb Z}
\newcommand{\Q}{\mathbb Q}
\newcommand{\R}{\mathbb R}
\newcommand{\C}{\mathbb C}
\newcommand{\K}{\mathbb K}
\newcommand{\T}{\mathbb T}
\newcommand{\E}{\mathbb E}
\newcommand{\V}{\mathbb V}
\renewcommand{\S}{\mathbb S}
\renewcommand{\P}{\mathbb P}
\newcommand{\1}{\mathbbm 1}
\newcommand{\G}{\mathbb G}

\newcommand{\iu}{\mathrm i}

% ---------------------------------------------------------------- %
% quantors

\newcommand{\Forall}        {\forall ~}
\newcommand{\Exists}        {\exists ~}
\newcommand{\nExists}       {\nexists ~}
\newcommand{\ExistsOnlyOne} {\exists! ~}
\newcommand{\nExistsOnlyOne}{\nexists! ~}
\newcommand{\ForAlmostAll}  {\forall^\infty ~}

% ---------------------------------------------------------------- %
% graphics boxed

\newcommand
{\includegraphicsboxed}
[2][0.75]
{
    \begin{center}
        \begin{tcolorbox}[standard jigsaw, opacityback = 0]

            \centering
            \includegraphics[width = #1 \textwidth]{#2}

        \end{tcolorbox}
    \end{center}
}

\newcommand
{\includegraphicsunboxed}
[2][0.75]
{
    \begin{center}
        \includegraphics[width = #1 \textwidth]{#2}
    \end{center}
}

\NewDocumentCommand
{\includegraphicsgraphicsboxed}
{ O{0.75} O{0.25} m m}
{
    \begin{center}
        \begin{tcolorbox}[standard jigsaw, opacityback = 0]

            \centering
            \includegraphics[width = #1 \textwidth]{#3} \\
            \vspace{#2 cm}
            \includegraphics[width = #1 \textwidth]{#4}

        \end{tcolorbox}
    \end{center}
}

\NewDocumentCommand
{\includegraphicsgraphicsunboxed}
{ O{0.75} O{0.25} m m}
{
    \begin{center}

        \centering
        \includegraphics[width = #1 \textwidth]{#3} \\
        \vspace{#2 cm}
        \includegraphics[width = #1 \textwidth]{#4}

    \end{center}
}

% ---------------------------------------------------------------- %
% braces

\newcommand{\pbraces}[1]{{\left  ( #1 \right  )}}
\newcommand{\bbraces}[1]{{\left  [ #1 \right  ]}}
\newcommand{\Bbraces}[1]{{\left \{ #1 \right \}}}
\newcommand{\vbraces}[1]{{\left  | #1 \right  |}}
\newcommand{\Vbraces}[1]{{\left \| #1 \right \|}}

\newcommand{\abraces}[1]{{\left \langle #1 \right \rangle}}

\newcommand{\floorbraces}[1]{{\left \lfloor #1 \right \rfloor}}
\newcommand{\ceilbraces} [1]{{\left \lceil  #1 \right \rceil }}

\newcommand{\dbbraces}    [1]{{\llbracket     #1 \rrbracket}}
\newcommand{\dpbraces}    [1]{{\llparenthesis #1 \rrparenthesis}}
\newcommand{\dfloorbraces}[1]{{\llfloor       #1 \rrfloor}}
\newcommand{\dceilbraces} [1]{{\llceil        #1 \rrceil}}

\newcommand{\dabraces}[1]{{\left \langle \left \langle #1 \right \rangle \right \rangle}}

\newcommand{\abs}  [1]{\vbraces{#1}}
\newcommand{\round}[1]{\bbraces{#1}}
\newcommand{\floor}[1]{\floorbraces{#1}}
\newcommand{\ceil} [1]{\ceilbraces{#1}}

% ---------------------------------------------------------------- %

% MISC

% metric spaces
\newcommand{\norm}[2][]{\Vbraces{#2}_{#1}}
\DeclareMathOperator{\metric}{d}
\DeclareMathOperator{\dist}  {dist}
\DeclareMathOperator{\diam}  {diam}

% O-notation
\newcommand{\landau}{{\scriptstyle \mathcal{O}}}
\newcommand{\Landau}{\mathcal{O}}

% ---------------------------------------------------------------- %

% math operators

% hyperbolic trigonometric function inverses
\DeclareMathOperator{\areasinh}{areasinh}
\DeclareMathOperator{\areacosh}{areacosh}
\DeclareMathOperator{\areatanh}{areatanh}

% special functions
\DeclareMathOperator{\id} {id}
\DeclareMathOperator{\sgn}{sgn}
\DeclareMathOperator{\Inv}{Inv}
\DeclareMathOperator{\erf}{erf}
\DeclareMathOperator{\pv} {pv}

% exponential function as power
\WithSuffix \newcommand \exp* [1]{\mathrm{e}^{#1}}

% operations on sets
\DeclareMathOperator{\meas}{meas}
\DeclareMathOperator{\card}{card}
\DeclareMathOperator{\Span}{span}
\DeclareMathOperator{\conv}{conv}
\DeclareMathOperator{\cof}{cof}
\DeclareMathOperator{\mean}{mean}
\DeclareMathOperator{\avg}{avg}
\DeclareMathOperator*{\argmax}{argmax}
\DeclareMathOperator*{\argsmax}{argsmax}

% number theory stuff
\DeclareMathOperator{\ggT}{ggT}
\DeclareMathOperator{\kgV}{kgV}
\DeclareMathOperator{\modulo}{mod}

% polynomial stuff
\DeclareMathOperator{\ord}{ord}
\DeclareMathOperator{\grad}{grad}

% function properties
\DeclareMathOperator{\ran}{ran}
\DeclareMathOperator{\supp}{supp}
\DeclareMathOperator{\graph}{graph}
\DeclareMathOperator{\dom}{dom}
\DeclareMathOperator{\Def}{def}
\DeclareMathOperator{\rg}{rg}

% matrix stuff
\DeclareMathOperator{\GL}{GL}
\DeclareMathOperator{\SL}{SL}
\DeclareMathOperator{\U}{U}
\DeclareMathOperator{\SU}{SU}
\DeclareMathOperator{\PSU}{PSU}
% \DeclareMathOperator{\O}{O}
% \DeclareMathOperator{\PO}{PO}
% \DeclareMathOperator{\PSO}{PSO}
\DeclareMathOperator{\diag}{diag}

% algebra stuff
\DeclareMathOperator{\At}{At}
\DeclareMathOperator{\Ob}{Ob}
\DeclareMathOperator{\Hom}{Hom}
\DeclareMathOperator{\End}{End}
\DeclareMathOperator{\Aut}{Aut}
\DeclareMathOperator{\Lin}{L}

% other function classes
\DeclareMathOperator{\Lip}{Lip}
\DeclareMathOperator{\Mod}{Mod}
\DeclareMathOperator{\Dil}{Dil}

% constants
\DeclareMathOperator{\NIL}{NIL}
\DeclareMathOperator{\eps}{eps}

% ---------------------------------------------------------------- %
% doubble & tripple powers

\newcommand
{\primeprime}
{{\prime \prime}}

\newcommand
{\primeprimeprime}
{{\prime \prime \prime}}

\newcommand
{\astast}
{{\ast \ast}}

\newcommand
{\astastast}
{{\ast \ast \ast}}

% ---------------------------------------------------------------- %
% derivatives

\NewDocumentCommand
{\derivative}
{ O{} O{} m m}
{
    \frac
    {\mathrm d^{#2} {#1}}
    {\mathrm d {#3}^{#2}}
}

\NewDocumentCommand
{\pderivative}
{ O{} O{} m m}
{
    \frac
    {\partial^{#2} {#1}}
    {\partial {#3}^{#2}}
}

\DeclareMathOperator{\Div}{div}
\DeclareMathOperator{\rot}{rot}

% ---------------------------------------------------------------- %
% integrals

\NewDocumentCommand
{\Int}
{ O{} O{} m m}
{\int_{#1}^{#2} #3 ~ \mathrm d #4}

\NewDocumentCommand
{\Iint}
{ O{} O{} m m m}
{\iint_{#1}^{#2} #3 ~ \mathrm d #4 ~ \mathrm d #5}

\NewDocumentCommand
{\Iiint}
{ O{} O{} m m m m}
{\iiint_{#1}^{#2} #3 ~ \mathrm d #4 ~ \mathrm d #5 ~ \mathrm d #6}

\NewDocumentCommand
{\Iiiint}
{ O{} O{} m m m m m}
{\iiiint_{#1}^{#2} #3 ~ \mathrm d #4 ~ \mathrm d #5 ~ \mathrm d #6 ~ \mathrm d #7}

\NewDocumentCommand
{\Idotsint}
{ O{} O{} m m m}
{\idotsint_{#1}^{#2} #3 ~ \mathrm d #4 \dots ~ \mathrm d #5}

\NewDocumentCommand
{\Oint}
{ O{} O{} m m}
{\oint_{#1}^{#2} #3 ~ \mathrm d #4}

% ---------------------------------------------------------------- %

% source:
% https://tex.stackexchange.com/questions/203257/tikz-chains-with-one-side-of-the-leftmost-node-thickbold

% #1 (optional): current state, e.g. $q_0$
% #2: cursor position, e.g. 1
% #3: number of displayed cells, e.g. 5
% #4: contents of cells, e.g. {$\triangleright$, $x_1$, \dots, $x_n$, \textvisiblespace}

\newcommand{\turingtape}[4][]
{
    \begin{tikzpicture}

        \tikzset{tape/.style={minimum size=.7cm, draw}}

        \begin{scope}[start chain=0 going right, node distance=0mm]
            \foreach \x [count=\i] in #4
            {
                \ifnum\i=#3 % if last node reset outer sep to 0pt
                    \node [on chain=0, tape, outer sep=0pt] (n\i) {\x};
                    \draw (n\i.north east) -- ++(.1,0) decorate [decoration={zigzag, segment length=.12cm, amplitude=.02cm}] {-- ($(n\i.south east)+(+.1,0)$)} -- (n\i.south east) -- cycle;
                \else
                    \node [on chain=0, tape] (n\i) {\x};
                \fi

                \ifnum\i=1 % if first node draw a thick line at the left
                    \draw [line width=.1cm] (n\i.north west) -- (n\i.south west);
                \fi
            }
 
            \node [right=.25cm of n#3] {$\cdots$};
            \node [tape, above left=.25cm and 1cm of n1] (q) {#1};
            \draw [>=latex, ->] (q) -| (n#2);

        \end{scope}

    \end{tikzpicture}
}

% ---------------------------------------------------------------- %

% ---------------------------------------------------------------- %
% amsthm-environments:

\theoremstyle{definition}

% numbered theorems
\newtheorem{theorem}             {Satz}[section]
\newtheorem{lemma}      [theorem]{Lemma}
\newtheorem{corollary}  [theorem]{Korollar}
\newtheorem{proposition}[theorem]{Proposition}
\newtheorem{remark}     [theorem]{Bemerkung}
\newtheorem{definition} [theorem]{Definition}
\newtheorem{example}    [theorem]{Beispiel}
\newtheorem{heuristics} [theorem]{Heuristik}

% unnumbered theorems
\newtheorem*{theorem*}    {Satz}
\newtheorem*{lemma*}      {Lemma}
\newtheorem*{corollary*}  {Korollar}
\newtheorem*{proposition*}{Proposition}
\newtheorem*{remark*}     {Bemerkung}
\newtheorem*{definition*} {Definition}
\newtheorem*{example*}    {Beispiel}
\newtheorem*{heuristics*} {Heuristik}

% ---------------------------------------------------------------- %
% exercise- and solution-environments:

% Please define this stuff in project ("main.tex"):
% \def \lastexercisenumber {...}

\newtheorem{exercise}{Aufgabe}
\setcounter{exercise}{\lastexercisenumber}

\newenvironment{solution}
{
  \begin{proof}[Lösung]
}{
  \end{proof}
}

% ---------------------------------------------------------------- %
% MISC translations for environment-names

\renewcommand{\proofname} {Beweis}
\renewcommand{\figurename}{Abbildung}
\renewcommand{\tablename} {Tabelle}

% ---------------------------------------------------------------- %

% ---------------------------------------------------------------- %
% https://www.overleaf.com/learn/latex/Code_listing

\definecolor{codegreen} {rgb}{0, 0.6, 0}
\definecolor{codegray}    {rgb}{0.5, 0.5, 0.5}
\definecolor{codepurple}{rgb}{0.58, 0, 0.82}
\definecolor{backcolour}{rgb}{0.95, 0.95, 0.92}

\lstdefinestyle{overleaf}
{
    backgroundcolor = \color{backcolour},
    commentstyle = \color{codegreen},
    keywordstyle = \color{magenta},
    numberstyle = \tiny\color{codegray},
    stringstyle = \color{codepurple},
    basicstyle = \ttfamily \footnotesize,
    breakatwhitespace = false,
    breaklines = true,
    captionpos = b,
    keepspaces = true,
    numbers = left,
    numbersep = 5pt,
    showspaces = false,
    showstringspaces = false,
    showtabs = false,
    tabsize = 2
}

% ---------------------------------------------------------------- %
% https://en.wikibooks.org/wiki/LaTeX/Source_Code_Listings

\lstdefinestyle{customc}
{
    belowcaptionskip = 1 \baselineskip,
    breaklines = true,
    frame = L,
    xleftmargin = \parindent,
    language = C,
    showstringspaces = false,
    basicstyle = \footnotesize \ttfamily,
    keywordstyle = \bfseries \color{green!40!black},
    commentstyle = \itshape \color{purple!40!black},
    identifierstyle = \color{blue},
    stringstyle = \color{orange},
}

\lstdefinestyle{customasm}
{
    belowcaptionskip = 1 \baselineskip,
    frame = L,
    xleftmargin = \parindent,
    language = [x86masm] Assembler,
    basicstyle = \footnotesize\ttfamily,
    commentstyle = \itshape\color{purple!40!black},
}

% ---------------------------------------------------------------- %
% https://tex.stackexchange.com/questions/235731/listings-syntax-for-literate

\definecolor{maroon}        {cmyk}{0, 0.87, 0.68, 0.32}
\definecolor{halfgray}      {gray}{0.55}
\definecolor{ipython_frame} {RGB}{207, 207, 207}
\definecolor{ipython_bg}    {RGB}{247, 247, 247}
\definecolor{ipython_red}   {RGB}{186, 33, 33}
\definecolor{ipython_green} {RGB}{0, 128, 0}
\definecolor{ipython_cyan}  {RGB}{64, 128, 128}
\definecolor{ipython_purple}{RGB}{170, 34, 255}

\lstdefinestyle{stackexchangePython}
{
    breaklines = true,
    %
    extendedchars = true,
    literate =
    {á}{{\' a}} 1 {é}{{\' e}} 1 {í}{{\' i}} 1 {ó}{{\' o}} 1 {ú}{{\' u}} 1
    {Á}{{\' A}} 1 {É}{{\' E}} 1 {Í}{{\' I}} 1 {Ó}{{\' O}} 1 {Ú}{{\' U}} 1
    {à}{{\` a}} 1 {è}{{\` e}} 1 {ì}{{\` i}} 1 {ò}{{\` o}} 1 {ù}{{\` u}} 1
    {À}{{\` A}} 1 {È}{{\' E}} 1 {Ì}{{\` I}} 1 {Ò}{{\` O}} 1 {Ù}{{\` U}} 1
    {ä}{{\" a}} 1 {ë}{{\" e}} 1 {ï}{{\" i}} 1 {ö}{{\" o}} 1 {ü}{{\" u}} 1
    {Ä}{{\" A}} 1 {Ë}{{\" E}} 1 {Ï}{{\" I}} 1 {Ö}{{\" O}} 1 {Ü}{{\" U}} 1
    {â}{{\^ a}} 1 {ê}{{\^ e}} 1 {î}{{\^ i}} 1 {ô}{{\^ o}} 1 {û}{{\^ u}} 1
    {Â}{{\^ A}} 1 {Ê}{{\^ E}} 1 {Î}{{\^ I}} 1 {Ô}{{\^ O}} 1 {Û}{{\^ U}} 1
    {œ}{{\oe}}  1 {Œ}{{\OE}}  1 {æ}{{\ae}}  1 {Æ}{{\AE}}  1 {ß}{{\ss}}  1
    {ç}{{\c c}} 1 {Ç}{{\c C}} 1 {ø}{{\o}} 1 {å}{{\r a}} 1 {Å}{{\r A}} 1
    {€}{{\EUR}} 1 {£}{{\pounds}} 1
}


% Python definition (c) 1998 Michael Weber
% Additional definitions (2013) Alexis Dimitriadis
% modified by me (should not have empty lines)

\lstdefinelanguage{iPython}{
    morekeywords = {access, and, break, class, continue, def, del, elif, else, except, exec, finally, for, from, global, if, import, in, is, lambda, not, or, pass, print, raise, return, try, while}, %
    %
    % Built-ins
    morekeywords = [2]{abs, all, any, basestring, bin, bool, bytearray, callable, chr, classmethod, cmp, compile, complex, delattr, dict, dir, divmod, enumerate, eval, execfile, file, filter, float, format, frozenset, getattr, globals, hasattr, hash, help, hex, id, input, int, isinstance, issubclass, iter, len, list, locals, long, map, max, memoryview, min, next, object, oct, open, ord, pow, property, range, raw_input, reduce, reload, repr, reversed, round, set, setattr, slice, sorted, staticmethod, str, sum, super, tuple, type, unichr, unicode, vars, xrange, zip, apply, buffer, coerce, intern}, %
    %
    sensitive = true, %
    morecomment = [l] \#, %
    morestring = [b]', %
    morestring = [b]", %
    %
    morestring = [s]{'''}{'''}, % used for documentation text (mulitiline strings)
    morestring = [s]{"""}{"""}, % added by Philipp Matthias Hahn
    %
    morestring = [s]{r'}{'},     % `raw' strings
    morestring = [s]{r"}{"},     %
    morestring = [s]{r'''}{'''}, %
    morestring = [s]{r"""}{"""}, %
    morestring = [s]{u'}{'},     % unicode strings
    morestring = [s]{u"}{"},     %
    morestring = [s]{u'''}{'''}, %
    morestring = [s]{u"""}{"""}, %
    %
    % {replace}{replacement}{lenght of replace}
    % *{-}{-}{1} will not replace in comments and so on
    literate = 
    {á}{{\' a}} 1 {é}{{\' e}} 1 {í}{{\' i}} 1 {ó}{{\' o}} 1 {ú}{{\' u}} 1
    {Á}{{\' A}} 1 {É}{{\' E}} 1 {Í}{{\' I}} 1 {Ó}{{\' O}} 1 {Ú}{{\' U}} 1
    {à}{{\` a}} 1 {è}{{\` e}} 1 {ì}{{\` i}} 1 {ò}{{\` o}} 1 {ù}{{\` u}} 1
    {À}{{\` A}} 1 {È}{{\' E}} 1 {Ì}{{\` I}} 1 {Ò}{{\` O}} 1 {Ù}{{\` U}} 1
    {ä}{{\" a}} 1 {ë}{{\" e}} 1 {ï}{{\" i}} 1 {ö}{{\" o}} 1 {ü}{{\" u}} 1
    {Ä}{{\" A}} 1 {Ë}{{\" E}} 1 {Ï}{{\" I}} 1 {Ö}{{\" O}} 1 {Ü}{{\" U}} 1
    {â}{{\^ a}} 1 {ê}{{\^ e}} 1 {î}{{\^ i}} 1 {ô}{{\^ o}} 1 {û}{{\^ u}} 1
    {Â}{{\^ A}} 1 {Ê}{{\^ E}} 1 {Î}{{\^ I}} 1 {Ô}{{\^ O}} 1 {Û}{{\^ U}} 1
    {œ}{{\oe}}  1 {Œ}{{\OE}}  1 {æ}{{\ae}}  1 {Æ}{{\AE}}  1 {ß}{{\ss}}  1
    {ç}{{\c c}} 1 {Ç}{{\c C}} 1 {ø}{{\o}} 1 {å}{{\r a}} 1 {Å}{{\r A}} 1
    {€}{{\EUR}} 1 {£}{{\pounds}} 1
    %
    {^}{{{\color{ipython_purple}\^ {}}}} 1
    { = }{{{\color{ipython_purple} = }}} 1
    %
    {+}{{{\color{ipython_purple}+}}} 1
    {*}{{{\color{ipython_purple}$^\ast$}}} 1
    {/}{{{\color{ipython_purple}/}}} 1
    %
    {+=}{{{+=}}} 1
    {-=}{{{-=}}} 1
    {*=}{{{$^\ast$ = }}} 1
    {/=}{{{/=}}} 1,
    literate = 
    *{-}{{{\color{ipython_purple} -}}} 1
     {?}{{{\color{ipython_purple} ?}}} 1,
    %
    identifierstyle = \color{black}\ttfamily,
    commentstyle = \color{ipython_cyan}\ttfamily,
    stringstyle = \color{ipython_red}\ttfamily,
    keepspaces = true,
    showspaces = false,
    showstringspaces = false,
    %
    rulecolor = \color{ipython_frame},
    frame = single,
    frameround = {t}{t}{t}{t},
    framexleftmargin = 6mm,
    numbers = left,
    numberstyle = \tiny\color{halfgray},
    %
    %
    backgroundcolor = \color{ipython_bg},
    % extendedchars = true,
    basicstyle = \scriptsize,
    keywordstyle = \color{ipython_green}\ttfamily,
}

% ---------------------------------------------------------------- %
% https://tex.stackexchange.com/questions/417884/colour-r-code-to-match-knitr-theme-using-listings-minted-or-other

\geometry{verbose, tmargin = 2.5cm, bmargin = 2.5cm, lmargin = 2.5cm, rmargin = 2.5cm}

\definecolor{backgroundCol}  {rgb}{.97, .97, .97}
\definecolor{commentstyleCol}{rgb}{0.678, 0.584, 0.686}
\definecolor{keywordstyleCol}{rgb}{0.737, 0.353, 0.396}
\definecolor{stringstyleCol} {rgb}{0.192, 0.494, 0.8}
\definecolor{NumCol}         {rgb}{0.686, 0.059, 0.569}
\definecolor{basicstyleCol}  {rgb}{0.345, 0.345, 0.345}

\lstdefinestyle{stackexchangeR}
{
    language = R,                                        % the language of the code
    basicstyle = \small \ttfamily \color{basicstyleCol}, % the size of the fonts that are used for the code
    % numbers = left,                                      % where to put the line-numbers
    numberstyle = \color{green},                         % the style that is used for the line-numbers
    stepnumber = 1,                                      % the step between two line-numbers. If it is 1, each line will be numbered
    numbersep = 5pt,                                     % how far the line-numbers are from the code
    backgroundcolor = \color{backgroundCol},             % choose the background color. You must add \usepackage{color}
    showspaces = false,                                  % show spaces adding particular underscores
    showstringspaces = false,                            % underline spaces within strings
    showtabs = false,                                    % show tabs within strings adding particular underscores
    % frame = single,                                      % adds a frame around the code
    % rulecolor = \color{white},                           % if not set, the frame-color may be changed on line-breaks within not-black text (e.g. commens (green here))
    tabsize = 2,                                         % sets default tabsize to 2 spaces
    captionpos = b,                                      % sets the caption-position to bottom
    breaklines = true,                                   % sets automatic line breaking
    breakatwhitespace = false,                           % sets if automatic breaks should only happen at whitespace
    keywordstyle = \color{keywordstyleCol},              % keyword style
    commentstyle = \color{commentstyleCol},              % comment style
    stringstyle = \color{stringstyleCol},                % string literal style
    literate = %
    *{0}{{{\color{NumCol} 0}}} 1
     {1}{{{\color{NumCol} 1}}} 1
     {2}{{{\color{NumCol} 2}}} 1
     {3}{{{\color{NumCol} 3}}} 1
     {4}{{{\color{NumCol} 4}}} 1
     {5}{{{\color{NumCol} 5}}} 1
     {6}{{{\color{NumCol} 6}}} 1
     {7}{{{\color{NumCol} 7}}} 1
     {8}{{{\color{NumCol} 8}}} 1
     {9}{{{\color{NumCol} 9}}} 1
}

% ---------------------------------------------------------------- %
% Fundament Mathematik

\lstdefinestyle{fundament}{basicstyle = \ttfamily}

% ---------------------------------------------------------------- %


\usepackage{algorithm}

\parskip 0pt
\parindent 0pt

\title
{
  Logik und Grundlagen der Mathematik \\
  \vspace{4pt}
  \normalsize
  \textit{11. Übung am 17.12.2020}
}
\author
{
  Richard Weiss
  \and
  Florian Schager
  \and
  Fabian Zehetgruber
}
\date{}

\begin{document}

\maketitle

\section*{Berechenbare Funktionen und (semi-)entscheidbare Mengen}

Die Menge der $\mu$-rekursiven Funktionen ist die kleinste Menge von (möglicherweise partiellen)
Funktionen, die alle primitiv rekursiven Funktionen enthält, unter Komposition und
primitiver Rekursion abgeschlossen ist, und außerdem Folgendes erfüllt:

\begin{adjustwidth}{1cm}{}
Wenn $f: \N^k \times \N$ total und $\mu$-rekursiv ist, \\
dann ist die partielle Funktion $\vv{x} \mapsto \min\{y: f(\vv{x},y) = 0\}$ auch
$\mu$-rekursiv.
\end{adjustwidth}

% --------------------------------------------------------------------------------

\begin{exercise}[224]

\phantom{}

\end{exercise}

% --------------------------------------------------------------------------------

\begin{solution}

\phantom{}

\end{solution}

% --------------------------------------------------------------------------------

\begin{exercise}

\phantom{}
	Betrachten Sie die skalare Reaktions-Diffusionsgleichung
	\begin{align*}
		u_t = \Delta u + \lambda u - u^3 \quad \text{für } (x,t) \in \Omega \times (0, \infty)
	\end{align*}
	für $u(x,t) \in \R$ auf einem beschränkten Gebiet $\Omega \subseteq \R^n$ mit glattem Rand $\partial \Omega$ und einem negativen Parameter $\lambda$.
	\begin{enumerate}[label = (\roman*)]
		\item Bestimmen Sie die räumlich homogenen Lösungen $u = u(t)$ und untersuchen Sie deren asymptotisches Verhalten für $t \to \infty$. \newline
		\textit{Hinweis:} Die räumlich homogenen Lösungen erfüllen eine gewöhnliche DGl. Bestimmen Sie die Stationärzustände dieser DGl und deren Stabilität.

		\item Betrachten Sie das ARWP mit der Randbedingung
		\begin{align*}
			u(x,t) = 0 \quad \text{für } (x,t) \in \partial \Omega \times (0, \infty),
		\end{align*}
		und beschränkten Anfangsdaten
		\begin{align*}
			m \leq u(x,0) \leq M \quad \text{für }  x \in \Omega, t \geq 0,
		\end{align*}
		Zeigen Sie, dass klassische Lösungen $u(x,t)$ des ARWP und die räumlich homogenen Lösungen $\underline{u}(t)$ bzw. $\overline{u}(t)$ von dem ARWP mit Anfangsbedingungen $\underline{u}(0) = \min\{0, m\}$ bzw. $\overline{u}(0) = \max\{0, M\}$ die Ungleichung
		\begin{align*}
			\underline{u}(t) \leq u(x,t) \leq \overline{u}(t) \quad \text{für } x \in \Omega, t \geq 0,
		\end{align*}
		erfüllen.

		\item Was können Sie aus diesen Ungleichungen für das zeitlich asymptotische Verhalten von klassischen Lösungen $u(x,t)$ des ARWP schließen?
	\end{enumerate}
\end{exercise}

% --------------------------------------------------------------------------------

\begin{solution}

\phantom{}
	\begin{enumerate}[label = (\roman*)]
		\item Für eine räumlich homogene Lösung $u$ gilt $\Delta u = 0$ also
		\begin{align*}
			u_t = \lambda u - u^3.
		\end{align*}
		Diese ODE ist separabel, um eine Lösung zu erhalten berechnen wir also eine Stammfunktion
		\begin{align*}
			\Int{\frac{1}{u(\lambda - u^2)}}{u} &\stackrel{u^2 = v}{=} \Int{\frac{1}{2 v(\lambda - v)}}{v} = \frac{1}{2} \pbraces{\Int{\frac{1}{\lambda v}}{v} + \Int{\frac{1}{\lambda (\lambda - v)}}{v}} \\
			 &= \frac{1}{2\lambda} \pbraces{\ln(v) - \ln(v - \lambda)} = \frac{1}{2 \lambda} \ln\pbraces{\frac{u^2}{u^2 - \lambda}}
		\end{align*}
		Nun lösen wir
		\begin{align*}
			\frac{1}{2 \lambda} \ln\pbraces{\frac{u^2}{u^2 - \lambda}} = t + \tilde{C}
		\end{align*}
		nach $u$ auf und erhalten
		\begin{align*}
			u(t) = \pm \frac{\sqrt{- \lambda} \exp(\lambda t)}{\sqrt{\exp(2\lambda t) - C^{-1}}}
		\end{align*}
		mit $C := \exp (2 \lambda \tilde{C}) > 0$. Sind das alle Lösungen? \newline
		Wir fragen uns auch welche Ruhelagen das System hat und erkennen
		\begin{align*}
			\lambda u - u^3 = - u (u - \sqrt{\lambda}) (u + \sqrt{\lambda}).
		\end{align*}
		Da $\lambda < 0$, also $\sqrt{\lambda} \in \C \setminus \R$, gibt es also nur die Ruhelage Null.
		
		\includegraphicsboxed{ODEs/ODEs - Satz 5.8.png}

		Wegen
    dem Prinzip der linearisierten Stabilität und
    \begin{align*}
			\partial_u (\lambda u - u^3)\mid_{u = 0} = (\lambda - 3 u^2)\mid_{u = 0} = \lambda < 0
		\end{align*}
		ist diese Ruhelage sogar asymptotisch stabil. Es gilt sogar ($\lambda < 0$)
    \begin{align*}
	  \lim_{t \to \infty} u(t)
	  =
	  \pm \lim_{t \to \infty} \frac{\sqrt{- \lambda} \exp(\lambda t)}{\sqrt{\exp(2\lambda t) - C^{-1}}} = 0.
    \end{align*}
    Somit konvergiert die homogene Lösung für alle Startwerte gegen die
    asymptotische Ruhelage.
		\item Für ein beliebiges $T > 0$ gilt
		\begin{align*}
			\underline{u}_t - \Delta \underline{u} - \lambda \underline{u} + \underline{u}^3 = 0 = u_t - \Delta u - \lambda u + u^3 = 0 = \overline{u}_t - \Delta \overline{u} - \lambda \overline{u} + \overline{u}^3 \quad \text{in } \Omega \times (0, T].
		\end{align*}
		Außerdem gilt
		\begin{align*}
			\underline{u}(0) = \min\{0, m\} \leq m \leq u(x, 0) \leq M \leq \max\{0, M\} \leq \overline{u}(0) \quad \text{für jedes } x \in \Omega.
		\end{align*}
		Schließlich gilt
		\begin{align*}
			\underline{u}(0) = \min\{0, m\} \leq 0 \leq \max\{0, M\} = \overline{u}(0)
		\end{align*}
		$0$ ist eine Ruhelage entsprechender ODE also auch eine Lösung.
		Da sich die Lösungen ($\underline{u}$, $0$ und $\overline{u}$) nicht schneiden dürfen gilt schon
		\begin{align*}
			\underline{u}(t) \leq 0 = u(x, t) = 0 \leq \overline{u}(t) \quad \text{für jedes } (x,t) \in \partial\Omega \times [0, T)
		\end{align*}
		Nach der vorherigen Aufgabe 1 (ii) gilt ($L_1 := -\Delta$ und $f(x, t, u) := u^3 - \lambda u$)
		\begin{align*}
			\underline{u}(t) \leq u(x,t) \leq \overline{u}(t) \quad \text{für jedes } (x,t) \in \Omega \times (0, T]
		\end{align*}
		und da $T > 0$ beliebig war folgt die Aussage.
		\item Wir erkennen, dass eine klassische Lösung $u$, wegen des Einschluss-Satzes und der letzten Ungleichung aus (ii), für $t \to \infty$ gegen $0$ konvergiert.
	\end{enumerate}

\end{solution}

% --------------------------------------------------------------------------------


\section*{Berechenbare Funktionen auf Strings}

Im Folgenden sei $S$ die Menge aller Strings über einem festen Alphabet $A$.
Für $x \in S$ sei $|x|$ die Länge von $x$. \\
Um zu zeigen, dass eine Menge $A$ von Strings entscheidbar oder semi-entscheidbar
ist, geben Sie (informell) einen Algorithmus an, der $\chi_A$ bzw. $\tilde{\chi}_A$
berechnet.

\begin{algebraUE}{361}
Gib das Minimialpolynom von $\sqrt{2} + \sqrt{3}$ und $\sqrt{3} + i$ über $\Q$ an.
\end{algebraUE}

\begin{solution}
  Gib das Minimialpolynom von $\sqrt{2} + \sqrt{3}$ und $\sqrt{3} + i$ über $\Q$ an.

  \begin{itemize}
  \item Folgendes Polynome hat $\sqrt{2} + \sqrt{3}$ als Nullstelle:

  $m_1(x) := (x^2 - 5)^2 - 24 = x^4 - 10x^2 + 1 = (x + \sqrt{5+\sqrt{24}})(x + \sqrt{5-\sqrt{24}})(x - \sqrt{5+\sqrt{24}})(x - \sqrt{5-\sqrt{24}}).$

  \item Folgendes Polynom hat $\sqrt{3} + i$ als Nullstelle:

  $m_2(x) := (x^2 -2)^2 + 12 = x^4 - 4x^2 + 16 = (x - \sqrt{3} + i)(x - \sqrt{3} - i)(x + \sqrt{3} + i)(x + \sqrt{3} - i).$

  Hier können wir jeweils die beiden Faktoren, deren Nullstellen sich nur durch Konjugation unterscheiden, zu einem quadratischen Polynom über $\R$ zusammenfassen, welches dann klarerweise irreduzibel ist. In beiden Fällen erhält man Polynome, die nichtrationale Koeffizienten haben. Daher sind beide Polynome irreduzibel über $\Q$.
  \end{itemize}
\end{solution}


\section*{Unentscheidbare Mengen; universelle Mengen}

% --------------------------------------------------------------------------------

\begin{exercise}

\phantom{}

\end{exercise}

% --------------------------------------------------------------------------------

\begin{solution}

\phantom{}
Seien $u$ und $v$ zwei Lösungen. Nach Satz 5.20 hat das Poisson-Problem
\begin{align}\label{poisson}
\begin{cases}
\Delta w(t) = u(t) - v(t) &\text{in~} \Omega,\\
w(t) = 0 &\text{auf~} \partial\Omega
\end{cases}
\end{align}
eine eindeutige schwache Lösung $w(t) \in H_0^1(\Omega);$ insbesondere gilt
\begin{align*}
    \int_\Omega (u-v)(t) w \mathrm{~d}x = \int_\Omega \nabla w(t) \cdot \nabla w \mathrm{~d}x.
\end{align*}

Wir können $w$ nun als Testfunktion für die schwache Formulierung unseres ARW-Problems verwenden: Es gilt
\begin{align*}
    \int_0^t \int_\Omega gw \mathrm{~d}x \mathrm{~d}s
    &= \int_0^t \int_\Omega \left(u_t - \Delta(u^\alpha)\right) w \mathrm{~d}x \mathrm{~d}s \\
    &= \int_0^t \left(\int_\Omega u_t w \mathrm{~d}x - \int_\Omega \mathrm{div}\nabla(u^\alpha) w \mathrm{~d}x\right) \mathrm{d}s\\
    &= \int_0^t \left(\int_\Omega u_t w \mathrm{~d}x + \int_\Omega \nabla(u^\alpha) \cdot \nabla w \mathrm{~d}x - \int_{\partial\Omega} \underbrace{w}_{=~0}(\nabla(u^\alpha)\cdot\nu) \mathrm{~d}\mathcal{H}^{n-1}\right) \mathrm{d}s\\
    &= \int_0^t \int_\Omega u_t w + \nabla(u^\alpha) \cdot \nabla w \mathrm{~d}x \mathrm{~d}s.
\end{align*}

Dieselbe Gleichheit gilt natürlich auch für $v.$ Wenn wir beide Gleichungen voneinander subtrahieren, erhalten wir
\begin{align*}
    0 &= \int_0^t \int_\Omega u_t w + \nabla(u^\alpha) \cdot \nabla w \mathrm{~d}x \mathrm{~d}s - \int_0^t \int_\Omega v_t w + \nabla(v^\alpha) \cdot \nabla w \mathrm{~d}x \mathrm{~d}s\\
    &= \int_0^t \left(\int_\Omega (u-v)_t w \mathrm{~d}x - \int_\Omega \nabla(u^\alpha - v^\alpha) \cdot \nabla w \mathrm{~d}x\right) \mathrm{d}s\\
    &= \int_0^t \left(- \int_\Omega \nabla w_t \cdot \nabla w \mathrm{~d}x - \int_\Omega (u^\alpha - v^\alpha) \Delta w \mathrm{~d}x\right) \mathrm{d}s\\
    &= - \int_0^t \left(\int_\Omega \nabla w_t \cdot \nabla w \mathrm{~d}x + \int_\Omega (u^\alpha - v^\alpha) (u-v) \mathrm{~d}x\right)\mathrm{d}s.
\end{align*}

Die dabei auftretenden Randintegrale fallen weg, weil $w$ respektive $u^\alpha - v^\alpha$ auf $\partial\Omega$ verschwinden.

Weil für $\alpha > 0$ die Abbildung $x \mapsto x^\alpha$ monoton steigend ist, gilt stets $(u^\alpha - v^\alpha) (u-v) \geq 0.$ Des Weiteren ist $w \equiv 0$ für $t = 0$ die eindeutig bestimmte schwache Lösung von \eqref{poisson}, womit $\nabla w(\cdot, 0) = 0$ ist. Nun gilt
\begin{align*}
    0 &\geq - \int_0^t \int_\Omega (u^\alpha - v^\alpha) (u-v) \mathrm{~d}x \mathrm{~d}s\\
    &= \int_0^t \int_\Omega \nabla w_t \cdot \nabla w \mathrm{~d}x \mathrm{~d}s\\
    &\stackrel{(\ast)}{=} \int_0^t \frac{1}{2} ~\frac{\mathrm{d}}{\mathrm{d}s} \left\| \nabla w(\cdot, s)\right\|_{L^2} \mathrm{d}s\\
    &= \frac{1}{2} \left\| \nabla w(\cdot, t)\right\|_{L^2} - \frac{1}{2} \| \underbrace{\nabla w(\cdot, 0)}_{=0}\|_{L^2}\\
    &= \frac{1}{2} \left\| \nabla w(\cdot, t)\right\|_{L^2};
\end{align*}

$w(\cdot, t)$ ist also für alle $t > 0$ eine konstante Funktion. Damit gilt $0 = \Delta w(t) = u(t) - v(t),$ was zu beweisen war.

Zeigen wir noch die Gleichheit ($\ast$): Für ein $u \in C^\infty(\Omega)$ ist
\begin{align*}
    \|u(t_0)\|^2_{L^2} + 2 \int_{t_0}^{t} (u_t(s), u(s))_{L^2} \mathrm{~d}s
    &= \|u(t_0)\|^2_{L^2} + \int_\Omega 2 \int_{t_0}^{t} u_t u \mathrm{~d}s \mathrm{~d}x\\
    &= \|u(t_1)\|^2_{L^2} + \int_\Omega u^2(x, t) - u^2(x, t_0) \mathrm{~d}x\\
    &= \| u(t)\|_{L^2}.
\end{align*}
Wenn wir nun beide Seiten nach $t$ differenzieren, erhalten wir
\begin{align*}
    \frac{\mathrm{d}}{\mathrm{d}t} \|u(t)\|_{L^2} = 2 \left(u_t(t), u(t)\right)_{L^2}.
\end{align*}
Aus der Dichtheit von $C^\infty(\Omega)$ in $L^2(\Omega)$ folgt die gewünschte Gleichheit für alle $u \in L^2(\Omega),$ insbesondere also für $\nabla w(\cdot, s).$
\end{solution}

% --------------------------------------------------------------------------------


\section*{Wohlordnungen}

Eine strikte lineare Ordnung $(A, <)$ (mit der zugehörigen reflexiven Ordnung $\leq$)
heißt Wohlordnung, wenn jede nicht-leere Teilmenge von $A$ ein kleinstes Element hat:
$\forall B \subseteq A: (B \neq \emptyset \Rightarrow \Exists b \in B\, \Forall x \in B: b \leq x)$

% -------------------------------------------------------------------------------- %

\begin{exercise}[\textbf{Comparing groups}]

    Health professionals warn that transmission of infectious diseases may occur during the
    traditional handshake greeting. Two alternative methods of greeting (popularized in sports)
    are the high five and the fist bump. Researchers compared the hygiene of these alternative
    greetings in a designed study and reported the results in the American Journal of Infection
    Control (Aug. 2014). A sterile-gloved hand was dipped into a culture of bacteria, then made
    contact for three seconds with another sterile-gloved hand via either a handshake, high five,
    or fist bump. The researchers then counted the number of bacteria present on the second,
    recipient, gloved hand. This experiment was replicated five times for each contact method.
    Simulated data (recorded as a percentage relative to the mean of the handshake), based on
    information provided by the journal article, are provided in the table.

    \begin{center}
        \begin{tabular}{c c c c c c}
            Handshake: & 131 & 74 & 129 & 96 & 92 \\
            High five: & 44 & 70 & 69 & 43 & 53 \\
            Fist bump: & 15 & 14 & 21 & 29 & 21
        \end{tabular}
    \end{center}


    \begin{enumerate}[label = (\alph*)]
        \item The researchers reported that more bacteria were transferred
        during a handshake compared with a high five. Use a 95\% confidence interval to support this statement statistically.
        \item The researchers also reported that the fist bump gave a lower transmission
        of bacteria than the high five. Use a 95\% confidence interval to support this statement statistically.
        \item Based on the results, parts (a) and (b), which greeting method would you
        recommend as being the most hygienic?
    \end{enumerate}

\end{exercise}

% -------------------------------------------------------------------------------- %

\begin{solution}

\phantom{}

\begin{enumerate}[label = (\alph*)]
    \item Again we look at the population of differences. We calculate the mean and sd:
    
    \begin{align*}
        \bar{d} &= 48.6 \\
        s_d &\approx 30.43518.
    \end{align*}

    The confidence interval for the two-sample hypothesis test then reads

    \begin{align*}
        48.6 \pm 1.96 \cdot \frac{30.43518}{\sqrt{5}} \approx 48.6 \pm 26.6776.
    \end{align*}

    Since $0 < 48.6 - 26.6776$ we can safely support this statement.
    \item

    \begin{align*}
        \bar{d} &= 35.8 \\
        s_d &\approx 16.52876.
    \end{align*}

    The confidence interval for the two-sample hypothesis test then reads

    \begin{align*}
        35.8 \pm 1.96 \cdot \frac{16.52876}{\sqrt{5}} \approx 35.8 \pm 14.4881.
    \end{align*}

    Since $0 < 35.8 - 14.4881$ we can safely support this statement.

    \item I don't trust the scientific methods of researchers who think that
    making contact with another hand for three seconds during a fist bump or high five
    is an appropriate depiction of reality.

    Furthermore, no physical contact at all is certainly the most hygienic option.
\end{enumerate}

\end{solution}

% -------------------------------------------------------------------------------- %

\begin{algebraUE}{382}
  Sei $L := \Q(x)$ der Körper der gebrochen rationalen Funktionen über $\Q$.
  \begin{itemize}
      \item Berechnen Sie $[L:K]$ für $K:= \Q(x^3) \leq L$, indem Sie das Minimalpolynom von $x$ über $K$ finden.
      \item Wie Teil 1, nur mit $K := \Q(x+\frac{1}{x}).$
      \item Z. z.: $[L:K] < \infty$ für jeden Körper $K := \Q(\alpha)$ mit $\alpha \in \Q(x) \backslash \Q.$
  \end{itemize}
\end{algebraUE}

\begin{solution}
\leavevmode \\
\begin{itemize}
  \item   Wir suchen zuerst ein Polynom $m \in \Q(x^3)[y]$ mit $m(x) = 0$.
    \begin{align*}
        m(y) = y^3 - x^3.
    \end{align*}
    Wir zeigen nun, dass $m$ auch irreduzibel über $\Q(x^3)$ ist. Angenommen $m$
    hätte eine Nullstelle  $\frac{p(x^3)}{q(x^3)} \in \Q(x^3)$. Dann gilt
    \begin{align*}
      \frac{p(x^3)^3}{q(x^3)^3} = x^3
      \iff \sum_{k=0}^np_kx^{3k} = p(x^3) = q(x^3)x = \sum_{k=0}^mq_kx^{3k+1}.
    \end{align*}
    Dies ist aufgrund der linearen Unabhängigkeit der Familie $(x^k)_{k \in \N}$
    nur in trivialer Weise möglich, was ein Widerspruch zu $q(x^3) \neq 0$ ist.
    Also ist $m$ irreduzibel über $\Q(x)$ und somit das Minimalpolynom von $x$.
  \item \begin{align*}
    m(y) = y^2 - (x + \frac{1}{x})y + 1
  \end{align*}
  erfüllt $m(x) = 0$. Um zu zeigen, dass $m$ das gewünschte Minimalpolynom ist,
  zeigen wir, dass $x$ nicht in $\Q(x + \frac{1}{x})$ liegt und somit keine
  Nullstelle eines linearen Polynoms sein kann. Angenommen dem wäre so, dann gäbe
  es $p(x+\frac{1}{x}),q(x+\frac{1}{x}) \in \Q[x+\frac{1}{x}]$ mit
  \begin{align*}
    &\frac{p(x+\frac{1}{x})}{q(x+\frac{1}{x})} = x \\
    &\iff p(x+\frac{1}{x}) = q(x + \frac{1}{x})x \\
    &\iff \sum_{k=0}^np_k\frac{(x^2+1)^k}{x^k} = \sum_{k=0}^nq_k\frac{(x^2+1)^k}{x^{k-1}} \\
    &\iff \sum_{k=0}^np_k(x^2+1)^kx^{n-k} = \sum_{k=0}^nq_k(x^2+1)^kx^{n-k+1} \\
    &\iff \sum_{k=0}^np_k\sum_{j=0}^k\binom{k}{j}x^{2(k-j)}x^{n-k} = \sum_{k=0}^nq_k\sum_{j=0}^k\binom{k}{j}x^{2(k-j)}x^{n-k+1} \\
    &\iff \sum_{k=0}^np_k\sum_{j=0}^k\binom{k}{j}x^{n+k-2j}= \sum_{k=0}^nq_k\sum_{j=0}^k\binom{k}{j}x^{n+k-2j+1} \\
  \end{align*}
  Jetzt steh ich da an.
  \item Man kann die gebrochen rationale Funktion mit dem Nenner multiplizieren
  und dann den Zähler abziehen und erhält mit dieser Vorschrift stets ein Polynom,
  dass $x$ als Nullstelle hat:
  Sei $\alpha = \frac{\sum_{i=1}^n a_i x^i}{\sum_{j=1}^m b_j x^j}$. Das Polynom
  \begin{align}
  m_\alpha(y) := \alpha \left(\sum_{j=1}^m b_j y^j\right) - \sum_{i=1}^n a_i y^i
  \end{align}

  erfüllt $m_\alpha(x) = 0.$
\end{itemize}



\end{solution}

% -------------------------------------------------------------------------------- %

\begin{exercise}[240]

Definieren Sie die lexikographische Ordnung auf $\{0,1\}^{\N}$. Gibt es ein kleinstes Element?
Zeigen Sie, dass diese Ordnung eine lineare Ordnung aber keine Wohlordnung ist.

\end{exercise}

% -------------------------------------------------------------------------------- %

\begin{solution}
	
	Wir definieren die lexikographische Ordnung

\begin{align*}
  (x_n)_{n\in\N} <_{\mathrm{lex}} (y_n)_{n\in\N}
  :\iff \exists n_0 \in \N: x_{n_0} < y_{n_0} \land \forall n < n_0: x_n = y_n
\end{align*}

	und rechnen nachk, dass es sich um eine lineare Ordnung, aber keine Wohlordnung handelt.
	
	\begin{enumerate}
		\item Irreflexivität: $\forall z \in \{0,1\}^\N: z \nless_{\mathrm{lex}} z$. Für $(x_n)_{n \in \N} \in \{0,1\}^\N$ gilt klarerweise für alle $k \in \N$, dass $x_k = x_k$ und daher ist $(x_n)_{n \in \N} \nless_{\mathrm{lex}} (x_n)_{n \in \N}$. 
		
		\item Transitivität: $\forall u,v,w \in \{0,1\}^\N: ((u < v \land v < w) \Rightarrow u < w)$. Seien $(x_n)_{n \in \N}, (y_n)_{n \in \N}, (z_n)_{n \in \N} \in \{0,1\}^\N$ mit $(x_n)_{n \in \N} <_{\mathrm{lex}} (y_n)_{n \in \N}$ und dem Zeugen $k \in \N$ sowie  $(y_n)_{n \in \N} <_{\mathrm{lex}} (z_n)_{n \in \N}$ mit dem Zeugen $l$. Wir definieren $m := \min\{k,l\}$. Es gilt
			\begin{align*}
				\forall i < m: x_i = y_i = z_i  \quad \text{und} \quad x_m < z_m
			\end{align*}
		und damit $(x_n)_{n \in \N} <_{\mathrm{lex}} (z_n)_{n \in \N}$.
		
		\item Trichotomie: $\forall u,v \in \{0,1\}^\N: (u < v \lor u = v \lor u > v)$. Seien also $(x_n)_{n \in \N}, (y_n)_{n \in \N} \in \{0,1\}^\N$ mit $(x_n)_{n \in \N} \nless_{\mathrm{lex}} (y_n)_{n \in \N}$ und $(x_n)_{n \in \N} \neq (y_n)_{n \in \N}$. Weil die beiden Folgen nicht gleich sind können wir 
		\begin{align*}
			k := \min\{i \in \N \mid x_i \neq y_i\}
		\end{align*}
		definieren. Es gilt also
			\begin{align*}
				\forall i < k: x_i = y_i \quad \text{und} \quad x_k \neq y_k
			\end{align*}
			und wegen $x_k \nless y_k$ gilt $x_k > y_k$ und daher $(x_n)_{n \in \N} >_{\mathrm{lex}} (y_n)_{n \in \N}$.
			
		\item keine Wohlordnung: $\exists C \subseteq \{0,1\}^\N: (C \neq \emptyset \land \forall z \in C \Exists x \in C: x <_{\mathrm{lex}} z)$. Das kleinste Element der Menge $\{0,1\}^\N$ ist $x^* :=(0,0,\dots)$. Das heißt wir müssen uns eine andere Menge suchen.
		Die Teilmenge $C := \{0,1\}^{\N} \setminus \{x^*\}$ hat kein kleinstes Element.
		Angenommen sie hätte ein kleinstes Element $x^{\prime}$, dann folgt aus
		$x^{\prime} \neq x^*$
		\begin{align*}
		\exists n_0 \in \N: x^{\prime}_{n_0} = 1
		\end{align*}
		und wir finden mit $(x^{\primeprime}_n)_{n \in \N} = (\delta_{n_0+1,n})_{n \in \N}$ ein echt kleineres
		Element. Ein Widerspruch.
	\end{enumerate}


\end{solution}


\section*{Bijektionen}

Beachten Sie, dass wir $f(x)$ für den Funktionswert von $f$ an der Stelle $x$ schreiben.
Für $U \subseteq \dom(f)$ nennen wir die Menge $\{f(u): u \in U\}$ nicht $f(U)$
sondern $f[U]$.

% -------------------------------------------------------------------------------- %

\begin{exercise}[243]

Seien $f: A \to B$ und $g: B \to A$ injektiv. Der Einfachheit halber seien
$A$ und $B$ disjunkt. Zeigen Sie, dass es eine Bijektion $h: A \to B$ gibt,
indem Sie den folgenden Beweis vervollständigen: \\
Wir definieren $A_0 := A, B_0 := B, A_{n+1} := g[B_n], B_{n+1} := f[A_n]$. \\
Sei $X_1 := \bigcup_{k\in\N}A_{2k}\setminus A_{2k+1},
X_2 := \bigcup_{k\in\N}A_{2k+1}\setminus A_{2k+2}, X_3 := \bigcap_{k \in \N} A_k$ \\
und $Y_1 := \bigcup_{k\in\N}B_{2k}\setminus B_{2k+1},
Y_2 := \bigcup_{k\in\N}B_{2k+1}\setminus B_{2k+2}, Y_3 := \bigcap_{k \in \N} B_k$.

\begin{enumerate}[label = \alph*.]
  \item Zeigen Sie, dass $\{X_1,X_2,X_3\}$ eine Partition von $A$ ist.
  \item Definieren Sie $h: A \to B$ mit einer Fallunterscheidung:
  Für $x \in X_1$ verwenden Sie $f$, um $h(x)$ zu definieren, für $x \in X_2$
  hingegen $g$. Und für $x \in X_3?$
  \item Zeigen Sie, dass die so definierte Funktion wohldefiniert ist,
  und überdies eine Bijektion.
\end{enumerate}

\end{exercise}

% -------------------------------------------------------------------------------- %

\begin{solution}

\phantom{}
\begin{enumerate}[label = \alph*.]
  \item Wir zeigen zuerst mit Induktion $\forall k \in \N: A_{k+2} \subseteq A_{k}$.
  \begin{align*}
    k = 0&: \quad A_2 = g[B_1] = g[f[A_0]] = g[f[A]] \subseteq g[B] \subseteq A = A_0 \\
    (k-1) \rightsquigarrow k&: \quad
    A_{k+2} = g[f[A_{k}]] \subseteq g[f[A_{k-2}]] = A_{k}.
  \end{align*}

  Als nächstes zeigen wir, dass $X_1,X_2,X_3$ paarweise disjunkt sind:
  Sei $x \in X_1 \cap X_2$: \\
  Dann existieren $k_1, k_2$, sodass $x \in A_{2k_1}\setminus A_{2k_1+1}$, sowie
  $x \in A_{2k_2 + 1}\setminus A_{2k_2+2}$. \\
  Fall 1: $k_1 \leq k_2$: Dann ist $x \in A_{2k_2 + 1} \subseteq A_{2k_1 + 1}$. Widerspruch! \\
  Fall 2: $k_1 > k_2$: Dann ist $x \in A_{2k_1} \subseteq A_{2k_2 + 2}$. Widerspruch! \\
  Die Disjunktheit von $X_3$ mit $X_1$ und $X_2$ ist offensichtlich. \\
  Ganz analog sieht man auch, dass $\{Y_1,Y_2,Y_3\}$ eine Partition von $B$ sein muss.

  Schließlich zeigen wir noch $X_1 \cup X_2 \cup X_3 = A$. \\
  Dazu sei $x \in A \setminus X_3$ und $k_0 \geq 1$ das kleinste $k \in \N$ sodass
  $x \notin A_k$. Dann gilt $x \in A_{k-1} \setminus A_k \subset X_1 \cup X_2$. \\
  Die andere Inklusion folgt aus $A_k \subseteq A, k \in \N$.

  \item
  Für Fall $x \in X_3$ können wir uns aussuchen, ob wir $f$ oder $g^{-1}$
  zur Definition hernehmen, wir entscheiden uns mal für $f$.
  \begin{align*}
    h(x) = \begin{cases}
      f(x) & x \in X_1 \\
      g^{-1}(x) & x \in X_2 \\
      f(x) & x \in X_3
    \end{cases}
  \end{align*}

  \item Die Wohldefiniertheit ist nur für $x \in X_2$ auf den ersten Blick fraglich,
  da aber $X_2 \subset A_1 = g[B]$ gilt, folgt sie sofort aus der Injektivität von $g$. \\
  Weiters gilt aufgrund der Injektivität von $f$
  \begin{align*}
    f(X_1) &= f\left[\bigcup_{k\in\N}A_{2k}\setminus A_{2k+1}\right]
    = \bigcup_{k\in\N}f\left[A_{2k}\setminus A_{2k+1}\right]
    = \bigcup_{k\in\N}f\left[A_{2k}\right]\setminus f[A_{2k+1}]
    = \bigcup_{k\in\N}B_{2k+1}\setminus B_{2k+2} = Y_2, \\
    g(Y_1) &= g\left[\bigcup_{k\in\N}B_{2k}\setminus B_{2k+1}\right]
    = \bigcup_{k\in\N}g\left[B_{2k}\setminus B_{2k+1}\right]
    = \bigcup_{k\in\N}g\left[B_{2k}] \setminus g[B_{2k+1}\right]
    = \bigcup_{k\in\N}A_{2k+1}\setminus A_{2k+2} = X_2, \\
    f(X_3) &= f\left[\bigcap_{k \in \N} A_k\right]
    = \bigcap_{k \in \N}f\left[A_k\right]
    = \bigcap_{k \in \N}B_{k+1} = Y_3, \\
    g(Y_3) &= g\left[\bigcap_{k \in \N} B_k\right]
    = \bigcap_{k \in \N}g\left[A_k\right]
    = \bigcap_{k \in \N}A_{k+1} = X_3.
  \end{align*}
  Also sehen wir, dass $f: X_1 \to Y_2$ und $g: X_2 \to Y_1$ Bijektionen sind,
  sowie $f: X_3 \to Y_3$ und $g: Y_3 \to X_3$, also können wir $h$ für $x \in X_3$
  tatsächlich auf beide Arten definieren und erhalten in jedem Fall eine Bijektion
  von $A$ nach $B$.
\end{enumerate}


\end{solution}


\end{document}
