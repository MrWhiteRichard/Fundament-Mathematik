% --------------------------------------------------------------------------------

\begin{exercise}[237]
	Wenn $(A, <)$ eine Wohlordnung ist, in der jede nichtleere Teilmenge ein größtes Element hat, dann ist $A$ endlich.

\end{exercise}

% --------------------------------------------------------------------------------

\begin{solution}
	Wir definieren $A_0 := A$ und für jedes $n \in \N$ die Menge $A_{n + 1} := A_n \setminus \{\min A_n\}$. Wir nehmen an, dass $A$ eine unendliche Menge ist und definieren die Funktion
	\begin{align*}
		f:\N \to A: n \mapsto \min A_n
	\end{align*}
	Nun bezeichnen wir mit $c \in A$ das größte Element von $f[\N] \subseteq A$. Sei $k \in \N$ ein Element mit $f(k) = c$. Es gilt
	\begin{align*}
		c = f(k) = \min A_k < \min A_{k + 1} = f(k + 1).
	\end{align*}
	Ein Widerspruch dazu, dass $c$ das größte Element in $f[\N]$ ist.

\end{solution}
