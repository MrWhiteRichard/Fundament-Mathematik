% -------------------------------------------------------------------------------- %

\begin{exercise}[270]

Sei $g: M \to M$ injektiv, aber nicht surjektiv. Geben Sie (in Abhängigkeit von $g$)
explizit eine injektive Abbildung $f: \N \to M$ an.
(Genauer: Geben Sie eine explizite Familie $(f_a: a \in I)$ von solchen Abbildungen
an, mit $I \neq \emptyset$.)

\end{exercise}

% -------------------------------------------------------------------------------- %

\begin{solution}

Weil $g$ nicht surjektiv ist, ist $I := M \setminus g[M] \neq \emptyset$.
Sei also $a \in I$.

\begin{align*}
    f_a:
    \N \to M:
    n
    \mapsto
	\begin{cases}
		a,             & \text{falls}~ n = 0, \\
		g(f_a(n - 1)), & \text{falls}~ n > 0 
	\end{cases}
\end{align*}

Wir wollen nun folgende Aussage, d.h. die Injektivität von $f_a$, mit Induktion nach $n$ beweisen.

\begin{align*}
    \Forall n, k \in \N:
    (f_a(k) = f_a(n) \implies k = n)
\end{align*}

IA ($n = 0$):
Sei $k \in \N$ mit $f_a(0) = f_a(n) = f_a(k)$.
Angenommen, $k \neq 0$, d.h. $k > 0$.

\begin{align*}
    \implies
    m = f_a(0) = f_a(k) = g(f_a(k - 1))
    \implies
    m \in g[M]
\end{align*}

Widerspruch!

IS ($n \mapsto n + 1$):
Für $n \in \N$ gelte bereits

\begin{align*}
    \Forall k \in \N:
    (f_a(n) = f_a(k) \implies n = k).
\end{align*}

Sei $k \in \N$ mit $f_a(n + 1) = f_a(k)$.
Wäre $k = 0$, dann würde wegen $n + 1 > 0 = k$, mit dem IA ($k = 0$) ein Widerspruch folgen.

\begin{align*}
    \implies
	g(f_a(n)) = f_a(n + 1) = f_a(k) = g(f_a(k - 1))
\end{align*}

Wegen der Injektivität von $g$, folgt $f_a(n) = f_a(k - 1)$.
Wegen der Induktionsvoraussetzung, folgt außerdem $n = k - 1$ und daher $n + 1 = k$. 

\end{solution}
