% -------------------------------------------------------------------------------- %

\begin{exercise}[109]

Skizzieren Sie einen (halb-)formalen Beweis einer der Formeln aus den vorigen
beiden Beispielen.
\end{exercise}

% -------------------------------------------------------------------------------- %

\begin{solution}

Wir skizzieren den Beweis für
$[\forall x A(x)] \land [\forall x B(x)] \rightarrow \forall x[A(x) \land B(x)]$: \\
Wir zeigen dazu $\{\forall x A(x), \forall x B(x)\} \vdash \forall x[A(x) \land B(x)]$
und erhalten einen Beweis für die eigentliche Formel durch zweimaliges Anwenden
des Deduktionstheorem und einer geeigneten Tautologie \eqref{eq:taut6}.
\begin{flalign*}
  1&: \forall x A(x), \forall x B(x) \vdash (\forall x A(x)) \rightarrow A(x) & \text{(Substitution)} \\
  2&: \forall x A(x), \forall x B(x) \vdash (\forall x B(x)) \rightarrow B(x) & \text{(Substitution)} \\
  3&: \forall x A(x), \forall x B(x) \vdash \forall x A(x) & \text{(Axiom)} \\
  4&: \forall x A(x), \forall x B(x) \vdash \forall x B(x) & \text{(Axiom)} \\
  5&: \forall x A(x), \forall x B(x) \vdash A(x) & \text{(MP)} \\
  6&: \forall x A(x), \forall x B(x) \vdash B(x) & \text{(MP)} \\
  7&: \forall x A(x), \forall x B(x) \vdash A(x) \rightarrow B(x) \rightarrow A(x) \land B(x)
  & \text{(Tautologie \eqref{eq:taut7})} \\
  8&: \forall x A(x), \forall x B(x) \vdash B(x) \rightarrow A(x) \land B(x) & \text{(MP)} \\
  9&: \forall x A(x), \forall x B(x) \vdash A(x) \land B(x) & \text{(MP)} \\
  27&: \forall x A(x), \forall x B(x) \vdash \forall x[A(x) \land B(x)] & \text{(Generalisierungstheorem)} \\
\end{flalign*}
Noch mehr Tautologien:
\begin{align}
(p \rightarrow (q \rightarrow r)) \rightarrow ((p \land q) \rightarrow r) \label{eq:taut6}\\
p \rightarrow (q \rightarrow (p \land q)) \label{eq:taut7}
\end{align}
\end{solution}

% -------------------------------------------------------------------------------- %
