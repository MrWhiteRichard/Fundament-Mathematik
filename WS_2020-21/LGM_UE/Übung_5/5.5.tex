% --------------------------------------------------------------------------------

\begin{exercise}[113]

Sei $\mathscr{L}$ eine Sprache der Prädikatenlogik, $\Sigma$ eine vollständige
konsistente Theorie in $\mathscr{L}$. Zeigen Sie für alle geschlossenen Formeln
$A$ und $B$:
\begin{itemize}
  \item $\Sigma \vdash A \lor B$ genau dann, wenn $\Sigma \vdash A$ oder $\Sigma \vdash B$.
  \item $\Sigma \vdash A \land B$ genau dann, wenn $\Sigma \vdash A$ und $\Sigma \vdash B$.
  \item $\Sigma \vdash \neg A$ genau dann, wenn $\Sigma \nvdash A$.
\end{itemize}
\end{exercise}

% --------------------------------------------------------------------------------

\begin{solution}

\phantom{}
\begin{itemize}
  \item Gelte $\Sigma \vdash A \lor B$, $\Sigma \nvdash A$ und $\Sigma \nvdash B$:\\
  Aus der Vollständigkeit von $\Sigma$ erhalten wir dadurch $\Sigma \vdash \neg A$
  und $\Sigma \vdash \neg B$. Mit der Tautologie $\neg A \rightarrow \neg B \rightarrow \neg(A \lor B)$
  erhalten wir damit auch
  $\Sigma \vdash \neg(A \lor B)$, was einen Widerspruch zur Konsistenz von $\Sigma$ liefert. \\
  Gelte umgekehrt $\Sigma \vdash A$. Dann folgt mit der Tautologie $A \rightarrow A \lor B$
  auch $\Sigma \vdash A \lor B$. Analoges gilt unter der Voraussetzung $\Sigma \vdash B$.
  \item Gelte $\Sigma \vdash A \land B$ und $\Sigma \nvdash A$.
  Hier passiert nichts Neues, wieder finden wir eine Tautologie $\neg A \rightarrow \neg(A \land B)$
  und somit einen Widerspruch zu Konsistenz. Analog für $\Sigma \nvdash B$. \\
  Gelte umgekehrt $\Sigma \vdash A $ und $\Sigma \vdash B$. Jetzt finden wir wieder
  eine Tautologie mit $A \rightarrow B \rightarrow A \land B$ und erhalten mit
  Modus Ponens $\Sigma \vdash A \land B$.
  \item Folgt unmittelbar aus Konsistenz und Vollständigkeit.
\end{itemize}

\end{solution}

% --------------------------------------------------------------------------------
