% --------------------------------------------------------------------------------

\begin{exercise}[25]

Sei $\Sigma \cup \Bbraces{A}$ eine Menge von aussagenlogischen Formeln.
Zeigen Sie:

\begin{enumerate}
    \item $\Sigma$ ist genau dann erfüllbar, wenn zumindest eine der Mengen $\Sigma \cup \Bbraces{A}$, $\Sigma \cup \Bbraces{\neg A}$ erfüllbar ist.
    \item $\Sigma$ ist genau dann $\ast$erfüllbar, wenn zumindest eine der Mengen $\Sigma \cup \Bbraces{A}$, $\Sigma \cup \Bbraces{\neg A}$ $\ast$erfüllbar ist.
\end{enumerate}

\end{exercise}

% --------------------------------------------------------------------------------

\begin{solution}
\phantom{}
\begin{enumerate}
	\item 
	\begin{enumerate}
		\item[`$\Rightarrow$']  Sei $\Sigma$ erfüllbar und $b$ eine Belegung aller in $\Sigma \cup \{A\}$ vorkommenden Variablen, sodass $\hat{b}(\Sigma) = 1$ gilt. 
		\begin{enumerate}[label = Fall \arabic*:]
			\item $\hat{b}(A) = 1$. Dann ist $\hat{b}(\Sigma \cup \{A\}) =1$.
			\item $\hat{b}(A) = 0$. Dann ist $\hat{b}(\neg A) = 1$ und damit $\hat{b}(\Sigma \cup \{\neg A\}) =1$.
		\end{enumerate}
		\item[`$\Leftarrow$'] Sei $\Sigma \cup \{A\}$ oder $\Sigma \cup \{\neg A\}$ erfüllbar. In jedem Fall ist natürlich $\Sigma$ als Teilmenge ebenfalls erfüllbar.
	\end{enumerate}
	\item Wir bemerken, dass aus der Erfüllbarkeit auch die *Erfüllbarkeit folgt. Nun werfen wir einen Blick auf die Lösung von Aufgabe 26 und erkennen, dass auch die umgekehrte Implikation gilt. Die Menge $\Sigma$ ist also *erfüllbar genau dann wenn sie erfüllbar ist. Nach dem vorherigen Punkt ist das genau dann der Fall, wenn $\Sigma \cup \{A\}$ oder $\Sigma \cup \{\neg A\}$ erfüllbar ist.
\end{enumerate}

\end{solution}

% --------------------------------------------------------------------------------
