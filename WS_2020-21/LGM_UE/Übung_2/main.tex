\documentclass{article}

% Hier befinden sich Pakete, die wir beinahe immer benutzen ...

\usepackage[utf8]{inputenc}

% Sprach-Paket:
\usepackage[ngerman]{babel}

% damit's nicht so, wie beim Grill aussieht:
\usepackage{fullpage}

% Mathematik:
\usepackage{amsmath, amssymb, amsfonts, amsthm}
\usepackage{bbm, mathrsfs, stmaryrd}
\usepackage{mathtools, mathdots}

% Makros mit mehereren Default-Argumenten:
\usepackage{twoopt}

% Anführungszeichen (Makro \Quote{}):
\usepackage{babel}

% if's für Makros:
\usepackage{xifthen}
\usepackage{etoolbox}

% tikz ist kein Zeichenprogramm (doch!):
\usepackage{tikz}

% bessere Aufzählungen:
\usepackage{enumitem}

% (bessere) Umgebung für Bilder:
\usepackage{graphicx, subfig, float}

% Umgebung für Code:
\usepackage{listings}

% Farben:
\usepackage{xcolor}

% Umgebung für "plain text":
\usepackage{verbatim}

% Umgebung für mehrerer Spalten:
\usepackage{multicol}

% "nette" Brüche
\usepackage{nicefrac}

% Spaltentypen verschiedener Dicke
\usepackage{tabularx}
\usepackage{makecell}

% Für Vektoren
\usepackage{esvect}

% (Web-)Links
\usepackage{hyperref}

% Zitieren & Literatur-Verzeichnis
\usepackage[style = authoryear]{biblatex}
\usepackage{csquotes}

% so ähnlich wie mathbb
%\usepackage{mathds}

% Keine Ahnung, was das macht ...
\usepackage{booktabs}
\usepackage{ngerman}
\usepackage{placeins}

% special letters:

\newcommand{\N}{\mathbb{N}}
\newcommand{\Z}{\mathbb{Z}}
\newcommand{\Q}{\mathbb{Q}}
\newcommand{\R}{\mathbb{R}}
\newcommand{\C}{\mathbb{C}}
\newcommand{\K}{\mathbb{K}}
\newcommand{\T}{\mathbb{T}}
\newcommand{\E}{\mathbb{E}}
\newcommand{\V}{\mathbb{V}}
\renewcommand{\P}{\mathbb{P}}
\newcommand{\1}{\mathbbm{1}}

\newcommand  {\B}{\mathfrak{B}}
\renewcommand{\S}{\mathfrak{S}}

% quantors:

\newcommand{\Forall}{\forall \,}
\newcommand{\Exists}{\exists \,}
\newcommand{\ExistsOnlyOne}{\exists! \,}
\newcommand{\nExists}{\nexists \,}

% MISC symbols:

\newcommand{\landau}[1]
{
  {\scriptstyle \mathcal{O}}
  \pbraces{#1}
}

\newcommand{\Landau}[1]
{
  \mathcal{O}
  \pbraces{#1}
}

\newcommand{\eps}{\mathrm{eps}}

% graphics in a box:

\newcommandtwoopt
{\includegraphicsboxed}[3][][]
{
  \begin{figure}[!h]
    \begin{boxedin}
      \ifthenelse{\isempty{#2}}
      {
        \begin{center}
          \includegraphics[width = 0.75 \textwidth]{#3}
          \label{fig:#1}
        \end{center}
      }{
        \begin{center}
          \includegraphics[width = 0.75 \textwidth]{#3}
          \caption{#2}
          \label{fig:#1}
        \end{center}
      }
    \end{boxedin}
  \end{figure}
}

% braces:

\newcommand{\pbraces}[1]{{\left  ( #1 \right  )}}
\newcommand{\bbraces}[1]{{\left  [ #1 \right  ]}}
\newcommand{\Bbraces}[1]{{\left \{ #1 \right \}}}
\newcommand{\vbraces}[1]{{\left  | #1 \right  |}}
\newcommand{\Vbraces}[1]{{\left \| #1 \right \|}}
\newcommand{\abraces}[1]{{\left \langle #1 \right \rangle}}
\newcommand{\round}[1]{\bbraces{#1}}

\newcommand
{\floor}[1]
{{\left \lfloor #1 \right \rfloor}}

\newcommand
{\ceil} [1]
{{\left \lceil  #1 \right \rceil }}

% special functions:

\newcommand{\norm}  [2][]{\Vbraces{#2}_{#1}}
\newcommand{\diag}  [1]{\mathrm{diag} \: #1}
\newcommand{\dist}  [1]{\mathrm{dist} \: #1}
\newcommand{\mean}  [1]{\mathrm{mean} \: #1}
\newcommand{\erf}   [1]{\mathrm{erf} \: #1}
\newcommand{\id}    [1]{\mathrm{id} \: #1}
\newcommand{\sgn}   [1]{\mathrm{sgn} \: #1}
\newcommand{\supp}  [1]{\mathrm{supp} \: #1}
\newcommand{\arsinh}[1]{\mathrm{arsinh} \: #1}
\newcommand{\arcosh}[1]{\mathrm{arcosh} \: #1}
\newcommand{\artanh}[1]{\mathrm{artanh} \: #1}
\newcommand{\card}  [1]{\mathrm{card} \: #1}
\newcommand{\Span}  [1]{\mathrm{span} \: #1}
\newcommand{\Aut}   [1]{\mathrm{Aut} \: #1}
\newcommand{\End}   [1]{\mathrm{End} \: #1}
\newcommand{\ggT}   [1]{\mathrm{ggT} \: #1}
\newcommand{\kgV}   [1]{\mathrm{kgV} \: #1}
\newcommand{\ord}   [1]{\mathrm{ord} \: #1}
\newcommand{\grad}  [1]{\mathrm{grad} \: #1}
\newcommand{\ran}   [1]{\mathrm{ran} \: #1}
\newcommand{\graph} [1]{\mathrm{graph} \: #1}
\newcommand{\Inv}   [1]{\mathrm{Inv} \: #1}
\newcommand{\pv}    [1]{\mathrm{pv} \: #1}
\newcommand{\Mod}{\: \mathrm{mod} \:}
\newcommand{\Char}{\mathrm{char}}
\newcommand{\At}{\mathrm{At}}
\newcommand{\Ob}{\mathrm{Ob}}
\newcommand{\Hom}{\mathrm{Hom}}
\newcommand{\orthogonal}[3][]{#2 ~\bot_{#1}~ #3}
\newcommand{\Rang}{\mathrm{Rang}}

\newcommand
{\GL}[2][]
{\mathrm{GL}_{#1} \pbraces{#2}}

% fractions:

\newcommand{\Frac}[2]{\frac{1}{#1} \pbraces{#2}}
\newcommand{\nfrac}[2]{\nicefrac{#1}{#2}}

% derivatives & integrals:

\newcommandtwoopt
{\Int}[4][][]
{\int_{#1}^{#2} #3 ~\mathrm{d} #4}

\newcommandtwoopt
{\derivative}[3][][]
{
  \frac
  {\mathrm{d}^{#1} #2}
  {\mathrm{d} #3^{#1}}
}

\newcommandtwoopt
{\pderivative}[3][][]
{
  \frac
  {\partial^{#1} #2}
  {\partial #3^{#1}}
}

\newcommand
{\primeprime}
{{\prime \prime}}

\newcommand
{\primeprimeprime}
{{\prime \prime \prime}}

% Text:

\newcommand{\Quote}[1]{\glqq #1\grqq{}}
\newcommand{\Text}[1]{{\text{#1}}}
\newcommand{\fastueberall}{\text{f.ü.}}
\newcommand{\fastsicher}{\text{f.s.}}

% -------------------------------- %
% amsthm-stuff:

\theoremstyle{definition}

% numbered theorems
\newtheorem{theorem}    {Satz}   [section]
\newtheorem{lemma}      [theorem]{Lemma}
\newtheorem{corollary}  [theorem]{Korollar}
\newtheorem{proposition}[theorem]{Proposition}
\newtheorem{remark}     [theorem]{Bemerkung}
\newtheorem{definition} [theorem]{Definition}
\newtheorem{example}    [theorem]{Beispiel}

% unnumbered theorems
\newtheorem*{theorem*}    {Satz}
\newtheorem*{lemma*}      {Lemma}
\newtheorem*{corollary*}  {Korollar}
\newtheorem*{proposition*}{Proposition}
\newtheorem*{remark*}     {Bemerkung}
\newtheorem*{definition*} {Definition}
\newtheorem*{example*}    {Beispiel}

% Please define this stuff in project ("main.tex"):

% \def \lastexercisenumber {...}
% This will be 0 by default

% \setcounter{section}{...}
% This will be 0 by default
% and hence, completely ignored

\ifnum \thesection = 0
{
  \newtheorem{exercise}{Aufgabe}
}
\else
{
  \newtheorem{exercise}{Aufgabe}[section]
}
\fi

\ifdef
{\lastexercisenumber}
{\setcounter{exercise}{\lastexercisenumber}}

\newenvironment{solution}
{
  \begin{proof}[Lösung]
}{
  \end{proof}
}

\renewcommand{\proofname}{Beweis}

% -------------------------------- %
% environment zum einkasteln:

% dickere vertical lines
\newcolumntype
{x}
[1]
{
  !{
    \centering
    \arraybackslash
    \vrule
    width #1}
}

% environment selbst (the big cheese)
\newenvironment
{boxedin}
{
  \begin{tabular}
  {
    x{1 pt}
    p{\textwidth}
    x{1 pt}
  }
  \Xhline
  {2 \arrayrulewidth}
}
{
  \\
  \Xhline{2 \arrayrulewidth}
  \end{tabular}
}

% -------------------------------- %
% MISC "Ein-Deutschungen"

\renewcommand{\figurename}{Abbildung}
\renewcommand{\tablename} {Tabelle}

% -------------------------------- %

\input{../../../Fundament-LaTeX/listings.tex}

\parskip 0pt
\parindent 0pt

\title
{
  Logik und Grundlagen der Mathematik \\
  \vspace{4pt}
  \normalsize
  \textit{2. Übung am 15.10.2020}
}
\author
{
  Richard Weiss
  \and
  Florian Schager
  % \and
  % Christian Sallinger
  \and
  Fabian Zehetgruber
  % \and
  % Paul Winkler
  % \and
  % Christian Göth
}
\date{}

\begin{document}

\maketitle

Eine Menge $\Sigma$ von aussagenlogischen Formeln heißt \blockquote{erfüllbar}, wenn es eine Belegung oder in $\Sigma$ vorkommenden Variablen gibt, die für alle $A \in \Sigma$ die Bedingung $\hat{b}(A) = 1$ erfüllt.
Wir nennen eine Menge $\Sigma$ $\ast$erfüllbar, wenn jede endliche Teilmenge von $\Sigma$ erfüllbar ist.

% -------------------------------------------------------------------------------- %

\begin{exercise}

Sei $L = \Bbraces{w c^n \mid w \in \Bbraces{a, b}^\ast, n_a(w) = n ~\text{oder}~ n_b(w) = n}$.
Geben Sie eine kontextfreie Grammatik an die $L$ erzeugt.

\end{exercise}

% -------------------------------------------------------------------------------- %

\begin{solution}

Die fehlende Quantisierung für $n$ lässt 2 zulässige Interpretationen von $L$ zu.

\begin{enumerate}

    \item Interpretation ($L = L_n := \Bbraces{w c^n \mid w \in \Bbraces{a, b}^\ast, n_a(w) = n ~\text{oder}~ n_b(w) = n}$, $n \in \N$):
    
    An diese habe ich ursprünglich gedacht.

    Zwecks besseren Verständnisses, listen wir die Elemente von $L$ für verschiedene $n \in \N$ auf.
    Dabei bezeichnet $x \cdot x = x^m$ für irgendein beliebiges $m \in \N$.
    Wenn der Ausdruck mehrmals vorkommt, dürfen die $m$ auch unterschiedlich sein.
    
    \begin{itemize}
    
        \item $n = 0$:
        
        \begin{align*}
            a \cdots a \\
            b \cdots b \\
        \end{align*}
    
        \item $n = 1$:
        
        \begin{align*}
            (a \cdots a) b (a \cdots a) c \\
            (b \cdots b) a (b \cdots b) c \\
        \end{align*}
    
        \item $n = 2$:
        
        \begin{align*}
            (a \cdots a) b (a \cdots a) b (a \cdots a) c^2 \\
            (b \cdots b) a (b \cdots b) a (b \cdots b) c^2 \\
        \end{align*}
    
        \item $n \in \N$:
        
        \begin{align*}
            \underbrace
            {
                ((a \cdots a) b)
                \cdots
                ((a \cdots a) b)
            }_{
                \text{$n$-mal}
            }
            (a \cdots a)
            c^n \\
            \underbrace
            {
                ((b \cdots b) a)
                \cdots
                ((b \cdots b) a)
            }_{
                \text{$n$-mal}
            }
            (b \cdots b)
            c^n \\
        \end{align*}
    \end{itemize}
    
    Wir wollen nun all diese Wörter systematisch aufbauen, mit einem endlichem Regelwerk $P$.
    Sei dabei aber o.B.d.A. $n \neq 0$.
    
    \begin{align*}
        P
        =
        \begin{cases}
            S \to X^n C^n \mid Y^n C^n, \\
            X \to X B \mid B X \mid A, \\
            Y \to Y A \mid A Y \mid B, \\
            A \to a, \\
            B \to b, \\
            C \to c            
        \end{cases}
    \end{align*}
    
    Die kontextfreie Grammatik $G = \abraces{N, T, P, S}$ erzeugt $L$.
    
    \begin{align*}
        N = \Bbraces{S, X, Y, A, B, C},
        \quad
        T = \Bbraces{a, b, c}
    \end{align*}

    \item Interpretation ($L = \bigcup_{n \in \N} L_n$):
    
    Man wäre möglicherweise nun dazu geneigt, Satz 2.2 zu verwenden; d.h. die Abschlusseigenschaft kontextfreier Sprachen unter Vereinigung.
    Diese wurde aber nur für endliche Vereinigungen bewiesen.
    Wir gehen also zu Fuß mit $G = \abraces{N, T, P, S}$, wobei

    \begin{align*}
        N = \Bbraces{S, A, B},
        \quad
        T = \Bbraces{a, b},
        \quad
        P
        =
        \begin{cases}
            S \to A \mid B, \\
            A \to a A c \mid b A \mid \varepsilon, \\
            B \to b B c \mid a B \mid \varepsilon.
        \end{cases}
    \end{align*}

\end{enumerate}

Die missverständliche Formulierung der Angabe kostet zwar etwas Zeit, wirft aber gleichzeitig die Frage auf, wann unendliche Vereinigeungen (oder vielleicht sogar Grenzwerte) von kontextfreien Sprachen wieder kontextfrei sind.

\end{solution}

% -------------------------------------------------------------------------------- %

% --------------------------------------------------------------------------------

\begin{exercise}

\phantom{}

\begin{enumerate}[label = (\roman*)]

    \item Lösen Sie das Randwertproblem für die Laplacegleichung in ebenen Polarkoordinaten
    
    \begin{align*}
        (\Delta u)(r, \varphi)
        & =
        0 ~\text{für}~ r < R, \\
        u(R, \varphi)
        & =
        f(\varphi) ~\text{für alle}~ \varphi,
    \end{align*}

    wobei

    \begin{align*}
        \Delta u
        =
        u_{rr} + \frac{1}{r} u + \frac{1}{r^2} u_{\varphi \varphi}
    \end{align*}

    der Laplaceoperator in Polarkoordinaten, $R > 0$ eine positive Konstante und $f$ eine stückweise stetig differenzierbare $2 \pi$-periodische Funktion ist?

    \item Wie sieht die Lösung konkret im Fall
    
    \begin{align*}
        f(\varphi)
        =
        \begin{cases}
             0 & \varphi = 0, \pi \\
             1 & 0 < \varphi < \pi \\
            -1 & \pi < \varphi < 2 \pi
        \end{cases}
    \end{align*}

    mit $R = 1$ aus?

\end{enumerate}

\textit{Hinweis:}
Verwenden Sie einen Separationsansatz.
Betrachten Sie dazu zunächst Einzellösungen $u_n$ der Gestalt $u_n(r, \varphi) = v_n(r) \cdot w_n(\varphi)$ (mit $w_n$ $2 \pi$-periodisch), welche die Differentialgleichung erfüllen und insbesondere $C^2$ im Nullpunkt sind.
Die gesuchte Gesamtlösung ergibt sich dann als Summe über die Einzellösungen $u_n$ mit geeigneten Koeffizienten.
Falls Sie dabei auf die homogene eulersche Differentialgleichung 2. Ordnung stoßen, verwenden Sie Aufgabe 6 von Blatt 1 oder schlagen sie in einer beliebigen Quelle ein Fundamentalsystem von Lösungen nach.

\end{exercise}

% --------------------------------------------------------------------------------

\begin{comment}

\begin{solution}

\phantom{}

\begin{enumerate}[label = (\roman*)]

    \item Weil $f$ stückweise stetig differenzierbar und $2 \pi$-periodisch ist, hat es die Darstellung

    \begin{align*}
        f = \sum_{n=1}^N \1_{A_n} f_n,
        \quad
        N \in \N,
        \quad
        \bigcup_{n=1}^N A_n = [0, 2 \pi),
        \quad
        \Forall n = 1, \ldots, N:
        f_n |_{A_n} \in C^1.
    \end{align*}

    Wir folgen dem Hinweis, und betrachten die PDE einmal nur auf einem $A_n$ mit $n = 1, \ldots, N$.
    Sei $\varphi \in (0, 2 \pi]$ fest gewählt.
    Die Randbedingung und Separation geben uns

    \begin{align*}
        f_n(\varphi) = u_n(R, \varphi) = v_n(R) w_n(\varphi)
        \stackrel{!}{\implies}
        f_n^\primeprime(\varphi) = v_n(R) w_n^\primeprime(\varphi).
    \end{align*}

    Die Bedingung an den Laplaceoperator führt andererseits zu

    \begin{align*}
        0 \stackrel{!}{=}
        \Delta u_n
        =
        \pderivative[2]{r} (v_n w_n)
        +
        \frac{1}{r} \pderivative{r} (v_n w_n)
        +
        \frac{1}{r^2} \pderivative[2]{\varphi} (v_n w_n)
        =
        v_n^\primeprime w_n
        +
        \frac{1}{r} v_n^\prime w_n
        +
        \frac{1}{r^2} v_n w_n^\primeprime \\
    \end{align*}

    Jetzt multiplizieren wir mit $r^2 v_n(R)$ und \Quote{stoßen} auf eine \Quote{eulersche Differentialgleichung 2. Ordnung}.

    \begin{align*}
        \implies
        0 =
        r^2 v_n^\primeprime v_n(R) w_n
        +
        r v_n^\prime v_n(R) w_n
        +
        v_n v_n(R) w_n^\primeprime
        =
        f_n(\varphi) r^2 v_n^\primeprime
        +
        f_n(\varphi) r v_n^\prime
        +
        f_n^\primeprime(\varphi) r v_n
    \end{align*}

    Unsere \Quote{beliebige Quelle} an der Stelle ist (NATÜRLICH, was sonst?) Wikipedia!

    \begin{figure}[h!]
        \centering
        \includegraphics[width = \textwidth]{Euler Wikipedia.png}
        \caption{Click \href{https://de.wikipedia.org/wiki/Eulersche_Differentialgleichung}{here}!}
    \end{figure}

    ACHTUNG: Die Nullstellen hängen formal immer noch von $\varphi$ ab!

    \begin{align*}
        \lambda_n(\varphi)
        =
        \frac{f_n(\varphi) - f_n(\varphi)}{2 f_n(\varphi)}
        \pm
        \sqrt
        {
            \frac{(f_n(\varphi) - f_n(\varphi))^2}{4 f_n(\varphi)^2}
            -
            \frac{f_n^\primeprime(\varphi)}{f_n(\varphi)}
        }
        =
        \sqrt{-\frac{f_n^\primeprime(\varphi)}{f_n(\varphi)}}
    \end{align*}

    Wir machen die Fallunterscheidungen und bekommen ein Fundamentalsystem $V_{n, \varphi, 1}, V_{n, \varphi, 2}$.
    Wir wählen zwei beliebige Koeffizienten $c_{n, 1}, c_{n, 2} \in \R$ an.
    (Die hängen NICHT von $\varphi$ ab.)

    \begin{align*}
        v_{n, \varphi, c_{n, 1}, c_{n, 2}} = c_{n, 1} V_{n, \varphi, 1} + c_{n, 2} V_{n, \varphi, 2}
        & \implies
        w_{n, c_{n, 1}, c_{n, 2}}(\varphi) = \frac{f_n(\varphi)}{v_{n, \varphi, c_{n, 1}, c_{n, 2}}(R)} \\
        & \implies
        u_{n, \varphi, c_{n, 1}, c_{n, 2}}
        =
        v_{n, \varphi, c_{n, 1}, c_{n, 2}}
        w_{n, c_{n, 1}, c_{n, 2}}(\varphi)
        =
        \frac
        {
            f_n(\varphi)
            v_{n, \varphi, c_{n, 1}, c_{n, 2}}
        }{
            v_{n, \varphi, c_{n, 1}, c_{n, 2}}(R)
        }
    \end{align*}

    Diese Einzellösungen setzen wir nun zusammen.

    \begin{align*}
        u_c(r, \varphi)
        =
        \sum_{n=1}^N \1_{A_n} u_{n, \varphi, c_{n, 1}, c_{n, 2}},
        \quad
        c = ((c_{n, 1}, c_{n, 2}))_{n=1}^N
    \end{align*}

    \item 

    \begin{align*}
        f = \1_{(0, \varphi)} - \1_{(\pi, 2 \pi)}
        \implies
        \lambda_n = 0
        \implies
        V_{n, 1} = 1, V_{n, 2} = \ln{r}
        \implies
        v_n = c_{n, 1} + c_{n, 2} \ln
    \end{align*}

    \begin{align*}
        v_n(R)
        =
        c_{n, 1} + \underbrace{c_{n, 2} \ln{R}}_0
        \implies
    \end{align*}

    \begin{enumerate}

        \item \Quote{$\varphi = 0, \pi$}:
        \begin{align*}
            w_1(\varphi)
            =
            \frac{f_1(\varphi)}{v_1(R)} = 0
            \implies
            u_1 = 0
        \end{align*}

        \item \Quote{$0 < \varphi < \pi$}:
        \begin{align*}
            w_2(\varphi)
            =
            \frac{f_2(\varphi)}{v_2(R)}
            =
            \frac{1}{c_{1, 2}}
            \implies
            u_2(r, \varphi)
            =
            \frac{v_2(r)}{c_{1, 2}}
            =
            \frac{c_{1, 2} + c_{2, 2} \ln{r}}{c_{1, 2}}
            =
            1 + \underbrace{\frac{c_{2, 2}}{c_{1, 2}}}_{=: \eta} \ln{r}
        \end{align*}

        \item \Quote{$\pi < \varphi < 2 \pi$}:
        \begin{align*}
            w_3(\varphi)
            =
            \frac{f_3(\varphi)}{v_3(R)}
            =
            \frac{-1}{c_{1, 3}}
            \implies
            u_3(r, \varphi)
            =
            -\frac{v_3(r)}{c_{1, 3}}
            =
            -\frac{c_{1, 3} + c_{2, 3} \ln{r}}{c_{1, 3}}
            =
            -1 \underbrace{-\frac{c_{2, 3}}{c_{1, 3}}}_{=: \zeta} \ln{r}
        \end{align*}

    \end{enumerate}

    \begin{align*}
        \implies
        u(r, \varphi)
        =
        \begin{cases}
            0,                & \varphi = 0, \pi, \\
            \eta  \ln{r} + 1, & 0 < \varphi < \pi, \\
            \zeta \ln{r} - 1, & \pi < \varphi < 2 \pi
        \end{cases}
    \end{align*}

\end{enumerate}

\end{solution}

\end{comment}

% --------------------------------------------------------------------------------

\begin{solution}

\phantom{}

\begin{enumerate}[label = (\roman*)]

    \item Wir verwenden den Hinweis und machen den Ansatz $u(r, \varphi) = v(\varphi) w(\varphi)$. Einsetzen in die Differentialgleichung liefert

    \begin{align*}
    v_{rr}(r)w(\varphi) + \frac{1}{r} v_r(r)w(\varphi) + \frac{1}{r^2} v(r)w_{\varphi\varphi}(\varphi) = 0 \Leftrightarrow \frac{r^2v_{rr}(r) + rv_r(r)}{v(r)} = -\frac{w_{\varphi\varphi}(\varphi)}{w(\varphi)} = \lambda
    \end{align*}
    
    mit einer Konstante $\lambda$, da die linke Seite nur von $r$ und die rechte Seite nur von $\varphi$ abhängt. Wir erhalten also die Differentialgleichung 
    
    \begin{align*}
        w_{\varphi\varphi} + \lambda w = 0 \quad \text{mit dem charakteristischen Polynom} \quad \chi(\mu) = \mu^2 + \lambda.
    \end{align*}
    
    Den Fall $\lambda = 0$ behandeln wir extra, ansonsten erhalten wir die Lösung 
    
    \begin{align*}
        w(\varphi) = c_1 \exp\left(i \varphi \sqrt{\lambda} \right) + c_2 \exp\left(-i \varphi \sqrt{\lambda} \right).
    \end{align*}

    Nach dem Hinweis soll diese Lösung $2\pi$-periodisch sein, also $\sqrt{\lambda} \in \Z$ und damit $\lambda = n^2$ mit $n \in \N$. Da wir nun wissen, dass $\lambda > 0$ ist können wir ein anderes Fundamentalsystem wählen und die Lösung $w_n(\varphi) = c_{n,1} \sin(n\varphi) + c_{n,2} \cos(n\varphi)$ anschreiben.    
    Als zweite Differentialgleichung erhalten wir die homogene eulersche Differentialgeichung 

    \begin{align*}
        a_2 r^2 v_{rr} + a_1 r v_r + a_0 v = 0 \quad \text{mit} \quad a_2 = 1, \quad a_1 = 1 \quad \text{und} \quad a_0 = -\lambda
    \end{align*}

    Wie wir solch eine Differentialgleichung lösen wissen wir schon von Aufgabe 6 auf Blatt 1. Wir definieren 

    \begin{align*}
        \mu_\pm := \frac{a_2 - a_1 \pm \sqrt{(a_2 - a_1)^2 - 4a_2a_0}}{2a_2} = \pm \sqrt{\lambda}  
    \end{align*}

    und erhalten unter Berücksichtigung von $\lambda = n^2$ mit $n \in \N$ die Lösung

    \begin{align*}
        v_n(r) = c_{n,3} r^{\mu_+} + c_{n,4} r^{\mu_-} = c_{n,3} r^{n} + c_{n,4} r^{-n}.
    \end{align*}

    Schauen wir nocheinmal in den Hinweis so sehen wir, dass zweimal stetige Differenzierbarkeit besonders im Nullpunkt wichtig ist, dort ist allerdings $v_n$ nur definiert, falls $c_{n,4} = 0$ erfüllt ist. Damit gilt $v_n(r) = c_{n,3} r^{n}$.
    Insgesamt erhalten wir also 

    \begin{align*}
        u_n(r,\varphi) = v_n(r) w_n(\varphi) = r^n \left(b_{n,1} \sin(n\varphi) + b_{n,2} \cos(n\varphi)\right)
    \end{align*}

    und gemäß hinweis die Summe

    \begin{align*}
        u(r, \varphi) = \sum_{n \in \N} r^n \left(b_{n,1} \sin(n\varphi) + b_{n,2} \cos(n\varphi)\right)
    \end{align*}
    
    als Gesamtlösung.

    \item 
    
    \begin{align*}
        u(R, \varphi)
        =
        \sum_{n \in \N}
        \underbrace{R^n}_1
        \pbraces
        {
            b_{n, 1} \sin(n\varphi)
            +
            b_{n, 2} \cos(n\varphi
        }
    \end{align*}

    Wir berechnen also die Fourierkoeffizienten $(b_{n, 1}, b_{n, 2})_{n \in \N}$ von $f$.
    Dazu, betrachten wir folgende Abbildung \ref{fig:FCS}.

    \includegraphicsboxed[FCS][Fourier-Cheat-Sheet]{Fourier-Cheat-Sheet.png}

    Nachdem $f$ ungerade ist, ist $(b_{1, n})_{n \in \N} = 0$, und wir müssen uns nur um die $\sin$-Koeffizienten kümmern.
    $\Forall n \in \N:$

    \begin{align*}
        b_{2, n}
        & =
        \frac{1}{\pi}
        \Bigg (
            \Int[0][\pi]
            {
                \underbrace{f(\varphi)}_1
                \sin{(n \varphi)}
            }{\varphi}
            +
            \Int[\pi][2 \pi]
            {
                \underbrace{f(\varphi)}_{-1}
                \sin(n \varphi)
            }{\varphi}
        \Bigg )
        =
        \frac{1}{\pi}
        \pbraces
        {
            -\frac{1}{n}
            \cos(n \varphi)
            \big |_{\varphi = 0}^\pi
            +
            \frac{1}{n}
            \cos(n \varphi)
            \big |_{\varphi = \pi}^{2 \pi}
        } \\
        & =
        \frac{1}{n \pi}
        \pbraces
        {
            1 - \cos{n \pi}
            +
            \cos{2n \pi} - \cos{n \pi}
        }
        =
        \frac{2}{n \pi}
        \pbraces
        {
            1 - (-1)^n
        }
        =
        \begin{cases}
            0,               & n \in 2 \N, \\
            \frac{4}{n \pi}, & n \in 2 \N - 1
        \end{cases}
    \end{align*}

\end{enumerate}



\end{solution}

% --------------------------------------------------------------------------------

% --------------------------------------------------------------------------------

\begin{exercise}

Sei $I \subseteq \R$ ein offenes Intervall um den Nullpunkt und $g, f \in C^1(\R)$.
Betrachten Sie das Cauchyproblem

\begin{align*}
    u_t + g(u) u_x = 0,
    \quad
    u(0, x) = f(x)
\end{align*}

für $(t, x) \in I \times \R$.

\begin{enumerate}[label = (\roman*)]

    \item Bestimmen Sie die Charakteristiken und überprüfen Sie, ob die Voraussetzungen für den Existenzsatz 2.3 erfüllt sind.

    \item Bestimmen Sie eine Lösung für die Burgers-Gleichung

    \begin{align*}
        u_t + u u_x = 0,
        \quad
        (t, x) \in \R^2
    \end{align*}

    mit den Anfangsdaten $u(0, x) = -x$ und geben Sie den Definitionsbereich der Lösung an.
    Skizzieren Sie die Charakteristiken und die Lösung zu verschiedenen Zeitpunkten.

\end{enumerate}

\end{exercise}

% --------------------------------------------------------------------------------

\begin{solution}

\phantom{}

\begin{enumerate}[label = (\roman*)]

	\item Unsere Differentialgleichung hat die Form

	\begin{gather*}
		\overline{t}(y) = 0,
		\overline{x}(y) = y,
		\overline{u}(y) = f(y), \\
		\Gamma = \Bbraces
		{
			(\overline{t}(y), \overline{x}(y)):
			y \in \R
		},
		\quad
		S = \Bbraces
		{
			(\overline{t}(y), \overline{x}(y), \overline{u}(y)):
			y \in \R
		}, \\
		a(t, x, u) u_t + b(t, x, u)u_x = c(t, x, u), \\
		a(t,x,u) = 1, \quad b(t,x,u) = g(u), \quad c(t,x,u) = 0.
	\end{gather*}

	Wir haben gelernt wie man so eine Differentialgleichung mit der Charakteristikenmethode löst. Dafür lösen wir als erstes das System an Differentialgleichungen

	\begin{align*}
		\pderivative[][t]{s} = a(t,x,u) = 1, \quad \pderivative[][x]{s} = b(t,x,u) = g(u), \quad \pderivative[][u]{s} = c(t,x,u) = 0.
	\end{align*}

	Aus der Nebenbedingung ergeben sich für die Lösungen noch zusätzlich die Bedingungen

	\begin{align*}
		t(0, y) = \overline{t}(y) = 0,
		\quad
		x(0, y) = \overline{x}(y) = y,
		\quad,
		u(0, y) = \overline{u}(y) = f(y).
	\end{align*}

	Die erste und letzte ODE löst man mit Integration nach $s$.
	Die Integrationskonstante ergibt sich aus dem Anfangswert.

	\begin{align*}
		\implies
		t(s,y) = s,
		\quad
		u(s,y) = f(y)
	\end{align*}

	Damit, und einer weiteren Integration (einer $s$-Konstanten), bekommt man die Lösung der zweiten ODE.

	\begin{align*}
		\pderivative[][x]{s}(s,y) = g(u(s,y)) = g(f(y))
		\implies
		x(s, y) = s \cdot g(f(y)) + y.
	\end{align*}

	Für festes $y$ ist also eine Charakteristik gegeben durch

	\begin{align*}
		(t^{(y)}(s), x^{(y)}(s), u^{(y)}(s))
		=
		(s, s \cdot g(f(y)) + y, f(y)).
	\end{align*}

	Sind die Voraussetzungen von Satz 2.3 erfüllt? Wir rechnen nach.

	\begin{align*}
		\det \pderivative[][(t,x)]{(s,y)} = \det
		\begin{pmatrix}
			t_s(0,y) & t_y(0,y) \\
			x_s(0,y) & x_y(0,y)
		\end{pmatrix}
		=
		\det
		\begin{pmatrix}
			a(t, x, u) & \overline{t}_y(y) \\
			b(t, x, u) & \overline{x}_y(y)
		\end{pmatrix}
		=
		\det
		\begin{pmatrix}
			1 & 0 \\
			g(f(y)) & 1
		\end{pmatrix}
		= 1 \neq 0
	\end{align*}

	Und wir sehen, dass die Voraussetzungen des Satzes erfüllt sind, es gibt also lokal eine Lösung.

	\item Nun haben wir es mit einem Spezialfall des bisher Betrachteten zu tun, nämlich mit $g(u) = u$ und $f(x) = -x$. Wir nützen die Ergebnisse vom vorigen Teil und erhalten

	\begin{align*}
		t(s,y) & = s, \\
		x(s,y) & = s \cdot g(f(y)) + y = s \cdot g(-y) + y = -sy + y = y(1 - s), \\
		u(s,y) & = f(y) = -y
	\end{align*}

	Es ergibt sich also

	\begin{align*}
		t = s
		& \iff
		s = t \\
		x = y(1 - s) = y(1 - t)
		& \iff
		y = \frac{x}{1 - t}
	\end{align*}

	und damit als Lösung

	\begin{align*}
		u(t,x) = u(s(t,x), y(t,x)) = y(t,x) = \frac{x}{t - 1},
	\end{align*}

	eine auf $(\R \setminus \Bbraces{1}) \times \R$ definierte Funktion.
	Plots dazu findet man in den Abbildungen \ref{fig:lsg_u_char} und \ref{fig:lsg_u_graph}.

	\begin{figure}[h!]
		\centering
		\includegraphics[width = 0.75 \textwidth]{2-3-1.png}
		\caption{Charakteristiken von $u$ im $\R^2$}
		\label{fig:lsg_u_char}
	\end{figure}

	\begin{figure}[h!]
		\centering
		\includegraphics[width = \textwidth]{2-3-2.png}
		\caption{Graph von $u$ im $\R^3$}
		\label{fig:lsg_u_graph}
	\end{figure}
\end{enumerate}

\end{solution}

% --------------------------------------------------------------------------------

% --------------------------------------------------------------------------------

\begin{exercise}

$f: [a, b] \to \R$ heißt Lipschitz-stetig, wenn es eine konstante $M$ gibt, sodass für $a \leq s \leq t \leq b$ $|f(t) - f(s)| \leq M (t - s)$ gilt.
Zeigen Sie:

\begin{enumerate}[label = (\alph*)]
    \item $f$ ist absolutstetig,
    \item $g = \derivative[][\mu_f]{\lambda}$ erfüllt $|g| \leq M$ fast überall.
\end{enumerate}

Insgesamt ist $f$ genau dann Lipschitz-stetig, wenn

\begin{align*}
    f(x)
    =
    \Int[\bbraces{a, x}]{g}{\lambda}
\end{align*}

mit einer beschränkten maessbaren Funktion $g$.

\end{exercise}

% --------------------------------------------------------------------------------

\begin{solution}

\phantom{}

\begin{enumerate}[label = (\alph*)]

    \item

    \begin{align*}
        \Forall \epsilon > 0:
            \Exists \delta > 0:
                \Forall a \leq s_1 < t_1 \leq \cdots \leq s_n < t_n \leq b:
                    \pbraces
                    {
                        \sum_{i=1}^n
                            |t_i - s_i|
                        \leq
                        \delta
                        \implies
                        \sum_{i=1}^n
                            |f(t_i) - f(s_i)|
                        \leq
                        \epsilon
                    }
    \end{align*}

    nämlich $\delta := \epsilon / M$, weil

    \begin{align*}
        \sum_{i=1}^n
            |f(t_i) - f(s_i)|
        \leq
        \sum_{i=1}^n
            M |t_i - s_i|
        \leq
        M \delta
        =
        \epsilon.
    \end{align*}

    \item Laut Satz 7.1: Radon-Nikodym, ist $g$ nicht-negativ.
    
    \includegraphicsboxed{MassWHT1&2/MassWHT1&2 - Satz 7.1 - Radon-Nikodym.png}

    \begin{align*}
        \Forall ]s, t] \subset [a, b]:
            \Int[s][t]{|g|}{\lambda}
            =
            \Int[s][t]{g}{\lambda}
            =
            \mu_f(]s, t])
            =
            |f(t) - f(s)|
            \leq
            M |t - s|
            =
            \Int[s][t]{M}{\lambda}
    \end{align*}

    Laut dem Fortsetzungssatz, gilt diese Gleichung nicht nur für Intervalle, sondern alle Borelmengen.
    Daraus folgt die Behauptung.

\end{enumerate}

\begin{itemize}

    \item [\enquote{$\Rightarrow$}]:

    \begin{align*}
        \implies
        f(x)
        \stackrel{?}{=}
        |f(x) - f(a)|
        =
        \mu_f(]a, x])
        =
        \Int[a][x]{g}{\lambda}
    \end{align*}

    \item [\enquote{$\Leftarrow$}]:
    Sei $g \leq M$ messbar und

    \begin{align*}
        f(x) = \Int[\bbraces{a, x}]{g}{\lambda}.
    \end{align*}

    \begin{align*}
        \Forall ]s, t] \subset [a, b]:
            |f(t) - f(s)|
            =
            \vbraces
            {
                \Int[a][t]{g}{\lambda}
                -
                \Int[a][s]{g}{\lambda}
            }
            =
            \vbraces
            {
                \Int[s][t]{g}{\lambda}
            }
            \leq
            \Int[s][t]{|g|}{\lambda}
            \leq
            M \lambda(]s, t])
            =
            M |t - s|
    \end{align*}    

\end{itemize}
    
\end{solution}

% --------------------------------------------------------------------------------
    
% --------------------------------------------------------------------------------

\begin{exercise}
Benutzen Sie das Jupyter-File
%FirstExample_error 
als Ausgangspunkt um mit NGSolve
Fehler-Konvergenzplots zu erstellen. Verifizieren Sie dazu mit Referenzgeraden, dass Sie für eine fixe
Polynomordnung $p$ die Konvergenzrate $h^p$ erhalten, wobei $h$ die maximale Mesh-size bezeichnet.
Was für eine Konvergenz erhält man, wenn für ein fixes Mesh die Polynomordnung erhöht wird?

\textbf{Anmerkung:} Die Rate $h^p$ ist für ein uniformes Mesh in 2D äquivalent zu $(\sqrt{\text{ndof}})^{-p}$, wobei ndof die Anzahl der Freiheitsgrade bezeichnet.
\end{exercise}

% --------------------------------------------------------------------------------

\begin{solution}

\end{solution}

% --------------------------------------------------------------------------------

% --------------------------------------------------------------------------------

\begin{exercise}[40]

Zeigen Sie, dass die leere Klausel mit (mehrfach ausgeführter) Resolution aus den Klauseln $M = \Bbraces{\Bbraces{\neg p, q}; \Bbraces{\neg r, s}; \Bbraces{p, r}; \Bbraces{\neg q}; \Bbraces{\neg s}}$ herleitbar ist.
Was hat dies mit Aufgabe 21 zu tun?

\end{exercise}

% --------------------------------------------------------------------------------

\begin{solution}

Definiere
\begin{align*}
  C_1 &:= \Bbraces{\neg p, q} \\
  C_2 &:= \Bbraces{\neg r, s} \\
  C_3 &:= \Bbraces{p, r} \\
  C_4 &:= \Bbraces{\neg q} \\
  C_5 &:= \Bbraces{\neg s} \\
  C_6 &:= \Res_p(C_3,C_1) = \{q, r\} \\
  C_7 &:= \Res_r(C_6,C_2) = \{q, s\} \\
  C_8 &:= \Res_q(C_7,C_4) = \{s\} \\
  C_9 &:= \Res_s(C_8,C_5) = \emptyset.
\end{align*}
Damit ist $(C_1,\dots,C_9)$ eine Resolutionswiderlegung von $M$. \\
Bei Aufgabe 21 galt es zu entscheiden, ob gewisse Formeln Tautologien sind oder nicht.
Dazu kann man natürlich auch zuerst die Formeln auf KNF bringen und darauf den
Resolutionsalgorithmus anwenden, welcher in endlich vielen Schritten eine
unter Resolution abgeschlossene Menge bringt. Enthält diese die leere Menge,
so ist die Formel widerlegbar, also keine Tautologie, anderenfalls schon.

\end{solution}

% --------------------------------------------------------------------------------

% --------------------------------------------------------------------------------

\begin{exercise}[Exercise 3.2]

Is the MDP framework adequate to usefully represent all goal-directed learning tasks?
Can you think of any clear exceptions?

\end{exercise}

% --------------------------------------------------------------------------------

\begin{solution}

ToDo!

\end{solution}

% --------------------------------------------------------------------------------

% --------------------------------------------------------------------------------

\begin{exercise}[Exercise 3.4]

Give a table analogous to that in Example 3.3 (texbook p. 52), but for $p(s^\prime, r \mid s, a)$.
It should have columns for $s$, $a$, $s^\prime$, $r$ and $p(s^\prime, r \mid s, a)$, and a row for every $4$-tuple for which $p(s^\prime, r \mid s, a) > 0$.

\end{exercise}

% --------------------------------------------------------------------------------

\begin{solution}

ToDo!

\end{solution}

% --------------------------------------------------------------------------------

% --------------------------------------------------------------------------------

\begin{exercise}

In einer Menge von $n$ Personen können 10 Personen Deutsch, $9$ Englisch, $9$ Russisch, $5$ Deutsch und Englisch, $7$ Deutsch und Russisch, $4$ Englisch und Russisch, $3$ alle drei Sprachen.
Wie groß ist $n$?

(Hinweis: Prinzip von Inklusion und Exklusion.)

\end{exercise}

% --------------------------------------------------------------------------------

\begin{solution}

\phantom{}

\includegraphicsboxed{Satz 2-1 (Prinzip von Inklusion und Exklusion).png}


\begin{figure}[h!]
  \centering
  \includegraphics[width = 0.75 \textwidth]{Grill - Maß- und Wahrscheinlichkeitstheorie - Satz 2-17.png}
  \caption{Grill - Maß- und Wahrscheinlichkeitstheorie}
\end{figure}

Wir definieren die Mengen $D, E, R$ jeweils als die Menge aller Deutsch-, Englisch-, Russisch-Sprachler.
Sei $\Omega$ die Menge aller Personen.

\begin{multline*}
  \implies
  |\Omega|
  =
  \underbrace{|(D \cup E \cup R)^C|}_0
  +
  |D \cup E \cup R| \\
  =
  |D| + |E| + |R| - |D \cap E| - |E \cap R| - |R \cap D| + |D \cap E \cap R|
  =
  10 + 9 + 9 - 5 - 7 - 4 + 3
  =
  15
\end{multline*}

\end{solution}

% --------------------------------------------------------------------------------

\begin{exercise}

Gegeben ist die Funktion $F: \R \to \R:$

\begin{align*}
  F(x) =
  \begin{cases}
    0   & \text{wenn} \enspace x < 0, \\
    1   & \text{wenn} \enspace 0 \leq x < 1, \\
    x^2 & \text{wenn} \enspace 1 \leq x < 2, \\
    5   & \text{wenn} \enspace x \geq 2.
  \end{cases}
\end{align*}

\begin{itemize}
  \item[(a)] Zeigen Sie, dass $F$ eine Verteilungsfunktion ist.
  \item[(b)] Bestimmen Sie $\mu_F(]0, 1[)$, $\mu_F([0, 2])$, $\mu_F(\Q)$.
  \item[(c)] Bestimmen Sie $\Int{e^x}{\mu_F(x)}$.
\end{itemize}

\end{exercise}

--------------------------------------------------------------------------------

\begin{solution}

(a)

\begin{itemize}

  \item \Quote{Rechtsstetigkeit}: $F$ ist stückweise stetig und $\Forall x = 0, 1, 2: F \text{ist rechtsstetig bei} \enspace x$.

  \item \Quote{Steigende Monotonie}: $F$ ist stückweise monoton steigend und $\Forall x = 0, 1, 2: F(x - 0) \leq F(x)$.

\end{itemize}

(b)

\begin{itemize}

  \item $\mu_F(]0, 1[) =$
  \begin{align*}
    \mu_F \pbraces{\bigcup_{n \in \N} \left ] 0, 1 - \frac{1}{n} \right ]}
    =
    \lim_{n \in \N} \mu_F \pbraces{\left ] 0, 1 - \frac{1}{n} \right ]}
    =
    \lim_{n \in \N} F \pbraces{1 - \frac{1}{n}} - F(0)
    = 1 - 1 = 0
  \end{align*}

  \item $\mu_F([0, 2]) =$
  \begin{align*}
    \mu_F \pbraces{\bigcap_{n \in \N} \left ] 0 - \frac{1}{n}, 2 \right ]}
    =
    \lim_{n \in \N} \mu_F \pbraces{\left ] 0 - \frac{1}{n}, 2 \right ]}
    =
    \lim_{n \in \N} F(2) - F \pbraces{0 - \frac{1}{n}}
    =
    5 - 0 = 5
  \end{align*}

  \item $\mu_F(\Q) =$
  \begin{align*}
    \mu_F \pbraces{\sum_{q \in \Q} \Bbraces{q}}
    & =
    \sum_{q \in \Q} \mu_F \pbraces
    {\bigcap_{n \in \N} \left ] q - \frac{1}{n}, q \right ]} \\
    & =
    \sum_{q \in \Q} \lim_{n \in \N} \mu_F \pbraces
    {\left ] q - \frac{1}{n}, q \right ]} \\
    & =
    \sum_{q \in \Q} F(q - 0) - F(q) \\
    & =
    \sum_{x = 0, 1, 2} (F(x) - F(x - 0)) \\
    & =
    (1 - 0) + (1 - 1) + (5 - 4) = 2
  \end{align*}

\end{itemize}

(c) Seien $f = \exp$ und $a_1, \ldots, a_n$ die Sprünge von $F$, sowie $a_0 = - \infty$ und $a_{n+1} = \infty$.

\begin{align*}
  \Int{f}{\mu_F}
  =
  \sum_{i=1}^{n+1} \Int[a_{i-1}][a_i]{f(x) F^\prime(x)}{x} +
  \sum_{i=1}^n f(a_i) (F(a_i) - F(a_i - 0))
\end{align*}

Also ...

\begin{align*}
  \Int{e^x}{\mu_F(x)}
  & =
  \underbrace{\Int[-\infty][0]{e^x 0}{x}}_0
  +
  \underbrace{\Int[0][1]{e^x 0}{x}}_0
  +
  \Int[1][2]{e^x 2x}{x}
  +
  \underbrace{\Int[2][\infty]{e^x 0}{x}}_0 \\
  & +
  e^0 \underbrace{(F(0) - F(0 - 0))}_{= 1-0 = 1}
  +
  e^1 \underbrace{(F(1) - F(1 - 0))}_{= 1-1 = 0}
  +
  e^2 \underbrace{(F(2) - F(2 - 0))}_{= 5-4 = 1} \\
  & =
  e^x 2x |_1^2 - 2 \Int[1][2]{e^x}{x} + 1 + e^2 \\
  & =
  (4e^2 - 2e) - 2 (e^2 - e) + 1 + e^2
  =
  1 + 3e^2
\end{align*}

\end{solution}

\begin{exercise}
What is the Bellman equation for action values, that is, for $q_\pi$?
It must give the action value $q_\pi(s, a)$ in terms of the action values, $q_\pi(s_0, a_0)$, of possible successors to the state-action pair $(s, a)$.
Hint:
the backup diagram to the right corresponds to this equation.
Show the sequence of equations analogous to \eqref{eq:2.15}, but for action values.

\begin{figure}[H]
    \centering
    \includegraphics[width = 0.2 \textwidth]{2.17.png}
    \caption{$q_\pi$ backup diagram}
    \label{fig:2.17}
\end{figure}

\end{exercise}

\begin{solution}
  We use the law of total expectation and get:

  \begin{align*}
    q_\pi(s,a)
    &=
    \E_\pi\big[G_t \mid S_t = s, A_t = a\big]
    =
    \E_\pi\big[R_{t+1} + \gamma G_{t+1}\mid S_t = s, A_t = a\big] \\
    &=
    \sum_{s^\prime, r} p(s^\prime, r \mid s,a) \E_\pi\big[R_{t+1} + \gamma G_{t+1} \mid S_t = s, A_t = a, S_{t+1} = s^\prime, R_{t+1} = r\big]  \\
    &=
    \sum_{s^\prime, r} p(s^\prime, r \mid s,a) \Big[r + \gamma \E_\pi\big[G_{t+1} \mid S_t = s, A_t = a, S_{t+1} = s^\prime, R_{t+1} = r\big]\Big] \\
    &=
    \sum_{s^\prime, r} p(s^\prime, r \mid s,a) \Big[r + \gamma \sum_{a^\prime}\pi(a^\prime\mid s^\prime)\E_\pi\big[G_{t+1}\mid S_{t+1} = s^\prime, A_{t+1} = a^\prime \big]\Big] \\
    &=
    \sum_{s^\prime, r} p(s^\prime, r \mid s,a) \Big[r + \gamma \sum_{a^\prime}\pi(a^\prime\mid s^\prime)q_\pi(s^\prime,a^\prime)\Big]
  \end{align*}
\end{solution}

% --------------------------------------------------------------------------------

\begin{exercise}[36]

ToDo!

\end{exercise}

% --------------------------------------------------------------------------------

\begin{solution}

Außer Konkurrenz!

\end{solution}

% --------------------------------------------------------------------------------

\begin{exercise}
The value of an action, $q_\pi(s, a)$, depends on the expected next reward and the expected sum of the remaining rewards.
Again we can think of this in terms of a small backup diagram, this one rooted at an action (state-action pair) and branching to the possible next states:

\begin{figure}[H]
    \centering
    \includegraphics[width = 0.5 \textwidth]{2.19.png}
    \caption{}
    \label{fig:2.19}
\end{figure}

Give the equation corresponding to this intuition and diagram for the action value, $q_\pi(s, a)$, in terms of the expected next reward, $R_{t+1}$ , and the expected next state value, $v_\pi(S_{t+1})$, given that $S_t = s$ and $A_t = a$.
This equation should include an expectation but not one conditioned on following the policy.
Then give a second equation, writing out the expected value explicitly in terms of $p(s_0, r \mid s, a)$ defined by \eqref{eq:2.13}, such that no expected value notation appears in the equation.

\end{exercise}

\begin{solution}
With the law of total expectation we get:

\begin{align*}
  q_\pi(s,a)
  &=
  \E_\pi\big[G_t\mid S_t = s, A_t = a\big]
  =
  \E_\pi\big[R_{t+1} + G_{t+1}\mid S_t = s, A_t = a\big] \\
  &=
  \sum_{r,s^\prime} p(s^\prime, r\mid s,a) \E_\pi\big[R_{t+1} + G_{t+1}\mid S_t = s, A_t = a, S_{t+1} = s^\prime, R_{t+1} = r\big] \\
  &=
  \sum_{r,s^\prime} p(s^\prime, r\mid s,a)\Big[r + \gamma \E_\pi\big[G_{t+1}\mid S_{t+1} = s^\prime\big] \Big]\\
  &=
  \sum_{r,s^\prime} p(s^\prime, r\mid s,a)\Big[r + \gamma v_\pi(s^\prime)\Big]
  =
  \E\big[R_{t+1} + v_\pi(S_{t+1})\mid S_t = s, A_t = a\big]
\end{align*}

\end{solution}


\end{document}
