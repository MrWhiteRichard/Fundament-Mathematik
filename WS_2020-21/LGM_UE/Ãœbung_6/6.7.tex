% --------------------------------------------------------------------------------

\begin{exercise}[138]

Geben Sie explizit eine konsistente Henkintheorie an, deren Sprache nur ein
einziges Konstantensymbol (möglicherweise aber weitere Prädikaten- und/oder
Funktionssymbole) enthält. \\
Wenn das zu leicht ist: Geben Sie zwei derartige Theorien an, wobei die eine
vollständig und die andere unvollständig sein soll.

\end{exercise}

% --------------------------------------------------------------------------------

\begin{solution}
Bezeichne $c$ das einzige Konstantensymbol in unserer Sprache. Wir definieren die Theorie
\begin{align*}
  \Sigma = \{\forall x \forall y\, (x = y)\}.
\end{align*}
Ein Modell erüllt die Theorie genau dann, wenn es einelementig ist. Da es einelementige Modelle gibt ist die Theorie also erfüllbar, nach dem Gödelschen Vollständigkeitssatz daher auch konsistent. Sei nun $\mathfrak{M}$ ein Modell das $\Sigma$ erfüllt, dann ist dieses Modell schon einelementig mit dem Element $m$ und daher $c^{\mathfrak{M}} = m$. Klarerweise ist jede Formel der Form $\exists x \varphi(x) \rightarrow \varphi(c)$ erfüllt, nach dem Vollständigkeitssatz gilt also $\Sigma \vdash (\exists x \varphi(x) \rightarrow \varphi(c))$ und daher ist $\Sigma$ eine Henkintheorie.
$\Sigma$ ist aber nicht vollständig,
da die geschlossene Formel $\exists x R(x)$ nicht entscheidbar ist.
\begin{align*}
  \Sigma^{\prime} = \{\forall x \forall y\, (x = y = c), \exists x R(x), R(c)\}
\end{align*}
ist dafür eine vollständige, konsisente Henkintheorie.
\end{solution}

% --------------------------------------------------------------------------------
