% --------------------------------------------------------------------------------

\begin{exercise}[138]

Geben Sie explizit eine konsistente Henkintheorie an, deren Sprache nur ein
einziges Konstantensymbol (möglicherweise aber weitere Prädikaten- und/oder
Funktionssymbole) enthält. \\
Wenn das zu leicht ist: Geben Sie zwei derartige Theorien an, wobei die eine
vollständig und die andere unvollständig sein soll.

\end{exercise}

% --------------------------------------------------------------------------------

\begin{solution}
Bezeichne $c$ das einzige Konstantensymbol in unserer Sprache und $R$ das
einzige Relationssymbol:
\begin{align*}
  \Sigma = \{\forall x \forall y\, (x = y = c)\}
\end{align*}
Jedes einelementige Modell erfüllt $\Sigma$, daher ist $\Sigma$ konsistent.
Ebenso ist $\Sigma$ eine Henkintheorie, da $\Sigma$ gar keine geschlossene
Formeln der Form $\exists x \varphi$ enthält. $\Sigma$ ist aber nicht vollständig,
da die geschlossene Formel $\exists x R(x)$ nicht entscheidbar ist.
\begin{align*}
  \Sigma^{\prime} = \{\forall x \forall y\, (x = y = c), \exists x R(x), R(c)\}
\end{align*}
ist dafür eine vollständige, konsisente Henkintheorie.
\end{solution}

% --------------------------------------------------------------------------------
