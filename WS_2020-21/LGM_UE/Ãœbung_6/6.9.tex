% --------------------------------------------------------------------------------

\begin{exercise}[137]

Sei $\mathcal{L}$ eine prädikatenlogische Sprache, und sei $\Sigma$ eine
konsistente Henkin-Theorie in der Sprache $\mathcal{L}$ mit der folgenden
Eigenschaft:
\begin{align*}
  \text{Für alle Konstantensymbole } c,d \text{ in } \mathcal{L}
  \text{ gilt entweder } \Sigma \vdash c = d \text{ oder } \Sigma \vdash c \neq d.
\end{align*}
Weiters habe $\Sigma$ die Eigenschaft, dass es zwei Konstantensymbole $0, 1$
mit $\Sigma \vdash 0 \neq 1$ gibt. Zeigen Sie, dass $\Sigma$ vollständig sein muss.

\end{exercise}

% --------------------------------------------------------------------------------

\begin{solution}

\phantom{}

\end{solution}

% --------------------------------------------------------------------------------
