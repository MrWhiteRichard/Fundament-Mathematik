% --------------------------------------------------------------------------------

\begin{exercise}[117]

Sei $\mathscr{L}$ eine (möglicherweise überabzählbare) Sprache der Prädikatenlogik,
$\Sigma_0$ eine konsistente Theorie in $\mathscr{L}$. Sei $P$ die Menge aller konsistenten
Theorien $\Sigma \supseteq \Sigma_0$ in der Sprache $\mathscr{L}$. Durch die
Relation $\subseteq$ wird $P$ partiell geordnet. \\
Zeigen Sie:
\begin{itemize}
  \item Jede Kette in $P$ (also jede durch $\subseteq$ total geordnete Teilmenge $K \subseteq P$)
  ist beschränkt. (Das heißt, für jede Kette $K \subseteq P$ gibt es $\Sigma^* \in P$,
  sodass für alle $\Sigma \in K$ die Beziehung $\Sigma \subseteq \Sigma^*$ gilt.)
  \item Wenn $\Sigma \in P_{\mathscr{L}}$ maximal ist (also: Es gibt kein
  $\Sigma^{\prime} \subsetneq$ in $P$), dann ist $\Sigma$ vollständig.
  \item Schließen Sie aus dem Lemma von Zorn, dass es zu jeder konsistenten Theorie
  $\Sigma_0$ eine vollständige konsistente Theorie $\Sigma_1 \supseteq \Sigma_0$ gibt.
\end{itemize}
\end{exercise}

% --------------------------------------------------------------------------------

\begin{solution}
\phantom{}
\begin{itemize}
  \item Sei $K$ beliebige Kette in $P$. Dann gilt sicher für alle
  $\Sigma \in K: \Sigma \subseteq \bigcup K$. \\
  Es bleibt also noch zu zeigen, dass $\bigcup K \in P$:
  Es gilt sicher $\Sigma_0 \subseteq \bigcup K$. \\
  Angenommen, $\bigcup K$
  wäre nicht konsisent, also $\bigcup K \vdash \bot$. Dann gibt es eine endliche
  Herleitung von $\bot$ aus $\bigcup K$. Daher gibt es eine endliche Teilmenge $K^{\prime} \subseteq K$,
  sodass alle nichtlogischen Axiome in der Herleitung von $\bot$ ist $\bigcup K^{\prime}$ liegen
  und somit $\bigcup K^{\prime} \vdash \bot$. Nun ist $K$ eine Totalordnung, also
  existiert ein größtes Element in
  $\Sigma_0 \in K^{\prime}: \forall \Sigma \in K: \Sigma_0 \supseteq \Sigma$.
  Daher gilt auch $\Sigma_0 \supseteq \bigcup K^{\prime}$ und damit erhalten wir
  den Widerspruch $\Sigma_0 \vdash \bot$.
  \item Sei $\Sigma^* \in P_{\mathscr{L}}$ maximal: Angenommen $\Sigma^*$ wäre nicht vollständig,
  also finden wir eine geschlossene Formel $\sigma$ in $\mathscr{L}$, sodass weder
  $\Sigma^* \vdash \sigma$ noch $\Sigma^* \vdash \neg \sigma$. Nun können wir eine
  der beiden Formeln zu $\Sigma^*$ hinzufügen und bleiben vollständig. Warum?
  Wäre dem nicht so, dann könnten wir $\Sigma^* \cup \sigma \vdash \bot$
  und $\Sigma^* \cup \neg \sigma \vdash \bot$ herleiten. Nach zweimaliger
  Anwendung des Deduktionstheorem haben wir hiermit einen halbformalen Beweis
  durch Fallunterscheidung für $\Sigma^* \vdash \bot$ dastehen. Widerspruch!
  Damit ist $\Sigma^*$ nicht maximal, nochmal Widerspruch! \\
  Also muss $\Sigma^*$ bereits vollständig sein.
  \item Jetzt müssen wir die Resultate nur noch zusammenfügen. In Punkt 1
  haben wir die Voraussetzung für das Lemma von Zorn überprüft, also
  gibt es maximale Elemente $\Sigma_1 \subseteq \Sigma_0$, welche laut Punkt 2
  bereits vollständig sein müssen.

\end{itemize}

\end{solution}

% --------------------------------------------------------------------------------
