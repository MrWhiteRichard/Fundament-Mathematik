% --------------------------------------------------------------------------------

\begin{exercise}[118]

Sei $\Sigma$ vollständige Henkintheorie, und sei $\mathscr{M}$ die zugehörige
in der Vorlesung definierte Struktur (Äquivalenzklassen von geschlossenen Termen).
Zeigen Sie, dass für alle geschlossenen Terme $t$ die Gleichung $t^{\mathscr{M}} = [t]_{\sim}$
gilt. (Was ist mit der linken Seite gemeint?)

\end{exercise}

% --------------------------------------------------------------------------------

\begin{solution}
In der Vorlesung haben wir definiert:
\begin{align*}
  \mathscr{M}&: M = \{[c]_{\sim}: c \text{ Konstantensymbol} \} \\
  c \sim d &:\iff \Sigma \vdash c = d. \\
  f^\mathscr{M}([t_1]_\sim, \dots, [t_n]_\sim) &:= [f(t_1, \dots, t_m)]_\sim
\end{align*}
Zuerst wollen wir uns überlegen, wie die Gleichung zu interpretieren ist. Für jeden geschlossenen Term $t$ erhalten wir mit dem Substitutionsaxiom, einem Existenzaxiom, einem Gleichheitsaxiom, einem Leibnizaxiom, sowie (zumindest) einer Tautologie und ein paar Mal Modus Ponens $\Sigma \vdash \exists x(x = t)$. Da $\Sigma$ eine Henkintheorie ist bekommen wir damit eine Konstante $c_t$ mit $\Sigma \vdash c_t = t$. Also ist $[t]_\sim := [c_t]_\sim$. Aufgrund der Gleichheitsaxiome ist die Defintion unabhängig vom Repräsentanten. \\
Im Folgenden wollen wir, ohne dies zu beweisen, verwenden, dass die Menge aller geschlossenen Terme die kleinste Menge ist, welche alle Konstanten enthält und bezüglich Funktionen abgeschlossen ist. So können wir also die geschlossenen Terme aufbauen, für Konstanten ist $c^\mathscr{M}$ ohnehin definiert, für $t = f(t_1, \dots, t_k)$ definieren wir $t^\mathscr{M} := f^\mathscr{M}(t_1^\mathscr{M}, \dots, t_k^\mathscr{M})$.  \\
Nun beweisen wir die Aussage mit Induktion nach dem Aufbau der geschlossenen Terme. \\
Für alle Konstantensymbole folgt die Aussage direkt aus der Definition: $c^{\mathscr{M}} = [c]_{\sim}$. \\
Sei $t = f(t_1,\dots,t_k)$ beliebig, wobei $t_1^\mathscr{M} = [t_1]_\sim, \dots, t_k^\mathscr{M} = [t_k]_\sim$. Ganz formal gilt 
\begin{align*}
t^\mathscr{M} := f^\mathscr{M}(t_1^\mathscr{M}, \dots, t_k^\mathscr{M}) = f^\mathscr{M}([t_1]_\sim, \dots, [t_k]_\sim) = [f(t_1, \dots, t_k)]_\sim = [t]_\sim.
\end{align*} 
Zu zeigen bleibt, dass es hier keine Abhängigkeit vom Repräsentanten gibt. Sei also $[s_1]_\sim = [t_1]_\sim, \dots, [s_k]_\sim = [t_k]_\sim$. Nach obiger Definition gibt es Konstanten $c_1,\dots,c_k$, sodass für alle $i \in \{1, \dots, k\}$ die Ausdrücke
$(\Sigma \vdash c_i  = t_i)$ und $(\Sigma \vdash c_i  = s_i)$ gelten. Mit den Gleichheitsaxiomen und den Leibniz-Axiomen leitet man schnell $\Sigma \vdash f(s_1,\dots,s_k) = f(t_1,\dots,t_k)$ her. Für $d \in [c]_\sim$ gilt $\Sigma \vdash c = d$ und daher $c^\mathscr{M} = d^\mathscr{M}$. Also sind alle Definitionen unabhängig von den Repräsentanten.
\end{solution}

% --------------------------------------------------------------------------------
