% --------------------------------------------------------------------------------

\begin{exercise}[118]

Sei $\Sigma$ vollständige Henkintheorie, und sei $\mathscr{M}$ die zugehörige
in der Vorlesung definierte Struktur (Äquivalenzklassen von geschlossenen Termen).
Zeigen Sie, dass für alle geschlossenen Terme $t$ die Gleichung $t^{\mathscr{M}} = [t]_{\sim}$
gilt. (Was ist mit der linken Seite gemeint?)

\end{exercise}

% --------------------------------------------------------------------------------

\begin{solution}
In der Vorlesung haben wir definiert:
\begin{align*}
  \mathscr{M}&: M = \{[c]_{\sim}: c \text{ Konstantensymbol} \} \\
  c \sim d &:\iff \Sigma \vdash c = d.
\end{align*}
Die Definition der Äquivalenzrelation kann auf natürliche Weise auf die Menge
aller geschlossenen Terme in $\mathscr{L}$ ausgedehnt werden. Mit $t^{\mathscr{M}} = [t]_{\sim}$
ist also gemeint: Es gibt ein Konstantensymbol $c$, sodass $\Sigma \vdash c = t$. \\
Die linke Seite macht aus den ersten Blick keinen Sinn, da unsere Interpretation
nur Konstanten, Funktionssymbole und Relationssymbole interpretiert. Da wir
es allerdings nur mit geschlossenen Termen zu tun haben, kommen keine Variablen in $t$
vor und der Wert $\overline{b}(t)$ ist daher unabhängig von der konkreten Wahl
der Belegung $b$ und kann somit als Definition für $t^{\mathscr{M}}$ gewählt werden. \\
Wir beweisen die Aussage mit Induktion nach dem Aufbau der geschlossenen Terme. \\
Für alle Konstantensymbole folgt die Aussage direkt aus der Definition: $c^{\mathscr{M}} = [c]_{\sim}$.
Da wir nur geschlossene Terme betrachten kommen keine Variablen darin vor. \\
Sei $t = f(t_1,\dots,t_k)$ beliebig, nach Voraussetzung existieren also $c_1,\dots,c_k$, sodass
$(\Sigma \vdash c_i  = t_i), i = 1,\dots,k$. Mit den Leibniz-Axiomen leitet man schnell
$\Sigma \vdash f(c_1,\dots,c_k) = f(t_1,\dots,t_k) = t$ her, also $t^{\mathscr{M}} = [t]_{\sim}$.
\end{solution}

% --------------------------------------------------------------------------------
