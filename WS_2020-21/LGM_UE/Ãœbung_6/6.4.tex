% --------------------------------------------------------------------------------

\begin{exercise}[130]

Sei $\mathscr{M}$ eine unendliche abzählbare Struktur. Zeigen Sie, dass es in
$\mathrm{Mod}(\mathrm{Th}(\mathscr{M}))$ eine überabzählbare Struktur gibt. \\
\textit{Hinweis:} Vollständigkeitssatz und Kompaktheitssatz gelten auch für
überabzählbare Sprachen. Verwenden Sie überabzählbar viele Konstantensymbole.

\end{exercise}

% --------------------------------------------------------------------------------

\begin{solution}
Mit $\mathrm{Mod}(\Sigma)$ bezeichnen wir die Klasse aller Strukturen $\mathscr{M}$
(einer vorgegebenen Sprache), die $\mathscr{M} \vDash \Sigma$ erfüllen.
Sei $\mathfrak{M}$ eine Klasse von Strukturen. Mit $\mathrm{Th}(\mathfrak{M})$ bezeichnen
wir die Theorie von $\mathfrak{M}$, also die Menge aller geschlossenen Formeln $\varphi$,
die in jedem $\mathscr{M} \in \mathfrak{M}$ erfüllt sind. \\
In unserem Fall gilt $\mathrm{Th}(\mathscr{M}) = \{\sigma: \mathscr{M} \vDash \sigma\}$.
Bezeichne $\mathscr{L}$ die Sprache von $\mathscr{M}$ und sei $C$ eine überabzählbare
Menge von neuen Konstantensymbolen. Definiere $\mathscr{L}^{\prime} := \mathscr{L} + C$, sowie
$\Sigma := \mathrm{Th}(\mathscr{M})\, \cup\, \{c \neq d: c,d \in C, c \neq d \}$.
Jede endliche Teilmenge davon ist erfüllbar, daher muss mit der Kontraposition
des Kompaktheitssatzes auch $\Sigma$ erfüllbar sein. Also existiert ein $\mathscr{M}^{\prime}$
mit $\mathscr{M}^{\prime} \vDash \mathrm{Th}(\mathscr{M})\, \cup\, \{c \neq d: c,d \in C, c \neq d \}$,
insbesondere ist $\mathscr{M}^{\prime} \in \mathrm{Mod}(\mathrm{Th}(\mathscr{M}))$.
Die hinzugefügten Formeln garantieren, dass verschiedene Konstantensymbole der
neuen überabzählbaren Konstantenmenge $C$ auch unterschiedlich interpretiert werden.
Damit muss $\mathscr{M}^{\prime}$ ebenso eine überabzählbare Struktur sein.

\end{solution}

% --------------------------------------------------------------------------------
