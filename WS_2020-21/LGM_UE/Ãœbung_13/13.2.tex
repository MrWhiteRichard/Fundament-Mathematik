% --------------------------------------------------------------------------------

\begin{exercise}[278]

Mit \glqq abzählbar\grqq\  meinen wir in dieser Aufgabe \glqq höchstens abzählbar\grqq.
Definieren Sie den Begriff \glqq Familie von abzählbarem Charakter\grqq\ und geben
Sie eine Familie an, die zwar abzählbaren Charakter hat, nicht aber endlichen Charakter.

\end{exercise}

% --------------------------------------------------------------------------------

\begin{solution}

Sei $\mathfrak{F} = (A_i: i \in I)$ eine Familie (oder Menge) von Mengen.
Wir sagen, $\mathfrak{F}$ hat \textit{abzählbaren Charakter}, falls

\begin{align*}
  \Forall A:
  \pbraces
  {
    A \in \mathfrak{F}
    \leftrightarrow
    \pbraces
    {
      \Forall B \subseteq A:
      B ~\text{abzählbar}~
      \to
      B \in \mathfrak{F}
    }
  }.
\end{align*}


  Sei $X$ abzählbar und $\mathfrak{F} := \Bbraces{A \subseteq X: A ~\text{abzählbar}}$.
  Dann hat $\mathfrak{F}$ abzählbaren Charakter.

  \begin{enumerate}[label = \texttt{ad}]

    \item \enquote{$\to$}:

    Sei $A \in \mathfrak{F}$, also $A \subseteq X$ und abzählbar.
    Sei weiters $B \subseteq A \subseteq X$ und abzählbar.
    Dann ist $B \in \mathfrak{F}$.

    \item \enquote{$\leftarrow$}:

    Natürlich ist $X \subseteq X$ abzählbar, also $X \in \mathfrak{F}$.

  \end{enumerate}

  Ein Gegenbeispiel für (gegen) endlichen Charakter (\enquote{$\not \leftarrow$}) wäre die Menge $X$ selbst.


\end{solution}

% --------------------------------------------------------------------------------
