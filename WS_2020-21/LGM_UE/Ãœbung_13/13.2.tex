% --------------------------------------------------------------------------------

\begin{exercise}[278]

Mit \glqq abzählbar\grqq\  meinen wir in dieser Aufgabe \glqq höchstens abzählbar\grqq.
Definieren Sie den Begriff \glqq Familie von abzählbarem Charakter\grqq\ und geben
Sie eine Familie an, die zwar abzählbaren Charakter hat, nicht aber endlichen Charakter.

\end{exercise}

% --------------------------------------------------------------------------------

\begin{solution}

Sei $\mathfrak{F} = (A_i: i \in I)$ eine Familie von Mengen. Wir sagen, $\mathfrak{F}$
hat abzählbaren Charakter, falls

\begin{align*}
  A \in \mathfrak{F} \iff (\forall B \subseteq A: B \text{ abzählbar } \implies B \in \mathfrak{F})
\end{align*}

Als Beispiel einer Mengenfamilie mit abzählbaren, aber nicht endlichen Charakter
betrachte die Menge $\mathcal{E}$ aller endlichen Teilmengen von $\omega$. \\
Aus $A \in \mathcal{E}$ folgt, dass $A$ endlich ist, und somit alle Teilmengen
von $A$ bereits in $\mathcal{E}$ liegen müssen. \\
Sei nun $B$ eine Menge, deren abzählbare Teilmengen allesamt in $\mathcal{E}$
liegen. Daraus folgt bereits $B \subseteq \omega$ und da $B$ damit selbst abzählbar ist,
muss bereits $B \in \mathcal{E}$ gelten. Also hat die $\mathcal{E}$ abzählbaren
Charakter. \\
$\mathcal{E}$ hat aber sicher nicht endlichen Charakter, wie man am Beispiel $\omega$
sieht: Alle endlichen Teilmengen von $\omega$ sind nach Definition in $\mathcal{E}$,
$\omega$ selbst allerdings nicht.
\end{solution}

% --------------------------------------------------------------------------------
