% --------------------------------------------------------------------------------

\begin{exercise}[278]

Mit \glqq abzählbar\grqq\  meinen wir in dieser Aufgabe \glqq höchstens abzählbar\grqq.
Definieren Sie den Begriff \glqq Familie von abzählbarem Charakter\grqq\ und geben
Sie eine Familie an, die zwar abzählbaren Charakter hat, nicht aber endlichen Charakter.

\end{exercise}

% --------------------------------------------------------------------------------

\begin{solution}

Sei $\mathfrak{F} = (A_i: i \in I)$ eine Familie (oder Menge) von Mengen.
Wir sagen, $\mathfrak{F}$ hat \textit{abzählbaren Charakter}, falls

\begin{align*}
  \Forall A:
  \pbraces
  {
    A \in \mathfrak{F}
    \leftrightarrow
    \pbraces
    {
      \Forall B \subseteq A:
      B ~\text{abzählbar}~
      \to
      B \in \mathfrak{F}
    }
  }.
\end{align*}

\begin{enumerate}[label = \arabic*.]

  \item Beispiel:

  Als Beispiel einer Mengenfamilie mit abzählbaren, aber nicht endlichen Charakter
  betrachte die Menge $\mathcal{E}$ aller endlichen Teilmengen von $\omega$. \\
  Aus $A \in \mathcal{E}$ folgt, dass $A$ endlich ist, und somit alle Teilmengen
  von $A$ bereits in $\mathcal{E}$ liegen müssen. \\
  Sei nun $B$ eine Menge, deren abzählbare Teilmengen allesamt in $\mathcal{E}$
  liegen. Daraus folgt bereits $B \subseteq \omega$ und da $B$ damit selbst abzählbar ist,
  muss bereits $B \in \mathcal{E}$ gelten. Also hat die Menge $\mathcal{E}$ abzählbaren
  Charakter. \\
  $\mathcal{E}$ hat aber sicher nicht endlichen Charakter, wie man am Beispiel $\omega$
  sieht: Alle endlichen Teilmengen von $\omega$ sind nach Definition in $\mathcal{E}$,
  $\omega$ selbst allerdings nicht.

  \item Beispiel:
  
  Sei $X$ abzählbar und $\mathfrak{F} := \Bbraces{A \subseteq X: A ~\text{abzählbar}}$.
  Dann hat $\mathfrak{F}$ abzählbaren Charakter.

  \begin{enumerate}[label = \texttt{ad}]

    \item \Quote{$\to$}:

    Sei $A \in \mathfrak{F}$, also $A \subseteq X$ und abzählbar.
    Sei weiters $B \subseteq A \subseteq X$ und abzählbar.
    Dann ist $B \in \mathfrak{F}$.

    \item \Quote{$\leftarrow$}:

    Natürlich ist $X \subseteq X$ abzählbar, also $X \in \mathfrak{F}$.

  \end{enumerate}

  Ein Gegenbeispiel für (gegen) endlichen Charakter (\Quote{$\not \leftarrow$}) wäre die Menge $X$ selbst.

\end{enumerate}

\end{solution}

% --------------------------------------------------------------------------------
