\subsection*{288 + 289}

% -------------------------------------------------------------------------------- %

\begin{exercise}[288]

Geben Sei eine explizite Wohlordnung von $V_{\omega}$ an. \\
\textit{Hinweis:} Geben Sie eine explizite Wohlordnung von $V_{n+1} \setminus V_n$ an.
\end{exercise}

% -------------------------------------------------------------------------------- %

\begin{solution}

\phantom{}

\end{solution}

% -------------------------------------------------------------------------------- %

\begin{exercise}[289]

Mit PWW bezeichnen wir den Satz: \glqq Für jede wohlgeordnete Menge $A$ gilt,
dass es auf $\mathfrak{P}(A)$ eine Wohlordnung gibt.\grqq\
Schließen Sie aus ZF+PWW dass es auf $V_{\omega + \omega}$ eine Wohlordnung gibt. \\
\textit{Achtung}: Das ist eine schwierige Aufgabe. Bitte kontrollieren Sie sorgfältig,
ob Sie nicht versehentlich das Auswahlaxiom verwendet haben. Wenn Sie eine einfachen
Beweis gefunden haben, dann ist er ziemlich sicher falsch, weil er nämlich in
versteckter Weise das Auswahlaxiom verwendet.

\end{exercise}

% -------------------------------------------------------------------------------- %

\begin{solution}

\phantom{}

\end{solution}

% -------------------------------------------------------------------------------- %
