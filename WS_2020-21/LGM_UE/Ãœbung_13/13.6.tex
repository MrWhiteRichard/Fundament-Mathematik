% --------------------------------------------------------------------------------

\begin{exercise}[286]

Beweisen Sie (in ZFC, informell):
Eine lineare Ordnung $(L, <)$ ist genau dann KEINE Wohlordnung, wenn es eine Funktion $f: \omega \to L$ gibt mit $\Forall n \in \omega: f(n + 1) < f(n)$.
An welchen Stellen Ihres Beweises (wenn überhaupt) verwenden Sie das Auswahlaxiom?

\end{exercise}

% --------------------------------------------------------------------------------

\begin{solution}

Sei $(L, <)$ eine lineare Ordnung.
Zu zeigen ist, dass

\begin{align*}
    (L, <) ~\text{keine Wohlordnung}~
    \iff
    \Exists f: \omega \to L:
        \Forall n \in \omega:
            f(n + 1) < f(n)
\end{align*}

\begin{enumerate}[label = \texttt{ad}]

    \item \blockquote{$\implies$}:

    Sei $(L, <)$ keine Wohlordnung, dann

    \begin{align} \label{eq:keine_Wohlordnung}
        \begin{split}
            \Exists E \in \mathcal{P}(L) \setminus \Bbraces{\emptyset}:
                & \nExists l \in E:
                    \Forall l^\prime \in E:
                        l^\prime \leq l \to l = l^\prime \\
                \iff
                & \Forall l \in E:
                    \Exists l^\prime \in E:
                        l^\prime \leq l \land l \neq l^\prime \\
                \stackrel{!}{\iff}
                & \Forall l \in E:
                    \Exists l^\prime \in E:
                        l^\prime < l.
        \end{split}
    \end{align}

    Letzteres \blockquote{!} gilt, weil $(L, <)$ ja eine lineare Ordnung ist.
    Wir finden jetzt mit dem Auswahlaxiom eine Funktion

    \begin{align*}
        g:
        \mathfrak{P}(L) \setminus \Bbraces{\emptyset} \to L:
            \Forall A \subseteq L:
                g(A) \in A.
    \end{align*}

    Nun definieren wir $E_0 := E$ und $f(0) := g(E_0)$.

    Sei nun $f(n) \in E_{n} := E_{< f(n-1)} \subseteq E$ bereits definiert.
    Weil $f(n) \in E$, gibt es, laut \eqref{eq:keine_Wohlordnung}, ein $f(n + 1) = f(n)^\prime \in E$, sodass $f(n + 1) < f(n)$.
    Also ist $E_{n+1} := E_{<f(n)} \neq \emptyset$ und wir definieren $f(n+1) := g(E_{n+1})$.

    \item \blockquote{$\impliedby$}:

    Es gebe eine streng monoton fallende Funktion $f: \omega \to L$.
    Sei $l \in E := f[\omega] \in \mathfrak{P}(L) \setminus \Bbraces{\emptyset}$.

    \begin{align*}
        \implies &
        \Exists n \in \omega:
            l = f(n) > f(n + 1) =: l^\prime \\
        \implies &
        l^\prime \leq l \land l^\prime \neq l
    \end{align*}

    Dafür wird das Auswahlaxiom allerdings nicht gebraucht.

\end{enumerate}

\end{solution}

% --------------------------------------------------------------------------------
