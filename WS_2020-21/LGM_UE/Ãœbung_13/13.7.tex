% --------------------------------------------------------------------------------

\begin{exercise}[287]

Zeigen Sie (in ZF, ohne AC): Wenn $(A,<)$ eine Wohlordnung ist, dann gibt es eine
lineare Ordnung auf der Potenzmenge von $A$.
\end{exercise}

% --------------------------------------------------------------------------------

\begin{solution}

\phantom{}

\begin{enumerate}[label = \arabic*.]

  \item Versuch (\textbf{gescheitert}):

  Für alle $B \subseteq A$: Definiere die partielle Funktion $f_B: \omega \to B$ durch
  \begin{align*}
    f_B(n)
    =
    \begin{cases}
      \min(B \setminus f[\{0,\dots,n-1\}]) & \text{falls }
      B \setminus f[\{0,\dots,n-1\}] \neq \emptyset \\
      \min(A) & \text{sonst}
    \end{cases}
  \end{align*}
  
  Wir definieren auf $\mathfrak{P}(A)$

  \begin{align*}
    B <_\mathfrak{P} C
    :\iff
    [\Exists n \in \omega: (f_B(n) < f_C(n)) \land \Forall k < n: f_B(k) = f_C(k)]
    \lor
    (B = \emptyset \land C \neq \emptyset).
  \end{align*}
  
  \begin{itemize}

    \item Irreflexivität:

    Klar.

    \item Transitivität:

    Gelte $B <_\mathfrak{P} C$ und $C <_\mathfrak{P} D$. \\
    Dann existieren $n$ und $m$, sodass $f_B(n) < f_C(n)$ und $f_C(m) < f_D(m)$.
    Für $l := \min(n, m)$ folgt $f_B(l) \leq f_C(l) \leq f_D(l)$, wobei bei mindestens einer Ungleichung echt ungleich gelten muss, also $f_B(l) < f_D(l)$ und für alle $k < l: f_B(k) = f_D(k)$.
  
    \item Trichotomie:

    Gelte $B \nless_\mathfrak{P} C$ und $C \nless_\mathfrak{P} B$ und $B, C \neq \emptyset$.
    Also gilt für alle $k \in \N: f_B(k) = f_C(k)$.
    Für endliche Mengen folgt daraus sicher bereits Gleichheit, bei unendlichen Mengen könnten vielleicht noch Tricks versteckt sein. \\

  \end{itemize}

  Problem:
  Betrachte die Wohlordnung $(\N \cup +\infty, <)$, wobei wir zu den natürlichen Zahlen ein maximales Element hinzugefügt haben.
  Dann wären die Mengen $\N$ und $\N \cup +\infty$ mit unserer linearen Ordnung nicht vergleichbar.

  \item Versuch (\textbf{erfolgreich}):
  
  Wir definieren $\overline{A} = A \cup \Bbraces{-\infty}$ mit $\min(\emptyset) := -\infty$ und

  \begin{align*}
    \overline < \: := \: < \cup \: \Bbraces{(-\infty, a): a \in A}.
  \end{align*}

  Der Einfachkeit halben, ignorieren wir das ursprüngliche $<$ und schreiben $<$ statt $\overline <$.
  Auf der Potenzmenge definieren wir
  
  \begin{align*}
    B <_\mathfrak{P} C
    : \iff
    \min(B \setminus C) < \min(C \setminus B),
    \quad
    B, C \in \mathfrak{P}(\overline A).
  \end{align*}

  \phantom{}
  
  \begin{center}

    \def \r {1}
    \def \R {1.3}

    \begin{tikzpicture}[scale=1.5]

      \foreach \ang/\name in {-150/b, -30/c, 90/d}
        \draw (\ang:\r) circle (\R) coordinate (\name) node {$\name$};

      \path (b) -- (c)  node [midway]   (bc)  {$bc$};
      \path (c) -- (d)  node [midway]   (cd)  {$cd$};
      \path (d) -- (b)  node [midway]   (db)  {$db$};
      \path (d) -- (bc) node [pos=.667] (bcd) {$bcd$};

    \end{tikzpicture}

  \end{center}
  
  \begin{itemize}

    \item Irreflexivität:

    Klar.
    
    \item Transitivität:

    Gelte $B <_\mathfrak{P} C$ und $C <_\mathfrak{P} D$. \\
    Wir verwenden ab nun die Kurznotationen aus der Skizze: \\
    (Man beachte, dass die Abkürzungen auch für die leere Menge stehen können.) \\
    Laut Voraussetzung gilt also
    
    \begin{align*}
      &
      \min(b \cup db)
      <
      \min(c \cup cd)
      \land
      \min(c \cup bc)
      <
      \min(d \cup db). \\
      \implies &
      \min(b \cup bc \cup db)
      =
      \min(b \cup bc \cup db \cup c \cup cd)
      =
      \min(b \cup bc \cup db \cup c \cup cd \cup d)
      \leq
      \min(d \cup cd)
    \end{align*}
    
    Angenommen, es gelte $\min(db) < \min(b \cup bc)$, dann erhalten wir mit
    
    \begin{align*}
      \min(db \cup d) &> \min(c \cup bc) \geq \min(c \cup bc \cup cd) \geq \min(b \cup bc \cup db)
      =
      \min(db)
    \end{align*}
    
    einen Widerspruch!
    Also gilt
    
    \begin{align*}
      \min(b \cup bc)
      =
      \min(b \cup bc \cup db) \leq \min(d \cup cd)
    \end{align*}
    
    und aufgrund der Disjunktheit gilt sogar die strikte Ungleichung und somit $B <_\mathfrak{P} D$.
  
    \item Trichotomie:

    Gelte $B \nless_\mathfrak{P} C$ und $C \nless_\mathfrak{P} B$, also $\min(B \setminus C) = \min(C \setminus B)$.
    Dies ist aufgrund der Disjunktheit nur möglich, wenn $B \setminus C = C \setminus B = \emptyset$, also $B = C$.

  \end{itemize}

\end{enumerate}

\end{solution}

% --------------------------------------------------------------------------------
