% --------------------------------------------------------------------------------

\begin{exercise}[287]

Zeigen Sie (in ZF, ohne AC): Wenn $(A,<)$ eine Wohlordnung ist, dann gibt es eine
lineare Ordnung auf der Potenzmenge von $A$.
\end{exercise}

% --------------------------------------------------------------------------------

\begin{solution}
\textbf{Gescheiterter} erster Versuch: \\
Für alle $B \subseteq A$: Definiere die partielle Funktion $f_B: \omega \to B$ durch
\begin{align*}
  f_B(n) = \begin{cases}
    \min(B \setminus f[\{0,\dots,n-1\}]) & \text{falls }
    B \setminus f[\{0,\dots,n-1\}] \neq \emptyset \\
    \min(A) & \text{sonst}
  \end{cases}
\end{align*}

Wir definieren auf $\mathfrak{P}(A)$
\begin{align*}
  B <_\mathfrak{P} C \iff [\exists n \in \omega: (f_B(n) < f_C(n)) \land \forall k < n: f_B(k) = f_C(k)] \lor (B = \emptyset \land C \neq \emptyset)
\end{align*}

\begin{itemize}
  \item[Irreflexivität:] Klar.
  \item[Transitivität:] Gelte $B <_\mathfrak{P} C$ und $C <_\mathfrak{P} D$. \\
  Dann existieren $n$ und $m$, sodass
  $f_B(n) < f_C(n)$ und $f_C(m) < f_D(m)$. Für $l := \min(n,m)$
  folgt $f_B(l) \leq f_C(l) \leq f_D(l)$, wobei bei mindestens einer Ungleichung
  echt ungleich gelten muss, also $f_B(l) < f_D(l)$ und für alle $k < l: f_B(k) = f_D(k)$.

  \item[Trichotomie:] Gelte $B \nless_\mathfrak{P} C$ und $C \nless_\mathfrak{P} B$
  und $B, C \neq \emptyset$. Also gilt für alle $k \in \N: f_B(k) = f_C(k)$.
  Für endliche Mengen folgt daraus sicher bereits Gleichheit, bei
  unendlichen Mengen könnten vielleicht noch Tricks versteckt sein. \\
\end{itemize}
Problem: Betrachte die Wohlordnung $(\N \cup +\infty, <)$, wobei wir zu den
natürlichen Zahlen ein maximales Element hinzugefügt haben. Dann wären die Mengen
$\N$ und $\N \cup +\infty$ mit unserer linearen Ordnung nicht vergleichbar.




\textbf{Erfolgreicher} zweiter Versuch: \\
Wir definieren $\overline{A} = A \cup \{-\infty\}$ mit $\forall x \in A: -\infty < x$
und weiters $\min(\emptyset) := -\infty$. \\
Auf der Potenzmenge definieren wir

\begin{align*}
  B <_\mathfrak{P} C :\min(B \setminus C) < \min(C \setminus B)
\end{align*}


\begin{center}
\def\r{1}
\def\R{1.3}
\begin{tikzpicture}[scale=1.5]
  \foreach\ang/\name in {-150/a,-30/b,90/c} \draw (\ang:\r) circle (\R) coordinate (\name) node {\name};
  \path (a) -- (b) node [midway] (ab) {ab} ;
  \path (b) -- (c) node [midway] (bc) {bc} ;
  \path (c) -- (a) node [midway] (ca) {ca} ;
  \path (c) -- (ab) node [pos=.667] (abc) {abc};
\end{tikzpicture}
\end{center}

\begin{itemize}
  \item[Irreflexivität:] Klar.
  \item[Transitivität:] Gelte $A <_\mathfrak{P} B$ und $B <_\mathfrak{P} C$. \\
  Wir verwenden ab nun die Kurznotationen aus der Skizze: \\
  (Man beachte, dass die Abkürzungen auch für die leere Menge stehen können)\\
  Laut Voraussetzung gilt also $\min(a \cup ca) < \min(b \cup bc)$ und
  $\min(b \cup ab) < \min(c \cup ca)$. Damit folgt
  \begin{align*}
    \min(a \cup ab \cup ca) = \min(a \cup ab \cup ca \cup b \cup bc)
    = \min(a \cup ab \cup ca \cup b \cup bc \cup c) \leq \min(c \cup bc)
  \end{align*}
  Angenommen, es gelte $\min(ca) < \min(a \cup ab)$, dann erhalten wir mit
  \begin{align*}
  \min(ca \cup c) &> \min(b \cup ab) \geq \min(b \cup ab \cup bc) \geq \min(a \cup ab \cup ca)
  = \min(ca)
  \end{align*}
  einen Widerspruch! Also gilt
  \begin{align*}
    \min(a \cup ab) = \min(a \cup ab \cup ca) \leq \min(c \cup bc)
  \end{align*}
  und aufgrund der Disjunktheit gilt sogar die strikte Ungleichung und somit $A <_\mathfrak{P} C$.

  \item[Trichotomie:] Gelte $A \nless_\mathfrak{P} B$ und $B \nless_\mathfrak{P} A$, also
  $\min(A \setminus B) = \min(B \setminus A)$. Dies ist aufgrund der Disjunktheit nur
  möglich, wenn $A \setminus B = B \setminus A = \emptyset$, also $A = B$.
\end{itemize}

\end{solution}

% --------------------------------------------------------------------------------
