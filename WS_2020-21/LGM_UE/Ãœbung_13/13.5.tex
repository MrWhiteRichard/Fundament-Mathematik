% -------------------------------------------------------------------------------- %

\begin{exercise}[281]

Sei $A$ eine endliche Menge mit $n$ Elementen. Sei $W(A)$ die Menge aller $(X, R)$, sodass $X \subseteq A$ ist und $R \subseteq X \times X$ eine lineare Ordnung von $X$ ist.
Sei $\simeq$ die Isomorphierelation auf $W(A)$ und $H(A)$ die Menge aller Äquivalenzklassen.

\begin{enumerate}[label = \alph*.]

  \item Für $A = \Bbraces{1, 2, 3}$:
  Wie viele Elemente hat $W(A)$?
  Geben Sie alle an.
  Wie viele Elemente hat $H(A)$?
  Geben Sie alle an.

  \item Für beliebiges $n$:
  Wie viele Elemente hat $H(A)$?

\end{enumerate}

\end{exercise}

% -------------------------------------------------------------------------------- %

\begin{solution}

\phantom{}

\begin{comment}

  \begin{enumerate}[label = \alph*.]
    \item
    \begin{align*}
      W(A)
      = \{ 1 \leq 2 \leq 3; 1 \leq 3 \leq 2; 2 \leq 1 \leq 3; 2 \leq 3 \leq 1;
      3 \leq 1 \leq 2; 3 \leq 2 \leq 1\}
    \end{align*}
    Wir können die Elemente aus $W(A)$ in kanonischer Weise mit den
    Permutationen auf der Menge $A = \{1, 2, 3\}$ identifizieren.
    Dabei sind alle linearen Ordnungen vermöge der jeweiligen Permutation
    isomorph,  also hat $H(A)$ genau ein Element.
    \item Ist nicht für alle $n$ $H(A)$ einfach ein-elementig?
  \end{enumerate}

\end{comment}

\begin{enumerate}[label = \alph*.]

  \item $W(A)$ und $H(A)$ sind der unteren Darstellung zu entnehmen.

  \begin{align*}
    \begin{array}{c|c|ccc|ccc|c}
      X & \emptyset & \Bbraces{1} & \Bbraces{2} & \Bbraces{3} & \Bbraces{1, 2} & \Bbraces{1, 3} & \Bbraces{2, 3} & \Bbraces{1, 2, 3} \\
        &           &             &             &             &                &                &                &                   \\
      R &           & 1           & 2           & 3           & 1 2            & 1 3            & 2 3            & 1 2 3             \\
        &           &             &             &             & 2 1            & 3 1            & 3 2            & 1 3 2             \\
        &           &             &             &             &                &                &                & 2 1 3             \\
        &           &             &             &             &                &                &                & 2 3 1             \\
        &           &             &             &             &                &                &                & 3 1 2             \\
        &           &             &             &             &                &                &                & 3 2 1             \\
    \end{array}
  \end{align*}

  \item $|H(A)| = n + 1$

\end{enumerate}

\end{solution}

% -------------------------------------------------------------------------------- %
