% --------------------------------------------------------------------------------

\begin{exercise}[110]

Skizzieren Sie einen weiteren (halb-)formalen Beweis einer der Formeln aus den vorigen
beiden Beispielen.

\end{exercise}

% --------------------------------------------------------------------------------

\begin{solution}

Wir skizzieren den Beweis für
$\exists x [A(x) \land B(x)] \rightarrow [\exists x A(x)] \land [\exists x B(x)]$:\\
Wir beweisen stattdessen
$[\forall x \neg A(x)] \land [\forall x \neg B(x)] \rightarrow \forall x [\neg(A(x) \land B(x))] $ und schließen
dann mit Tautologien auf die eigentliche Formel.
Wir kennen bereits einen (halbformalen) Beweis von
$[\forall x \neg A(x)] \land [\forall x \neg B(x)] \rightarrow \forall x [\neg A(x) \land \neg B(x)]$.
Jetzt müssen wir nur noch die offensichtliche Tautologie
mit dem Distributivitäts-Axiom kombinieren und wir sind fertig:
\begin{flalign*}
  &\vdash \forall x \underbrace{(\neg A(x) \land \neg B(x))}_{A} \rightarrow
  \underbrace{(\neg(A(x) \land B(x)))}_{B}
  & \text{(Allquantor vor Tautologie)}\\
  &\vdash [\forall x (A \rightarrow B)]
  \rightarrow [\forall x A \rightarrow \forall x B] & \text{(Distributivität)} \\
  &\vdash [\forall x \neg A(x)] \land [\forall x \neg B(x)] \rightarrow \forall x [\neg(A(x) \land B(x))]
  & \text{(Tautologie + MP)}
\end{flalign*}
\end{solution}

% --------------------------------------------------------------------------------
