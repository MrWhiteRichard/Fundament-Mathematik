% --------------------------------------------------------------------------------

\begin{exercise}[104 + 105]
\phantom{}
\begin{enumerate}
  \item Beweisen Sie die starke $\exists$-Einführung. Das heißt: Wenn $\Phi \vdash A \rightarrow B$,
  und $x$ weder in $\Phi$ noch in $B$ frei vorkommt, dann $\Phi \vdash \exists x A \rightarrow B$. \\
  \textit{Hinweis:} Verwenden Sie die starke $\forall$-Einführung.
  \item Formulieren und beweisen Sie die schwache $\exists$-Einführung.

\end{enumerate}


\end{exercise}

% --------------------------------------------------------------------------------

\begin{solution}
\phantom{}
\begin{enumerate}
  \item Unter Verwendung der starken Einführung des Allquantors erhalten wir den halbformalen Beweis:
  \begin{flalign*}
    & \Phi \vdash A \rightarrow B \\
    & \Phi \vdash (A \rightarrow B) \rightarrow (\neg B \rightarrow \neg A) & \text{(Tautologie \eqref{eq:taut3})}\\
    & \Phi \vdash \neg B \rightarrow \neg A & \text{(MP)} \\
    & \Phi \vdash \neg B \rightarrow \forall x \neg A & \text{(starke Einführung von $\forall$)} \\
    & \Phi \vdash \exists x A \rightarrow \neg \forall x \neg A & \text{(Existenz-Axiom)} \\
    & \Phi \vdash (\neg B \rightarrow \forall x \neg A) \rightarrow
    (\exists x A \rightarrow \neg \forall x \neg A) \rightarrow (\exists x A \rightarrow B)
    & \text{(Tautologie \eqref{eq:taut5})} \\
    & \Phi \vdash (\exists x A \rightarrow \neg \forall x \neg A)
    \rightarrow (\exists x A \rightarrow B) & \text{(MP)} \\
    & \Phi \vdash \exists x A \rightarrow B & \text{(MP)} \\
  \end{flalign*}
  Hier noch weitere verwendete Tautologien:
  \begin{align}
  (\neg p \rightarrow q) \rightarrow ((r \rightarrow \neg q) \rightarrow (r \rightarrow p)) \label{eq:taut5}
  \end{align}
  \item Formulierung der schwachen $\exists$-Einführung:
  \begin{align*}
    \frac{\Phi \vdash \varphi \rightarrow  \psi[x/t]}{\Phi \vdash \varphi \rightarrow \exists x \psi},
    \qquad \text{wenn die Substitution erlaubt ist.}
  \end{align*}
  Nun der halbformale Beweis (Tautologien bitte dazudenken):
  \begin{flalign*}
    &\Phi \vdash \neg \psi[x/t] \rightarrow \neg \varphi & \text{(Tautologie \eqref{eq:taut3} + MP)} \\
    &\Phi \vdash \forall x \neg \psi \rightarrow \neg \psi[x/t] & \text{(Substitutions-Axiom)} \\
    &\Phi \vdash \forall x \neg \psi \rightarrow \neg \varphi & \text{(Tautologie \eqref{eq:taut4} + MP)} \\
    &\Phi \vdash \varphi \rightarrow \exists x \psi & \text{(Existenzaxiom + Tautologien \eqref{eq:taut3} und \eqref{eq:taut4} + MP)} \\
  \end{flalign*}
\end{enumerate}


\end{solution}


% --------------------------------------------------------------------------------
