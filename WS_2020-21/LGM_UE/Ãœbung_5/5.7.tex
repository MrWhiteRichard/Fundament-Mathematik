% -------------------------------------------------------------------------------- %

\begin{exercise}[117]

Sei $\mathscr{L}$ eine (möglicherweise überabzählbare) Sprache der Prädikatenlogik,
$\Sigma_0$ eine konsistente Theorie in $\mathscr{L}$. Sei $P$ die Menge aller konsistenten
Theorien $\Sigma \supseteq \Sigma_0$ in der Sprache $\mathscr{L}$. Durch die
Relation $\subseteq$ wird $P$ partiell geordnet. \\
Zeigen Sie:
\begin{itemize}
  \item Jede Kette in $P$ (also jede durch $\subseteq$ total geordnete Teilmenge $K \subseteq P$)
  ist beschränkt. (Das heißt, für jede Kette $K \subseteq P$ gibt es $\Sigma^* \in P$,
  sodass für alle $\Sigma \in K$ die Beziehung $\Sigma \subseteq \Sigma^*$ gilt.)
  \item Wenn $\Sigma \in P_{\mathscr{L}}$ maximal ist (also: Es gibt kein
  $\Sigma^{\prime} \subsetneq$ in $P$), dann ist $\Sigma$ vollständig.
  \item Schließen Sie aus dem Lemma von Zorn, dass es zu jeder konsistenten Theorie
  $\Sigma_0$ eine vollständige konsistente Theorie $\Sigma_1 \supseteq \Sigma_0$ gibt.
\end{itemize}
\end{exercise}

% -------------------------------------------------------------------------------- %

\begin{solution}

\phantom{}

\end{solution}

% -------------------------------------------------------------------------------- %
