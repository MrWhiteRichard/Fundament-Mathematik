% --------------------------------------------------------------------------------

\begin{exercise}[100]

Sei $\psi$ eine Formel mit der freien Variable $z$. Seien $x$ und $y$ zwei
(verschiedene) Variablen, die nicht in $\psi$ vorkommen. Wir schreiben $\psi(x)$
und $\psi(y)$ statt $\psi(z/x)$ bzw. $\psi(z/y)$. \\
Zeigen Sie $\neg \exists y \psi(y) \vdash \exists y \psi(y) \rightarrow \chi$
und $\exists x  \psi(x) \vdash \exists x (\varphi \rightarrow \psi(x))$
für alle Formeln $\chi, \varphi$ und schließen Sie daraus
$\vdash \exists x (\exists y \psi(y) \rightarrow \psi(x))$. \\
(Die Formel $\exists y A \rightarrow B$) wird als $(\exists y A) \rightarrow B$ gelesen.)

\end{exercise}

% --------------------------------------------------------------------------------

\begin{solution}
Formaler Beweis für $\neg \exists y \psi(y) \vdash \exists y \psi(y) \rightarrow \chi$:
\begin{flalign*}
1&: \neg \exists y \psi(y) & \\
2&: \neg \exists y \psi(y) \rightarrow \exists y \psi(y) \rightarrow \chi & \text{(Tautologie)} \\
3&: \exists y \psi(y) \rightarrow \chi & \text{MP(2,1)}
\end{flalign*}
Formaler Beweis für $\exists x  \psi(x) \vdash \exists x (\varphi \rightarrow \psi(x))$:
\begin{flalign*}
1&: \exists x \psi(x) & \\
2&: \exists x \psi(x) \rightarrow \neg \forall x \neg \psi(x) & \text{(Existenz-Axiom)} \\
3&: \forall x( \neg (\varphi \rightarrow \psi(x)) \rightarrow \neg \psi(x)) & \text{(Allquantor vor Tautologie = Axiom)} \\
4&: \forall x( \neg (\varphi \rightarrow \psi(x)) \rightarrow \neg \psi(x)) \rightarrow
\forall x( \neg (\varphi \rightarrow \psi(x))) \rightarrow \forall x \neg \psi(x) & \text{(Distributivität)} \\
5&: \forall x( \neg (\varphi \rightarrow \psi(x))) \rightarrow \forall x \neg \psi(x) & \text{(MP(4,3))} \\
6&: (\forall x( \neg (\varphi \rightarrow \psi(x))) \rightarrow \forall x \neg \psi(x))
\rightarrow (\neg \forall x \neg \psi(x) \rightarrow \neg \forall x( \neg (\varphi \rightarrow \psi(x)))) & \text{(Tautologie)} \\
7&: \neg \forall x \neg \psi(x) \rightarrow \neg \forall x( \neg (\varphi \rightarrow \psi(x))) & \text{(MP(6,5))} \\
8&: \neg \forall x \neg \psi(x) & \text{(MP(2,1))} \\
9&: \neg \forall x( \neg (\varphi \rightarrow \psi(x))) & \text{(MP(7,8))} \\
10&: \neg \forall x( \neg (\varphi \rightarrow \psi(x))) \rightarrow \exists x (\varphi \rightarrow \psi(x))
& \text{(Existenz-Axiom)} \\
11&: \exists x (\varphi \rightarrow \psi(x)) & \text{(MP(10,9))}
\end{flalign*}
Für den (halbformalen) Beweis von $\vdash \exists x (\exists y \psi(y) \rightarrow \psi(x))$
verwenden wir zunächst zweimal das Deduktionstheorem, sowie eine schwache Einführung
des Existenzquantors und erhalten
für die Wahl $\chi = \phi(x)$ und $\varphi = \exists y \psi(y)$:
\begin{flalign*}
  &\vdash \neg \exists y \psi(y) \rightarrow \exists y \psi(y) \rightarrow \psi(x) & \text{(Deduktion)}\\
  &\vdash \neg \exists y \psi(y) \rightarrow \exists z(\exists y \psi(y) \rightarrow \psi) &
  \text{(schwache Einführung von $\exists$)}\\
  & \vdash \exists x  \psi(x) \rightarrow \exists x (\exists y \psi(y) \rightarrow \psi(x)) & \text{(Deduktion)}\\
\end{flalign*}
Weiters wissen wir bereits $\vdash \exists x \psi(x) \rightarrow \exists y \psi(y)$.
Damit ist die Aussage mittels Beweis durch Fallunterscheidung gezeigt.
\end{solution}

% --------------------------------------------------------------------------------
