% --------------------------------------------------------------------------------

\begin{exercise}[114]

Die 3 Äquivalenzen in der vorigen Aufgabe lassen sich in der Form von 6
Implikationen schreiben. Welche dieser 6 Implikationen
\begin{itemize}
  \item ... gelten für alle Theorien $\Sigma$?
  \item ... gelten für alle konsistenten Theorien $\Sigma$?
  \item ... implizieren Vollständigkeit von $\Sigma$? (Unter der Voraussetzung,
  dass $\Sigma$ konsistent ist.)
\end{itemize}
\end{exercise}

% --------------------------------------------------------------------------------

\begin{solution}

Sicher gelten die Folgenden Implikationen, denn da kann man den Beweis einfach übernehmen:
\begin{itemize}
  \item Rückrichtungen von 1. und Äquivalenz von 2.
  \item Rückrichtungen von 1. und Äquivalenz von 2. und Hinrichtung von 3.
  \item Hinrichtung von 1. und Rückrichtung von 3.
\end{itemize}
Sei nun eine konsistente Theorie $\Sigma$ gegeben für welche die Hinrichtung von 1. gilt. Wir wählen eine beliebige geschlossene Formel $A$. Es gilt $\Sigma \vdash A \lor \neg A$, da es sich um eine Tautologie handelt. Wegen der Hinrichtung von 1. gilt also $\Sigma \vdash A$ oder $\Sigma \vdash \neg A$. Gälten beide so könnten wir mit der Tautologie $A \rightarrow (\neg A \rightarrow \perp)$ auch den Beweis $\Sigma \rightarrow \perp$ führen, ein Widerspruch zur konsitenz. Also impliziert auch die Hinrichtung von 1. schon Vollständigkeit. \\
Als Gegenbeispiel dafür, dass die Hinrichtung von 3. nicht für alle Theorien gilt nehmen wir eine beliebige geschlossene Formel $\sigma$ und definieren $\Sigma := \Bbraces{\sigma, \neg \sigma}$. Dann gilt natürlich $\Sigma \vdash \sigma$ und $\Sigma \vdash \neg \sigma$. \\
Als Gegenbeispiel für eine konsistente Theorie, für welche die Hinrichtung von 1. und die Rückrichtung nicht beweisbar ist, nehmen wir eine beliebige konsistente und nicht vollständige Theorie $\Sigma$. Darin finden wir eine Aussage $A$ mit $\Sigma \nvdash A$ und $\Sigma \nvdash \neg A$. Nun gilt aber, da es sich bei $A \lor \neg A$ um eine Tautologie handelt, $\Sigma \vdash A \lor \neg A$.

\end{solution}

% --------------------------------------------------------------------------------
