% --------------------------------------------------------------------------------

\begin{exercise}

In einer Menge von $n$ Personen können 10 Personen Deutsch, 9 Englisch, 9 Russisch,
5 Deutsch und Englisch, 7 Deutsch und Russisch, 4 Englisch und Russisch, 3 alle drei
Sprachen. Wie groß ist $n$? \\
(Hinweis: Prinzip von Inklusion und Exklusion.)

\end{exercise}

% --------------------------------------------------------------------------------

\begin{solution}
Wir definieren die Mengen $A_1,A_2,A_3$ jeweils als die Menge aller Deutsch-,
Englisch-, Russisch-sprachler und erhalten direkt über das Prinzip von Inklusion und Exklusion.
\begin{align*}
  n = 10 + 9 + 9 - 5 - 7 - 4 + 3 = 15.
\end{align*}

\end{solution}

% --------------------------------------------------------------------------------
