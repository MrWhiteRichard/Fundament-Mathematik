% --------------------------------------------------------------------------------

\begin{exercise}

Gegeben sei die Adjazenzmatrix eines gerichteten Graphens $G = (V,E)$, welcher keine Schlingen (also Kanten $(v,v), v \in V$) und keine Mehrfachkanten enthält.
Eine universelle Senke in solch einem gerichteten Graphen $G$ ist ein Knoten $s$ mit Hingrad $d^-(s) = |V| - 1$ und Weggrad $d^+(s) = 0$.
Man zeige, dass es möglich ist, durch Untersuchen der Adjazenzmatrix $A$ in Laufzeit $O(|V|)$ festzustellen, ob $G$ solch eine universelle Senke enthält oder nicht.

\end{exercise}

% --------------------------------------------------------------------------------

\begin{solution}

\phantom{}

\includegraphicsboxed{Definition 2-1}
\includegraphicsboxed{Definition 2-3}

Offensichtlich sind universelle Senken (u.S.) eindeutig.
Wir führen weiters folgende Übersetzungen durch:

\begin{itemize}

  \item keine Schlingen: \\
  d.h. alle Diagonaleinträge sin $0$ \\
  d.h. $\Forall i = 1, \dots, |V|: A_{ii} = 0$

  \item keine Mehrfachkanten:\\
  d.h. gespiegelte $1$-Einträge sind $0$-Einträge \\
  d.h. $\Forall i, j = 1, \dots, |V|: A_{ij} = 1 \implies A_{j, i} = 0$

  \item $s$ u.S. \\
  d.h. $s$-te Zeile hat nur $0$er \& $s$-te Spalte hat genau einen $0$er \\
  d.h. $A_s^T = 0 \land  A_s = (\underbrace{1, \dots, 1}_{(s-1) \text{-mal}}, 0, \underbrace{1, \dots, 1}_{|V| \text{-mal}})^T$

\end{itemize}

\begin{align*}
  \begin{array}{c|ccccccccccc}
            & 1 & 2 & 3 & \cdots & s - 1 & s      & s + 1 & \cdots & |V| - 2 & |V| - 1 & |V| \\
    \hline
    1       & 0 &   &   &        &       & 1      &       &        &         &         &     \\
    2       &   & 0 &   &        &       & 1      &       &        &         &         &     \\
    3       &   &   & 0 &        &       & 1      &       &        &         &         &     \\
    \vdots  &   &   &   & \ddots &       & \vdots &       &        &         &         &     \\
    s - 1   &   &   &   &        & 0     & 1      &       &        &         &         &     \\
    s       & 0 & 0 & 0 & \cdots & 0     & 0      & 0     & \cdots & 0       & 0       & 0   \\
    s + 1   &   &   &   &        &       & 1      & 0     &        &         &         &     \\
    \vdots  &   &   &   &        &       & \vdots &       & \ddots &         &         &     \\
    |V| - 2 &   &   &   &        &       & 1      &       &        & 0       &         &     \\
    |V| - 1 &   &   &   &        &       & 1      &       &        &         & 0       &     \\
    |V|     &   &   &   &        &       & 1      &       &        &         &         & 0
  \end{array}
\end{align*}

\begin{flalign*}
   1&: \textbf{Prozedur}~ \textsc{universelle Senke} (G) \\
   2&: \quad (V, E) := G \\
   3&: \quad A := \textsc{Adjazenzmatrix}(G) \\
   4&: \quad i := 1 \\
   5&: \quad j := 2 \\
   6&: \quad \textbf{Solange}~ \max \Bbraces{i, j} \leq |V| \\
   7&: \quad \quad \textbf{Wenn}~ A_{ij} = 0 \\
    &: \quad \quad \implies \text{$j$-te Spalte hat mehr als (nicht genau) einen $0$er!} \\
    &: \quad \quad \implies \text{$j$ keine u.S.} \\
   8&: \quad \quad \quad s := i \\
   9&: \quad \quad \quad j := \max \Bbraces{i, j} + 1 \\
  10&: \quad \quad \textbf{Sonst} \\
    &: \quad \quad \implies A_{ij} = 1 \\
    &: \quad \quad \implies \text{$i$-te Zeile hat nicht nur $0$er!} \\
    &: \quad \quad \implies \text{$i$ keine u.S.} \\
  11&: \quad \quad \quad s := j \\
  12&: \quad \quad \quad i := \max \Bbraces{i, j} + 1 \\
  13&: \quad \quad \textbf{Ende Wenn} \\
  14&: \quad \textbf{Ende Solange} \\
  15&: \quad \textbf{Für}~ r := 1, \dots, |V| \\
  16&: \quad \quad \textbf{Wenn}~ r \neq s \\
    &: \quad \quad \implies (r, s) ~\text{kein Diagonaleintrag} \\
  17&: \quad \quad \quad \textbf{Wenn}~ A_{rs} = 0 \\
    &: \quad \quad \quad \implies \text{$s$-te Spalte hat mehr als (nicht genau) einen $0$er!} \\
    &: \quad \quad \quad \implies \text{$s$ keine u.S.} \\
  18&: \quad \quad \quad \quad s := \NIL \\
  19&: \quad \quad \quad \quad \textbf{Ende Prozedur} \\
    &: \quad \quad \quad \textbf{Sonst} \\
    &: \quad \quad \quad \implies A_{rs} = 1 \\
    &: \quad \quad \quad \implies A_{sr} = 0 \\
  20&: \quad \quad \quad \textbf{Ende Wenn} \\
    &: \quad \quad \textbf{Sonst} \\
    &: \quad \quad \implies r = s =: t \\
    &: \quad \quad \implies (t, t) ~\text{Diagonaleintrag} \\
    &: \quad \quad \implies A_{tt} = 0 \\
  21&: \quad \quad \textbf{Ende Wenn} \\
  22&: \quad \textbf{Ende Für} \\
    &: \quad \implies \Forall r = 1, \dots, |V|: (A_{sr} = 0) \land (s \neq r \implies A_{rs} = 1) \\
    &: \quad \implies \text{$s$-te Zeile hat nur $0$er \& $s$-te Spalte hat genau einen $0$er} \\
    &: \quad \implies \text{$s$ u.S.} \\
  23&: \textbf{Ende Prozedur}
\end{flalign*}

\end{solution}

% --------------------------------------------------------------------------------

\begin{solution}
Wir bemerken, dass wir das Problem eine universelle Senke zu finden,
reformulieren können zu: Finde Index $i$, sodass die $i$-te Zeilensumme von $A$
gleich $0$ und die $i$-te Spaltensumme gleich $|V| - 1$ ist. \\
Anschaulich gesprochen, wird bei jedem Matrixeintrag außerhalb der Diagonale den wir überprüfen ein
Senkenkandidat ausgeschlossen.
Das liegt daran, dass für $A[i, j] = 1$ für $i$ die Bedingung, dass die Zeilensumme
$0$ sein muss verletzt wird, und für $A[i,j] = 0$ für $j$ die Bedingung, dass
die Spaltensumme gleich $|V| - 1$ sein muss verletzt wird.
Damit bleibt uns nach $n - 1$ Überprüfungen nur noch 1 möglicher Senkenkandidat übrig.
Von diesem müssen wir noch überprüfen, ob er tatsächlich die Senkenbedingung erfüllt,
was in maximal $2n$ Überprüfungen gelingt. \\
Weiter unten ist ein Algorithmus ausgeführt, der diese eben beschriebenen Schritte präzisiert.
\begin{flalign*}
  1&: \textbf{Prozedur}~ \textsc{Finde universelle Senke} (A) & \\
  2&:  \quad n := A.\textit{Zeilenanzahl} & \\
  3&:  \quad Senke := NIL & \\
  4&:  \quad i := 1 & \\
  5&:  \quad j := 2 & \\
  6&:  \quad s := 1 & \\
  7&:  \quad \textbf{Solange } \max\{i,j\}  \leq n: & \\
  8&:  \quad \quad \textbf{Wenn } A[i,j] = 0: & \\
  9&:  \quad \quad \quad j := \max\{i, j\} + 1 & \\
  10&:  \quad \quad \quad s := i & \\
  11&:  \quad \quad \textbf{Ende Wenn} & \\
  12&:  \quad \quad \textbf{Wenn } A[i,j] = 1: & \\
  13&:  \quad \quad \quad  i := \max\{i, j\} + 1 & \\
  14&:  \quad \quad \quad s := j & \\
  15&:  \quad \quad \textbf{Ende Wenn} & \\
  16&:  \quad  \textbf{Ende Solange} & \\
  17&: \quad \textbf{Für } k = 1,\dots,n: & \\
  18&: \quad \quad \textbf{Wenn } k \neq s: & \\
  19&: \quad \quad \quad \textbf{Wenn } A[k,j] = 0: & \\
  20&: \quad \quad \quad \quad s := NIL & \\
  21&:  \quad \quad \quad \textbf{Ende Wenn} & \\
  22&:  \quad \quad \textbf{Ende Wenn} & \\
  23&:  \quad  \textbf{Ende Für} & \\
  24&: \textbf{Ende Prozedur} &
\end{flalign*}
Der Solange-Block wird maximal $|V|$-Mal ausgeführt, ebenso die Für-Schleife weiter unten,
insgesamt entscheidet der Algorithmus also in $\Landau{|V|}$ Schritten, ob es eine
universelle Senke gibt.
\end{solution}

% --------------------------------------------------------------------------------

\begin{solution}
Alternativer Code:

\begin{flalign*}
   1&: \textbf{Prozedur}~ \textsc{universelle Senke} (G) \\
   2&: \quad (V, E) := G \\
   3&: \quad A := \textsc{Adjazenzmatrix}(G) \\
   4&: \quad i := 1 \\
   5&: \quad j := 2 \\
   6&: \quad s := 1 \\
   7&: \quad \textbf{Solange}~ j \leq |V| \\
   8&: \quad \quad \textbf{Wenn}~ A_{ij} = 0 \\
   9&: \quad \quad \quad j := j + 1 \\
  10&: \quad \quad \textbf{Sonst} \\
  11&: \quad \quad \quad s := j \\
  12&: \quad \quad \quad {i,j} := {j, j + 1} \\
  13&: \quad \quad \textbf{Ende Wenn} \\
  14&: \quad \textbf{Ende Solange} \\
  15&: \quad \textbf{Für}~ r := 1, \dots, |V| \textbf{und}~ r \neq s\\
  16&: \quad \quad \quad \textbf{Wenn}~ A_{rs} = 0 \\
  17&: \quad \quad \quad \quad s := \NIL \\
  18&: \quad \quad \quad \quad \textbf{Ende Prozedur} \\
  19&: \quad \quad \quad \textbf{Ende Wenn} \\
  20&: \quad \textbf{Ende Für} \\
  21&: \textbf{Ende Prozedur}
\end{flalign*}



\end{solution}

\begin{comment}

\begin{solution}
Alternativer Code:

\begin{flalign*}
   1&: \textbf{Prozedur}~ \textsc{universelle Senke} (G) \\
   2&: \quad (V, E) := G \\
   3&: \quad A := \textsc{Adjazenzmatrix}(G) \\
   4&: \quad i := 2 \\
   5&: \quad j := 1 \\
   6&: \quad s := 1 \\
   7&: \quad \textbf{Solange}~ i \leq |V| \\
   8&: \quad \quad \textbf{Wenn}~ A(i,j) = 1 \\
   9&: \quad \quad \quad i := i + 1 \\
  10&: \quad \quad \textbf{Sonst} \\
  11&: \quad \quad \quad s := i \\
  12&: \quad \quad \quad {i,j} := {i+1, i} \\
  13&: \quad \quad \textbf{Ende Wenn} \\
  14&: \quad \textbf{Ende Solange} \\
  15&: \quad \textbf{Für}~ r := 1, \dots, |V| \textbf{und}~ r \neq s\\
  16&: \quad \quad \quad \textbf{Wenn}~ A_(r,s) = 0 \\
  17&: \quad \quad \quad \quad s := \NIL \\
  18&: \quad \quad \quad \quad \textbf{Ende Prozedur} \\
  19&: \quad \quad \quad \textbf{Ende Wenn} \\
  20&: \quad \textbf{Ende Für} \\
  21&: \textbf{Ende Prozedur}
\end{flalign*}



\end{solution}

\end{comment}
