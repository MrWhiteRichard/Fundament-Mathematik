% --------------------------------------------------------------------------------

\begin{exercise}

Zum asymptotischen Vergleich von Folgen:
\begin{enumerate}[label = (\alph*)]
  \item Vergleichen Sie das asymptotische Verhalten von $f(n) = n!$ und $g(n) = (n + 2)!$,
  also überlegen Sie sich ob eine (welche) der Funktionen ein ${\scriptstyle \mathcal{O}}, \mathcal{O}, \omega, \Omega, \Theta$
  der anderen Funktion ist.
  \item Vergleichen Sie das asymptotische Verhalten von $f(n) = n^{\log_2(4)}$ und
  $g(n) = 3^{\log_2(n)}$, also überlegen Sie sich ob eine (welche) der Funktionen
  ein ${\scriptstyle \mathcal{O}}, \mathcal{O}, \omega, \Omega, \Theta$ der anderen Funktion ist.
  \item Zeigen Sie an Hand der Defintion, dass für positive Funktionen $f$ und $g$
  die Beziehung
  \begin{align*}
    \max\{f(n),g(n)\} = \Theta(f(n) + g(n))
  \end{align*}
  gilt.
  \item Gilt selbige Beziehung ebenfalls für $\min\{f(n),g(n)\}$?
  \item Folgt aus $f(n) = \Landau{g(n)}$, dass $2^{f(n)} = \Landau{2^{g(n)}}$?
  \item Gilt für alle positiven Funktionen $f$ die Beziehung $f(n) = \Landau{f(n)^2}$?
  \item Finden Sie eine Funktion $f$, sodass weder $f(n) = \Landau{n}$ noch $f(n) = \Omega(n)$ gilt.
\end{enumerate}

\end{exercise}

% --------------------------------------------------------------------------------

\begin{solution}
Definitionen:
\begin{align*}
  \Landau{f} &:= \{g: \N \to \R: \exists C > 0, n_0 \in \N: \forall n \geq n_0:
  |g(n)| \leq C|f(n)|\} \\
  \Omega(f) &:= \{g: \N \to \R: \exists C > 0, n_0 \in \N: \forall n \geq n_0:
  C|f(n)| \leq |g(n)|\} \\
  \Theta(f) &:= \{g: \N \to \R: \exists C, D > 0, n_0 \in \N: \forall n \geq n_0:
  C|f(n)| \leq |g(n)|\leq D|f(n)|\} \\
  \landau{f} &:= \{g: \N \to \R: \forall C > 0: \exists n_0 \in \N: \forall n \geq n_0:
  |g(n)| \leq C|f(n)|\} \\
  \omega(f) &:= \{g: \N \to \R: \forall C > 0: \exists n_0 \in \N: \forall n \geq n_0:
  C|f(n)| \leq |g(n)|\}
\end{align*}
\begin{enumerate}[label = (\alph*)]
  \item Es gilt
  \begin{align*}
    g(n) &= (n + 2)(n + 1)f(n) = (n^2 + 3n + 2)f(n) \\
    \limsup \left|\frac{f(n)}{g(n)}\right| &= \limsup \left|\frac{1}{n^2 + 3n + 2}\right|
    =0 \iff f = \landau{g} \iff g = \omega(f).
  \end{align*}

  \item
  Da $\log_2(3/4) < 0$ folgt
  \begin{align*}
    g(n) &= 3^{\log_2(n)}n = 3^{\log_3(n)\log_2(3)} = n^{\log_2(3)} \\
    \limsup \left|\frac{g(n)}{f(n)}\right| &= \limsup \left|\frac{n^{\log_2(3)}}{n^{\log_2(4)}}\right|
    = \limsup \left|n^{\log_2(3/4)}\right| = 0 \iff f = \landau{g} \iff g = \omega(f).
  \end{align*}
  \item Es gilt
  \begin{align*}
    \max\{f(n), g(n)\} \leq |f(n) + g(n)| \leq 2\max\{|f(n)|,|g(n)|\} = 2\max\{f(n),g(n)\}.
  \end{align*}
  \item Gegenbeispiel: $f(n) = 0, \quad g(n) = 1$.
  \item Gegenbeispiel: $f(n) = n, \quad g(n) = -n$. \\
  Es gilt $|f(n)| = |g(n)|$, aber
  \begin{align*}
    \limsup \left|\frac{2^{f(n)}}{2^{g(n)}}\right| = \limsup \left|2^{2n}\right| = \infty \iff f(n) \neq \Landau{g(n)}.
  \end{align*}
  \item Gegenbeispiel: $f(n) = 1/n$. Es gilt
  \begin{align*}
    \limsup \left|\frac{f(n)}{f(n)^2}\right| = \limsup \left|n\right| = \infty \iff f(n) \neq \Landau{f(n)^2}.
  \end{align*}
  \item Definiere
  \begin{align*}
    f(n) = \begin{cases}
      0, & n \in 2\N \\
      n^2, & n \in 2\N + 1.
    \end{cases}
  \end{align*}
  Es folgt
  \begin{align*}
    \liminf \left|\frac{f(n)}{n}\right| &= 0 \iff f \neq \Omega(n) \\
    \limsup \left|\frac{f(n)}{n}\right| &= \infty \iff f \neq \Landau{n}.
  \end{align*}
\end{enumerate}

\end{solution}

% --------------------------------------------------------------------------------
