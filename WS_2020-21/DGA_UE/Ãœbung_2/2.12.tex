% --------------------------------------------------------------------------------

\begin{exercise}

Beweisen Sie, dass es unter je neun Punkten in einem Würfel der Kantenlänge $2$ stets zwei Punkte gibt, deren Abstand höchstens $\sqrt{3}$ ist.

(Hinweis: Schubfachprinzip)

\end{exercise}

% --------------------------------------------------------------------------------

\begin{solution}

Man partitioniere den Würfel $W$ in $8$ Teilwürfel aus $\mathcal{W} := \Bbraces{W_{abc}: a, b, c = 1, 2}$ der Kantenlänge $1$.

\begin{align*}
    \implies
    W = \sum_{a, b, c = 1}^2 W_{abc}
\end{align*}

Sei $P := \Bbraces{p_1, \dots, p_9} \subseteq W$ die Menge der $9$ Punkte.
Nun ist aber $|P| = 9 > 8 = |\mathcal{W}|$.
Laut dem Schubfachprinzip existiert daher keine injektive Funktion $P \to \mathcal{W}$.
Das bedeutet, es gibt einen Teilwürfel $\widetilde{W} \in \mathcal{W}$, in welchem mindestens zwei verschiedene Punkte $\widetilde{p_1}, \widetilde{p_2} \in P \cap \widetilde{W}$ zu finden sind.
Der maximale Abstand zweier Punkte im Einheitswürfel (also auch in $\widetilde{W} \subseteq \R^3$) ist $\diam[2]{\widetilde{W}} = \sqrt{3}$.
Insbesondere, ist daher auch $d_2(\widetilde{p_1}, \widetilde{p_2}) \leq \sqrt{3}$.

\end{solution}

% --------------------------------------------------------------------------------
