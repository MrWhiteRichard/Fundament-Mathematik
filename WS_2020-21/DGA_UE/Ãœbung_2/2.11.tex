% --------------------------------------------------------------------------------

\begin{exercise}

Sei $G$ der vollständige Graph auf 6 Knoten (also eine Kante zwischen je 2 von ihnen).
Jede Kante ist rot oder blau gefärbt. Zeigen Sie: Es existiert in dem Graphen
ein Dreieck (induzierter Teilgraph mit 3 Knoten) mit nur roten Kanten oder ein Dreieck
mit nur blauen Kanten. \\
(Hinweis: Schubfachprinzip)
\end{exercise}

% --------------------------------------------------------------------------------

\begin{solution}

Wir bemerken zuerst, dass es insgesamt 15 Kanten in $G$ gibt. Teile $G$
nun in die beiden Graphen $G_{\text{blau}}$ und $G_{\text{rot}}$ auf, wobei
die beiden neuen Graphen alle Knoten und genau die Kanten ihrer jeweiligen Farbe
enthalten. Nun muss einer dieser beiden Graphen mindestens 8 Kanten enthalten. \\
Da wir wissen, dass für einen endlichen, nichtleeren, zyklenfreien Graphen $G = (V,E)$
\begin{align*}
  |E| \leq |V| - 1
\end{align*}
gilt, folgt daraus, dass einer der beiden Graphen nicht zyklenfrei sein kann,
was genau bedeutet, dass es ein Dreieck mit nur roten oder nur blauen Kanten gibt.
\end{solution}

% --------------------------------------------------------------------------------
