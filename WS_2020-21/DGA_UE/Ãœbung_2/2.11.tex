% --------------------------------------------------------------------------------

\begin{exercise}

Sei $G$ der vollständige Graph auf $6$ Knoten (also eine Kante zwischen je $2$ von ihnen).
Jede Kante ist rot oder blau gefärbt.
Zeigen Sie:
Es existiert in dem Graphen ein Dreieck (induzierter Teilgraph mit $3$ Knoten) mit nur roten Kanten oder ein Dreieck mit nur blauen Kanten.

(Hinweis: Schubfachprinzip)

\end{exercise}

% --------------------------------------------------------------------------------

\begin{solution}


\includegraphicsboxed[Schubfachprinzip]{Schubfachprinzip.png}

Bezeichne die Knoten von $G$ mit $A,B,C,D,E,F$. \\
Vom Knoten $A$ gehen insgesamt 5 Kanten aus, also gibt es mindestens 3 Kanten, die die gleiche Farbe haben.
O.b.d.A. seien die Kanten $\overline{AB}, \overline{AC}, \overline{AD}$ allesamt rot. \\
Nun betrachte die Kanten $\overline{BC}, \overline{CD}, \overline{BD}$. \\
Fall 1: Alle diese Kanten sind blau. Dann besteht das Dreieck $\overline{BCD}$ nur aus blauen Kanten. \\
Fall 2: Es existiert eine rote Kante unter den dreien: Dann besteht eines
der Dreiecke $\overline{ABC}, \overline{ABD}, \overline{ACD}$ nur aus roten Kanten.

\end{solution}

% --------------------------------------------------------------------------------
