% --------------------------------------------------------------------------------

\begin{exercise}

Sei $G$ der vollständige Graph auf $6$ Knoten (also eine Kante zwischen je $2$ von ihnen).
Jede Kante ist rot oder blau gefärbt.
Zeigen Sie:
Es existiert in dem Graphen ein Dreieck (induzierter Teilgraph mit $3$ Knoten) mit nur roten Kanten oder ein Dreieck mit nur blauen Kanten.

(Hinweis: Schubfachprinzip)

\end{exercise}

% --------------------------------------------------------------------------------

\begin{solution}

Seien $V$ die Ecken (\Quote{vertices}) und $E$ die Kanten (\Quote{edges}) von $G$.
Offensichtlich ist $|V| = 6$.
Jede Ecke ist mit jeder anderen ($5$ Stück) verbunden.
Wenn wir ein Eck wegnehmen und seine Kanten \Quote{ernten} (vgl. \Quote{Blätter} $\rightsquigarrow$ \Quote{Früchte}?), dann verlieren die übrigen Ecken jeweils eine Kante.
Insgesamt erntet man daher nicht $6 \cdot 5 = 30$ Kanten, sondern bloß die Hälfte.

\begin{align*}
  \implies
  |E|
  =
  \sum_{n=6}^1 (n - 1)
  =
  \sum_{n=1}^6 (n - 1)
  =
  \sum_{n=0}^5 n
  =
  \sum_{n=1}^5 n
  =
  \frac{5 (5 + 1)}{2}
  =
  15
\end{align*}

Sei $f$ eine von $2^{15}$ beliebigen \Quote{Färbungs-Funktionen}.

\begin{align*}
  f:
  E \to \Bbraces{\text{blau}, \text{rot}}:
  e \mapsto ~\text{Farbe von}~ e
\end{align*}

Um das Schubfachprinzip anzuwenden, teile $G$ nun in die beiden Graphen $G_{\text{blau}}$ und $G_{\text{rot}}$ auf.

\begin{align*}
  G_{\text{blau}} := (V_{\text{blau}}, E_{\text{blau}}) := (V, f^{-1}(\text{blau})),
  \quad
  G_{\text{rot}}  := (V_{\text{rot}}, E_{\text{rot}})   :=(V, f^{-1}(\text{rot}))
  \quad
  \implies
  E_{\text{rot}} + E_{\text{blue}} = E
\end{align*}

Die beiden neuen Graphen enthalten alle Knoten von $G$, und genau die Kanten ihrer jeweiligen Farbe enthalten.

Weil $|E_{\text{blau}}| + |E_{\text{rot}}| = |E| = 15$, muss einer dieser beiden Graphen, laut Schubfachprinzip, mindestens $8$ Kanten enthalten.
o.B.d.A sei dies $G_{\text{blau}} = (V_{\text{blau}}, E_{\text{blau}})$.

\includegraphicsboxed{Lemma 2.2.png}

Weil $G_{\text{blau}}$ endlich und nicht-leer ist, aber $|E_{\text{blau}}| \geq 8 > 5 = 6 - 1 = |V| - 1$, ist $G_{\text{blau}}$ nicht, laut Lemma 2.2, nicht zyklenfrei.
Das bedeutet, dass es ein Dreieck in $G_{\text{blau}}$ mit nur blauen Kanten gibt.

\end{solution}

% --------------------------------------------------------------------------------
