% --------------------------------------------------------------------------------

\begin{exercise}

Erklären Sie die Funktionsweise der Algorithmen von Kruskal und Prim anhand der Konstruktion eines maximalen Spannbaums (eines spannenden Baums mit maximalen Gewicht) mittels Kruskal's Algorithmus und eines minimalen Spannbaums mittels Prim's Algorithmus für den Wurzelknoten $s$ im folgenden kantenbewerteten Graph:

\begin{center}
    \input{7.38_tikz/7.38_tikz_picture_0.tex}
\end{center}

% \begin{center}
%     \includegraphics[width = 0.5 \textwidth]{7.38.png}
% \end{center}

\end{exercise}

% --------------------------------------------------------------------------------

\begin{solution}

\phantom{}

\begin{enumerate}[label = \arabic*.]

    \item Algorithmus (von Kruskal):

    Der Algorithmus von Kruskal erhält als Eingabe einen zusammenhängenden endlichen (möglicherweise sogar gerichteten) Graphen $G = (V, E)$ und eine Kostenfunktion $c: E  \to \R^+$.

    \begin{align*}
        G = (V, E),
        \quad
        V = \Bbraces{a, b, c, d, e, f, g, h, i, j, k, l, m, n, s},
    \end{align*}

    \begin{multline*}
        E =
        \{
            \Bbraces{a, b},
            \Bbraces{a, d},
            \Bbraces{a, s},
            \Bbraces{b, c},
            \Bbraces{b, d},
            \Bbraces{c, e},
            \Bbraces{c, s},
            \Bbraces{d, e},
            \Bbraces{e, f},
            \Bbraces{e, g},
            \Bbraces{e, h}, \\
            \Bbraces{g, h},
            \Bbraces{h, i},
            \Bbraces{h, j},
            \Bbraces{h, k},
            \Bbraces{h, l},
            \Bbraces{i, l},
            \Bbraces{j, k},
            \Bbraces{j, m},
            \Bbraces{k, l},
            \Bbraces{k, m},
            \Bbraces{l, n},
            \Bbraces{m, n}
        \}
    \end{multline*}

    \begin{align*}
        \begin{matrix}
            c(\Bbraces{a, b}) = 8, & c(\Bbraces{a, d}) = 5, & c(\Bbraces{a, s}) = 2, & \\
            c(\Bbraces{b, c}) = 3, & c(\Bbraces{b, d}) = 3, & & \\
            c(\Bbraces{c, e}) = 4, & c(\Bbraces{c, s}) = 2, & & \\
            c(\Bbraces{d, e}) = 6, & & & \\
            c(\Bbraces{e, f}) = 8, & c(\Bbraces{e, g}) = 7, & c(\Bbraces{e, h}) = 5, & \\
            c(\Bbraces{g, h}) = 3, & & & \\
            c(\Bbraces{h, i}) = 2, & c(\Bbraces{h, j}) = 1, & c(\Bbraces{h, k}) = 2, & c(\Bbraces{h, l}) = 6, \\
            c(\Bbraces{i, l}) = 4, & & & \\
            c(\Bbraces{j, k}) = 3, & c(\Bbraces{j, m}) = 6, & & \\
            c(\Bbraces{k, l}) = 5, & c(\Bbraces{k, m}) = 5, & & \\
            c(\Bbraces{l, n}) = 3, & & & \\
            c(\Bbraces{m, n}) = 7  & & &
        \end{matrix}
    \end{align*}    

    $B$ wird die Kanten-Menge des gewünschten minimalen Spannbaums von $G$.
    Diese startet mit der leeren Menge $\emptyset$ und es werden sukzessive passende Kanten hinzugefügt.

    Zuerst wird allerdings der Graph in einen Wald $P$ aus $1$-punktigen Bäumen partitioniert.
    
    \begin{multline*}
        P_0
        :=
        \{
            (\Bbraces{a}, \emptyset),
            (\Bbraces{b}, \emptyset),
            (\Bbraces{c}, \emptyset),
            (\Bbraces{d}, \emptyset),
            (\Bbraces{e}, \emptyset),
            (\Bbraces{f}, \emptyset),
            (\Bbraces{g}, \emptyset), \\
            (\Bbraces{h}, \emptyset),
            (\Bbraces{i}, \emptyset),
            (\Bbraces{j}, \emptyset),
            (\Bbraces{k}, \emptyset),
            (\Bbraces{l}, \emptyset),
            (\Bbraces{m}, \emptyset),
            (\Bbraces{n}, \emptyset),
            (\Bbraces{s}, \emptyset)
        \}
    \end{multline*}
    
    Diese Bäume werden mit passenden Kanten $\in E$ zu einem minimalen Spannbaum verknüpft.
    Dazu iteriert man über das aus $E$ bzgl. $c$ von oben nach unten sortierte Datenfeld.

    \begin{align*}
        \begin{matrix}
            c(\Bbraces{a, b}) = 8 & \geq & c(\Bbraces{e, f}) = 8 & \geq &                       &      &                       &      &                       &      \\
            c(\Bbraces{e, g}) = 7 & \geq & c(\Bbraces{m, n}) = 7 & \geq &                       &      &                       &      &                       &      \\
            c(\Bbraces{d, e}) = 6 & \geq & c(\Bbraces{h, l}) = 6 & \geq & c(\Bbraces{j, m}) = 6 & \geq &                       &      &                       &      \\
            c(\Bbraces{a, d}) = 5 & \geq & c(\Bbraces{e, h}) = 5 & \geq & c(\Bbraces{k, l}) = 5 & \geq & c(\Bbraces{k, m}) = 5 & \geq &                       &      \\
            c(\Bbraces{c, e}) = 4 & \geq & c(\Bbraces{i, l}) = 4 & \geq &                       &      &                       &      &                       &      \\
            c(\Bbraces{b, c}) = 3 & \geq & c(\Bbraces{b, d}) = 3 & \geq & c(\Bbraces{g, h}) = 3 & \geq & c(\Bbraces{j, k}) = 3 & \geq & c(\Bbraces{l, n}) = 3 & \geq \\
            c(\Bbraces{a, s}) = 2 & \geq & c(\Bbraces{c, s}) = 2 & \geq & c(\Bbraces{h, i}) = 2 & \geq & c(\Bbraces{h, k}) = 2 & \geq &                       &      \\
            c(\Bbraces{h, j}) = 1 &      &                       &      &                       &      &                       &      &                       & 
        \end{matrix}
    \end{align*}

    \begin{multline*}
        E \mapsto
        [
            \Bbraces{a, b},
            \Bbraces{e, f},
            \Bbraces{e, g},
            \Bbraces{m, n},
            \Bbraces{d, e},
            \Bbraces{h, l},
            \Bbraces{j, m},
            \Bbraces{a, d},
            \Bbraces{e, h},
            \Bbraces{k, l},
            \Bbraces{k, m}, \\
            \Bbraces{c, e},
            \Bbraces{i, l},
            \Bbraces{b, c},
            \Bbraces{b, d},
            \Bbraces{g, h},
            \Bbraces{j, k},
            \Bbraces{l, n},
            \Bbraces{a, s},
            \Bbraces{c, s},
            \Bbraces{h, i},
            \Bbraces{h, k},
            \Bbraces{h, j}
        ]
    \end{multline*}    

    Es werden also am ehesten jene Kanten als Verknüpfung verwendet, die maximale Kosten liefern.
    Wir führen exemplarisch ein paar Schritte durch, sodass alle ($2$) Fälle abgedeckt sind.

    \begin{multline*}
        P_0
        \stackrel
        {
            \Bbraces{a, b}
        }{
            \mapsto
        }
        P_1 :=
        \{
            (\Bbraces{a, b}, \Bbraces{\Bbraces{a, b}}),
            (\Bbraces{c}, \emptyset),
            (\Bbraces{d}, \emptyset),
            (\Bbraces{e}, \emptyset),
            (\Bbraces{f}, \emptyset),
            (\Bbraces{g}, \emptyset), \\
            (\Bbraces{h}, \emptyset),
            (\Bbraces{i}, \emptyset),
            (\Bbraces{j}, \emptyset),
            (\Bbraces{k}, \emptyset),
            (\Bbraces{l}, \emptyset),
            (\Bbraces{m}, \emptyset),
            (\Bbraces{n}, \emptyset),
            (\Bbraces{s}, \emptyset)
        \}
    \end{multline*}

    \begin{multline*}
        P_1
        \stackrel
        {
            \Bbraces{e, f}
        }{
            \mapsto
        }
        P_2 :=
        \{
            (\Bbraces{a, b}, \Bbraces{\Bbraces{a, b}}),
            (\Bbraces{c}, \emptyset),
            (\Bbraces{d}, \emptyset),
            (\Bbraces{e, f}, \Bbraces{\Bbraces{e, f}}),
            (\Bbraces{g}, \emptyset), \\
            (\Bbraces{h}, \emptyset),
            (\Bbraces{i}, \emptyset),
            (\Bbraces{j}, \emptyset),
            (\Bbraces{k}, \emptyset),
            (\Bbraces{l}, \emptyset),
            (\Bbraces{m}, \emptyset),
            (\Bbraces{n}, \emptyset),
            (\Bbraces{s}, \emptyset)
        \}
    \end{multline*}

    \begin{multline*}
        P_2
        \stackrel
        {
            \Bbraces{e, g}
        }{
            \mapsto
        }
        P_3 :=
        \{
            (\Bbraces{a, b}, \Bbraces{\Bbraces{a, b}}),
            (\Bbraces{c}, \emptyset),
            (\Bbraces{d}, \emptyset),
            (\Bbraces{e, f, g}, \Bbraces{\Bbraces{e, f}, \Bbraces{e, g}}), \\
            (\Bbraces{h}, \emptyset),
            (\Bbraces{i}, \emptyset),
            (\Bbraces{j}, \emptyset),
            (\Bbraces{k}, \emptyset),
            (\Bbraces{l}, \emptyset),
            (\Bbraces{m}, \emptyset),
            (\Bbraces{n}, \emptyset),
            (\Bbraces{s}, \emptyset)
        \}
    \end{multline*}

    \begin{multline*}
        P_3
        \stackrel
        {
            \Bbraces{m, n}
        }{
            \mapsto
        }
        P_4 :=
        \{
            (\Bbraces{a, b}, \Bbraces{\Bbraces{a, b}}),
            (\Bbraces{c}, \emptyset),
            (\Bbraces{d}, \emptyset),
            (\Bbraces{e, f, g}, \Bbraces{\Bbraces{e, f}, \Bbraces{e, g}}), \\
            (\Bbraces{h}, \emptyset),
            (\Bbraces{i}, \emptyset),
            (\Bbraces{j}, \emptyset),
            (\Bbraces{k}, \emptyset),
            (\Bbraces{l}, \emptyset),
            (\Bbraces{m, n}, \Bbraces{\Bbraces{m, n}}),
            (\Bbraces{s}, \emptyset)
        \}
    \end{multline*}

    \begin{multline*}
        P_4
        \stackrel
        {
            \Bbraces{d, e}
        }{
            \mapsto
        }
        P_5 :=
        \{
            (\Bbraces{a, b}, \Bbraces{\Bbraces{a, b}}),
            (\Bbraces{c}, \emptyset),
            (\Bbraces{d, e, f, g}, \Bbraces{\Bbraces{d, e}, \Bbraces{e, f}, \Bbraces{e, g}}), \\
            (\Bbraces{h}, \emptyset),
            (\Bbraces{i}, \emptyset),
            (\Bbraces{j}, \emptyset),
            (\Bbraces{k}, \emptyset),
            (\Bbraces{l}, \emptyset),
            (\Bbraces{m, n}, \Bbraces{\Bbraces{m, n}}),
            (\Bbraces{s}, \emptyset)
        \}
    \end{multline*}

    \begin{multline*}
        P_5
        \stackrel
        {
            \Bbraces{h, l}
        }{
            \mapsto
        }
        P_6 :=
        \{
            (\Bbraces{a, b}, \Bbraces{\Bbraces{a, b}}),
            (\Bbraces{c}, \emptyset),
            (\Bbraces{d, e, f, g}, \Bbraces{\Bbraces{d, e}, \Bbraces{e, f}, \Bbraces{e, g}}), \\
            (\Bbraces{h, l}, \Bbraces{\Bbraces{h, l}}),
            (\Bbraces{i}, \emptyset),
            (\Bbraces{j}, \emptyset),
            (\Bbraces{k}, \emptyset),
            (\Bbraces{m, n}, \Bbraces{\Bbraces{m, n}}),
            (\Bbraces{s}, \emptyset)
        \}
    \end{multline*}

    \begin{multline*}
        P_6
        \stackrel
        {
            \Bbraces{j, m}
        }{
            \mapsto
        }
        P_7 :=
        \{
            (\Bbraces{a, b}, \Bbraces{\Bbraces{a, b}}),
            (\Bbraces{c}, \emptyset),
            (\Bbraces{d, e, f, g}, \Bbraces{\Bbraces{d, e}, \Bbraces{e, f}, \Bbraces{e, g}}), \\
            (\Bbraces{h, l}, \Bbraces{\Bbraces{h, l}}),
            (\Bbraces{i}, \emptyset),
            (\Bbraces{j, m, n}, \Bbraces{\Bbraces{j, m}, \Bbraces{m, n}}),
            (\Bbraces{k}, \emptyset),
            (\Bbraces{s}, \emptyset)
        \}
    \end{multline*}

    \begin{multline*}
        P_7
        \stackrel
        {
            \Bbraces{a, d}
        }{
            \mapsto
        }
        P_8 :=
        \{
            (\Bbraces{a, b, d, e, f, g}, \Bbraces{\Bbraces{a, b}, \Bbraces{d, e}, \Bbraces{e, f}, \Bbraces{e, g}}),
            (\Bbraces{c}, \emptyset), \\
            (\Bbraces{h, l}, \Bbraces{\Bbraces{h, l}}),
            (\Bbraces{i}, \emptyset),
            (\Bbraces{j, m, n}, \Bbraces{\Bbraces{j, m}, \Bbraces{m, n}}),
            (\Bbraces{k}, \emptyset),
            (\Bbraces{s}, \emptyset)
        \}
    \end{multline*}

    \begin{multline*}
        P_8
        \stackrel
        {
            \Bbraces{e, h}
        }{
            \mapsto
        }
        P_9 :=
        \{
            (\Bbraces{a, b, d, e, f, g, h, l}, \Bbraces{\Bbraces{a, b}, \Bbraces{d, e}, \Bbraces{e, f}, \Bbraces{e, g}, \Bbraces{h, l}}),
            (\Bbraces{c}, \emptyset), \\
            (\Bbraces{i}, \emptyset),
            (\Bbraces{j, m, n}, \Bbraces{\Bbraces{j, m}, \Bbraces{m, n}}),
            (\Bbraces{k}, \emptyset),
            (\Bbraces{s}, \emptyset)
        \}
    \end{multline*}

    \begin{multline*}
        P_9
        \stackrel
        {
            \Bbraces{k, l}
        }{
            \mapsto
        }
        P_{10} :=
        \{
            (\Bbraces{a, b, d, e, f, g, h, k, l}, \Bbraces{\Bbraces{a, b}, \Bbraces{d, e}, \Bbraces{e, f}, \Bbraces{e, g}, \Bbraces{h, l}, \Bbraces{k, l}}),
            (\Bbraces{c}, \emptyset), \\
            (\Bbraces{i}, \emptyset),
            (\Bbraces{j, m, n}, \Bbraces{\Bbraces{j, m}, \Bbraces{m, n}}),
            (\Bbraces{s}, \emptyset)
        \}
    \end{multline*}

    \begin{multline*}
        P_{10}
        \stackrel
        {
            \Bbraces{k, m}
        }{
            \mapsto
        }
        P_{11} :=
        \{
            (\Bbraces{a, b, d, e, f, g, h, j, k, l, m, n}, \\ \Bbraces{\Bbraces{a, b}, \Bbraces{d, e}, \Bbraces{e, f}, \Bbraces{e, g}, \Bbraces{h, l}, \Bbraces{j, m}, \Bbraces{k, l}, \Bbraces{m, n}}),
            (\Bbraces{c}, \emptyset),
            (\Bbraces{i}, \emptyset),
            (\Bbraces{s}, \emptyset)
        \}
    \end{multline*}

    \begin{multline*}
        P_{11}
        \stackrel
        {
            \Bbraces{c, e}
        }{
            \mapsto
        }
        P_{12} :=
        \{
            (\Bbraces{a, b, c, d, e, f, g, h, j, k, l, m, n}, \\ \Bbraces{\Bbraces{a, b}, \Bbraces{c, e}, \Bbraces{d, e}, \Bbraces{e, f}, \Bbraces{e, g}, \Bbraces{h, l}, \Bbraces{j, m}, \Bbraces{k, l}, \Bbraces{m, n}}),
            (\Bbraces{i}, \emptyset),
            (\Bbraces{s}, \emptyset)
        \}
    \end{multline*}

    \begin{multline*}
        P_{12}
        \stackrel
        {
            \Bbraces{i, l}
        }{
            \mapsto
        }
        P_{13} :=
        \{
            (\Bbraces{a, b, c, d, e, f, g, h, i, j, k, l, m, n}, \\ \Bbraces{\Bbraces{a, b}, \Bbraces{c, e}, \Bbraces{d, e}, \Bbraces{e, f}, \Bbraces{e, g}, \Bbraces{h, l}, \Bbraces{i, l} \Bbraces{j, m}, \Bbraces{k, l}, \Bbraces{m, n}}),
            (\Bbraces{s}, \emptyset)
        \}
    \end{multline*}

    \begin{align*}
        P_{13}
        \stackrel
        {
            \Bbraces{b, c}
        }{
            \mapsto
        }
        P_{14} := P_{13}
    \end{align*}

    \begin{align*}
        \dots
    \end{align*}

    Nachdem $G$ als zusammenhängend vorausgesetzt war, werden alle ursprünglichen Äquivalenzklassen (d.h. $1$-punktige Bäume) durch eine Kante getroffen.

    \item Algorithmus (von Prim):
    
    \includegraphicsboxed{DGA/DGA - Definition 7.1.png}
    \includegraphicsboxed{DGA/DGA - Algorithmus von Prim.png}

    \begin{center}
        \input{7.38_tikz/7.38_tikz_picture_1.tex}
    \end{center}

    \begin{center}
        \input{7.38_tikz/7.38_tikz_picture_2.tex}
    \end{center}

    \begin{center}
        \input{7.38_tikz/7.38_tikz_picture_3.tex}
    \end{center}

    \begin{center}
        \input{7.38_tikz/7.38_tikz_picture_4.tex}
    \end{center}

    \begin{center}
        \input{7.38_tikz/7.38_tikz_picture_5.tex}
    \end{center}

    \begin{center}
        \input{7.38_tikz/7.38_tikz_picture_6.tex}
    \end{center}

    \begin{center}
        \input{7.38_tikz/7.38_tikz_picture_7.tex}
    \end{center}

    \begin{center}
        \input{7.38_tikz/7.38_tikz_picture_8.tex}
    \end{center}

    \begin{center}
        \input{7.38_tikz/7.38_tikz_picture_9.tex}
    \end{center}

    \begin{center}
        \input{7.38_tikz/7.38_tikz_picture_10.tex}
    \end{center}

    \begin{center}
        \input{7.38_tikz/7.38_tikz_picture_11.tex}
    \end{center}

    \begin{center}
        \input{7.38_tikz/7.38_tikz_picture_12.tex}
    \end{center}

    \begin{center}
        \input{7.38_tikz/7.38_tikz_picture_13.tex}
    \end{center}

    \begin{center}
        \input{7.38_tikz/7.38_tikz_picture_14.tex}
    \end{center}

    \begin{center}
        \input{7.38_tikz/7.38_tikz_picture_15.tex}
    \end{center}

\end{enumerate}

\end{solution}

% --------------------------------------------------------------------------------
