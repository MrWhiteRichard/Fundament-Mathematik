% -------------------------------------------------------------------------------- %

\begin{exercise}

Bestimmen Sie die minimale Anzahl von Knoten in einem balancierten Baum der Höhe $h$.

\end{exercise}

% -------------------------------------------------------------------------------- %

\begin{solution}

\phantom{}

\includegraphicsboxed{DGA/DGA - Definition (Höhe Knoten & Baum).png}
\includegraphicsboxed{DGA/DGA - Definition 5.2.png}

Der balancierte Baum $T_1$ der Höhe $h = 1$ besteht nur aus der Wurzel, besitzt also nur $|E(T_1)| = 1$ Knoten.
Der balancierte Baum $T_2$ der Höhe $h = 2$ besteht nur aus maximal $3$ Knoten.
Einen unteren Knote darf man entfernen und der Baum bleibt balanciert, d.h. $|E(T_2)| = 2$.

Für höhere Bäume $T_h$ mit $h > 2$ können wir $|E(T_h)|$ bestimmen, indem wir rekursiv vorgehen:
Der linke und rechte Teilbaum $(T_h)_\mathrm{l}$ bzw. $(T_h)_\mathrm{r}$ haben jeweils maximal Höhe $h-1$.
Einer davon darf sogar Höhe $h-2$ haben, $T_h$ bleibt balanciert und hat minimale Anzahl an Knoten.

Weil $T_h$ balanciert ist, müssen auch die Teilbäume $(T_h)_\mathrm{l}$ und $(T_h)_\mathrm{r}$ balanciert sein.
Weil $T_h$ minimale Knoten-Zahl hat, müssen auch die Teilbäume $(T_h)_\mathrm{l}$ und $(T_h)_\mathrm{r}$ minimale Knoten-Zahl haben.

\begin{align*}
  \implies
  (T_h)_\mathrm{l} = T_{h-1},
  \quad
  (T_h)_\mathrm{r} = T_{h-2},
  \quad
  \text{oder}
  \quad  
  (T_h)_\mathrm{l} = T_{h-2},
  \quad
  (T_h)_\mathrm{r} = T_{h-1}
\end{align*}

\begin{align*}
  \implies
  |E(T_h)|
  =
  |E((T_h)_\mathrm{l})| + |E((T_h)_\mathrm{r})| + 1
  =
  |E(T_{h-1})| + |E(T_{h-2})| + 1
\end{align*}

Wir haben also für die minimale Anzahl an Knoten die folgende Rekursionsgleichung.

\begin{align*}
  m_h = m_{h-1} + m_{h-2} + 1,
  \quad
  m_1 = 1,
  \quad
  m_2 = 2
\end{align*}

Wir haben es also mit einer inhomogenen Rekursionsgleichung $2$-ter Ordnung zu tun.

\includegraphicsboxed{DGA/DGA - Satz 4.3.png}

Zuerst lösen wir die zugehörige homogene Rekursionsgleichung.

\begin{align*}
  n_h = n_{h-1} + n_{h-2},
  \quad
  n_1 = 2,
  \quad
  n_2 = 3
\end{align*}

Das ist genau die, um $1$ im Index verschobene, Fibonacci-Folge.

\begin{align*}
  \implies
  n_h
  =
  F_{h+1}
  =
  \frac{1}{\sqrt{5}}
  \pbraces
  {
    \frac{1 + \sqrt{5}}{2}
  }^{h+1}
  -
  \frac{1}{\sqrt{5}}
  \pbraces
  {
    \frac{1 - \sqrt{5}}{2}
  }^{h+1}
\end{align*}

Nun wieder zur inhomogenen Rekursionsgleichung.
Diese hat schließlich die folgende Lösung.

\begin{align*}
  \implies
  m_h
  =
  n_h - 1
  =
  \frac{1}{\sqrt{5}}
  \pbraces
  {
    \frac{1 + \sqrt{5}}{2}
  }^{h+1}
  -
  \frac{1}{\sqrt{5}}
  \pbraces
  {
    \frac{1 - \sqrt{5}}{2}
  }^{h+1}
  -
  1
\end{align*}

\begin{comment}

Das charakteristische Polynom und seine Nullstellen:

\begin{align*}
  \chi(x) = x^2 - x - 1 &\stackrel{!}{=} 0
  \iff
  x_{1,2} = \frac{1}{2} \pm \frac{\sqrt{5}}{2} \\
  c_1 + c_2 &\stackrel{!}{=} 2 \\
  c_1\Big(\frac{1}{2} + \frac{\sqrt{5}}{2}\Big) + c_2 \Big(\frac{1}{2} - \frac{\sqrt{5}}{2}\Big) &\stackrel{!}{=} 3 \\
  \implies c_1 = \frac{2\sqrt{5}}{5} + 1, c_2 &= \frac{-2\sqrt{5}}{5} + 1
\end{align*}

also
\begin{align*}
  n_h = \Big(\frac{2\sqrt{5}}{5} + 1\Big)\Big(\frac{1}{2} + \frac{\sqrt{5}}{2}\Big)^h + \Big(\frac{-2\sqrt{5}}{5} + 1 \Big)\Big(\frac{1}{2} - \frac{\sqrt{5}}{2}\Big)^h \\
  \implies
  m_h = \Big(\frac{2\sqrt{5}}{5} + 1\Big)\Big(\frac{1}{2} + \frac{\sqrt{5}}{2}\Big)^h + \Big(\frac{-2\sqrt{5}}{5} + 1 \Big)\Big(\frac{1}{2} - \frac{\sqrt{5}}{2}\Big)^h - 1
\end{align*}

\end{comment}

\end{solution}

% -------------------------------------------------------------------------------- %
