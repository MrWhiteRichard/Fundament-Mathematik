% --------------------------------------------------------------------------------

\begin{exercise}

Bestimmen Sie die minimale Anzahl von Knoten in einem balancierten Baum der Höhe h.

\end{exercise}

% --------------------------------------------------------------------------------

\begin{solution}
Die Höhe $h(v)$ ist definiert als die maximale Anzahl von Knoten auf einem Pfad von $v$
zur Wurzel. Die Höhe des Baums die dann die Höhe der Wurzel. \\
Der balancierte Baum der Höhe $h = 1$ besteht nur aus der Wurzel, besitzt also nur einen Knoten.
Die minimale Anzahl von Knoten im balanciertem Baum der Höhe $h = 2$ ist klarerweise $2$.
Für höhere Bäume können wir den balancierten Baum mit minimaler Anzahl an Knoten bestimmen,
indem wir rekursiv vorgehen: Von unserer Wurzel ausgehend bilden wir links den
balancierten Baum mit minimaler Anzahl von Knoten mit Höhe $h-1$ und rechts den mit Höhe $h-2$.
Dieser Baum ist somit wieder balanciert und hat minimale Anzahl an Knoten.
Wir haben also für die minimale Anzahl an Knoten die Rekursionsgleichung

\begin{align*}
  m_h = m_{h-1} + m_{h-2} + 1, \quad m_1 = 1, m_2 = 2
\end{align*}

Wir haben es also mit einer inhomogenen Rekursionsgleichung zweiter Ordnung zu tun, wie wir diese Lösen wissen wir aus der Vorlesung. Zuerst lösen wir

\begin{align*}
  n_h = n_{h-1} + n_{h-2}, \quad n_1 = 2, n_2 = 3
\end{align*}

Das charakteristische Polynom und seine Nullstellen:

\begin{align*}
  \chi(x) = x^2 - x - 1 &\stackrel{!}{=} 0
  \iff
  x_{1,2} = \frac{1}{2} \pm \frac{\sqrt{5}}{2} \\
  c_1 + c_2 &\stackrel{!}{=} 2 \\
  c_1\Big(\frac{1}{2} + \frac{\sqrt{5}}{2}\Big) + c_2 \Big(\frac{1}{2} - \frac{\sqrt{5}}{2}\Big) &\stackrel{!}{=} 3 \\
  \implies c_1 = \frac{2\sqrt{5}}{5} + 1, c_2 &= \frac{-2\sqrt{5}}{5} + 1
\end{align*}

also
\begin{align*}
  n_h = \Big(\frac{2\sqrt{5}}{5} + 1\Big)\Big(\frac{1}{2} + \frac{\sqrt{5}}{2}\Big)^h + \Big(\frac{-2\sqrt{5}}{5} + 1 \Big)\Big(\frac{1}{2} - \frac{\sqrt{5}}{2}\Big)^h \\
  \implies
  m_h = \Big(\frac{2\sqrt{5}}{5} + 1\Big)\Big(\frac{1}{2} + \frac{\sqrt{5}}{2}\Big)^h + \Big(\frac{-2\sqrt{5}}{5} + 1 \Big)\Big(\frac{1}{2} - \frac{\sqrt{5}}{2}\Big)^h - 1
\end{align*}
\end{solution}

% --------------------------------------------------------------------------------
