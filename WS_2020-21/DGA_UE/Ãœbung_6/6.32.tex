% --------------------------------------------------------------------------------

\begin{exercise}

Geben Sie, für jedes gerade $n \in \N$ paarweise unterschiedliche Schlüssel $x_1, \dots, x_n$ an, so dass ein Suchbaum in Kammform (sh. Abbildung) entsteht wenn, beginnend mit dem leeren Suchbaum, die Schlüssel $x_1, \dots, x_n$ in dieser Reihenfolge eingefügt werden.

\end{exercise}

% --------------------------------------------------------------------------------

\begin{solution}

\phantom{}

\includegraphicsboxed{DGA/DGA - Definition 5.1.png}
\includegraphicsboxed{DGA/DGA - Algorithmus 14 - Einfügen in einen Suchbaum.png}

\begin{center}
    \begin{tikzpicture}

        \coordinate (x_1) at ( 0,  0);
        \coordinate (x_2) at (-1, -1);
        \coordinate (x_3) at ( 1, -1);
        \coordinate (x_4) at ( 0, -2);
        \coordinate (x_5) at ( 2, -2);
        \coordinate (x_6) at ( 1, -3);

        \coordinate (x_n_minus_1) at (5, -5);
        \coordinate (x_n)         at (4, -6);

        \draw (x_1) -- (x_2);
        \draw (x_3) -- (x_4);
        \draw (x_5) -- (x_6);

        \draw (x_n_minus_1) -- (x_n);

        \draw (x_1) -- (x_3) -- (x_5);
        \draw [dotted] (x_5) -- (x_n_minus_1);

        \filldraw [color = black, fill = white] (x_1) circle (12 pt) node {$x_1$};
        \filldraw [color = black, fill = white] (x_2) circle (12 pt) node {$x_2$};
        \filldraw [color = black, fill = white] (x_3) circle (12 pt) node {$x_3$};
        \filldraw [color = black, fill = white] (x_4) circle (12 pt) node {$x_4$};
        \filldraw [color = black, fill = white] (x_5) circle (12 pt) node {$x_5$};
        \filldraw [color = black, fill = white] (x_6) circle (12 pt) node {$x_6$};

        \filldraw [color = black, fill = white] (x_n_minus_1) circle (12 pt) node {$x_{n-1}$};
        \filldraw [color = black, fill = white] (x_n)         circle (12 pt) node {$x_n$};

        \draw (x_1) node [right, xshift =  15 pt] {$:= 2$};
        \draw (x_2) node [left,  xshift = -15 pt] {$1 =:$};
        \draw (x_3) node [right, xshift =  15 pt] {$:= 4$};
        \draw (x_4) node [left,  xshift = -15 pt] {$3 =:$};
        \draw (x_5) node [right, xshift =  15 pt] {$:= 6$};
        \draw (x_6) node [left,  xshift = -15 pt] {$5 =:$};

        \draw (x_n_minus_1) node [right, xshift =  15 pt] {$:= n$};
        \draw (x_n)         node [left,  xshift = -15 pt] {$n-1 =:$};

    \end{tikzpicture}


\end{center}
Also lauten unsere Schlüssel für $i \leq n$:

\begin{align*}
  x_i
  :=
  \begin{cases}
  i - 1, & \text{falls}~ i \in 2\N \\
  i + 1, & \text{falls}~ i \in 2\N +1
  \end{cases}
\end{align*}
\end{solution}

% --------------------------------------------------------------------------------
