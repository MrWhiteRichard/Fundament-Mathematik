% --------------------------------------------------------------------------------

\begin{exercise}

Ein binärer Suchbaum mit Blättern ist ein binärer Baum, sodass jeder Knoten entweder genau $2$ oder keinen Nachfolger hat.
Erstere Knoten werden \textit{innere Knoten} genannt, letztere \textit{Blätter} oder \textit{externe Knoten}.
Die \textit{interne Pfadlänge} $I(T)$ eines Binärbaums $T$ ist die Summe der Abstände (gemessen in Anzahl von Kanten) von der Wurzel zu allen internen Knoten, die \textit{externe Pfadlänge} $E(T)$ ist die Summe der Abstände zu allen externen Knoten.
Beweisen Sie:

\begin{enumerate}[label = \alph*)]

    \item Ein Binärbaum mit $n$ internen Knoten hat $n + 1$ Blätter.

    \item Zwischen interner und externer Pfadlänge besteht folgender Zusammenhang:

    \begin{align*}
        E(T) = I(T) + 2 |T|,
    \end{align*}

    wobei $|T|$ die Anzahl interner Knoten von $T$ ist.

\end{enumerate}

\end{exercise}

% --------------------------------------------------------------------------------

\begin{solution}

So ein BSB ist entweder der $1$-knotige ($0$-kantige) Baum oder besteht aus lauter verkehrt-rummen \blockquote{v}-s.

\begin{enumerate}[label = \alph*)]

    \item IA($n = 0$):

    Es gibt nur den Wurzel-Knoten.
    Dieser ist extern.

    IS($n-1 \mapsto n$):

    Sei $T$ ein BSB mit $n$ internen Knoten.
    Lösche zwei Blätter, die ein \textit{Kirschen-Paar} $\Bbraces{v_1, v_2}$ bilden, d.h. sodass wieder ein BSB $T^\prime$ entsteht.
    (Wenn es kein Kirschen-Paar gäbe, dann hätte ein Knoten von $T$ genau einen Nachfolger!)

    Der zugehörige $T$-interne Knoten, der die Kirschen verbindet $v$, wurde somit $T^\prime$-extern.
    $T^\prime$ hat also einen internen Knoten ($v$) weniger als $T$, d.h. $n - 1$.

    Laut IV hat $T^\prime$ daher $(n - 1) + 1 = n$ Blätter.
    Wenn wir das Kirschen-Paar $\Bbraces{v_1, v_2}$ wieder an $T^\prime$ drankleben (und die Kanten $\Bbraces{v, v_1}$ und $\Bbraces{v, v_2}$ ergänzen), kommt $T$ raus.
    $v$ ist nicht mehr Blatt (wir ziehen $1$ ab) und die Kirschen kommen jeweils als Blatt dazu (wir addieren $2$).

    \begin{align*}
        n - 1 + 2 = n + 1
    \end{align*}
    \textit{Alternative}:
    Die Anzahl der Kanten in unserem Baum ist genau $2n$ und daraus folgt bereits,
    dass unser Baum $2n + 1$ Knoten haben muss.
    \item Sei $T$ ein BSB.

    IA($|T| = 0$):

    Wieder trivial, alles $0$.

    IS($n-1 \mapsto n$):

    Lösche wieder ein Kirschen-Paar $\Bbraces{v_1, v_2}$, wobei die Kirschen jeweils Tiefe $m$ haben, und erhalte $T^\prime$.
    Genauso wie vorher sieht man

    \begin{align*}
        |T^\prime| = |T| - 1.
    \end{align*}

    Der $T^\prime$-externe bzw. $T$-interne Knoten $v$, der die Kirschen $v_1$ und $v_2$ verbindet hat Tiefe $m - 1$.

    \begin{align*}
        \implies
        I(T^\prime) = I(T) - (m - 1),
        \quad
        E(T^\prime) = E(T) - 2 m + (m - 1) = E(T) - (m + 1)
    \end{align*}

    Es ist Zeit, alles zusammenzustöpseln.

    \begin{multline*}
        \implies
        E(T)
        =
        E(T^\prime) + (m + 1)
        \stackrel
        {
            \mathrm{IV}
        }{=}
        I(T^\prime) + 2 |T^\prime| + m + 1 \\
        =
        I(T^\prime) + 2 (|T| - 1) + m + 1
        =
        I(T^\prime) + (m - 1) + 2 |T|
        =
        I(T) + 2 |T|
    \end{multline*}

\end{enumerate}

\end{solution}

% --------------------------------------------------------------------------------
