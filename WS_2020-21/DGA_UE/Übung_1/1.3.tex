% --------------------------------------------------------------------------------

\begin{exercise}

\phantom{}

\begin{enumerate}[label = (\alph*)]

    \item Binäre Suche:
    Gegeben sei ein (aufsteigend) sortiertes Datenfeld $A[1, \ldots, n]$ und ein Wert $x$.
    Das sogenannte Suchproblem, also einen Index $j$ mit $x = A[j]$ auszugeben, falls $x$ in $A$ enthalten ist, und einen speziellen Wert $\NIL$ auszugeben, falls $x$ nicht in $A$ vorkommt, kann hier mittels Divide-and-Conquer gelöst werden.
    Man vergleicht $x$ mit dem mittleren Element des Datenfelds und ist nach diesem Vergleichen entweder fündig geworden oder braucht nur noch das halbe Datenfeld mit der gleichen Prozedur zu durchsuchen.
    Schreiben Sie ein Programm in Pseudocode für die binäre Suche.
    Begründen Sie, warum die Laufzeit der binären Suche im schlechtesten Fall $\Landau{\log{n}}$ ist.

    \item Beim Algorithmus Einfügesortieren wird die sequentielle Suche verwendet, um das bereits sortierte Teilfeld $A[1, \ldots, n]$ (rückwärts) zu durchsuchen.
    Kann stattdessendie binäre Suche verwendet werden, um die worst-case-Laufzeit von Insertion Sort auf $\Landau{\log{n}}$ zu verbessern?

\end{enumerate}

\end{exercise}

% --------------------------------------------------------------------------------

\begin{solution}

ToDo!

\end{solution}

% --------------------------------------------------------------------------------
