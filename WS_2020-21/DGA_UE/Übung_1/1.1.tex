% --------------------------------------------------------------------------------

\begin{exercise}

Überlegen Sie sich einen Pseudocode für die folgenden Algorithmen und bestimmen Sie die Anzahl der notwendigen Schritte, die (in Ihrem Pseudocode) nötig sind, um einen-elementige Menge zu sortieren.
Wenden Sie die Algorithmen aud den Datensatz $6, 77, 45, 103, 4, 17$ an.

\begin{enumerate}[label = (\alph*)]

    \item Selection Sort:
    der Algorithmus sucht zunächst das kleinste Element und bringt es an die erste Position.
    Anschließend sucht er das zweitkleinste Element und bringtes an die zweite Position, usw.
    
    \item Bubble-Sort:
    Der Algorithmus vergleicht der Reihe nach je zwei benachbarte Zahlenund vertauscht diese, falls sie nicht in der richtigen Reihenfolge angeordnet sind.
    Dieses Verfahren wird so lange wiederholt, bis alle Zahlen der Eingabe sortiert sind.
    
\end{enumerate}

\end{exercise}

% --------------------------------------------------------------------------------

\begin{solution}

ToDo!

\end{solution}

% --------------------------------------------------------------------------------
