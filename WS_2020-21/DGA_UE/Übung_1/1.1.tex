% --------------------------------------------------------------------------------

\begin{exercise}

Überlegen Sie sich einen Pseudocode für die folgenden Algorithmen und bestimmen Sie die Anzahl der notwendigen Schritte, die (in Ihrem Pseudocode) nötig sind, um eine $n$-elementige Menge zu sortieren.
Wenden Sie die Algorithmen auf den Datensatz $6, 77, 45, 103, 4, 17$ an.

\begin{enumerate}[label = (\alph*)]

    \item Selection Sort:
    der Algorithmus sucht zunächst das kleinste Element und bringt es an die erste Position.
    Anschließend sucht er das zweitkleinste Element und bringtes an die zweite Position, usw.

    \item Bubble-Sort:
    Der Algorithmus vergleicht der Reihe nach je zwei benachbarte Zahlen und vertauscht diese, falls sie nicht in der richtigen Reihenfolge angeordnet sind.
    Dieses Verfahren wird so lange wiederholt, bis alle Zahlen der Eingabe sortiert sind.

\end{enumerate}

\end{exercise}

% --------------------------------------------------------------------------------

\begin{solution}

  \phantom{}

\begin{enumerate}[label = (\alph*)]

  \item

  \begin{flalign*}
    &1: \textbf{Prozedur}~ \textsc{Selection Sort} (A) & \\
    &2:  \quad n := A.\textit{Länge} & \\
    &3:  \quad \textbf{Für}~ i := 1, \dots, n-1 & \\
    &4:  \quad \quad i_{min} := i & \\
    &5:  \quad \quad \textbf{Für}~ j := i+1, \dots, n & \\
    &6:  \quad \quad \quad \textbf{Wenn}~ A[j] < A[i_{min}] & \\
    &7:  \quad \quad \quad \quad i_{min} := j & \\
    &8:  \quad \quad \quad \textbf{Ende Wenn}~ & \\
    &9:  \quad \quad \textbf{Ende Für}~ & \\
    &10: \quad  \quad A[i_{min}], A[i] := A[i], A[i_{min}] & \\
    &11: \quad  \textbf{Ende Für}~ & \\
    &12: \textbf{Ende Prozedur}
  \end{flalign*}

  $Z := \Bbraces{2, \dots, 7, 10}$ ist die Menge der Zeilen, die beim Algorithmus wesentliche Operationen ausführen.
  Wir können also voraussetzen dass für $z \in Z$ eine Konstante $c_z$ existiert, welche die Zeit angibt, die zur Ausführung von der $z$-ten Zeile benötigt wird.
  Sei $A$ das Eingabedatenfeld und $n$ die Länge von $A$.
  Dann belaufen sich die Kosten der Anwendung von Selection-Sort auf das Datenfeld $A$ auf

  \begin{align*}
    T(A)
    =
    c_2
    +
    \sum_{i=1}^{n-1}
    \pbraces
    {
      c_3 + c_4
      +
      \sum_{j=i+1}^n
      (
        c_5
        +
        c_6
        +
        c_7 \cdot t_{ij}
      )
      +
      c_{10}
    },
  \end{align*}

  wobei $t_{ij}$ mit $1$ oder $0$ angibt, ob die $7$-te Zeile ausgeführt wird oder nicht.
  Die $t_{ij}$ hängen offensichtlich von $A$ ab.
  Um die Anzahl der Schritte zu bekommen, setzen wir für alle $z \in Z$ einfach $c_z := 1$.

  \begin{align*}
    S(A)
    =
    1 + \sum_{i=1}^{n-1}
    \pbraces
    {
      3 + \sum_{j=i+1}^n
      (
        2 + t_{ij}
      )
    }
  \end{align*}

  Im worst-case gilt die Bedingung in der $6$-ten Zeile immer und

  \begin{align*}
    \Forall i = 1, \dots, n-1:
    \Forall j = i+1, \dots, n:
    t_{ij} = 1.
  \end{align*}

  Wir setzen dies in $S(A)$ ein und erhalten

  \begin{multline*}
    S_{\mathrm{s}}(A)
    =
    1 + \sum_{i=1}^{n-1} \pbraces{3 +\sum_{j=i+1}^n 3}
    =
    1 + \sum_{i=1}^{n-1} (3 + 3 (n - i))
    =
    1 + \sum_{i=1}^{n-1} 3 + \sum_{i=1}^{n-1} 3n - \sum_{i=1}^{n-1} 3i \\
    =
    1 + 3 (n - 1) + 3n(n-1) - 3 \frac{(n-1)n}{2}
    =
    \frac{3n^2}{2} + \frac{3n}{2} - 2
    =
    \Landau{n^2}.
  \end{multline*}

  Im best-case ist die Liste aufsteigend sortiert.
  Die Bedingung in der $6$-ten Zeile hält also nie und

  \begin{align*}
    \Forall i = 1, \dots, n-1:
    \Forall j = i+1, \dots, n:
    t_{ij} = 0.
  \end{align*}

  Wir setzen dies in $S(A)$ ein und erhalten

  \begin{multline*}
    S_{\mathrm{b}}(A)
    =
    1 + \sum_{i=1}^{n-1} \pbraces{3 +\sum_{j=i+1}^n 2}
    =
    1 + \sum_{i=1}^{n-1} (3 + 2 (n - i))
    =
    1 + \sum_{i=1}^{n-1} 3 + \sum_{i=1}^{n-1} 2 n - \sum_{i=1}^{n-1} 2i \\
    =
    1 + 3 (n - 1) + 2n (n-1) - 2 \frac{(n-1) n}{2}
    =
    n^2 + 2n - 2
    =
    \Landau{n^2}.
  \end{multline*}


  \item

  \begin{flalign*}
  &1: \textbf{Prozedur}~ \textsc{Bubble Sort}(A) & \\
  &2: \quad n := A.\textit{Länge} & \\
  &3: \quad \textbf{Für}~ i := 1, \dots, n & \\
  &4: \quad \quad \textbf{Für}~ j := 1, \dots, n-i & \\
  &5: \quad \quad \quad \textbf{Wenn}~ A[j+1] < A[j] & \\
  &6: \quad \quad \quad \quad A[j], A[j+1] := A[j+1], A[j] & \\
  &7: \quad \quad \quad \textbf{Ende Wenn} & \\
  &8: \quad \quad \textbf{Ende Für} & \\
  &9: \quad \textbf{Ende Für} & \\
  &10: \textbf{Ende Prozedur}
  \end{flalign*}

  \begin{align*}
    \implies
    T(A) = c_2 + \sum_{i=1}^n \pbraces{c_3 + \sum_{j=1}^{n-i} (c_5 + c_6 \cdot t_{ij})} \\
  \end{align*}

  \begin{align*}
    \implies
    S(A) = 1 + \sum_{i=1}^n \pbraces{1 + \sum_{j=1}^{n-i} (1 + t_{ij})}
  \end{align*}

  \begin{multline*}
    \implies
    S_{\mathrm{s}}(A)
    =
    1 + \sum_{i=1}^n \pbraces {1 + \sum_{j=1}^{n-i} (1 + 1)}
    =
    1 + \sum_{i=1}^n (1 + 2(n-i))
    =
    1 + \sum_{i=1}^n 1 + \sum_{i=1}^n 2n - \sum_{i=1}^n i \\
    =
    1 + n + 2n^2 - \frac{n (n + 1)}{2}
    =
    \frac{3n^2}{2} + \frac{n}{2} + 1
    =
    \Landau{n^2}
  \end{multline*}

  \begin{multline*}
    \implies
    S_{\mathrm{b}}(A)
    =
    1 + \sum_{i=1}^n \pbraces {1 + \sum_{j=1}^{n-i} (1 + 0)}
    =
    1 + \sum_{i=1}^n (1 + (n-i))
    =
    1 + \sum_{i=1}^n 1 + \sum_{i=1}^n n - \sum_{i=1}^n i \\
    =
    1 + n + n^2 - \frac{n (n + 1)}{2}
    =
    \frac{n^2}{2} + \frac{n}{2} + 1
    =
    \Landau{n^2}
  \end{multline*}

\end{enumerate}

\end{solution}

% --------------------------------------------------------------------------------
