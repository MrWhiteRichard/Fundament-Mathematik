% --------------------------------------------------------------------------------

\begin{exercise}

Gegeben sei die Adjazenzmatrix eines gerichteten Graphens $G = (V,E)$, welcher
keine Schlingen (also Kanten $(v,v), v \in V$) und keine Mehrfachkanten enthält.
Eine universelle Senke in solch einem gerichteten Graphen $G$ ist ein Knoten $s$
mit Hingrad $d^-(s) = |V| - 1$ und Weggrad $d^+(s) = 0$. Man zeige, dass es möglich
ist, durch Untersuchen der Adjazenzmatrix $A$ in Laufzeit $O(|V|)$ festzustellen,
ob $G$ solch eine universelle Senke enthält oder nicht. \\

\end{exercise}

% --------------------------------------------------------------------------------

\begin{solution}
Wir bemerken, dass wir das Problem eine universelle Senke zu finden,
reformulieren können zu: Finde Index $i$, sodass die $i$-te Zeilensumme von $A$
gleich $0$ und die $i$-te Spaltensumme gleich $|V| - 1$ ist. \\
Anschaulich gesprochen, wird bei jedem Matrixeintrag außerhalb der Diagonale den wir überprüfen ein
Senkenkandidat ausgeschlossen.
Das liegt daran, dass für $A[i, j] = 1$ für $i$ die Bedingung, dass die Zeilensumme
$0$ sein muss verletzt wird, und für $A[i,j] = 0$ für $j$ die Bedingung, dass
die Spaltensumme gleich $|V| - 1$ sein muss verletzt wird.
Damit bleibt uns nach $n - 1$ Überprüfungen nur noch 1 möglicher Senkenkandidat übrig.
Von diesem müssen wir noch überprüfen, ob er tatsächlich die Senkenbedingung erfüllt,
was in maximal $2n$ Überprüfungen gelingt. \\
Weiter unten ist ein Algorithmus ausgeführt, der diese eben beschriebenen Schritte präzisiert.
\begin{flalign*}
  1&: \textbf{Prozedur}~ \textsc{Finde universelle Senke} (A) & \\
  2&:  \quad n := A.\textit{Zeilenanzahl} & \\
  3&:  \quad Senke := NIL & \\
  4&:  \quad i := 1 & \\
  4&:  \quad j := 2 & \\
  6&:  \quad \textbf{Solange } i \leq n \land j \leq n: & \\
  7&:  \quad \quad b := A[i,j] & \\
  8&:  \quad \quad \textbf{Wenn } b = 0: & \\
  9&:  \quad \quad \quad j := \max\{i + 1, j + 1\} & \\
  8&:  \quad \quad \textbf{Ende Wenn} & \\
  10&:  \quad \quad \textbf{Wenn } b = 1: & \\
  11&:  \quad \quad \quad  i := \max\{i + 1, j + 1\} & \\
  8&:  \quad \quad \textbf{Ende Wenn} & \\
  8&:  \quad  \textbf{Ende Solange} & \\
  12&: \quad s := 0 & \\
  13&: \quad z := 0 & \\
  12&: \quad \textbf{Wenn } i = n: & \\
  13&: \quad \quad \textbf{Für } k = 1,\dots,n: & \\
  14&: \quad \quad \quad z := z + A[j,k] & \\
  14&: \quad \quad \quad s := s + A[k,j] & \\
  8&:  \quad \quad \textbf{Ende Für} & \\
  12&: \quad \quad \textbf{Wenn } z = 0 \land s = n - 1: & \\
  13&: \quad \quad \quad \text{Senke} := j & \\
  8&:  \quad \quad \textbf{Ende Wenn} & \\
  12&: \quad \textbf{Sonst Wenn } j = n: & \\
  13&: \quad \quad \textbf{Für } k = 1,\dots,n: & \\
  14&: \quad \quad \quad z := z + A[i,k] & \\
  14&: \quad \quad \quad s := s + A[k,i] & \\
  8&:  \quad \quad \textbf{Ende Für} & \\
  12&: \quad \quad \textbf{Wenn } z = 0 \land s = n - 1: & \\
  13&: \quad \quad \quad \text{Senke} := i & \\
  8&:  \quad \quad \textbf{Ende Wenn} & \\
  8&:  \quad \textbf{Ende Wenn} & \\
  12&: \textbf{Ende Prozedur} &
\end{flalign*}
Der Solange-Block wird maximal $|V|$-Mal ausgeführt, ebenso die Für-Schleife weiter unten,
insgesamt entscheidet der Algorithmus also in $\Landau{|V|}$ Schritten, ob es eine
universelle Senke gibt.
\end{solution}

% --------------------------------------------------------------------------------
