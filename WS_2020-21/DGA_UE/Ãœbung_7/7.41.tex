% -------------------------------------------------------------------------------- %

\begin{exercise}

Beweisen Sie, dass ein Unabhängigkeitssystem $(E, S)$ genau dann ein Matriod ist, wenn für alle $A \subseteq E$ gilt, dass alle maximalen unabhängigen Teilmengen von $A$ gleichmächtig sind.

\end{exercise}

% -------------------------------------------------------------------------------- %

\begin{solution}

\phantom{}

\includegraphicsboxed{DGA/DGA - Definition 7.2.png}

\textbf{Definition (Unabhängigkeitssystem).}
Ein \textit{Unabhängigkeitssystem} $(E, S)$ ist ein Matriod, ohne geforderter Austauscheigenschaft.
Die Elemente von $S$ heißen \textit{unabhängig} und die von $\mathcal{P}(E) \setminus S$ heißen \textit{abhängig}.

\begin{enumerate}[label = \arabic*.]

    \item Richtung (\enquote{$\Rightarrow$}):

    \includegraphicsboxed{DGA/DGA - Definition 7.3.png}
    \includegraphicsboxed{DGA/DGA - Satz 7.5.png}

    Sei $(E, S)$ ein Matriod und $A \subseteq E$.

    \begin{align*}
        S_A := \Bbraces{B \in S: B \subseteq A}
    \end{align*}

    Offensichtlich ist $(A, S_A)$ ein Matriod.
    Die Behauptung folgt daher aus Satz 7.5. Dieser hat einen sehr kurzen Beweis, der wie folgt aussieht.
    
   	Nehmen wir an es wären $B,C \in S$ zwei maximal linear unabhängige Teilmengen von $A$ mit $|B| \neq |C|$. Ohne Beschränkung der Allgemeinheit nehmen wir $|B| > |C|$ an. Dann gibt es ein $x \in (B \setminus C) \subseteq A$  mit $C \cup \{x\} \in S$, ein Widerspruch zur Maximalität von $C$. 

    \item Richtung (\enquote{$\Leftarrow$}):
    
    Seien $A, B \in S$ und $|A| > |B|$.
    Sei weiters $C := A \cup B$.
    Es gibt eine $S$-unabhängige $C$-maximale Obermenge von $A$.

    \begin{align*}
        \implies
        \Exists A^\prime \in S:
        A \subseteq A^\prime \subseteq C ~\text{maximal}
    \end{align*}

    $B \subseteq C = A \cup B$ ist nicht maximal, weil sonst wäre wegen der gegebenen Zusatzeigenschaft

    \begin{align*}
        |B| = |A^\prime| \geq |A| > |B|.
    \end{align*}

    Es gibt daher eine $S$-unabhängige $C$-maximale echte Obermenge von $B$.

    \begin{align*}
        \implies
        \Exists B^\prime \in S:
        B \subsetneq B^\prime \subseteq C ~\text{maximal}
    \end{align*}

    Wenn $B$ zu $B^\prime \subseteq C = A \cup B$ erweitert wird, müssen Elemente aus $A$ hinzugefügt werden, d.h.

    \begin{align*}
        & \implies
        \emptyset \neq B^\prime \setminus B \subseteq A. \\
        & \implies
        \emptyset \neq B^\prime \setminus B \subseteq A \setminus B
    \end{align*}

    Sei $x \in B^\prime \setminus B \subseteq B^\prime$, dann ist $B \cup \Bbraces{x} \subseteq B^\prime \in S$.
    Weil $S$ als Unabhängigkeitssystem nach unten $\subseteq$-abgeschlossen ist, folgt $B \cup \Bbraces{x} \in S$.

\end{enumerate}

\end{solution}

% -------------------------------------------------------------------------------- %
