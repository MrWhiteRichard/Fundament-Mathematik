% --------------------------------------------------------------------------------

\begin{exercise}

Sei $G = (V, E)$ ein ungerichteter Graph und $k$ eine beliebige nichtnegative ganze Zahl. Ferner sei $M_k(G) = (E, S)$. Die Menge $S$ sei definiert als die Menge aller Teilmengen von $E$, die sich inder Form $M \cup F$ darstellen lassen, wobei $M$ eine höchstens $k$-elementige Menge bezeichnet und $F$ eine Menge, für die $(V, F)$ ein Wald ist. Beweisen Sie, dass $M_k(G)$ ein Matroid ist.

\end{exercise}

% --------------------------------------------------------------------------------

\begin{solution}

  Klarerweise ist $\emptyset \in S.$ Falls $B \in S$ und $A \subseteq B$ gilt, lässt sich $B$ schreiben als $M_B \cup F_B$, wobei $|M_B| \leq k$ und $(V, F_B)$ ein Wald ist. Wenn man $M_A := M_B \cap A$ und $F_A := F_B \cap A$ setzt, erhält man $A \in S$.

  Interessanter ist die Austauscheigenschaft. Seien $A, B \in S$ mit $|A| > |B|$ und seien
  \begin{align*}
      A = M_A \cup F_A, B = M_B \cup F_B
  \end{align*}
  einschlägige Zerlegungen.
  Wir machen eine Fallunterscheidung:
  \begin{itemize}
      \item Fall 1: $|M_B| < k:$ In diesem Fall wählen wir $x \in A\backslash B$ beliebig. Dann gilt
      \begin{align*}
      B \cup \{x\} = (M_B \cup \{x\}) \cup F_B \in S.
      \end{align*}
      \item Fall 2: $|M_B| = k:$ Nach Korollar 2.1 besteht der Wald $(V, F_A)$ aus $|V| - |F_A|$ und der Wald $(V, F_B)$ aus $|V| - |F_B|$ Bäumen.

      Aus $|A| > |B|$ und $|M_B| = k$ folgt $|F_A| > |F_B|$ und daher $|V| - |F_A| < |V| - |F_B|.$

      Nach dem Schubfachprinzip existiert ein Baum $T$ im Wald $(V, F_A),$ dessen Knoten zu mindestens zwei verschiedenen Bäumen in $(V, F_B)$ gehören.

      Wie wird das Schubfachprinzip hier genau angewendet? Die \Quote{Gegenstände}, welche wir auf die Schubfächer aufteilen wollen sind die Bäume im Wald $(V,F_B)$. Ein solcher Baum $R$ ist nicht leer, er enthält also mindestens einen Knoten $v$. Da der Wald $(V,F_A)$ alle Knoten aus $V$ enthält gibt es schon einen Baum $R^\prime$ aus diesem Wald, der $v$ als Knoten enthält. Beschriften wir nun jedes Schubfach mit einem Baum aus dem Wald $(V,F_A)$ so können wir jeden Baum $R$ aus dem Wald $(V,F_B)$ in ein Schubfach legen, das mit einem Baum $R^\prime$ aus dem Wald $(V,F_A)$ beschriftet ist, wobei sich $R$ und $R^\prime$ mindestens einen Knoten Teilen. Da der Wald $(V,F_A)$ aber weniger Bäume hat als der Wald $(V,F_B)$ müssen in mindestens einem Schubfach dann schon mindestens zwei Bäume aus dem Wald $(V,F_B)$ liegen.

      $T$ ist zusammenhängend, daher gibt es eine Kante $\{v, w\},$ sodass $v$ und $w$ in verschiedenen Bäumen von $(V, F_B)$ liegen. Daher ist $(V, F_B \cup \{\{v, w\}\})$ azyklisch und somit
      \begin{align*}
          B \cup \{\{v, w\}\} \in S.
      \end{align*}
  \end{itemize}
\end{solution}

% --------------------------------------------------------------------------------
