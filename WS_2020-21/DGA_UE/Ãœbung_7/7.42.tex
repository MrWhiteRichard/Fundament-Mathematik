% --------------------------------------------------------------------------------

\begin{exercise}

Sei $G = (V, E)$ ein ungerichteter Graph und $k$ eine beliebige nichtnegative ganze Zahl.
Ferner sei $M_k(G) = (E, S)$.
Die Menge $S$ sei definiert als die Menge aller Teilmengen von $E$, die sich in der Form $M \cup F$ darstellen lassen, wobei $M$ eine höchstens $k$-elementige Menge bezeichnet und $F$ eine Menge, für die $(V, F)$ ein Wald ist. Beweisen Sie, dass $M_k(G)$ ein Matroid ist.

\end{exercise}

% --------------------------------------------------------------------------------

\begin{solution}

\begin{align*}
    M_k(G) = (E, S),
    \quad
    S = \Bbraces
    {
        M \cup F \subseteq E:
        |M| \leq k,
        (V, F) ~\text{Wald}
    }
\end{align*}

\begin{enumerate}[label = \arabic*.]

    \item Eigenschaft ($\emptyset \in S$):
    
    Dafür müssen $M, F = \emptyset$.
    Für $M$ ist das sowieso möglich.
    Für $F$ geht das auch, weil jeder Graph der Form $(V, \emptyset)$ ja keine Kanten, also keine Zyklen haben kann.
    Insofern ist $(V, \emptyset)$ zyklenfrei, also ein Wald.

    \item Eigenschaft ($\subseteq$-Abgeschlossenheit):

    Sei $A \subseteq B \in S$, dann lässt sich $B = M_B \cup F_B$ schreiben, wobei $|M_B| \leq k$ und $(V, F_B)$ ein Wald (d.h. zyklenfrei) ist.
    Wenn man $M_A := M_B \cap A$ und $F_A := F_B \cap A$ setzt, erhält man $A \in S$.

    \begin{align*}
        \implies
        & |M_A| = |M_B \cap A| \leq |M_B| \leq k, \\
        & F_A = F_B \cap A \subseteq F_B
        \implies
        (V, F_B) ~\text{zyklenfrei}~
        \supseteq
        (V, F_A) ~\text{zyklenfrei}, \\
        & M_A \cup F_A
        =
        (M_B \cap A) \cup (F_B \cap A)
        =
        \underbrace
        {
            (M_B \cup F_B)
        }_B
        \cap A
        =
        A \\
        \implies
        & A \in S
    \end{align*}

    \item Eigenschaft (Austauscheigenschaft):

    Seien $A, B \in S$ mit $|A| > |B|$, dann lassen sich $A$ und $B$ darstellen als

    \begin{gather*}
        A = M_A \dot \cup F_A,
        \quad
        B = M_B \dot \cup F_B \\
        |M_A|, |M_B| \leq k,
        \quad
        (V, F_A), (V, F_B) ~\text{zyklenfrei}.
    \end{gather*}

    Weil $M_A$ und $M_B$ auch kleiner sein dürfen, haben wir $M \cap F = \emptyset$ gewählt.

    \begin{enumerate}[label = \arabic*.]
        
        \item Fall ($|M_B| < k$):
        
        In diesem Fall wählen wir $x \in A \setminus B \neq \emptyset$ beliebig.

        \begin{align*}
            \implies &
            |M_B \cup \Bbraces{x}| \leq |M_B| + 1 \leq k \\
            \implies &
            B \cup \Bbraces{x}
            =
            M_B \cup F_B \cup \Bbraces{x}
            =
            (M_B \cup \Bbraces{x}) \cup F_B
            \in
            S
        \end{align*}
        
        \item Fall ($|M_B| = k$):
        
        Nach Korollar 2.1 besteht der Wald $(V, F_A)$ aus $|V| - |F_A|$ und der Wald $(V, F_B)$ aus $|V| - |F_B|$ Bäumen.

        \begin{align*}
            \implies &
            |F_A| + |M_A|  = |A| > |B| = |M_B| + |F_B| = k + |F_B| \geq |M_A| + |F_B| \\
            \implies &
            |F_A| > |F_B| \\
            \implies &
            |V| - |F_A| < |V| - |F_B|
        \end{align*}

        Nach dem Schubfachprinzip existiert $1$ Baum $T$ im Wald $(V, F_A),$ dessen Knoten zu mindestens $2$ verschiedenen Bäumen $T_1$ und $T_2$ im Wald $(V, F_B)$ gehören.

        Wie wird das Schubfachprinzip hier genau angewendet?
        Die \blockquote{Gegenstände}, welche wir auf die Schubfächer aufteilen wollen sind die Bäume im Wald $(V, F_B)$.
        Ein solcher Baum $R$ ist nicht leer, er enthält also mindestens einen Knoten $v$.
        Da der Wald $(V, F_A)$ alle Knoten aus $V$ enthält gibt es schon einen Baum $R^\prime$ aus diesem Wald, der $v$ als Knoten enthält.
        Beschriften wir nun jedes Schubfach mit einem Baum aus dem Wald $(V, F_A)$ so können wir jeden Baum $R$ aus dem Wald $(V, F_B)$ in ein Schubfach legen, das mit einem Baum $R^\prime$ aus dem Wald $(V, F_A)$ beschriftet ist, wobei sich $R$ und $R^\prime$ mindestens einen Knoten Teilen.
        Da der Wald $(V, F_A)$ aber weniger Bäume hat als der Wald $(V, F_B)$ müssen in mindestens einem Schubfach dann schon mindestens zwei Bäume aus dem Wald $(V, F_B)$ liegen.

        $T$ ist als Baum zusammenhängend.
        Daher gibt es eine Kante $e = \Bbraces{v, w} \in_\mathrm{Kanten} T$, sodass $v \in T_1$ und $w \in T_2$ in verschiedenen Bäumen vom Wald $(V, F_B)$ liegen.

        \begin{align*}
            \implies
            & e \in_\mathrm{Kanten} T \subseteq (V, F_A) \subseteq_\mathrm{Kanten} = M_A \cup F_A = A, \\
            & e \not \in_\mathrm{Kanten} T_1, T_2 \subseteq (V, F_B) \subseteq_\mathrm{Kanten} = M_B \cup F_B = B \\
            \implies
            & e \in A \setminus B
        \end{align*}

        $T^\prime = T_1 \cup T_2 \cup_\mathrm{Kanten} \Bbraces{e}$ ist weiterhin zusammenhängend, weil $T_1$ und $T_2$ es sind und vermöge $e$ verbunden.
        $T^\prime$ ist auch zyklenfrei, weil $T_1$ und $T_2$ es sind und es keinen Zyklus gibt, der durch $e$ geht.
        $T^\prime$ ist also ein Baum aus $(V, F_B \cup \Bbraces{e})$.

        Die restlichen Bäume daraus sind jene $\neq T_1, T_2$ vom Wald $(V, F_B)$.
        Daher ist $(V, F_B \cup \Bbraces{e})$ zyklenfrei, d.h. ein Wald.
        Die Kante $e \in A \setminus B$ kann als Austausch-Element verwendet werden.

        \begin{align*}
            \implies
            B \cup \Bbraces{e}
            =
            M_B \cup (F_B \cup \Bbraces{e})
            \in S
        \end{align*}

    \end{enumerate}

\end{enumerate}

\end{solution}

% --------------------------------------------------------------------------------
