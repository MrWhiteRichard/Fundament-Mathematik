% --------------------------------------------------------------------------------

\newpage

\begin{exercise}

Alternative Algorithmen zur Bestimmung minimaler Spannbäume.
Nachfolgend sind die Pseudocodes von drei verschiedenen Algorithmen angegeben.
Alle drei erhalten als Eingabe einen zusammenhängenden kantenbewerteten Graphen und geben eine Kantenmenge $T$ zurück.
Beweisen oder widerlegen Sie für jeden der drei Algorithmen die Behauptung, dass $T$ in jedem Fall ein minimaler Spannbaum ist.

\begin{algorithmic}
    \State Maybe-MST-A($G, w$)
    \State sortiere die Kanten in nichtsteigender Reihenfolge nach ihren Gewichten $w$
    \State $T := E$
    \For{$e \in E$ (in der soeben berechneten Reihenfolge)}
        \If{$T \setminus \Bbraces{e}$ ist ein zusammenhängender Graph}
            \State $T := T \setminus \Bbraces{e}$
        \EndIf
    \EndFor
    \State \Return $T$
\end{algorithmic}

\begin{algorithmic}
    \State Maybe-MST-B($G, w$)
    \State $T := \emptyset$
    \For{$e \in E$ (in einer beliebigen Reihenfolge)}
        \If{$T \cup \Bbraces{e}$ kreisfrei}
            \State $T := T \cup \Bbraces{e}$
        \EndIf
    \EndFor
    \State \Return $T$
\end{algorithmic}

\begin{algorithmic}
    \State Maybe-MST-C($G, w$)
    \State $T := \emptyset$
    \For{$e \in E$ (in einer beliebigen Reihenfolge)}
        \State $T := T \cup \Bbraces{e}$
        \If{$T$ enthält einen Zyklus $c$}
            \State sei $e_0$ eine Kante von $c$ mit maximalem Gewicht
            \State $T := T \setminus \Bbraces{e_0}$
        \EndIf
    \EndFor
    \State \Return $T$
\end{algorithmic}

\end{exercise}

% --------------------------------------------------------------------------------

\begin{solution}

\phantom{}

\begin{enumerate}[label = (\Alph*)]

    \item Der Algorithmus von Kruskal startet leer und bezieht nur die nötigsten Kanten (d.h. jene mit minimalen Kosten) in den potentiellen MST mit ein.
    Unser Algorithmus startet jedoch voll und entfernt die \Quote{unnötigsten} Kanten (d.h. jene mit maximalen Kosten) von dem potentiellen MST.
    
    Die Vermutung, dass Letzterer korrekt ist, klingt daher durchaus plausibel.
    Um dies zu beweisen, orientieren wir uns entsprechend am Beweis von Satz 7.2.

    Sei $e_1, \dots, e_n$ eine Sortierung von $E$, so dass $c(e_1) \geq \cdots \geq c(e_n)$, sei $E_0 = \emptyset$ und

    \begin{align*}
        E_{i+1}
        =
        \begin{cases}
            E_i \setminus \Bbraces{e_{i+1}} & \text{falls}~ (V, E_i \setminus \Bbraces{e_{i+1}}) ~\text{zusammenhängenden ist und} \\
            E_i                             & \text{sonst.}
        \end{cases}
    \end{align*}

    Der Algorithmus antwortet mit $(V, E_n)$.

    \textit{Beweis.}
    Zunächst ist klar, dass $B = (V, E_n)$ zusammenhängend ist.
    $B$ ist aber auch zyklenfrei:
    Angenommen, $B$ wäre nicht zyklenfrei, sei $e_{i_1}, \dots, e_{i_k} \in E_n$ ein Zyklus in $B$.
    $e_{i_1}, \dots, e_{i_k}$ wurden vom Algorithmus nicht entfernt, also waren $(V, E_{i_1 - 1} \setminus \Bbraces{e_{i_1}}), \dots, (V, E_{i_k - 1} \setminus \Bbraces{e_{i_k}})$ nicht zusammenhängend.
    Sei $\ell = 1, \dots, k$.
    Weil nun $E_{i_\ell - 1} \setminus \Bbraces{e_{i_\ell}} \supset E_n$, wäre somit aber $(V, E_n \setminus \Bbraces{e_{i_\ell}})$ ebenso nicht zusammenhängend.
    Aus dem Zyklus $e_{i_1}, \dots, e_{i_k} \in E_n$ darf man aber eine beliebige Kante löschen und der resultierende Graph $(V, E_n \setminus \Bbraces{e_{i_\ell}})$ wäre noch immer zusammenhängend.
    Widerspruch!

    Für die Minimalität von $B$ reicht es zu zeigen, dass für alle $i \in \Bbraces{0, \dots, n}$ ein minimaler Spannbaum von $G$ mit Kantenmenge $M_i$ existiert so dass $E_i \supseteq M_i$.
    Dann ist nämlich $E_n = M_n$ und damit $B$ minimal.

    Wir gehen mit Induktion nach $i$ vor.
    Der Fall $i = 0$ ist trivial.
    Für den Induktionsschritt definieren wir $M_{i+1} = M_i$ falls keine Kante weggenommen wird oder die weggenommene Kante $e_{i+1} \not \in M_i$.
    Sei nun also $E_{i+1} = E_i \setminus \Bbraces{e_{i+1}}$, $(V, E_{i+1})$ zusammenhängend und $e_{i+1} \in M_i$.
    Weil $(V, M_i)$ als Baum minimal zusammenhängend ist, ist $(V, M_i \setminus \Bbraces{e_{i+1}})$ nicht mehr zusammenhängend.
    Weil $E_{i+1}$ zusammenhängend ist, können die beiden Zusammenhangskomponenten von $(V, E_i \setminus \Bbraces{e_{i+1}}) \supseteq (V, M_i \setminus \Bbraces{e_{i+1}}) = (V, E_{i+1})$ durch ein $e_j \in E_{i+1}$ verbunden werden.
    Dabei ist $e_{i+1} \neq e_j \in E_{i+1} = E_i \setminus \Bbraces{e_{i+1}}$.
    Ebenso ist $e_j \not \in M_i \setminus \Bbraces{e_{i+1}}$, weil die Zusammenhangskomponenten sonst bereits verunden gewesen wären.
    Daher muss $e_j \not \in M_i$.

    \begin{align*}
        M_{i+1}
        :=
        (
            \underbrace
            {
                \underbrace{M_i}_{\subseteq E_{i+1}}
                \setminus
                \Bbraces{e_{i+1}}
            }_{
                \subseteq E_{i+1}
            }
            \dot \cup
            \underbrace
            {
                \Bbraces{e_j}
            }_{
                \subseteq E_{i+1}
            }
        )
        \subseteq
        E_{i+1}
    \end{align*}

    $(V, M_{i+1})$ ist also zusammenhängend.

    Weiters ist $|M_{i+1}| = |M_i| = |V| - 1$ da ja $e_j \not \in$ und $e_{i+1} \in M_i$ ist und $(V, M_{i+1})$ mit Satz 2.3 also ein Spannbaum von $G$.

    Außerdem ist $c(e_j) \leq c(e_{i+1})$, denn $c(e_j) > c(e_{i+1})$ impliziert $j < i + 1$ und damit $j-1 < j \leq i < i+1$.
    Weil die $E$-Kantenmengen monoton nicht-steigen, heißt das $E_{j-1} \supseteq E_j \subseteq E_i \subseteq E_{i+1}$.

    \begin{align*}
        \implies
        E_{j-1} \setminus \Bbraces{e_j}
        \stackrel{!}{\neq}
        E_j
        \supseteq
        E_{i+1}
        \ni
        e_j
    \end{align*}

    Laut Konsturktion, und weil $e_j \not \in M_i$, wäre dann aber

    \begin{multline*}
        (V, E_{j-1} \setminus \Bbraces{e_j}) ~\text{nicht zusammenhängend}~
        \supseteq
        (V, E_i     \setminus \Bbraces{e_j}) ~\text{nicht zusammenhängend} \\
        \supseteq
        (V, M_i     \setminus \Bbraces{e_j}) ~\text{nicht zusammenhängend}~
        =
        (V, M_i)                             ~\text{nicht zusammenhängend.}
    \end{multline*}

    Widerspruch!
    Also ist $c(M_{i+1}) \leq c(M_i)$ und, da $M_i$ minimal ist, $c(M_{i+1}) = c(M_i)$ und $M_{i+1}$ also ebenfalls ein minimaler Spannbaum.

    Q.E.D.

\end{enumerate}

\end{solution}

% --------------------------------------------------------------------------------
