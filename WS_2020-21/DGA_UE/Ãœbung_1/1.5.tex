% --------------------------------------------------------------------------------

\begin{exercise}

Das Horner-Schema dient zur Auswertung von Polynomen.
Die Grundidee dahinter ist die Umformung $p(x) = a_0 + a_1 x + a_2 x^2 + \cdots + a_n x^n = a_0 + x (a_1 + x (a_2 + \cdots + x (a_{n - 1} + x a_n) \ldots))$.

\begin{enumerate}[label = (\roman*)]
    \item Wiederholen Sie die Funktionsweise des Horner-Schemas und schreiben sie einen Pseudocode für die Auswertung von Polynomen mittels Horner-Schema.
    \item Schreiben Sie einen Pseudocode für die direkte Auswerung von Polynomen (Einsetzen).
    \item Vergleichen Sie die Schrittzahlen beider Codes.
\end{enumerate}

\end{exercise}

% --------------------------------------------------------------------------------

\begin{solution}

\phantom{}

\begin{enumerate}[label = (\roman*)]

  \item Wir nehmen an, dass Polynome als Arrays übergeben werden mit $p[0] = a_0, \dots, p[n] = a_n$.

  \begin{flalign*}
    1&: \textbf{Prozedur}~ \textsc{Horner-Schema} (p,x) & \\
    2&: \quad n := \grad{p} & \\
    3&: \quad y := p[n] \cdot x & \\
    4&: \quad \textbf{Für}~ i = n-1, \dots, 1 & \\
    5&: \quad \quad y := y + p[i] & \\
    6&: \quad \quad y := x \cdot y & \\
    7&: \quad \textbf{Ende Für} & \\
    8&: \quad y := y + p[0] & \\
    9&: \textbf{Ende Prozedur}
  \end{flalign*}

  \begin{flalign*}
    1&: \textbf{Prozedur}~ \textsc{Horner-Schema} (p,x) & \\
    2&: \quad n := \grad{p} & \\
    3&: \quad y := p[n] & \\
    4&: \quad \textbf{Für}~ i = n-1, \dots, 1 & \\
    5&: \quad \quad y := p[i] + x \cdot y & \\
    6&: \quad \textbf{Ende Für} & \\
    7&: \textbf{Ende Prozedur}
  \end{flalign*}
  
  \item
  
  \begin{flalign*}
    1&: \textbf{Prozedur}~ \textsc{Polynom auswerten} (p, x) & \\
    2&: \quad n := \grad{p} & \\
    3&: \quad y := 0 & \\
    4&: \quad z := 1 & \\
    5&: \quad \textbf{Für}~ i = 0, \dots, n & \\
    6&: \quad \quad y := y + p[i] \cdot z & \\
    7&: \quad \quad z := z \cdot x & \\
    8&: \quad \textbf{Ende Für} & \\
    9&: \textbf{Ende Prozedur}
  \end{flalign*}

  \item Wir zählen als Schritte nur reine Rechenschritte und nicht jede ausgeführte Zeile Code:
  Additionen / Multiplikationen Horner-Schema (in Abhängigkeit vom Grad des Polynoms):
  
  \begin{align*}
    A_{\text{Horner-Schema}}(n) = n,
    \quad
    M_{\text{Horner-Schema}}(n) = n
  \end{align*}

  Additionen / Multiplikationen Polynom-Auswerten (in Abhängigkeit vom Grad des Polynoms):

  \begin{align*}
    A_{\text{gewöhnlich}}(n) = n + 1,
    \quad
    M_{\text{gewöhnlich}}(n) = 2 (n + 1).
  \end{align*}

  Da Additionen wesentlich billiger sind als Multiplikationen fallen diese nicht so ins Gewicht.
  Das Horner-Schema spart also fast die Hälfte der Zeit im Vergleich zur gewöhnlichen Auswertung des Polynoms.

\end{enumerate}

\end{solution}

% --------------------------------------------------------------------------------
