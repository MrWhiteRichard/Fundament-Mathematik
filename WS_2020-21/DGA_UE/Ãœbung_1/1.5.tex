% --------------------------------------------------------------------------------

\begin{exercise}

Das Horner-Schema dient zur Auswertung von Polynomen.
Die Grundidee dahinter ist die Umformung $p(x) = a_0 + a_1 x + a_2 x^2 + \cdots + a_n x^n = a_0 + x (a_1 + x (a_2 + \cdots + x (a_n - 1 + x a_n) \ldots))$.

\begin{enumerate}[label = (\roman*)]
    \item Wiederholen Sie die Funktionsweise des Horner-Schemas und schreiben sie einen Pseudocode für die Auswertung von Polynomen mittels Horner-Schema.
    \item Schreiben Sie einen Pseudocode für die direkte Auswerung von Polynomen (Ein-setzen).
    \item Vergleichen Sie die Schrittzahlen beider Codes.    
\end{enumerate}

\end{exercise}

% --------------------------------------------------------------------------------

\begin{solution}

ToDo!

\end{solution}

% --------------------------------------------------------------------------------
