% --------------------------------------------------------------------------------

\begin{exercise}

Das Horner-Schema dient zur Auswertung von Polynomen.
Die Grundidee dahinter ist die Umformung $p(x) = a_0 + a_1 x + a_2 x^2 + \cdots + a_n x^n = a_0 + x (a_1 + x (a_2 + \cdots + x (a_{n - 1} + x a_n) \ldots))$.

\begin{enumerate}[label = (\roman*)]
    \item Wiederholen Sie die Funktionsweise des Horner-Schemas und schreiben sie einen Pseudocode für die Auswertung von Polynomen mittels Horner-Schema.
    \item Schreiben Sie einen Pseudocode für die direkte Auswerung von Polynomen (Ein-setzen).
    \item Vergleichen Sie die Schrittzahlen beider Codes.
\end{enumerate}

\end{exercise}

% --------------------------------------------------------------------------------

\begin{solution}
\phantom{}
\begin{enumerate}[label = (\roman*)]
    \item Wir nehmen an, dass Polynome als Arrays übergeben werden mit $p[0] = a_0, \dots, p[n] = a_n$.
    \begin{flalign*}
    &1: \textbf{Prozedur}~ \textsc{Horner-Schema} (p,x) &\\
    &2: \quad y := p[n]*x &\\
    &3: \quad \textbf{Für}~ i = n-1,\dots,1: &\\
    &4: \quad \quad  y := y + p[i] &\\
    &5: \quad \quad  y := x*y &\\
    &6: \quad \textbf{Ende Für} &\\
    &7: \quad y := y + p[0] &\\
    &8: \textbf{Ende Prozedur}
    \end{flalign*}
    \begin{flalign*}
    &1: \textbf{Prozedur}~ \textsc{Horner-Schema} (p,x) &\\
    &2: \quad y := p[n] &\\
    &3: \quad \textbf{Für}~ i = n-1,\dots,1: &\\
    &4: \quad \quad  y := p[i] + x * y &\\
    &5: \quad \textbf{Ende Für} &\\
    &6: \textbf{Ende Prozedur}
    \end{flalign*}
    \item
    \begin{flalign*}
    &1: \textbf{Prozedur}~ \textsc{Polynom-Auswerten} (p,x) &\\
    &2: \quad y := p[0] &\\
    &3: \quad z := 1 &\\
    &4: \quad \textbf{Für}~ i = 1,\dots,n: &\\
    &5: \quad \quad  z := z*x &\\
    &6: \quad \quad  y := y + p[i]*z &\\
    &7: \quad \textbf{Ende Für} &\\
    &8: \textbf{Ende Prozedur}
    \end{flalign*}
    \begin{flalign*}
    &1: \textbf{Prozedur}~ \textsc{Polynom-Auswerten} (p,x) &\\
    &2: \quad y := p[0] &\\
    &3: \quad \textbf{Für}~ i = 1,\dots,n: &\\
    &4: \quad \quad  y := y + p[i]*x &\\
    &5: \quad \quad  x := x * x &\\
    &6: \quad \textbf{Ende Für} &\\
    &7: \textbf{Ende Prozedur}
    \end{flalign*}
    \item Ich zähle als Schritte jetzt nur reine Rechenschritte und nicht jede ausgeführte Code-Zeile:
    Additionen / Multiplikationen Horner-Schema (in Abhängigkeit vom Grad des Polynoms):
    \begin{align*}
      A(n) &= n \\
      M(n) &= n
    \end{align*}
    Additionen / Multiplikationen Polynom-Auswerten (in Abhängigkeit vom Grad des Polynoms):
    \begin{align*}
      A^{\prime}(n) &= n \\
      M^{\prime}(n) &= 2n.
    \end{align*}
    Da Additionen wesentlich billiger sind als Multiplikationen,
    spart das Horner-Schema fast die Hälfte der Zeit im Vergleich zur gewöhnlichen Auswertung des Polynoms.
\end{enumerate}

\end{solution}

% --------------------------------------------------------------------------------
