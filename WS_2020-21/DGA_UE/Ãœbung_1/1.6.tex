% --------------------------------------------------------------------------------

\begin{exercise}

\phantom{}

\begin{enumerate}[label = (\alph*)]

    \item Zeigen Sie mittels vollständiger Induktion, dass ein Algorithmus, dessen Laufzeit $t(n)$ (wobei $n = 2^k$, $k \in \Z^+$) der Rekursion

    \begin{align*}
        T(n)
        =
        \begin{cases}
            1                    & \text{für}~ n = 2 \\
            2T (\frac{n}{2}) + 1 & \text{für}~ n = 2^k, k > 1
        \end{cases}
    \end{align*}

    genügt, $T(n) = n - 1$ erfüllt.

    \item Zeigen Sie mittels vollständiger Induktion, dass ein Algorithmus, dessen Laufzeit $T(n)$ (wobei $n = 2^k$, $k \in \Z^+$) der Rekursion

    \begin{align*}
        T(n)
        =
        \begin{cases}
            2                    & \text{für}~ n = 2 \\
            2T (\frac{n}{2}) + 1 & \text{für}~ n = 2^k, k > 1
        \end{cases}
    \end{align*}

    genügt, $T(n) = n \log_2{(n)}$ erfüllt.

\end{enumerate}


\end{exercise}

% --------------------------------------------------------------------------------

\begin{solution}

\begin{enumerate}[label = (\alph*)]

\item Für den Induktionsanfang setzen wir $k=1$, also $n=2^1$ und sehen sofort

\begin{align*}
  T(2^1)
  =
  2 - 1
  =
  1
\end{align*}

Um den Induktionsschritt durchzuführen sei die Induktionsbehauptung für alle $2^j$,
wobei $j < k$, wahr. Dann erhalten wir für $n=2^k$:

\begin{align*}
  T(2^k)
  =
  2 T(2^{k-1}) + 1
  \stackrel{\text{IV}}{=}
  2(2^{k-1} - 1) + 1
  =
  2^k - 1
  =
  n - 1
\end{align*}

\item Für den Induktionsanfang setzen wir $k=1$, also $n=2^1$ und sehen sofort

\begin{align*}
  T(2^1)
  =
  2 \cdot 1
  =
  2
\end{align*}

Um den Induktionsschritt durchzuführen sei die Induktionsbehauptung für alle $2^j$,
wobei $j < k$, wahr. Dann erhalten wir für $n=2^k$:

\begin{align*}
  T(2^k)
  =
  2 T(2^{k-1}) + 2^k
  \stackrel{\text{IV}}{=}
  2 (2^{k-1} \log_2{(2^{k-1})}) + 2^k
  =
  2^k(k-1) + 2^k
  =
  2^k k - 2^k + 2^k
  =
  2^k \log_2{(2^k)}
  =
  n \log_2{(n)}
\end{align*}

\end{enumerate}
\end{solution}

% --------------------------------------------------------------------------------
