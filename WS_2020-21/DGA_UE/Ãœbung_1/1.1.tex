% --------------------------------------------------------------------------------

\begin{exercise}

Überlegen Sie sich einen Pseudocode für die folgenden Algorithmen und bestimmen Sie die Anzahl der notwendigen Schritte, die (in Ihrem Pseudocode) nötig sind, um eine $n$-elementige Menge zu sortieren.
Wenden Sie die Algorithmen auf den Datensatz $6, 77, 45, 103, 4, 17$ an.

\begin{enumerate}[label = (\alph*)]

    \item Selection Sort:
    der Algorithmus sucht zunächst das kleinste Element und bringt es an die erste Position.
    Anschließend sucht er das zweitkleinste Element und bringtes an die zweite Position, usw.

    \item Bubble-Sort:
    Der Algorithmus vergleicht der Reihe nach je zwei benachbarte Zahlenund vertauscht diese, falls sie nicht in der richtigen Reihenfolge angeordnet sind.
    Dieses Verfahren wird so lange wiederholt, bis alle Zahlen der Eingabe sortiert sind.

\end{enumerate}

\end{exercise}

% --------------------------------------------------------------------------------

\begin{solution}
\phantom{}

\begin{enumerate}[label = (\alph*)]

\item
\begin{flalign*}
&1: \textbf{Prozedur}~ \textsc{Selection Sort} (A) &\\
&2:  \quad n := A.\textit{Länge} &\\
&3:  \quad \textbf{Für}~ i := 1, \dots, n &\\
&4:  \quad \quad x := A[i] &\\
&5:  \quad \quad \textbf{Für}~ j := i+1, \dots, n &\\
&6:  \quad \quad \quad \textbf{Wenn}~ x > A[j] &\\
&7:  \quad \quad \quad \quad x := A[j] &\\
&8:  \quad \quad \quad \textbf{Ende Wenn}~ &\\
&9:  \quad \quad \textbf{Ende Für}~ &\\
&10: \quad  \quad A[j],A[i] := A[i],x &\\
&11: \quad  \textbf{Ende Für}~ &\\
&12: \textbf{Ende Prozedur}
\end{flalign*}

Die notwendigen Schritte setzen sich wie folgt zusammen
(die $t_j$ sind dabei $1$ wenn die Codezeile $7$ ausgeführt werden muss und $0$ sonst):
\begin{align*}
  1 + \sum_{i = 1}^n \pbraces{1 +\pbraces{\sum_{j = i+1}^n (1 + t_j)} +1} \\
\end{align*}
Für den worst-case (also $\Forall j \in \N :t_j = 1$) erhalten wir

\begin{multline*}
  1 + \sum_{i = 1}^n \pbraces{2 +\sum_{j = i+1}^n 2}
  =
  1 + \sum_{i = 1}^n (2 + 2(n - i))
  =
  1 + \sum_{i = 1}^n 2 + 2\sum_{i = 1}^n n - 2\sum_{i = 1}^n i \\
  =
  1 + 2n + 2n^2 - 2\frac{n(n+1)}{2}
  =
  n^2 + n + 1
  =
  \Landau{n^2}
\end{multline*}

Für den best-case (also $\Forall j \in \N :t_j = 0$) erhalten wir

\begin{multline*}
  1 + \sum_{i = 1}^n \pbraces{2 +\sum_{j = i+1}^n 1}
  =
  1 + \sum_{i = 1}^n (2 + n - i)
  =
  1 + \sum_{i = 1}^n 2 + \sum_{i = 1}^n n - \sum_{i = 1}^n i \\
  =
  1 + 2n + n^2 - \frac{n(n+1)}{2}
  =
  \frac{n^2 + 3n}{2} + 1
  =
  \Landau{n^2}
\end{multline*}


\item
\begin{flalign*}
&1: \textbf{Prozedur}~ \textsc{Bubble Sort}(A) &\\
&2: \quad n := A.\textit{Länge} &\\
&3: \quad \textbf{Für}~ i := 1, \dots, n &\\
&4: \quad \quad \textbf{Für}~ j := 1, \dots, n-i &\\
&5: \quad \quad \quad \textbf{Wenn}~ A[j+1] < A[j] &\\
&6: \quad \quad \quad \quad A[j], A[j+1] := A[j+1], A[j] &\\
&7: \quad \quad \quad \textbf{Ende Wenn} &\\
&8: \quad \quad \textbf{Ende Für} &\\
&9: \quad \textbf{Ende Für} &\\
&10: \textbf{Ende Prozedur}
\end{flalign*}

Die notwendigen Schritte setzen sich wie folgt zusammen
(die $t_j$ sind dabei $1$ wenn die Codezeile $7$ ausgeführt werden muss und $0$ sonst):

\begin{align*}
  1 + \sum_{i=1}^n \sum_{j=1}^{n-i} (1 + t_j)
\end{align*}

Für den worst-case (also $\Forall j \in \N :t_j = 1$) erhalten wir

\begin{align*}
    1 + \sum_{i=1}^n \sum_{j=1}^{n-i} (1 + 1)
    =
    1 + \sum_{i=1}^n 2(n-i)
    =
    1 + 2\sum_{i=1}^n n - 2\sum_{i=1}^n i
    =
    1 + 2n^2 - 2\frac{n(n+1)}{2}
    =
    n^2-n + 1
    =
    \Landau{n^2}
\end{align*}

Für den best-case (also $\Forall j \in \N :t_j = 0$) erhalten wir

\begin{align*}
  1 + \sum_{i=1}^n \sum_{j=1}^{n-i} 1
  =
  1 + \sum_{i=1}^n n - \sum_{i=1}^n i
  =
  1 + n^2 - \frac{n(n+1)}{2}
  =
  \frac{n^2 - n}{2} + 1
  =
  \Landau{n^2}
\end{align*}
\end{enumerate}
\end{solution}

% --------------------------------------------------------------------------------
