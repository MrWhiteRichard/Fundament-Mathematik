% --------------------------------------------------------------------------------

\begin{exercise}

Sequentielle Suche:
Gegeben sei ein $n$-elementiger Datenfeld $A[1, \ldots, n]$ und ein Wert $x$.
Überlegen Sie sich einen Pseudocode, der $x$ durch sukzessive Vergleiche mit den Elementen $A[1], A[2], \ldots$ sucht und einen Wert $j \in \Bbraces{1, \ldots n}$ ausgibt, falls $x = A[j]$, oder $\NIL$ ausgibt, falls $x$ nicht in der Liste $A$ enthalten ist.

\begin{enumerate}[label = (\alph*)]

    \item Beweisen Sie, dass Ihr Algorithmus ein korrektes Ergebnis liefert (etwa mit Zuhilfenahme einer Schleifeninvariante).

    \item Machen Sie für diesen Algorithmus eine best-case-Analyse, eine worst-case-Analyse und eine average-case-Analyse (für die average-case-Analyse soll das Modell der Zufallspermutationen verwendet werden, d.h. alle Permutationen von $\Bbraces{1, \ldots, n}$ sind gleich wahrscheinlich.
    Weiters soll angenommen werden, dass $x$ im Datensatz enthalten ist).
    
\end{enumerate}

\end{exercise}

% --------------------------------------------------------------------------------

\begin{solution}

ToDo!

\end{solution}

% --------------------------------------------------------------------------------
