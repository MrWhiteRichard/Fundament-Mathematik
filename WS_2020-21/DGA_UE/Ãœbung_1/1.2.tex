% --------------------------------------------------------------------------------

\begin{exercise}

Sequentielle Suche:
Gegeben sei ein $n$-elementiger Datenfeld $A[1, \ldots, n]$ und ein Wert $x$.
Überlegen Sie sich einen Pseudocode, der $x$ durch sukzessive Vergleiche mit den Elementen $A[1], A[2], \ldots$ sucht und einen Wert $j \in \Bbraces{1, \ldots n}$ ausgibt, falls $x = A[j]$, oder $\NIL$ ausgibt, falls $x$ nicht in der Liste $A$ enthalten ist.

\begin{enumerate}[label = (\alph*)]

    \item Beweisen Sie, dass Ihr Algorithmus ein korrektes Ergebnis liefert (etwa mit Zuhilfenahme einer Schleifeninvariante).

    \item Machen Sie für diesen Algorithmus eine best-case-Analyse, eine worst-case-Analyse und eine average-case-Analyse (für die average-case-Analyse soll das Modell der Zufallspermutationen verwendet werden, d.h. alle Permutationen von $\Bbraces{1, \ldots, n}$ sind gleich wahrscheinlich.
    Weiters soll angenommen werden, dass $x$ im Datensatz enthalten ist).

\end{enumerate}

\end{exercise}

% --------------------------------------------------------------------------------

\begin{solution}

\begin{flalign*}
&1: \textbf{Prozedur}~ \textsc{Sequentielle Suche} (A,x) &\\
&2: \quad n := A.\textit{Länge} &\\
&3: \quad j := \NIL &\\
&4: \quad \textbf{Für}~ i := 1, \dots, n &\\
&5: \quad \quad \textbf{Wenn}~ x = A[i] &\\
&6: \quad \quad \quad j := i &\\
&7: \quad \quad \quad \textbf{Abbruch} &\\
&8: \quad \quad \textbf{Ende Wenn} &\\
&9: \quad \textbf{Ende Für} &\\
&10: \textbf{Ende Prozedur}
\end{flalign*}

\begin{enumerate}[label = (\alph*)]

  \item Unsere Schleifeninvariante ist:
  \begin{align*}
    I(\texttt{A},\texttt{x},\texttt{j},\texttt{i}):
    (\texttt{j} = \NIL \land \Forall \texttt{k} < \texttt{i}: \texttt{x} \neq \texttt{A}[\texttt{k}])
    \lor
    \texttt{x} = \texttt{A}[\texttt{j}]
  \end{align*}

  \begin{enumerate}[label = \arabic*.]
    \item $I$ ist zu Beginn des ersten Durchlaufs der Schleife wahr.
    Dort ist $\texttt{j} = \NIL$.
    \item
    Es gelte $I(\texttt{A},\texttt{x},\texttt{j},\texttt{i})$. Entweder $\texttt{x}=\texttt{A}[\texttt{j}]$, dann wurde die Prozedur bereits beendet, oder wir befinden uns in der Codezeile $4$.

    Es gilt $\texttt{A}^\prime = \texttt{A}$ sowie $\texttt{x}^\prime = \texttt{x}$ weil sich $\texttt{A}$ und $\texttt{x}$ nie ändern. Weil die For-Schleife nach oben zählt, ist $\texttt{i}^\prime = \texttt{i} + 1$.

    \begin{enumerate}[label = \Roman*.]
      \item Fall ($\texttt{x}^\prime = \texttt{A}^\prime[\texttt{i}^\prime]$): Dann ist $\texttt{j}^\prime = \texttt{i}^\prime$ und daher $\texttt{x}^\prime = \texttt{A}^\prime[\texttt{j}^\prime]$ und die Prozedur wird beendet.

      \item Fall ($\texttt{x}^\prime \neq \texttt{A}^\prime[\texttt{i}^\prime]$): Dann bleibt $\texttt{j}^\prime = \texttt{j}$ und $\Forall \texttt{k} < \texttt{i}^\prime: \texttt{x} \neq \texttt{A}[\texttt{k}]$ wegen der Schleifeninvariante und der Fallunterscheidungsbedingung.
    \end{enumerate}

    Also gilt auch $I(\texttt{A}^\prime, \texttt{x}^\prime, \texttt{j}^\prime, \texttt{i}^\prime)$

    \item Offensichtlich ist $\texttt{j}$ für $\texttt{i} = \texttt{n}$ das gewünschte Ergebnis.
  \end{enumerate}
  \item Best-Case: $A[1] = x:$
  \begin{align*}
    T(A) = c_2 + c_3 + c_4 + c_5 + c_6 + c_7 + c_8 + c_9 = \Landau{1}
  \end{align*}
  Worst-Case: $A[n] = x:$
  \begin{align*}
    T(A) = (n+1)(c_2 + c_9) + c_3 + c_4 + n(c_5 + c_8) + c_6 + c_7 = \Landau{n}
  \end{align*}
  Average-Case: Sei $\pi \in S_n$ beliebig. Für jedes $x \in \{1,\dots,n\}$
  und $i \in \{1,\dots,n\}$
  gibt es genau $n!/n$ Permuationen, die $x = \pi(i)$ erfüllen.
  Für $x = \pi(i)$ muss die Wenn-Bedingung genau $i$-Mal überprüft werden und wir erhalten
  für die durchschnittliche Anzahl der Wenn-Bedingungsüberprüfungen
  \begin{align*}
    \overline{t} = \sum_{i=1}^n \frac{i}{n} = \frac{n+1}{2}.
  \end{align*}
  Damit gilt
  \begin{align*}
    T(A) = \frac{n+1}{2}(c_2 + c_9) + c_3 + c_4 + \frac{n+1}{2}(c_5 + c_8) + c_6 + c_7 = \Landau{n}
  \end{align*}
\end{enumerate}

\end{solution}

% --------------------------------------------------------------------------------
