% --------------------------------------------------------------------------------

\begin{exercise}

\begin{enumerate}[label = (\alph*)]
  \item Wie schnell könnte man eine $(kn \times n)$-Matrix $A$ mit einer $(n \times kn)$-Matrix
  $B$ multiplizieren, d.h. $C = A \cdot B$ berechenen, wenn man Strassens Algorithmus als Unterprogramm verwendet?
  \item Benantworten Sie die gleiche Frage, wenn die Reihenfolge der Eingabematrizen vertauscht ist, man also
  $\tilde{C} = B \cdot A$ bestimmen möchte.
\end{enumerate}

\end{exercise}

% --------------------------------------------------------------------------------

\begin{solution}

\begin{enumerate}[label = (\alph*)]
  \item Man kann die Matrizen als Blockmatrizen auffassen und muss somit zur Berechnung der $(kn \times kn)$-Matrix $k^2$ Blöcke der Größe $n \times n$ durch je eine Multiplikation von zwei $(n \times n)$-Matrizen berechnen. Es ergibt sich also eine Laufzeit von $T(n,k) = \Theta(k^2n^{\log 7})$. (Wenn wir für den Strassen-Algorithmus wieder voraussetzen, dass $n$ eine Zweierpotenz ist).

  \item Analog zu oben betrachten wir wieder die Multiplikation von $(n \times n)$-Blockmatrizen. Hier benötigen wir $k$ Multiplikationen und $k-1$ Additionen. Es ergibt sich also eine Laufzeit von $T(n,k) = \Theta(kn^{\log 7})$.
\end{enumerate}

\end{solution}

% --------------------------------------------------------------------------------
