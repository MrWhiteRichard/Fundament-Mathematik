% --------------------------------------------------------------------------------

\begin{exercise}

Die Fibonacci-Zahlen seien durch die Rekursion $F_n = F_{n-1} + F_{n-2}$ für $n \geq 2$ mit Anfangswerten $F_0 = 0$ und $F_1 = 1$ definiert.
Die Fibonacci-Zahlen können effizient mittels folgender auf Matrizenmultiplikation beruhender Formel berechnet werden:

\begin{align*}
  \begin{pmatrix}
    F_{n+1} & F_n \\
    F_n     & F_{n-1}
  \end{pmatrix}
  =
  \begin{pmatrix}
    1 & 1 \\
    1 & 0
  \end{pmatrix}^n
  \quad
  ~\text{für}~
  n \geq 1
\end{align*}

\begin{enumerate}[label = \alph*.]

  \item Beweisen Sie diese Formel durch vollständige Induktion.

  \item Überlegen Sie sich einen Algorithmus, der
  $\begin{pmatrix}
    1 & 1 \\
    1 & 0
  \end{pmatrix}^n$
  in nur logarithmisch vielen Schritten berechnet.

\end{enumerate}

\end{exercise}

% --------------------------------------------------------------------------------

\begin{solution}

Wir nennen die linke Matrix $L_n$ und die rechte $R_n$.

\begin{enumerate}[label = \alph*.]

  \item IA($n = 1$):

  \begin{align*}
    \implies
    F_2 = F_1 + F_0 = 1 + 0
    \implies
    L_1 =
    \begin{pmatrix}
      F_2 & F_1 \\
      F_1 & F_0
    \end{pmatrix}
    =
    \begin{pmatrix}
      1 & 1 \\
      1 & 0
    \end{pmatrix}^1
    = R_1
  \end{align*}

  IS($n \mapsto n + 1$):

  \begin{align*}
    \implies
    R_{n+1}
    &=
    R_n
    \begin{pmatrix}
      1 & 1 \\
      1 & 0
    \end{pmatrix}
    =
    L_n
    \begin{pmatrix}
      1 & 1 \\
      1 & 0
    \end{pmatrix}
    =
    \begin{pmatrix}
      F_{n+1} & F_n \\
      F_n     & F_{n-1}
    \end{pmatrix}
    \begin{pmatrix}
      1 & 1 \\
      1 & 0
    \end{pmatrix} \\
    &=
    \begin{pmatrix}
      F_{n+1} + F_n & F_{n+1} \\
      F_n + F_{n-1} & F_n
    \end{pmatrix}
    =
    \begin{pmatrix}
      F_{n+2} & F_{n+1} \\
      F_{n+1} & F_n
    \end{pmatrix}
    =
    L_{n+1}
  \end{align*}

  \item Sei $k \in \N$.

  \begin{align*}
    & \implies
    P_k
    :=
    \begin{pmatrix}
      1 & 1 \\
      1 & 0
    \end{pmatrix}^{2^k}
    =
    \underbrace
    {
      \pbraces
      {
        \begin{pmatrix}
          1 & 1 \\
          1 & 0
        \end{pmatrix}^2
        \cdots
      }^2
    }_{
      k \text{-mal}
    }
    =
    P_{k-1}^2
    \implies
    P_0
    =
    \begin{pmatrix}
      1 & 1 \\
      1 & 0
    \end{pmatrix}
  \end{align*}

  Stelle $n \in \N_0$ (eindeutig) binär dar, d.h.

  \begin{align*}
    n = \sum_{k=0}^m a_k 2^k,
    \quad
    m = \floorbraces{\log(n)},
    \quad
    a_0, \dots, a_m \in \Bbraces{0, 1}.
  \end{align*}

  \begin{align*}
    \implies
    F_n
    :=
    \begin{pmatrix}
      1 & 1 \\
      1 & 0
    \end{pmatrix}^n
    =
    \prod_{k=0}^m
    \begin{pmatrix}
      1 & 1 \\
      1 & 0
    \end{pmatrix}^{a_k 2^k}
    =
    \prod_{
      \substack
      {
        k = 0 \\
        a_k = 1
      }
    }^m
    P_k
  \end{align*}

  \begin{flalign*}
   1&: \textbf{Prozedur}~ \textsc{Fibonacci Matrix}(n) & \\
   2&: \quad a := \textsc{Binärkoeffizienten}(n) & \\
   3&: \quad m := a.\textit{Länge} & \\
   4&: \quad F :=
   \begin{pmatrix}
    1 & 0 \\
    0 & 1
  \end{pmatrix} & \\
   5&: \quad P :=
   \begin{pmatrix}
    1 & 1 \\
    1 & 0
  \end{pmatrix} & \\
   6&: \quad \textbf{Für}~ k = 0, \dots, m & \\
   7&: \quad \quad \textbf{Falls}~ a_k = 1 & \\
   8&: \quad \quad \quad F := F \cdot P & \\
   9&: \quad \quad \textbf{Ende Falls} & \\
  11&: \quad \quad \textbf{Falls}~ k < m & \\
  12&: \quad \quad \quad P := P^2 & \\
  13&: \quad \quad \textbf{Ende Falls} & \\
  14&: \quad \textbf{Ende Für} & \\
  15&: \quad \textbf{Antworte}~ F & \\
  16&: \textbf{Ende Prozedur}
  \end{flalign*}
  
  Eine andere Möglichkeit die womöglich funktioniert ist die Folgende.
  
  \begin{flalign*}
  1&: \textbf{Prozedur}~ \textsc{MatrixPotenzieren}(M, n) & \\
  2&: \textbf{Falls } n = 1 & \\
  3&: \quad \textbf{Antworte } M & \\
  4&: \textbf{Sonst} & \\
  5&: \quad \textbf{Antworte } \textit{MatrixPotenzieren}\pbraces{M, \left\lfloor \frac{n}{2} \right \rfloor} \cdot \textit{MatrixPotenzieren}\pbraces{M, \left\lceil \frac{n}{2} \right\rceil} & \\
  6&: \textbf{Ende Falls} & \\
  7&: \textbf{Ende Prozedur}
  \end{flalign*}

\end{enumerate}

\end{solution}

% --------------------------------------------------------------------------------
