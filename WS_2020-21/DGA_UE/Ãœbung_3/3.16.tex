% --------------------------------------------------------------------------------

\begin{exercise}

Die Fibonacci-Zahlen seien durch die Rekursion $F_n = F_{n-1} + F_{n-2}$ für $n \geq 2$ mit Anfangswerten $F_0 = 0$ und $F_1 = 1$ definiert.
Die FIbonacci-Zahlen können effizient mittels folgender auf Matrizenmultiplikation beruhender Formel berechnet werden:

\begin{align*}
  \begin{pmatrix}
    F_{n+1} & F_n \\
    F_n     & F_{n-1}
  \end{pmatrix}
  =
  \begin{pmatrix}
    1 & 1 \\
    1 & 0
  \end{pmatrix}^n
  \quad
  ~\text{für}~
  n \geq 1
\end{align*}

\begin{enumerate}[label = \alph*.]

  \item Beweisen Sie diese Formel durch vollständige Induktion.

  \item Überlegen Sie sich einen Algorithmus, der
  $\begin{pmatrix}
    1 & 1 \\
    1 & 0
  \end{pmatrix}^n$
  in nur logarithmisch vielen Schritten berechnet.

\end{enumerate}

\end{exercise}

% --------------------------------------------------------------------------------

\begin{solution}

Wir nennen die linke Matrix $L_n$ und die rechte $R_n$.

\begin{enumerate}[label = \alph*.]

  \item IA($n = 0$):

  \begin{align*}
    F_{-1} := 0
    \implies
    A_0 = I_2 = B_0
  \end{align*}

  IS($n \mapsto n + 1$):
  
  \begin{multline*}
    \implies
    R_{n+1}
    =
    R_n
    \begin{pmatrix}
      1 & 1 \\
      1 & 0
    \end{pmatrix}
    =
    L_n
    \begin{pmatrix}
      1 & 1 \\
      1 & 0
    \end{pmatrix}
    =
    \begin{pmatrix}
      F_{n+1} & F_n \\
      F_n     & F_{n-1}
    \end{pmatrix}
    \begin{pmatrix}
      1 & 1 \\
      1 & 0
    \end{pmatrix} \\
    =
    \begin{pmatrix}
      F_{n+1} + F_n & F_{n+1} \\
      F_n + F_{n-1} & F_n
    \end{pmatrix}
    =
    \begin{pmatrix}
      F_{n+2} & F_{n+1} \\
      F_{n+1} & F_n
    \end{pmatrix}
    =
    L_{n+1}
  \end{multline*}

  \item Sei $n$ eine Zweierpotenz, d.h. $n \in \Bbraces{2^k: k \in \N_0}$.
  
  \begin{align*}
    \implies
    \begin{pmatrix}
      1 & 1 \\
      1 & 0
    \end{pmatrix}^n
    =
    \begin{pmatrix}
      1 & 1 \\
      1 & 0
    \end{pmatrix}^{2^k}
    =
    \underbrace
    {
      \pbraces
      {
        \begin{pmatrix}
          1 & 1 \\
          1 & 0
        \end{pmatrix}^2
        \cdots
      }^2
    }_{
      k \text{-mal}
    }
  \end{align*}

  Hätte man die explizite Darstellung der Fibonacci-Zahlen, so ginge dies sogar in konstanter Zeit.

\end{enumerate}

\end{solution}

% --------------------------------------------------------------------------------
