% --------------------------------------------------------------------------------

\begin{exercise}

Natural Merge-Sort ist eine Variante von Merge-Sort, die bereits vorsortierte Teilfolgen
(sogenannte runs) ausnutzt. Ein run ist eine Teilfolge aufeinanderfolgender Glieder $x_i, x_{i+1}, \dots, x_{i+k}$ mit $x_i \leq x_{i+1} \leq \dots \leq x_{i+k}$. Die Basis für den Verschmelzen-Vorgang bilden hier nicht
die rekursiv oder iterativ gewonnenen Zweiergruppen, sondern die runs. Im ersten Durchlauf des Algorithmus
bestimmt man die runs, anschließend fügt man die runs mittels VERSCHMELZEN zusammen.

\begin{enumerate}[label = (\alph*)]
  \item Sortieren Sie folgende Liste mittels Natural Merge-Sort:
  \begin{align}
    2,4,3,1,7,6,8,9,0,5
  \end{align}
  \item Schreiben Sie einen Pseudocode für Natural Merge-Sort.
  \item Machen Sie eine Best- sowie eine Worst-Case-Analyse für das Laufzeiverhalten von Natural Merge-Sort
  bei einem Eingabefeld der Größe $n$.
\end{enumerate}
\end{exercise}

% --------------------------------------------------------------------------------

\begin{solution}

ToDo!

\end{solution}

% --------------------------------------------------------------------------------
