% --------------------------------------------------------------------------------

\begin{exercise}

Natural Merge-Sort ist eine Variante von Merge-Sort, die bereits vorsortierte Teilfolgen (sogenannte runs) ausnutzt. Ein run ist eine Teilfolge aufeinanderfolgender Glieder $x_i, x_{i+1}, \dots, x_{i+k}$ mit $x_i \leq x_{i+1} \leq \dots \leq x_{i+k}$.
Die Basis für den Verschmelzen-Vorgang bilden hier nicht die rekursiv oder iterativ gewonnenen Zweiergruppen, sondern die runs.
Im ersten Durchlauf des Algorithmus bestimmt man die runs, anschließend fügt man die runs mittels VERSCHMELZEN zusammen.

\begin{enumerate}[label = (\alph*)]

  \item Sortieren Sie folgende Liste mittels Natural Merge-Sort:

  \begin{align}
    2, 4, 3, 1, 7, 6, 8, 9, 0, 5
  \end{align}

  \item Schreiben Sie einen Pseudocode für Natural Merge-Sort.

  \item Machen Sie eine Best- sowie eine Worst-Case-Analyse für das Laufzeitverhalten von Natural Merge-Sort bei einem Eingabefeld der Größe $n$.

\end{enumerate}

\end{exercise}

% --------------------------------------------------------------------------------

\begin{solution}

\phantom{}

\begin{enumerate}[label = (\alph*)]

  \item Wir haben dafür ein Python-Programm geschrieben.
  Der Output lautet wie folgt.

  \begin{verbatim}
    # ---------------------------------------------------------------- #

    [[2, 4], [3], [1, 7], [6, 8, 9], [0, 5]]

    [2, 4]
    [2, 3, 4]
    [1, 2, 3, 4, 7]
    [1, 2, 3, 4, 6, 7, 8, 9]
    [0, 1, 2, 3, 4, 5, 6, 7, 8, 9]

    [0, 1, 2, 3, 4, 5, 6, 7, 8, 9]

    # ---------------------------------------------------------------- #
  \end{verbatim}

  Jetzt machen wir's manuell.
  Zuerst bestimmen wir die Runs.

  \begin{align*}
    [2, 4], [3], [1, 7], [6, 8, 9], [0, 5]
  \end{align*}

  Nun verschmelzen wir.

  \begin{align*}
    [2, 4], [3], [1, 7], [6, 8, 9], [0, 5]
    & \rightsquigarrow
    [2, 3, 4], [1, 7], [6, 8, 9], [0, 5] \\
    & \rightsquigarrow
    [1, 2, 3, 4, 7], [6, 8, 9], [0, 5] \\
    & \rightsquigarrow
    [1, 2, 3, 4, 6, 7, 8, 9], [0, 5] \\
    & \rightsquigarrow
    [0, 1, 2, 3, 4, 5, 6, 7, 8, 9]
  \end{align*}

  \item

  \begin{flalign*}
     1&: \textbf{Prozedur}~ \textsc{Runs Verschmelzen} (R) & \\
     2&: \quad m := R.\textit{Länge} & \\
     3&: \quad C := R[1] & \\
     4&: \quad \textbf{Für}~ i = 2, \dots, m & \\
     5&: \quad \quad C := \textsc{Verschmelzen} (C, R[i]) & \\
     6&: \quad \textbf{Ende Für} & \\
     7&: \quad \textbf{Antworte~} C & \\
     8&: \textbf{Ende Prozedur}
  \end{flalign*}

  \begin{flalign*}
     1&: \textbf{Prozedur}~ \textsc{Natural Merge Sort} (A) & \\
     2&: \quad \text{Sei $R$ ein neuer Datenfeld der Länge $0$.} & \\
     3&: \quad n := A.\textit{Länge} & \\
     4&: \quad i := 1 & \\
     5&: \quad \textbf{Solange}~ i \leq n & \\
     6&: \quad \quad j := 0 & \\
     7&: \quad \quad r := [A[i]] & \\
     8&: \quad \quad \textbf{Solange}~ i+j+1 \leq n~ \textbf{und}~ A[i+j] \leq A[i+j+1] & \\
     9&: \quad \quad \quad r.\textit{append}(A[i+j+1]) & \\
     10&: \quad \quad \quad j := j + 1 & \\
     11&: \quad \quad \textbf{Ende Solange} & \\
     12&: \quad \quad R.\textit{append}(r) & \\
     13&: \quad \quad i := i + j + 1 & \\
     14&: \quad \textbf{Ende Solange} & \\
     15&: \quad \textbf{Antworte}~ \textsc{Runs Verschmelzen} (R) & \\
     16&: \textbf{Ende Prozedur}
  \end{flalign*}

  Man könnte den Algorithmus evtl. noch optimieren, indem man die Runs aufsteigend nach ihrer Länge sortiert.
  Damit spart man sich beim Verschmelzen etwas Aufwand.

  \item Sei $A$ ein Datenfeld der Länge $n$.
  Das Erstellen der Runs gelingt immer in linearem Aufwand.

  \begin{itemize}

    \item Best-Case:

    Sei das Datenfeld bereits aufsteigend sortiert.
    Der Aufwand liegt also nur in der Erstellung der Runs.
    Somit erhalten wir linearen Aufwand.

    \item Worst-Case:

    Sei das Datenfeld absteigend sortiert.
    Wir erhalten also $n$ Runs der Länge $1$.
    Abgesehen vom Erstellen der Runs, müssen wir die Prozedur Verschmelzen für $i = 2, \dots, n$, also $(n-1)$-mal, auf die Datenfelder $C$ und $R[i]$ der Länge $i-1$ bzw. $1$ anwenden.
    Verschmelzen hat linearen Aufwand.
    Wir kommen also insgesamt auf quadratischen Aufwand.

  \end{itemize}

\end{enumerate}

\end{solution}



% --------------------------------------------------------------------------------
