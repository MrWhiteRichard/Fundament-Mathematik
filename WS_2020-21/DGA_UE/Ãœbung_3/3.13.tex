% --------------------------------------------------------------------------------

\begin{exercise}

Gegeben sei ein zusammenhängender ungerichteter Graph $G = (V,E)$ mit einer geraden Anzahl an Knoten.
Zeigen Sie, dass es einen (nicht notwendigerweise zusammenhängenden) Untergraph mit
Knotenmenge $V$ gibt (also einen Graph $G^\prime = (V, E^\prime)$ mit $E^\prime \subseteq E$),
in dem alle Knotengrade ungerade sind.

(Hinweis: beweisen Sie die Behauptung für Bäume und begründen Sie, warum diese Annahme reicht.)

\end{exercise}

% --------------------------------------------------------------------------------

\begin{solution}
Der Grad eines Knoten $v \in V$
ist definert als $\grad(x) = |\{\{x,y\}: \{x,y\} \in E\}|$. \\
Wir beweisen die Aussage mit Induktion nach $n := |V|$: \\
Für $n = 2$ gibt es genau eine Kante, welche die beiden Knoten verbindet,
also gilt die Aussage bereits für $E^{\prime} = E$. \\
Induktionsschritt: $n \rightsquigarrow n + 2$: \\
Betrachte einen beliebigen Baum $G$ mit $V = \{x_1,\dots,x_{n+2}\}$.
Nun existiert ein $v \in V$ mit $\grad(v) = 1$ (o.B.d.A. $v = x_{n+2}$).
Nun ist $G_1 = (V_1,E_1)$ mit
\begin{align*}
  V_1 &:= \{x_1,\dots,x_{n+1}\} \\
  E_1 &:= \{\{x,y\} \in E: x,y \in V_1\}
\end{align*}
klarerweise immer noch zyklenfrei und zusammenhängend und wir können
ein weiteres $v \in V_1$ mit $\grad(v) = 1$ (o.B.d.A: $v = x_{n+1}$) finden, sodass
dann $G = (V_2,E_2)$ mit
\begin{align*}
  V_2 &:= \{x_1,\dots,x_{n}\} \\
    E_1 &:= \{\{x,y\} \in E: x,y \in V_2\}
\end{align*}
ein zusammenhängender, zyklenfreier Graph ist, auf den wir die Induktionsvoraussetzung
anwenden können, also erhalten wir $E_2^{\prime} \subseteq E_2$, sodass
für alle $v \in V_2: \grad(v) = 1$. \\
Nun definieren wir $E^{\prime} = E_2^{\prime} \cup \{\{x_{n+1},x_{n+2}\}\}$, welches
die Bedingung dann für alle $v \in V$ erfüllt. \\


Für einen beliebig zusammenhängenden Graphen $G = (V,E)$ mit gerader Knotenanzahl
erinnern wir uns daran, dass
ein Baum genau ein minimal zusammenhängender Graph ist.
Also finden wir in jedem Fall ein $E_1 \subseteq E$, sodass $G_1 = (V,E_1)$
ein Baum ist und wir das soeben gezeigte anwenden können.  \\


---------------------Alternative Lösung: ---------------\\

Der Grad eines Knoten $v \in V$
ist definert als $\grad(x) = |\{\{x,y\}: \{x,y\} \in E\}|$. \\
Wir beweisen die Aussage mit Induktion nach $n := |E| = |V-1|$: \\

Für $n = 1$ gibt es genau eine Kante, welche die beiden Knoten verbindet,
also gilt die Aussage bereits für $E^{\prime} = E$. \\

Induktionsschritt: $n \rightsquigarrow n + 2$: \\
Betrachte einen beliebigen Baum $G = (V,E)$.
O.B.d.A. existiert ein $v \in V$ mit $\grad(v) = 2k$ (sonst ist die Aussage schon erfüllt). Sei $\{x_1,\dots,x_{2k}\} = \{x \in E: \{x,v \} \in V\}$.
Da ganz $G$ aus einer geraden Anzahl von Knoten besteht, gibt es also mindestens einen von $v$ ausgehenden \textquoteleft{Arm}\textquoteright ~$G_i$, der auch aus einer geraden Anzahl von Knoten besteht.

Mit einem \textquoteleft{Arm}\textquoteright ~ $G_i$ ist hier jeweils die Zusammenhangskomponente von $G$ gemeint, die genau einen Knoten $\{x_i,v\}$ enthält und alle anderen Knoten $\{x_j,v\}$ für $j \neq i$ nicht.

Wenn wir von $G$ nun den Knoten $\{x_i,v\}$ des \textquoteleft{Armes}\textquoteright ~$G_i$ mit gerader Knotenanzahl entfernen, bekommen wir zwei Zusammenhangskomponenten mit ungerader Knotenanzahl (die natürlich zyklenfrei und somit Bäume sind). Beide haben eine Knotenanzahl $\leq n$ und somit können wir die IV auf beide anwenden. \\

---------------------------------------------------------\\


Für einen beliebig zusammenhängenden Graphen $G = (V,E)$ mit gerader Knotenanzahl
erinnern wir uns daran, dass
ein Baum genau ein minimal zusammenhängender Graph ist.
Also finden wir in jedem Fall ein $E_1 \subseteq E$, sodass $G_1 = (V,E_1)$
ein Baum ist und wir das soeben gezeigte anwenden können.

\end{solution}

% --------------------------------------------------------------------------------
