% --------------------------------------------------------------------------------

\begin{exercise}

Gegeben sei ein zusammenhängender ungerichteter Graph $G = (V, E)$ mit einer geraden Anzahl an Knoten.
Zeigen Sie, dass es einen (nicht notwendigerweise zusammenhängenden) Untergraph mit Knotenmenge $V$ gibt (also einen Graph $G^\prime = (V, E^\prime)$ mit $E^\prime \subseteq E$),  in dem alle Knotengrade ungerade sind.

(Hinweis: beweisen Sie die Behauptung für Bäume und begründen Sie, warum diese Annahme reicht.)

\end{exercise}

% --------------------------------------------------------------------------------

\begin{comment}

# Die Aussage, die hier gezeigt wird ist zu stark!
# Es gibt ein Gegenbeispiel in Beispiel 2.11!

\begin{solution}
Der Grad eines Knoten $v \in V$ ist definert als

\begin{align*}
  \grad(x)
  =
  |\Bbraces{\Bbraces{x, y}: \Bbraces{x, y} \in E}|.
\end{align*}

Wir beweisen die Aussage mit Induktion nach $n := |V|$. \\

IA($n = 2$):

Es gibt genau eine Kante, welche die beiden Knoten verbindet.
Also gilt die Aussage bereits für $E^\prime = E$. \\

IS($n \mapsto n + 2$):

Betrachte einen beliebigen Baum $G$ mit $V = \Bbraces{x_1, \dots, x_{n+2}}$.

\begin{align*}
  \implies
  \Exists v \in V:
  \grad(v) = 1,
  \quad
  \text{(o.B.d.A. $v = x_{n+2}$)}
\end{align*}

\begin{align*}
  G_1 & := (V_1, E_1) \\
  V_1 & := \Bbraces{x_1, \dots, x_{n+1}} \\
  E_1 & := \Bbraces{\Bbraces{x, y} \in E: x, y \in V_1}
\end{align*}

Klarerweise ist $G_1$ immer noch zyklenfrei und zusammenhängend.

\begin{align*}
  \implies
  \Exists v \in V_1:
  \grad(v) = 1,
  \quad
  \text{(o.B.d.A. $v = x_{n+1}$)}
\end{align*}

\begin{align*}
  G_2 & := (V_2, E_2) \\
  V_2 & := \Bbraces{x_1, \dots, x_{n}} \\
  E_2 & := \Bbraces{\Bbraces{x, y} \in E: x, y \in V_2}
\end{align*}

$G_2$ ist ein zusammenhängender, zyklenfreier Graph, auf den wir die Induktionsvoraussetzung anwenden können.

\begin{align*}
  \implies
  \Exists E_2^\prime \subseteq E_2:
  \Forall v \in V_2:
  \grad(v) = 1
\end{align*}

# Wir wissen aber nicht, ob $\Bbraces{x_{n+1}, x_{n+2}} \in E$!
Schließlich setzen wir $E^\prime := E_2^\prime \cup \Bbraces{\Bbraces{x_{n+1}, x_{n+2}}}$. \\

\end{solution}

\end{comment}

% --------------------------------------------------------------------------------

\begin{solution}

Der Grad eines Knoten $v \in V$ ist definert als

\begin{align*}
  \grad(x)
  =
  |\Bbraces{\Bbraces{x, y}: \Bbraces{x, y} \in E}|.
\end{align*}

Wir beweisen die Aussage mit Induktion nach $n := |E| = |V-1|$. \\

IA($n = 1$):

Es gibt genau eine Kante, welche die beiden Knoten verbindet.
Also gilt die Aussage bereits für $E^\prime = E$. \\

IS($n \mapsto n + 2$):

Betrachte einen beliebigen Baum $G = (V, E)$ mit einer geraden Anzahl an Knoten.
O.B.d.A. $\Exists v \in V: \grad(v) = 2k$, $k \in \N$, sonst wäre die Aussage schon erfüllt.
Wir betrachten Menge aller Kanten aus $E$, die an $v$ angrenzen.

\begin{align*}
  \Bbraces{x_1, \dots, x_{2k}}
  =
  \Bbraces{x \in E: \Bbraces{x, v} \in V}  
\end{align*}

Löscht man diese Kanten und den Knoten $v$ aus $G$, so zerfällt der neue Graph in Zusammenhangskomponenten $G_1^\prime, \dots, G_{2k}^\prime$.
Für $i = 1, \dots, 2k$, ergänzen wir $G_i^\prime$ um den Knoten $v$ und Kante $\Bbraces{x_i, v}$ zum \blockquote{Ast} $G_i$. \\

$G$ besteht aus einer geraden Anzahl von Knoten und somit aus einer ungeraden Anzahl an Kanten.
Also gibt es also mindestens einen (von $v$ ausgehenden) Ast $G_i$, der eine gerade Kanten-Zahl hat.
(Wenn alle Äste (gerade viele) ungerade Kanten-Zahl hätten, wäre die Gesamt-Kanten-Zahl gerade mal ungerade, also gerade!)
\blockquote{$G \setminus G_i}$ hat dann eine ungerade Kanten-Zahl, weil \blockquote{$G = G_i + G \setminus G_i$} ja eine ungerade Kanten-Zahl hat. \\

Wir löschen von $G$ nun die Kante $\Bbraces{x_i, v}$ (des Asts $G_i$ (mit gerader Kanten-Zahl)).
Dann erhalten wir zwei Zusammenhangskomponenten mit ungerader Kantenanzahl $\leq n$.
Die sind wieder Bäume.
Somit können wir die Induktionsvoraussetzung auf Beide anwenden.
Wir vereinigen die resultierenden Graphen und erhalten $G^\prime = (V, E^\prime)$. \\

Sei $G = (V, E)$ ein beliebiger zusammenhängender Graph mit gerader Knotenanzahl.
Ein Baum ist genau ein minimal zusammenhängender Graph.
Wir finden einen solchen Teilgraphen (\blockquote{Spannbaum}) $G_0 = (V, E_0)$ mit $E_0 \subseteq E$.
Auf den wenden wir das oben gezeigte an.

\end{solution}

% --------------------------------------------------------------------------------
