% --------------------------------------------------------------------------------

\begin{exercise}

Gegeben sei ein zusammenhängender ungerichteter Graph $G = (V, E)$ mit einer geraden Anzahl an Knoten.
Zeigen Sie, dass es einen (nicht notwendigerweise zusammenhängenden) Untergraph mit Knotenmenge $V$ gibt (also einen Graph $G^\prime = (V, E^\prime)$ mit $E^\prime \subseteq E$),  in dem alle Knotengrade ungerade sind.

(Hinweis: beweisen Sie die Behauptung für Bäume und begründen Sie, warum diese Annahme reicht.)

\end{exercise}

% --------------------------------------------------------------------------------

\begin{solution}
Der Grad eines Knoten $v \in V$ ist definert als

\begin{align*}
  \grad(x)
  =
  |\Bbraces{\Bbraces{x, y}: \Bbraces{x, y} \in E}|.
\end{align*}

Wir beweisen die Aussage mit Induktion nach $n := |V|$. \\

IA($n = 2$):

Es gibt genau eine Kante, welche die beiden Knoten verbindet.
Also gilt die Aussage bereits für $E^\prime = E$. \\

IS($n \mapsto n + 2$):

Betrachte einen beliebigen Baum $G$ mit $V = \Bbraces{x_1, \dots, x_{n+2}}$.

\begin{align*}
  \implies
  \Exists v \in V:
  \grad(v) = 1,
  \quad
  \text{(o.B.d.A. $v = x_{n+2}$)}
\end{align*}

\begin{align*}
  G_1 & := (V_1, E_1) \\
  V_1 & := \Bbraces{x_1, \dots, x_{n+1}} \\
  E_1 & := \Bbraces{\Bbraces{x, y} \in E: x, y \in V_1}
\end{align*}

Klarerweise ist $G_1$ immer noch zyklenfrei und zusammenhängend.

\begin{align*}
  \implies
  \Exists v \in V_1:
  \grad(v) = 1,
  \quad
  \text{(o.B.d.A. $v = x_{n+1}$)}
\end{align*}

\begin{align*}
  G_2 & := (V_2, E_2) \\
  V_2 & := \Bbraces{x_1, \dots, x_{n}} \\
  E_2 & := \Bbraces{\Bbraces{x, y} \in E: x, y \in V_2}
\end{align*}

$G_2$ ist ein zusammenhängender, zyklenfreier Graph, auf den wir die Induktionsvoraussetzung anwenden können.

\begin{align*}
  \implies
  \Exists E_2^\prime \subseteq E_2:
  \Forall v \in V_2:
  \grad(v) = 1
\end{align*}

Schließlich setzen wir $E^\prime := E_2^\prime \cup \Bbraces{\Bbraces{x_{n+1}, x_{n+2}}}$. \\

Sei $G = (V, E)$ ein beliebiger zusammenhängender Graph mit gerader Knotenanzahl.
Ein Baum ist genau ein minimal zusammenhängender Graph.
Wir finden einen solchen Teilgraphen (\Quote{Spannbaum}) $G_0 = (V, E_0)$ mit $E_0 \subseteq E$.
Auf den wenden wir das oben gezeigte an. \\

\end{solution}

% --------------------------------------------------------------------------------

\begin{solution}

Der Grad eines Knoten $v \in V$
ist definert als $\grad(x) = |\{\{x, y\}: \{x, y\} \in E\}|$. \\
Wir beweisen die Aussage mit Induktion nach $n := |E| = |V-1|$: \\

Für $n = 1$ gibt es genau eine Kante, welche die beiden Knoten verbindet, 
also gilt die Aussage bereits für $E^\prime = E$. \\

Induktionsschritt: $n \rightsquigarrow n + 2$: \\
Betrachte einen beliebigen Baum $G = (V, E)$.
O.B.d.A. existiert ein $v \in V$ mit $\grad(v) = 2k$ (sonst ist die Aussage schon erfüllt). Sei $\{x_1, \dots, x_{2k}\} = \{x \in E: \{x, v \} \in V\}$.
Da ganz $G$ aus einer geraden Anzahl von Knoten besteht (und somit aus einer ungeraden Anzahl an Kanten), gibt es also mindestens einen von $v$ ausgehenden \textquoteleft{Arm}\textquoteright ~$G_i$, der aus einer geraden Anzahl von Kanten besteht.

Mit einem \textquoteleft{Arm}\textquoteright ~ $G_i$ ist hier jeweils die Zusammenhangskomponente von $G$ gemeint, die genau einen Kanten $\{x_i, v\}$ enthält und alle anderen Kanten $\{x_j, v\}$ für $j \neq i$ nicht.

Wenn wir von $G$ nun die Kante $\{x_i, v\}$ des \textquoteleft{Armes}\textquoteright ~$G_i$ mit gerader Kantenanzahl entfernen, bekommen wir zwei Zusammenhangskomponenten mit ungerader Kantenanzahl (die natürlich zyklenfrei und somit Bäume sind). Beide haben eine Kantenanzahl $\leq n$ und somit können wir die IV auf beide anwenden. \\  

\end{solution}

% --------------------------------------------------------------------------------
