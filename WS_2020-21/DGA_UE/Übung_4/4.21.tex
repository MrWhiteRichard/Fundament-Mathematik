\begin{exercise}
Gegeben sei die durch folgenden Algorithmus definierte Funktion
\begin{flalign*}
  &F(n,a,b,c): & \\
  &\textbf{if } n = 0 \textbf{ then} & \\
  &\quad \text{return } 1 & \\
  &\textbf{end if} & \\
  &\textbf{if } n = 1 \textbf{ then} & \\
  &\quad \text{return } 2 & \\
  &\textbf{end if} & \\
  &A = F(n-2,b,c,a) & \\
  &B = F(n-2,c,a,b) & \\
  &C = F(n-2,b,a,c) & \\
  &\textbf{if } F(n-1,A,B,C) \equiv 0\ (\mathrm{mod}\ 3) \textbf{ then} & \\
  & \quad \text{return } F(n-1,A-1,B,C+1) + F(0,A,B,C) & \\
  &\textbf{else} & \\
  & \quad \text{return } F(n-1,B-1,A+B+C,A\cdot C) + F(0,A,B,C) & \\
  &\textbf{end if} &
\end{flalign*}
Bezeichne $a(n)$ die Anzahl der Aufrufe von $F(0,x,y,z)$ mit irgendwelchen Parametern
$x,y$ und $z$ bei der Berechnung von $F(n,a,b,c)$. Bestimmen Sie eine Rekursionsgleichung
(inklusive Anfangsbedingungen) für $(a(n))_{n\in\N}$ und lösen Sie diese.
\end{exercise}

% --------------------------------------------------------------------------------

\begin{solution}
Klarerweise gilt $a(0) = 1, a(1) = 0$. Wir erhalten außerdem die Rekursion
\begin{align*}
  a(n) = a(n-1) + 3a(n-2) + 1.
\end{align*}
Diese inhomogene, lineare Rekursionsgleichung 2-ter Ordnung mit konstanten
Koeffizienten lässt sich durch $a(n) = y(n) - \frac{d}{C-1} = y(n) - \frac{1}{4}$ lösen, wobei $y(n)$
Lösung von
\begin{align*}
  y(0) &= a(0) + \frac{1}{4} = \frac{5}{4}, \quad y(1) = a(1) + \frac{1}{4} = \frac{1}{4}, \\
  y(n) &= y(n-1) + 3y(n-2), \quad n \geq 2.
\end{align*}
Das charakteristische Polynom davon lautet
\begin{align*}
  \chi(\lambda) = \lambda^2 - \lambda - 3 \implies \lambda_{1,2} = \frac{1}{2} \pm \sqrt{\frac{1}{4} + 3}
  = \frac{1}{2} \pm \frac{\sqrt{13}}{2}.
\end{align*}
Also erhalten wir die Lösung
\begin{align*}
  y(n) = c_0\left(\frac{1}{2} + \frac{\sqrt{13}}{2}\right)^n +
        c_1\left(\frac{1}{2} - \frac{\sqrt{13}}{2}\right)^n.
\end{align*}
Lösen wir die Konstanten nach den Anfangsbedingungen auf:
\begin{align*}
  \frac{5}{4} = y(0) &\stackrel{!}{=} c_0 + c_1 \\
  \frac{1}{4} = y(1) &\stackrel{!}{=} c_0\left(\frac{1}{2} + \frac{\sqrt{13}}{2}\right) +
        c_1\left(\frac{1}{2} - \frac{\sqrt{13}}{2}\right)
        =c_0\left(\frac{1}{2} + \frac{\sqrt{13}}{2}\right) +
              (\frac{5}{4} - c_0)\left(\frac{1}{2} - \frac{\sqrt{13}}{2}\right) \\
        &= c_0\left(2\frac{\sqrt{13}}{2}\right) + \frac{5}{4}\left(\frac{1}{2} - \frac{\sqrt{13}}{2}\right)
        = c_0\sqrt{13} + \frac{5 - 5\sqrt{13}}{8} \\
        \iff c_0 &= \frac{\frac{1}{4} - \frac{5 - 5\sqrt{13}}{8}}{\sqrt{13}}
        = \frac{5\sqrt{13}- 3}{8\sqrt{13}} = \frac{5}{8} - \frac{3}{8\sqrt{13}} \\
        \iff c_1 &= \frac{5}{4} - \frac{5}{8} + \frac{3}{8\sqrt{13}}
        = \frac{5}{8} + \frac{3}{8\sqrt{13}}
\end{align*}
Schließlich erhalten wir für $a(n)$:
\begin{align*}
  a(n) = y(n) - \frac{1}{4} = \left(\frac{5}{8} - \frac{3}{8\sqrt{13}}\right)\left(\frac{1}{2} +
   \frac{\sqrt{13}}{2}\right)^n + \left(\frac{5}{8} + \frac{3}{8\sqrt{13}}\right)\left(\frac{1}{2} - \frac{\sqrt{13}}{2}\right)^n - \frac{1}{4}.
\end{align*}
\end{solution}

% --------------------------------------------------------------------------------
