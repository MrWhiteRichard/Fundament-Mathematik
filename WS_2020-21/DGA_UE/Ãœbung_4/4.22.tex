\begin{exercise}
Die Türme von Hanoi: Gegeben seien drei Stäbe $A,B,C$ und $n$ verschieden große
Scheiben. Anfangs seien alle Scheiben der Größe nach auf Stab $A$ aufgereiht, die
größte ganz unten. Dieser Turm von Scheiben soll nun unter folgenden Regeln von
$A$ nach $B$ transferiert werden:
\begin{itemize}
  \item In jedem Zug darf nur eine Scheibe bewegt werden.
  \item Eine größere Scheibe darf nie über einer kleineren platziert werden.
\end{itemize}
Sei $a_n$ die minimale Anzahl der benötigten Züge. Ermitteln Sie $a_n$ durch
Aufstellen und Lösen einer Rekursion.
\end{exercise}

% --------------------------------------------------------------------------------

\begin{solution}
In den Regeln in der Angabe ist nicht explizit verlangt, dass immer nur die
oberste Scheibe bewegt werden kann. Also bewegen wir einfach die unterste
zuerst und erhalten einfach $a_n = n$. :) \\
Ernsthafte Lösung: \\
Wir haben in EProg einen rekurisven Algorithmus dafür programmiert, der
folgende Rekursion verwendet:
\begin{enumerate}
  \item Verschiebe die obersten $n-1$ Scheiben von Pfosten $A$ auf Pfosten $B$.
  \item Verschiebe die größte Scheibe von Pfosten $A$ auf Pfosten $C$.
  \item Verschibe die $n-1$ Scheiben von Pfosten $B$ auf Pfosten $C$.
\end{enumerate}
Also erhalten wir für $a_n$ folgende Rekursionsgleichung:
\begin{align*}
  a_n = 2a_{n-1} + 1, \quad n \geq 2.
\end{align*}
mit der expliziten Lösung
\begin{align*}
  a_n = \sum_{i=1}^n\prod_{j=i+1}^n2 = 2^n - 1.
\end{align*}
\end{solution}

% --------------------------------------------------------------------------------
