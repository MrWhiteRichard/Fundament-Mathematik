\begin{exercise}
  Lösen Sie folgendes System von Rekursionen

  \begin{align*}
    a_{n+1} = 2 a_n + b_n, \quad b_{n+1} = 4 a_n - b_n, \quad \text{für}~ n \geq 1
  \end{align*}

  mit den Anfangsbedingungen $a_0 = 1, b_0 = 0$ indem Sie das System in eine äquivalente Rekursion $2.$ Ordnung umschreiben.
\end{exercise}

% -------------------------------------------------------------------------------- %
% Vorzeichenfehler
\begin{comment}
\begin{solution}
  Wir formen zuerst die erste Rekursion um und setzen in die zweite ein

  \begin{align*}
    &a_n = 2a_{n-1} + b_{n-1}
    \Leftrightarrow
    b_{n-1} = a_n - 2a_{n-1} \\
    \Rightarrow~
    & b_n = 4a_{n-1} - 2a_{n-1} + a_n
    =
    2a_{n-1} + a_n\\
    \Rightarrow~
    &b_{n-1} = 2a_{n-2} + a_{n-1}
  \end{align*}

  Setzen wir nun $b_{n-1}$ in die Gleichung aus der ersten Zeile ein erhalten wir also

  \begin{align*}
    a_n
    =
    3a_{n-1} + 2a_{n-2}, \quad \text{für}~ n \geq 2
  \end{align*}

  Mit den Anfangsbedingungen $a_0 = 1, a_1 = 2$. Diese Rekursion $2.$ Ordnung lösen wir nun mithilfe von Satz $4.2$, das charakteristische Polynom unserer Rekursionsgleichung lautet:

  \begin{align*}
    \chi(z)
    =
    z^2 - 3z - 2
  \end{align*}

  Die Nullstellen sind:

  \begin{align*}
    \chi(z) = 0
    \Leftrightarrow
    z_{1,2}
    =
    \frac{3}{2} \pm \frac{\sqrt{17}}{2}
  \end{align*}

  Allgemeine Lösungen unserer Rekursion haben also die Form:

  \begin{align*}
    a_n
    =
    c_0 \Big(\frac{3 +\sqrt{17}}{2}\Big)^n + c_1 \Big(\frac{3 -\sqrt{17}}{2}\Big)^n
  \end{align*}

  Um nun $c_0, c_1$ zu bestimmen sehen wir uns die Anfangsbedingungen an.

  \begin{align*}
    c_0 + c_1 = 1 \\
    c_0 \Big(\frac{3 +\sqrt{17}}{2}\Big) + c_1\Big(\frac{3 -\sqrt{17}}{2}\Big) = 2
  \end{align*}

  Die Lösung dieses LGS ist gegeben durch

  \begin{align*}
    c_0 = \frac{\sqrt{17} + 17}{34} \quad
    c_1 = \frac{-\sqrt{17} +17}{34}
  \end{align*}

  Somit lautet die Lösung der Rekursion $2.$ Ordnung

  \begin{align*}
  a_n
  =
  \frac{\sqrt{17} + 17}{34} \Big(\frac{3 +\sqrt{17}}{2}\Big)^n
    + \frac{-\sqrt{17} +17}{34} \Big(\frac{3 -\sqrt{17}}{2}\Big)^n
  \end{align*}

  Jetzt gilt es wohl noch diese Lösung auf das System zurückzuführen.
\end{solution}
\end{comment}
% -------------------------------------------------------------------------------- %

\begin{solution}
	Wir formen zuerst die erste Rekursion um und setzen in die zweite ein

	\begin{align*}
	&a_n = 2a_{n-1} + b_{n-1}
	\Leftrightarrow
	b_{n-1} = a_n - 2a_{n-1} \\
	\Rightarrow~
	& b_n = 4a_{n-1} - b_{n - 1} = 4a_{n-1} - (a_n - 2a_{n-1}) = 6a_{n-1} - a_n\\
	\Rightarrow~
	&b_{n-1} = 6a_{n-2} - a_{n-1}
	\end{align*}

	Setzen wir nun $b_{n-1}$ in die Gleichung aus der ersten Zeile ein erhalten wir also

	\begin{align*}
	a_n
	= 2a_{n-1} + b_{n-1} = 2a_{n-1} + 6a_{n-2} - a_{n-1} = a_{n-1} + 6a_{n - 2} , \quad \text{für}~ n \geq 2
	\end{align*}

	Mit den Anfangsbedingungen $a_0 = 1, a_1 = 2a_0 + b_0 = 2$. Diese Rekursion $2.$ Ordnung lösen wir nun mithilfe von Satz $4.2$, das charakteristische Polynom unserer Rekursionsgleichung lautet:

	\begin{align*}
	\chi(z)
	=
	z^2 - z -6 = (z - 3)(z + 2)
	\end{align*}

	Allgemeine Lösungen unserer Rekursion haben also die Form:

	\begin{align*}
	a_n
	=
	c_0 (-2)^n + c_1 3^n
	\end{align*}
  Jetzt lösen wir noch für die Anfangsbedingungen:
  \begin{align*}
    1 = a_0 &\stackrel{!}{=} c_0 + c_1 \\
    2 = a_1 &\stackrel{!}{=} -2c_0 + 3c_1 = 2(c_1 - 1) + 3c_1 = 5c_1 - 2 \iff c_1 = \frac{4}{5} \iff c_0 =
    \frac{1}{5}.
  \end{align*}
  Die konkrete Lösung lautet also für $n\geq 1$
  \begin{align*}
    a_n &= \frac{1}{5}(-2)^n + \frac{4}{5}3^n \\
    b_n &= 6a_{n-1} - a_n = \frac{6}{5}(-2)^{n-1} + \frac{24}{5}3^{n-1} - \frac{1}{5}(-2)^n - \frac{4}{5}3^n \\
    &= \frac{8}{5}(-2)^{n-1} + \frac{12}{5}3^{n-1} = \frac{4}{5}3^n - \frac{4}{5}(-2)^n 
  \end{align*}

\end{solution}
