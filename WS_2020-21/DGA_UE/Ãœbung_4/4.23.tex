\begin{exercise}
In Übungsblatt 3 haben wir gesehen, wie sich die Fibonacci-Zahlen mithilfe von
Potenzen der Matrix $\begin{pmatrix}1 & 1 \\ 1 & 0\end{pmatrix}$ effizient berechnen
lassen. Man kann diese Methode adaptieren, um Zahlenfolgen, die andere Rekursionsgleichungen
erfüllen, zu berechnen. Wir betrachten im Folgenden die Potenzen der Matrix
\begin{align*}
  M = \begin{pmatrix}
    3 & 2 \\ 2 & 0
  \end{pmatrix}.
\end{align*}
Definiere
\begin{align*}
  M_n := M^n = \begin{pmatrix}3 & 2 \\ 2 & 0\end{pmatrix}^n
  = \begin{pmatrix}a_n & b_n \\ c_n & d_n\end{pmatrix}
\end{align*}
und betrachte die Koeffizienten dieser Matrix.
\begin{enumerate}[label = \alph*)]
  \item Geben Sie eine Rekursion für die Zahlenfolge $(a_n)_{n \geq 0}$ an.
  \item Lösen Sie diese Rekursion, um eine explizite Formel für $a_n$ zu erhalten.
  \item Geben Sie einen Algorithmus zur effizienten Berechnung von $M^n$ an
  und analysieren Sie dessen Laufzeit in Abhängigkeit von $n$ ($\Landau$-Notation genügt).
\end{enumerate}
\end{exercise}

% --------------------------------------------------------------------------------

\begin{solution}
\phantom{}
\begin{enumerate}[label = \alph*)]
  \item
  \begin{align*}
    \begin{pmatrix}a_n & b_n \\ c_n & d_n\end{pmatrix}
    = \begin{pmatrix}3 & 2 \\ 2 & 0\end{pmatrix}\cdot
    \begin{pmatrix}a_{n-1} & b_{n-1} \\ c_{n-1} & d_{n-1}\end{pmatrix}
    = \begin{pmatrix}
      3a_{n-1} + 2c_{n-1} & 3b_{n-1} + 2d_{n-1} \\ 2a_{n-1} & 2b_{n-1}
    \end{pmatrix}
  \end{align*}
  Wir erhalten die Rekursion
  \begin{align*}
    a_n &= 3a_{n-1} + 2c_{n-1} = 3a_{n-1} + 4a_{n-2}.
  \end{align*}
  \item Nullstellen des charakteristischen Polynoms:
  \begin{align*}
    \chi(\lambda) = \lambda^2 - 3\lambda - 4, \quad \lambda_{1,2} = \frac{3}{2} \pm \sqrt{\frac{9}{4} + 4}
    = \frac{3}{2} \pm \frac{5}{2}
  \end{align*}
  allgemeine Lösung:
  \begin{align*}
    a_n = \alpha_0(-1)^n + \alpha_1 4^n
  \end{align*}
  Anfangsbedingungen:
  \begin{align*}
    a_0 = 1 &\stackrel{!}{=} \alpha_0 + \alpha_1 \\
    a_1 = 3 &\stackrel{!}{=} -\alpha_0 + 4\alpha_1 = 4\alpha_1 + \alpha_1 - 1 \iff \alpha_1 = \frac{4}{5}
    \iff \alpha_0 = \frac{1}{5}.
  \end{align*}
  explizite Lösung:
  \begin{align*}
    a_n &= \frac{1}{5}(-1)^n  + \frac{4}{5}4^n \\
    c_n &= 2a_{n-1} = \frac{2}{5}(-1)^{n-1} + \frac{8}{5}4^{n-1} = -\frac{2}{5}(-1)^n + \frac{2}{5}4^n
  \end{align*}
  \item Wir erkennen aus der Darstellung in a), dass wir $b_n, d_n$ durch die
  selbe Rekursiongleichung berechnen können mit Anfangswerten $b_0 = 0, b_1 = 2$.
  \begin{align*}
    b_0 = 0 &\stackrel{!}{=} \beta_0 + \beta_1 \\
    b_1 = 2 &\stackrel{!}{=} -\beta_0 + 4\beta_1 = 4\beta_1 + \beta_1 \iff \beta_1 = \frac{2}{5} \iff \beta_0 = -\frac{2}{5}.
  \end{align*}
  explizite Lösung von $b_n,d_n$:
  \begin{align*}
    b_n &= -\frac{2}{5}(-1)^n + \frac{2}{5}4^n \\
    d_n &= 2b_{n-1} = -\frac{4}{5}(-1)^{n-1} + \frac{4}{5}4^{n-1} = \frac{4}{5}(-1)^n + \frac{1}{5}4^n
  \end{align*}
  Also besteht der Aufwand des Algorithmus lediglich darin, die $4$-er Potenzen
  zu berechnen, sowie eine konstante Anzahl von Additionen und Multiplikationen.
  Wie beim Algorithmus zur expliziten Berechnung der Fibonacci-Zahlen können
  wir $n = \sum_{i=0}^k a_i2^i$ in Binärform darstellen und $4^n = \prod_{i=0}^k 4^{a_i 2^i}$
  in logarithmischer Zeit berechnen, insgesamt erhalten wir also $\Landau(\log(n))$
  für den Aufwand. Im folgenden Algorithmus ist $m \leq \left\lfloor \log(n) \right\rfloor$.
    \begin{flalign*}
  1&: \textbf{Prozedur}~ \textsc{ZahlPotenzieren}(x,n) \\
  2&: \quad u := \textsc{Binärkoeffizienten}(n) \\
  3&: \quad m := u.\textit{Länge} \\
  4&: \quad y := 1 \\
  6&: \quad \textbf{Für}~ k = 0, \dots, m \\
  7&: \quad \quad \textbf{Falls}~ a_k = 1 \\
  8&: \quad \quad \quad y := y \cdot x \\
  9&: \quad \quad \textbf{Ende Falls} \\
  11&: \quad \quad \textbf{Falls}~ k < m \\
  12&: \quad \quad \quad x := x^2 \\
  13&: \quad \quad \textbf{Ende Falls} \\
  14&: \quad \textbf{Ende Für} \\
  15&: \quad \textbf{Antworte}~ y \\
  16&: \textbf{Ende Prozedur}
  \end{flalign*}
\end{enumerate}


\end{solution}

% --------------------------------------------------------------------------------
