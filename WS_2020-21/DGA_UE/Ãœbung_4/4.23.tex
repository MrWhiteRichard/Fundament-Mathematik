\begin{exercise}
In Übungsblatt 3 haben wir gesehen, wie sich die Fibonacci-Zahlen mithilfe von
Potenzen der Matrix $\begin{pmatrix}1 & 1 \\ 1 & 0\end{pmatrix}$ effizient berechnen
lassen. Man kann diese Methode adaptieren, um Zahlenfolgen, die andere Rekursionsgleichungen
erfüllen, zu berechnen. Wir betrachten im Folgenden die Potenzen der Matrix
\begin{align*}
  M = \begin{pmatrix}
    3 & 2 \\ 2 & 0
  \end{pmatrix}.
\end{align*}
Definiere
\begin{align*}
  M_n := M^n = \begin{pmatrix}3 & 2 \\ 2 & 0\end{pmatrix}^n
  = \begin{pmatrix}a_n & b_n \\ c_n & d_n\end{pmatrix}
\end{align*}
und betrachte die Koeffizienten dieser Matrix.
\begin{enumerate}[label = \alph*)]
  \item Geben Sie eine Rekursion für die Zahlenfolge $(a_n)_{n \geq 0}$ an.
  \item Lösen Sie diese Rekursion, um eine explizite Formel für $a_n$ zu erhalten.
  \item Geben Sie einen Algorithmus zur effizienten Berechnung von $M^n$ an
  und analysieren Sie dessen Laufzeit in Abhängigkeit von $n$ ($\Landau$-Notation genügt).
\end{enumerate}
\end{exercise}

% --------------------------------------------------------------------------------

\begin{solution}
\phantom{}
\end{solution}

% --------------------------------------------------------------------------------
