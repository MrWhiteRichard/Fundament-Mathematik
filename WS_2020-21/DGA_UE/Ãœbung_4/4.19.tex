% -------------------------------------------------------------------------------- %

\begin{exercise}

	Lösen Sie die Rekursion $n a_n = (n + 1) a_{n - 1} + 2n$ mit $a_0 = 0$.

\end{exercise}


\begin{solution}

	Wir dividieren auf beiden Seiten durch $n$ und erhalten die Rekursionsgleichung
	\begin{align*}
	a_n = \frac{n + 1}{n} a_{n - 1} + 2
	\end{align*}
	Dies ist eine aus der Vorlesung, siehe dazu Satz 4.1 im Skriptum, bekannte Form mit $b_0 = 0, b_n = 2, n \geq 1, c_n = \frac{n+1}{n}$. Die Lösung ist auch aus der Vorlesung bekannt und hat die Form
	\begin{align*}
	a_n = \sum_{i=0}^nb_i\prod_{j=i+1}^nc_j
	= 2\sum_{i=1}^n\prod_{j=i+1}^n\frac{j+1}{j} = 2\sum_{i=1}^n\frac{n+1}{i+1}
	= 2(n+1)\sum_{i=2}^{n+1}\frac{1}{i}
	\end{align*}
	Eine Lösung der Rekursionsgleichung ist. Mit Induktion nach $n$ erhalten wir für festes $i$
	\begin{align*}
	\prod_{j=i+1}^i = 1 = \frac{i + 1}{i + 1}, \quad \prod_{j = i+1}^{n} \frac{j + 1}{j} =
	\frac{n + 1}{n} \frac{n}{i + 1} = \frac{n + 1}{i + 1}
	\end{align*}
	Machen wir noch die Probe.
	\begin{align*}
	a_0 &= 0, \\
	2n(n + 1) \sum_{i = 2}^{n + 1} \frac{1}{i} &= n a_n \stackrel{!}{=} (n + 1) a_{n - 1} + 2n = (n + 1) 2n \sum_{i = 2}^{n} \frac{1}{i} + 2n \\
	&= 2n (n + 1)  \sum_{i = 2}^{n} \frac{1}{i} + 2n (n + 1) \frac{1}{n + 1} = \quad 2n(n + 1) \sum_{i = 2}^{n + 1} \frac{1}{i}
	\end{align*}
\end{solution}
