\begin{exercise}
\phantom{}
  \begin{enumerate}[label = \alph*)]
  \item Lösen Sie die Rekursion $T(1) = 2$ und $T(n) = T(n-1) + 2$ für $n \geq 2$, indem Sie widerholt in die Rekursion einsetzen, bis Sie $T(n)$ erkennen. Verifizieren Sie ihr Ergebnis anschließend mit der Substitutionsmethode.

  \item Lösen Sie die Rekursion $T(1) = 2$ und $T(n) = 3T(n-1) + 2$ für $n \geq 2$, indem Sie widerholt in die Rekursion einsetzen, bis Sie $T(n)$ erkennen. Verifizieren Sie ihr Ergebnis anschließend mit der Substitutionsmethode.
  \end{enumerate}
\end{exercise}

% -------------------------------------------------------------------------------- %

\begin{solution}
\phantom{}
\begin{enumerate}[label = \alph*)]
  \item Setzen wir also zuerst ein paar mal ein:

  \begin{align*}
    T(1) = 2 \\
    T(2) = 4 \\
    T(3) = 6 \\
  \end{align*}
 Die Vermutung ist, dass $T(n) = 2n$. Mit Induktion lässt sich dies leicht bestätigen: \\
  IA$(n=2)$:

  \begin{align*}
    T(2) = 4 = 2 \cdot 2
  \end{align*}
  IV: $T(n) = 2n$ \\

  IS:$n \mapsto n+1$
  \begin{align*}
    T(n+1)
    =
    T(n) + 2
    \stackrel{IV}{=}
    2n + 2
    =
    2(n+1)
  \end{align*}

  \item Einsetzen in die Rekursion liefert

  \begin{align*}
    T(1) = 2 \\
    T(2) = 8 \\
    T(3) = 26 \\
    T(4) = 80
  \end{align*}

  Die Vermutung ist, dass $T(n) = 3^n - 1$. Wiederum zeigen wir das mit Induktion: \\
  IA$(n=2)$:

  \begin{align*}
    T(2) = 8 = 3^2 - 1
  \end{align*}
  IV: $T(n) = 3^n - 1$ \\

  IS:$n \mapsto n+1$
  \begin{align*}
    T(n+1)
    =
    3T(n) + 2
    \stackrel{IV}{=}
    3(3^n - 1) + 2
    =
    3^{n+1} - 1
  \end{align*}
  \end{enumerate}

\end{solution}

% -------------------------------------------------------------------------------- %
