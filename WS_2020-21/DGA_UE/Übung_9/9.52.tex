% --------------------------------------------------------------------------------

\begin{exercise}

Zur Analyse der erwarteten Laufzeit von Quickselect: Bezeichne $T(n)$ die Laufzeit
von Quickselect bei einem Feld der Größe $n$ und sei $X_k$ das Ereignis, dass das
Pivotelement an $k$-ter Stelle steht.
\begin{enumerate}[label = \alph*)]
  \item Weisen Sie nach, dass $\E(X_k) = \frac{1}{n}$ für $k = 1,\dots,n$ und dass
  für $T(n)$ gilt, dass
  \begin{align*}
    T(n) \leq \sum_{k=1}^n X_k \cdot T(\max(k-1,n-k)) + \Landau(n).
  \end{align*}
  \item Da $X_k$ und $T(\max(k-1,n-k))$ unabhängig sind, lässt sich also folgern, dass
  \begin{align*}
    \E(T(n)) \leq \sum_{k=1}^n \frac{1}{n} \cdot \E(T(\max(k-1,n-k))) + \Landau(n).
  \end{align*}
  Zeigen Sie (mittels Betrachtung von $\max(k-1,n-k)$), dass
  \begin{align*}
    \E(T(n)) \leq \frac{2}{n}\sum_{k=\ceilbraces{n/2}}^n \E(T(k)) + \Landau(n).
  \end{align*}
  \item Folgern Sie daraus nun mittels Substitutionsmethode, dass $\E(T(n)) = \Landau(n)$
  (also $\E(T(n)) \leq cn$ für passendes $c > 0$).
\end{enumerate}

\end{exercise}

% --------------------------------------------------------------------------------


\begin{solution}

\phantom{}

\end{solution}
