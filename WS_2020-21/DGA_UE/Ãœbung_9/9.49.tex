% --------------------------------------------------------------------------------

\begin{exercise}

\phantom{}

\begin{enumerate}[label = \alph*)]
  \item Erzeugt die folgende Modifikation des Fisher-Yates-Algorithmus ebenfalls
  eine zufällige Permutation von $A$? Weshalb oder weshalb nicht?

  \begin{algorithmic}
      \State PERMUTE-WITH-ALL($A[1,\dots,n]$)
      \State $n = A.$length
      \For{$i = 1 ~\text{to}~ n$}
          \State Vertausche $A[i]$ mit $A[RANDOM(1,n)]$
      \EndFor
  \end{algorithmic}

  \item Eine weitere Abwandlung des Fisher-Yates-Algorithmus: Diesmal wird das
  Element $A[i]$ in jedem Schritt mit einem zufälligen Element aus dem Teilfeld
  $A[i+1,\dots,n]$ vertauscht, also

  \begin{algorithmic}
    \State PERMUTE-WITHOUT-IDENTITY($A[1,\dots,n]$)
    \State $n = A.$length
    \For{$i = 1 \text{ to } n$}
      \State Vertausche $A[i]$ mit $A[RANDOM(i+1,n)]$
    \EndFor
  \end{algorithmic}

  Man könnte zunächst meinen, dass dieser Algorithmus alle von der Identität
  verschiedenen Permutationen zufällig erzeugt. Weisen Sie nach, dass dies aber
  nicht der Fall ist. Überlegen Sie sich, welche Klasse von Permutationen von
  diesem Algorithmus tatsächlich zufällig erzeugt werden.
\end{enumerate}

\end{exercise}

% --------------------------------------------------------------------------------


\begin{solution}

\phantom{}

\begin{enumerate}[label = \alph*)]
  \item Ja, warum denn nicht? Im zweiten Beispiel kommt der falsche Algorithmus,
  also ist das hier der Richtige.
  \item
\end{enumerate}

\end{solution}
