% --------------------------------------------------------------------------------

\begin{exercise}

Quickselect ist ein Algorithmus zum Auffinden des $j$-kleinsten Elements eines
Feldes, der auf einer ähnlichen Idee wie Quicksort basiert: Wähle ein zufälliges
Pivotelement und bringe es wie bei Quicksort an die richtige Stelle, sodass im
Teilfeld links davon nur kleinere Elemente und im Teilfeld rechts davon nur
größere Elemente sind. Ist das Pivotelement genau an
Stelle $j$, so sind wir fertig. Hat das linke Teilfeld mehr als $j$ Elemente,
so suche dort weiter, ansonsten im rechten Teilfeld (anstatt wie bei Quicksort beide
Seiten rekursiv zu betrachten, macht man dies jetzt nur mit der Seite, wo das Element ist). \\
Schreiben Sie einen Pseudocode für den Algorithmus Quickselect.

\end{exercise}

% --------------------------------------------------------------------------------


\begin{solution}

\phantom{}

\begin{algorithmic}[1]
  \State \textsc{Quickselect}($A,j,l,r$)
  \If{$l < r$}
    \State $m := \textsc{Teilen}(A,l,r)$
    \If{$m > j$}
      \State \textsc{Quickselect}($A,j,l,m-1$)
    \ElsIf{$m < j$}
      \State \textsc{Quickselect}($A,j,m+1,r$)
    \ElsIf{$m = j$}
      \State \Return $m$
    \EndIf
  \EndIf
\end{algorithmic}

\begin{algorithmic}[1]
  \State \textsc{Teilen}($A,l,r$)
  \State $x := A[r]$
  \State $i := l$
  \For{$j = l,\dots,r-1$}
    \If{$A[j] \leq x$}
      \State Vertausche $A[i]$ mit $A[j]$
      \State $i := i + 1$
    \EndIf
  \EndFor
  \State Vertausche $A[i]$ mit $A[r]$
  \State \Return $i$
\end{algorithmic}

\end{solution}
