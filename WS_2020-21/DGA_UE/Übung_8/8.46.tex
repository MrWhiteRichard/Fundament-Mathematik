% --------------------------------------------------------------------------------

\begin{exercise}

Das (diskrete) Rucksack-Problem. Sie haben schon wieder gewonnen! Diesmal dürfen
Sie sich unter $n$ Gegenständen (die Sie aber nicht zerteilen dürfen), wobei der
$i$-te Gegenstand $v_i$ Euro wert ist und $w_i$ Kilo wiegt, so viele aussuchen und
in Ihrem Rucksack hineinpacken, so viel Sie tragen können. Wir nehmen dabei weiters
immer an, dass $v_i,w_i$ und $W$ positive ganze Zahlen sind. Die Aufgabe beim
\glqq Rucksack-Problem \grqq besteht nun darin, anzugeben, welche Gegenstände
Sie mitnehmen sollen, sodass der Gesamtwert der eingepackten Gegenstände möglichst groß ist. \\
Etwas genauer: Gesucht ist $f(n,W)$, wenn
\begin{align*}
  f(m, W^{\prime}) = \max\left\{\sum_{i=1}^m x_iv_i \mid \sum_{i=1}^m x_iw_i \leq
  W^{\prime} \text{und} x_i \in \{0,1\}\right\},
\end{align*}
für $m \in \{1,2,\dots,n\}$ und $W^{\prime} \in \{0,\dots,W\}$.
\begin{enumerate}[label = \alph*)]
  \item Überlegen Sie sich, dass das Rucksack-Problem die optimale Teilstruktur-Eigenschaft
  (wie lautet diese hier?) besitzt.
  \item Geben Sie eine Rekursion für $f(m,W^{\prime})$, also für den Wert von
  optimalen Teillösungen des ursprünglichen Problems an. \\
  \textit{Anmerkung:} Diese Rekursion hängt nun anders als beim vorigen Beispiel von beiden Parametern $m$ und $W^{\prime}$
  ab.
  \item Man gebe einen Algorithmus an, der unter Zuhilfenahme von dynamischer Programmierung
  das Rucksack-Problem löst.
\end{enumerate}
\end{exercise}

% --------------------------------------------------------------------------------


\begin{solution}

\phantom{}

\begin{enumerate}[label = \alph*)]
  \item Die optimale Teilstruktur-Eigenschaft lautet hier, dass für eine optimale Lösung $\sum_{i=1}^n x_iv_i$ zum Gewicht $W$ auch (für beliebiges $m \in \{1,\dots,n\}$) die Lösung $\sum_{i=1}^m x_iv_i$  eine optimale Lösung zum Gewicht $W^{\prime} := W - \sum_{i=m+1}^n x_iw_i$ und zum Auswählen aus den ersten $m$ Gegenständen ist - also der Wert von $f(m, W^{\prime})$. (Zum Gewicht $\sum_{i=m+1}^n x_iw_i$ und zum Auswählen aus den Gegenständen $m+1, \dots, n$ ist dann auch $\sum_{i=m+1}^n x_iv_i$ die optimale Lösung.) Beweis wieder recht leicht durch Widerspruch.
  \item Wir verwenden die oben gezeigt Teilstruktur-Eigenschaft für $m = n-1$.
  Dann folgt $f(\cdot,0) = 0$ und

  \begin{align*}
    f(n,W) = \begin{cases}
      f(n-1,W) & w_n > W \\
      \max\{f(n-1,W), v_n + f(n-1, W-w_n)\} & w_n \leq W.
    \end{cases}
  \end{align*}

  (Natürlich gilt nach Definition $f(0,\cdot) = 0$.)

  \item
  \begin{algorithm}
      \caption{Lösung des Rucksack-Problems}
      \begin{algorithmic}[1]
          \Procedure{$f$}{$n,W$}
              Sei $f[0,\dots,n][1,\dots,W]$ ein neues Datenfeld mit $0$-Einträgen
              \For{$i = 1,\dots, n$}
                  \For{$j = 1,\dots,W$}
                      \If{$w_i \leq j$}
                          \State $f[i,j] = \max\{v_i + f[i-1,j-w_i], f[i-1,j]\}$
                      \Else
                          \State $f[i,j] = f[i-1,j]$
                      \EndIf
                  \EndFor
              \EndFor
              \State \Return $f[n,W]$
          \EndProcedure
      \end{algorithmic}
  \end{algorithm}
\end{enumerate}

\end{solution}
