% --------------------------------------------------------------------------------

\begin{exercise}

Gegeben ist die Rekursion

\begin{align*}
  a_n = a_{f(n)} + a_{g(n)} + a_{h(n)} + 1
\end{align*}

für $n > t$ und $a_n = 1$ für $n \leq t$ mit $f,g,h: \N \rightarrow \N^+$ und $f(n) + g(n) + h(n) = n$. Beweisen Sie, dass $a_n = \Theta(n)$ gilt.
\end{exercise}

% --------------------------------------------------------------------------------

\begin{solution}
Wir bemerken zu Beginn, dass $t \geq 2$, sonst haben wir undefinierte Werte. Als Induktionsanfang erhalten wir
\begin{align*}
\forall n \in \{1, \dots, \lfloor t \rfloor\} \ \pbraces{a_n = 1 \leq 2n - 1}
\end{align*}
Nun zeigen wir noch den Induktionsschritt um schließlich $a_n \leq 2n - 1$ zu erhalten:
\begin{align*}
  a_n = a_{f(n)} + a_{g(n)} + a_{h(n)} + 1 \leq 2(f(n) + g(n) + h(n)) - 3 + 1 \leq 2n - 1.
\end{align*}
Untere Schranke: $a_n \geq \frac{n}{t}$:
\begin{align*}
  a_n = a_{f(n)} + a_{g(n)} + a_{h(n)} + 1 \geq \frac{f(n) + g(n) + h(n)}{t} + 1 = \frac{n}{t} + 1 \geq \frac{n}{t}.
\end{align*}
Bei der Abschätzung nach unten wollen wir für den Induktionsanfang noch
\begin{align*}
\forall m \in \{1, \dots, \lfloor t \rfloor \} \ \pbraces{a_m = 1 \geq \frac{m}{t}}
\end{align*}
was für $m \leq t$ natürlich erreicht ist. Damit haben wir insgesamt $a_n = \Theta(n)$ gezeigt.
\end{solution}

% --------------------------------------------------------------------------------
