% -------------------------------------------------------------------------------- %

\begin{exercise}

Lösen Sie die Rekursion $T(1) = 1$ und $T(n) = 3T(\frac{n}{2}) + n^2 + n$ für $n = 2^k \geq 2$, indem Sie wiederholt in die Rekursion einsetzen, bis Sie $T(n)$ erkennen. Verifizieren Sie ihr Ergebnis anschließend mit der Substitutionsmethode.

\end{exercise}

% -------------------------------------------------------------------------------- %

\begin{solution}

Setzen wir also zuerst ein paar mal ein (und addieren dabei nicht direkt damit man das Schema erkennen kann):

\begin{align*}
  T(2^0) &= 1 \\
  T(2^1) &= 3 + 4 + 2 = 2 + 3 + 2^2 \\
  T(2^2) &= 3\cdot2 + 3^2 +3\cdot2^2 + 2^4 + 2^2 = 2^2 + 2\cdot3 + 3^2+ 2^4+ 3\cdot2^2\\
  T(2^3) &= 3^2\cdot2 + 3^3 +2^2\cdot3^2 +3\cdot2^4 +3\cdot2^2 + 2^6 +2^3 = 2^3 + 2^2\cdot3 + 2\cdot3^2 + 3^3 + 2^4+ 2^6 +3\cdot2^4 + 3^2\cdot2^2
\end{align*}

Wir stellen die Vermutung

\begin{align*}
  T(2^k) = \sum_{i=0}^k 2^{k-i}3^i + \sum_{i=0}^{k-1} 2^{2(k-i)} 3^i
\end{align*}
auf. Um diese zu Verifizieren verwenden wir Induktion (über $k$). \\
IA$(k=0)$:

\begin{align*}
  T(1) = 1 = 2^0 \cdot 3^0
\end{align*}
IV: $T(2^k) = \sum_{i=0}^k 2^{k-i}3^i + \sum_{i=0}^{k-1} 2^{2(k-i)} 3^i$ \\

IS:~$k \mapsto k+1$
\begin{align*}
  T(2^{k+1})
  &=
  3T(2^k) + 2^{2(k+1)} + 2^{k+1}
  \stackrel{IV}{=}
  3(\sum_{i=0}^k 2^{k-i}3^i + \sum_{i=0}^{k-1} 2^{2(k-i)} 3^i) + 2^{2(k+1)} + 2^{k+1} \\
  &=
  \sum_{i=0}^k 2^{k-i}3^{i+1} + \sum_{i=0}^{k-1} 2^{2(k-i)} 3^{i+1} + 2^{2(k+1)} + 2^{k+1}
  =
  \sum_{i=1}^{k+1} 2^{k-(i-1)}3^{i} + \sum_{i=1}^{k} 2^{2[k-(i-1)]} 3^{i} + 2^{2(k+1)} + 2^{k+1} \\
  &=
  \sum_{i=0}^{k+1} 2^{k+1-i}3^{i} + \sum_{i=0}^{k} 2^{2[(k+1)-i]} 3^{i}
\end{align*}

\end{solution}

% -------------------------------------------------------------------------------- %
