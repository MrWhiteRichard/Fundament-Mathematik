% --------------------------------------------------------------------------------

\begin{exercise}

Verwenden Sie einen Rekursionsbaum, um eine asymptotische obere Schranke für die Rekursion

\begin{align*}
  T(n) = T(n-a) + T(a) + n \quad \text{für}~ n > a,\quad T(n) = 0 \quad \text{für}~ n < a,
\end{align*}

zu bestimmen ($a \geq 1$). Überprüfen Sie Ihre Schranke mittels Substitutionsmethode.
\end{exercise}

% --------------------------------------------------------------------------------

\begin{solution}
\begin{align*}
  \Tree [.na [.(n-1)a [ a a ].(...) a ].(n-1)a [.a ].a ]
\end{align*}

Aus ebenjenem Baum liest man leicht unsere Vermutung für die asymptotische obere Schranke ab. \\
Wir zeigen mittels vollständiger Induktion: $T(na) \leq n^2 + nT(a)$.
\begin{align*}
  n = 1&: \quad T(a) \leq 1^2 + 1T(a) \\
  n \rightsquigarrow n+1&: \quad T((n+1)a) = T(na) + T(a) + n \leq n^2 + nT(a) + T(a) + n
  \leq (n+1)^2 + (n+1)T(a).
\end{align*}
Daher erhalten wir $T(na) = \Landau(n^2)$
\end{solution}

\begin{solution}

	\begin{align*}
	\Tree [.n [.n-a [ n-3a T(a) ].n-2a T(a) ].n-a [.T(a) ].T(a) ]
	\end{align*}

	Der Baum hat eine Tiefe von $\frac{n}{a}$ und auf jeder Höhe hat er zwei Blätter also insgesamt $2\frac{n}{a}$ Blätter. Außerdem ist der Aufwand in Tiefe $k$ gegeben durch $n - ka + T(a)$. Wir berechenen
	\begin{align*}
	\sum_{k=0}^{\frac{n}{a} - 1} (n - ka + T(a)) + 2 \frac{n}{a} = a \sum_{k = 1}^{\frac{n}{a} - 1} k + \frac{n}{a} (T(a) + 2)
	\end{align*}
	Also ist die Vermutung $T(n) = \Landau(n^2)$. Wir wollen also zeigen
	\begin{align*}
	\exists c >0 \ \exists n_0 \in \N \ \forall n \geq n_0 \ T(n) \leq c n^2
	\end{align*}
	Für alle $n < a$ gilt dies können wir ein beliebiges $c > 0$ aussuchen, da $T(n) = 0$. Außerdem wollen wir $c$ hinreichend groß, dass $T(a) \leq c a^2$ gilt. Unter der Annahme, dass für $n > a$ alle $k < n$ gilt, dass $T(k) \leq c k^2$ wünschen wir uns
	\begin{align*}
	c n^2 \stackrel{!}{\geq} T(n) = T(n - a) + T(a) + n \leq c(n - a)^2 + ca^2 + n = cn^2 - 2cna + 2ca^2 + n \\ 
	\Leftarrow 2ca(n - a) \geq n \Leftarrow c \geq \frac{n}{2a(n - a)}
	\end{align*}
	Zusammengefasst wollen wir
	\begin{enumerate}
		\item $n_0 = 0$
		\item $c \geq \frac{T(a)}{a^2}$
		\item $\forall n > a \ (c \geq \frac{n}{2a(n - a)})$
	\end{enumerate}
\end{solution}

% --------------------------------------------------------------------------------

\begin{solution}
\begin{align*}
\Tree [.n [.n-a [ n-3a T(a) ].n-2a T(a) ].n-a [.T(a) ].T(a) ]
\end{align*}

Der Baum hat eine Tiefe von $\frac{n}{a}$ und auf jeder Höhe hat er zwei Blätter also insgesamt $2\frac{n}{a}$ Blätter. Außerdem ist der Aufwand in Tiefe $k$ gegeben durch $n - ka + T(a)$. Wir berechenen

\begin{align*}
  \sum_{k=0}^{\frac{n}{a} - 1} (n - ka + T(a)) + 2 \frac{n}{a}
  =
  \frac{n^2}{a} - a\sum_{k = 0}^{\frac{n}{a} - 1} k + \frac{n}{a} (T(a) + 2)
  \leq
  \frac{n^2}{a} + \frac{n}{a}(T(a) + 2)
\end{align*}

Also ist die Vermutung $T(n) = \Landau(n^2)$. Wir wollen also zeigen

\begin{align*}
\exists c >0 \ \exists n_0 \in \N \ \forall n \geq n_0 :~ T(n) \leq c n^2 + c a^2 + T(a)
\end{align*}

Für unseren Induktionsanfang sei $n_0 = 0 \implies T(n_0) = 0 \leq c a^2 + T(a)$. Nehmen wir nun also an, dass die Ungleichung für alle $k < n$ gilt. Dann gilt auch

\begin{align*}
  T(n)
  =
  T(n-a) + T(a) + n
  \leq
  c (n - a)^2 + c a^2 + T(a) + n
  =
  cn^2 - 2cna + T(a) + n
  \leq
  c n^2 + c a^2 + T(a)
\end{align*}

Wobei wir hier $\forall n \geq n_0 :~ 2cna \geq n$ als Bedingung an unser $c$ stellen.

Damit haben wir also gezeigt $T(n) = \Landau(c n^2 +c a^2 + T(a)) = \Landau(n^2)$.
\end{solution}

% --------------------------------------------------------------------------------
