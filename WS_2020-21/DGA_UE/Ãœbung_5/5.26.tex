% --------------------------------------------------------------------------------

\begin{exercise}

Verwenden Sie einen Rekursionsbaum, um eine asymptotische obere Schranke für die Rekursion

\begin{align*}
  T(n) = T(n-a) + T(a) + n \quad \text{für}~ n > a,\quad T(n) = 0 \quad \text{für}~ n < a,
\end{align*}

zu bestimmen ($a \geq 1$). Überprüfen Sie Ihre Schranke mittels Substitutionsmethode.
\end{exercise}

% --------------------------------------------------------------------------------

\begin{solution}

Hier stelle man sich einen schönen Rekursionsbaum vor. \\
Aus ebenjenem Baum liest man leicht unsere Vermutung für die asymptotische obere Schranke ab. \\
Wir zeigen mittels vollständiger Induktion: $T(na) \leq n^2 + nT(a)$.
\begin{align*}
  n = 1&: \quad T(a) \leq 1^2 + 1T(a) \\
  n \rightsquigarrow n+1&: \quad T((n+1)a) = T(na) + T(a) + n \leq n^2 + nT(a) + T(a) + n
  \leq (n+1)^2 + (n+1)T(a).
\end{align*}
Daher erhalten wir $T(na) = \Landau(n^2)$
\end{solution}

% --------------------------------------------------------------------------------
