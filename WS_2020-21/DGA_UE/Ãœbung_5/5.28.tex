% --------------------------------------------------------------------------------

\begin{exercise}

Finden Sie für folgende Rekursionen $T(n)$ asymptotische untere und obere Schranken mittels Master-Theorem.

\begin{enumerate}[label = \alph*)]
  \item $T(n) = 2T(\frac{n}{3}) + n \log n$
  \item $T(n) = T(\frac{8n}{9}) + n$
  \item $T(n) = 11T(\frac{n}{3}) + n^{1.5}$
  \item $T(n) = 4T(\frac{n}{2}) + n$
  \item $T(n) = 4T(\frac{n}{2}) + n^2$
  \item $T(n) = 4T(\frac{n}{2}) + n^3$
\end{enumerate}

\end{exercise}

% --------------------------------------------------------------------------------

\begin{solution}
\phantom{}
\includegraphicsboxed{DGA/DGA - Satz 4.4 (Master-Theorem für Teile-und-herrsche-Rekursionsgleichungen).png}

\begin{enumerate}[label = \alph*)]
  \item Es ist $a = 2, b = 3$ und wir sind im Fall $3.$ (mit $c := \frac{2}{3}$), denn

    \begin{align*}
      \forall n \in \N: 2f\left(\frac{n}{3}\right)
      =
      \frac{2}{3}n \log\left(\frac{n}{3}\right)
      =
      \frac{2}{3}n \left( \log n - \log 3\right)
      \leq
      \frac{2}{3} n\log n
    \end{align*}
  Also gilt $T(n) = \Theta(n\log n)$.

  \item Es ist $a = 1, b = \frac{9}{8}$ und wir sind im Fall $3.$ (mit $c := \frac{8}{9}$), denn
    \begin{align*}
      f\left(\frac{8n}{9}\right)
      =
      \frac{8}{9}n
    \end{align*}
  Also gilt $T(n) = \Theta(n)$.

  \item Es ist $a = 11, b = 3$ und wir sind im Fall $1.$ (mit $\varepsilon := \log_3(11)-1.5$), denn

    \begin{align*}
      \log_3(11) \approx 2.182 > 1.5 \implies \varepsilon > 0 \\
      f(n)
      = n^{3/2} =
      \Landau(n^{\log_3(11)-\varepsilon})
    \end{align*}
  Also gilt $T(n) = \Theta(n^{\log_3(11)})$.

  \item Es ist $a = 4, b = 2$ und wir sind im Fall $1.$ (mit $\varepsilon := 1$), denn

    \begin{align*}
      f(n)
      = n =
      \Landau(n^{\log_2(4)-1})
    \end{align*}
  Also gilt $T(n) = \Theta(n^2)$.

  \item Es ist $a = 4, b = 2$ und wir sind im Fall $2.$, denn
    \begin{align*}
      f(n) = n^2 = \Theta(n^2)
    \end{align*}
  Also gilt $T(n) = \Theta(n^2 \log n)$.

  \item Es ist $a = 4, b = 2$ und wir sind im Fall $3.$ (mit $c := \frac{1}{2}$), denn

    \begin{align*}
      4f\left(\frac{n}{2}\right)
      =
      4 \left(\frac{n}{2}\right)^3
      =
      \frac{1}{2} n^3
    \end{align*}
  Also gilt $T(n) = \Theta(n^3)$
\end{enumerate}

\end{solution}

% --------------------------------------------------------------------------------
