% -------------------------------------------------------------------------------- %

\begin{exercise}

\phantom{}
Nehmen Sie an, Sie wollen die Ausgaben $0$ und $1$ mit Wahrscheinlichkeit je $\frac{1}{2}$ erhalten.
Dazu steht Ihnen die Prozedur Biased-Random zur Verfügung, die den Wert $1$ mit Wahrscheinlichkeit $p$ und den Wert $0$ mit Wahrscheinlichkeit $1 - p$ ausgibt ($0 < p < 1$).
Sie wissen aber nicht, wie groß $p$ ist.
Geben Sie einen Algorithmus (in Pseudocode) an, der Biased-Random als Unterroutine verwendet und den Wert $0$ mit Wahrscheinlichkeit $\frac{1}{2}$ und den Wert $1$ ebenfalls mit Wahrscheinlichkeit $\frac{1}{2}$ zurückgibt.
Wie groß ist die erwartete Laufzeit ihres Algorithmus als Funktion von $p$?

\end{exercise}

% -------------------------------------------------------------------------------- %


\begin{solution}

Seien $X, Y$ unabhängige Instanzen von \textsc{Biased-Random}().

\begin{align*}
    \implies
    \P(X = 0, Y = 0) & = \P(X = 0) \P(Y = 0) = p^2, \\
    \P(X = 0, Y = 1) & = \P(X = 0) \P(Y = 1) = p (1 - p) = \\
    \P(X = 1, Y = 0) & = \P(X = 1) \P(Y = 0) = (1 - p) p, \\
    \P(X = 1, Y = 1) & = \P(X = 1) \P(Y = 1) = (1 - p)^2
\end{align*}

Die folgende Prozedur liefert also, mit $X$ bzw. $Y$, in einer Hälfte der Fälle $0$ und in der anderen $1$.

\phantom{}

\begin{algorithmic}
    \Procedure{Random}{}
    \State $X := Y := 0$
    \While{$X = Y$}
        \State $X$ = \textsc{Biased-Random}()
        \State $Y$ = \textsc{Biased-Random}()
    \EndWhile
    \State \Return $X$ (oder $Y$)
    \EndProcedure
\end{algorithmic}

\phantom{}

\begin{align*}
    \P(\text{continue})
    & :=
    \P(X = Y) \\
    & =
    \P((X = 0 \land Y = 0) \lor (X = 1 \land Y = 1)) \\
    & =
    \P(X = 0) \P(Y = 0) + \P(X = 1) \P(Y = 1) \\
    & =
    p^2 + (1 - p)^2 \\
    & =
    p^2 + 1 - 2 p + p^2 \\
    & =
    1 - 2 p + 2 p^2 \\
    & =
    1 - 2 p (1 - p)
\end{align*}

\begin{align*}
    \P(\text{break})
    :=
    1 - \P(\text{continue})
    =
    2 p (1 - p)
    =:
    q \in (0, 1)
\end{align*}

\begin{align*}
    \P(\text{Durchläufe} = n)
    :=
    (1 - q)^{n-1} q
\end{align*}

\begin{align} \tag{$\ast$}
    \Forall x \in (-1, 1):
    \sum_{n \in \N}
    n x^{n-1}
    =
    \sum_{n \in \N}
        \derivative{x}
            x^n
    =
    \derivative{x}
        \sum_{n \in \N}
            x^n
    =
    \derivative{x}
        \frac{1}{1 - x}
    =
    \frac{1}{(1 - x)^2}
\end{align}

\begin{align*}
        \E(\text{Durchläufe})
        :=
        \sum_{n \in \N}
            n \P(\text{Durchläufe} = n)
        =
        q
        \sum_{n \in \N}
            n (1 - q)^{n-1}
        \stackrel
        {
            (\ast)
        }{=}
        \frac{q}{(1 - (1 - q))^2}
        =
        \frac{q}{q^2}
        =
        \frac{1}{2 p (1 - p)}
\end{align*}

\end{solution}
