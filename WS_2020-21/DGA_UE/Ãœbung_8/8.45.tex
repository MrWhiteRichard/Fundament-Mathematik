% --------------------------------------------------------------------------------

\begin{exercise}

Wir betrachten \glqq Wechsel-Geld-Problem \grqq (WGK).
In einer Währung existieren (unlimitiert viele) Münzen im Wert von $1 = d_1 < d_2 < \cdots  d_k$
Cent. Ziel des WGK ist es, mit möglichst wenigen Münzen einen gegebenen Betrag von $n$ Cent zu wechseln.
\begin{enumerate}[label = \alph*)]
  \item Bestätigen Sie, dass das WGK die optimale Teilstruktur-Eigenschaft erfüllt,
  also zeigen Sie, dass eine optimale Lösung für $n$ Cent, also $n = \sum_{i} c_i d_i$
  und $\sum_i c_i$ ist minimal, auch eine optimale Lösung für $b$ und $n - b$ darstellt,
  wenn $b$ ein Anteil von $n$ ist, welcher durch die in der Darstellung von $n$
  verwendeten Münzen zustande kommt, also $b = \sum_i c^{\prime}_i d_i$ mit $0 \leq c^{\prime}_i \leq c_i$.
  \item Sei $A(n)$ die Anzahl der Münzen, die zum Wechseln des Betrags $n$ mindestens notwendig sind.
  Überlegen Sie sich eine Rekursion für $A(n)$.
  \item Erklären Sie die Funktionsweise des folgenden Algorithmus \textsc{Wechsel}($d,k,n$),
  welcher für den Wert $n$ und für ein gegebenes Münzen-Array $d = (d_1,\dots,d_k)$
  die optimale Wechsellösung liefert am Beispiel $(d_1,d_2,d_3) = (1,2,5)$ und $n = 8$.
  \begin{algorithm}
      \begin{algorithmic}[1]
          \Procedure{Wechsel}{$d,k,n$}
          \State Seien $C[0,\dots,n]$ und $S[0,\dots,n]$ neue Felder
          \State $C[0] := 0$
          \For{$j = 1$ to $n$}
              \State $C[j] := \infty$
              \For{$i = 1$ to $k$}
                  \If{ $d_i \leq j$ and $1 + C[j - d_i] < C[j]$}
                      \State $C[j] := 1 + C[j - d_i]$
                      \State $S[j] := d_i$
                  \EndIf
              \EndFor
          \EndFor
          \EndProcedure
      \end{algorithmic}
  \end{algorithm}

\item Wie groß ist asymptotisch die Komplexität von \textsc{Wechsel}($d,k,n$)?
\end{enumerate}


\end{exercise}

% --------------------------------------------------------------------------------


\begin{solution}

\phantom{}

\begin{enumerate}[label = \alph*)]

\item Sei $c = (c_1,\dots,c_k)$ die optimale Münzenanzahl für den Wert $n$.
und $c^{\prime}\cdot d = b$ mit $0 \leq c^{\prime}_i \leq c_i$ für $i=1,\dots,k$.
Angenommen, es gäbe $\overline{c}$ mit $\overline{c}\cdot d = b$ und
$\sum_{i=1}^k \overline{c}_i < \sum_{i=1}^k c^{\prime}_i$. \\
Dann können wir $n$ auch in folgender Münzenkombination wechseln:
\begin{align*}
  (\overline{c} + c - c^{\prime})d = b + n - b = n
\end{align*}
mit $\sum_{i=1}^k c_i + (\overline{c}_i - c^{\prime}_i) < \sum_{i=1}^k c_i$. Widerspruch!

\item
\begin{align*}
  A(0) &= 0 \\
  A(n) &= \min\{ 1 + A(n - d_i): i = 1,\dots,k \land d_i \leq n\}
\end{align*}

\item Zuerst betrachten wir die optimale Lösung für den Wechsel $j = 1$er Münze und erhalten 
\begin{align*}
	C = [0, 1] \\
	S = [\ast, d_1 = 1],
\end{align*}
wobei $\ast$ für einen beliebigen Wert steht. Danach basierend auf dieser Lösung die optimale Lösung für $j = 2$ Münzen. Wir erhalten
\begin{align*}
	C = [0,1,1]\\
	S = [\ast, 1, d_2 = 2].
\end{align*}
Beim nächsten Schritt für $j = 3$ Münzen wird es interessanter. Hier haben wir prinipiell zwei mögliche vorgangsweisen. Wir könnten $S[3] = d_1$ setzen und als Fortsetzung zu $j = 2$ betrachten oder aber wir setzen $S[3] = d_2$ und betrachten es als Fortsetzung von $j = 1$. In beiden Fällen erreichen wir die gleiche Aufteilung $d_1 + d_2 = 1 + 2 = 3$. Für welchen Wert $S[3]$ entscheidet sich unser Algorithmus? Er nimmt jenen mit kleinerem Index, also 
\begin{align*}
	C = [0, 1, 1, 2]\\
	S = [\ast, 1, 2, d_1 = 1]
\end{align*}
So geht das weiter. Wie können wir die Münzen welche wir brauchen, herausfinden wenn wir die Ausgaben $C$ und $S$ vom Algorithmus bekommen?

\item Offensichtlich ist der asymptotische Aufwand $\Landau(nk)$. 
\end{enumerate}

\end{solution}
