% --------------------------------------------------------------------------------

\begin{exercise}

Das (diskrete) Rucksack-Problem. Sie haben schon wieder gewonnen! Diesmal dürfen
Sie sich unter $n$ Gegenständen (die Sie aber nicht zerteilen dürfen), wobei der
$i$-te Gegenstand $v_i$ Euro wert ist und $w_i$ Kilo wiegt, so viele aussuchen und
in Ihrem Rucksack hineinpacken, so viel Sie tragen können. Wir nehmen dabei weiters
immer an, dass $v_i,w_i$ und $W$ positive ganze Zahlen sind. Die Aufgabe beim
\glqq Rucksack-Problem \grqq besteht nun darin, anzugeben, welche Gegenstände
Sie mitnehmen sollen, sodass der Gesamtwert der eingepackten Gegenstände möglichst groß ist. \\
Etwas genauer: Gesucht ist $f(n,W)$, wenn
\begin{align*}
  f(m, W^{\prime}) = \max\left\{\sum_{i=1}^m x_iv_i \mid \sum_{i=1}^m x_iw_i \leq
  W^{\prime} \text{und} x_i \in \{0,1\}\right\},
\end{align*}
für $m \in \{1,2,\dots,n\}$ und $W^{\prime} \in \{0,\dots,W\}$.
\begin{enumerate}[label = \alph*)]
  \item Überlegen Sie sich, dass das Rucksack-Problem die optimale Teilstruktur-Eigenschaft
  (wie lautet diese hier?) besitzt.
  \item Geben Sie eine Rekursion für $f(m,W^{\prime})$, also für den Wert von
  optimalen Teillösungen des ursprünglichen Problems an. \\
  \textit{Anmerkung:} Diese Rekursion hängt nun anders als beim vorigen Beispiel von beiden Parametern $m$ und $W^{\prime}$
  ab.
  \item Man gebe einen Algorithmus an, der unter Zuhilfenahme von dynamischer Programmierung
  das Rucksack-Problem löst.
\end{enumerate}
\end{exercise}

% --------------------------------------------------------------------------------


\begin{solution}

\phantom{}

\end{solution}
