% --------------------------------------------------------------------------------

\begin{exercise}

Es gibt zwei Arten von randomisierten Algorithmen:
Las-Vegas-Algorithmen nutzen randomisierten Input, um die erwartete Laufzeit zu verringern, produzieren aber immer ein korrektes Ergebnis (man gambelt mit der Laufzeit).
Monte-Carlo-Algorithmen hingegen haben deterministische Laufzeit, produzieren aber mit einer gewissen Wahrscheinlichkeit ein falsches oder gar kein Ergebnis (insofern ist die Bezeichung \enquote{Algorithmus} auch etwas fragwürdig).

Betrachten wir nun das Problem, in einem Array $A$ der Größe $2 n$ bestehend aus $n$ Einträgen mit Wert $a$ und $n$ Einträgen mit Wert $b$ ein $a$ zu finden, sowie einen Las-Vegas- und einen Monte-Carlo-Algorithmus, der dieses Problem löst:

\phantom{}

Las-Vegas-Algorithmus:

\begin{algorithmic}
    \State LV-FIND-a($A$)
    \While{true}
        \State $i = \text{RANDOM}(1, 2 n)$
        \If{$A[i] = a$}
            \State \Return $i$
        \EndIf
    \EndWhile
\end{algorithmic}

\phantom{}

Dieser Algorithmus liefert immer ein richtiges Ergebnis.
Bestimmen Sie seine erwartete Laufzeit.

\phantom{}

Monte-Carlo-Algorithmus:

\begin{algorithmic}
    \State MC-FIND-a($A, k$)
    \State $i = 0$
    \While{$i < k$}
        \State $i = \text{RANDOM}(1, 2 n)$
        \If{$A[i] = a$}
            \State \Return $i$
        \EndIf
    \EndWhile
\end{algorithmic}

\phantom{}

Dieser Algorithmus braucht höchstens $k$ Schritte ($k$ fest), hat also konstante Laufzeit.
Bestimmen Sie die Wahrscheinlichkeit, dass nach $k$ Durchläufen ein $a$ gefunden wurde.

\end{exercise}

\begin{solution}

\begin{align*}
    & \Forall i = 1, \dots, 2 n:
    \P(A[i] = a) = \P(A[i] = b) = \frac{1}{2} \\
    \implies
    & \P(A[\text{RANDOM}(1, 2 n)] = a) = \P(A[\text{RANDOM}(1, 2 n)] = b) = \frac{1}{2}
\end{align*}

\begin{enumerate}[label = \arabic*.]

    \item Algorithmus (Las-Vegas):
    
    \begin{multline*}
        \Forall k \in \N:
        \P(\text{Durchläufe} = k)
        =
        \P(A[\text{RANDOM}(1, 2 n)] = b)^{k-1}
        \P(A[\text{RANDOM}(1, 2 n)] = a) \\
        =
        \pbraces{\frac{1}{2}}^{k-1}
        \frac{1}{2}
    \end{multline*}
    
    \begin{align*}
        \E(\text{Durchläufe})
        =
        \sum_{k \in \N}
            k
            \P(\text{Durchläufe} = k)
        =
        \sum_{k \in \N}
            n
            \pbraces{\frac{1}{2}}^{k-1}
            \frac{1}{2}
        =
        \frac{1}{2}
        \sum_{k \in \N}
            n
            \pbraces{\frac{1}{2}}^{k-1}
        \stackrel
        {
            (\ast)
        }{=}
        \frac{1}{2}
        \frac{1}{\pbraces{1 - \frac{1}{2}}^2}
        =
        2
    \end{align*}

    \item Algorithmus (Monte-Carlo):

    \begin{align*}
        \P(a \in A[1:k])
        =
        1 - \P(a \not \in A[1:k])
        =
        1 - \P(A[1] = b, \dots, A[k] = b)
        =
        1 - \pbraces{\frac{1}{2}}^k
    \end{align*}

\end{enumerate}

\end{solution}
