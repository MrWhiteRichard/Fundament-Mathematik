% --------------------------------------------------------------------------------

\begin{exercise}

Die Fibonacci-Zahlen seien durch die Rekursion $F_n = F_{n-1} + F_{n-2}$ für $n \geq 2$ mit Anfangswerten $F_0 = 0$ und $F_1 = 1$ definiert.
Die Fibonacci-Zahlen können effizient mittels folgender auf Matrizenmultiplikation beruhender Formel berechnet werden:

\begin{align*}
  \begin{pmatrix}
    F_{n+1} & F_n \\
    F_n     & F_{n-1}
  \end{pmatrix}
  =
  \begin{pmatrix}
    1 & 1 \\
    1 & 0
  \end{pmatrix}^n
  \quad
  ~\text{für}~
  n \geq 1
\end{align*}

\begin{enumerate}[label = \alph*.]

  \item Beweisen Sie diese Formel durch vollständige Induktion.

  \item Überlegen Sie sich einen Algorithmus, der
  $\begin{pmatrix}
    1 & 1 \\
    1 & 0
  \end{pmatrix}^n$
  in nur logarithmisch vielen Schritten berechnet.

\end{enumerate}

\end{exercise}

% --------------------------------------------------------------------------------

\begin{solution}

Wir nennen die linke Matrix $L_n$ und die rechte $R_n$.

\begin{enumerate}[label = \alph*.]

  \item IA($n = 1$): $F_2 = F_1 + F_0 = 1$.

  \begin{align*}
  \begin{pmatrix}
    F_2 & F_1 \\
    F_1 & F_0
  \end{pmatrix}
  =
  \begin{pmatrix}
    1 & 1 \\
    1 & 0
  \end{pmatrix}^1
  \end{align*}

  IS($n \mapsto n + 1$):

  \begin{multline*}
    \implies
    R_{n+1}
    =
    R_n
    \begin{pmatrix}
      1 & 1 \\
      1 & 0
    \end{pmatrix}
    =
    L_n
    \begin{pmatrix}
      1 & 1 \\
      1 & 0
    \end{pmatrix}
    =
    \begin{pmatrix}
      F_{n+1} & F_n \\
      F_n     & F_{n-1}
    \end{pmatrix}
    \begin{pmatrix}
      1 & 1 \\
      1 & 0
    \end{pmatrix} \\
    =
    \begin{pmatrix}
      F_{n+1} + F_n & F_{n+1} \\
      F_n + F_{n-1} & F_n
    \end{pmatrix}
    =
    \begin{pmatrix}
      F_{n+2} & F_{n+1} \\
      F_{n+1} & F_n
    \end{pmatrix}
    =
    L_{n+1}
  \end{multline*}

  \item Stelle $n$ binär dar, also $n = \sum_{k= 0}^{\lfloor \log(n)\rfloor}a_k2^k$.
  Hat man den Koeffizientenvektor $A$ berechnet, führt folgender Algorithmus zum Ziel:

  \begin{flalign*}
  1&: \textbf{Prozedur}~ \textsc{Berechne} (A) & \\
  2&: \quad m := A.\textit{Länge} & \\
  3&: \quad F := [[1,0],[0,1]] & \\
  3&: \quad B := [[1,0],[0,1]] & \\
  4&: \quad \textbf{Für}~ i = 0,\dots,m & \\
  5&: \quad B = B*[[1,1],[1,0]] & \\
  4&: \quad \quad \textbf{Falls}~ A[i] = 1 & \\
  5&: \quad \quad \quad F = F*B & \\
  6&: \quad \textbf{Ende Für} & \\
  7&: \quad \textbf{Antworte~} F & \\
  8&: \textbf{Ende Prozedur}
  \end{flalign*}

\end{enumerate}

\end{solution}

% --------------------------------------------------------------------------------
