% --------------------------------------------------------------------------------

\begin{exercise}

\begin{enumerate}[label = (\alph*)]
  \item Wie schnell könnte man eine $(kn \times n)$-Matrix $A$ mit einer $(n \times kn)$-Matrix $B$ multiplizieren, d.h. $C = A \cdot B$ berechenen, wenn man Strassens Algorithmus als Unterprogramm verwendet?
  \item Benantworten Sie die gleiche Frage, wenn die Reihenfolge der Eingabematrizen vertauscht ist, man also $\tilde{C} = B \cdot A$ bestimmen möchte.
\end{enumerate}

\end{exercise}

% --------------------------------------------------------------------------------

\begin{solution}

\phantom{}

\begin{enumerate}[label = (\alph*)]

  \item Man kann die Matrizen als Blockmatrizen auffassen.
  
  \begin{align*}
    A
    =
    \begin{pmatrix}
      A_1 \\ \vdots \\ A_k
    \end{pmatrix},
    \quad
    B
    =
    (B_1 \cdots B_k),
    \quad
    A_1, \dots, A_k,
    B_1, \dots, B_k \enspace
    (n \times n) \text{-Matrizen}
  \end{align*}

  Man muss somit, zur Berechnung der $(kn \times kn)$-Matrix $C$, für $k^2$ Blöcke der Größe $n \times n$ durch je eine $(n \times n)$-Matrix-Multiplikation durchführen.
  Dies kann jeweils mit Strassens Algorithmus gemacht werden.

  \begin{align*}
    C
    =
    AB
    =
    \begin{pmatrix}
      A_1 \\ \vdots \\ A_k
    \end{pmatrix}
    (B_1 \cdots B_k)
    =
    \begin{pmatrix}
      A_1 B_1 & \cdots & A_1 B_k \\
      \vdots  & \ddots & \vdots \\
      A_k B_1 & \cdots & A_k B_k
    \end{pmatrix}
  \end{align*}
  
  Sei $n$ eine Zweierpotenz.
  Strassens Algorithmus hat den Aufwand $S(n) = \Theta(n^{\log 7})$.
  Es ergibt sich also eine Laufzeit von $T(n, k) = k^2 S(n) = \Theta(k^2 n^{\log 7})$.

  \item Analog zu oben betrachten wir wieder die $(n \times n)$-Matrix-Multiplikation von $(n \times n)$-Blockmatrizen.
  Hier benötigen wir $k$ viele $(n \times n)$-Matrix-Multiplikationen und $k - 1$ viele $(n \times n)$-Matrix-Additionen.

  \begin{align*}
    \tilde{C}
    =
    BA
    =
    (B_1 \cdots B_k)
    \begin{pmatrix}
      A_1 \\ \vdots \\ A_k
    \end{pmatrix}
    =
    \sum_{i=1}^k
    B_i A_i
  \end{align*}

  Es ergibt sich also eine Laufzeit von $T(n, k) = k S(n) = \Theta(k n^{\log 7})$.

\end{enumerate}

\end{solution}

% --------------------------------------------------------------------------------
