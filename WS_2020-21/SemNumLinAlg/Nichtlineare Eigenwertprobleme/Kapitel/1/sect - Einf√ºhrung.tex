\chapter{Einführung}

Die herkömmliche (lineare) Formulierung des Eigenwert-Problems dürfte bekannt sein:
Sei $A \in \C^{N \times N}$ eine Matrix.
Suche $\lambda \in \C$ und $v \in \C^N \setminus \Bbraces{0}$, sodass $A v = \lambda v$, oder
äquivalent

\begin{align} \label{eq:lineares_ewp}
    B(\lambda) v = 0,
    \quad
    B(\lambda) := A - \lambda I_N.
\end{align}

Dabei bezeichnet $\ker B(\lambda)$ den Eigenraum von $\lambda$.
In manchen Anwendungen ist die Funktion $B$ jedoch nicht mehr linear oder gar polynomiell.
Herkömmliche Algorithmen zur Suche von Eigenwerten sind jedoch bloß für lineare $B$, so wie in \eqref{eq:lineares_ewp}, konzipiert.

Die wohl naheliegendste Variante, einer nichtlinearen Eigenwert-Suche ist eine Nullstellen-Suche
des nichtlinearen charakteristischen Polynoms $\lambda \mapsto \det B(\lambda)$.
Bereits bei linearen Eigenwertproblemen kann man folgende Nachteile erkennen:

\begin{enumerate}[label = \arabic*.]
    \item Die Berechnung der Determinante ist (insbesondere für große Matrizen) überaus kostspielig
    und numerisch instabil.
    \item Wir bekommen bloß eine Nullstelle (Eigenwert).
    \item Selbst wenn ein Algorithmus wie das Newton-Verfahren zur Nullstellen-Suche verwendet wird, wird immer noch ein geeigneter Startwert benötigt.
\end{enumerate}

Die, in \cite{BEYN20123839} vorgeschlagene Konturintegral-Methode wird all diese Probleme umgehen.
Deren Grundbaustein ist folgendes Resultat aus der komplexen Analysis.
