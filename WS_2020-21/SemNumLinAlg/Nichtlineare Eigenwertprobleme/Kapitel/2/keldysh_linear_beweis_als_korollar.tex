\begin{proof}[Beweis (als Korollar)]

    Wir wollen Satz \ref{keldysh_nicht_linear} auf $\lambda \mapsto A - I_N \lambda$ anwenden.
    Wir überprüfen also die Voraussetzungen.

    \begin{enumerate}[label = \arabic*.]

        \item Unsere (lineare) Matrix-Funktion $\lambda \mapsto A - I_N \lambda$ ist offensichtlich (komponentenweise) holomorph.

        \item Für alle $\lambda \in \rho(A)$, ist $A - I_N \lambda \in \GL_N(\C)$.

        \item Die Existenz einer Orthonormalbasis $V_1$ von $\ker (A - I_N \lambda_1)$, wenn $\lambda_1 \in \sigma(A)$ halb-einfach ist, ist ein Resultat der Linearen Algebra.
        
        \begin{comment}

            Seien $\lambda_2, \dots, \lambda_k$ die restlichen (paarweise verschiedenen) Eigenwerte von $A$.
            Seien $L_n^\mathrm{geo}$ und $L_n^\mathrm{alg}$ die geometrische bzw. algebraische Vielfachheit von $\lambda_n$ für $n = 2, \dots, k$.
            Betrachte die Jordan Normalform von $A$.
        
            \begin{align*}
                T J T^{-1} & = A, \\
                J & = \diag (J_1, \dots, J_k),
                \quad
                T \in \GL_N(\C) \\
                J_n
                & =
                \diag
                \underbrace
                {
                    \pbraces
                    {
                        \begin{pmatrix}
                            \lambda_n & 1      &        &           \\
                                    & \ddots & \ddots &           \\
                                    &        & \ddots & 1         \\
                                    &        &        & \lambda_n \\
                        \end{pmatrix},
                        \dots,
                        \begin{pmatrix}
                            \lambda_n & 1      &        &           \\
                                    & \ddots & \ddots &           \\
                                    &        & \ddots & 1         \\
                                    &        &        & \lambda_n \\
                        \end{pmatrix}
                    }
                }_{
                    \displaystyle
                    L_n^\mathrm{geo} \text{-viele}
                }
                \in
                \C^{
                    L_n^\mathrm{alg}
                    \times
                    L_n^\mathrm{alg}
                },
                \quad
                n = 1, \dots, k
            \end{align*}
        
            Weil $\lambda_1$ halb-einfach ist, muss $J_1 = I_{L_1} \lambda_1$.
            Seien $\hat v_1, \dots, \hat v_N$ die linear unabhängig Spalten der \\ Transformations-Matrix $T$.
        
            \begin{align*}
                \implies
                (A \hat v_1, \dots, A \hat v_{L_1}, \ast)
                =
                A T
                \stackrel
                {
                    \text{JNF}
                }{=}
                T J
                =
                (\hat v_1, \dots, \hat v_{L_1}, \ast)
                \underbrace
                {
                    \begin{pmatrix}
                        J_1 & 0 \\
                        0   & \ast
                    \end{pmatrix}
                }_J
                =
                (\lambda_1 \hat v_1, \dots, \lambda_1 \hat v_{L_1}, \ast)
            \end{align*}
        
            Wir können die linear unabhängig $\hat v_1, \dots, v_{L_1}$ also orthonormalisieren (Gram-Schmidt) und erhalten die Orthonormalbasis $V_1 := (v_{1, 1}, \dots, v_{1, L_1})$.

        \end{comment}

        \item Wir müssen uns also eigentlich bloß überlegen, dass

        \begin{align*}
            \Forall l = 1, \dots, L_1:
                \derivative{\lambda} (A - I_N \lambda) \Big |_{\lambda = \lambda_1} v_{1, l}
                =
                -I_N v_{1, l}
                =
                -v_{1, l}
                \not \in
                \ran (A - I_N \lambda_1).
        \end{align*}

        Weil $\lambda_1$ halb-einfach ist, sein Jordan-Block genau $I_{L_1} \lambda_1$.
        Alle zugehörigen Jordan-Kästchen haben also die Größe $1 \times 1$.
        Es gibt somit bloß Hauptvektoren $1$-ter Stufe (Eigenwerte), und keine $2$-ter oder höherer.

        Sei $y \in \ker (A - I_N \lambda_1) \cap \ran (A - I_N \lambda_1)$, dann

        \begin{align*}
            \Exists x \in \C^N:
                (A - I_N \lambda_1) x = y,
                \quad
                (A - I_N \lambda_1)^2 x = (A - I_N \lambda_1) y = 0.
        \end{align*}

        Angenommen, $y \neq 0$, dann wäre $x$ ein Hauptvektor $2$-ter Stufe.
        Widerspruch!
        Damit gilt also

        \begin{align*}
            \ker (A - I_N \lambda_1) \cap \ran (A - I_N \lambda_1)
            =
            \Bbraces{0}
            \not \ni
            v_{1, 1}, \dots, v_{1, L_1}
            \begin{cases}
                     \in \ker (A - I_N \lambda), \\
                \not \in \ran (A - I_N \lambda).
            \end{cases}
        \end{align*}
    
    \end{enumerate}

    Wir können also Satz \ref{keldysh_nicht_linear} anwenden.
    
    \begin{enumerate}[label = (\roman*), start = 2]

        \item Der Satz \ref{keldysh_nicht_linear} gibt uns eine Basis $w_{1, 1}, \dots, w_{1, L_1}$ von $\ker (A - I_N \lambda_1)^\ast = \ker (A^\ast - I_N \overline \lambda_1)$, sodass
        
        \begin{align*}
            \Forall l, k = 1, \dots, L_1:
                (v_{1, k}, w_{1, l})_2
                =
                w_{1, l}^\ast v_{1, k}
                =
                -w_{1, l}^\ast (-I_N) v_{1, k}
                =
                -w_{1, l}^\ast \derivative{\lambda}(A - I_N \lambda) \Big |_{\lambda = \lambda_1} v_{1, k}
                =
                -\delta_{l, k}.
        \end{align*}

        \item Diese Tatsache kann $1 : 1$ aus Satz \ref{keldysh_nicht_linear} übernommen werden.

    \end{enumerate}
    
\end{proof}
