\begin{proof}[Beweis (als Korollar)]

    Wir überprüfen also die Voraussetzungen von Satz \ref{keldysh_nicht_linear}.

    \begin{enumerate}[label = \arabic*.]

        \item Unsere (lineare) Matrix-Funktion $\lambda \mapsto A - I_N \lambda$ ist offensichtlich holomorph.
        
        \item Wenn $\lambda$ kein Eigenwert von $A$ ist, dann ist $\ker(A - I_N \lambda) = \Bbraces{0}$, also $A - I_N \lambda \in \GL_N(\C)$ d.h. invertierbar.
        
        \item Seien $\lambda_2, \dots, \lambda_k$ die restlichen (paarweise verschiedenen) Eigenwert von $A$.
        Seien $L_n^\mathrm{geo}$ und $L_n^\mathrm{alg}$ die geometrische bzw. algebraische Vielfachheit von $\lambda_n$ für $n = 2, \dots, k$.
        Betrachte die JNF von $A$.
    
        \begin{align*}
            T J T^{-1} & = A, \\
            J & = \diag (J_1, \dots, J_k),
            \quad
            T \in \GL_N(\C) \\
            J_n
            & =
            \diag
            \underbrace
            {
                \pbraces
                {
                    \begin{pmatrix}
                        \lambda_n & 1      &        &           \\
                                  & \ddots & \ddots &           \\
                                  &        & \ddots & 1         \\
                                  &        &        & \lambda_n \\
                    \end{pmatrix},
                    \dots,
                    \begin{pmatrix}
                        \lambda_n & 1      &        &           \\
                                  & \ddots & \ddots &           \\
                                  &        & \ddots & 1         \\
                                  &        &        & \lambda_n \\
                    \end{pmatrix}
                }
            }_{
                \displaystyle
                L_n^\mathrm{geo} \text{-viele}
            }
            \in
            \C^{
                L_n^\mathrm{geo}
                \times
                L_n^\mathrm{alg}
            },
            \quad
            n = 1, \dots, k
        \end{align*}
    
        Weil $\lambda_1$ halb-einfach ist, muss $J_1 = I_{L_1} \lambda_1$.
        Seien $\hat v_1, \dots, \hat v_N$ die linear unabhängig Spalten der \\ Transformations-Matrix $T$.
    
        \begin{align*}
            \implies
            (A \hat v_1, \dots, A \hat v_{L_1}, \ast)
            =
            A T
            \stackrel
            {
                \text{JNF}
            }{=}
            T J
            =
            (\hat v_1, \dots, \hat v_{L_1}, \ast)
            \underbrace
            {
                \begin{pmatrix}
                    J_1 & 0 \\
                    0   & \ast
                \end{pmatrix}
            }_J
            =
            (\lambda_1 \hat v_1, \dots, \lambda_1 \hat v_{L_1}, \ast)
        \end{align*}
    
        Wir können die linear unabhängig $\hat v_1, \dots, v_{L_1}$ also orthonormalisieren (Gram-Schmidt) und erhalten die Orthonormalbasis $V_1 := (v_{1, 1}, \dots, v_{1, L_1})$.

        \item Die Größe der Jordan-Kästchen entspricht der Länge der Jordan-Ketten (Hauptvektor-Ketten).
        Alle Jordan-Ketten sind also $1$-gliedrig, bestehen also nur aus \Quote{echten} Eigenwerten.
        Es gibt also keine Hauptvektoren $2$-ter Stufe.
    
        Sei $y \in \ker(A - I_N \lambda_1) \cap \ran(A - I_N \lambda_1)$, dann gibt es ein $x \in \C^N$ mit $y = (A - I_N \lambda_1) x$ und $(A - I_N \lambda_1) y = 0$.
        Angenommen, $y \neq 0$, dann wäre $x$ ein Hauptvektor $2$-ter Stufe, weil

        \begin{align*}
            & \implies
            (A - I_N \lambda_1) x = y \neq 0,
            \quad
            (A - I_N \lambda_1)^2 x = (A - I_N \lambda_1) y = 0 \\
            & \implies
            x \in \ker(A - I_N \lambda_1)^2 \setminus \ker(A - I_N \lambda_1) = \emptyset.
        \end{align*}

        Widerspruch!
    
        \begin{align*}
            \implies
            \ker(A - I_N \lambda_1) \cap \ran(A - I_N \lambda_1) = \Bbraces{0}
        \end{align*}
    
        Die Ableitung unserer Matrixfunktion berechnet man komponentenweise.
        Weil $0 \neq v_{1, 1}, \dots, v_{1, L_1} \in \ker(A - I_N \lambda)$, folgt damit die letzte Voraussetzung.
        $\Forall l = 1, \dots, L_1:$

        \begin{align*}
            \derivative{\lambda} (A - I_N \lambda) \Big |_{\lambda = \lambda_1} v_{1, l}
            =
            -I_N v_{1, l}
            =
            -v_{1, l}
            \not \in
            \ran(A - I_N \lambda_1)
        \end{align*}

    \end{enumerate}

    Wir können also Satz \ref{keldysh_nicht_linear} anwenden.
    
    \begin{enumerate}[label = (\roman*)]

        \item Die Orthonormalbasis $V_1 = (v_{1, 1}, \dots, v_{1, L_1})$ von $\ker(A - I_N \lambda_1)$ haben wir bereits konstruiert.
        
        \item Der Satz \ref{keldysh_nicht_linear} gibt uns eine Basis $w_{1, 1}, \dots, w_{1, L_1}$ von $\ker(A - I_N \lambda_1)^\ast = \ker(A^\ast - I_N \overline \lambda_1)$, sodass $\Forall l, k = 1, \dots, L_1:$
        
        \begin{align*}
            (v_{1, k}, w_{1, l})_2
            =
            w_{1, l}^\ast v_{1, k}
            =
            -w_{1, l}^\ast (-I_N) v_{1, k}
            =
            -w_{1, l}^\ast \derivative{\lambda} (A - I_N \lambda) \Big |_{\lambda = \lambda_1} v_{1, k}
            =
            -\delta_{l, k}.
        \end{align*}

        \item Diese Tatsache kann $1 : 1$ aus Satz \ref{keldysh_nicht_linear} übernommen werden.

    \end{enumerate}
    
\end{proof}
