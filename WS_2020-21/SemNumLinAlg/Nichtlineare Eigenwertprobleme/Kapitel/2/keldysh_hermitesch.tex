\begin{remark}[Keldysh, hermitesch] \label{keldysh_hermitesch}

    Sei $A \in \C^{N \times N}$ hermitesch (selbstadjungiert).
    Seien $\lambda_1 < \cdots < \lambda_k$ deren Eigenwerte, jeweils zu den Vielfachheiten $L_1, \dots, L_k$, und für alle $n = 1, \dots, k$ sei $V_n = (v_{n, 1}, \dots, v_{n, L_n})$ eine Orthonormalbasis von $\ker (A - I_N \lambda_n)$, sodass $A$ unitär diagonalisierbar ist, d.h.

    \begin{align*}
        A = V^\ast D V,
        \quad
        \text{mit}
        \quad
        V := (V_1, \dots, V_k) \in \U_N(\C),
        \quad
        D := \diag (I_{L_1} \lambda_1, \dots, I_{L_k} \lambda_k).
    \end{align*}

    In Analogie zu Satz \ref{keldysh_nicht_linear}, bilden nun $-v_{1, 1}, \dots, -v_{1, L_1}$ eine Basis von $\ker (A^\ast - I_N \overline \lambda_1)$, sodass

    \begin{align*}
        \Forall l, k = 1, \dots, L_1:
            -v_{1, l}^\ast (-I_N) v_{1, k} = \delta_{l, k},
    \end{align*}

    und es gilt

    \begin{align*}
        \Forall \lambda \in U_1 \setminus \Bbraces{\lambda_1}:
            (A - I_N \lambda)^{-1}
            & =
            (V^\ast D V - V^\ast \lambda V)^{-1} \\
            & =
            (V^\ast \diag ((\lambda_1 - \lambda) I_{L_1}, \dots, (\lambda_k - \lambda) I_{L_k}) V)^{-1} \\
            & =
            V^\ast \diag \pbraces{\frac{1}{\lambda_1 - \lambda} I_{L_1}, \dots, \frac{1}{\lambda_N - \lambda} I_{L_k}} V \\
            & \stackrel{!}{=}
            \sum_{n=1}^k
                \frac{1}{\lambda_n - \lambda}
                \sum_{l=1}^{L_n}
                    v_{n, l}^\ast v_{n, l} \\
            & =
            \frac{1}{\lambda - \lambda_1}
            \sum_{l=1}^{L_1}
                v_{1, l} (-v_{1, l})^\ast
            +
            R_1(\lambda),
    \end{align*}

    wobei $U_1 \subseteq \rho(A) \cup \Bbraces{\lambda_1}$ eine nichtleere Umgebung von $\lambda_1$ ist und

    \begin{align*}
        R_1 \in H(U_1, \C^{N \times N}),
        \quad
        \lambda
        \mapsto
        \sum_{n=2}^k
            \frac{1}{\lambda - \lambda_n}
            \sum_{l=1}^{L_n}
                v_{n, l} (-v_{n, l})^\ast.
    \end{align*}

    Die Gleichheit mit dem \Quote{!}, erhält man durch eine elementare, aber sperrige Rechnung.

\end{remark}