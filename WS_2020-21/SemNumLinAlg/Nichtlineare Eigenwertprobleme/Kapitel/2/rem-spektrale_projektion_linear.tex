\begin{remark}[Spektrale Projektion] \label{rem:spektrale_projektion_linear}

    Weil $P_1$ eine Projektion ist, gilt $\C^N = X \oplus Y$, wobei

    \begin{align*}
        X & := \ker P_1 = \ran (I_N - P_1) \stackrel{!}{=} \ran (A - \lambda_1 I_N) \xrightarrow{A - \lambda_1 I_N} \ran (A - \lambda_1 I_N)^2 \subseteq \ran (A - \lambda_1 I_N) = X, \\
        Y & := \ker (I_N - P_1) = \ran P_1 \stackrel{!}{=} \ker (A - \lambda_1 I_N) \xrightarrow{A - \lambda_1 I_N} \Bbraces{0} \subseteq Y.
    \end{align*}

    Das $2$-te \enquote{!} gilt, weil als Spektrale Projektion

    \begin{align*}
        P_1: \C^N \to \ker (A - \lambda_1 I_N),
        \quad
        \Forall x \in \ker (A - \lambda_1 I_N):
            x = P_1 x \in \ran P_1.
    \end{align*}

    Das $1$-te \enquote{!} gilt, weil einerseits $\ran (A - \lambda_1 I_N) \subseteq \ran (I_N - P_1)$,

    \begin{align*}
        P_1 (A - \lambda_1 I_N)
        =
        -\sum_{l=1}^{L_1}
            v_{1, l} w_{1, l}^\ast
        (A - \lambda_1 I_N)
        =
        -\sum_{l=1}^{L_1}
            v_{1, l}
            \underbrace
            {
                (w_{1, l}^\ast A
                -
                w_{1, l}^\ast \lambda_1)
            }_{=0}
        =
        0.
    \end{align*}

    und andererseits gilt laut der Rangformel

    \begin{align*}
        \dim \ran (I_N - P_1)
        =
        N - \dim \ker (I_N - P_1)
        =
        N - \dim \ker (A - \lambda I_N)
        =
        \dim \ran (A - \lambda I_N).
    \end{align*}

    $X$ und $Y$ sind also $(A - \lambda_1 I_N)$-invariant.
    Wir sagen, dass $A - \lambda_1 I_N$ auch $X$- und $Y$-invariant ist.

    Außerdem, kommutieren $A - \lambda_1 I_N$ und $I_N - P_1$, weil

    \begin{multline*}
        (A - \lambda_1 I_N) (I_N - P_1)
        =
        (A - \lambda_1 I_N) - \underbrace{(A - \lambda_1 I_N) P_1}_{=0}
        =
        A - \lambda_1 I_N \\
        =
        (A - \lambda_1 I_N) - \underbrace{P_1 (A - \lambda_1 I_N)}_{=0}
        =
        (I_N - P_1) (A - \lambda_1 I_N).
    \end{multline*}

\end{remark}
