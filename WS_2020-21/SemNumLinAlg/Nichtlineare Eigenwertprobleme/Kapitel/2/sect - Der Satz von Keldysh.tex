\chapter{Theoretische Grundlagen}

\section{Der Satz von Keldysh}

Bevor wir starten, benötigen wir noch die folgende Begriffsbildung.

\begin{definition} \label{def:links_rechts_eigen_}

    Seien $A \in \C^{N \times N}$, $\lambda \in \C$, und $v \in \C^N$ mit $A v = \lambda v$.
    Wir nennen $(\lambda, v)$ ein \textit{Rechts-Eigenpaar} von $A$, $v$ einen \textit{Rechts-Eigenvektor} von $A$ zum Eigenwert $\lambda$, und $\ker (A - \lambda I_N)$ den \textit{Rechts-Eigenraum} von $\lambda$.

    Seien $B \in \C^{N \times N}$, $\mu \in \C$ und $w \in \C$, mit $w^\ast B = \mu w^\ast$.
    Wir nennen $(\mu, w)$ ein \textit{Links-Eigenpaar} von $B$, $w$ einen \textit{Links-Eigenvektor} von $B$ zum Eigenwert $\mu$, $\ker (B^\ast - \overline \mu I_N)$ den \textit{Links-Eigenraum} von $\mu$.

\end{definition}

\begin{remark} \label{rem:links_rechts_eigen_}

    Wir schließen direkt an die Definition \ref{def:links_rechts_eigen_} an.
    Der Rechts-Eigenvektor $v$ von $A$ zum Eigenwert $\lambda$ ist auch Links-Eigenvektor von $A^\ast$ zum Eigenwert $\overline \lambda$, weil

    \begin{align*}
        v^\ast A^\ast
        =
        (A v)^\ast
        =
        (\lambda v)^\ast
        =
        \overline \lambda v^\ast.
    \end{align*}

    Der Links-Eigenvektor $w$ von $B$ zum Eigenwert $\mu$ ist auch Rechts-Eigenvektor von $B^\ast$ zum Eigenwert $\overline \mu$, weil

    \begin{align*}
        B^\ast w
        =
        (w^\ast B)^\ast
        =
        (\mu w^\ast)^\ast
        =
        \overline \mu w.
    \end{align*}

    Daher ist die Definition von $\ker (B^\ast - \overline \mu I_N)$ als Links-Eigenraum von $\mu$ sinnvoll.

    Sollte $A$ hermitesch (selbstadjungiert) sein, so sind alle Eigenwerte reell, und damit alle Rechts-Eigenvektoren auch Links-Eigenvektoren zum selben Eigenwert.

\end{remark}

\begin{comment}

    $A^\ast$ ist tatsächlich die Adjungierte von $A$ im Sinne der Funktionalanalysis, weil $\Forall x, y \in \C^N:$

    \begin{align*}
        (A x, y)_2
        =
        y^\ast A x
        =
        (A^\ast y)^\ast x
        =
        (x, A^\ast y)_2.
    \end{align*}

    Nun ist $\overline \lambda$ Eigenwert von $A^\ast$ mit derselben algebraischen Vielfachheit wie $\lambda$, weil

    \begin{align*}
        \chi_{A^\ast}(\lambda)
        =
        \det(A^\ast - \lambda I_N)
        =
        \overline{\det(A - \lambda I_N)^\top}
        =
        \overline{\chi_A(\lambda)}.
    \end{align*}

\end{comment}


Wir führen den Satz von Keldysh zunächst in seiner allgemeinsten Form aus.

\begin{theorem}[Keldysh, nicht-linear] \label{thm:keldysh_nicht_linear}

    Sei $\Lambda \subset \C$ ein beschränktes Gebiet und $A \in H(\Lambda, \C^{N \times N})$ holomorph.
    Es existiere ein $\lambda \in \Lambda$, sodass $A(\lambda) \in \GL_N(\C)$.

    Weiter sei $\lambda_1 \in \Lambda$ ein halb-einfacher Eigenwert, d.h. es existiere eine Orthonormalbasis aus \\ (Rechts-)Eigenvektoren $v_{1, 1}, \dots, v_{1, L_1}$ von $\ker A(\lambda_1)$.
    Für diese gelte

    \begin{align} \label{eq:keldysh_nicht_linear_1}
        \Forall l = 1, \dots, L_1:
            A^\prime(\lambda_1) v_{1, l}
            \not \in
            \ran A(\lambda_1).
    \end{align}

    Dann existiert eine Basis aus Links-Eigenvektoren $w_{1, 1}, \dots, w_{1, L_1}$ von $\ker A^\ast(\lambda_1)$, sodass

    \begin{align} \label{eq:keldysh_nicht_linear_2}
        \Forall l, k = 1, \dots, L_1:
            w_{1, l}^\ast A^\prime(\lambda_1) v_{1, k} = \delta_{l, k}.
    \end{align}

    Weiterhin existiert eine Umgebung $U_1$ von $\lambda_1$ und $R_1 \in H(U_1, \C^{N \times N})$ holomorph, sodass

    \begin{align} \label{eq:keldysh_nicht_linear_3}
        \Forall \lambda \in U_1 \setminus \Bbraces{\lambda_1}:
            A(\lambda)^{-1}
            =
            \frac{1}{\lambda - \lambda_1} S_1
            +
            R_1(\lambda),
            \quad
            \text{mit}
            \quad
            S_1
            :=
            \sum_{l=1}^{L_1}
                v_{1, l} w_{1, l}^\ast.
    \end{align}

\end{theorem}


Der Beweis dieses Satzes ist sehr aufwändig und nicht zentraler Teil dieser Arbeit.
Wir verweisen dazu also auf \cite{BEYN20123839}.
Wir wollen aber dennoch eine Intuition dafür vermitteln.
Dazu behandeln wir dessen linearen Spezialfall mit der zusätzlichen Voraussetzung, dass die Matrix dabei hermitesch ist.

\begin{remark}[Keldysh, linear, hermitesch] \label{keldysh_linear_hermitesch}

    Sei $A \in \C^{N \times N}$ hermitesch (selbstadjungiert).
    Seien $\lambda_1 < \cdots < \lambda_k$ deren Eigenwerte, jeweils zu den Vielfachheiten $L_1, \dots, L_k$, und für alle $n = 1, \dots, k$ sei $V_n = (v_{n, 1}, \dots, v_{n, L_n})$ eine Orthonormalbasis von $\ker (A - \lambda_n I_N)$, sodass $A$ unitär diagonalisierbar ist, d.h.

    \begin{align*}
        A = V^\ast D V,
        \quad
        \text{mit}
        \quad
        V := (V_1, \dots, V_k) \in \U_N(\C),
        \quad
        D := \diag (\lambda_1 I_{L_1}, \dots, \lambda_k I_{L_k}).
    \end{align*}

    In Analogie zu Satz \ref{thm:keldysh_nicht_linear}, bilden nun $-v_{1, 1}, \dots, -v_{1, L_1}$ eine Basis von $\ker (A^\ast - \overline \lambda_1 I_N)$, sodass

    \begin{align*}
        \Forall l, k = 1, \dots, L_1:
            -v_{1, l}^\ast (-I_N) v_{1, k} = \delta_{l, k},
    \end{align*}

    und es gilt

    \begin{align*}
        \Forall \lambda \in U_1 \setminus \Bbraces{\lambda_1}:
            (A - \lambda I_N)^{-1}
            & =
            (V^\ast D V - V^\ast \lambda V)^{-1} \\
            & =
            (V^\ast \diag ((\lambda_1 - \lambda) I_{L_1}, \dots, (\lambda_k - \lambda) I_{L_k}) V)^{-1} \\
            & =
            V^\ast \diag \pbraces{\frac{1}{\lambda_1 - \lambda} I_{L_1}, \dots, \frac{1}{\lambda_N - \lambda} I_{L_k}} V \\
            & \stackrel{!}{=}
            \sum_{n=1}^k
                \frac{1}{\lambda_n - \lambda}
                \sum_{l=1}^{L_n}
                    v_{n, l}^\ast v_{n, l} \\
            & =
            \frac{1}{\lambda - \lambda_1}
            \sum_{l=1}^{L_1}
                v_{1, l} (-v_{1, l})^\ast
            +
            R_1(\lambda),
    \end{align*}

    wobei $U_1 \subseteq \rho(A) \cup \Bbraces{\lambda_1}$ eine nichtleere Umgebung von $\lambda_1$ ist und

    \begin{align*}
        R_1 \in H(U_1, \C^{N \times N}),
        \quad
        \lambda
        \mapsto
        \sum_{n=2}^k
            \frac{1}{\lambda - \lambda_n}
            \sum_{l=1}^{L_n}
                v_{n, l} (-v_{n, l})^\ast.
    \end{align*}

    Die Gleichheit mit dem \Quote{!} erhält man durch eine elementare, aber sperrige Rechnung.

\end{remark}


Die Eigenschaften aus Satz \ref{thm:keldysh_nicht_linear} sind Verallgemeinerungen bekannter Begriffe und Tatsachen aus dem linearen Setting.
\eqref{eq:keldysh_nicht_linear_1} fordert, dass es keine Hauptvektoren $2$-ter Stufe gibt und \eqref{eq:keldysh_nicht_linear_2} beschreibt eine Orthogonalität zwischen Links- und Rechts-Eigenvektoren.
Die Zusammenhänge zwischen nicht-linear und linear wollen wir nun noch mit \ref{prop:keldysh_linear} verdeutlichen.

\begin{proposition}[Keldysh, linear] \label{prop:keldysh_linear}

    Sei $\lambda_1$ ein halb-einfacher Eigenwert einer Matrix $A \in \C^{N \times N}$, d.h. geometrische Vielfachheit $L_1^\mathrm{geo}$ und algebraische Vielfachheit $L_1^\mathrm{alg}$ stimmen überein, d.h.

    \begin{align*}
        L_1
        :=
        L_1^\mathrm{geo} := \Def(A - \lambda_1 I_N)
        =
        L_1^\mathrm{alg} := \mu_1 = \max \Bbraces{\mu \in \N: (\lambda - \lambda_1)^\mu \mid \chi_A(\lambda)}.
    \end{align*}

    Dann gilt Folgendes.

    \begin{enumerate}[label = (\roman*)]
        \item Es gibt eine Orthonormalbasis $V_1 = (v_{1, 1}, \dots, v_{1, L_1})$ von $\ker (A - \lambda_1 I_N)$.
    \end{enumerate}

    Weiters gelten folgende $2$ Analoga zu Satz \ref{thm:keldysh_nicht_linear}.

    \begin{enumerate}[label = (\roman*), start = 2]

        \item Es gibt eine Basis $W_1 = (w_{1, 1}, \dots, w_{1, L_1})$ von $\ker (A^\ast - \overline \lambda_1 I_N)$, sodass

        \begin{align*}
            \Forall l, k = 1, \dots, L_1:
            (v_{1, k}, w_{1, l})_2 = -\delta_{l, k}.
        \end{align*}

        \item Es existiert eine Umgebung $U_1$ von $\lambda_1$ und $R_1 \in H(U_1, \C^{N \times N})$ holomorph, sodass $\Forall \lambda \in U_1 \setminus \Bbraces{\lambda_1}:$

        \begin{align*}
            (A - \lambda I_N)^{-1}
            =
            \frac{1}{\lambda - \lambda_1} S_1
            +
            R_1(\lambda),
            \quad
            S_1
            :=
            \sum_{l=1}^{L_1}
                v_{1, l} w_{1, l}^\ast.
        \end{align*}

    \end{enumerate}

\end{proposition}

\begin{proof}[Beweis (als Korollar)]

    Wir wollen Satz \ref{thm:keldysh_nicht_linear} auf $\lambda \mapsto A - \lambda I_N$ anwenden.
    Wir überprüfen also die Voraussetzungen.

    \begin{enumerate}[label = \arabic*.]

        \item Unsere (lineare) Matrix-Funktion $\lambda \mapsto A - \lambda I_N$ ist offensichtlich (komponentenweise) holomorph.

        \item Für alle $\lambda \in \rho(A)$, ist $A - \lambda I_N \in \GL_N(\C)$.

        \item Die Existenz einer Orthonormalbasis $V_1$ von $\ker (A - \lambda_1 I_N)$, wenn $\lambda_1 \in \sigma(A)$ halb-einfach ist, ist einfache Linearen Algebra.

        \begin{comment}

            Seien $\lambda_2, \dots, \lambda_k$ die restlichen (paarweise verschiedenen) Eigenwerte von $A$.
            Seien $L_n^\mathrm{geo}$ und $L_n^\mathrm{alg}$ die geometrische bzw. algebraische Vielfachheit von $\lambda_n$ für $n = 2, \dots, k$.
            Betrachte die Jordan Normalform von $A$.
        
            \begin{align*}
                T J T^{-1} & = A, \\
                J & = \diag (J_1, \dots, J_k),
                \quad
                T \in \GL_N(\C) \\
                J_n
                & =
                \diag
                \underbrace
                {
                    \pbraces
                    {
                        \begin{pmatrix}
                            \lambda_n & 1      &        &           \\
                                    & \ddots & \ddots &           \\
                                    &        & \ddots & 1         \\
                                    &        &        & \lambda_n \\
                        \end{pmatrix},
                        \dots,
                        \begin{pmatrix}
                            \lambda_n & 1      &        &           \\
                                    & \ddots & \ddots &           \\
                                    &        & \ddots & 1         \\
                                    &        &        & \lambda_n \\
                        \end{pmatrix}
                    }
                }_{
                    \displaystyle
                    L_n^\mathrm{geo} \text{-viele}
                }
                \in
                \C^{
                    L_n^\mathrm{alg}
                    \times
                    L_n^\mathrm{alg}
                },
                \quad
                n = 1, \dots, k
            \end{align*}
        
            Weil $\lambda_1$ halb-einfach ist, muss $J_1 = \lambda_1 I_{L_1}$.
            Seien $\hat v_1, \dots, \hat v_N$ die linear unabhängig Spalten der \\ Transformations-Matrix $T$.
        
            \begin{align*}
                \implies
                (A \hat v_1, \dots, A \hat v_{L_1}, \ast)
                =
                A T
                \stackrel
                {
                    \text{JNF}
                }{=}
                T J
                =
                (\hat v_1, \dots, \hat v_{L_1}, \ast)
                \underbrace
                {
                    \begin{pmatrix}
                        J_1 & 0 \\
                        0   & \ast
                    \end{pmatrix}
                }_J
                =
                (\lambda_1 \hat v_1, \dots, \lambda_1 \hat v_{L_1}, \ast)
            \end{align*}
        
            Wir können die linear unabhängig $\hat v_1, \dots, v_{L_1}$ also orthonormalisieren (Gram-Schmidt) und erhalten die Orthonormalbasis $V_1 := (v_{1, 1}, \dots, v_{1, L_1})$.

        \end{comment}

        \item Wir müssen uns also eigentlich bloß überlegen, dass

        \begin{align*}
            \Forall l = 1, \dots, L_1:
                \derivative{\lambda} (A - \lambda I_N) \Big |_{\lambda = \lambda_1} v_{1, l}
                =
                -I_N v_{1, l}
                =
                -v_{1, l}
                \not \in
                \ran (A - \lambda_1 I_N).
        \end{align*}

        Weil $\lambda_1$ halb-einfach ist, sein Jordan-Block genau $\lambda_1 I_{L_1}$.
        Alle zugehörigen Jordan-Kästchen haben also die Größe $1 \times 1$.
        Es gibt somit bloß Hauptvektoren $1$-ter Stufe (Eigenwerte), und keine $2$-ter oder höherer.

        Sei $y \in \ker (A - \lambda_1 I_N) \cap \ran (A - \lambda_1 I_N)$, dann

        \begin{align*}
            \Exists x \in \C^N:
                (A - \lambda_1 I_N) x = y,
                \quad
                (A - \lambda_1 I_N)^2 x = (A - \lambda_1 I_N) y = 0.
        \end{align*}

        Angenommen, $y \neq 0$, dann wäre $x$ ein Hauptvektor $2$-ter Stufe.
        Widerspruch!
        Damit gilt also

        \begin{align*}
            \ker (A - \lambda_1 I_N) \cap \ran (A - \lambda_1 I_N)
            =
            \Bbraces{0}
            \not \ni
            v_{1, 1}, \dots, v_{1, L_1}
            \begin{cases}
                     \in \ker (A - \lambda I_N), \\
                \not \in \ran (A - \lambda I_N).
            \end{cases}
        \end{align*}
    
    \end{enumerate}

    Wir können also Satz \ref{thm:keldysh_nicht_linear} anwenden.
    
    \begin{enumerate}[label = (\roman*), start = 2]

        \item Der Satz \ref{thm:keldysh_nicht_linear} gibt uns eine Basis $w_{1, 1}, \dots, w_{1, L_1}$ von $\ker (A - \lambda_1 I_N)^\ast = \ker (A^\ast - \overline \lambda_1 I_N)$, sodass
        
        \begin{multline*}
            \Forall l, k = 1, \dots, L_1:
                (v_{1, k}, w_{1, l})_2
                =
                w_{1, l}^\ast v_{1, k}
                =
                -w_{1, l}^\ast (-I_N) v_{1, k} \\
                =
                -w_{1, l}^\ast \derivative{\lambda}(A - \lambda I_N) \Big |_{\lambda = \lambda_1} v_{1, k}
                =
                -\delta_{l, k}.
        \end{multline*}

        \item Diese Tatsache kann $1 : 1$ aus Satz \ref{thm:keldysh_nicht_linear} übernommen werden.

    \end{enumerate}
    
\end{proof}


\begin{remark}[Spektrale Projektion] \label{rem-spektrale_projektion_nicht_linear}

    $P_1 := S_1 A^\prime(\lambda_1)$ ist eine Projektion von $\C^N$ auf den (Rechts-)Eigenraum $\ker A(\lambda_1)$.
    Sei nämlich $x \in \C^N$, so gilt

    \begin{align*}
        P_1 x
        =
        \sum_{l=1}^{L_1}
            v_{1, l}
            \underbrace
            {
                w_{1, l}^\ast
                A^\prime(\lambda_1)
                x
            }_{
                \in \C
            }
        \in
        \ker A(\lambda).
    \end{align*}

    Außerdem ist $P_1$ idempotent, weil

    \begin{multline*}
        \Forall x \in \ker A(\lambda_1):
            P_1 x
            =
            \sum_{l=1}^{L_1}
                v_{1, l} w_{1, l}^\ast
                A^\prime(\lambda_1)
                \sum_{k=1}^{L_1}
                    (x, v_{1, k})_2 v_{1, k} \\
            =
            \sum_{l=1}^{L_1}
                \sum_{k=1}^{L_1}
                    (x, v_{1, k})_2
                    v_{1, l}
                    \underbrace
                    {
                        w_{1, l}^\ast
                        A^\prime(\lambda_1)
                        v_{1, k}
                    }_{
                        = \delta_{l, k}
                    }
            =
            \sum_{l=1}^{L_1}
                (x, v_{1, l})_2
                v_{1, l}
            =
            x.
    \end{multline*}

\end{remark}

% \begin{remark}[Spektrale Projektion] \label{rem:spektrale_projektion_linear}

    Weil $P_1$ eine Projektion ist, gilt $\C^N = X \oplus Y$, wobei

    \begin{align*}
        X & := \ker P_1 = \ran (I_N - P_1) \stackrel{!}{=} \ran (A - \lambda_1 I_N) \xrightarrow{A - \lambda_1 I_N} \ran (A - \lambda_1 I_N)^2 \subseteq \ran (A - \lambda_1 I_N) = X, \\
        Y & := \ker (I_N - P_1) = \ran P_1 \stackrel{!}{=} \ker (A - \lambda_1 I_N) \xrightarrow{A - \lambda_1 I_N} \Bbraces{0} \subseteq Y.
    \end{align*}

    Das $2$-te \blockquote{!} gilt, weil als Spektrale Projektion

    \begin{align*}
        P_1: \C^N \to \ker (A - \lambda_1 I_N),
        \quad
        \Forall x \in \ker (A - \lambda_1 I_N):
            x = P_1 x \in \ran P_1.
    \end{align*}

    Das $1$-te \blockquote{!} gilt, weil einerseits $\ran (A - \lambda_1 I_N) \subseteq \ran (I_N - P_1)$,

    \begin{align*}
        P_1 (A - \lambda_1 I_N)
        =
        -\sum_{l=1}^{L_1}
            v_{1, l} w_{1, l}^\ast
        (A - \lambda_1 I_N)
        =
        -\sum_{l=1}^{L_1}
            v_{1, l}
            \underbrace
            {
                (w_{1, l}^\ast A
                -
                w_{1, l}^\ast \lambda_1)
            }_{=0}
        =
        0.
    \end{align*}

    und andererseits gilt laut der Rangformel

    \begin{align*}
        \dim \ran (I_N - P_1)
        =
        N - \dim \ker (I_N - P_1)
        =
        N - \dim \ker (A - \lambda I_N)
        =
        \dim \ran (A - \lambda I_N).
    \end{align*}

    $X$ und $Y$ sind also $(A - \lambda_1 I_N)$-invariant.
    Wir sagen, dass $A - \lambda_1 I_N$ auch $X$- und $Y$-invariant ist.

    Außerdem, kommutieren $A - \lambda_1 I_N$ und $I_N - P_1$, weil

    \begin{multline*}
        (A - \lambda_1 I_N) (I_N - P_1)
        =
        (A - \lambda_1 I_N) - \underbrace{(A - \lambda_1 I_N) P_1}_{=0}
        =
        A - \lambda_1 I_N \\
        =
        (A - \lambda_1 I_N) - \underbrace{P_1 (A - \lambda_1 I_N)}_{=0}
        =
        (I_N - P_1) (A - \lambda_1 I_N).
    \end{multline*}

\end{remark}


% \begin{remark} \label{semi_inverse}

    Sei $f_1$ die lineare Abbildung, die von $A - \lambda_1 I_N$ dargestellt wird.
    Seien $X$ und $Y$ so wie in Bemerkung \ref{rem:spektrale_projektion_linear}.
    Wir wählen eine Basis von $X$ und erweitern diese, vermöge der Basis $V_1$ von $Y$, zu einer von ganz $\C^N = X \oplus Y$.
    Wenn wir bzgl. dieser koordinatisieren, dann bekommen folgende Darstellungen (modulo Basis-Transformation).

    \begin{gather*}
        f_1
        \cong
        \begin{pmatrix}
            B_1 & 0 \\
            0   & 0
        \end{pmatrix},
        \quad
        B_1 \in \C^{(N - L_1) \times (N - L_1)}, \\
        \Forall v \in \C^N:
            \ExistsOnlyOne x \in X:
            \ExistsOnlyOne y \in Y:
                v = x + y, \\
        x
        \cong
        (
            v_1, \dots, v_{N - L_1},
            0, \dots, 0
        )^\top,
        \quad
        y
        \cong
        (
            0, \dots, 0,
            v_{N - L_1 + 1}, \dots, v_N
        )^\top
    \end{gather*}

    Nur (d.h. genau) Eigenvektoren $y \in Y$, und $y = 0$ erfüllen $(A - \lambda_1 I_N) y = 0$.
    Weil $P_1$ ja die Spektrale Projektion ist, gilt aber $P_1 y = y \neq 0$, also $y \not \in \ker P_1 = X$.

    \begin{align*}
        \implies
        \ker f_1 |_X = \Bbraces{0}
        \implies
        \GL(X) \ni f |_X \cong B \in \GL_{N - L_1}(\C)
    \end{align*}

    Wir identifizieren daher (modulo Basis-Transformation)

    \begin{align*}
        (A - \lambda_1 I_N) |_X
        \cong
        \begin{pmatrix}
            B_1 & 0 \\ 0 & 0
        \end{pmatrix},
        \quad
        (A - \lambda_1 I_N) |_X^{-1}
        \cong
        \begin{pmatrix}
            B_1^{-1} & 0 \\ 0 & 0
        \end{pmatrix}.
    \end{align*}

\end{remark}
% \input{Kapitel/2/prop-keldysh_linear_beweis_zu_fuß.tex}

Das folgende Resultat spricht nicht nur über einen einzelnen Eigenwert, sondern mehrere.
Es wird eine Grundlegende Rolle in unserer Konstruktion spielen.

\begin{corollary} \label{corr:keldysh_multi}

    Sei $\Lambda \subset \C$ ein beschränktes Gebiet und $A \in H(\Lambda, \C^{N \times N})$ holomorph.
    Es existiere ein $\lambda \in \Lambda$, sodass $A(\lambda) \in \GL_N(\C)$, d.h. invertierbar ist.
    Mögen $\lambda_1, \dots, \lambda_k$, paarweise verschiedene Eigenwerte von $A$, die Voraussetzungen von Satz \ref{thm:keldysh_nicht_linear} erfüllen.
    Seien $S_1, \dots, S_k$ die entsprechenden Summen aus \eqref{eq:keldysh_nicht_linear_3}.

    Dann existiert ein $R \in H(\Lambda, \C^{N \times N})$, sodass $\Forall \lambda \in \Lambda \setminus \Bbraces{\lambda_1, \dots, \lambda_k}:$

    \begin{align*}
        A(\lambda)^{-1}
        =
        \sum_{n=1}^k
            \frac{1}{\lambda - \lambda_n} S_n
        +
        R(\lambda).
    \end{align*}

\end{corollary}

\begin{proof}

    Laut \ref{thm:keldysh_nicht_linear}, existiert für $m = 1, \dots, k$ eine Umgebung $U_m$ von $\lambda_m$ und $R_m \in H(U_m, \C^{N \times N})$ holomorph, sodass

    \begin{align*}
        \Forall \lambda \in U_m \setminus \Bbraces{\lambda_m}:
            A(\lambda)^{-1}
            =
            \frac{1}{\lambda - \lambda_m} S_m
            +
            R_m(\lambda).
    \end{align*}

    Die beiden Funktionen $R_\Lambda \in H(\Lambda \setminus \Bbraces{\lambda_1, \dots, \lambda_k}, \C^{N \times N})$ und $R_{U_m} \in H(U_m, \C^{N \times N})$ stimmen auf $U_m \setminus \{\lambda_m\}$ überein:

    \begin{multline*}
        \Forall \lambda \in U_m \setminus \Bbraces{\lambda_m}:
            R_\Lambda(\lambda)
            :=
            A(\lambda)^{-1}
            -
            \sum_{n=1}^k
                \frac{1}{\lambda - \lambda_n} S_n \\
            =
            \frac{1}{\lambda - \lambda_m} S_m
            +
            R_m(\lambda)
            -
            \sum_{n=1}^k
                \frac{1}{\lambda - \lambda_n} S_n
            =
            R_m(\lambda)
            -
            \sum_{\substack{n = 1 \\ n \neq m}}^k
                \frac{1}{\lambda - \lambda_n} S_n
            =:
            R_{U_m}(\lambda).
    \end{multline*}
    Damit können wir den Identitätssatz für holomorphe Funktionen anwenden.
    Die folgende Funktion $R$ ist also wohldefiniert, holomorph auf $\Lambda$ und leistet das gewünschte:

    \begin{align*}
        R:
        \Lambda \to C^{N \times N}:
        \lambda
        \mapsto
        \begin{cases}
            R_\Lambda(\lambda), & \lambda \in \Lambda \setminus \Bbraces{\lambda_1, \dots, \lambda_k}, \\
            R_{U_m}(\lambda),   & \Exists m = 1, \dots, k: \lambda \in U_m.
        \end{cases}
    \end{align*}

\end{proof}

