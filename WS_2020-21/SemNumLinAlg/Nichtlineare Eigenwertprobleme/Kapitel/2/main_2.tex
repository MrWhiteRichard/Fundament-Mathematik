\section{Der Satz von Keldysh}

Bevor wir starten, benötigen wir noch die folgende Begriffsbildung.

\begin{remark} \label{links_rechts_eigen_}
    
    Sei $(\lambda, v)$ ein Eigenpaar der Matrix $A \in \C^{N \times N}$.
    Wir nennen $v$ einen \textit{Rechts-Eigenvektor} von $A$ zum Eigenwert $\lambda$.
    Dieser besitzt den bekannten \textit{Rechts-Eigenraum} $\ker(A - I_N \lambda)$.

    $A^\ast$ ist tatsächlich die Adjungierte von $A$ im Sinne der Funktionalanalysis, weil $\Forall x, y \in \C^N:$

    \begin{align*}
        (A x, y)_2
        =
        y^\ast A x
        =
        (A^\ast y)^\ast x
        =
        (x, A^\ast y)_2.
    \end{align*}

    Nun ist $\overline \lambda$ Eigenwert von $A^\ast$ mit derselben algebraischen Vielfachheit wie $\lambda$, weil

    \begin{align*}
        \chi_{A^\ast}(\lambda)
        =
        \det(A^\ast - I_N \lambda)
        =
        \overline{\det(A - I_N \lambda)^\top}
        =
        \overline{\chi_A(\lambda)}.
    \end{align*}

    $v$ ist dann auch ein sogenannter \textit{Links-Eigenvektor} von $A^\ast$ zum Eigenwert $\overline \lambda$, weil

    \begin{align*}
        v^\ast A^\ast
        =
        (A v)^\ast
        =
        (\lambda v)^\ast
        =
        \overline \lambda v^\ast.
    \end{align*}

    Sämtliche Links-Eigenvektoren bilden (gemeinsam mit der $0$) den \textit{Links-Eigenraum} von $\lambda$ bzgl. $A$.
    Dieser ist in der Tat ein Unterraum.

\end{remark}


Wir führen den Satz von Keldysh zunächst in seiner allgemeinsten Form aus.

\begin{theorem}[Keldysh, nicht-linear] \label{keldysh_nicht_linear}

    Sei $\Lambda \subset \C$ ein beschränktes Gebiet und $A \in H(\Lambda, \C^{N \times N})$ holomorph.
    Es existiere ein $\lambda \in \Lambda$, sodass $A(\lambda) \in \GL_N(\C)$.

    Weiter sei $\lambda_1 \in \Lambda$ ein halb-einfacher Eigenwert, d.h. es existiere eine Orthonormalbasis aus \\ (Rechts-)Eigenvektoren $v_{1, 1}, \dots, v_{1, L_1}$ von $\ker A(\lambda_1)$.
    Für diese gelte

    \begin{align*}
        \Forall l = 1, \dots, L_1:
            A^\prime(\lambda_1) v_{1, l}
            \not \in
            \ran A(\lambda_1).
    \end{align*}

    Dann existiert eine Basis aus Links-Eigenvektoren $w_{1, 1}, \dots, w_{1, L_1}$ von $\ker A^\ast(\lambda_1)$, sodass

    \begin{align}
        \Forall l, k = 1, \dots, L_1:
            w_{1, l}^\ast A^\prime(\lambda_1) v_{1, k} = \delta_{l, k}.
    \end{align}

    Weiterhin existiert eine Umgebung $U_1$ von $\lambda_1$ und $R_1 \in H(U_1, \C^{N \times N})$ holomorph, sodass

    \begin{align} \label{eq:darstellung_nicht_linear}
        \Forall \lambda \in U_1 \setminus \Bbraces{\lambda_1}:
            A(\lambda)^{-1}
            =
            \frac{1}{\lambda - \lambda_1} S_1
            +
            R_1(\lambda),
            \quad
            \text{mit}
            \quad
            S_1
            :=
            \sum_{l=1}^{L_1}
                v_{1, l} w_{1, l}^\ast.
    \end{align}

\end{theorem}


Der Beweis dieses Satzes ist sehr aufwändig und nicht zentraler Teil dieser Arbeit.
Wir verweisen dazu also auf \cite{BEYN20123839}.
Wir wollen aber dennoch eine Intuition dafür vermitteln.
Dazu behandeln wir dessen linearen Spezialfall, und wenn die zugehörige Matrix hermitesch ist.
Es wird sich herausstellen, dass die Eigenschaften aus Satz \ref{keldysh_nicht_linear} Verallgemeinerungen bekannter (oder zumindest plausibler) Begriffe und Tatsachen aus dem linearen Setting sind.

\begin{remark}[Keldysh, hermitesch] \label{keldysh_hermitesch}

    Sei $A \in \C^{N \times N}$ hermitesch (selbstadjungiert).
    Seien $\lambda_1 < \cdots < \lambda_k$ deren Eigenwerte, jeweils zu den Vielfachheiten $L_1, \dots, L_k$, und für alle $n = 1, \dots, k$ sei $V_n = (v_{n, 1}, \dots, v_{n, L_n})$ eine Orthonormalbasis von $\ker (A - I_N \lambda_n)$, sodass $A$ unitär diagonalisierbar ist, d.h.

    \begin{align*}
        A = V^\ast D V,
        \quad
        \text{mit}
        \quad
        V := (V_1, \dots, V_k) \in \U_N(\C),
        \quad
        D := \diag (I_{L_1} \lambda_1, \dots, I_{L_k} \lambda_k).
    \end{align*}

    In Analogie zu Satz \ref{keldysh_nicht_linear}, bilden nun $-v_{1, 1}, \dots, -v_{1, L_1}$ eine Basis von $\ker (A^\ast - I_N \overline \lambda_1)$, sodass

    \begin{align*}
        \Forall l, k = 1, \dots, L_1:
            -v_{1, l}^\ast (-I_N) v_{1, k} = \delta_{l, k},
    \end{align*}

    und es gilt

    \begin{align*}
        \Forall \lambda \in U_1 \setminus \Bbraces{\lambda_1}:
            (A - I_N \lambda)^{-1}
            & =
            (V^\ast D V - V^\ast \lambda V)^{-1} \\
            & =
            (V^\ast \diag ((\lambda_1 - \lambda) I_{L_1}, \dots, (\lambda_k - \lambda) I_{L_k}) V)^{-1} \\
            & =
            V^\ast \diag \pbraces{\frac{1}{\lambda_1 - \lambda} I_{L_1}, \dots, \frac{1}{\lambda_N - \lambda} I_{L_k}} V \\
            & \stackrel{!}{=}
            \sum_{n=1}^k
                \frac{1}{\lambda_n - \lambda}
                \sum_{l=1}^{L_n}
                    v_{n, l}^\ast v_{n, l} \\
            & =
            \frac{1}{\lambda - \lambda_1}
            \sum_{l=1}^{L_1}
                v_{1, l} (-v_{1, l})^\ast
            +
            R_1(\lambda),
    \end{align*}

    wobei $U_1 \subseteq \rho(A) \cup \Bbraces{\lambda_1}$ eine nichtleere Umgebung von $\lambda_1$ ist und

    \begin{align*}
        R_1 \in H(U_1, \C^{N \times N}),
        \quad
        \lambda
        \mapsto
        \sum_{n=2}^k
            \frac{1}{\lambda - \lambda_n}
            \sum_{l=1}^{L_n}
                v_{n, l} (-v_{n, l})^\ast.
    \end{align*}

    Die Gleichheit mit dem \Quote{!}, erhält man durch eine elementare, aber sperrige Rechnung.

\end{remark}

\begin{remark} \label{spektrale_projektion}
    
    $P_1^- := -P_1^+$ nennt man die \textit{Spektrale Projektion} auf den Eigenraum von $\lambda_1$.
    Offensichtlich bildet $P_1^-$ ganz $\C^N$ auf den Rechts-Eigenraum von $\lambda_1$ ab, sie ist aber auch idempotent, weil $\Forall x \in \ker(A - I_N \lambda_1):$

    \begin{align*}
        P_1^- x
        =
        -\sum_{l=1}^{L_1}
            v_{1, l}
            w_{1, l}^\ast
            \sum_{k=1}^{L_1}
                (x, v_{1, k})_2
                v_{1, k}
        =
        -\sum_{l=1}^{L_1}
            v_{1, l}
            \sum_{k=1}^{L_1}
                \underbrace{(v_{1, k}, w_{1, l})_2}_{-\delta_{l, k}}
                (x, v_{1, k})_2
        =
        \sum_{l=1}^{L_1}
            (x, v_{1, l})_2
            v_{1, l}
        =
        x.
    \end{align*}

    Also ist $P_1^-$ tatsächlich eine Projektion.
    Damit gilt also $ \C^N = X \oplus Y$.

    \begin{align*}
        Y & := \ker (I_n - P_1^-) = \ran P_1^- \stackrel{!}{=} \Span V_1 = \ker(A - I_N \lambda_1) & \xrightarrow{A - I_N \lambda_1} \quad & \Bbraces{0} \subseteq Y \\
        X & := \ker P_1^- = \ran(I_N - P_1^-) \stackrel{!}{=} \ran(A - I_N \lambda_1)              & \xrightarrow{A - I_N \lambda_1} \quad & \ran(A - I_N \lambda_1) = X
    \end{align*}

    Das erste \Quote{!} gilt, weil als Spektrale Projektion

    \begin{align*}
        P_1^-: \C^N \to \Span V_1,
        \quad
        \Forall l = 1, \dots, L_1:
            v_{1, l} = P_1^- v_{1, l} \in \ran P_1^-;
    \end{align*}

    das zweite \Quote{!} gilt, weil $A - I_N \lambda_1$ und $I_N - P_1^-$ kommutieren.

    \begin{align*}
        P_1^- (A - I_N \lambda_1)
        =
        -\sum_{l=1}^{L_1}
            v_{1, l} w_{1, l}^\ast
        (A - I_N \lambda_1)
        =
        -\sum_{l=1}^{L_1}
            v_{1, l}
            \underbrace
            {
                (w_{1, l}^\ast A
                -
                w_{1, l}^\ast \lambda_1)
            }_0
        =
        0
    \end{align*}

    \begin{multline*}
        (A - I_N \lambda_1) (I_N - P_1^-)
        =
        (A - I_N \lambda_1) - \underbrace{(A - I_N \lambda_1) P_1^-}_0 \\
        =
        (A - I_N \lambda_1) - \underbrace{P_1^- (A - I_N \lambda_1)}_0
        =
        (I_N - P_1^-) (A - I_N \lambda_1)
    \end{multline*}

    $X$ und $Y$ sind also $(A - I_N \lambda_1)$-invariant.
    Wir sagen, dass $A - I_N \lambda_1$ auch $X$- und $Y$-invariant ist.

\end{remark}

\begin{theorem}[Keldysh, linear] \label{keldysh_linear}

    Sei $\lambda_1$ ein halb-einfacher Eigenwert einer Matrix $A \in \C^{N \times N}$, d.h. geometrische Vielfachheit $L_1^\mathrm{geo}$ und algebraische Vielfachheit $L_1^\mathrm{alg}$ stimmen überein.

    \begin{align*}
        L_1
        :=
        L_1^\mathrm{geo} := \Def(A - I_N \lambda_1)
        =
        L_1^\mathrm{alg} := \mu_1 = \max \Bbraces{\mu \in \N: (\lambda - \lambda_1)^\mu \mid \chi_A(\lambda)},
    \end{align*}

    Dann gelten folgende $3$ Analoga zu Satz \ref{keldysh_nicht_linear}.

    \begin{enumerate}[label = (\roman*)]

        \item Es gibt eine Orthonormalbasis $V_1 = (v_{1, 1}, \dots, v_{1, L_1})$ von $\ker (A - I_N \lambda_1)$.

        \item Es gibt eine Basis $W_1 = (w_{1, 1}, \dots, w_{1, L_1})$ von $\ker (A^\ast - I_N \overline \lambda_1)$, sodass

        \begin{align*}
            \Forall l, k = 1, \dots, L_1:
            (v_{1, k}, w_{1, l})_2 = -\delta_{l, k}.
        \end{align*}

        \item Es existiert eine Umgebung $U_1$ von $\lambda_1$ und $R_1 \in H(U_1, \C^{N \times N})$ holomorph, sodass $\Forall \lambda \in U_1 \setminus \Bbraces{\lambda_1}:$

        \begin{align*}
            (A - I_N \lambda)^{-1}
            =
            \frac{1}{\lambda - \lambda_1} P_1^+
            +
            R_1(\lambda),
            \quad
            P_1^+
            :=
            \sum_{l=1}^{L_1}
                v_{1, l} w_{1, l}^\ast.
        \end{align*}

    \end{enumerate}

\end{theorem}

\begin{proof}[Beweis (als Korollar)]

    Wir überprüfen also die Voraussetzungen von Satz \ref{keldysh_nicht_linear}.

    \begin{enumerate}[label = \arabic*.]

        \item Unsere (lineare) Matrix-Funktion $\lambda \mapsto A - I_N \lambda$ ist offensichtlich holomorph.
        
        \item Wenn $\lambda$ kein Eigenwert von $A$ ist, dann ist $\ker(A - I_N \lambda) = \Bbraces{0}$, also $A - I_N \lambda \in \GL_N(\C)$ d.h. invertierbar.
        
        \item Seien $\lambda_2, \dots, \lambda_k$ die restlichen (paarweise verschiedenen) Eigenwert von $A$.
        Seien $L_n^\mathrm{geo}$ und $L_n^\mathrm{alg}$ die geometrische bzw. algebraische Vielfachheit von $\lambda_n$ für $n = 2, \dots, k$.
        Betrachte die JNF von $A$.
    
        \begin{align*}
            T J T^{-1} & = A, \\
            J & = \diag (J_1, \dots, J_k),
            \quad
            T \in \GL_N(\C) \\
            J_n
            & =
            \diag
            \underbrace
            {
                \pbraces
                {
                    \begin{pmatrix}
                        \lambda_n & 1      &        &           \\
                                  & \ddots & \ddots &           \\
                                  &        & \ddots & 1         \\
                                  &        &        & \lambda_n \\
                    \end{pmatrix},
                    \dots,
                    \begin{pmatrix}
                        \lambda_n & 1      &        &           \\
                                  & \ddots & \ddots &           \\
                                  &        & \ddots & 1         \\
                                  &        &        & \lambda_n \\
                    \end{pmatrix}
                }
            }_{
                \displaystyle
                L_n^\mathrm{geo} \text{-viele}
            }
            \in
            \C^{
                L_n^\mathrm{geo}
                \times
                L_n^\mathrm{alg}
            },
            \quad
            n = 1, \dots, k
        \end{align*}
    
        Weil $\lambda_1$ halb-einfach ist, muss $J_1 = I_{L_1} \lambda_1$.
        Seien $\hat v_1, \dots, \hat v_N$ die linear unabhängig Spalten der \\ Transformations-Matrix $T$.
    
        \begin{align*}
            \implies
            (A \hat v_1, \dots, A \hat v_{L_1}, \ast)
            =
            A T
            \stackrel
            {
                \text{JNF}
            }{=}
            T J
            =
            (\hat v_1, \dots, \hat v_{L_1}, \ast)
            \underbrace
            {
                \begin{pmatrix}
                    J_1 & 0 \\
                    0   & \ast
                \end{pmatrix}
            }_J
            =
            (\lambda_1 \hat v_1, \dots, \lambda_1 \hat v_{L_1}, \ast)
        \end{align*}
    
        Wir können die linear unabhängig $\hat v_1, \dots, v_{L_1}$ also orthonormalisieren (Gram-Schmidt) und erhalten die Orthonormalbasis $V_1 := (v_{1, 1}, \dots, v_{1, L_1})$.

        \item Die Größe der Jordan-Kästchen entspricht der Länge der Jordan-Ketten (Hauptvektor-Ketten).
        Alle Jordan-Ketten sind also $1$-gliedrig, bestehen also nur aus \Quote{echten} Eigenwerten.
        Es gibt also keine Hauptvektoren $2$-ter Stufe.
    
        Sei $y \in \ker(A - I_N \lambda_1) \cap \ran(A - I_N \lambda_1)$, dann gibt es ein $x \in \C^N$ mit $y = (A - I_N \lambda_1) x$ und $(A - I_N \lambda_1) y = 0$.
        Angenommen, $y \neq 0$, dann wäre $x$ ein Hauptvektor $2$-ter Stufe, weil

        \begin{align*}
            & \implies
            (A - I_N \lambda_1) x = y \neq 0,
            \quad
            (A - I_N \lambda_1)^2 x = (A - I_N \lambda_1) y = 0 \\
            & \implies
            x \in \ker(A - I_N \lambda_1)^2 \setminus \ker(A - I_N \lambda_1) = \emptyset.
        \end{align*}

        Widerspruch!
    
        \begin{align*}
            \implies
            \ker(A - I_N \lambda_1) \cap \ran(A - I_N \lambda_1) = \Bbraces{0}
        \end{align*}
    
        Die Ableitung unserer Matrixfunktion berechnet man komponentenweise.
        Weil $0 \neq v_{1, 1}, \dots, v_{1, L_1} \in \ker(A - I_N \lambda)$, folgt damit die letzte Voraussetzung.
        $\Forall l = 1, \dots, L_1:$

        \begin{align*}
            \derivative{\lambda} (A - I_N \lambda) \Big |_{\lambda = \lambda_1} v_{1, l}
            =
            -I_N v_{1, l}
            =
            -v_{1, l}
            \not \in
            \ran(A - I_N \lambda_1)
        \end{align*}

    \end{enumerate}

    Wir können also Satz \ref{keldysh_nicht_linear} anwenden.
    
    \begin{enumerate}[label = (\roman*)]

        \item Die Orthonormalbasis $V_1 = (v_{1, 1}, \dots, v_{1, L_1})$ von $\ker(A - I_N \lambda_1)$ haben wir bereits konstruiert.
        
        \item Der Satz \ref{keldysh_nicht_linear} gibt uns eine Basis $w_{1, 1}, \dots, w_{1, L_1}$ von $\ker(A - I_N \lambda_1)^\ast = \ker(A^\ast - I_N \overline \lambda_1)$, sodass $\Forall l, k = 1, \dots, L_1:$
        
        \begin{align*}
            (v_{1, k}, w_{1, l})_2
            =
            w_{1, l}^\ast v_{1, k}
            =
            -w_{1, l}^\ast (-I_N) v_{1, k}
            =
            -w_{1, l}^\ast \derivative{\lambda} (A - I_N \lambda) \Big |_{\lambda = \lambda_1} v_{1, k}
            =
            -\delta_{l, k}.
        \end{align*}

        \item Diese Tatsache kann $1 : 1$ aus Satz \ref{keldysh_nicht_linear} übernommen werden.

    \end{enumerate}
    
\end{proof}

% \begin{remark} \label{semi_inverse}

    Sei $f_1$ die lineare Abbildung, die von $A - \lambda_1 I_N$ dargestellt wird.
    Seien $X$ und $Y$ so wie in Bemerkung \ref{spektrale_projektion}.
    Wir wählen eine Basis von $X$ und erweitern diese, vermöge der Basis $V_1$ von $Y$, zu einer von ganz $\C^N = X \oplus Y$.
    Wenn wir bzgl. dieser koordinatisieren, dann bekommen folgende Darstellungen (modulo Basis-Transformation).

    \begin{gather*}
        f_1
        \cong
        \begin{pmatrix}
            B_1 & 0 \\
            0   & 0
        \end{pmatrix},
        \quad
        B_1 \in \C^{(N - L_1) \times (N - L_1)}, \\
        \Forall v \in \C^N:
            \ExistsOnlyOne x \in X:
            \ExistsOnlyOne y \in Y:
                v = x + y, \\
        x
        \cong
        (
            v_1, \dots, v_{N - L_1},
            0, \dots, 0
        )^\top,
        \quad
        y
        \cong
        (
            0, \dots, 0,
            v_{N - L_1 + 1}, \dots, v_N
        )^\top
    \end{gather*}

    Nur (d.h. genau) Eigenvektoren $y \in Y$, und $y = 0$ erfüllen $(A - \lambda_1 I_N) y = 0$.
    Weil $P_1^-$ ja die Spektrale Projektion ist, gilt aber $P_1^- y = y \neq 0$, also $y \not \in \ker P_1^- = X$.

    \begin{align*}
        \implies
        \ker f_1 |_X = \Bbraces{0}
        \implies
        \GL(X) \ni f |_X \cong B \in \GL_{N - L_1}(\C)
    \end{align*}

    Wir identifizieren daher (modulo Basis-Transformation)

    \begin{align*}
        (A - \lambda_1 I_N) |_X
        \cong
        \begin{pmatrix}
            B_1 & 0 \\ 0 & 0
        \end{pmatrix},
        \quad
        (A - \lambda_1 I_N) |_X^{-1}
        \cong
        \begin{pmatrix}
            B_1^{-1} & 0 \\ 0 & 0
        \end{pmatrix}.
    \end{align*}

\end{remark}
% \input{Kapitel/2/keldysh_linear_beweis_zu_fuß.tex}
\begin{corollary} \label{keldysh_multi}

    Sei $\Lambda \subset \C$ ein beschränktes Gebiet und $A \in H(\Lambda, \C^{N \times N})$ holomorph.
    Es existiere ein $\lambda \in \Lambda$, sodass $A(\lambda) \in \GL_N(\C)$, d.h. invertierbar ist.
    Mögen $\lambda_1, \dots, \lambda_k$, paarweise verschiedene Eigenwerte von $A$, die Voraussetzungen von Satz \ref{keldysh_nicht_linear} erfüllen.
    Seien $S_1, \dots, S_k$ die entsprechenden Summen aus \eqref{eq:darstellung_nicht_linear}.

    Dann existiert ein $R \in H(\Lambda, \C^{N \times N})$, sodass $\Forall \lambda \in \Lambda \setminus \Bbraces{\lambda_1, \dots, \lambda_k}:$

    \begin{align*}
        A(\lambda)^{-1}
        =
        \sum_{n=1}^k
            \frac{1}{\lambda - \lambda_n} S_n
        +
        R(\lambda).
    \end{align*}

\end{corollary}

\begin{proof}

    Laut \ref{keldysh_nicht_linear}, existiert für $m = 1, \dots, k$ eine Umgebung $U_m$ von $\lambda_m$ und $R_m \in H(U_m)$ holomorph, sodass $\Forall \lambda \in U_m \setminus \Bbraces{\lambda_m}:$

    \begin{align*}
        A(\lambda)^{-1}
        =
        \frac{1}{\lambda - \lambda_m} P_m^+
        +
        R_m(\lambda).
    \end{align*}

    Die beiden Funktionen $R_\Lambda \in H(\Lambda \setminus \Bbraces{\lambda_1, \dots, \lambda_k})$ und $R_{U_m} \in H(U_m)$ stimmen überein $\Forall \lambda \in U_m \setminus \Bbraces{\lambda_m}:$

    \begin{multline*}
        R_\Lambda(\lambda)
        :=
        A(\lambda)^{-1}
        -
        \sum_{n=1}^k
            \frac{1}{\lambda - \lambda_n} P_n^+
        =
        \frac{1}{\lambda - \lambda_m} P_m^+
        +
        R_m(\lambda)
        -
        \sum_{n=1}^k
            \frac{1}{\lambda - \lambda_n} P_n^+ \\
        =
        R_m(\lambda)
        -
        \sum_{\substack{n = 1 \\ n \neq m}}^k
            \frac{1}{\lambda - \lambda_n} P_n^+
        =:
        R_{U_m}(\lambda).
    \end{multline*}

    Die folgende Funktion $R$ ist also wohldefiniert, $\in H(\Lambda)$ und leistet das Gewünscht.

    \begin{align*}
        R:
        \Lambda \to C^{N \times N}:
        \lambda
        \mapsto
        \begin{cases}
            R_\Lambda(\lambda), & \lambda \in \dom R_\Lambda, \\
            R_{U_m}(\lambda),   & \Exists m = 1, \dots, k: \lambda \in \dom R_{U_m}
        \end{cases}
    \end{align*}

\end{proof}

