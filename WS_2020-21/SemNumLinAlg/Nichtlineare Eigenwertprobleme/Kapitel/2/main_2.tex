\section{Der Satz von Keldysh}

Bevor wir starten, benötigen wir noch die folgende Begriffsbildung.

\begin{remark} \label{links_rechts_eigen_}
    
    Sei $(\lambda, v)$ ein Eigenpaar der Matrix $A \in \C^{N \times N}$.
    Wir nennen $v$ einen \textit{Rechts-Eigenvektor} von $A$ zum Eigenwert $\lambda$.
    Dieser besitzt den bekannten \textit{Rechts-Eigenraum} $\ker(A - I_N \lambda)$.

    $A^\ast$ ist tatsächlich die Adjungierte von $A$ im Sinne der Funktionalanalysis, weil $\Forall x, y \in \C^N:$

    \begin{align*}
        (A x, y)_2
        =
        y^\ast A x
        =
        (A^\ast y)^\ast x
        =
        (x, A^\ast y)_2.
    \end{align*}

    Nun ist $\overline \lambda$ Eigenwert von $A^\ast$ mit derselben algebraischen Vielfachheit wie $\lambda$, weil

    \begin{align*}
        \chi_{A^\ast}(\lambda)
        =
        \det(A^\ast - I_N \lambda)
        =
        \overline{\det(A - I_N \lambda)^\top}
        =
        \overline{\chi_A(\lambda)}.
    \end{align*}

    $v$ ist dann auch ein sogenannter \textit{Links-Eigenvektor} von $A^\ast$ zum Eigenwert $\overline \lambda$, weil

    \begin{align*}
        v^\ast A^\ast
        =
        (A v)^\ast
        =
        (\lambda v)^\ast
        =
        \overline \lambda v^\ast.
    \end{align*}

    Sämtliche Links-Eigenvektoren bilden (gemeinsam mit der $0$) den \textit{Links-Eigenraum} von $\lambda$ bzgl. $A$.
    Dieser ist in der Tat ein Unterraum.

\end{remark}


Wir führen den Satz von Keldysh zunächst in seiner allgemeinsten Form aus.

\begin{theorem}[Keldysh, nicht-linear] \label{keldysh_nicht_linear}

    Sei $\Lambda \subset \C$ ein beschränktes Gebiet und $A \in H(\Lambda, \C^{N \times N})$ holomorph.
    Es existiere ein $\lambda \in \Lambda$, sodass $A(\lambda) \in \GL_N(\C)$, d.h. invertierbar ist.

    Weiter sei $\lambda_1 \in \Lambda$ ein halb-einfacher Eigenwert, d.h. es existiere eine $L_1$-dimensionale Orthonormalbasis aus (Rechts-)Eigenvektoren $v_{1, 1}, \dots, v_{1, L_1} \in \ker A(\lambda_1)$.
    Für diese gelte $\Forall l = 1, \dots, L_1:$
    
    \begin{align*}
        A^\prime(\lambda_1) v_{1, l}
        \not \in
        \ran A(\lambda_1).
    \end{align*}

    Dann existiert eine Basis $w_{1, 1}, \dots, w_{1, L_1}$ von $\ker(A^\ast(\lambda_1))$, sodass $\Forall l, k = 1, \dots, L_1:$

    \begin{align}
        w_{1, l}^\ast A^\prime(\lambda_1) v_{1, k} = \delta_{l, k}.
    \end{align}

    Weiterhin existiert eine Umgebung $U_1$ von $\lambda_1$ und $R_1 \in H(U_1)$ holomorph, sodass $\Forall \lambda \in U_1 \setminus \Bbraces{\lambda_1}:$

    \begin{align}
        A(\lambda)^{-1}
        =
        \frac{1}{\lambda - \lambda_1} P_1^+
        +
        R_1(\lambda),
        \quad
        P_1^+
        :=
        \sum_{l=1}^{L_1}
            v_{1, l} w_{1, l}^\ast.
    \end{align}

\end{theorem}


Der Beweis dieses Satzes ist sehr aufwändig und nicht zentraler Teil dieser Arbeit.
Wir verweisen dazu also auf \cite{BEYN20123839}.
Wir wollen aber dennoch eine Intuition dafür vermitteln.
Dazu behandeln wir dessen linearen Spezialfall mit der zusätzlichen Voraussetzung, dass
die Matrix dabei hermitesch ist.
Es wird sich herausstellen, dass die Eigenschaften aus Satz \ref{keldysh_nicht_linear} Verallgemeinerungen bekannter (oder zumindest plausibler) Begriffe und Tatsachen aus dem linearen Setting sind.

\input{Kapitel/2/keldysh_linear_hermitesch.tex}

\begin{remark} \label{spektrale_projektion}
    
    $P_1^- := -P_1^+$ nennt man die \textit{Spektrale Projektion} auf den Eigenraum von $\lambda_1$.
    Offensichtlich bildet $P_1^-$ ganz $\C^N$ auf den Rechts-Eigenraum von $\lambda_1$ ab, sie ist aber auch idempotent, weil $\Forall x \in \ker(A - I_N \lambda_1):$

    \begin{align*}
        P_1^- x
        =
        -\sum_{l=1}^{L_1}
            v_{1, l}
            w_{1, l}^\ast
            \sum_{k=1}^{L_1}
                (x, v_{1, k})_2
                v_{1, k}
        =
        -\sum_{l=1}^{L_1}
            v_{1, l}
            \sum_{k=1}^{L_1}
                \underbrace{(v_{1, k}, w_{1, l})_2}_{-\delta_{l, k}}
                (x, v_{1, k})_2
        =
        \sum_{l=1}^{L_1}
            (x, v_{1, l})_2
            v_{1, l}
        =
        x.
    \end{align*}

    Also ist $P_1^-$ tatsächlich eine Projektion.
    Damit gilt also $ \C^N = X \oplus Y$.

    \begin{align*}
        Y & := \ker (I_n - P_1^-) = \ran P_1^- \stackrel{!}{=} \Span V_1 = \ker(A - I_N \lambda_1) & \xrightarrow{A - I_N \lambda_1} \quad & \Bbraces{0} \subseteq Y \\
        X & := \ker P_1^- = \ran(I_N - P_1^-) \stackrel{!}{=} \ran(A - I_N \lambda_1)              & \xrightarrow{A - I_N \lambda_1} \quad & \ran(A - I_N \lambda_1) = X
    \end{align*}

    Das erste \Quote{!} gilt, weil als Spektrale Projektion

    \begin{align*}
        P_1^-: \C^N \to \Span V_1,
        \quad
        \Forall l = 1, \dots, L_1:
            v_{1, l} = P_1^- v_{1, l} \in \ran P_1^-;
    \end{align*}

    das zweite \Quote{!} gilt, weil $A - I_N \lambda_1$ und $I_N - P_1^-$ kommutieren.

    \begin{align*}
        P_1^- (A - I_N \lambda_1)
        =
        -\sum_{l=1}^{L_1}
            v_{1, l} w_{1, l}^\ast
        (A - I_N \lambda_1)
        =
        -\sum_{l=1}^{L_1}
            v_{1, l}
            \underbrace
            {
                (w_{1, l}^\ast A
                -
                w_{1, l}^\ast \lambda_1)
            }_0
        =
        0
    \end{align*}

    \begin{multline*}
        (A - I_N \lambda_1) (I_N - P_1^-)
        =
        (A - I_N \lambda_1) - \underbrace{(A - I_N \lambda_1) P_1^-}_0 \\
        =
        (A - I_N \lambda_1) - \underbrace{P_1^- (A - I_N \lambda_1)}_0
        =
        (I_N - P_1^-) (A - I_N \lambda_1)
    \end{multline*}

    $X$ und $Y$ sind also $(A - I_N \lambda_1)$-invariant.
    Wir sagen, dass $A - I_N \lambda_1$ auch $X$- und $Y$-invariant ist.

\end{remark}

\begin{theorem}[Keldysh, linear] \label{keldysh_linear}
    
    Sei $\lambda_1$ ein halb-einfacher Eigenwert einer Matrix $A \in \C^{N \times N}$, d.h. geometrische Vielfachheit $L_1^\mathrm{geo}$ und algebraische Vielfachheit $L_1^\mathrm{alg}$ stimmen überein.

    \begin{align*}
        L_1^\mathrm{geo} = \Def(A - I_N \lambda),
        \quad
        L_1^\mathrm{alg} = \mu_1 = \max \Bbraces{\mu \in \N: (\lambda - \lambda_1)^\mu \mid \chi_A(\lambda)},
        \quad
        L_1 := L_1^\mathrm{geo} + L_1^\mathrm{alg}
    \end{align*}

    Dann gelten folgende $3$ Analoga zu Satz \ref{keldysh_nicht_linear}.

    \begin{enumerate}[label = (\roman*)]

        \item Es gibt eine Orthonormalbasis $V_1 = (v_{1, 1}, \dots, v_{1, L_1})$ von $\ker(A - I_N \lambda_1)$.

        \item Es gibt eine Basis $W_1 = (w_{1, 1}, \dots, w_{1, L_1})$ von $\ker(A^\ast - I_N \overline \lambda_1)$, sodass
        
        \begin{align*}
            \Forall l, k = 1, \dots, L_1:
            (v_{1, k}, w_{1, l})_2 = -\delta_{l, k}.
        \end{align*}

        \item Es existiert eine Umgebung $U_1$ von $\lambda_1$ und $R_1 \in H(U_1)$ holomorph, sodass $\Forall \lambda \in U_1 \setminus \Bbraces{\lambda_1}:$

        \begin{align*}
            (A - I_N \lambda)^{-1}
            =
            \frac{1}{\lambda - \lambda_1} P_1^+
            +
            R_1(\lambda),
            \quad
            P_1^+
            :=
            \sum_{l=1}^{L_1}
                v_{1, l} w_{1, l}^\ast.
        \end{align*}

    \end{enumerate}

\end{theorem}

\begin{proof}[Beweis (als Korollar)]

    Wir wollen Satz \ref{keldysh_nicht_linear} auf $\lambda \mapsto A - I_N \lambda$ anwenden.
    Wir überprüfen also die Voraussetzungen.

    \begin{enumerate}[label = \arabic*.]

        \item Unsere (lineare) Matrix-Funktion $\lambda \mapsto A - I_N \lambda$ ist offensichtlich (komponentenweise) holomorph.

        \item Für alle $\lambda \in \rho(A)$, ist $A - I_N \lambda \in \GL_N(\C)$.

        \item Die Existenz einer Orthonormalbasis $V_1$ von $\ker (A - I_N \lambda_1)$, wenn $\lambda_1 \in \sigma(A)$ halb-einfach ist, ist ein Resultat der Linearen Algebra.
        
        \begin{comment}

            Seien $\lambda_2, \dots, \lambda_k$ die restlichen (paarweise verschiedenen) Eigenwerte von $A$.
            Seien $L_n^\mathrm{geo}$ und $L_n^\mathrm{alg}$ die geometrische bzw. algebraische Vielfachheit von $\lambda_n$ für $n = 2, \dots, k$.
            Betrachte die Jordan Normalform von $A$.
        
            \begin{align*}
                T J T^{-1} & = A, \\
                J & = \diag (J_1, \dots, J_k),
                \quad
                T \in \GL_N(\C) \\
                J_n
                & =
                \diag
                \underbrace
                {
                    \pbraces
                    {
                        \begin{pmatrix}
                            \lambda_n & 1      &        &           \\
                                    & \ddots & \ddots &           \\
                                    &        & \ddots & 1         \\
                                    &        &        & \lambda_n \\
                        \end{pmatrix},
                        \dots,
                        \begin{pmatrix}
                            \lambda_n & 1      &        &           \\
                                    & \ddots & \ddots &           \\
                                    &        & \ddots & 1         \\
                                    &        &        & \lambda_n \\
                        \end{pmatrix}
                    }
                }_{
                    \displaystyle
                    L_n^\mathrm{geo} \text{-viele}
                }
                \in
                \C^{
                    L_n^\mathrm{alg}
                    \times
                    L_n^\mathrm{alg}
                },
                \quad
                n = 1, \dots, k
            \end{align*}
        
            Weil $\lambda_1$ halb-einfach ist, muss $J_1 = I_{L_1} \lambda_1$.
            Seien $\hat v_1, \dots, \hat v_N$ die linear unabhängig Spalten der \\ Transformations-Matrix $T$.
        
            \begin{align*}
                \implies
                (A \hat v_1, \dots, A \hat v_{L_1}, \ast)
                =
                A T
                \stackrel
                {
                    \text{JNF}
                }{=}
                T J
                =
                (\hat v_1, \dots, \hat v_{L_1}, \ast)
                \underbrace
                {
                    \begin{pmatrix}
                        J_1 & 0 \\
                        0   & \ast
                    \end{pmatrix}
                }_J
                =
                (\lambda_1 \hat v_1, \dots, \lambda_1 \hat v_{L_1}, \ast)
            \end{align*}
        
            Wir können die linear unabhängig $\hat v_1, \dots, v_{L_1}$ also orthonormalisieren (Gram-Schmidt) und erhalten die Orthonormalbasis $V_1 := (v_{1, 1}, \dots, v_{1, L_1})$.

        \end{comment}

        \item Wir müssen uns also eigentlich bloß überlegen, dass

        \begin{align*}
            \Forall l = 1, \dots, L_1:
                \derivative{\lambda} (A - I_N \lambda) \Big |_{\lambda = \lambda_1} v_{1, l}
                =
                -I_N v_{1, l}
                =
                -v_{1, l}
                \not \in
                \ran (A - I_N \lambda_1).
        \end{align*}

        Weil $\lambda_1$ halb-einfach ist, sein Jordan-Block genau $I_{L_1} \lambda_1$.
        Alle zugehörigen Jordan-Kästchen haben also die Größe $1 \times 1$.
        Es gibt somit bloß Hauptvektoren $1$-ter Stufe (Eigenwerte), und keine $2$-ter oder höherer.

        Sei $y \in \ker (A - I_N \lambda_1) \cap \ran (A - I_N \lambda_1)$, dann

        \begin{align*}
            \Exists x \in \C^N:
                (A - I_N \lambda_1) x = y,
                \quad
                (A - I_N \lambda_1)^2 x = (A - I_N \lambda_1) y = 0.
        \end{align*}

        Angenommen, $y \neq 0$, dann wäre $x$ ein Hauptvektor $2$-ter Stufe.
        Widerspruch!
        Damit gilt also

        \begin{align*}
            \ker (A - I_N \lambda_1) \cap \ran (A - I_N \lambda_1)
            =
            \Bbraces{0}
            \not \ni
            v_{1, 1}, \dots, v_{1, L_1}
            \begin{cases}
                     \in \ker (A - I_N \lambda), \\
                \not \in \ran (A - I_N \lambda).
            \end{cases}
        \end{align*}
    
    \end{enumerate}

    Wir können also Satz \ref{keldysh_nicht_linear} anwenden.
    
    \begin{enumerate}[label = (\roman*), start = 2]

        \item Der Satz \ref{keldysh_nicht_linear} gibt uns eine Basis $w_{1, 1}, \dots, w_{1, L_1}$ von $\ker (A - I_N \lambda_1)^\ast = \ker (A^\ast - I_N \overline \lambda_1)$, sodass
        
        \begin{align*}
            \Forall l, k = 1, \dots, L_1:
                (v_{1, k}, w_{1, l})_2
                =
                w_{1, l}^\ast v_{1, k}
                =
                -w_{1, l}^\ast (-I_N) v_{1, k}
                =
                -w_{1, l}^\ast \derivative{\lambda}(A - I_N \lambda) \Big |_{\lambda = \lambda_1} v_{1, k}
                =
                -\delta_{l, k}.
        \end{align*}

        \item Diese Tatsache kann $1 : 1$ aus Satz \ref{keldysh_nicht_linear} übernommen werden.

    \end{enumerate}
    
\end{proof}

% \begin{remark} \label{semi_inverse}

    Sei $X := \ker P_1^-$ und $Y := \ker (I_N - P_1^-)$ so wie in Bemerkung \ref{spektrale_projektion}.
    Wir erweitern die Basis $V_1$ von $Y$ zu einer von ganz $\C^N = X \oplus Y$, vermöge einer Basis von $X$.
    Wenn wir bzgl. dieser koordinatisieren, dann bekommen folgende Darstellungen.

    \begin{align*}
        A - I_N \lambda_1
        =
        \begin{pmatrix}
            B_1 & 0 \\
            0   & B_2
        \end{pmatrix},
        \quad
        \Forall x \in X:
            x
            =
            (
                \underbrace
                {
                    x_1, \dots, x_{N - (L_1 + 1)}
                }_{
                    \in \C
                },
                \underbrace
                {
                    0, \dots, 0
                }_{
                    \displaystyle
                    \text{$L_1$-mal}
                }
            )^\top
    \end{align*}

    Die $0$-er in der Block-Matrix treten auf, weil $A - I_N \lambda_1$ ja $X$- und $Y$-invariant ist, und $X \oplus Y = \C^N$.
    Nur (d.h. genau) Eigenvektoren $x$ von $A$ bzgl. $\lambda_1$, und $0$ erfüllen

    \begin{align*}
        A x = \lambda_1 x
        \quad
        \text{d.h.}
        \quad
        (A - I_N \lambda_1) x = 0.
    \end{align*}

    Weil $P_1^-$ ja die Spektrale Projektion gilt aber $P_1^- x = x \neq 0$, also $x \not \in \ker P_1^- = X$.

    \begin{align*}
        & \implies
        \ker (A - I_N \lambda_1) |_X = \Bbraces{0} \\
        & \implies
        \GL(X) \ni (A - I_N \lambda_1) |_X \cong B_1 \in \GL_{N - (L_1 + 1)}(\C)
    \end{align*}

    Wir verwenden die kanonische Einbettung $\iota: X \hookrightarrow \C^N$ und interpretieren daher

    \begin{align*}
        (A - I_N \lambda_1) |_X
        ~\text{als}~
        \begin{pmatrix}
            B_1 & 0 \\ 0 & 0
        \end{pmatrix}
        ~\text{und}~
        (A - I_N \lambda_1) |_X^{-1}
        ~\text{als}~
        \begin{pmatrix}
            B_1^{-1} & 0 \\ 0 & 0
        \end{pmatrix},
    \end{align*}

    ohne die Verkettung mit $\iota$ jedes Mal dazuzuschreiben.

\end{remark}
% \begin{proof}[Beweis (zu Fuß)]

    \phantom{}

    \begin{enumerate}[label = (\roman*)]

        \item Siehe vorheriger Beweis.

        \item Betrachte abermals die JNF.
        Es gibt $\hat W_1 = (\hat w_{1, 1}, \dots, \hat w_{1, L_1})$ linear unabhängig, sodass
        
        \begin{align*}
            \begin{pmatrix}
                \lambda_1 \hat w_{1,   1}^\ast \\
                \vdots                         \\
                \lambda_1 \hat w_{1, L_1}^\ast \\
                \ast
            \end{pmatrix}
            =
            \underbrace
            {
                \begin{pmatrix}
                    J_1 & 0 \\
                    0   & \ast
                \end{pmatrix}
            }_J
            \begin{pmatrix}
                \hat w_{1,   1}^\ast \\
                \vdots               \\
                \hat w_{1, L_1}^\ast \\
                \ast
            \end{pmatrix}
            =
            J T^{-1}
            \stackrel
            {
                \text{JNF}
            }{=}
            T^{-1} A
            =
            \begin{pmatrix}
                \hat w_{1,   1}^\ast \\
                \vdots               \\
                \hat w_{1, L_1}^\ast \\
                \ast
            \end{pmatrix}
            A
            =
            \begin{pmatrix}
                \hat w_{1,   1}^\ast A \\
                \vdots                 \\
                \hat w_{1, L_1}^\ast A \\
                \ast
            \end{pmatrix}.
        \end{align*}

        $\hat W_1$ sind also Links-Eigenvektoren von $A$ zum Eigenwert $\lambda_1$, d.h. Rechts-Eigenvektoren von $A^\ast$ zum Eigenwert $\overline \lambda_1$.
        Weil die algebraische Vielfachheit von $\overline \lambda_1$ bzgl. $A^\ast$ ja $L_1$ ist, bilden diese eine Basis von $\ker (A^\ast - I_N \overline \lambda_1)$.
        Wir setzen an mit

        \begin{align*}
            w_{1, l}
            \stackrel{!}{=}
            \sum_{i=1}^{L_1}
                \alpha_{1, l, i} \hat w_{1, i},
            \quad
            l = 1, \dots, L_1.
        \end{align*}

        Wir formulieren unseren Wunsch an $W_1$ etwas um.

        \begin{align*}
            \iff
            \Forall l, k = 1, \dots, L_1:
                -\delta_{l, k}
                \stackrel{!}{=}
                (v_{1, k}, w_{1, l})_2
                =
                \pbraces
                {
                    v_{1, k},
                    \sum_{i=1}^{L_1}
                        \alpha_{1, l, i} \hat w_{1, i}
                }_2
                =
                \sum_{i=1}^{L_1}
                    \overline \alpha_{1, l, i} (v_{1, k}, \hat w_{1, i})_2
        \end{align*}

        \begin{align*}
            \iff
            \Forall l = 1, \dots, L_1:
                -e_l
                \stackrel{!}{=}
                \overline{M \alpha_{1, l}}
            \iff
                e_l
                =
                \overline e_l
                \stackrel{!}{=}
                -M \alpha_{1, l}
        \end{align*}

        \begin{align*}
            e_l
            :=
            \begin{pmatrix}
                \delta_{l 1} \\ \vdots \\ \delta_{l L_1}
            \end{pmatrix},
            \quad
            \alpha_{1, l}
            :=
            \begin{pmatrix}
                \alpha_{1, l, 1} \\ \vdots \\ \alpha_{1, l, L_1}
            \end{pmatrix}
        \end{align*}

        \begin{multline*}
            M
            :=
            \overline
            {
                \begin{pmatrix}
                    (v_{1,   1}, \hat w_{1, 1})_2 & \cdots & (v_{1,   1}, \hat w_{1, L_1})_2 \\
                    \vdots                        & \ddots & \vdots                          \\
                    (v_{1, L_1}, \hat w_{1, 1})_2 & \cdots & (v_{1, L_1}, \hat w_{1, L_1})_2 \\
                \end{pmatrix}
            }
            =
            \begin{pmatrix}
                (\hat w_{1, 1}, v_{1,   1})_2 & \cdots & (\hat w_{1, L_1}, v_{1,   1})_2 \\
                \vdots                        & \ddots & \vdots                          \\
                (\hat w_{1, 1}, v_{1, L_1})_2 & \cdots & (\hat w_{1, L_1}, v_{1, L_1})_2 \\
            \end{pmatrix} \\
            =
            \begin{pmatrix}
                v_{1,   1}^\ast \hat w_{1, 1} & \cdots & v_{1,   1}^\ast \hat w_{1, L_1} \\
                \vdots                        & \ddots & \vdots                          \\
                v_{1, L_1}^\ast \hat w_{1, 1} & \cdots & v_{1, L_1}^\ast \hat w_{1, L_1} \\
            \end{pmatrix}
            =
            \begin{pmatrix}
                v_{1, 1}^\ast \\ \vdots \\ v_{1, L_1}^\ast
            \end{pmatrix}
            \hat W_1
            =
            V_1^\ast \hat W_1
            \in
            \GL_{L_1}(\C),
        \end{multline*}

        Man beachte dabei, dass $M \in \GL_{L_1}(\C)$, weil $\hat W_1 \in \GL_{L_1}(\C)$ und

        \begin{align*}
            V_1 \in \GL_{L_1}(\C)
            \implies
            \det V_1 \neq 0
            \implies
            0 \neq \overline{\det V_1^\top} = \det V_1^\ast
            \implies
            V_1^\ast \in \GL_{L_1}(\C).
        \end{align*}

        $W_1 = (w_{1, 1}, \dots, w_{1, L_1})$ ist linear unabhängig, also eine Basis, weil

        \begin{align*}
            & \implies
            \alpha_1
            :=
            (\alpha_{1, 1}, \dots, \alpha_{1, L_1})
            =
            -M^{-1} (e_1, \dots, e_{L_1})
            =
            -M^{-1} I_{L_1}
            =
            -M^{-1}
            \in
            \GL_{L_1}(\C) \\
            & \implies
            W_1
            =
            \hat W_1 \alpha_1
            \in
            \GL_{L_1}(\C).
        \end{align*}

        \item Weil ja $\dim \C^\N < \infty$, genügt es, nachzurechnen, dass die rechte Seite eine Rechtsinverse ist.
        
        \begin{align*}
            (A - I_N \lambda)^{-1}
            & \stackrel{!}{=}
            \frac{1}{\lambda - \lambda_1} P_1^+
            +
            R_1(\lambda) \\
            \iff
            I_N
            & =
            (A - I_N \lambda)
            \pbraces
            {
                \frac{1}{\lambda - \lambda_1} P_1^+
                +
                R_1(\lambda)
            } \\
            & =
            (A - I_N \lambda)
            \pbraces
            {
                -\frac{1}{\lambda_1 - \lambda}
                \sum_{l=1}^{L_1}
                    v_{1, l} w_{1, l}^\ast
                +
                R_1(\lambda)
            } \\
            & =
            -\frac{1}{\lambda_1 - \lambda}
            \sum_{l=1}^{L_1}
                (
                    A v_{1, l}
                    -
                    \lambda v_{1, l}
                )
                w_{1, l}^\ast
            +
            (A - I_N \lambda) R_1(\lambda) \\
            & =
            -\frac{1}{\lambda_1 - \lambda}
            \sum_{l=1}^{L_1}
                (\lambda_1 - \lambda) v_{1, l} w_{1, l}^\ast
            +
            (A - I_N \lambda)
            R_1(\lambda) \\
            & =
            -\sum_{l=1}^{L_1}
                v_{1, l} w_{1, l}^\ast
            +
            (A - I_N \lambda)
            R_1(\lambda) \\
            & =
            P_1^-
            +
            (A - I_N \lambda)
            R_1(\lambda) \\
            \stackrel{!}{\iff}
            I_N - P_1^-
            & =
            (A - I_N \lambda) R_1(\lambda)
        \end{align*}

        Das motiviert folgende Definition des Residuums.
        Dabei ist $X$ so wie in \ref{spektrale_projektion}.

        \begin{multline*}
            R_1(\lambda)
            :=
            \begin{cases}
                (A - I_N \lambda)    ^{-1} (I_N - P_1^-), & \lambda \neq \lambda_1, \\
                (A - I_N \lambda) |_X^{-1} (I_N - P_1^-), & \lambda =    \lambda_1, \\
            \end{cases} \\
            \lambda
            \in
            U_1
            :=
            B
            (
                \lambda_1,
                \min
                \Bbraces
                {
                    \abs{\lambda - \lambda^\prime}:
                    \lambda, \lambda^\prime \in \sigma(A),
                    \lambda \neq \lambda^\prime
                }
            )
        \end{multline*}

        Für $\lambda \in U_1 \setminus \Bbraces{\lambda_1}$ ist $R(\lambda)$ wohldefiniert, weil $\lambda$ kein Eigenwert und somit $\ker (A - I_N \lambda) = \Bbraces{0}$, also $A - I_N \lambda$ invertierbar ist.
        $A - I_N \id$ ist holomorph.
        Wegen der Kofaktor-Darstellung der Inversen, als Matrix mit komponentenweise rationalen Polynomen, ist $\lambda \mapsto (A - I_N \lambda)^{-1}$, und auch somit $R_1$, in $\lambda$ holomorph.

        \begin{align*}
            (A - I_N \lambda)^{-1}
            =
            \frac{1}{\chi_A(\lambda)}
            \cof (A - I_N \lambda)
        \end{align*}

        Dies soll nun auch für $\lambda_1$ gelten.
        Laut Bemerkung \ref{semi_inverse}, ist $(A - I_N \lambda_1) |_X^{-1}$, also auch $R_1(\lambda_1)$, wohldefiniert.

        Weil $(A - I_N \lambda_1) |_X^{-1} (I_N - P_1^-) \in \Lin(\C^N, X)$, gilt

        \begin{align*}
            (A - I_N \lambda_1)
            (A - I_N \lambda_1) |_X^{-1}
            (I_N - P_1^-)
            =
            (A - I_N \lambda_1) |_X
            (A - I_N \lambda_1) |_X^{-1}
            (I_N - P_1^-)
            =
            (I_N - P_1^-)
        \end{align*}

        Darauf basierend, machen wir zunächst eine kleine Nebenrechnung, die wir mehrmals einsetzen werden.
        Dabei sollte man eigentlich immer $I_N - P_1^-$ links dazuschreiben.

        \begin{align*}
            (A - I_N \lambda)^{-1}
            -
            (A - I_N \lambda_1) |_X^{-1}
            & =
            (A - I_N \lambda)^{-1}
            \pbraces
            {
                (A - I_N \lambda_1) |_X
                -
                (A - I_N \lambda)
            }
            (A - I_N \lambda_1) |_X^{-1} \\
            & =
            (A - I_N \lambda)^{-1}
            \pbraces
            {
                (A - I_N \lambda_1)
                -
                (A - I_N \lambda)
            }
            (A - I_N \lambda_1) |_X^{-1} \\
            & =
            (A - I_N \lambda)^{-1}
            (\lambda - \lambda_1)
            (A - I_N \lambda_1) |_X^{-1}
        \end{align*}

        Wir versuchen jetzt, $R^\prime(\lambda_1)$ auszurechnen.

        \begin{align*}
            \frac{1}{\lambda - \lambda_1}
            (R(\lambda) - R(\lambda_1))
            & =
            \frac{1}{\lambda - \lambda_1}
            \pbraces
            {
                (A - I_N \lambda)^{-1}
                -
                (A - I_N \lambda_1) |_X^{-1}
            }
            (I_N - P_1^-) \\
            & =
            \frac{1}{\lambda - \lambda_1}
            (A - I_N \lambda)^{-1}
            (\lambda - \lambda_1)
            (A - I_N \lambda_1) |_X^{-1}
            (I_N - P_1^-) \\
            & =
            (A - I_N \lambda)^{-1}
            (A - I_N \lambda_1) |_X^{-1}
            (I_N - P_1^-) \\
            & \xrightarrow[\lambda \to \lambda_1]{!}
            (A - I_N \lambda_1) |_X^{-1}
            (A - I_N \lambda_1) |_X^{-1}
            (I_N - P_1^-)
        \end{align*}

        Letzterer Grenzwert muss noch gerechtfertigt werden.

        \begin{align*}
            &
            \norm[\C^N \to X]
            {
                (A - I_N \lambda)^{-1}
                (A - I_N \lambda_1) |_X^{-1}
                (I_N - P_1^-)
                -
                (A - I_N \lambda_1) |_X^{-1}
                (A - I_N \lambda_1) |_X^{-1}
                (I_N - P_1^-)
            } \\
            & \leq
            \norm[X \to X]
            {
                (A - I_N \lambda)^{-1}
                -
                (A - I_N \lambda_1) |_X^{-1}
            }
            \norm[X \to X]{(A - I_N \lambda_1) |_X^{-1}}
            \norm[\C^N \to X]{I_N - P_1^-} \\
            & \leq
            \underbrace
            {
                \abs{\lambda - \lambda_1}
            }_{
                \xrightarrow[\lambda \to \lambda_1]{} 0
            }
            \underbrace
            {
                \norm[X \to X]{(A - I_N \lambda)^{-1}}
            }_{
                \stackrel{!}{\leq} C
            }
            \norm[X \to X]{(A - I_N \lambda_1) |_X^{-1}}
            \norm[X \to X]{(A - I_N \lambda_1) |_X^{-1}}
            \norm[\C^N \to X]{I_N - P_1^-}
        \end{align*}

        Die Beschränktheit durch $C > 0$, eine Schranke, die von $\lambda$ unabhängig ist, sieht man für $\abs{\lambda - \lambda_1}$ hinreichend klein.

        \begin{multline*}
            \norm[X \to X]{(A - I_N \lambda)^{-1}}
            \leq
            \norm[X \to X]{(A - I_N \lambda_1) |_X^{-1}}
            +
            \norm[X \to X]
            {
                (A - I_N \lambda)^{-1}
                -
                (A - I_N \lambda_1) |_X^{-1}
            } \\
            \leq
            \norm[X \to X]{(A - I_N \lambda_1) |_X^{-1}}
            +
            \abs{\lambda - \lambda_1}
            \norm[X \to X]{(A - I_N \lambda)^{-1}}
            \norm[X \to X]{(A - I_N \lambda_1) |_X^{-1}}           
        \end{multline*}

        \begin{align*}
            \implies
            &
            \norm[X \to X]{(A - I_N \lambda)^{-1}}
            (
                1
                -
                \abs{\lambda - \lambda_1}
                \norm[X \to X]{(A - I_N \lambda_1) |_X^{-1}}
            ) \\
            & =
            \norm[X \to X]{(A - I_N \lambda)^{-1}}
            -
            \abs{\lambda - \lambda_1}
            \norm[X \to X]{(A - I_N \lambda)^{-1}}
            \norm[X \to X]{(A - I_N \lambda_1) |_X^{-1}} \\
            & \leq
            \norm[X \to X]{(A - I_N \lambda_1) |_X^{-1}}
        \end{align*}

        \begin{align*}
            \implies
            \norm[X \to X]{(A - I_N \lambda)^{-1}}
            \leq
            \frac
            {
                \norm[X \to X]{(A - I_N \lambda_1) |_X^{-1}}
            }{
                1
                -
                \abs{\lambda - \lambda_1}
                \norm[X \to X]{(A - I_N \lambda_1) |_X^{-1}}
            }
            \leq
            C
        \end{align*}

    \end{enumerate}

\end{proof}

\begin{corollary} \label{keldysh_multi}

    Sei $\Lambda \subset \C$ ein beschränktes Gebiet und $A \in H(\Lambda, \C^{N \times N})$ holomorph.
    Es existiere ein $\lambda \in \Lambda$, sodass $A(\lambda) \in \GL_N(\C)$, d.h. invertierbar ist.
    Mögen $\lambda_1, \dots, \lambda_k$, paarweise verschiedene Eigenwerte von $A$, die Voraussetzungen von Satz \ref{keldysh_nicht_linear} erfüllen.
    Seien $S_1, \dots, S_k$ die entsprechenden Summen aus \eqref{eq:darstellung_nicht_linear}.

    Dann existiert ein $R \in H(\Lambda, \C^{N \times N})$, sodass $\Forall \lambda \in \Lambda \setminus \Bbraces{\lambda_1, \dots, \lambda_k}:$

    \begin{align*}
        A(\lambda)^{-1}
        =
        \sum_{n=1}^k
            \frac{1}{\lambda - \lambda_n} S_n
        +
        R(\lambda).
    \end{align*}

\end{corollary}

\begin{proof}

    Laut \ref{keldysh_nicht_linear}, existiert für $m = 1, \dots, k$ eine Umgebung $U_m$ von $\lambda_m$ und $R_m \in H(U_m)$ holomorph, sodass $\Forall \lambda \in U_m \setminus \Bbraces{\lambda_m}:$

    \begin{align*}
        A(\lambda)^{-1}
        =
        \frac{1}{\lambda - \lambda_m} P_m^+
        +
        R_m(\lambda).
    \end{align*}

    Die beiden Funktionen $R_\Lambda \in H(\Lambda \setminus \Bbraces{\lambda_1, \dots, \lambda_k})$ und $R_{U_m} \in H(U_m)$ stimmen überein $\Forall \lambda \in U_m \setminus \Bbraces{\lambda_m}:$

    \begin{multline*}
        R_\Lambda(\lambda)
        :=
        A(\lambda)^{-1}
        -
        \sum_{n=1}^k
            \frac{1}{\lambda - \lambda_n} P_n^+
        =
        \frac{1}{\lambda - \lambda_m} P_m^+
        +
        R_m(\lambda)
        -
        \sum_{n=1}^k
            \frac{1}{\lambda - \lambda_n} P_n^+ \\
        =
        R_m(\lambda)
        -
        \sum_{\substack{n = 1 \\ n \neq m}}^k
            \frac{1}{\lambda - \lambda_n} P_n^+
        =:
        R_{U_m}(\lambda).
    \end{multline*}

    Die folgende Funktion $R$ ist also wohldefiniert, $\in H(\Lambda)$ und leistet das Gewünscht.

    \begin{align*}
        R:
        \Lambda \to C^{N \times N}:
        \lambda
        \mapsto
        \begin{cases}
            R_\Lambda(\lambda), & \lambda \in \dom R_\Lambda, \\
            R_{U_m}(\lambda),   & \Exists m = 1, \dots, k: \lambda \in \dom R_{U_m}
        \end{cases}
    \end{align*}

\end{proof}

