\begin{remark} \label{rem:links_rechts_eigen_}

    Wir schließen direkt and die Definition \ref{def:links_rechts_eigen_} an.
    Der Rechts-Eigenvektor $v$ von $A$ zum Eigenwert $\lambda$ ist auch Links-Eigenvektor von $A^\ast$ zum Eigenwert $\overline \lambda$, weil

    \begin{align*}
        v^\ast A^\ast
        =
        (A v)^\ast
        =
        (\lambda v)^\ast
        =
        \overline \lambda v^\ast.
    \end{align*}

    Der Links-Eigenvektor $w$ von $B$ zum Eigenwert $\mu$ ist auch Rechts-Eigenvektor von $B^\ast$ zum Eigenwert $\overline \mu$, weil

    \begin{align*}
        B^\ast w
        =
        (w^\ast B)^\ast
        =
        (\mu w^\ast)^\ast
        =
        \overline \mu w.
    \end{align*}

    $\ker (B^\ast - \overline \mu I_N)$ den Links-Eigenraum von $\mu$ zu nennen, macht also durchaus Sinn.

    Sollte $A$ hermitesch (selbstadjungiert) sein, so sind alle Eigenwerte reell, und damit alle Rechts-Eigenvektoren auch Links-Eigenvektoren, zum selben Eigenwert, und umgekehrt.

\end{remark}

\begin{comment}

    $A^\ast$ ist tatsächlich die Adjungierte von $A$ im Sinne der Funktionalanalysis, weil $\Forall x, y \in \C^N:$

    \begin{align*}
        (A x, y)_2
        =
        y^\ast A x
        =
        (A^\ast y)^\ast x
        =
        (x, A^\ast y)_2.
    \end{align*}

    Nun ist $\overline \lambda$ Eigenwert von $A^\ast$ mit derselben algebraischen Vielfachheit wie $\lambda$, weil

    \begin{align*}
        \chi_{A^\ast}(\lambda)
        =
        \det(A^\ast - \lambda I_N)
        =
        \overline{\det(A - \lambda I_N)^\top}
        =
        \overline{\chi_A(\lambda)}.
    \end{align*}

\end{comment}
