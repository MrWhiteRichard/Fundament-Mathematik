\begin{remark} \label{spektrale_projektion}

    $P_1 := S_1 A^\prime(\lambda_1)$ ist eine Projektion von $\C^N$ auf den (Rechts-)Eigenraum $\ker A(\lambda_1)$.
    Sei nämlich $x \in \C^N$, so gilt

    \begin{align*}
        P_1 x
        =
        \sum_{l=1}^{L_1}
            v_{1, l}
            \underbrace
            {
                w_{1, l}^\ast
                A^\prime(\lambda_1)
                x
            }_{
                \in \C
            }
        \in
        \ker A(\lambda).
    \end{align*}

    Außerdem ist $P_1$ idempotent, weil

    \begin{multline*}
        \Forall x \in \ker A(\lambda_1):
            P_1 x
            =
            \sum_{l=1}^{L_1}
                v_{1, l} w_{1, l}^\ast
                A^\prime(\lambda_1)
                \sum_{k=1}^{L_1}
                    (x, v_{1, k})_2 v_{1, k} \\
            =
            \sum_{l=1}^{L_1}
                \sum_{k=1}^{L_1}
                    (x, v_{1, k})_2
                    v_{1, l}
                    \underbrace
                    {
                        w_{1, l}^\ast
                        A^\prime(\lambda_1)
                        v_{1, k}
                    }_{
                        = \delta_{l, k}
                    }
            =
            \sum_{l=1}^{L_1}
                (x, v_{1, l})_2
                v_{1, l}
            =
            x.
    \end{multline*}

    \begin{comment}

        $P_1^- := -P_1^+$ nennt man die \textit{Spektrale Projektion} auf den Eigenraum $\ker (A - \lambda_1 I_N)$.
        Offensichtlich bildet $P_1^-$ ganz $\C^N$ auf $\ker (A - \lambda_1 I_N)$ ab, ist aber auch idempotent, weil $\Forall x \in \ker (A - \lambda_1 I_N):$

        \begin{align*}
            P_1^- x
            =
            -\sum_{l=1}^{L_1}
                v_{1, l}
                w_{1, l}^\ast
                \sum_{k=1}^{L_1}
                    (x, v_{1, k})_2
                    v_{1, k}
            =
            -\sum_{l=1}^{L_1}
                v_{1, l}
                \sum_{k=1}^{L_1}
                    \underbrace{(v_{1, k}, w_{1, l})_2}_{-\delta_{l, k}}
                    (x, v_{1, k})_2
            =
            \sum_{l=1}^{L_1}
                (x, v_{1, l})_2
                v_{1, l}
            =
            x.
        \end{align*}

        Also ist $P_1^-$ tatsächlich eine Projektion.

        Damit gilt also $ \C^N = X \oplus Y$.

        \begin{align*}
            X & := \ker P_1^- = \ran (I_N - P_1^-) \stackrel{!}{=} \ran (A - \lambda_1 I_N) \xrightarrow{A - \lambda_1 I_N} \ran (A - \lambda_1 I_N)^2 \subseteq \ran (A - \lambda_1 I_N) = X \\
            Y & := \ker (I_N - P_1^-) = \ran P_1^- \stackrel{!}{=} \ker (A - \lambda_1 I_N) \xrightarrow{A - \lambda_1 I_N} \Bbraces{0} \subseteq Y
        \end{align*}

        Das $2$-te \Quote{!} gilt, weil als Spektrale Projektion

        \begin{align*}
            P_1^-: \C^N \to \ker (A - \lambda_1 I_N),
            \quad
            \Forall x \in \ker (A - \lambda_1 I_N):
                x = P_1^- x \in \ran P_1^-.
        \end{align*}

        Das $1$-te \Quote{!} gilt, weil einerseits $\ran (A - \lambda_1 I_N) \subseteq \ran (I_N - P_1^-)$;

        \begin{align*}
            P_1^- (A - \lambda_1 I_N)
            =
            -\sum_{l=1}^{L_1}
                v_{1, l} w_{1, l}^\ast
            (A - \lambda_1 I_N)
            =
            -\sum_{l=1}^{L_1}
                v_{1, l}
                \underbrace
                {
                    (w_{1, l}^\ast A
                    -
                    w_{1, l}^\ast \lambda_1)
                }_{=0}
            =
            0.
        \end{align*}

        und andererseits gilt laut der Rangformel

        \begin{align*}
            \dim \ran (I_N - P_1^-)
            =
            N - \dim \ker (I_N - P_1^-)
            =
            N - \dim \ker (A - \lambda I_N)
            =
            \dim \ran (A - \lambda I_N).
        \end{align*}

        $X$ und $Y$ sind also $(A - \lambda_1 I_N)$-invariant.
        Wir sagen, dass $A - \lambda_1 I_N$ auch $X$- und $Y$-invariant ist.

        Tatsächlich kommutieren $A - \lambda_1 I_N$ und $I_N - P_1^-$.

        \begin{multline*}
            (A - \lambda_1 I_N) (I_N - P_1^-)
            =
            (A - \lambda_1 I_N) - \underbrace{(A - \lambda_1 I_N) P_1^-}_{=0}
            =
            A - \lambda_1 I_N \\
            =
            (A - \lambda_1 I_N) - \underbrace{P_1^- (A - \lambda_1 I_N)}_{=0}
            =
            (I_N - P_1^-) (A - \lambda_1 I_N)
        \end{multline*}
    
    \end{comment}

\end{remark}
