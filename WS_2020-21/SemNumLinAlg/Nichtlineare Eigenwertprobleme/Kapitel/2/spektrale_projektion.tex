\begin{remark} \label{spektrale_projektion}
    
    $P_1^- := -P_1^+$ nennt man die \textit{Spektrale Projektion} auf den Eigenraum von $\lambda_1$.
    Offensichtlich bildet $P_1^-$ ganz $\C^N$ auf den Rechts-Eigenraum von $\lambda_1$ ab, sie ist aber auch idempotent, weil $\Forall x \in \ker(A - I_N \lambda_1):$

    \begin{align*}
        P_1^- x
        =
        -\sum_{l=1}^{L_1}
            v_{1, l}
            w_{1, l}^\ast
            \sum_{k=1}^{L_1}
                (x, v_{1, k})_2
                v_{1, k}
        =
        -\sum_{l=1}^{L_1}
            v_{1, l}
            \sum_{k=1}^{L_1}
                \underbrace{(v_{1, k}, w_{1, l})_2}_{-\delta_{l, k}}
                (x, v_{1, k})_2
        =
        \sum_{l=1}^{L_1}
            (x, v_{1, l})_2
            v_{1, l}
        =
        x.
    \end{align*}

    Also ist $P_1^-$ tatsächlich eine Projektion.
    Damit gilt also $ \C^N = X \oplus Y$.

    \begin{align*}
        Y & := \ker (I_n - P_1^-) = \ran P_1^- \stackrel{!}{=} \Span V_1 = \ker(A - I_N \lambda_1) & \xrightarrow{A - I_N \lambda_1} \quad & \Bbraces{0} \subseteq Y \\
        X & := \ker P_1^- = \ran(I_N - P_1^-) \stackrel{!}{=} \ran(A - I_N \lambda_1)              & \xrightarrow{A - I_N \lambda_1} \quad & \ran(A - I_N \lambda_1) = X
    \end{align*}

    Das erste \Quote{!} gilt, weil als Spektrale Projektion

    \begin{align*}
        P_1^-: \C^N \to \Span V_1,
        \quad
        \Forall l = 1, \dots, L_1:
            v_{1, l} = P_1^- v_{1, l} \in \ran P_1^-;
    \end{align*}

    das zweite \Quote{!} gilt, weil $A - I_N \lambda_1$ und $I_N - P_1^-$ kommutieren.

    \begin{align*}
        P_1^- (A - I_N \lambda_1)
        =
        -\sum_{l=1}^{L_1}
            v_{1, l} w_{1, l}^\ast
        (A - I_N \lambda_1)
        =
        -\sum_{l=1}^{L_1}
            v_{1, l}
            \underbrace
            {
                (w_{1, l}^\ast A
                -
                w_{1, l}^\ast \lambda_1)
            }_0
        =
        0
    \end{align*}

    \begin{multline*}
        (A - I_N \lambda_1) (I_N - P_1^-)
        =
        (A - I_N \lambda_1) - \underbrace{(A - I_N \lambda_1) P_1^-}_0 \\
        =
        (A - I_N \lambda_1) - \underbrace{P_1^- (A - I_N \lambda_1)}_0
        =
        (I_N - P_1^-) (A - I_N \lambda_1)
    \end{multline*}

    $X$ und $Y$ sind also $(A - I_N \lambda_1)$-invariant.
    Wir sagen, dass $A - I_N \lambda_1$ auch $X$- und $Y$-invariant ist.

\end{remark}
