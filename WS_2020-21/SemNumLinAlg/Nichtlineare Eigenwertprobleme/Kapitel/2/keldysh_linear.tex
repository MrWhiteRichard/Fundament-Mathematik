\begin{theorem}[Keldysh, linear] \label{keldysh_linear}

    Sei $\lambda_1$ ein halb-einfacher Eigenwert einer Matrix $A \in \C^{N \times N}$, d.h. geometrische Vielfachheit $L_1^\mathrm{geo}$ und algebraische Vielfachheit $L_1^\mathrm{alg}$ stimmen überein:

    \begin{align*}
        L_1
        :=
        L_1^\mathrm{geo} := \Def(A - \lambda_1 I_N)
        =
        L_1^\mathrm{alg} := \mu_1 = \max \Bbraces{\mu \in \N: (\lambda - \lambda_1)^\mu \mid \chi_A(\lambda)}.
    \end{align*}

    \begin{enumerate}[label = (\roman*)]
        \item Es gibt also eine Orthonormalbasis $V_1 = (v_{1, 1}, \dots, v_{1, L_1})$ von $\ker (A - \lambda_1 I_N)$.
    \end{enumerate}

    Weiters gelten folgende $2$ Analoga zu Satz \ref{keldysh_nicht_linear}.

    \begin{enumerate}[label = (\roman*), start = 2]

        \item Es gibt eine Basis $W_1 = (w_{1, 1}, \dots, w_{1, L_1})$ von $\ker (A^\ast - \overline \lambda_1 I_N)$, sodass

        \begin{align*}
            \Forall l, k = 1, \dots, L_1:
            (v_{1, k}, w_{1, l})_2 = -\delta_{l, k}.
        \end{align*}

        \item Es existiert eine Umgebung $U_1$ von $\lambda_1$ und $R_1 \in H(U_1, \C^{N \times N})$ holomorph, sodass $\Forall \lambda \in U_1 \setminus \Bbraces{\lambda_1}:$

        \begin{align*}
            (A - \lambda I_N)^{-1}
            =
            \frac{1}{\lambda - \lambda_1} S_1
            +
            R_1(\lambda),
            \quad
            S_1
            :=
            \sum_{l=1}^{L_1}
                v_{1, l} w_{1, l}^\ast.
        \end{align*}

    \end{enumerate}

\end{theorem}
