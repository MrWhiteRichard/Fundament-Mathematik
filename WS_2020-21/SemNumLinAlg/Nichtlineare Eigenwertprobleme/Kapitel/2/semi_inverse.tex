\begin{remark} \label{semi_inverse}

    Sei $X := \ker P_1^-$ und $Y := \ker (I_N - P_1^-)$ so wie in Bemerkung \ref{spektrale_projektion}.
    Wir erweitern die Basis $V_1$ von $Y$ zu einer von ganz $\C^N = X \oplus Y$, vermöge einer Basis von $X$.
    Wenn wir bzgl. dieser koordinatisieren, dann bekommen folgende Darstellungen.

    \begin{align*}
        A - I_N \lambda_1
        =
        \begin{pmatrix}
            B_1 & 0 \\
            0   & B_2
        \end{pmatrix},
        \quad
        \Forall x \in X:
            x
            =
            (
                \underbrace
                {
                    x_1, \dots, x_{N - (L_1 + 1)}
                }_{
                    \in \C
                },
                \underbrace
                {
                    0, \dots, 0
                }_{
                    \displaystyle
                    \text{$L_1$-mal}
                }
            )^\top
    \end{align*}

    Die $0$-er in der Block-Matrix treten auf, weil $A - I_N \lambda_1$ ja $X$- und $Y$-invariant ist, und $X \oplus Y = \C^N$.
    Nur (d.h. genau) Eigenvektoren $x$ von $A$ bzgl. $\lambda_1$, und $0$ erfüllen

    \begin{align*}
        A x = \lambda_1 x
        \quad
        \text{d.h.}
        \quad
        (A - I_N \lambda_1) x = 0.
    \end{align*}

    Weil $P_1^-$ ja die Spektrale Projektion gilt aber $P_1^- x = x \neq 0$, also $x \not \in \ker P_1^- = X$.

    \begin{align*}
        & \implies
        \ker (A - I_N \lambda_1) |_X = \Bbraces{0} \\
        & \implies
        \GL(X) \ni (A - I_N \lambda_1) |_X \cong B_1 \in \GL_{N - (L_1 + 1)}(\C)
    \end{align*}

    Wir verwenden die kanonische Einbettung $\iota: X \hookrightarrow \C^N$ und interpretieren daher

    \begin{align*}
        (A - I_N \lambda_1) |_X
        ~\text{als}~
        \begin{pmatrix}
            B_1 & 0 \\ 0 & 0
        \end{pmatrix}
        ~\text{und}~
        (A - I_N \lambda_1) |_X^{-1}
        ~\text{als}~
        \begin{pmatrix}
            B_1^{-1} & 0 \\ 0 & 0
        \end{pmatrix},
    \end{align*}

    ohne die Verkettung mit $\iota$ jedes Mal dazuzuschreiben.

\end{remark}