\begin{theorem}[Keldysh, nicht-linear] \label{thm:keldysh_nicht_linear}

    Sei $\Lambda \subset \C$ ein beschränktes Gebiet und $A \in H(\Lambda, \C^{N \times N})$ holomorph.
    Es existiere ein $\lambda \in \Lambda$, sodass $A(\lambda) \in \GL_N(\C)$.

    Weiter sei $\lambda_1 \in \Lambda$ ein halb-einfacher Eigenwert, d.h. es existiere eine $L_1$-dimensionale Orthonormalbasis aus (Rechts-)Eigenvektoren $v_{1, 1}, \dots, v_{1, L_1}$ von $\ker A(\lambda_1)$.
    Für diese gelte

    \begin{align} \label{eq:keldysh_nicht_linear_1}
        \Forall l = 1, \dots, L_1:
            A^\prime(\lambda_1) v_{1, l}
            \not \in
            \ran A(\lambda_1).
    \end{align}

    Dann existiert eine Basis aus Links-Eigenvektoren $w_{1, 1}, \dots, w_{1, L_1}$ von $\ker A^\ast(\lambda_1)$, sodass

    \begin{align} \label{eq:keldysh_nicht_linear_2}
        \Forall l, k = 1, \dots, L_1:
            w_{1, l}^\ast A^\prime(\lambda_1) v_{1, k} = \delta_{l, k}.
    \end{align}

    Weiterhin existiert eine Umgebung $U_1$ von $\lambda_1$ und $R_1 \in H(U_1, \C^{N \times N})$ holomorph, sodass

    \begin{align} \label{eq:keldysh_nicht_linear_3}
        \Forall \lambda \in U_1 \setminus \Bbraces{\lambda_1}:
            A(\lambda)^{-1}
            =
            \frac{1}{\lambda - \lambda_1} S_1
            +
            R_1(\lambda),
            \quad
            \text{mit}
            \quad
            S_1
            :=
            \sum_{l=1}^{L_1}
                v_{1, l} w_{1, l}^\ast.
    \end{align}

\end{theorem}
