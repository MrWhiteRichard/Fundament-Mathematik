\begin{remark} \label{semi_inverse}

    Sei $f_1$ die lineare Abbildung, die von $A - \lambda_1 I_N$ dargestellt wird.
    Seien $X$ und $Y$ so wie in Bemerkung \ref{rem:spektrale_projektion_linear}.
    Wir wählen eine Basis von $X$ und erweitern diese, vermöge der Basis $V_1$ von $Y$, zu einer von ganz $\C^N = X \oplus Y$.
    Wenn wir bzgl. dieser koordinatisieren, dann bekommen folgende Darstellungen (modulo Basis-Transformation).

    \begin{gather*}
        f_1
        \cong
        \begin{pmatrix}
            B_1 & 0 \\
            0   & 0
        \end{pmatrix},
        \quad
        B_1 \in \C^{(N - L_1) \times (N - L_1)}, \\
        \Forall v \in \C^N:
            \ExistsOnlyOne x \in X:
            \ExistsOnlyOne y \in Y:
                v = x + y, \\
        x
        \cong
        (
            v_1, \dots, v_{N - L_1},
            0, \dots, 0
        )^\top,
        \quad
        y
        \cong
        (
            0, \dots, 0,
            v_{N - L_1 + 1}, \dots, v_N
        )^\top
    \end{gather*}

    Nur (d.h. genau) Eigenvektoren $y \in Y$, und $y = 0$ erfüllen $(A - \lambda_1 I_N) y = 0$.
    Weil $P_1$ ja die Spektrale Projektion ist, gilt aber $P_1 y = y \neq 0$, also $y \not \in \ker P_1 = X$.

    \begin{align*}
        \implies
        \ker f_1 |_X = \Bbraces{0}
        \implies
        \GL(X) \ni f |_X \cong B \in \GL_{N - L_1}(\C)
    \end{align*}

    Wir identifizieren daher (modulo Basis-Transformation)

    \begin{align*}
        (A - \lambda_1 I_N) |_X
        \cong
        \begin{pmatrix}
            B_1 & 0 \\ 0 & 0
        \end{pmatrix},
        \quad
        (A - \lambda_1 I_N) |_X^{-1}
        \cong
        \begin{pmatrix}
            B_1^{-1} & 0 \\ 0 & 0
        \end{pmatrix}.
    \end{align*}

\end{remark}