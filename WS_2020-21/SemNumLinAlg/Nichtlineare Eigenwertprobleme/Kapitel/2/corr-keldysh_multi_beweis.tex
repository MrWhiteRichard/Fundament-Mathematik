\begin{proof}

    Laut \ref{thm:keldysh_nicht_linear}, existiert für $m = 1, \dots, k$ eine Umgebung $U_m$ von $\lambda_m$ und $R_m \in H(U_m, \C^{N \times N})$ holomorph, sodass

    \begin{align*}
        \Forall \lambda \in U_m \setminus \Bbraces{\lambda_m}:
            A(\lambda)^{-1}
            =
            \frac{1}{\lambda - \lambda_m} S_m
            +
            R_m(\lambda).
    \end{align*}

    Die beiden Funktionen $R_\Lambda \in H(\Lambda \setminus \Bbraces{\lambda_1, \dots, \lambda_k}, \C^{N \times N})$ und $R_{U_m} \in H(U_m, \C^{N \times N})$ stimmen auf $U_m \setminus \Bbraces{\lambda_m}$ überein, weil

    \begin{multline*}
        \Forall \lambda \in U_m \setminus \Bbraces{\lambda_m}:
            R_\Lambda(\lambda)
            :=
            A(\lambda)^{-1}
            -
            \sum_{n=1}^k
                \frac{1}{\lambda - \lambda_n} S_n \\
            =
            \frac{1}{\lambda - \lambda_m} S_m
            +
            R_m(\lambda)
            -
            \sum_{n=1}^k
                \frac{1}{\lambda - \lambda_n} S_n
            =
            R_m(\lambda)
            -
            \sum_{\substack{n = 1 \\ n \neq m}}^k
                \frac{1}{\lambda - \lambda_n} S_n
            =:
            R_{U_m}(\lambda).
    \end{multline*}

    Damit können wir den Identitätssatz für holomorphe Funktionen anwenden.
    Die Funktion

    \begin{align*}
        R:
        \Lambda \to C^{N \times N}:
        \lambda
        \mapsto
        \begin{cases}
            R_\Lambda(\lambda), & \lambda \in \Lambda \setminus \Bbraces{\lambda_1, \dots, \lambda_k}, \\
            R_{U_m}(\lambda),   & \Exists m = 1, \dots, k: \lambda \in U_m,
        \end{cases}
    \end{align*}

    ist also wohldefiniert, holomorph auf $\Lambda$, und leistet das Gewünschte.

\end{proof}
