\begin{definition} \label{links_rechts_eigen_}

    Seien $A \in \C^{N \times N}$, $\lambda \in \C$, und $v \in \C^N$, mit $A v = \lambda v$.
    Wir nennen $(\lambda, v)$ ein \textit{Rechts-Eigenpaar} von $A$, $v$ einen \textit{Rechts-Eigenvektor} von $A$ zum Eigenwert $\lambda$, und $\ker (A - I_N \lambda)$ den \textit{Rechts-Eigenraum} von $\lambda$.

    Seien $B \in \C^{N \times N}$, $\mu \in \C$ und $w \in \C$, mit $w^\ast B = \mu w^\ast$.
    Wir nennen $(\mu, w)$ ein \textit{Links-Eigenpaar} von $B$, $w$ einen \textit{Links-Eigenvektor} von $B$ zum Eigenwert $\mu$, $\ker (B^\ast - I_N \overline \mu)$ den \textit{Links-Eigenraum} von $\mu$.

\end{definition}

\begin{remark}
    
    Wir schließen direkt and die Definition \ref{links_rechts_eigen_} an.
    Der Rechts-Eigenvektor $v$ von $A$ zum Eigenwert $\lambda$ ist auch Links-Eigenvektor von $A^\ast$ zum Eigenwert $\overline \lambda$, weil

    \begin{align*}
        v^\ast A^\ast
        =
        (A v)^\ast
        =
        (\lambda v)^\ast
        =
        \overline \lambda v^\ast.
    \end{align*}

    Der Links-Eigenvektor $w$ von $B$ zum Eigenwert $\mu$ ist auch Rechts-Eigenvektor von $B^\ast$ zum Eigenwert $\overline \mu$.
    $\ker (B^\ast - I_N \overline \mu)$ den Links-Eigenraum von $\mu$ zu nennen, macht also durchaus Sinn.

    Sollte $A$ hermitesch (selbstadjungiert) sein, so sind alle Eigenwerte reell, und damit alle Rechts-Eigenvektore auch Links-Eigenvektoren und umgekehrt.

\end{remark}

\begin{comment}


    % ------------------------------------------------

    $A^\ast$ ist tatsächlich die Adjungierte von $A$ im Sinne der Funktionalanalysis, weil $\Forall x, y \in \C^N:$

    \begin{align*}
        (A x, y)_2
        =
        y^\ast A x
        =
        (A^\ast y)^\ast x
        =
        (x, A^\ast y)_2.
    \end{align*}

    Nun ist $\overline \lambda$ Eigenwert von $A^\ast$ mit derselben algebraischen Vielfachheit wie $\lambda$, weil

    \begin{align*}
        \chi_{A^\ast}(\lambda)
        =
        \det(A^\ast - I_N \lambda)
        =
        \overline{\det(A - I_N \lambda)^\top}
        =
        \overline{\chi_A(\lambda)}.
    \end{align*}

\end{comment}