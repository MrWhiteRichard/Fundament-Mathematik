\begin{remark} \label{links_rechts_eigen_}
    
    Sei $(\lambda, v)$ ein Eigenpaar der Matrix $A \in \C^{N \times N}$.
    Wir nennen $v$ einen \textit{Rechts-Eigenvektor} von $A$ zum Eigenwert $\lambda$.
    Dieser besitzt den bekannten \textit{Rechts-Eigenraum} $\ker (A - I_N \lambda)$.

    $A^\ast$ ist tatsächlich die Adjungierte von $A$ im Sinne der Funktionalanalysis, weil $\Forall x, y \in \C^N:$

    \begin{align*}
        (A x, y)_2
        =
        y^\ast A x
        =
        (A^\ast y)^\ast x
        =
        (x, A^\ast y)_2.
    \end{align*}

    Nun ist $\overline \lambda$ Eigenwert von $A^\ast$ mit derselben algebraischen Vielfachheit wie $\lambda$, weil

    \begin{align*}
        \chi_{A^\ast}(\lambda)
        =
        \det(A^\ast - I_N \lambda)
        =
        \overline{\det(A - I_N \lambda)^\top}
        =
        \overline{\chi_A(\lambda)}.
    \end{align*}

    $v$ ist dann auch ein sogenannter \textit{Links-Eigenvektor} von $A^\ast$ zum Eigenwert $\overline \lambda$, weil

    \begin{align*}
        v^\ast A^\ast
        =
        (A v)^\ast
        =
        (\lambda v)^\ast
        =
        \overline \lambda v^\ast.
    \end{align*}

    Sämtliche Links-Eigenvektoren bilden (gemeinsam mit der $0$) den \textit{Links-Eigenraum} von $\lambda$ bzgl. $A$.
    Dieser ist in der Tat ein Unterraum.

\end{remark}
