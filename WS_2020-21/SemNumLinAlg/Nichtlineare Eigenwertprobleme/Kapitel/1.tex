\section{Einführung}

Die herkömmliche (lineare) Formulierung des Eigenwert-Problems dürfte bekannt sein:
Sei $A \in \C^{N \times N}$ eine Matrix.
Suche $\lambda \in \C$ und $v \in \C^N \setminus \Bbraces{0}$, sodass $A v = \lambda v$.

\begin{align*}
    \iff
    B(\lambda) v := 0,
    \quad
    B(\lambda) := A - I_N \lambda
\end{align*}

Dabei bezeichnet $\Ker B(\lambda)$ den Eigenraum von $\lambda$.
In manchen Anwendungen ist die Funktion $B$ jedoch nicht mehr linear oder gar polynomiell.
Herkömmliche Algorithmen zur Suche von Eigenwerten sind jedoch bloß für lineare $B$ konzipiert.

Die wohl naheliegendste Variante, einer nicht-linearen Eigenwert-Suche ist eine Nullstellen-Suche des nicht-linearen charakteristischen Polynoms $\lambda \mapsto \det B(\lambda)$.
Bereits bei linearen Eigenwertproblemen kann man folgende Nachteile erkennen.

\begin{enumerate}[label = \arabic*.]
    \item Die Berechnung der Determinante ist (insbesondere für große Matrizen) überaus kostspielig.
    \item Wir bekommen bloß einen Eigenwert (Nullstelle).
    \item Selbst wenn ein Algorithmus wie das Newton-Verfahren zur Nullstellen-Suche verwendet wird, wird noch immer ein geeigneter Start benötigt.
\end{enumerate}

Die, von \textbf{Beyn} vorgeschlagene, Konturintegral-Methode wird all diese Probleme auf eine sehr elegante Art und Weise umgehen.
Deren Grundbaustein ist folgendes Resultat aus der Komplexen Analysis.