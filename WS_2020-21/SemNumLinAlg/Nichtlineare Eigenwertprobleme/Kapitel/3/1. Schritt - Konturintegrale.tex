\subsection*{1. Schritt: Konturintegrale}

Nachdem es gilt, $D$ zu bestimmen, bieten $A_0$ und $A_1$ einen Vielversprechenden Anfang.

\begin{align} \label{eq:integral_matrizen_resultat}
    \stackrel
    {
        \eqref{eq:integral_matrizen_resultat_compact}
    }{\implies}
    A_0 = V W^\ast \hat V,
    \quad
    A_1 = V D W^\ast \hat V
\end{align}

Diese werden wir im Wesentlichen durch eine komplexe Variante der Summierten Trapezregel approximieren.
Betrachte dazu folgendes Ringgebiet.

\begin{align*}
    U := \Bbraces{z \in \C: R / a_- < |z| < R a_+},
    \quad
    R > 0,
    \quad
    1 < a_- < a_+
\end{align*}

Wir parametrisieren den Kreis $\partial B(0, R) \subset U$.

\begin{align*}
    \gamma: [0, 1) \to \partial B(0, R): t \mapsto R \exp*{2 \pi i t},
    \quad
    \gamma^\prime: t \mapsto 2 \pi i R \exp*{2 \pi i t}
\end{align*}

Wir approximieren ihn durch die $m$-ten Einheitswurzeln.
Diese wählen wir dann auch als Quadraturknoten, gemeinsam mit gleichmäßigen Gewichten.

\begin{align*}
    \omega_m := \exp \frac{2 \pi i}{m},
    \quad
    m \in \N
\end{align*}

Sei $f \in H(U, \C)$ holomorph.

\begin{multline*}
    Q(f)
    :=
    \frac{1}{2 \pi i}
    \Int[|\lambda| = R]
    {
        f(\lambda)
    }{\lambda}
    =
    \frac{1}{2 \pi i}
    \Int[\gamma]
    {
        f(\lambda)
    }{\lambda}
    =
    \frac{1}{2 \pi i}
    \Int[0][1]
    {
        \gamma^\prime(t)
        f(\gamma(t))
    }{t} \\
    =
    \Int[0][1]
    {
        R
        \exp*{2 \pi i t}
        f(R \exp*{2 \pi i t})
    }{t}
    \approx
    \sum_{\nu = 0}^{m-1}
        \frac{1}{m}
        R \exp \frac{2 \pi i \nu}{m}
        f
        \pbraces
        {
            R \exp \frac{2 \pi i \nu}{m}
        }
    =
    \frac{R}{m}
    \sum_{\nu = 0}^{m-1}
        \omega_m^\nu
        f(R \omega_m^\nu)
    =:
    Q_m(f)
\end{multline*}

$f$ repräsentiert eine Komponente der Integranden von $A_0$ und $A_1$.
Diese müssen vorhin passend zum Ursprung translatiert werden, damit über $\ran \gamma$ anstelle von $\Gamma$ integriert werden kann.
Die (überaus hohe) Genauigkeit der Quadraturformel $Q_m$ wird in ... diskutiert.

Um $A(\lambda)^{-1} \hat V$ in den Integranden von \eqref{eq:integral_matrizen} zu bestimmen, werden wir nicht $A(\lambda)^{-1}$ direkt berechnen.
Stattdessen, bestimmen wir eine LU-Zerlegung von $A(\lambda)$ und führen $j$-mal eine Vorwärts-Rückwärts-Substitution durch.
$\Forall i = 1, \dots, j:$

\begin{align*}
    A(\lambda)^{-1} \hat V_i = x
    \iff
    \hat V_i = A(\lambda) x
\end{align*}
