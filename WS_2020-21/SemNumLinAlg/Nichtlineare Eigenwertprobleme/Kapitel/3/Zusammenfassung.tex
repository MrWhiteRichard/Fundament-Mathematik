\subsection*{Zusammenfassung}

Wenn wir in \eqref{eq:integral_matrizen_resultat_compact} explizit für $i = 0, 1$ einsetzen, erhalten wir

\begin{align} \label{eq:integral_matrizen_resultat}
    A_0 = V W^\ast \hat V,
    \quad
    A_1 = V D W^\ast \hat V.
\end{align}

Nun gilt es, $D$, die Diagonal-Matrix aller Eigenwerte, zu bestimmen.
$A_0$ und $A_1$ bieten dazu einen vielversprechenden Anfang.
Unsere Hoffnung besteht nämlich darin, $D$ aus $A_1$ mittels $A_0$ irgendwie zu \Quote{isolieren}.

Dazu müssen $A_0$ und $A_1$ zunächst \Quote{berechnet}, bzw. mit einer Quadraturformel approximiert, werden.
Das motivieren wir im $1$-ten Schritt mit einer komplexen Variante der summierten Trapezregel;
eine genaue Konvergenz-Analyse folgt aber erst in $4$-ten Kapitel.
Wie diese \Quote{Isolation} von $D$ vonstattengehen soll, erklären wir vollständig im 2. Schritt.

Die Konturintegral-Methode ist zunächst oberflächlich im Algorithmus \ref{alg:integral_methode_zusammenfassung} zusammengefasst worden.
Genauere Erklärungen zur $1$-ten und $2$-ten Zeile folgen im $1$-ten und $2$-ten Schritt.
Die genauere Auseinandersetzung mit der $3$-ten Zeile (bzw. dem $3$-ten Schritt) ist nicht Teil dieser Arbeit.
Wir verwenden lediglich den Befehl \texttt{linalg.eigvals}, aus der \texttt{scipy}-Bibliothek.

\begin{algorithm}[H]
	\caption{Integral-Methode}
    \begin{algorithmic}[0]
        \Procedure{Integral-Methode Zusammanfassung}{}
            \State Berechne (bzw. approximiere) $A_0, A_1 \in \C^{N \times j}$;
            \State Berechne und reduziere eine Singulärwert-Zerlegung $A_0 = \tilde V \Sigma \tilde W^\ast$ auf $J$ Singulärwerte;
            \State Berechne Eigenwerte $\lambda_1, \dots, \lambda_k$ der Matrix $\tilde V A_1 \tilde W \Sigma^{-1} \approx D$;
            \State \Return $\lambda_1, \dots, \lambda_k$
		\EndProcedure
    \end{algorithmic}
    \caption{}
    \label{alg:integral_methode_zusammenfassung}
\end{algorithm}
