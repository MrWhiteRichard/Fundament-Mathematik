\subsection*{Zusammenfassung}

Wenn wir in \eqref{eq:integral_matrizen_resultat_compact} explizit für $i = 0, 1$ einsetzen, erhalten wir

\begin{align} \label{eq:integral_matrizen_resultat}
    A_0 = V W^\ast \hat V,
    \quad
    A_1 = V D W^\ast \hat V.
\end{align}

Nun gilt es, $D$, die Diagonal-Matrix aller Eigenwerte, zu bestimmen.
$A_0$ und $A_1$ bieten dazu einen vielversprechenden Anfang.
Unsere Hoffnung besteht nämlich darin, $D$ aus $A_1$ mittels $A_0$ irgendwie zu \Quote{isolieren}.

Dazu müssen $A_0$ und $A_1$ zunächst \Quote{berechnet}, bzw. mit einer Quadraturformel approximiert werden.
Das motivieren wir im ersten Schritt mit einer komplexen Variante der summierten Trapezregel.
Nachdem dieser Schritt der weitaus teuerste ist, folgt eine genaue Konvergenz-Analyse im vierten Kapitel.
Wie diese \Quote{Isolation} von $D$ vonstattengehen soll, erklären wir vollständig im zweiten Schritt.

Die Konturintegral-Methode ist zunächst oberflächlich im Algorithmus \ref{alg:integral_methode_zusammenfassung} zusammengefasst worden.
Genauere Erklärungen zur ersten und zweiten Zeile folgen im ersten und zweiten Schritt.
Die Auseinandersetzung mit der dritten Zeile (bzw. dem dritten Schritt) ist nicht Teil dieser Arbeit.
Wir verwenden lediglich den Befehl \texttt{linalg.eigvals}, aus der \texttt{scipy}-Bibliothek.

\begin{algorithm}[H]
	\caption{Integral-Methode}
    \begin{algorithmic}[0]
        \Procedure{Integral-Methode Zusammenfassung}{}
            \State Berechne (bzw. approximiere) $A_0, A_1 \in \C^{N \times j}$;
            \State Berechne und reduziere eine Singulärwert-Zerlegung $A_0 = \tilde V \Sigma \tilde W^\ast$ auf $J$ Singulärwerte;
            \State Berechne Eigenwerte $\lambda_1, \dots, \lambda_k$ der Matrix $\tilde V A_1 \tilde W \Sigma^{-1} \sim D$;
            \State \Return $\lambda_1, \dots, \lambda_k$
		\EndProcedure
    \end{algorithmic}
    \caption{}
    \label{alg:integral_methode_zusammenfassung}
\end{algorithm}
