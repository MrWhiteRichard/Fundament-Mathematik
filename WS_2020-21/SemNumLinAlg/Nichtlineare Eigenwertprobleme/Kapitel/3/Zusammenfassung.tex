\subsection*{Zusammenfassung}

Wir werden in diesem Kapitel die folgende Heuristik \ref{zusammenfassung} und Algorithmus \ref{alg:integral_methode_zusammenfassung} herleiten.

\begin{heuristics} \label{zusammenfassung}

    Es gelten die Voraussetzungen von Korollar \ref{keldysh_multi}.
    Seien $\hat V \in \C^{N \times j}$ mit $\sum_{n=1}^k L_n =: J \leq j \ll N$, $A_0, A_1$ wie in \eqref{eq:integral_matrizen} und $D$ wie in \eqref{eq:diagonal_matrix}.
    Sei $\tilde V \Sigma \tilde W$ eine reduzierte Singulärwertzerlegung von $A_0$.

    Dann sind $\tilde V A_1 \tilde W \Sigma^{-1} \approx D$ ähnlich.
    Sie besitzen also dieselben Eigenwerte $\lambda_1, \dots, \lambda_k$ von $A$.

\end{heuristics}

\begin{algorithm}[H]
	\caption{Integral-Methode}
    \begin{algorithmic}[1]
        \Procedure{Integral-Methode Zusammanfassung}{}
            \State Berechne $A_0, A_1 \in \C^{N \times j}$;
            \State Berechne und reduziere eine Singulärwert-Zerlegung $A_0 = \tilde V \Sigma \tilde W^\ast$ auf $J$ Singulärwerte;
            \State Berechne Eigenwerte $\lambda_1, \dots, \lambda_k$ der Matrix $\tilde V A_1 \tilde W \Sigma^{-1}$ (z.B. mit QR-Verfahren);
            \State \Return $\lambda_1, \dots, \lambda_k$
		\EndProcedure
    \end{algorithmic}
    \caption{}
    \label{alg:integral_methode_zusammenfassung}
\end{algorithm}
