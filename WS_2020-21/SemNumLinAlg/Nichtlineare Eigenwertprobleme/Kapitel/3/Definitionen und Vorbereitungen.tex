\subsection*{Definitionen und Vorbereitungen}

Es gelten die Bezeichnungen des vorherigen Kapitels.
Sei $\Gamma \subset \Lambda$ zudem eine positiv orientierten Jordan-Kurve (d.h. ein geschlossener Weg).
$\Gamma$ umschließe alle Eigenwerte $\lambda_1, \dots, \lambda_k \not \in \Gamma$, und sonst keine weiteren.

\begin{figure}[!ht]
    \centering
    \begin{tikzpicture}

        \draw [->] (-1, 0) -- (9, 0) node [right] {$\Re$};
        \draw [->] (0, -1) -- (0, 5) node [above] {$\Im$};

        \draw (1,   2)   .. controls (1,   1)   and (2,   0.5) ..
              (3,   0.5) .. controls (5,   0.5) and (5,   1.5) ..
              (7,   1.5) .. controls (8,   1.5) and (8.5, 2)   ..
              (8.5, 3)   .. controls (8.5, 4)   and (8,   4.5) ..
              (7,   4.5) .. controls (6,   4.5) and (6,   4)   ..
              (4,   4)   .. controls (3,   4)   and (1,   3.5) ..
              cycle;
        \draw (7.5, 3) node {$\Lambda$};

        \draw (4, 2.5) ellipse [x radius = 2, y radius = 1];
        \draw (4, 3.5) node {$<$};
        \draw (6, 3.5) node {$\Gamma$};

        \filldraw (3, 2.5) circle (1 pt) node [below right] {$\lambda_1$};
        \draw (4, 2.5) node {$\cdots$};
        \filldraw (5, 2.5) circle (1 pt) node [below right] {$\lambda_k$};

    \end{tikzpicture}    
    \caption{$\Lambda$ Gebiet, $\Gamma$ Jordan-Kurve, $\lambda_1, \dots, \lambda_k$ Eigenwerte}
    \label{fig:gebiet_kurve_ews}
\end{figure}


Sei nun $f \in H(\C, \C)$ holomorph.
Wir wenden die Cauchyche Integralformel und den Cauchychen Integralsatz \cite{KAna} an.
Für $n = 1, \dots, k$, sei dazu $r_n$ hinreichend klein, sodass $B(\lambda_n, r_n) \subset U_n$.

\begin{multline} \label{eq:cauchysche_integral_formel_satz}
    \implies
    \frac{1}{2 \pi i}
    \Int[\Gamma]
    {
        f(\lambda) A(\lambda)^{-1}
    }{\lambda}
    \stackrel
    {
        \ref{keldysh_multi}
    }{=}
    \frac{1}{2 \pi i}
    \Int[\Gamma]
    {
        f(\lambda)
        \pbraces
        {
            \sum_{n=1}^k
                \frac{1}{\lambda - \lambda_n} P_n^+
                +
                R(\lambda)
        }
    }{\lambda} \\
    =
    \sum_{n=1}^k
        \underbrace
        {
            \frac{1}{2 \pi i}
            \Int[B(\lambda_n, r_n)]
            {
                \frac{f(\lambda)}{\lambda - \lambda_n}
            }{\lambda}
        }_{f(\lambda_n)}
        P_n^+
    +
    \frac{1}{2 \pi i}
    \underbrace
    {
        \Int[\Gamma]
        {
            f(\lambda) R(\lambda)
        }{\lambda}
    }_0
    =
    \sum_{n=1}^k
        f(\lambda_n)
        \sum_{l=n}^{L_n}
            v_{n, l} w_{n, l}^\ast
\end{multline}

Diese Formel bildet die Grundlage zur Berechnung der Eigenwerte.
Sei $J$ die Summe aller Eigenraum-Dimensionen.
\Quote{!} würde für lineare Eigenwertprobleme gelten.

\begin{align*}
    J
    :=
    \sum_{n=1}^k
        L_n
    \stackrel{!}{=}
    \sum_{n=1}^k
        \Def(A - I_N \lambda_n)
    =
    \dim
    \bigoplus_{n=1}^k
        \ker (A - I_N \lambda_n)
    \ll
    \dim \C^N
    =
    N
\end{align*}

Wir vereinigen die Basen all jener (Rechts)-Eigenräume.

\begin{multline*}
    V
    :=
    (V_1, \dots, V_k)
    =
    (v_{1, 1}, \dots, v_{1, L_1}, \dots, v_{k, 1}, \dots, v_{k, L_k}) \\
    =
    \begin{pmatrix}
        v_{1, 1, 1} & \cdots & v_{1, L_1, 1} & \cdots & v_{k, 1, 1} & \cdots & v_{k, L_k, 1} \\
        \vdots      & \ddots & \vdots        & \ddots & \vdots      & \ddots & \vdots        \\
        v_{1, 1, N} & \cdots & v_{1, L_1, N} & \cdots & v_{k, 1, N} & \cdots & v_{k, L_k, N}
    \end{pmatrix}
    \in
    \C^{N \times J}
\end{multline*}

Dasselbe machen wir für die Links-Eigenräume.

\begin{multline*}
    W
    :=
    (W_1, \dots, W_k)
    =
    (w_{1, 1}, \dots, w_{1, L_1}, \dots, w_{k, 1}, \dots, w_{k, L_k}) \\
    =
    \begin{pmatrix}
        w_{1, 1, 1} & \cdots & w_{1, L_1, 1} & \cdots & w_{k, 1, 1} & \cdots & w_{k, L_k, 1} \\
        \vdots      & \ddots & \vdots        & \ddots & \vdots      & \ddots & \vdots        \\
        w_{1, 1, N} & \cdots & w_{1, L_1, N} & \cdots & w_{k, 1, N} & \cdots & w_{k, L_k, N}
    \end{pmatrix}
    \in
    \C^{N \times J}
\end{multline*}

Sei $\hat V \in \C^{N \times j}$ mit $J \leq j \ll N$.

\begin{align} \label{eq:integral_matrizen}
    A_0 := \frac{1}{2 \pi i} \Int[\Gamma]{\lambda^0 A(\lambda)^{-1} \hat V}{\lambda} \in \C^{N \times j},
    \quad
    A_1 := \frac{1}{2 \pi i} \Int[\Gamma]{\lambda^1 A(\lambda)^{-1} \hat V}{\lambda} \in \C^{N \times j}
\end{align}

Sei $D$ die Diagonal-Matrix der Eigenwerte, ihrer Vielfachheit nach aufgeführt.

\begin{align} \label{eq:diagonal_matrix}
    D
    :=
    \diag
    (
        \underbrace
        {
            \lambda_1, \dots, \lambda_1
        }_{
            \displaystyle
            \text{$L_1$-viele}
        },
        \dots,
        \underbrace
        {
            \lambda_k, \dots, \lambda_k
        }_{
            \displaystyle
            \text{$L_k$-viele}
        }
    )
    =
    \diag(I_{L_1} \lambda_1, \dots, I_{L_k} \lambda_k)
    \in
    \C^{N \times N}
\end{align}

Das erstere Ergebnis lässt uns nun die Matrix-Integrale $A_0$ und $A_1$ wir folgt umformulieren.
Letztere Gleichheit ist dabei eine elementare, sperrige Rechnung.

\begin{align} \label{eq:integral_matrizen_resultat_compact}
    A_i
    \stackrel
    {
        \eqref{eq:integral_matrizen}
    }{=}
    \frac{1}{2 \pi i}
    \Int[\Gamma]
    {
        \lambda^i
        A(\lambda)^{-1}
        \hat V
    }{\lambda}
    \stackrel
    {
        \eqref{eq:integral_matrizen_resultat}
    }{}
    \sum_{n=1}^k
        \lambda_n^i
        \sum_{l=1}^{L_n}
            v_{n, l} w_{n, l}^\ast
    \hat V
    \stackrel{!}{=}
    V D^i W^\ast \hat V,
    \quad
    i = 0, 1
\end{align}
