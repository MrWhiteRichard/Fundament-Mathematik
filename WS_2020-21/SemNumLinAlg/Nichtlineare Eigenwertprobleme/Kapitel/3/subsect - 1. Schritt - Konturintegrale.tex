\subsection*{1. Schritt: Konturintegrale}

Betrachte folgendes Ringgebiet

\begin{align*}
    U := \Bbraces{z \in \C: R / a_- < |z| < R a_+},
    \quad
    \text{mit}
    \quad
    R > 0,
    \quad
    1 < a_- < a_+.
\end{align*}

Wir parametrisieren den Kreis $\partial B(0, R) \subset U$ vermöge dem komplexen Weg

\begin{align*}
    \gamma: [0, 1) \to \partial B(0, R): t \mapsto R \exp(2 \pi i t),
    \quad
    \text{mit}
    \quad
    \gamma^\prime: t \mapsto 2 \pi i R \exp(2 \pi i t).
\end{align*}

Wir approximieren ihn durch die $m$-ten Einheitswurzeln.
Diese wählen wir dann auch als Quadraturknoten, gemeinsam mit gleichmäßigen Gewichten (bzw. Gewicht) $1 / m$.
In Formeln geschrieben, lauten diese

\begin{align*}
    \omega_m^0, \dots, \omega_m^{m - 1},
    \quad
    \text{mit}
    \quad
    \omega_m := \exp \pbraces{\frac{2 \pi i}{m}},
    \quad
    m \in \N.
\end{align*}

Sei $f \in H(U, \C)$ holomorph.
Wir verwenden nun tatsächlich eine klassische summierte Rechtecksregel und erhalten

\begin{align*}
    Q(f)
    & :=
    \frac{1}{2 \pi i}
    \Int[|\lambda| = R]
    {
        f(\lambda)
    }{\lambda}
    =
    \frac{1}{2 \pi i}
    \Int[\gamma]
    {
        f(\lambda)
    }{\lambda}
    =
    \frac{1}{2 \pi i}
    \Int[0][1]
    {
        \gamma^\prime(t)
        f(\gamma(t))
    }{t} \\
    & =
    \Int[0][1]
    {
        R
        \exp(2 \pi i t)
        f(R \exp(2 \pi i t))
    }{t}
    \stackrel{!}{\approx}
    \sum_{\nu = 0}^{m-1}
        \frac{1}{m}
        R \exp \pbraces{\frac{2 \pi i \nu}{m}}
        f
        \pbraces
        {
            R \exp \pbraces{\frac{2 \pi i \nu}{m}}
        } \\
    & =
    \frac{R}{m}
    \sum_{\nu = 0}^{m-1}
        \omega_m^\nu
        f(R \omega_m^\nu)
    =:
    Q_m(f).
\end{align*}

Für $f$ werden wir später dann die Integranden von $A_0$ und $A_1$ einsetzen.
Diese müssen vorhin aber noch passend zum Ursprung translatiert werden, damit über $\ran \gamma$ anstelle von $\Gamma$ integriert (bzw. quadriert) werden kann.
Die (überaus hohe) Genauigkeit der Quadraturformel $Q_m$ wird im $4$-ten Kapitel diskutiert.

Um $A(\lambda)^{-1} \hat V$ in den Integranden von $A_0$ und $A_1$ zu bestimmen, werden wir nicht $A(\lambda)^{-1}$ direkt berechnen.
Stattdessen, bestimmen wir eine LU-Zerlegung von $A(\lambda)$.
Der Einfachkeit halber sei diese $L U = A(\lambda)$, mit $L$ und $U$ als untere bzw. obere Dreiecks-Matrizen.
Dann führen wir für alle Spalten $\hat v_1, \dots, \hat v_j$ der Zufallsmatrix $\hat V \in \C^{N \times j}$ jeweils eine Vorwärts- und Rückwärts-Substitution durch.
Es gilt ja schließlich

\begin{align*}
    \Forall i = 1, \dots, j:
        A(\lambda)^{-1} \hat v_i = x_i
        \iff
        \hat v_i = A(\lambda) x_i = L U x_i = L y_i,
        \quad
        y_i = U x_i.
\end{align*}

Somit können dann die Spalten $x_1, \dots, x_j$ von $A(\lambda)^{-1} \hat V$ effizient ermittelt werden, ohne $A(\lambda)^{-1}$ explizit auszurechnen.
Um technische Details der LU-Zerlegung, z.B. das Arbeiten mit Permutationsmatrizen, Pivotsuche usw., machen wir uns hierbei aber keine Gedanken.
Wir verwenden lediglich die Befehle \texttt{linalg.lu\_factor} und \texttt{linalg.lu\_solve}, aus der \texttt{scipy}-Bibliothek.
