\subsection*{Voraussetzungen}

\begin{enumerate}[label = \arabic*.]

    \item $\lambda_1, \dots, \lambda_k \in \Lambda$ sind alle Eigenwerte.
    Diese sind zudem jeweils halb-einfach und liegen im Inneren von $\Gamma$, d.h. insbesondere nicht darauf.

    \item $V, W \in \C^{N \times J}$ haben vollen Rang $J$, d.h. deren Spalten linear unabhängig sind.
    Diese Annahme ist sinnvoll.

    $V, W$ bestehen ja schließlich aus Rechts- bzw. Links-Eigenvektoren.
    Die Voraussetzung gilt also zumindest im linearen Fall.
    
    \item $\hat V$ sei eine hinreichend große, gleichverteilt gewählte Zufallsmatrix, mit vollem Rang $j$.
    Diese Annahme ist sinnvoll.

    Anstelle von $\C$, sei dazu $K$ ein endlicher Körper und $\hat V \in K^{j \times j}$.
    Durch Abzählen der möglichen Komponenten der Spalten der regulären Matrizen, kommt man auf folgende Wahrscheinlichkeit.

    \begin{multline*}
        \mathbf{P}(\hat V \in \GL_j(K))
        =
        \frac
        {
            |\GL_j(K)|
        }{
            |K^{j \times j}|
        }
        =
        \frac{1}
        {
            |K|^{j \cdot j}
        }
        \prod_{i=1}^j
            \pbraces
            {
                |K|^j - |K|^{i-1}
            } \\
        =
        \prod_{i=1}^j
            \frac
            {
                |K|^j - |K|^{i-1}
            }{
                |K|^j
            }
        =
        \prod_{i=1}^j
            \pbraces
            {
                1 - \frac{1}{|K|^{j + 1 - i}}
            }
        \xrightarrow{|K| \to |\C|}
        1
    \end{multline*}

\end{enumerate}
