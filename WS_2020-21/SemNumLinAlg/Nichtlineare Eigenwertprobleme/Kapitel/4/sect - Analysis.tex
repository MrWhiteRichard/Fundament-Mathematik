\chapter{Analyse des Algorithmus}

In diesem Kapitel untersuchen wir nun den teuersten Schritt des Verfahrens: \\
Die Approximation von $A_0$ und $A_1$ durch eine geeigente Quadraturformel.


\section{Konvergenz der Quadraturformel}

In diesem Abschnitt werden wir die exponentielle Konvergenz der Quadraturformel
auf einem Kreis für holomorphe Funktionen auf einem Gebiet rund um den Kreis nachweisen.
\begin{lemma} \label{lem:quadratur_konvergenz}

    Sei $R > 0$, $1 < a_- < a_+$ und $U$ ein Ringgebiet.

    \begin{align*}
        U := \Bbraces{z \in \C: R / a_- < \abs{z} < R a_+}
    \end{align*}

    Weiter sei $f \in H(U, \C^N)$ eine holomorphe Funktion und

    \begin{align*}
        Q(f)
        :=
        \frac{1}{2 \pi i}
        \Int[\abs{\lambda} = R]{f(\lambda)}{\lambda}.
    \end{align*}

    Dann gilt für die summierte Rechtecksregel (hier gleichbedeutend mit einer summierten Trapezregel)

    \begin{align*}
        Q_m(f)
        :=
        \frac{R}{m}
        \sum_{\nu = 0}^{m - 1}
            \omega_m^\nu
            f(R \omega_m^\nu),
        \quad
        \omega_m
        :=
        \exp \frac{2 \pi i}{m},
    \end{align*}

    mit $m \in \N$ Quadraturknoten für alle $\rho_\pm \in (1, a_\pm)$ die Fehlerabschätzung

    \begin{align*}
        \abs{Q_m(f) - Q(f)}
        \leq
        \max_{\abs{\lambda} = R \rho_+}
            \norm{f(\lambda)}
        \frac{\rho_+^{-m}}{1 - \rho_+^{-m}}
        +
        \max_{\abs{\lambda} = R \rho_-}
            \norm{f(\lambda)}
        \frac{\rho_-^{-m}}{1 - \rho_-^{-m}}.
    \end{align*}

\end{lemma}

\begin{proof}

    Wir können $f$ in eine Laurent-Reihe entwickeln, d.h.

    \begin{align*}
        f(\lambda) = \sum_{\mu = - \infty}^\infty \lambda^\mu f_\mu,
        \quad
        \text{mit}
        \quad
        f_\mu := \frac{1}{2\pi i} \Int[|\lambda| = R]{\lambda^{-\mu - 1} f(\lambda)}{\lambda}.
    \end{align*}

    Aufgrund der Holomorphie von $f$ konvergiert die Reihe für alle $\lambda$ gleichmäßig in einer kompakten Teilmenge von $D$.
    Betrachte die Parametrisierung

    \begin{align*}
        \gamma:
        [0,1] \to \partial B_R^\C(0), \quad
        t \mapsto R\exp(2\pi i t).
    \end{align*}

    Wir berechnen für das Integral

    \begin{align*}
        Q(\lambda \to \lambda^\mu)
        & =
        \frac{1}{2 \pi i}
        \Int[\abs{\lambda} = R]{\lambda^\mu}{\lambda}
         =
        \frac{1}{2 \pi i}
        \Int[\gamma]{\lambda^\mu}{\lambda}
         =
        \frac{1}{2 \pi i}
        \Int[0][1]{\gamma^\prime(t) \gamma(t)^\mu}{t} \\ \\
        & =
        R
        \Int[0][1]{\exp(2 \pi i t)(R \exp(2 \pi i t))^\mu}{t} \\
        & =
        R^{\mu + 1}
        \Int[0][1]{\exp(2\pi i t(\mu + 1))}{t} \\
        & =
        \begin{cases}
            R^{\mu + 1}, & \mu + 1 = 0, \\
            0,           & \text{sonst},
        \end{cases}
    \end{align*}

    sowie die Quadratur

    \begin{align*}
        Q_m(\lambda \to \lambda^\mu)
        &=
        \frac{R}{m}
        \sum_{\nu = 0}^{m - 1}
            \omega_m^\nu (R \omega_m^\nu)^\mu
        =
        \frac{R^{\mu + 1}}{m}
        \sum_{\nu = 0}^{m - 1}
            \exp \pbraces{\frac{2\pi i}{m}(\mu + 1)}^v \\
        &=
        \begin{cases}
            R^{\mu + 1},
            & \mu + 1 \in m \Z \\
            \frac{R^{\mu + 1}}{m}
            \frac
            {
                1 - \exp \pbraces{2 \pi i(\mu + 1)}
            }{
                1 - \exp \pbraces{\frac{2\pi i}{m}(\mu + 1)}
            }
            =
            0,
            & \text{sonst}.
        \end{cases}
    \end{align*}

    Zusammengesetzt ergibt das

    \begin{align*}
        Q_m(\lambda \to \lambda^\mu) - Q(\lambda \to \lambda^\mu)
        =
        \begin{cases}
            R^{\mu + 1}, & \mu + 1 \in m \Z \setminus \Bbraces{0}, \\
            0,           & \text{sonst}.
        \end{cases}
    \end{align*}

    Mit der Linearität und Stetigkeit des Integrals, sowie der Quadraturformel erhalten wir somit

    \begin{align*}
        Q_m(f) - Q(f)
        =
        \sum_{l=1}^\infty
            (f_{l m}R^{l m} + f_{-l m} R^{-l m}).
    \end{align*}

    Zur Berechnung der Koeffizienten können wir, mit Hilfe des Cauchy'schen Integralsatzes, die Kurve über die integriert wird innerhalb von $D$ beliebig verändern.
    Damit berechnen wir

    \begin{align*}
        |f_{l m}R^{l m}|
        & =
        \abs
        {
            \frac{R^{l m}}{2\pi i}
            \Int[|\lambda| = \rho^+R]{\lambda^{-l m - 1} f(\lambda)}{\lambda}
        } \\
        & =
        \abs
        {
            R^{l m+1} \rho_+
            \Int[0][1]
            {
                \exp(2\pi i t)
                (R \rho_+ \exp(2 \pi i t))^{-l m -1} f(\exp(2 \pi i t))
            }{t}
        } \\
        & =
        \abs
        {
            \rho_+^{-l m}
            \Int[0][1]
            {
                \exp(2 \pi i t)^{-l m} f(\exp(2\pi i t))
            }{t}
        }
        \leq
        \max_{|\lambda| = \rho^+R}
            \norm{f(\lambda)}
            \rho_+^{-l m}
    \end{align*}

    und analog

    \begin{align*}
        |f_{-l m}R^{-l m}|
        \leq
        \max_{|\lambda| = \rho_-R}
            \norm{f(\lambda)}
            \rho_-^{-l m}.
    \end{align*}

    Daraus erhalten wir zusammen

    \begin{align*}
        |Q_m(f) - Q(f)|
        &\leq
        \sum_{l=1}^\infty
            \pbraces
            {
                \max_{|\lambda| = \rho^+R}
                    \norm{f(\lambda)}
                    \rho_+^{-l m}
                +
                \max_{|\lambda| = \rho_-R}
                    \norm{f(\lambda)}
                    \rho_-^{-l m}
            } \\
        &=
        \max_{\abs{\lambda} = R \rho_+}
            \norm{f(\lambda)}
        \frac{\rho_+^{-m}}{1 - \rho_+^{-m}}
        +
        \max_{\abs{\lambda} = R \rho_-}
            \norm{f(\lambda)}
        \frac{\rho_-^{-m}}{1 - \rho_-^{-m}}.
    \end{align*}

\end{proof}


Für weitergehende Konvergenzanalyse zitieren wir aus \cite{EWPs} noch ohne Beweis  folgenden, auf Lemma \ref{lem:quadratur_konvergenz} aufbauenden Satz.
\begin{theorem} \label{thm:quadratur_konvergenz}

    Werden die Kurven in der Definition \eqref{eq:integral_matrizen} von $A_0$ und $A_1$ als Kreise der Form $z_0 + R \exp*{i t}$ mit $t \in [0, 2 \pi)$ gewählt und durch eine Rechtecksregel mit $m$ äquidistant verteilten Punkten diskretisiert, so erhalten wir für die angenäherten Matrizen $A_{0, 1}^{(m)} \in \C^{N \times j}$ die Fehlerabschätzung

    \begin{align*}
        \norm{A_{0, 1}^{(m)} - A_{0, 1}}
        \leq
        C (\rho_-^{m - r + 1} + \rho_+^{m - r + 1}),
    \end{align*}

    wobei $r$ die maximale Ordnung der Pole von $A(\lambda)^{-1}$ und

    \begin{align*}
        \rho_-
        :=
        \max_{\substack{\lambda \in \sigma(A) \\ \abs{\lambda - z_0} < R}}
            \frac{\abs{\lambda - z_0}}{R},
        \quad
        \rho_+
        :=
        \max_{\substack{\lambda \in \sigma(A) \\ \abs{\lambda - z_0} > R}}
            \frac{R}{\abs{\lambda - z_0}}
    \end{align*}

    ist.

\end{theorem}


Als unmittelbare Folgerung dieser Resultate sehen wir, dass die Konvergenzrate
unserer Quadraturformel wesentlich von der Distanz des nähesten Eigenwert zu
unserer Kurve abhängt. Daher ist es wohl von Vorteil im Algorithmus die Kurve
über die integriert wird mit gewissen Abstand zu den Eigenwerten zu wählen
und gegebenfalls anzupassen.

\section{Aufwand}

Trotz der exponentiellen Konvergenzrate der Quadratur, ist die Berechnung der angenäherten
Matrizen $A_{0, 1}^{(m)}$ relativ aufwändig.

Für jeden Quadraturpunkt und jede Spalte von $\hat V$ muss ein lineares Gleichungssystem gelöst werden.
Dies entspricht somit $m j$ linearen Gleichungssystemen der Größe $N \times N$.
Verwendet man eine LU-Zerlegung von $A(\lambda)$ und nehmen wir für diese den Aufwand $\Landau(N^3)$ an, so erhalten wir in etwa einen Gesamtaufwand zur Berechnung von $A_0^{(m)}$ und $A_1^{(m)}$ von $\Landau(m (N^3 + 2 N^2 j))$.
