\section{Aufwand}

Trotz der exponentiellen Konvergenzrate der Quadratur, ist die Berechnung der angenäherten Matrizen $A_{0, 1}^{(m)}$ relativ aufwändig.

Für jeden Quadraturpunkt und jede Spalte von $\hat V$ muss ein lineares Gleichungssystem gelöst werden.
Dies entspricht somit $m j$ linearen Gleichungssystemen der Größe $N \times N$.
Verwendet man eine LU-Zerlegung von $A(\lambda)$ und nehmen wir für diese den Aufwand $\Landau(N^3)$ an, so erhalten wir in etwa einen Gesamtaufwand zur Berechnung von $A_0^{(m)}$ und $A_1^{(m)}$ von $\Landau(m (N^3 + 2 N^2 j))$.
