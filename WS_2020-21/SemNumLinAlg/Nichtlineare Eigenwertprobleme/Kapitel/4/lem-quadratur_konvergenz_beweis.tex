\begin{proof}

    Wir können $f$ in eine Laurent-Reihe entwickeln, d.h.

    \begin{align*}
        f(\lambda) = \sum_{\mu = - \infty}^\infty \lambda^\mu f_\mu,
        \quad
        \text{mit}
        \quad
        f_\mu := \frac{1}{2\pi i} \Int[|\lambda| = R]{\lambda^{-\mu - 1} f(\lambda)}{\lambda}.
    \end{align*}

    Aufgrund der Holomorphie von $f$ konvergiert die Reihe für alle $\lambda$ gleichmäßig in einer kompakten Teilmenge von $D$.
    Betrachte die Parametrisierung

    \begin{align*}
        \gamma:
        [0,1] \to \partial B_R^\C(0), \quad
        t \mapsto R\exp(2\pi i t).
    \end{align*}

    Wir berechnen für das Integral

    \begin{align*}
        Q(\lambda \to \lambda^\mu)
        & =
        \frac{1}{2 \pi i}
        \Int[\abs{\lambda} = R]{\lambda^\mu}{\lambda}
         =
        \frac{1}{2 \pi i}
        \Int[\gamma]{\lambda^\mu}{\lambda}
         =
        \frac{1}{2 \pi i}
        \Int[0][1]{\gamma^\prime(t) \gamma(t)^\mu}{t} \\ \\
        & =
        R
        \Int[0][1]{\exp(2 \pi i t)(R \exp(2 \pi i t))^\mu}{t} \\
        & =
        R^{\mu + 1}
        \Int[0][1]{\exp(2\pi i t(\mu + 1))}{t} \\
        & =
        \begin{cases}
            R^{\mu + 1}, & \mu + 1 = 0, \\
            0,           & \text{sonst},
        \end{cases}
    \end{align*}

    sowie die Quadratur

    \begin{align*}
        Q_m(\lambda \to \lambda^\mu)
        &=
        \frac{R}{m}
        \sum_{\nu = 0}^{m - 1}
            \omega_m^\nu (R \omega_m^\nu)^\mu
        =
        \frac{R^{\mu + 1}}{m}
        \sum_{\nu = 0}^{m - 1}
            \exp \pbraces{\frac{2\pi i}{m}(\mu + 1)}^v \\
        &=
        \begin{cases}
            R^{\mu + 1},
            & \mu + 1 \in m \Z \\
            \frac{R^{\mu + 1}}{m}
            \frac
            {
                1 - \exp \pbraces{2 \pi i(\mu + 1)}
            }{
                1 - \exp \pbraces{\frac{2\pi i}{m}(\mu + 1)}
            }
            =
            0,
            & \text{sonst}.
        \end{cases}
    \end{align*}

    Zusammengesetzt ergibt das

    \begin{align*}
        Q_m(\lambda \to \lambda^\mu) - Q(\lambda \to \lambda^\mu)
        =
        \begin{cases}
            R^{\mu + 1}, & \mu + 1 \in m \Z \setminus \Bbraces{0}, \\
            0,           & \text{sonst}.
        \end{cases}
    \end{align*}

    Mit der Linearität und Stetigkeit des Integrals, sowie der Quadraturformel erhalten wir somit

    \begin{align*}
        Q_m(f) - Q(f)
        =
        \sum_{l=1}^\infty
            (f_{l m}R^{l m} + f_{-l m} R^{-l m}).
    \end{align*}

    Zur Berechnung der Koeffizienten können wir, mit Hilfe des Cauchy'schen Integralsatzes, die Kurve über die integriert wird innerhalb von $D$ beliebig verändern.
    Damit berechnen wir

    \begin{align*}
        |f_{l m}R^{l m}|
        & =
        \abs
        {
            \frac{R^{l m}}{2\pi i}
            \Int[|\lambda| = \rho^+R]{\lambda^{-l m - 1} f(\lambda)}{\lambda}
        } \\
        & =
        \abs
        {
            R^{l m+1} \rho_+
            \Int[0][1]
            {
                \exp(2\pi i t)
                (R \rho_+ \exp(2 \pi i t))^{-l m -1} f(\exp(2 \pi i t))
            }{t}
        } \\
        & =
        \abs
        {
            \rho_+^{-l m}
            \Int[0][1]
            {
                \exp(2 \pi i t)^{-l m} f(\exp(2\pi i t))
            }{t}
        }
        \leq
        \max_{|\lambda| = \rho^+R}
            \norm{f(\lambda)}
            \rho_+^{-l m}
    \end{align*}

    und analog

    \begin{align*}
        |f_{-l m}R^{-l m}|
        \leq
        \max_{|\lambda| = \rho_-R}
            \norm{f(\lambda)}
            \rho_-^{-l m}.
    \end{align*}

    Daraus erhalten wir zusammen

    \begin{align*}
        |Q_m(f) - Q(f)|
        &\leq
        \sum_{l=1}^\infty
            \pbraces
            {
                \max_{|\lambda| = \rho^+R}
                    \norm{f(\lambda)}
                    \rho_+^{-l m}
                +
                \max_{|\lambda| = \rho_-R}
                    \norm{f(\lambda)}
                    \rho_-^{-l m}
            } \\
        &=
        \max_{\abs{\lambda} = R \rho_+}
            \norm{f(\lambda)}
        \frac{\rho_+^{-m}}{1 - \rho_+^{-m}}
        +
        \max_{\abs{\lambda} = R \rho_-}
            \norm{f(\lambda)}
        \frac{\rho_-^{-m}}{1 - \rho_-^{-m}}.
    \end{align*}

\end{proof}
