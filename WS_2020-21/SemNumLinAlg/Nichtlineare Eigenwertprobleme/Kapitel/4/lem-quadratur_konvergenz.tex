\begin{lemma} \label{lem:quadratur_konvergenz}

    Sei $R > 0$, $1 < a_- < a_+$ und $U$ ein Ringgebiet.

    \begin{align*}
        U := \Bbraces{z \in \C: R / a_- < \abs{z} < R a_+}
    \end{align*}

    Weiter sei $f \in H(U, \C^N)$ eine holomorphe Funktion und

    \begin{align*}
        Q(f)
        :=
        \frac{1}{2 \pi i}
        \Int[\abs{\lambda} = R]{f(\lambda)}{\lambda}.
    \end{align*}

    Dann gilt für die summierte Rechtecksregel (hier gleichbedeutend mit einer summierten Trapezregel)

    \begin{align*}
        Q_m(f)
        :=
        \frac{R}{m}
        \sum_{\nu = 0}^{m - 1}
            \omega_m^\nu
            f(R \omega_m^\nu),
        \quad
        \omega_m
        :=
        \exp \frac{2 \pi i}{m},
    \end{align*}

    mit $m \in \N$ Quadraturknoten für alle $\rho_\pm \in (1, a_\pm)$ die Fehlerabschätzung

    \begin{align*}
        \abs{Q_m(f) - Q(f)}
        \leq
        \max_{\abs{\lambda} = R \rho_+}
            \norm{f(\lambda)}
        \frac{\rho_+^{-m}}{1 - \rho_+^{-m}}
        +
        \max_{\abs{\lambda} = R \rho_-}
            \norm{f(\lambda)}
        \frac{\rho_-^{-m}}{1 - \rho_-^{-m}}.
    \end{align*}

\end{lemma}
