\section{Fazit}

Die Konturintegral-Methode ist ein Verfahren zur Lösung nichtlinearer Eigenwertprobleme, basierend auf dem \Quote{Satz von Keldysh}.
Die zentrale Formel, die dem Algorithmus zu Grunde liegt ist

\begin{gather*}
    \frac{1}{2 \pi i}
    \Int[\Gamma]{f(\lambda) A(\lambda)^{-1}}{\lambda}
    =
    \sum_{n=1}^k
        f(\lambda_n)
        \sum_{l=1}^{L_n}
            v_{n, l} w_{n, l}^\ast.
\end{gather*}

Um den Algorithmus durzuführen, müssen wir die Matrizen

\begin{align*}
    A_0
    & :=
    \frac{1}{2 \pi i}
    \Int[\Gamma]{\lambda^0 A(\lambda)^{-1} \hat V}{\lambda}
    =
    V D^0 W^\ast \hat V
    \in
    \C^{N \times j}, \\
    A_1
    & :=
    \frac{1}{2 \pi i}
    \Int[\Gamma]{\lambda^1 A(\lambda)^{-1} \hat V}{\lambda}
    =
    V D^1 W^\ast \hat V
    \in
    \C^{N \times j}
\end{align*}

mit der Quadratur-Formel

\begin{gather*}
    Q(f)
    :=
    \frac{1}{2 \pi i}
    \Int[|\lambda| = R]{f(\lambda)}{\lambda}
    \approx
    Q_m(f)
    :=
    \frac{R}{m}
    \sum_{\nu = 0}^{m-1}
        \omega_m^\nu f(R \omega_m^\nu), \\
    \omega_m
    :=
    \exp \frac{2 \pi i}{m},
    \quad
    R > 0,
    \quad
    m \in \N,
    \quad
    f \in H(\C),
\end{gather*}

approximieren, die reduzierte Singulärwertzerlegung

\begin{align*}
    A_0
    =
    \tilde V \Sigma \tilde W^\ast
\end{align*}

bestimmen und das lineare Eigenwertproblem der Matrix

\begin{align*}
    \tilde V^\ast A_1 \tilde W \Sigma^{-1}
    \approx
    D
    =
    \operatorname{diag}(\lambda_1, \dots, \lambda_k)
\end{align*}

(beispielsweise mit dem QR-Verfahren) lösen.
Der Algorithmus hat die Vorteile,

\begin{itemize}
    \item dass er eine deutliche Vereinfachung der Lösung von nichtlinearen Eigenwertproblemen bietet,
    \item alle Eigenwerte innerhalb einer gegebenen Kurve berechnet (approximiert),
\end{itemize}

und die Nachteile,

\begin{itemize}
    \item dass ein hoher Berechnungsaufwand erforderlich ist,
    \item und Schwierigkeiten bei der Bestimmung der \Quote{korrekten} Singulärwerte entstehen können.
\end{itemize}
