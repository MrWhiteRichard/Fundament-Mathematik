\chapter{Fazit}

Die Konturintegral-Methode ist ein Verfahren zur Lösung nichtlinearer Eigenwertprobleme, basierend auf dem \Quote{Satz von Keldysh}.
Die zentrale Formel, die dem Algorithmus zu Grunde liegt ist

\begin{gather*}
    \frac{1}{2 \pi i}
    \Int[\Gamma]{f(\lambda) A(\lambda)^{-1}}{\lambda}
    =
    \sum_{n=1}^k
        f(\lambda_n)
        \sum_{l=1}^{L_n}
            v_{n, l} w_{n, l}^\ast.
\end{gather*}

Um den Algorithmus durzuführen, müssen wir die Matrizen

\begin{align*}
    A_0
    & :=
    \frac{1}{2 \pi i}
    \Int[\Gamma]{\lambda^0 A(\lambda)^{-1} \hat V}{\lambda}
    =
    V D^0 W^\ast \hat V
    \in
    \C^{N \times j}, \\
    A_1
    & :=
    \frac{1}{2 \pi i}
    \Int[\Gamma]{\lambda^1 A(\lambda)^{-1} \hat V}{\lambda}
    =
    V D^1 W^\ast \hat V
    \in
    \C^{N \times j}
\end{align*}

mit der Quadratur-Formel

\begin{gather*}
    Q(f)
    :=
    \frac{1}{2 \pi i}
    \Int[|\lambda| = R]{f(\lambda)}{\lambda}
    \approx
    Q_m(f)
    :=
    \frac{R}{m}
    \sum_{\nu = 0}^{m-1}
        \omega_m^\nu f(R \omega_m^\nu), \\
    \omega_m
    :=
    \exp \frac{2 \pi i}{m},
    \quad
    R > 0,
    \quad
    m \in \N,
    \quad
    f \in H(\C),
\end{gather*}

approximieren, die reduzierte Singulärwertzerlegung

\begin{align*}
    A_0
    =
    \tilde V \Sigma \tilde W^\ast
\end{align*}

bestimmen und das lineare Eigenwertproblem der Matrix

\begin{align*}
    \tilde V^\ast A_1 \tilde W \Sigma^{-1}
    \sim
    D
    =
    \operatorname{diag}(\lambda_1, \dots, \lambda_k)
\end{align*}

(beispielsweise mit dem QR-Verfahren) lösen.

Die Integralmethode zur Berechnung der Eigenwerte nichtlinearer Eigenwertprobleme ist vergleichsweise neu und Gegenstand aktueller Forschung.
Die Methode hat die Lösung von nichtlinearen Eigenwertproblemen deutlich vereinfacht,
dennoch birgt sie auch wesentliche Probleme.
Neben dem relativ hohen Aufwand wird durch die Verwendung von $A_{0, 1}^{(m)}$ die Anzahl der nicht-verschwindenden Singulärwerte deutlich erhöht.
Solange die \Quote{korrekten} Singulärwerte um Größenordnungen über den zusätzlichen liegen, können letztere einfach aussortiert werden.
Sind die Approximationen von $A_0, A_1$ nicht genau genug, kann es passieren, dass
die \Quote{korrekten} Singulärwerte kaum mehr von den zusätzlichen Singulärwerten
zu unterscheiden sind, und der Algorithmus eventuell falsche Resultate liefert. \\
Dennoch hat die Integralmethode die Lösung von nichtlinearen Eigenwertproblemen massiv vereinfacht.
Sie ermöglicht bei entsprechender Genauigkeit die zuverlässige Berechnung aller Eigenwerte innerhalb einer gewählten Kurve.
Dies ist ein sehr großer Vorteil zu iterativen Verfahren, bei denen typischerweise die Lage der gefundenen Eigenwerte nicht zuverlässig kontrolliert werden kann.
