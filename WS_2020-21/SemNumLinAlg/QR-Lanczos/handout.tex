\documentclass{article}

% Hier befinden sich Pakete, die wir beinahe immer benutzen ...

\usepackage[utf8]{inputenc}

% Sprach-Paket:
\usepackage[ngerman]{babel}

% damit's nicht so, wie beim Grill aussieht:
\usepackage{fullpage}

% Mathematik:
\usepackage{amsmath, amssymb, amsfonts, amsthm}
\usepackage{bbm}
\usepackage{mathtools, mathdots}

% Makros mit mehereren Default-Argumenten:
\usepackage{twoopt}

% Anführungszeichen (Makro \Quote{}):
\usepackage{babel}

% if's für Makros:
\usepackage{xifthen}
\usepackage{etoolbox}

% tikz ist kein Zeichenprogramm (doch!):
\usepackage{tikz}

% bessere Aufzählungen:
\usepackage{enumitem}

% (bessere) Umgebung für Bilder:
\usepackage{graphicx, subfig, float}

% Umgebung für Code:
\usepackage{listings}

% Farben:
\usepackage{xcolor}

% Umgebung für "plain text":
\usepackage{verbatim}

% Umgebung für mehrerer Spalten:
\usepackage{multicol}

% "nette" Brüche
\usepackage{nicefrac}

% Spaltentypen verschiedener Dicke
\usepackage{tabularx}
\usepackage{makecell}

% Für Vektoren
\usepackage{esvect}

% (Web-)Links
\usepackage{hyperref}

% Zitieren & Literatur-Verzeichnis
\usepackage[style = authoryear]{biblatex}
\usepackage{csquotes}

% so ähnlich wie mathbb
%\usepackage{mathds}

% Keine Ahnung, was das macht ...
\usepackage{booktabs}
\usepackage{ngerman}
\usepackage{placeins}

% special letters:

\newcommand{\N}{\mathbb{N}}
\newcommand{\Z}{\mathbb{Z}}
\newcommand{\Q}{\mathbb{Q}}
\newcommand{\R}{\mathbb{R}}
\newcommand{\C}{\mathbb{C}}
\newcommand{\K}{\mathbb{K}}
\newcommand{\T}{\mathbb{T}}
\newcommand{\E}{\mathbb{E}}
\newcommand{\V}{\mathbb{V}}
\renewcommand{\S}{\mathbb{S}}
\renewcommand{\P}{\mathbb{P}}
\newcommand{\1}{\mathbbm{1}}

% quantors:

\newcommand{\Forall}{\forall \,}
\newcommand{\Exists}{\exists \,}
\newcommand{\ExistsOnlyOne}{\exists! \,}
\newcommand{\nExists}{\nexists \,}
\newcommand{\ForAlmostAll}{\forall^\infty \,}

% MISC symbols:

\newcommand{\landau}{{\scriptstyle \mathcal{O}}}
\newcommand{\Landau}{\mathcal{O}}


\newcommand{\eps}{\mathrm{eps}}

% graphics in a box:

\newcommandtwoopt
{\includegraphicsboxed}[3][][]
{
  \begin{figure}[!h]
    \begin{boxedin}
      \ifthenelse{\isempty{#1}}
      {
        \begin{center}
          \includegraphics[width = 0.75 \textwidth]{#3}
          \label{fig:#2}
        \end{center}
      }{
        \begin{center}
          \includegraphics[width = 0.75 \textwidth]{#3}
          \caption{#1}
          \label{fig:#2}
        \end{center}
      }
    \end{boxedin}
  \end{figure}
}

% braces:

\newcommand{\pbraces}[1]{{\left  ( #1 \right  )}}
\newcommand{\bbraces}[1]{{\left  [ #1 \right  ]}}
\newcommand{\Bbraces}[1]{{\left \{ #1 \right \}}}
\newcommand{\vbraces}[1]{{\left  | #1 \right  |}}
\newcommand{\Vbraces}[1]{{\left \| #1 \right \|}}
\newcommand{\abraces}[1]{{\left \langle #1 \right \rangle}}
\newcommand{\round}[1]{\bbraces{#1}}

\newcommand
{\floorbraces}[1]
{{\left \lfloor #1 \right \rfloor}}

\newcommand
{\ceilbraces} [1]
{{\left \lceil  #1 \right \rceil }}

% special functions:

\newcommand{\norm}  [2][]{\Vbraces{#2}_{#1}}
\newcommand{\diam}  [2][]{\mathrm{diam}_{#1} \: #2}
\newcommand{\diag}  [1]{\mathrm{diag} \: #1}
\newcommand{\dist}  [1]{\mathrm{dist} \: #1}
\newcommand{\mean}  [1]{\mathrm{mean} \: #1}
\newcommand{\erf}   [1]{\mathrm{erf} \: #1}
\newcommand{\id}    [1]{\mathrm{id} \: #1}
\newcommand{\sgn}   [1]{\mathrm{sgn} \: #1}
\newcommand{\supp}  [1]{\mathrm{supp} \: #1}
\newcommand{\arsinh}[1]{\mathrm{arsinh} \: #1}
\newcommand{\arcosh}[1]{\mathrm{arcosh} \: #1}
\newcommand{\artanh}[1]{\mathrm{artanh} \: #1}
\newcommand{\card}  [1]{\mathrm{card} \: #1}
\newcommand{\Span}  [1]{\mathrm{span} \: #1}
\newcommand{\Aut}   [1]{\mathrm{Aut} \: #1}
\newcommand{\End}   [1]{\mathrm{End} \: #1}
\newcommand{\ggT}   [1]{\mathrm{ggT} \: #1}
\newcommand{\kgV}   [1]{\mathrm{kgV} \: #1}
\newcommand{\ord}   [1]{\mathrm{ord} \: #1}
\newcommand{\grad}  [1]{\mathrm{grad} \: #1}
\newcommand{\ran}   [1]{\mathrm{ran} \: #1}
\newcommand{\graph} [1]{\mathrm{graph} \: #1}
\newcommand{\Inv}   [1]{\mathrm{Inv} \: #1}
\newcommand{\pv}    [1]{\mathrm{pv} \: #1}
\newcommand{\GL}    [1]{\mathrm{GL} \: #1}
\newcommand{\Mod}{\mathrm{Mod} \:}
\newcommand{\Th}{\mathrm{Th} \:}
\newcommand{\Char}{\mathrm{char}}
\newcommand{\At}{\mathrm{At}}
\newcommand{\Ob}{\mathrm{Ob}}
\newcommand{\Hom}{\mathrm{Hom}}
\newcommand{\orthogonal}[3][]{#2 ~\bot_{#1}~ #3}
\newcommand{\Rang}{\mathrm{Rang}}
\newcommand{\NIL}{\mathrm{NIL}}
\newcommand{\Res}{\mathrm{Res}}
\newcommand{\lxor}{\dot \lor}
\newcommand{\Div}{\mathrm{div} \:}
\newcommand{\meas}{\mathrm{meas} \:}

% fractions:

\newcommand{\Frac}[2]{\frac{1}{#1} \pbraces{#2}}
\newcommand{\nfrac}[2]{\nicefrac{#1}{#2}}

% derivatives & integrals:

\newcommandtwoopt
{\Int}[4][][]
{\int_{#1}^{#2} #3 ~\mathrm{d} #4}

\newcommandtwoopt
{\derivative}[3][][]
{
  \frac
  {\mathrm{d}^{#1} #2}
  {\mathrm{d} #3^{#1}}
}

\newcommandtwoopt
{\pderivative}[3][][]
{
  \frac
  {\partial^{#1} #2}
  {\partial #3^{#1}}
}

\newcommand
{\primeprime}
{{\prime \prime}}

\newcommand
{\primeprimeprime}
{{\prime \prime \prime}}

% Text:

\newcommand{\Quote}[1]{\glqq #1\grqq{}}
\newcommand{\Text}[1]{{\text{#1}}}
\newcommand{\fastueberall}{\text{f.ü.}}
\newcommand{\fastsicher}{\text{f.s.}}

% -------------------------------- %
% amsthm-stuff:

\theoremstyle{definition}

% numbered theorems
\newtheorem{theorem}{Satz}
\newtheorem{lemma}{Lemma}
\newtheorem{corollary}{Korollar}
\newtheorem{proposition}{Proposition}
\newtheorem{remark}{Bemerkung}
\newtheorem{definition}{Definition}
\newtheorem{example}{Beispiel}

% unnumbered theorems
\newtheorem*{theorem*}{Satz}
\newtheorem*{lemma*}{Lemma}
\newtheorem*{corollary*}{Korollar}
\newtheorem*{proposition*}{Proposition}
\newtheorem*{remark*}{Bemerkung}
\newtheorem*{definition*}{Definition}
\newtheorem*{example*}{Beispiel}

% Please define this stuff in project ("main.tex"):

% \def \lastexercisenumber {...}
% This will be 0 by default

% \setcounter{section}{...}
% This will be 0 by default
% and hence, completely ignored

\ifnum \thesection = 0
{\newtheorem{exercise}{Aufgabe}}
\else
{\newtheorem{exercise}{Aufgabe}[section]}
\fi

\ifdef
{\lastexercisenumber}
{\setcounter{exercise}{\lastexercisenumber}}

\newcommand{\solution}
{
    \renewcommand{\proofname}{Lösung}
    \renewcommand{\qedsymbol}{}
    \proof
}

\renewcommand{\proofname}{Beweis}

% -------------------------------- %
% environment zum einkasteln:

% dickere vertical lines
\newcolumntype
{x}
[1]
{!{\centering\arraybackslash\vrule width #1}}

% environment selbst (the big cheese)
\newenvironment
{boxedin}
{
  \begin{tabular}
  {
    x{1 pt}
    p{\textwidth}
    x{1 pt}
  }
  \Xhline
  {2 \arrayrulewidth}
}
{
  \\
  \Xhline{2 \arrayrulewidth}
  \end{tabular}
}

% -------------------------------- %
% MISC "Ein-Deutschungen"

\renewcommand
{\figurename}
{Abbildung}

\renewcommand
{\tablename}
{Tabelle}

% -------------------------------- %


\parindent 0pt

\begin{document}
\title
{
	Eigenwertberechnung mithilfe des Lanczos-Verfahrens (Handout)
}
\author
{
	Göth Christian
	\and
	Moik Matthias
	\and
	Sallinger Christian
}
\date{\today}
\maketitle


\section{Definitionen, Lemmata und Sätze}

\begin{definition}
	Sei $v_0 \in \K^{N}$ und $A \in \K^{N \times N}$. Dann bezeichnet
	\begin{align*}
		\mathcal{K}_m(A,v_0) := \textbf{span}\{v_0, Av_0, \dots, A^{m-1}v_0\}, \quad m \in \N
	\end{align*}
	den Krylov-Raum von $A$ und $v_0$. Es bezeichne $\mathcal{P}_m: \K^N \to \mathcal{K}_m$ die orthogonale Projektion auf $\mathcal{K}_m$.
\end{definition}

\begin{lemma}
	Sei $\Pi_m$ der Raum der Polynome in einer Veränderlichen mit maximalem Grad $m$. Dann ist $v \in \mathcal{K}_m(A,v_0) \subset \K^N$ genau dann, wenn ein Polynom $p \in \Pi_{m-1}$ existiert mit $v = p(A)v_0$. \newline
	Ist $A$ diagonalisierbar mit Eigenwerten $\lambda_1, \dots, \lambda_N$ und zugehörigen Eigenvektoren $u_1, \dots, u_N$, dann existiert eine eindeutige Darstellung $v_0 = \sum_{j=1}^{N} \alpha_j u_j$ und es gilt
	\begin{align*}
		v \in \mathcal{K}_m \Leftrightarrow \Exists p \in \Pi_{m-1}: v = \sum_{j=1}^{N} p(\lambda_j)\alpha_j u_j.
	\end{align*}
\end{lemma}

\begin{lemma}
	Sei $A \in \C^{N \times N}$ hermitesch bezüglich des Skalarproduktes $(\cdot, \cdot)$, $\lambda_1 \geq \lambda_2 \geq \dots \geq \lambda_N$ die Eigenwerte von $A$ (gemäß Vielfachheit gezählt) und $u_1, \dots, u_N$ die zugehörigen normierten Eigenvektoren. Dann gilt
	\begin{align}\label{lambdamax}
		\lambda_1 = \max_{v \in \C^N \setminus \{0\}} \frac{(Av,v)}{(v,v)}, \quad \lambda_k = \max_{\substack{v \in \C^N \setminus \{0\}\\
    (u_j,v) = 0, j= 1,\dots, k-1}} \frac{(Av,v)}{(v,v)}, \quad k= 2, \dots, N
	\end{align}
	und
	\begin{align}\label{lambdamaxmin}
		\lambda_k = \max_{\substack{S \subset \C^N\\
    \dim S = k}} \min_{v \in S \setminus \{0\}} \frac{(Av,v)}{(v,v)}.
	\end{align}
\end{lemma}

\begin{definition}
	Für $m \in \N$ sind die Chebyshev-Polynome $T_m \in \Pi_m$ definiert druch
	\begin{align}\label{chebyshev}
		T_m(x) := \frac{1}{2}((x+ \sqrt{x^2 - 1})^m + (x- \sqrt{x^2 - 1})^m), \quad x \in \R.
	\end{align}
\end{definition}

Alternative Definitionen der Chebyshev-Polynome:
\begin{align}\label{chebyalt}
	T_(x) := \cos(m \arccos x), \quad x \in [-1,1].
\end{align}

\begin{lemma}
	Sei $[a,b]$ ein nicht-leeres Intervall in $\R$ und sei $c \geq b$. Dann gilt mit $\gamma := 1 + 2 \frac{c-b}{b-a} > 0$
	\begin{align}\label{polminmax}
		\min_{\substack{p \in \Pi_m \\
		p(c) = 1}} \max_{x \in [a,b]} |p(x)| \leq \frac{1}{|T_m(\gamma)|} \leq 2 (\gamma + \sqrt{\gamma^2 -1})^{-m}.
	\end{align}
\end{lemma}

\begin{theorem}[Konvergenz der Eigenwerte hermitscher Matrizen]
	Sei $A \in \K^{n \times n}$ eine hermitesche Matrix mit paarweise verschiedenen Eigenwerten $\lambda_1 > \lambda_2 > \dots > \lambda_n$ und der zugehörigen Orthonormalbasis aus Eigenvektoren $u_1,\dots,u_n$. Für $1 \le m < n$ werden die Eigenwerte der linearen Abbildung $\mathcal{A}_m: \mathcal{K}_m(A,v_0)\rightarrow \mathcal{K}_m(A,v_0)$, die durch $v \rightarrowtail \mathcal{P}_mAv$ gegeben ist,  mit $\lambda_1^{(m)} \ge \lambda_2^{(m)} \ge \dots \ge \lambda_m^{(m)}$ bezeichnet. Dabei ist $v_0$ ein beliebiger Startvektor, der nicht orthogonal zu den ersten $m-1$ Eigenvektoren von $A$ ist. Dann gilt

	\begin{equation}
		0 \le \lambda_i - \lambda_i^{(m)} \le (\lambda_i -\lambda_n) (\tan\theta_i)^2 \kappa_i^{(m)} \left(\frac{1}{T_{m-i}(\gamma_i)}\right)^2, \quad i=1,\dots,m-1
	\end{equation}

		wobei
	\begin{equation*}
		\tan \theta_i \coloneqq \frac{\norm{(\id - \mathcal{P}_{u_i})v_0}}{\norm{ \mathcal{P}_{u_i}v_0}}, \quad \gamma_i \coloneqq 1+2 \frac{\lambda_i-\lambda_{i+1}}{\lambda_{i+1} -\lambda_n}
	\end{equation*}
	
	und
	\begin{equation*}
		\kappa_1^{(m)} \coloneqq 1, \quad \kappa_i^{(m)} \coloneqq \left(\prod_{j=1}^{i-1} \frac{\lambda_j^{(m)} - \lambda_n}{\lambda_j^{(m)} - \lambda_i}\right)^2, \quad i = 2,\dots,m.
	\end{equation*}
\end{theorem}

\end{document}
