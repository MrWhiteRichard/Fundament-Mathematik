\documentclass{article}

\input{../../../../Fundament-LaTeX/packages_de.tex}
\input{../../../../Fundament-LaTeX/macros_de.tex}
% ---------------------------------------------------------------- %
% amsthm-environments:

\theoremstyle{definition}

% numbered theorems
\newtheorem{theorem}             {Satz}[section]
\newtheorem{lemma}      [theorem]{Lemma}
\newtheorem{corollary}  [theorem]{Korollar}
\newtheorem{proposition}[theorem]{Proposition}
\newtheorem{remark}     [theorem]{Bemerkung}
\newtheorem{definition} [theorem]{Definition}
\newtheorem{example}    [theorem]{Beispiel}
\newtheorem{heuristics} [theorem]{Heuristik}

% unnumbered theorems
\newtheorem*{theorem*}    {Satz}
\newtheorem*{lemma*}      {Lemma}
\newtheorem*{corollary*}  {Korollar}
\newtheorem*{proposition*}{Proposition}
\newtheorem*{remark*}     {Bemerkung}
\newtheorem*{definition*} {Definition}
\newtheorem*{example*}    {Beispiel}
\newtheorem*{heuristics*} {Heuristik}

% ---------------------------------------------------------------- %
% exercise- and solution-environments:

\newtheorem{exercise}{Aufgabe}

% if the exercise counter should start at a given exercise number please include the following in the main.tex document
% \setcounter{exercise}{<last exercise number>}

\newenvironment{solution}
{
  \begin{proof}[Lösung]
}{
  \end{proof}
}

% ---------------------------------------------------------------- %
% a tcolorbox-preset designed to mimic the text boxes typically used by Prof. Stefan Hetzl

% starting template:
% https://tex.stackexchange.com/a/527829

% provide box title as optional argument
\newtcolorbox[auto counter]{hetzlbox}[1][]{%
    colback = white,
    coltitle = black,
    fonttitle = \bfseries,
    sharp corners,
    detach title,
    width = 12cm,
    #1,
    code = {\ifdefempty{\tcbtitletext}{}{\tcbset{before upper = {{\centering \tcbtitle \par} \medskip}}}},
    boxrule = 0.5pt
}

% ---------------------------------------------------------------- %
% MISC translations for environment-names

\renewcommand{\proofname} {Beweis}
\renewcommand{\figurename}{Abbildung}
\renewcommand{\tablename} {Tabelle}

% ---------------------------------------------------------------- %


\parskip 0pt
\parindent 0pt

\begin{document}
\title
{
	Eigenwertberechnung mithilfe des Lanczos-Verfahrens (Handout)
}
\author
{
	Göth Christian
	\and
	Moik Matthias
	\and
	Sallinger Christian
}
\date{\today}
\maketitle


\section{Definitionen, Lemmata und Sätze}

\begin{enumerate}
  \item\textbf{Satz}: Die Eigenwerte für das Eigenwertproblem

	\begin{align*}
		\begin{cases}
		-\Delta u = \lambda u \quad &\text{in}~ \Omega, \\
		\frac{\partial u}{\partial \nu} = 0 \quad &\text{auf}~ \partial \Omega,
		\end{cases}
	\end{align*}

	auf dem Rechteck $\Omega := (0,a) \times (0,b)$ sind gegeben durch

	\begin{align}
		\lambda_{n,m}
		=
		\pi^2 \Big(
			\frac{n^2}{a^2} + \frac{m^2}{b^2}
		\Big), \quad n,m \in \N
	\end{align}

	\item\textbf{Definition}: Sei $v_0 \in \C^{N}$ und $A \in \C^{N \times N}$. Dann bezeichnet
	\begin{align*}
		\mathcal{K}_m(A,v_0) := \textbf{span}\{v_0, Av_0, \dots, A^{m-1}v_0\}, \quad m \in \N
	\end{align*}
	den Krylov-Raum von $A$ und $v_0$. Es bezeichne $\mathcal{P}_m: \C^N \to \mathcal{K}_m$ die orthogonale Projektion auf $\mathcal{K}_m$.

	\item\textbf{Lemma}:	Sei $\Pi_m$ der Raum der Polynome in einer Veränderlichen mit maximalem Grad $m$. Dann ist $v \in \mathcal{K}_m(A,v_0) \subset \C^N$ genau dann, wenn ein Polynom $p \in \Pi_{m-1}$ existiert mit $v = p(A)v_0$. \newline
	Ist $A$ diagonalisierbar mit Eigenwerten $\lambda_1, \dots, \lambda_N$ und zugehörigen Eigenvektoren $u_1, \dots, u_N$, dann existiert eine eindeutige Darstellung $v_0 = \sum_{j=1}^{N} \alpha_j u_j$ und es gilt
	\begin{align*}
		v \in \mathcal{K}_m \Leftrightarrow \Exists p \in \Pi_{m-1}: v = \sum_{j=1}^{N} p(\lambda_j)\alpha_j u_j.
	\end{align*}

	\item\textbf{Lemma}: 	Sei $A \in \C^{N \times N}$ hermitesch bezüglich des Skalarproduktes $(\cdot, \cdot)$, $\lambda_1 \geq \lambda_2 \geq \dots \geq \lambda_N$ die Eigenwerte von $A$ (gemäß Vielfachheit gezählt) und $u_1, \dots, u_N$ die zugehörigen normierten Eigenvektoren. Dann gilt
	\begin{align}\label{lambdamax}
		\lambda_1 = \max_{v \in \C^N \setminus \{0\}} \frac{(Av,v)}{(v,v)}, \quad \lambda_k = \max_{\substack{v \in \C^N \setminus \{0\}\\
				(u_j,v) = 0, j= 1,\dots, k-1}} \frac{(Av,v)}{(v,v)}, \quad k= 2, \dots, N
	\end{align}
	und
	\begin{align}\label{lambdamaxmin}
		\lambda_k = \max_{\substack{S \subset \C^N\\
				\dim S = k}} \min_{v \in S \setminus \{0\}} \frac{(Av,v)}{(v,v)}.
	\end{align}

	\item\textbf{Definition}: Für $m \in \N$ sind die Chebyshev-Polynome $T_m \in \Pi_m$ definiert druch
	\begin{align}\label{chebyshev}
		T_m(x) := \frac{1}{2}((x+ \sqrt{x^2 - 1})^m + (x- \sqrt{x^2 - 1})^m), \quad x \in \R.
	\end{align}

	\item\textbf{Definition}: Alternative Definitionen der Chebyshev-Polynome:
	\begin{align}\label{chebyalt}
		T_(x) := \cos(m \arccos x), \quad x \in [-1,1].
	\end{align}

	\item\textbf{Lemma}: Sei $[a,b]$ ein nicht-leeres Intervall in $\R$ und sei $c \geq b$. Dann gilt mit $\gamma := 1 + 2 \frac{c-b}{b-a} > 0$
		\begin{align}\label{polminmax}
			\min_{\substack{p \in \Pi_m \\
					p(c) = 1}} \max_{x \in [a,b]} |p(x)| \leq \frac{1}{|T_m(\gamma)|} \leq 2 (\gamma + \sqrt{\gamma^2 -1})^{-m}.
		\end{align}

	\item\textbf{Satz}: Sei $A \in \C^{n \times n}$ eine hermitesche Matrix mit paarweise verschiedenen Eigenwerten $\lambda_1 > \lambda_2 > \dots > \lambda_n$ und der zugehörigen Orthonormalbasis aus Eigenvektoren $u_1,\dots,u_n$. Für $1 \le m < n$ werden die Eigenwerte der linearen Abbildung $\mathcal{A}_m: \mathcal{K}_m(A,v_0)\rightarrow \mathcal{K}_m(A,v_0)$, die durch $v \rightarrowtail \mathcal{P}_mAv$ gegeben ist,  mit $\lambda_1^{(m)} \ge \lambda_2^{(m)} \ge \dots \ge \lambda_m^{(m)}$ bezeichnet. Dabei ist $v_0$ ein beliebiger Startvektor, der nicht orthogonal zu den ersten $m-1$ Eigenvektoren von $A$ ist. Dann gilt

	\begin{equation}
		\label{konvergenz Eigenwerte}
		0 \le \lambda_i - \lambda_i^{(m)} \le (\lambda_i -\lambda_n) (\tan\theta_i)^2 \kappa_i^{(m)} \left(\frac{1}{T_{m-i}(\gamma_i)}\right)^2, \quad i=1,\dots,m-1
	\end{equation}

	wobei
	\begin{equation*}
		\tan \theta_i \coloneqq \frac{\norm{(\id - \mathcal{P}_{u_i})v_0}}{\norm{ \mathcal{P}_{u_i}v_0}}, \quad \gamma_i \coloneqq 1+2 \frac{\lambda_i-\lambda_{i+1}}{\lambda_{i+1} -\lambda_n}
	\end{equation*}

	und
	\begin{equation*}
		\kappa_1^{(m)} \coloneqq 1, \quad \kappa_i^{(m)} \coloneqq \left(\prod_{j=1}^{i-1} \frac{\lambda_j^{(m)} - \lambda_n}{\lambda_j^{(m)} - \lambda_i}\right)^2, \quad i = 2,\dots,m.
	\end{equation*}
\end{enumerate}

\end{document}
