% --------------------------------------------------------------------------------

\begin{exercise}

Bestimmen Sie für das Randwertproblem
\begin{align*}
  -\frac{d}{dx}\left(k(x)\frac{du}{dx}\right) &= f(x), \quad 0 < x < 1 \\
  u(0) = u(1) &= 0
\end{align*}
mit $k(x) > 0$ für $0 \leq x \leq 1$ eine Funktion $g(x,y)$ (genannt
\textit{greensche Funktion}) sodass
\begin{align*}
  u(x) = \int_0^1g(x,y)f(y)dy
\end{align*}
das Randwertproblem löst. Gehen Sie dazu wie folgt vor:
\begin{enumerate}[label = (\roman*)]
  \item Multiplizieren Sie die Differentialgleichung mit einer Funktion $K(x)$,
  welche das homogene Problem
  \begin{align*}
    -\frac{d}{dx}\left(k(x)\frac{du}{dx}\right) = 0
  \end{align*}
  löst und die Randbedingung bei $x = 1$ erfüllt.
  \item Integrieren Sie die erhaltene Gleichung von $x = 0$ bis $x = 1$, um einen
  Ausdruck für $u^{\prime}(0)$ zu erhalten.
  \item Leiten Sie daraus eine Formel für $u^{\prime}(x)$ und dann $u(x)$ her.
  \item Bringen Sie die Formel für $u(x)$ auf die gewünschte Form.
\end{enumerate}
Überprüfen Sie am Ende, dass
\begin{align*}
  -\frac{d}{dy}\left(k(y)\frac{dg}{dy}(x,y)\right) &= \delta(x-y) \\
  g(x,0) = g(x,1) &= 0
\end{align*}
gilt. \\
\textit{Hinweis:} Die Methode ist im Spezialfall $k(x) = 1$ mit $K(x) = 1 - x$ etwas einfacher.
\end{exercise}

% --------------------------------------------------------------------------------

\begin{solution}

\phantom{}

\end{solution}

% --------------------------------------------------------------------------------
