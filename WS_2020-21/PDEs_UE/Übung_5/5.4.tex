% --------------------------------------------------------------------------------

\begin{exercise}

Betrachten Sie die \textit{Helmholtz-Gleichung}

\begin{align*}
  (\Delta + k^2) u = f ~\text{in}~ \R^3
\end{align*}

mit der Wellenzahl $k > 0$, der Wellenquelle $f \in \mathcal{D}(\R^3)$ und dem Wellenfeld $u: \R^3 \to \C$.

\begin{enumerate}[label = (\roman*)]

  \item Berechnen Sie eine radialsymmetrische Funktion $G(x, \xi) = g(|x - \xi|)$,
  für die $(\Delta_x + k^2)G = \delta_\xi$ gilt, und die der \textit{sommerfeldschen
  Ausstrahlungsbedingung}

  \begin{align*}
    \lim_{r \to \infty} r \pbraces{\pderivative{r} - i k} g(r) = 0
  \end{align*}

  genügt.

  \textit{Anmerkung:}
  Diese Bedingung stellt eine Randbedingung (im Unendlichen) dar, weshalb man $G$ auch eine greensche Funktion nennt.

  \item Stellen Sie die/eine Lösung $u$ der Helmholtz-Gleichung mit Hilfe von $G(x, \xi)$ dar.
  Zeigen Sie, dass für eine radialsymmetrische Funktion $f$ diese Lösung auch radialsymmetrisch ist.

  \item Konstruieren Sie aus $G$ zwei weitere Funktionen $G_{\mathrm{Dir}}$ und $G_{\mathrm{Neu}}$, welche greensche Funktionen auf $\Omega := \R^2 \times \R_+$ sind und auf $\partial\Omega$ homogene Dirichlet- bzw. homogene Neumann-Randbedingunen erfüllen.

\end{enumerate}

\end{exercise}

% --------------------------------------------------------------------------------

\begin{solution}

Da unser Ableitungsoperator nur konstante Koeffizienten hat suchen wir eine Lösung mit Pol in $0$ und verschieben diese dann nach Satz $3.21$ um eine Polstelle in $\xi$ zu erhalten.
Unsere Funktion $g$ soll nun radialsymmetrisch sein mit Radius $r = |x| = \sqrt{x_1^2 + x_2^2 +x_3^2}$.Wir berechnen zuerst für $i \in \{1,2,3\}$

\begin{align*}
  \frac{\partial r}{\partial x_i}
  =
  \frac{x_i}{\sqrt{x_1^2 + x_2^2 + x_3^2}}
  =
  \frac{x_i}{r}
\end{align*}

Dann gilt

\begin{align*}
  \frac{\partial G}{\partial x_i}
  =
  g^\prime (r) \frac{\partial r}{\partial x_i}
  =
  g^\prime (r) \frac{x_i}{r}
\end{align*}

Ferner ist

\begin{align*}
  \frac{\partial^2 G}{\partial x_i^2}
  =
  g^\primeprime(r) \frac{x_i^2}{r^2}
    + g^\prime(r) \frac{1}{r}
    - g^\prime(r) \frac{x_i}{r}\frac{x_i}{r^2}
\end{align*}

Also erhalten wir insgesamt

\begin{align*}
  (\Delta_x +k^2) G
  =
  \sum_{i=1}^3 \frac{\partial^2 G}{\partial x_i^2} + k^2 G
  =
  g^\primeprime(r) \sum_{i=1}^3 \frac{x_i^2}{r^2}
    + g^\prime(r) \frac{3}{r}
    - g^\prime(r) \sum_{i=1}^3 \frac{x_i^2}{r^3}
    + k^2 g(r)
  =
  g^\primeprime(r) + \frac{2}{r} g^\prime(r) + k^2g(r)
\end{align*}

Also wird $(\Delta_x +k^2) G = 0$ zu einer gewöhnlichen Differentialgleichung für $g$ die für $r>0$ zu lösen ist.

\begin{align*}
  0 = g^\primeprime(r) + \frac{2}{r} g^\prime(r) + k^2g(r)
\end{align*}

Durch Multiplikation mit $r$ und unter der Verwendung der Produktregel bekommen wir

\begin{align*}
  \frac{\partial^2}{\partial r^2}(rg) + k^2(rg) = 0
\end{align*}

Lösungen dieser Differentialgleichung sind gegeben durch $g = C \frac{1}{r}e^{\pm ikr}, C \in \C$. Um der sommerfeldschen Ausstrahlungsbedingung zu genügen
wählen wir $g = C \frac{1}{r}e^{ikr}$ mit der noch unbestimmten Konstante $C$ und rechnen nach

\begin{multline*}
  \lim_{r \to \infty} r \pbraces{\pderivative{r} - i k} g(r)
  =
  \lim_{r \to \infty} r \pbraces{C \frac{ik e^{ikr}r - e^{ikr}}{r^2} - ikC e^{ikr} \frac{1}{r}} \\
  =
  \lim_{r \to \infty} r \pbraces{\frac{1}{r^2} e^{ikr}(ikr - 1) - ikC e^{ikr} \frac{1}{r}}
  =
  \lim_{r \to \infty} \frac{1}{r} e^{ikr}ikr C - \frac{1}{r}e^{ikr}C - e^{ikr}ik C
  =
  0
\end{multline*}

Um zu zeigen, dass die Funktion auch Fundamentallösung ist sei $\phi \in \mathcal{D}(\R^3)$ beliebig, $\Omega_\varepsilon = \R^3 \setminus B_\varepsilon(0)$, dann ist (mit PI nach Gauss)

\begin{align*}
  \langle (\Delta_x + k^2)G(\cdot, 0), \phi \rangle
  &=
  \langle G(\cdot, 0), (\Delta_x + k^2) \phi \rangle
  =
  \Int[\Omega_\varepsilon]{G(x, 0) (\Delta + k^2) \phi(x)}{x}
  =
  \Int[\Omega_\varepsilon]{G(x, 0) \Delta \phi(x)}{x} + \Int[\Omega_\varepsilon]{G(x, 0)k^2 \phi(x)}{x} \\
  &=
  -\Int[\Omega_\varepsilon]{\nabla G(x, 0) \cdot \nabla \phi(x)}{x}
    + \Int[\Omega_\varepsilon]{G(x, 0)k^2 \phi(x)}{x}
    + \Int[\partial \Omega_\varepsilon]{G(x,0) \nabla \phi \cdot \nu}{s} \\
  &=
  \underbrace{\Int[\Omega_\varepsilon]{(\Delta_x + k^2)G(x, 0) \phi(x)}{x}}_0
    + \Int[\partial \Omega_\varepsilon]{G(x,0) \nabla \phi \cdot \nu}{s}
    - \Int[\partial \Omega_\varepsilon]{\phi \nabla G(x,0) \cdot \nu}{s} \\
\end{align*}

Die beiden Randintegrale schauen wir uns für $\varepsilon \rightarrow 0$ noch einmal genauer an.

\begin{align*}
  \pbraces{\Int[\partial \Omega_\varepsilon]{G(x,0) \nabla \phi \cdot \nu}{s}}
  \leq
  \underbrace{|g(\varepsilon)|}_{\frac{C}{\varepsilon}} \underbrace{\text{meas}(\partial \Omega_\varepsilon)}_{\varepsilon^2 S_3} \max_{x \in \R^3}|\nabla \phi(x)|
  \stackrel{\varepsilon \to 0}{\longrightarrow}
  0
\end{align*}

Beim zweiten verwenden wir $\nu(x) = - r$, wenden den MWS der Integralrechnung an und wählen $C = -\frac{1}{S_3}$

\begin{align*}
  - \Int[\partial \Omega_\varepsilon]{\phi \nabla g(|x|) \cdot \nu}{s}
  &\stackrel{\text{MWS}}{=}
  \phi(x_\varepsilon) \frac{\partial g}{\partial r}(\varepsilon) \Int[\partial \Omega_\varepsilon]{s} \\
  =
  \phi(x_\varepsilon) C \frac{ik e^{ik\varepsilon}\varepsilon - e^{ik\varepsilon}}{\varepsilon^2} \varepsilon^2 S_3
  &=
  -\phi(x_\varepsilon) \underbrace{(ik e^{ik\varepsilon}\varepsilon - e^{ik\varepsilon})}_{\stackrel{\varepsilon \to 0}{\longrightarrow} -1}
  \stackrel{\varepsilon \to 0}{\longrightarrow}
  \phi(0)
\end{align*}
Also insgesamt

\begin{align*}
  \langle (\Delta_x + k^2)G(\cdot, 0), \phi \rangle
  =
  \langle \delta, \phi \rangle
\end{align*}
\end{solution}

% --------------------------------------------------------------------------------
