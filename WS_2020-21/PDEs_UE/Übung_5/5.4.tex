% --------------------------------------------------------------------------------

\begin{exercise}

Betrachten Sie die \textit{Helmholtz-Gleichung}

\begin{align*}
  (\Delta + k^2) u = f ~\text{in}~ \R^3
\end{align*}

mit der Wellenzahl $k > 0$, der Wellenquelle $f \in \mathcal{D}(\R^3)$ und dem Wellenfeld $u: \R^3 \to \C$.

\begin{enumerate}[label = (\roman*)]

  \item Berechnen Sie eine radialsymmetrische Funktion $G(x, \xi) = g(|x - \xi|)$,
  für die $(\Delta_x + k^2)G = \delta_\xi$ gilt, und die der \textit{sommerfeldschen
  Ausstrahlungsbedingung}

  \begin{align*}
    \lim_{r \to \infty} r \pbraces{\pderivative{r} - i k} g(r) = 0
  \end{align*}

  genügt.

  \textit{Anmerkung:}
  Diese Bedingung stellt eine Randbedingung (im Unendlichen) dar, weshalb man $G$ auch eine greensche Funktion nennt.

  \item Stellen Sie die/eine Lösung $u$ der Helmholtz-Gleichung mit Hilfe von $G(x, \xi)$ dar.
  Zeigen Sie, dass für eine radialsymmetrische Funktion $f$ diese Lösung auch radialsymmetrisch ist.

  \item Konstruieren Sie aus $G$ zwei weitere Funktionen $G_{\mathrm{Dir}}$ und $G_{\mathrm{Neu}}$, welche greensche Funktionen auf $\Omega := \R^2 \times \R_+$ sind und auf $\partial\Omega$ homogene Dirichlet- bzw. homogene Neumann-Randbedingunen erfüllen.

\end{enumerate}

\end{exercise}

% --------------------------------------------------------------------------------

\begin{solution}

\phantom{}

\end{solution}

% --------------------------------------------------------------------------------
