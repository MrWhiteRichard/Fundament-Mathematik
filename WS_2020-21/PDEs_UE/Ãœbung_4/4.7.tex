% --------------------------------------------------------------------------------

\begin{exercise}

Bestimmen Sie Fundamentallösungen mit Pol an $\xi \in \R$ folgender Differentialoperatoren
auf $\R$:
\begin{enumerate}[label = (\roman*)]
  \item $L(u) = u^{\prime}$
  \item $L(u) = u^{\primeprime}$
  \item $L(u) = u^{\prime} - au (a \neq 0)$ \\
  \textit{Hinweis:} Bestimmen Sie die allgemeine Lösung $u_{hom}$ der homogenen
  Differentialgleichung $Lu = 0$ und verwenden Sie einen Ansatz der Form
  $U_0(x) = C_1u_{hom}(x)$ für $x < 0, U_0(x) = C_2u_{hom}(x)$ für $x > 0$ mit
  $C_1,C_2 \in \R)$.
\end{enumerate}

\end{exercise}

% --------------------------------------------------------------------------------

\begin{solution}

\phantom{}
\begin{enumerate}[label = (\roman*)]
  \item Wir wählen die Heaviside-Funktion
  \begin{align*}
    H(x) = \begin{cases}
      0 &: x < 0 \\
      1 &: x \geq 0
    \end{cases}
  \end{align*}
  und rechnen für $\phi \in \mathcal{D}(\R)$ beliebig nach
  \begin{align*}
    \langle H^{\prime}, \phi \rangle = - \langle H, \phi^{\prime} \rangle
    = -\int_0^{\infty}\phi^{\prime}(x) dx = \phi(0) = \langle \delta, \phi \rangle.
  \end{align*}
  Also ist die Heaviside-Funktion eine Fundamentallösung zur Polstelle $0$.
  Für eine Fundamentallösung $U_{\xi}$ zur Polstelle $\xi \in \R$ gehen wir nach folgender Formel vor:
  \begin{align*}
    U_{\xi}(x) = \tau_{-\xi}U_0(x) = \tau_{-\xi}H(x) = H(x -\xi) =  \begin{cases}
      0 &: x < \xi \\
      1 &: x \geq \xi
    \end{cases}.
  \end{align*}
  \item Die Lösung ist aus dem Skript bekannt:
  \begin{align*}
    g(x,\xi) = \begin{cases}
      x(\xi - 1) &: x \leq \xi \\
      \xi(x - 1) &: x  > \xi
    \end{cases}
  \end{align*}
  \item
  Wir lösen $Lu = u^{\prime} - au \stackrel{!}{=} 0$.
  \begin{align*}
    \frac{u^{\prime}}{u} = a \iff \ln(u)^{\prime} = a
    \iff \ln(u) = ax + C
    \iff u = D\exp(ax)
  \end{align*}
  Wir machen den Ansatz
  \begin{align*}
    U_0(x) = \begin{cases}
      C_1\exp(ax) &: x < 0 \\
      C_2\exp(ax) &: x \geq 0
    \end{cases}
  \end{align*}
  und setzen für $\phi \in \mathcal{D}(\R)$ beliebig ein:

  \begin{align*}
    \langle L(U_0), \phi \rangle
    &= -\langle U_0, \phi^{\prime} + a\phi \rangle
  = -\int_{\R}U_0(\phi^{\prime} + a\phi)dx \\
  &= -\int_{-\infty}^0C_1\exp(ax)(\phi^{\prime} + a\phi)dx
  -\int_0^{\infty}C_2\exp(ax)(\phi^{\prime} + a\phi)dx \\
  &= C_1\exp(ax)\phi\mid_{-\infty}^0 - C_2\exp(ax)\phi\mid_{0}^{\infty}\\
  &= (C_1-C_2)\exp(0)\phi(0) = (C_1 - C_2)\langle \delta, \phi \rangle
  \end{align*}
  Für $C_1 - C_2 = 1$ ist also $U_0$ eine Fundamentallösung
  an der Polstelle $0$. Wählen wir also der Einfachheit halber $C_1 = 1$ und $C_2 = 0$, dann erhalten wir
  \begin{align*}
    U_{\xi}(x) = \tau_{-\xi}U_0(x) =
    \begin{cases}
      \exp(a(x-\xi)) &: x < \xi \\
      0 &: x \geq \xi
    \end{cases}.
  \end{align*}
\end{enumerate}


\end{solution}

% --------------------------------------------------------------------------------
