% --------------------------------------------------------------------------------

\begin{exercise}

Gegeben $v \in C^2(\R)$, sei $u(x,t) = v\left(x/\sqrt{t}\right)$ für $t > 0$ und $x \in \R$.
\begin{enumerate}[label = (\roman*)]
  \item Zeigen Sie:
  \begin{align*}
    u_t = u_{xx} \iff v^{\primeprime}(z) + \frac{z}{2}v^{\prime}(z) = 0.
  \end{align*}
  Berechnen Sie die allgemeine Lösung $v$ und damit $u$.
  \item Wählen Sie die Konstanten in $u$ so, dass
  \begin{align*}
    \lim_{t \to 0^+} u(t,x) = 0 \text{ für } x < 0, \quad \lim_{t \to 0^+} u(t,x) = 1
    \text{ für } x < 0.
  \end{align*}
  \item Zeigen Sie, dass für $\varphi \in \mathcal{D}(\R)$ die Funktion
  $f(x,t) = (\partial_x u(\cdot,t)\ast \varphi)(x)$ (Faltung in der $x$-Variablen)
  folgendes Anfangswertproblem für die Wärmeleitungsgleichung löst:
  \begin{align*}
    f_t - f_{xx} &= 0 \\
    \lim_{t \to 0^+} f(t,x) &= \varphi(x)
  \end{align*}
\end{enumerate}
\end{exercise}

% --------------------------------------------------------------------------------

\begin{solution}

\phantom{}

\end{solution}

% --------------------------------------------------------------------------------
