% -------------------------------------------------------------------------------- %

\begin{exercise}

Gegeben $v \in C^2(\R)$, sei $u(x,t) = v\left(x/\sqrt{t}\right)$ für $t > 0$ und $x \in \R$.
\begin{enumerate}[label = (\roman*)]
  \item Zeigen Sie:
  \begin{align*}
    u_t = u_{xx} \iff v^{\primeprime}(z) + \frac{z}{2}v^{\prime}(z) = 0.
  \end{align*}
  Berechnen Sie die allgemeine Lösung $v$ und damit $u$.
  \item Wählen Sie die Konstanten in $u$ so, dass
  \begin{align*}
    \lim_{t \to 0^+} u(x,t) = 0 \text{ für } x < 0, \quad \lim_{t \to 0^+} u(x,t) = 1
    \text{ für } x > 0.
  \end{align*}
  \item Zeigen Sie, dass für $\varphi \in \mathcal{D}(\R)$ die Funktion
  $f(x,t) = (\partial_x u(\cdot,t)\ast \varphi)(x)$ (Faltung in der $x$-Variablen)
  folgendes Anfangswertproblem für die Wärmeleitungsgleichung löst:
  \begin{align*}
    f_t - f_{xx} &= 0 \\
    \lim_{t \to 0^+} f(t,x) &= \varphi(x)
  \end{align*}
\end{enumerate}
\end{exercise}

% -------------------------------------------------------------------------------- %

\begin{solution}
\phantom{}
\begin{enumerate}[label = (\roman*)]
  \item Da $t > 0$ gilt
  \begin{align*}
    0 &= u_t - u_{xx} = \partial_t v(x/\sqrt{t}) - \partial_{xx}v(x/\sqrt{t})
    = -\frac{x}{2\sqrt{t}^3}v^{\prime}(x/\sqrt{t}) - \frac{1}{t}v^{\primeprime}(x/\sqrt{t})
    = - \frac{z}{2t}v^{\prime}(z) - \frac{1}{t}v^{\primeprime}(z) \\
    &\iff \frac{z}{2}v^{\prime}(z) + v^{\primeprime}(z) = 0.
  \end{align*}
  Nun lösen wir die gewöhnliche Differentialgleichung für $w := v^{\prime}$
  \begin{align*}
    w^{\prime}(z) + \frac{z}{2}w(z) = 0 &\iff \frac{w^{\prime}(z)}{w(z)} = -\frac{z}{2}
    \iff \ln(w(z))^{\prime} = -\frac{z}{2} \iff \ln(w(z)) = -\frac{z^2}{4} + C_0 \\
    &\iff w(z) = C_1\exp\left(-z^2/4\right).
  \end{align*}
  Also erhalten wir $v(z) = C_1\int_0^z \exp(-s^2/4) ds + C_2$ und
  $u(x,t) = v(x/\sqrt{t}) = C_1\int_0^{x/\sqrt{t}} \exp(-s^2/4) ds + C_2$.
  \item
  \begin{align*}
    \lim_{t \to 0^+} u(t,x) = \lim_{t \to 0^+} C_1\int_0^{x/\sqrt{t}} \exp(-s^2/4) ds + C_2
    = \sgn(x)\sqrt{\pi}C_1 + C_2.
  \end{align*}
  Wir erhalten also die Gleichungen
  \begin{align*}
    -\sqrt{\pi}C_1 + C_2 &= 0   \iff C_2 = C_1\sqrt{\pi}\\
    \sqrt{\pi}C_1 + C_2 &= 1 \iff 2C_1\sqrt{\pi} = 1 \iff C_1 = \frac{1}{2\sqrt{\pi}}
    \iff C_2 = \frac{1}{2}.
  \end{align*}
  Unter diesen zusätzlichen Bedingungen lautet unsere Lösung nun
  \begin{align*}
    u(x,t) = \frac{1}{2\sqrt{\pi}}\int_0^{x/\sqrt{t}}\exp(-s^2/4)ds + \frac{1}{2}.
  \end{align*}
  \item
  Da $u(\cdot,t)$ und $\partial_x u(\cdot,1)$ stetig sind, sind sie insbesondere
  auch lokal integrierbar und es folgt mit Lemma 3.14
  \begin{align*}
    f_t - f_{xx} &= \partial_t((\partial_x u(\cdot,t)\ast \varphi)(x))
    - \partial_{xx}(\partial_x u(\cdot,t)\ast \varphi)(x) \\
    &= (\partial_{tx} u(\cdot,t) \ast \varphi)(x) - (\partial_{xxx} u(\cdot,t) \ast \varphi)(x) \\
    &= ((\partial_{tx} u(\cdot,t) - \partial_{xxx} u(\cdot,t)) \ast \varphi)(x) \\
    &= ((\partial_{x} (\partial_t u(\cdot,t) - \partial_{xx} u(\cdot,t))) \ast \varphi)(x) = 0.
  \end{align*}

  \begin{align*}
    \lim_{t \to 0^+} f(t,x) &= \lim_{t \to 0^+} (\partial_x u(\cdot,t)\ast \varphi)(x)
    = \lim_{t \to 0^+} (u(\cdot,t)\ast \varphi^{\prime})(x)
    = \lim_{t \to 0^+} \int_{\R}u(x-y,t) \varphi^{\prime}(y) dy \\
    &= \int_{\R}\lim_{t \to 0^+} u(x-y,t) \varphi^{\prime}(y) dy
    = \int_{-\infty}^x \lim_{t \to 0^+} u(x-y,t) \varphi^{\prime}(y) dy
    + \int_{x}^{\infty} \lim_{t \to 0^+} u(x-y,t) \varphi^{\prime}(y) dy \\
    &= \int_{-\infty}^x \varphi^{\prime}(y) dy = \varphi(x).
  \end{align*}
  Die Vertauschung von Grenzwert und Integral gelingt mittels dominierter Konvergenz
  wegen der Abschätzung
  \begin{align*}
    |u(x-y,t)| = \frac{1}{2\sqrt{\pi}}\int_0^{(x-y)/\sqrt{t}}\exp(-s^2/4)ds + \frac{1}{2}
    \leq \frac{1}{2\sqrt{\pi}}\int_0^{\infty}\exp(-s^2/4)ds + \frac{1}{2}
  \end{align*} \\
\end{enumerate}

\end{solution}

% -------------------------------------------------------------------------------- %
