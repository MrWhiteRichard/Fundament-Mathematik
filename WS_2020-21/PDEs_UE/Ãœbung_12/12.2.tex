% --------------------------------------------------------------------------------

\begin{exercise}

Seien das elektrische Feld $E = (E_1, E_2, E_3)^\top$ und das magnetische Feld $B = (B_1, B_2, B_3)^\top$ glatte Lösungen der Maxwell-Gleichungen

\begin{align*}
    E_t = \rot B, \quad B_t = -\rot E,
    \quad
    \Div E = \Div B = 0
\end{align*}

ohne Ladungen und Ströme, wobei $x \in \R^3$ und $t > 0$.

\begin{enumerate}[label = (\roman*)]
    \item Zeigen Sie, dass $u = E_i$ bzw. $u = B_i, i = 1, 2, 3$, die Wellengleichung $u_{tt} - \Delta u = 0$ löst.
    \item Zeigen Sie, dass die Energiedichte $\frac{1}{2} \Int[\R^3]{(|E|^2+|B|^2)}{x}$ zeitlich konstant ist.
\end{enumerate}

\end{exercise}

% --------------------------------------------------------------------------------

\begin{solution}

\phantom{}

\begin{enumerate}[label = (\roman*)]

    \item Wir erinnern uns an folgende Definitionen.
    
    \begin{align*}
        \rot f
        =
        \begin{pmatrix}
            \partial_{x_2} f_3 - \partial_{x_3} f_2 \\
            \partial_{x_3} f_1 - \partial_{x_1} f_3 \\
            \partial_{x_1} f_2 - \partial_{x_2} f_1
        \end{pmatrix},
        \quad
        \Div f
        =
        \sum_{i=1}^3
        \partial_{x_i} f_i
    \end{align*}

    \begin{align*}
        \rot \rot f
        & =
        \rot
        \begin{pmatrix}
            \partial_{x_2} f_3 - \partial_{x_3} f_2 \\
            \partial_{x_3} f_1 - \partial_{x_1} f_3 \\
            \partial_{x_1} f_2 - \partial_{x_2} f_1
        \end{pmatrix} \\
        & =
        \begin{pmatrix}
            \partial_{x_2} (\partial_{x_1} f_2 - \partial_{x_2} f_1) - \partial_{x_3} (\partial_{x_3} f_1 - \partial_{x_1} f_3) \\
            \partial_{x_3} (\partial_{x_2} f_3 - \partial_{x_3} f_2) - \partial_{x_1} (\partial_{x_1} f_2 - \partial_{x_2} f_1) \\
            \partial_{x_1} (\partial_{x_3} f_1 - \partial_{x_1} f_3) - \partial_{x_2} (\partial_{x_2} f_3 - \partial_{x_3} f_2)
        \end{pmatrix} \\
        & =
        \begin{pmatrix}
            \partial_{x_1 x_2} f_2 - \partial_{x_2 x_2} f_1 - \partial_{x_3 x_3} f_1 + \partial_{x_1 x_3} f_3 \\
            \partial_{x_2 x_3} f_3 - \partial_{x_3 x_3} f_2 - \partial_{x_1 x_1} f_2 + \partial_{x_1 x_2} f_1 \\
            \partial_{x_1 x_3} f_1 - \partial_{x_1 x_1} f_3 - \partial_{x_2 x_2} f_3 + \partial_{x_2 x_3} f_2
        \end{pmatrix} \\
        & =
        \begin{pmatrix}
            \partial_{x_1} \partial_{x_1} f_1 + \partial_{x_2} \partial_{x_1} f_2 + \partial_{x_3} \partial_{x_1} f_3 \\
            \partial_{x_1} \partial_{x_2} f_1 + \partial_{x_2} \partial_{x_2} f_2 + \partial_{x_3} \partial_{x_2} f_3 \\
            \partial_{x_1} \partial_{x_3} f_1 + \partial_{x_2} \partial_{x_3} f_2 + \partial_{x_3} \partial_{x_3} f_3
        \end{pmatrix}
        -
        \begin{pmatrix}
            \partial_{x_1 x_1} f_1 + \partial_{x_2 x_2} f_1 + \partial_{x_3 x_3} f_1 \\
            \partial_{x_1 x_1} f_2 + \partial_{x_2 x_2} f_2 + \partial_{x_3 x_3} f_2 \\
            \partial_{x_1 x_1} f_3 + \partial_{x_2 x_2} f_3 + \partial_{x_3 x_3} f_3 \\
        \end{pmatrix} \\
        & =
        \begin{pmatrix}
            \partial_{x_1} \Div f \\
            \partial_{x_2} \Div f \\
            \partial_{x_3} \Div f
        \end{pmatrix}
        -
        \begin{pmatrix}
            \Delta f_1 \\
            \Delta f_2 \\
            \Delta f_3
        \end{pmatrix} \\
        & =
        \nabla \Div f
        -
        \Delta f
    \end{align*}

    Wir erinnern uns an die Bedingungen an das elektrische und magnetische Feld $E$ bzw. $B$.

    \begin{align*}
        \implies
        B_{tt} & = -\rot E_t = -\rot \rot B = \Delta B, \\
        E_{tt} & = \rot B_t = \rot(-\rot E) = \Delta E
    \end{align*}

    \item Wir zeigen, dass die Ableitung der Energiedichte verschwindet.
    
    \begin{align*}
        \derivative{t}
        \frac{1}{2}
        \Int[\R^3]{\pbraces{|E|^2 + |B|^2}}{x}
        & =
        \derivative{t}
        \lim_{R \to \infty}
        \frac{1}{2}
        \Int[B_R(0)]{\pbraces{|E|^2 + |B|^2}}{x} \\
        & \stackrel{!}{=}
        \lim_{R \to \infty}
        \derivative{t}
        \frac{1}{2}
        \Int[B_R(0)]{\pbraces{|E|^2 + |B|^2}}{x} \\
        & \stackrel
        {
            \mathrm{MK}
        }{=}
        \lim_{R \to \infty}
        \frac{1}{2}
        \Int[B_R(0)]
        {
            \derivative{t}
            \pbraces{|E|^2 + |B|^2}
        }{x} \\
        & \stackrel{!}{=}
        0
    \end{align*}
    
    \begin{align*}
        \derivative{t}
        \pbraces{|E|^2 + |B|^2}
        & =
        \derivative{t}
        \sum_{i=1}^3 E_i^2 + B_i^2 \\
        & =
        \sum_{i=1}^3 2 E_i (E_t)_i + B_i (B_t)_i \\
        & =
        2 \sum_{i=1}^3 E_i (\rot B)_i - B_i (\rot E)_i \\
        & =
        2
        (
            E_1 (\partial_{x_2} B_3 - \partial_{x_3} B_2)
            +
            E_2 (\partial_{x_3} B_1 - \partial_{x_1} B_3)
            +
            E_3 (\partial_{x_1} B_2 - \partial_{x_2} B_1) \\
            & \quad
            -
            B_1 (\partial_{x_2} E_3 - \partial_{x_3} E_2)
            -
            B_2 (\partial_{x_3} E_1 - \partial_{x_1} E_3)
            -
            B_3 (\partial_{x_1} E_2 - \partial_{x_2} E_1)
        ) \\
        & =
        2
        (
            (
                E_1 \partial_{x_2} B_3
                +
                B_3 \partial_{x_2} E_1
            )
            -
            (
                E_1 \partial_{x_3} B_2
                +
                B_2 \partial_{x_3} E_1
            ) \\
            & \quad
            +
            (
                E_2 \partial_{x_3} B_1
                +
                B_1 \partial_{x_3} E_2
            )
            -
            (
                E_2 \partial_{x_1} B_3
                +
                B_3 \partial_{x_1} E_2
            ) \\
            & \quad
            +
            (
                E_3 \partial_{x_1} B_2
                +
                B_2 \partial_{x_1} E_3
            )
            -
            (
                E_3 \partial_{x_2} B_1
                +
                B_1 \partial_{x_2} E_3
            )
        ) \\
        & =
        2
        (
            \partial_{x_2} (E_1 B_3)
            -
            \partial_{x_3} (E_1 B_2)
            +
            \partial_{x_3} (E_2 B_1)
            -
            \partial_{x_1} (E_2 B_3)
            +
            \partial_{x_1} (E_3 B_2)
            -
            \partial_{x_2} (E_3 B_1)
        ) \\
        & =
        2
        \pbraces
        {
            \Div
            \begin{pmatrix}
                E_3 B_2 \\ E_1 B_3 \\ E_2 B_1
            \end{pmatrix}
            -
            \Div
            \begin{pmatrix}
                E_2 B_3 \\ E_3 B_1 \\ E_1 B_2
            \end{pmatrix}
        } \\
        & =
        2 \Div(E \times B)
    \end{align*}

    Wir gehen davon aus, dass $E(\cdot, t), B(\cdot, t) \in L^2(\R^3)$, $t > 0$.
    Sei $f$ eine dieser Funktionen.

    \begin{align*}
        \implies
        \Int[0][\infty]
        {
            \norm[L^2(\partial B_R(0))]{f}^2
        }{R}
        & =
        \Int[0][\infty]
        {
            \Int[\partial B_R(0)]{|f|^2}{s}
        }{R} \\
        & \stackrel
        {
            \mathrm{KOFLFO}
        }{=}
        \Int[\R^3]{|f|^2}{x} \\
        & =
        \norm[L^2(\R^3)]{f}^2 \\
        & <
        \infty \\
        \implies
        \norm[L^2(\partial B_R(0))]{f}
        & \xrightarrow{R \to \infty}
        0
    \end{align*}

    \begin{align*}
        \implies
        \frac{1}{2}
        \Int[B_R(0)]{\derivative{t} \pbraces{|E|^2 + |B|^2}}{x}
        & =
        \Int[B_R(0)]{\Div(E \times B)}{x} \\
        & \stackrel
        {
            \mathrm{Gauß}
        }{=}
        \Int[\partial B_R(0)]{(E \times B) \cdot \nu}{s} \\
        & =
        \Int[\partial B_R(0)]{(E \times \nu) \cdot B}{s} \\
        & \stackrel
        {
            \mathrm{CSB}
        }{\leq}
        \sqrt
        {
            \Int[\partial B_R(0)]{|E \times \nu|^2}{s}
        }
        \sqrt
        {
            \Int[\partial B_R(0)]{|B|^2}{s}
        } \\
        & =
        \sqrt
        {
            \Int[\partial B_R(0)]{|E|^2 |\nu|^2 \sin^2 \theta}{s}
        }
        \sqrt
        {
            \Int[\partial B_R(0)]{|B|^2}{s}
        } \\
        & \leq
        \norm[L^2(\partial B_R(0))]{E}
        \norm[L^2(\partial B_R(0))]{B}
        \xrightarrow{R \to \infty}
        0
    \end{align*}

\end{enumerate}
    
\end{solution}

% --------------------------------------------------------------------------------
