% --------------------------------------------------------------------------------

\begin{exercise}

Sei $u$ eine klassische Lösung der Telegraphengleichung

\begin{align*}
    \begin{cases}
        u_{tt} + d u_t - \Delta u = 0 & \text{in}~ \Omega, t > 0, \\
        u = 0                         & \text{auf}~ \partial \Omega, t > 0, \\
        u(\cdot, 0) = u_0             & \text{in}~ \Omega, \\
        u_t(\cdot, 0) = u_1           & \text{in}~ \Omega,
    \end{cases}
\end{align*}

wobei $d>0$ konstant, $u_0 \in H_0^1(\Omega), u_1 \in L^2(\Omega)$ und $\Omega \subset \R^n$ ein beschränktes Gebiet mit glattem Rand ist.

\begin{enumerate}[label = (\roman*)]

    \item Zeigen Sie durch eine formale Rechnung, dass die Energie $\Int[\Omega]{(u_t^2 + |\nabla u|^2)}{x}$ uniform beschränkt in $t \in (0, \infty)$ ist.

    \item Bestimmen Sie formal eine Lösung bzgl. eines geeigneten ONS.

    \item Zeigen Sie, dass $\norm[L^2(\Omega)]{u_t}$ exponentiell schnell für $t \rightarrow \infty$ gegen $0$ konvergiert, falls $u_1 = 0$.
    Gilt diese Aussage auch für $d = 0$?

\end{enumerate}

\end{exercise}

% --------------------------------------------------------------------------------

\begin{solution}

\phantom{}

\begin{enumerate}[label = (\roman*)]

    \item Wir gehen analog zu Seite 121 im Skriptum vor.

    Dann können wir die Differentialgleichung mit $u_t$ multiplizieren und integrieren und erhalten

    \begin{align*}
        \implies
        0
        =
        \Int[\Omega]{u_{tt} u_t + d u_t^2 - (\Delta u) u_t}{x}
        =
        \Int[\Omega]{u_{tt} u_t}{x}
        +
        d \Int[\Omega]{u_t^2}{x}
        -
        \Int[\Omega]{(\Delta u) u_t}{x}.
    \end{align*}

    \begin{align*}
        \derivative{t} u_t^2 = 2 u_{tt} u_t,
        \quad
        \derivative{t} |\nabla u|^2 = 2 \nabla u \cdot \nabla u_t
    \end{align*}

    Für das linke Integral erhalten wir daher

    \begin{align*}
        \implies
        \Int[\Omega]{u_{tt} u_t}{x}
        =
        \frac{1}{2}
        \derivative{t}
        \Int[\Omega]{u_t^2}{x}.
    \end{align*}

    Für das rechte Integral erhalten wir (daher)

    \begin{align*}
        \implies
        -\Int[\Omega]{(\Delta u) u_t}{x}
        \stackrel
        {
            \mathrm{PI}
        }{=}
        \Int[\Omega]{\nabla u \cdot \nabla u_t}{x}
        -
        \underbrace
        {
            \Int[\Omega]{u_t (\nabla u \cdot \nu)}{s}
        }_0
        =
        \frac{1}{2}
        \derivative{t}
        \Int[\Omega]{|\nabla u|^2}{x}.
    \end{align*}

    Insgesamt erhalten wir (daher)

    \begin{align*}
        \implies
        \Int[\Omega]{u_{tt} u_t}{x}
        +
        d \Int[\Omega]{u_t^2}{x}
        -
        \Int[\Omega]{(\Delta u) u_t}{x}
        =
        \frac{1}{2}
        \derivative{t}
        \Int[\Omega]{u_t^2 + |\nabla u|^2}{x}
        +
        d \norm[L^2(\Omega)]{u_t(\cdot, t)}^2.
    \end{align*}

    Die Energie bezeichnen wir mit $E(t)$.

    \begin{align*}
        \implies
        E^\prime(t)
        =
        -2 d \norm[L^2(\Omega)]{u_t(\cdot, t)}^2
        \leq
        0
    \end{align*}

    Das heißt, dass die Energie höchstens abnimmt.
    Wir berechnen die Anfangs-Energie $E(0)$.

    \begin{align*}
        \implies
        E(0)
        =
        \Int[\Omega]{u_t(x, 0)^2 + |\nabla u(x, 0)|^2}{x}
        =
        \norm[L^2(\Omega)]{u_1}^2 + \norm[L^2(\Omega)]{\nabla u_0}^2
        \leq
        \underbrace
        {
            \norm[L^2(\Omega)]{u_1}^2
        }_{
            < \infty
        }
        +
        \underbrace
        {
            \norm[H^1(\Omega)]{u_0}^2
        }_{
            < \infty
        }
        <
        \infty
    \end{align*}

    Wir benutzen den Hauptsatz der Differential- und Integralrechnung, um zu zeigen, dass die Anfangs-Energie $E(0)$ tatsächlich eine uniforme Schranke für $E(t)$, $t > 0$ ist.

    \begin{align*}
        \implies
        E(t)
        =
        E(0)
        +
        \underbrace
        {
            \Int[0][t]{E^\prime(s)}{s}
        }_{
            \leq 0
        }
        \leq
        E(0)
        <
        \infty
    \end{align*}

    \item Sei $(v_k)_{k \in \N}$ ein ONS von $L^2(\Omega)$ bestehend aus Eigenvektoren von $- \Delta $ zu den positiven und aufsteigend sortierten Eigenwerten $(\lambda_k)_{k \in \N}$. Wir berechnen für $j \in \N$ und $\tilde{u}_j := (u,v_j)_{L^2(\Omega)}$ formal
    \begin{align*}
    	(\partial_{tt} + d \partial_t)u_j = (u_{tt} + du_t, v_j)_{L^2(\Omega)} = (\Delta u, v_j)_{L^2(\Omega)} = -\lambda_j u_j.
    \end{align*}
	Hier haben wir es also mit einer gewöhnlichen Differentialgleichung zu tun. Die Nullstellen des charakteristischen Polynoms sind
	\begin{align*}
		\mu_j := - \frac{d}{2} + \sqrt{\frac{d^2}{4} - \lambda_j}, \quad \nu_j := - \frac{d}{2} - \sqrt{\frac{d^2}{4} - \lambda_j}.
	\end{align*}
	Also ist eine Allgemeine Lösung gegeben durch
	\begin{align*}
		\tilde{u}_j(t) = c_1 e^{\mu_j t} + c_2 e^{\nu_j t}
	\end{align*}
	Nun sollen auch noch die Anfangsbedingungen
	\begin{align*}
		(u_0, v_j)_{L^2(\Omega)} = \tilde{u}_j(0) = c_1 + c_2, \quad (u_1, v_j)_{L^2(\Omega)} = \partial_t \tilde{u}_j(0) = \mu_j c_1 + \nu_j c_2
	\end{align*}
	Wenn wir das nun lösen erhalten wir
	\begin{align*}
		\tilde{u}_j(t) = \frac{\mu_j(u_0, v_j)_{L^2(\Omega)} - (u_1, v_j)_{L^2(\Omega)}}{\mu_j - \nu_j} e^{\nu_j t} - \frac{\nu_j(u_0, v_j)_{L^2(\Omega)} - (u_1, v_j)_{L^2(\Omega)}}{\mu_j - \nu_j} e^{\mu_j t}
	\end{align*}
	und somit die formale Lösung
	\begin{align*}
		u(t) &= \sum_{j=1}^\infty \tilde{u}_j(t) v_j \\
		&= \sum_{j=1}^\infty \pbraces{\frac{\mu_j(u_0, v_j)_{L^2(\Omega)} - (u_1, v_j)_{L^2(\Omega)}}{\mu_j - \nu_j} e^{\nu_j t} - \frac{\nu_j(u_0, v_j)_{L^2(\Omega)} - (u_1, v_j)_{L^2(\Omega)}}{\mu_j - \nu_j} e^{\mu_j t}} v_j
	\end{align*}

	\item Nun setzen wir $u_1 := 0$. Damit gilt
	\begin{align*}
		u(t) = \sum_{j=1}^\infty \pbraces{\frac{\mu_j(u_0, v_j)_{L^2(\Omega)}}{\mu_j - \nu_j} e^{\nu_j t} - \frac{\nu_j(u_0, v_j)_{L^2(\Omega)}}{\mu_j - \nu_j} e^{\mu_j t}} v_j
	\end{align*}
	und mit einer formalen Grenzwertvertauschung und dem Satz von Parseval weiters
	\begin{align*}
		\norm[L^2(\Omega)]{u_t(t)}^2 &= \sum_{j=1}^\infty \vbraces{\frac{\mu_j\nu_j(u_0, v_j)_{L^2(\Omega)}}{\mu_j - \nu_j} e^{\nu_j t} - \frac{\mu_j\nu_j(u_0, v_j)_{L^2(\Omega)}}{\mu_j - \nu_j} e^{\mu_j t}}^2 \\
		&= \sum_{j=1}^\infty \frac{\vbraces{\mu_j\nu_j(u_0, v_j)_{L^2(\Omega)}}^2}{\vbraces{\mu_j - \nu_j}^2} \vbraces{e^{\nu_j t} - e^{\mu_j t}}^2 \\
		&= \sum_{j=1}^\infty \frac{\vbraces{\mu_j\nu_j(u_0, v_j)_{L^2(\Omega)}}^2}{\vbraces{\mu_j - \nu_j}^2} e^{-d t}\vbraces{\exp\pbraces{-t\sqrt{\frac{d^2}{4} - \lambda_j} } - \exp\pbraces{t\sqrt{\frac{d^2}{4} - \lambda_j} }}^2
	\end{align*}
	Wir wissen, dass für jedes $j \in \N$ der Eigenwert $\lambda_j$ positiv ist und dass die Eigenwerte aufsteigend geordnet sind. Sei also $l \in \N$ der kleinste Index mit $\frac{d^2}{4} < \lambda_l$. Für alle $k \geq l$ gilt nun
	\begin{align*}
		\mu_k \nu_k = \pbraces{-\frac{d}{2} + i \sqrt{\lambda_k - \frac{d^2}{4}}}\pbraces{-\frac{d}{2} - i \sqrt{\lambda_k - \frac{d^2}{4}}} = \frac{d^2}{4} + \pbraces{\lambda_k - \frac{d^2}{4}} = \lambda_k.
	\end{align*}
	Außerdem gilt
	\begin{align*}
		\vbraces{\mu_k - \nu_k}^2 = \vbraces{2 i \sqrt{\lambda_k - \frac{d^2}{4}}}^2 = 4 \pbraces{\lambda_k - \frac{d^2}{4}} \geq 4 \pbraces{\lambda_l - \frac{d^2}{4}}
	\end{align*}
	sowie
	\begin{align*}
		\vbraces{\exp\pbraces{-i t \sqrt{\lambda_k - \frac{d^2}{4}} } - \exp\pbraces{i t \sqrt{\lambda_k - \frac{d^2}{4}} }}^2 &= 4 \vbraces{\frac{\exp\pbraces{i t \sqrt{\lambda_k - \frac{d^2}{4}}} - \exp\pbraces{-i t \sqrt{\lambda_k - \frac{d^2}{4}} }}{2i}}^2  \\
		&= 4 \vbraces{\sin\pbraces{t \sqrt{\lambda_k - \frac{d^2}{2}}}}^2 \leq 4
	\end{align*}
	also gilt mit der Besselschen Ungleichung ganz am Ende
	\begin{align*}
		\sum_{j=l}^\infty \frac{\vbraces{\mu_j\nu_j(u_0, v_j)_{L^2(\Omega)}}^2}{\vbraces{\mu_j - \nu_j}^2} e^{-d t}\vbraces{\exp\pbraces{-t\sqrt{\frac{d^2}{4} - \lambda_j} } - \exp\pbraces{t\sqrt{\frac{d^2}{4} - \lambda_j} }}^2 \\
		\leq \pbraces{\lambda_l - \frac{d^2}{4}}^{-1} e^{-d t} \sum_{j=l}^\infty \vbraces{\lambda_j (u_0, v_j)_{L^2(\Omega)}}^2 \\
		= \pbraces{\lambda_l - \frac{d^2}{4}}^{-1} e^{-d t} \sum_{j=l}^\infty \vbraces{ (u_0, \lambda_j v_j)_{L^2(\Omega)}}^2 \\
		= \pbraces{\lambda_l - \frac{d^2}{4}}^{-1} e^{-d t} \sum_{j=l}^\infty \vbraces{ (u_0, \Delta v_j)_{L^2(\Omega)}}^2 \\
		= \pbraces{\lambda_l - \frac{d^2}{4}}^{-1} e^{-d t} \sum_{j=l}^\infty \vbraces{ (\nabla u_0, \nabla v_j)_{L^2(\Omega)}}^2 \\
		\leq \pbraces{\lambda_l - \frac{d^2}{4}}^{-1} e^{-d t} \norm[L^2(\Omega)]{\nabla u_0}^2
	\end{align*}

	Also erhalten wir insgesamt
	\begin{align*}
		\norm[L^2(\Omega)]{u_t(t)}^2 \\
		 \leq e^{- d t} \pbraces{ \pbraces{\lambda_l - \frac{d^2}{4}}^{-1} \norm[L^2(\Omega)]{\nabla u_0}^2 + \sum_{j=1}^{l-1} \frac{\vbraces{\mu_j\nu_j(u_0, v_j)_{L^2(\Omega)}}^2}{\vbraces{\mu_j - \nu_j}^2} \vbraces{\exp\pbraces{-t\sqrt{\frac{d^2}{4} - \lambda_j} } - \exp\pbraces{t\sqrt{\frac{d^2}{4} - \lambda_j} }}^2}
	\end{align*}
	Für $d = 0$ haben wir es einfach mit der homogenen Wellengleichung zu tun. Nach den überlegungen auf Seite $118$ im Skript bekommen wir die Lösungsformel

  \begin{align*}
    u(\cdot, t)
    =
    \sum_{k=1}^\infty \cos(\sqrt{\lambda_k}t) (u_0, v_k)_{L^2(\Omega)} v_k
    +
    \sum_{k=1}^\infty \frac{\sin(\sqrt{\lambda_j }t)}{\sqrt{\lambda_j}} (u_1, v_k)_{L^2(\Omega)} v_k
  \end{align*}

  Nach Parseval gilt also (wiederum mit formaler Grenzwertvertauschung)

  \begin{align*}
    \norm[L^2(\Omega)]{u_t(\cdot, t)}^2
    &=
    \sum_{k=1}^\infty | \sqrt{\lambda_k} \sin(\sqrt{\lambda_k}t) (u_0, v_k)_{L^2(\Omega)}|^2
    +
    \sum_{k=1}^\infty \vbraces{\cos(\sqrt{\lambda_k}t) (u_1, v_k)_{L^2(\Omega)}}^2 \\
    &\geq
    \lambda_1 \sum_{k=1}^\infty |\sin(\sqrt{\lambda_k}t) (u_0, v_k)_{L^2(\Omega)}|^2
    +
    \sum_{k=1}^\infty \vbraces{\cos(\sqrt{\lambda_k}t) (u_1, v_k)_{L^2(\Omega)}}^2
  \end{align*}

  Durch die Sinus- und Cosinusterme bekommen wir nun keine (erst recht keine exponentiell schnelle) Konvergenz gegen 0.
\end{enumerate}

\end{solution}

% --------------------------------------------------------------------------------
