% --------------------------------------------------------------------------------

\begin{exercise}

Sei $u$ eine klassische Lösung der Telegraphengleichung

\begin{align*}
    \begin{cases}
        u_{tt} + d u_t - \Delta u = 0 & \text{in}~ \Omega, t > 0, \\
        u = 0                         & \text{auf}~ \partial \Omega, t > 0, \\
        u(\cdot, 0) = u_0             & \text{in}~ \Omega, \\
        u_t(\cdot, 0) = u_1           & \text{in}~ \Omega,
    \end{cases}
\end{align*}

wobei $d>0$ konstant, $u_0 \in H_0^1(\Omega), u_1 \in L^2(\Omega)$ und $\Omega \subset \R^n$ ein beschränktes Gebiet mit glattem Rand ist.

\begin{enumerate}[label = (\roman*)]

    \item Zeigen Sie durch eine formale Rechnung, dass die Energie $\Int[\Omega]{(u_t^2 + |\nabla u|^2)}{x}$ uniform beschränkt in $t \in (0, \infty)$ ist.

    \item Bestimmen Sie formal eine Lösung bzgl. eines geeigneten ONS.

    \item Zeigen Sie, dass $\norm[L^2(\Omega)]{u_t}$ exponentiell schnell für $t \rightarrow \infty$ gegen $0$ konvergiert, falls $u_1 = 0$.
    Gilt diese Aussage auch für $d = 0$?

\end{enumerate}

\end{exercise}

% --------------------------------------------------------------------------------

\begin{solution}

\phantom{}

\begin{enumerate}[label = (\roman*)]

    \item Wir gehen analog zu Seite 121 im Skriptum vor.
    
    Dann können wir die Differentialgleichung mit $u_t$ multiplizieren und integrieren und erhalten

    \begin{align*}
        \implies
        0
        =
        \Int[\Omega]{u_{tt} u_t + d u_t^2 - (\Delta u) u_t}{x}
        =
        \Int[\Omega]{u_{tt} u_t}{x}
        +
        d \Int[\Omega]{u_t^2}{x}
        -
        \Int[\Omega]{(\Delta u) u_t}{x}.
    \end{align*}

    \begin{align*}
        \derivative{t} u_t^2 = 2 u_{tt} u_t,
        \quad
        \derivative{t} |\nabla u|^2 = 2 \nabla u \cdot \nabla u_t
    \end{align*}

    Für das linke Integral erhalten wir daher

    \begin{align*}
        \implies
        \Int[\Omega]{u_{tt} u_t}{x}
        =
        \frac{1}{2}
        \derivative{t}
        \Int[\Omega]{u_t^2}{x}.        
    \end{align*}

    Für das rechte Integral erhalten wir (daher)

    \begin{align*}
        \implies
        -\Int[\Omega]{(\Delta u) u_t}{x}
        \stackrel
        {
            \mathrm{PI}
        }{=}
        \Int[\Omega]{\nabla u \cdot \nabla u_t}{x}
        -
        \underbrace
        {
            \Int[\Omega]{u_t (\nabla u \cdot \nu)}{s}
        }_0
        =
        \frac{1}{2}
        \derivative{t}
        \Int[\Omega]{|\nabla u|^2}{x}.
    \end{align*}

    Insgesamt erhalten wir (daher)

    \begin{align*}
        \implies
        \Int[\Omega]{u_{tt} u_t}{x}
        +
        d \Int[\Omega]{u_t^2}{x}
        -
        \Int[\Omega]{(\Delta u) u_t}{x}
        =
        \frac{1}{2}
        \derivative{t}
        \Int[\Omega]{u_t^2 + |\nabla u|^2}{x}
        +
        d \norm[L^2(\Omega)]{u_t(\cdot, t)}.
    \end{align*}

    Die Energie bezeichnen wir mit $E(t)$.

    \begin{align*}
        \implies
        E^\prime(t)
        =
        -2 d \norm[L^2(\Omega)]{u_t(\cdot, t)}
        \leq
        0
    \end{align*}

    Das heißt, dass die Energie höchstens abnimmt.
    Wir berechnen die Anfangs-Energie $E(0)$.

    \begin{align*}
        \implies
        E(0)
        =
        \Int[\Omega]{u_t(x, 0)^2 + |\nabla u(x, 0)|^2}{x}
        =
        \norm[L^2(\Omega)]{u_1}^2 + \norm[L^2(\Omega)]{\nabla u_0}^2
        \leq
        \underbrace
        {
            \norm[L^2(\Omega)]{u_1}^2
        }_{
            < \infty
        }
        +
        \underbrace
        {
            \norm[H^1(\Omega)]{u_0}^2
        }_{
            < \infty
        }
        <
        \infty
    \end{align*}

    Wir benutzen den Hauptsatz der Differential- und Integralrechnung, um zu zeigen, dass die Anfangs-Energie $E(0)$ tatsächlich eine uniforme Schranke für $E(t)$, $t > 0$ ist.

    \begin{align*}
        \implies
        E(t)
        =
        E(0)
        +
        \underbrace
        {
            \Int[0][t]{E^\prime(s)}{s}
        }_{
            \leq 0
        }
        \leq
        E(0)
        <
        \infty
    \end{align*}

\end{enumerate}
    
\end{solution}

% --------------------------------------------------------------------------------
