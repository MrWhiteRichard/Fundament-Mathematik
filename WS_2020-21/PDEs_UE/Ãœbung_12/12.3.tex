% -------------------------------------------------------------------------------- %

\begin{exercise}

Betrachten Sie die lineare Wellengleichung in $\R^3 \times [0, \infty)$

\begin{align*}
    \begin{cases}
        u_{tt} - \Delta u = 0 & \text{für}~ (x, t) \in \R^3 \times [0, \infty), \\
        u(x, 0) = u_0(x)    & \text{für}~ x \in \R^3, \\
        u_t(x, 0) = u_1(x)  & \text{für}~ x \in \R^3.
    \end{cases}
\end{align*}

Leiten Sie die Kirchhoffsche Formel

\begin{align*}
    u(x, t)
    =
    \frac{1}{4 \pi t}
    \Int[\partial B(x, t)]{u_1(y)}{s(y)}
    +
    \pderivative{t}
    \pbraces
    {
        \frac{1}{4 \pi t}
        \Int[\partial B(x, t)]{u_0(y)}{s(y)}
    }
\end{align*}

für die Lösung der Wellengleichung her, wobei $B(x, t)$ die Kugel mit Mittelpunkt $x$ und Radius $t$ ist.

\textit{Hinweis:}
Betrachten Sie für eine Lösung $u \in C^2 \pbraces{\R^3 \times [0, \infty)}$ die Mittelwerte

\begin{align*}
    U(x, r, t) & := \frac{1}{4 \pi r^2} \Int[\partial B(x, r)]{u(y, t)}{s(y)}, \\
    G(x, r)    & := \frac{1}{4 \pi r^2} \Int[\partial B(x, r)]{u_0(y)}{s(y)}, \\
    H(x, r)    & := \frac{1}{4 \pi r^2} \Int[\partial B(x, r)]{u_1(y)}{s(y)}.
\end{align*}

Zeigen Sie, dass für $x \in \R^3$ der Mittelwert $U \in C^2([0, \infty) \times [0, \infty))$  die \textit{Euler-Poisson-Darboux-Gleichung}

\begin{align*}
    \begin{cases}
        U_{tt} - U_{rr} - \frac{2}{r} U_r = 0 & \text{für}~(r, t) \in (0, \infty) \times (0, \infty), \\
        U(r, 0) = G                             & \text{für}~ r \in (0, \infty), \\
        U_t(r, 0)                           = H & \text{für}~ r \in (0, \infty),
    \end{cases}
\end{align*}

erfüllt und $\tilde U := r U$ die Differentialgleichung

\begin{align*}
    \begin{cases}
        \tilde U_{tt} - \tilde U_{rr} = 0 & \text{für}~ (r, t) \in (0, \infty) \times (0, \infty), \\
        \tilde U(r, 0) = r G              & \text{für}~ r \in (0, \infty), \\
        \tilde U_t(r, 0) = r H            & \text{für}~ r \in (0, \infty), \\
        \tilde U(0, t) = 0                & \text{für}~ t \in (0, \infty).
    \end{cases}
\end{align*}

\end{exercise}

% -------------------------------------------------------------------------------- %

\begin{solution}

Eine $1000$ Formeln sagen mehr als $1000$ Worte!

\begin{enumerate}[label = \arabic*.]

    \item Schritt (Euler-Poisson-Darboux-Gleichung):

    \begin{enumerate}[label = 1.\arabic*.]

        \item Schritt ($\Forall (r, t) \in (0, \infty) \times (0, \infty): U_{tt} - U_{rr} - \frac{2}{r} U_r = 0$):

        \begin{align*}
            \implies
            U(x, t, r)
            =
            \frac{1}{4 \pi r^2}
            \Int[\partial B(x, r)]{u(y, t)}{s(y)}
            \stackrel
            {
                \mathrm{TRAFO}
            }{=}
            \frac{1}{4 \pi}
            \Int[\partial B(0, 1)]{u(x + r z, t)}{s(z)}
        \end{align*}

        \begin{align*}
            \implies
            U_r(x, t, r)
            & =
            \frac{1}{4 \pi}
            \Int[\partial B(0, 1)]{\nabla u(x + r z, t) \cdot z}{s(z)} \\
            & \stackrel
            {
                \mathrm{TRAFO}
            }{=}
            \frac{1}{4 \pi r^2}
            \Int[\partial B(x, r)]{\nabla u(y, t) \cdot \frac{y - x}{r}}{s(y)} \\
            & =
            \frac{1}{4 \pi r^2}
            \Int[\partial B(x, r)]{\nabla u(y, t) \cdot \nu}{s(y)} \\
            & \stackrel
            {
                \mathrm{Gauß}
            }{=}
            \frac{1}{4 \pi r^2}
            \Int[B(x, r)]{\Div \nabla u(y, t)}{y} \\
            & =
            \frac{1}{4 \pi r^2}
            \Int[B(x, r)]{\Delta u(y, t)}{y} \\
            & =
            \frac{1}{4 \pi r^2}
            \Int[B(x, r)]{u_{tt}(y, t)}{y} \\
            & \stackrel
            {
                \mathrm{KOFLFO}
            }{=}
            \frac{1}{4 \pi r^2}
            \Int[0][r]
            {
                \Int[\partial B(x, R)]{u_{tt}(y, t)}{y}
            }{R}
        \end{align*}

        \begin{align*}
            \implies
            r^2 U_r(x, f, t)
            =
            \frac{1}{4 \pi}
            \Int[0][r]
            {
                \Int[\partial B(x, R)]{u_{tt}(y, t)}{y}
            }{R}
        \end{align*}

        \begin{multline*}
            \implies
            2 r U_r(x, r, t) + r^2 U_{rr}(x, r, t)
            =
            (r^2 U_r(x, r, t))_r \\
            =
            \frac{1}{4 \pi}
            \Int[\partial B(x, r)]{u_{tt}(y, t)}{s(y)}
            =
            \partial_{tt}
            \frac{1}{4 \pi}
            \Int[\partial B(x, r)]{u(y, t)}{s(y)}
        \end{multline*}

        \begin{align*}
            \implies
            \frac{2}{r}
            U_r(x, r, t) + U_{rr}(x, r, t)
            =
            \partial_{tt}
            \frac{1}{4 \pi r^2}
            \Int[\partial B(x, r)]{u(y, t)}{s(y)}
            =
            U_{tt}(x, r, t)
        \end{align*}

        \item Schritt ($\Forall r \in (0, \infty): U(r, 0) = G$):

        \begin{align*}
            \implies
            U(x, r, 0)
            =
            \frac{1}{4 \pi r^2}
            \Int[\partial B(x, r)]{u(y, 0)}{s(y)}
            =
            \frac{1}{4 \pi r^2}
            \Int[\partial B(x, r)]{u_0(y)}{s(y)}
            =
            G(x, r)
        \end{align*}

        \item Schritt ($\Forall r \in (0, \infty): U_t(r, 0) = H$):

        \begin{multline*}
            \implies
            U_t(x, r, 0)
            =
            \bbraces
            {
                \partial_t
                \frac{1}{4 \pi r^2}
                \Int[\partial B(x, r)]{u(y, t)}{s(y)}
            }_{t = 0}
            =
            \bbraces
            {
                \frac{1}{4 \pi r^2}
                \Int[\partial B(x, r)]{u_t(y, t)}{s(y)}
            }_{t = 0} \\
            =
            \frac{1}{4 \pi r^2}
            \Int[\partial B(x, r)]{u_t(y, 0)}{s(y)}
            =
            \frac{1}{4 \pi r^2}
            \Int[\partial B(x, r)]{u_1(y)}{s(y)}
            =
            H(x, r)
        \end{multline*}

    \end{enumerate}

    \item Schritt (homogene eindimensionale Wellengleichung):

    \begin{align*}
        \tilde U := r U,
        \quad
        \tilde G := r G,
        \quad
        \tilde H := r H
    \end{align*}

    \begin{enumerate}[label = 2.\arabic*.]

        \item Schritt ($\Forall (r, t) \in (0, \infty) \times (0, \infty): \tilde U_{tt} - \tilde U_{rr} = 0$):

        \begin{align*}
            \implies
            \tilde U_{tt}
            =
            r U_{tt}
            \stackrel
            {
                \mathrm{1.1}
            }{=}
            r \pbraces{U_{rr} + \frac{2}{r} U_r}
            =
            r U_{rr} + 2 U_r
            =
            (U + r U_r)_r
            =
            \tilde U_{rr}
        \end{align*}

        \item Schritt ($\Forall r \in (0, \infty): \tilde U(r, 0) = \tilde G$):

        \begin{align*}
            \implies
            \tilde U(r, 0)
            =
            r U(r, 0)
            \stackrel
            {
                \mathrm{1.2}
            }{=}
            r G(r, 0)
            =
            \tilde G(r, 0)
        \end{align*}

        \item Schritt ($\Forall r \in (0, \infty): \tilde U_t(r, 0) = \tilde H$):

        \begin{align*}
            \implies
            \tilde U_t(r, 0)
            =
            r U_t(r, 0)
            \stackrel
            {
                \mathrm{1.3}
            }{=}
            r H(r, 0)
            =
            \tilde H(r, 0)
        \end{align*}

        \item Schritt ($\Forall t \in (0, \infty): \tilde U(0, t) = 0$):

        \begin{align*}
            \implies
            \tilde U(0, t)
            =
            0 \cdot U(0, t)
            =
            0
        \end{align*}

    \end{enumerate}

    \item Schritt:

    \includegraphicsboxed{PDEs/PDEs_-_(7-1).png}
    \includegraphicsboxed{PDEs/PDEs_-_(7-2).png}

    \begin{multline*}
        \implies
        \tilde U(x, r, t)
        =
        \frac{1}{2}
        \pbraces
        {
            \tilde G(x, t + r)
            -
            \tilde G(x, t - r)
        }
        +
        \frac{1}{2}
        \Int[t-r][t+r]{\tilde H(x, y)}{y} \\
        =
        \frac{1}{2}
        \pbraces
        {
            \tilde G(x, t + r)
            -
            \tilde G(x, t - r)
        }
        +
        \frac{1}{2}
        \pbraces
        {
            \Int[0][t+r]{\tilde H(x, y)}{y}
            -
            \Int[0][t-r]{\tilde H(x, y)}{y}
        }
    \end{multline*}

    \begin{align*}
        \implies
        u(x, t)
        & =
        \frac{1}{4 \pi}
        u(x, t)
        \Int[\partial B(0, 1)]{}{s(z)} \\
        & =
        \lim_{r \to 0}
        \frac{1}{4 \pi}
        \Int[\partial B(0, 1)]{u(x + r z, t)}{s(z)} \\
        & =
        \lim_{r \to 0}
        U(x, r, t) \\
        & =
        \lim_{r \to 0}
        \frac{\tilde U(x, r, t)}{r} \\
        & =
        \lim_{r \to 0}
        \pbraces
        {
            \frac{1}{2 r}
            \pbraces
            {
                \tilde G(x, t + r)
                -
                \tilde G(x, t - r)
            }
            +
            \frac{1}{2 r}
            \pbraces
            {
                \Int[0][t+r]{\tilde H(x, y)}{y}
                -
                \Int[0][t-r]{\tilde H(x, y)}{y}
            }
        } \\
        & =
        \partial_t \tilde G(x, t) + \tilde H(x, t) \\
        & =
        \frac{1}{4 \pi t}
        \Int[\partial B(x, t)]{u_1(y)}{s(y)}
        +
        \pderivative{t}
        \pbraces
        {
            \frac{1}{4 \pi t}
            \Int[\partial B(x, t)]{u_0(y)}{s(y)}
        }
    \end{align*}

\end{enumerate}

\end{solution}

% -------------------------------------------------------------------------------- %
