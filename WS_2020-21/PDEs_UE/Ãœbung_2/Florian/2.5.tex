% --------------------------------------------------------------------------------

\begin{exercise}

Lösen Sie folgendes Problem mithilfe der Methode der Charakteristiken und geben Sie an, wo die Lösung definiert ist:

\begin{align*}
    (y + u) u_x + y u_y
    & =
    x - y, \\
    u(x, 1)
    & =
    1 + x
\end{align*}

\end{exercise}

\begin{solution}

Wie bereits in den vorigen beiden Aufgaben lösen wir

\begin{align*}
    \pderivative[][x]{s} = y + u \quad \text{mit} \quad x(0,t) = t, \qquad
    \pderivative[][y]{s} = y \quad \text{mit} \quad y(0,t) = 1, \qquad
    \pderivative[][u]{s} = x - y \quad \text{mit} \quad u(0,t) = 1 + t .
\end{align*}

Die zweite Differentialgleichung lässt sich leicht lösen, nämlich ist $y(s,t) = e^s$ eine Lösung die zudem noch die zusätzliche Bedingung erfüllt. Die anderen beiden Differentialgleichungen wollen wir ein weiteres Mal ableiten und erhalten so

\begin{align*}
    \pderivative[2][x]{s} = \pderivative[][y]{s} + \pderivative[][u]{s} = y  + x - y = x \quad \text{und} \quad \pderivative[2][u]{s} = \pderivative[][x]{s} - \pderivative[][y]{s} = y + u - y = u,
\end{align*}

also zweimal die gleiche homogene Differentialgleichung. Diese haben das charakteristische Polynom $\chi(\lambda) = \lambda^2 -1$ also die Lösung $x(s,t) = c_1 e^s + c_2 e^{-s}$ und $u(s,t) = c_3 e^s + c_4 e^{-s}$, wobei die scheinbaren Konstanten noch von $t$ abhängen können. Nun wissen wir außerdem

\begin{align*}
    c_1 e^s - c_2 e^{-s} = \pderivative[][x]{s}(s)\stackrel{!}{=} y + u = e^s + c_3 e^s + c_4 e^{-s}, \quad \text{also} \quad c_1 = 1 + c_3 \quad \text{und} \quad -c_2 = c_4
\end{align*}

Diese beiden Gleichungen können wir nützen und erhalten

\begin{align*}
    t = x(0,t) = c_1 + c_2 \quad \text{und} \quad 1 + t = u(0,t) = c_3 + c_4 = c_1 - 1 - c_2
\end{align*}

Das Lösen dieses Gleichungssystems ist nun wirklich nicht mehr schwer und wir erhalten

\begin{align*}
    c_1 = 1 + t, \quad c_2 = -1, \quad c_3 = t \quad \text{und} \quad c_4 = 1
\end{align*}

also die Funktionen

\begin{align*}
    x(s,t) = (1 + t) e^s - e^{-s} \quad \text{und} \quad u(s,t) = t e^s + e^{-s}
\end{align*}

Aus $y = e^s$ erhalten wir schnell $s = \ln(y)$ und damit

\begin{align*}
    x = (1+t) e^s - e^{-s} = (1 + t)y - y^{-1} \quad \text{umgeformt also} \quad t = \frac{x}{y} + y^{-2} - 1
\end{align*}

woraus sich insgesamt

\begin{align*}
    u(x,y) = t e^s + e^{-s} = \left(\frac{x}{y} + y^{-2} - 1 \right) y + y^{-1} = x + \frac{2}{y} - y
\end{align*}

ergibt.
Die Lösung ist auf $\R \times \R \setminus \{0\}$ wohldefiniert.

\end{solution}

% --------------------------------------------------------------------------------
