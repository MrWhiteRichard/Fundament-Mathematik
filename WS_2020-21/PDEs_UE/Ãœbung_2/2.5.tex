% --------------------------------------------------------------------------------

\begin{exercise}

Lösen Sie folgendes Problem mithilfe der Methode der Charakteristiken und geben Sie an, wo die Lösung definiert ist:

\begin{align*}
    (y + u) u_x + y u_y
    & =
    x - y, \\
    u(x, 1)
    & =
    1 + x
\end{align*}

\end{exercise}

% --------------------------------------------------------------------------------

%\begin{solution}
%
%Das dazugehorige Anfangswertproblem lautet
%
%\begin{align*}
%    \pderivative[][x]{s} = y + u,
%    \pderivative[][y]{s} = y,
%    \pderivative[][u]{s} = x - y,
%\end{align*}
%
%mit Anfangswerten
%
%\begin{align*}
%    x(0, t) = t,
%    y(0, t) = 1,
%    u(0, t) = 1 + t.
%\end{align*}
%
%Anstatt sofort ein lineares $(3 \times 3)$-System aufzustellen, bemerken wir zunächst, dass $y(s, t) = e^s$.
%Wir benötigen also bloß folgendes inhomogene lineare $(2 \times 2)$-System (mit konstanten koeffizienten im linearen Teil).
%
%\begin{align*}
%    v(s)
%    :=
%    \begin{pmatrix}
%        x(s, t) \\ u(s, t)
%    \end{pmatrix}
%    \implies
%    v^\prime
%    =
%    \underbrace
%    {
%        \begin{pmatrix}
%            0 & 1 \\
%            1 & 0
%        \end{pmatrix}
%    }_{=: A} v
%    +
%    \underbrace
%    {
%        \begin{pmatrix}
%            e^s \\ -e^s
%        \end{pmatrix}
%    }_{=: b(s)},
%    \quad
%    v(0)
%    =
%    \begin{pmatrix}
%        t \\ 1 + t
%    \end{pmatrix}
%\end{align*}
%
%Wir bestimmen die Eigenwerte $\lambda_{1, 2}$.
%
%\begin{align*}
%    & \implies
%    \chi_A(\lambda)
%    =
%    \begin{vmatrix}
%        -\lambda & 1 \\
%         1       & -\lambda
%    \end{vmatrix}
%    =
%    \lambda^2 - 1^2
%    =
%    (\lambda + 1)
%    (\lambda - 1) \\
%    & \implies
%    \lambda_{1, 2} = \pm 1
%\end{align*}
%
%Wir bestimmen die zugehörigen Eigenvektoren $v_{1, 2}$.
%
%\begin{align*}
%    A - \lambda_1
%    =
%    \begin{pmatrix}
%        -1 &  1 \\
%         1 & -1
%    \end{pmatrix}
%    \mapsto
%    \begin{pmatrix}
%        1 & -1 \\
%        0 &  0
%    \end{pmatrix}
%    & \implies
%    v_1 \in
%    \Span \Bbraces
%    {
%        \begin{pmatrix}
%            1 \\ 1
%        \end{pmatrix}
%    } \\
%    A - \lambda_2
%    =
%    \begin{pmatrix}
%        1 & 1 \\
%        1 & 1
%    \end{pmatrix}
%    \mapsto
%    \begin{pmatrix}
%        1 & 1 \\
%        0 & 0
%    \end{pmatrix}
%    & \implies
%    v_2 \in
%    \Span \Bbraces
%    {
%        \begin{pmatrix}
%            1 \\ -1
%        \end{pmatrix}
%    }
%\end{align*}
%
%Wir fassen zusammen:
%
%\begin{align*}
%    \Lambda := \diag{(\lambda_1, \lambda_2)},
%    \quad
%    V := (v_1, v_2)
%\end{align*}
%
%Wir berechnen die Inverse von $V$.
%
%\begin{align*}
%    \begin{bmatrix}
%        V \\
%        \hline
%        I_2
%    \end{bmatrix}
%    =
%    \begin{bmatrix}
%        1 &  1 \\
%        1 & -1 \\
%        \hline
%        1 & 0 \\
%        0 & 1
%    \end{bmatrix}
%    \mapsto
%    \begin{bmatrix}
%        1 & 2 \\
%        1 & 0 \\
%        \hline
%        1 & 1 \\
%        1 & 0
%    \end{bmatrix}
%    \mapsto
%    \begin{bmatrix}
%        2 & 1 \\
%        0 & 1 \\
%        \hline
%        1 & 1 \\
%        1 & 0
%    \end{bmatrix}
%    \mapsto
%    \begin{bmatrix}
%        1 & 1 \\
%        0 & 1 \\
%        \hline
%        1/2 & 1 \\
%        1/2 & 0
%    \end{bmatrix}
%    \mapsto
%    \begin{bmatrix}
%        1 & 0 \\
%        0 & 1 \\
%        \hline
%        1/2 &  1/2 \\
%        1/2 & -1/2
%    \end{bmatrix}
%    =
%    \begin{bmatrix}
%        I_2 \\
%        \hline
%        V^{-1}
%    \end{bmatrix}
%\end{align*}
%
%Somit, können wir die Exponentialmatrix von $A$ ausrechnen.
%
%\begin{align*}
%    \implies
%    e^{sA}
%    =
%    V s^{s \Lambda} V^{-1}
%    & =
%    \begin{pmatrix}
%        1 &  1 \\
%        1 & -1
%    \end{pmatrix}
%    \begin{pmatrix}
%        e^{s \lambda_1} & 0 \\
%        0               & e^{s \lambda_2}
%    \end{pmatrix}
%    \frac{1}{2}
%    \begin{pmatrix}
%        1 &  1 \\
%        1 & -1
%    \end{pmatrix} \\
%    & =
%    \begin{pmatrix}
%        e^s &  e^{-s} \\
%        e^s & -e^{-s}
%    \end{pmatrix}
%    \frac{1}{2}
%    \begin{pmatrix}
%        1 &  1 \\
%        1 & -1
%    \end{pmatrix}
%    =
%    \frac{1}{2}
%    \begin{pmatrix}
%        e^s + e^{-s} & e^s - e^{-s} \\
%        e^s - e^{-s} & e^s + e^{-s}
%    \end{pmatrix}
%\end{align*}
%
%Nachdem ein Fundamentalsystem nach Spaltenumformungen wieder ein Fundamentalsystem bleibt, werden wir die Exponentialmatrix transformieren.
%Das soll die folgenden Rechnungen vereinfachen.
%
%\begin{align*}
%    e^{sA}
%    \mapsto
%    \begin{pmatrix}
%        e^s & -e^{-s} / s \\
%        e^s &  e^{-s} / 2
%    \end{pmatrix}
%    \mapsto
%    \begin{pmatrix}
%        e^s & -e^{-s} \\
%        e^s &  e^{-s}
%    \end{pmatrix}
%    =: Y(s)
%\end{align*}
%
%Die Inverse bekommen wir mit dem Determinanten-Trick.
%
%\begin{align*}
%    Y(s)^{-1}
%    =
%    \begin{pmatrix}
%         e^{-s} & e^{-s} \\
%        -e^s    & e^s
%    \end{pmatrix}
%\end{align*}
%
%Damit, können wir die Partikulärlösung nun konkret hinschreiben.
%
%\begin{align*}
%    \implies
%    \begin{pmatrix}
%        x(s, t)  \\ u(s, t)
%    \end{pmatrix}
%    =
%    v(s)
%    =
%    Y(s)
%    \pbraces
%    {
%        Y(0)^{-1} v_0
%        +
%        \Int[0][s]{Y(\xi)^{-1} b(\xi)}{\xi}
%    }
%    =
%    \begin{pmatrix}
%        t e^s + e^s - e^{-s} \\
%        t e^s + e^{-s}
%    \end{pmatrix}
%\end{align*}
%
%Wir erhalten insgesamt also folgende Lösung unserer ODE.
%
%\begin{align*}
%    x(s, t) = t e^s + e^s - e^{-s},
%    \quad
%    y(s, t) = e^s,
%    \quad
%    u(s, t) = t e^s + e^{-s}
%\end{align*}
%
%Wir transformieren $(s, t) \mapsto (x, y)$, indem wir die ersten beiden Gleichungen nach $s$ und $t$ auflösen und in die dritte einsetzen.
%Das funktioniert, weil
%
%\begin{align*}
%    \det \pderivative[][(x, y)]{(s, t)}
%    =
%    \begin{vmatrix}
%        x_s & x_t \\
%        y_s & y_t
%    \end{vmatrix}
%    =
%    \begin{vmatrix}
%        t e^s + e^s + e^{-s} & e^s \\
%        e^s                  & 0
%    \end{vmatrix}
%    =
%    -e^{2s} \neq 0.
%\end{align*}
%
%Konkret ...
%
%\begin{align*}
%    & \implies
%    s = \ln{y} \\
%    & \implies
%    t = x/y - 1 + 1/y^2 \\
%    & \implies
%    u = x - y + 2/y
%\end{align*}
%
%Die Lösung ist, laut unseren Rechnungen, auf $\R \times \R^+$ definiert.
%\blockquote{Zufälligerweise} gilt sie sogar auf $\R \times (\R \setminus \Bbraces{0})$.
%
%\end{solution}
%
%% --------------------------------------------------------------------------------

\begin{solution}

Wie bereits in den vorigen beiden Aufgaben lösen wir

\begin{align*}
    \pderivative[][x]{s} = y + u \quad \text{mit} \quad x(0,t) = t, \qquad
    \pderivative[][y]{s} = y \quad \text{mit} \quad y(0,t) = 1, \qquad
    \pderivative[][u]{s} = x - y \quad \text{mit} \quad u(0,t) = 1 + t .
\end{align*}

Die zweite Differentialgleichung lässt sich leicht lösen, nämlich ist $y(s,t) = e^s$ eine Lösung die zudem noch die zusätzliche Bedingung erfüllt. Die anderen beiden Differentialgleichungen wollen wir ein weiteres Mal ableiten und erhalten so

\begin{align*}
    \pderivative[2][x]{s} = \pderivative[][y]{s} + \pderivative[][u]{s} = y  + x - y = x \quad \text{und} \quad \pderivative[2][u]{s} = \pderivative[][x]{s} - \pderivative[][y]{s} = y + u - y = u,
\end{align*}

also zweimal die gleiche homogene Differentialgleichung. Diese haben das charakteristische Polynom $\chi(\lambda) = \lambda^2 -1$ also die Lösung $x(s,t) = c_1 e^s + c_2 e^{-s}$ und $u(s,t) = c_3 e^s + c_4 e^{-s}$, wobei die scheinbaren Konstanten noch von $t$ abhängen können. Nun wissen wir außerdem

\begin{align*}
    c_1 e^s - c_2 e^{-s} = \pderivative[][x]{s}(s)\stackrel{!}{=} y + u = e^s + c_3 e^s + c_4 e^{-s}, \quad \text{also} \quad c_1 = 1 + c_3 \quad \text{und} \quad -c_2 = c_4
\end{align*}

Diese beiden Gleichungen können wir nützen und erhalten

\begin{align*}
    t = x(0,t) = c_1 + c_2 \quad \text{und} \quad 1 + t = u(0,t) = c_3 + c_4 = c_1 - 1 - c_2
\end{align*}

Das Lösen dieses Gleichungssystems ist nun wirklich nicht mehr schwer und wir erhalten

\begin{align*}
    c_1 = 1 + t, \quad c_2 = -1, \quad c_3 = t \quad \text{und} \quad c_4 = 1
\end{align*}

also die Funktionen

\begin{align*}
    x(s,t) = (1 + t) e^s - e^{-s} \quad \text{und} \quad u(s,t) = t e^s + e^{-s}
\end{align*}

Aus $y = e^s$ erhalten wir schnell $s = \ln(y)$ und damit

\begin{align*}
    x = (1+t) e^s - e^{-s} = (1 + t)y - y^{-1} \quad \text{umgeformt also} \quad t = \frac{x}{y} + y^{-2} - 1
\end{align*}

woraus sich insgesamt

\begin{align*}
    u(x,y) = t e^s + e^{-s} = \left(\frac{x}{y} + y^{-2} - 1 \right) y + y^{-1} = x + \frac{2}{y} - y
\end{align*}

ergibt.
Die Lösung ist auf $\R \times \R \setminus \{0\}$ wohldefiniert.
\end{solution}

% --------------------------------------------------------------------------------


\begin{solution}
Wie bereits in den vorigen beiden Aufgaben lösen wir

\begin{align*}
    \pderivative[][x]{s} = y + u \quad \text{mit} \quad x(0,t) = t, \qquad
    \pderivative[][y]{s} = y \quad \text{mit} \quad y(0,t) = 1, \qquad
    \pderivative[][u]{s} = x - y \quad \text{mit} \quad u(0,t) = 1 + t .
\end{align*}

Wir haben hier also ein System gewöhnlicher Differentialgleichungen:

\begin{align*}
  A =\left(
  \begin{array}{ccc}
  0 & 1 & 1 \\
  0 & 1 & 0 \\
  1 & -1 & 0
  \end{array}
  \right) \\
  \left(
  \begin{array}{c}
    x_s \\
    y_s \\
    u_s
  \end{array}
  \right) =
  A
  \left(
  \begin{array}{c}
    x \\
    y \\
    u
  \end{array}
  \right)
\end{align*}

Wir wissen, dass $e^{sA}$ ein Fundamentalsystem bildet. Um dieses zu berechnen,
sehen wir uns die Diagonalzerlegung von $A$ (mithilfe eines Matrizenrechners) an:

\begin{align*}
  A =
  \left(
  \begin{array}{ccc}
    1 & 1 & -1 \\
    1 & 0 & 0 \\
    0 & 1 & 1
  \end{array}
  \right)
  \left(
  \begin{array}{ccc}
    1 & & \\
    & 1 & \\
    & & -1
  \end{array}
  \right)
  \left(
  \begin{array}{ccc}
  0 & 1 & 0 \\
  \frac{1}{2} & - \frac{1}{2} & \frac{1}{2} \\
  - \frac{1}{2} & \frac{1}{2} & \frac{1}{2}
  \end{array}
  \right)
\end{align*}

Dann gilt:

\begin{align*}
  e^{sA} = V
  \left(
  \begin{array}{ccc}
  e^s & & \\
  & e^s & \\
  & & e^{-s}
  \end{array}
  \right) V^{-1}
  =
  \left(
  \begin{array}{ccc}
    \frac{e^s + e^{-s}}{2} & \frac{e^s - e^{-s}}{2} & \frac{e^s - e^{-s}}{2} \\
    0 & e^s & 0 \\
    \frac{e^s - e^{-s}}{2} & \frac{e^{-s}-e^s}{2} & \frac{e^s + e^{-s}}{2}
  \end{array}
  \right)
\end{align*}

Unsere Partikulärlösung ist schließlich gegeben durch:

\begin{align*}
\left(
\begin{array}{c}
  x \\
  y \\
  u
\end{array}
\right)(s)
=
e^{sA} \left(
\begin{array}{c}
  t \\
  1 \\
  1+t
\end{array}
\right)
=
\left(
\begin{array}{c}
  te^s + e^s - e^{-s} \\
  e^s \\
  te^s +e^{-s}
\end{array}
\right)
\end{align*}

Also
\begin{align*}
  x(s,t) = te^s + e^s - e^{-s} \\
  y(s,t) = e^s \\
  u(s,t) =   te^s +e^{-s}
\end{align*}

Wir können Rücktransformieren da

\begin{align*}
    \det \pderivative[][(x, y)]{(s, t)}
    =
    \begin{vmatrix}
        x_s(0,t) & x_t(0,t) \\
        y_s(0,t) & y_t(0,t)
    \end{vmatrix}
    =
    \begin{vmatrix}
        t & 1 \\
        1 & 0
    \end{vmatrix}
    =
    -1 \neq 0.
\end{align*}

Wir sehen direkt $ s = \ln(y)$, also auch

\begin{align*}
  x = y(t+1) - \frac{1}{y}
  \Rightarrow
  t = \frac{x}{y} + \frac{1}{y^2} - 1
\end{align*}

Setzen wir nun in $u(s,t)$ ein erhalten wir unsere Lösungsfunktion die auf
$\R \times \R \setminus \{0 \}$ wohldefiniert ist:

\begin{align*}
  u(x,y) = x + \frac{2}{y} - y
\end{align*}
\end{solution}
