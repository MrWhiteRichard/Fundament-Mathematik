% --------------------------------------------------------------------------------

\begin{exercise}

Sei $\Omega = (0,1)$. Lösen Sie das folgende Problem: Gesucht ist ein $u \in H_0^1(\Omega)$ mit
\begin{align*}
  \int_\Omega a(x)u^{\prime}(x)v^{\prime}(x) - v(x) dx = 0 \text{ für alle } v \in H_0^1(\Omega),
\end{align*}
wobei $a(x) = \1_{(0,\frac{1}{2}]}(x) + 2\cdot\1_{(\frac{1}{2},1)}(x)$.
\end{exercise}

% --------------------------------------------------------------------------------

\begin{solution}
Motivation: Für $u \in C^1(\Omega) \cap H^2(\Omega)$ und $v \in H_0^1(\Omega)$ gilt
mit partieller Integration für Sobolevfunktionen, sowie der Tatsache, dass aufgrund
des Sobolevschen Einbettungssatzes $v \in C(\overline{\Omega})$ ist und somit
$Tv = v|\partial\Omega, Tu = u|\partial\Omega$:
\begin{align*}
  \int_\Omega a(x)u^{\prime}(x)v^{\prime}(x) &= \int_\Omega v(x) dx \\
  \iff \int_0^{\frac{1}{2}} u^{\prime}(x)v^{\prime}(x)dx + 2\int_{\frac{1}{2}}^1 u^{\prime}(x)v^{\prime}(x) dx &= \int_0^1 v(x) dx \\
  \iff u^{\prime}\pbraces{\frac{1}{2}-}v\pbraces{\frac{1}{2}} - 2u^{\prime}\pbraces{\frac{1}{2}+}v\pbraces{\frac{1}{2}} -
  2\int_{\frac{1}{2}}^1 u^{\primeprime}(x)v(x) dx - \int_0^{\frac{1}{2}}u^{\primeprime}(x)v(x) &= \int_0^1 v(x) dx \\
  \iff u^{\prime}\pbraces{\frac{1}{2}-}v\pbraces{\frac{1}{2}} - 2u^{\prime}\pbraces{\frac{1}{2}+}v\pbraces{\frac{1}{2}} - 2\int_{\frac{1}{2}}^1 u^{\primeprime}(x)v(x) dx - \int_0^{\frac{1}{2}}u^{\primeprime}(x)v(x) &= \int_0^1 v(x) dx \\
  \iff \int_0^1 v(x)(a(x)u^{\primeprime}(x) + 1) dx &=
  \left(u^{\prime}\pbraces{\frac{1}{2}-}- 2u^{\prime}\pbraces{\frac{1}{2}-}\right)v\pbraces{\frac{1}{2}}
\end{align*}
Ansatz: Suche $u \in C(\Omega) \subset H_1(\Omega)$, sodass $u^{\primeprime}(x)a(x) + 1 = 0$ punktweise, sowie $u(0) = u(1) = 0, u^{\prime}(1/2-) - 2u^{\prime}(1/2+) = 0$.
\begin{align*}
  u|_{[0,1/2]}(x) = -\int\int \frac{1}{a(z)} dz dy  &=
  -\int\int 1 dz dy  \\
  &= -\int y + C dy \\
  &= -\frac{x^2}{2} + C_1x + D_1
\end{align*}
Bestimme nun die Konstanten:
\begin{align*}
  0 &\stackrel{!}{=} u(0) = D_1 \iff D_1 = 0 \\
  u\pbraces{\frac{1}{2}-} &= -\frac{1}{8} + \frac{C_1}{2} \\
  u^{\prime}\pbraces{\frac{1}{2}-} &= -\frac{1}{2} + C_1
\end{align*}
\begin{align*}
  u|_{[1/2,1]}(x) = -\int\int \frac{1}{a(z)} dz dy  &=
  -\int\int \frac{1}{2} dz dy  \\
  &= -\int \frac{y}{2} + C_2 dy \\
  &= -\frac{x^2}{4} + C_2x + D_2
\end{align*}
\begin{align*}
  0 &\stackrel{!}{=} u(1) = -\frac{1}{4} + C_2 + D_2 \iff C_2 + D_2 = \frac{1}{4} \\
  0 &\stackrel{!}{=} u^{\prime}\pbraces{\frac{1}{2}-} - 2u^{\prime}\pbraces{\frac{1}{2}} =
  -\frac{1}{2} + C_1 + \frac{1}{2} - 2C_2 \iff C_1 - 2C_2 = 0 \\
  0 &\stackrel{!}{=} u\pbraces{\frac{1}{2}-} - u\pbraces{\frac{1}{2}+}
  = -\frac{1}{8} + \frac{C_1}{2} + \frac{1}{16} - \frac{C_2}{2} - D_2
  \iff C_1 - C_2 - 2D_2 = \frac{1}{8} \\
  &\iff 2C_2 - C_2 -2\left(\frac{1}{4} - C_2\right) = \frac{1}{8}
  \iff 3C_2  = \frac{5}{8} \iff C_2 = \frac{5}{24} \\
  &\iff C_1 = \frac{5}{12} \iff D_2 = \frac{1}{24}
\end{align*}
Damit lautet unser Lösungskandidat
\begin{align*}
  u(x) &= \begin{cases}
    -\frac{x^2}{2} + \frac{5x}{12}, & 0\leq x \leq \frac{1}{2} \\
    -\frac{x^2}{4} + \frac{5x}{24} + \frac{1}{24}, & \frac{1}{2} < x \leq 1.
  \end{cases} \\
  u^{\prime}(x) &= \begin{cases}
    -x + \frac{5}{12}, & 0\leq x \leq \frac{1}{2} \\
    -\frac{x}{2} + \frac{5}{24}, & \frac{1}{2} < x \leq 1.
  \end{cases} \\
\end{align*}
Da $u$ stetig ist, stimmt zumindest die erste distributionelle Ableitung
mit der klassischen Ableitung überein. \\
Berechne nun die distributionelle Ableitung von $u^{\prime}(x)$:
\begin{align*}
  \langle u^{\primeprime}, \phi \rangle &= \langle u^{\prime}, -\phi^{\prime}\rangle
  = -\int_0^1 u^{\prime}(x)\phi^{\prime}(x) dx =
  \frac{1}{24}\left(\int_0^{\frac{1}{2}}(4x-2)\phi^{\prime}(x) dx +
  \int_{\frac{1}{2}}^1(2x-1)\phi^{\prime}(x) dx\right) \\
  &= \frac{1}{4}\left([(4x-2)\phi(x)]_{x=0}^{\frac{1}{2}} + [(2x-1)\phi(x)]_{x=1/2}^{1} -
  4\int_0^{\frac{1}{2}}\phi(x) - 2\int_{\frac{1}{2}}^1\phi(x)dx\right) \\
  &= -\int_0^1 \frac{1}{a(x)}\phi(x) dx.
\end{align*}
Also ist $u \in H_0^2(\Omega)$, es gilt $a(x)u^{\primeprime}(x) + 1 = 0$
fast überall. Damit können wir alle Schritte aus der Motivation genauso wieder
zurückgehen und erhalten
\begin{align*}
  \int_0^1 v(x)(a(x)u^{\primeprime}(x) + 1) dx &= 0 \\
  \iff - \int_0^{\frac{1}{2}}u^{\primeprime}(x)v(x) -2\int_{\frac{1}{2}}^1 u^{\primeprime}(x)v(x) dx
  &= \int_0^1 v(x) dx \\
   \iff \underbrace{u^{\prime}\pbraces{\frac{1}{2}}v\pbraces{\frac{1}{2}} - u^{\prime}\pbraces{\frac{1}{2}}v\pbraces{\frac{1}{2}}}_{=0} - \int_0^{\frac{1}{2}}u^{\primeprime}(x)v(x)
  -2 \int_{\frac{1}{2}}^1 u^{\primeprime}(x)v(x) dx  &=
  \int_0^1 v(x) dx \\
  \stackrel{PI}{\iff} \int_0^{\frac{1}{2}} u^{\prime}(x)v^{\prime}(x)dx +
  2\int_{\frac{1}{2}}^1 u^{\prime}(x)v^{\prime}(x) dx &= \int_0^1 v(x) dx \\
  \iff \int_\Omega a(x)u^{\prime}(x)v^{\prime}(x) &= \int_\Omega v(x) dx
\end{align*}
\end{solution}

% --------------------------------------------------------------------------------

\begin{solution}

Wir stellen zunächst einen Ansatz für $u$ auf, indem wir partiell integrieren (streng genommen nach Satz 5.8 (Gauß für Sobolev-Funktionen)).

\begin{align*}
  0
  & \stackrel{!}{=}
  \Int[\Omega]{a(x) u^\prime(x) v^\prime(x) - v(x)}{x}
  =
  \Int[0][\frac{1}{2}]{u^\prime(x) v^\prime(x)}{x}
  +
  \Int[\frac{1}{2}][1]{2 u^\prime(x) v^\prime(x)}{x}
  -
  \Int[\Omega]{v(x)}{x} \\
  & \stackrel
  {
    \mathrm{PI}
  }{=}
  u^\prime(x) v(x) \Big |_{x=0}^{\frac{1}{2}}
  -
  \Int[0][\frac{1}{2}]{u^\primeprime(x) v(x)}{x}
  +
  2 u^\prime(x) v(x) \Big |_{x=\frac{1}{2}}^1
  -
  \Int[\frac{1}{2}][1]{2 u^\primeprime(x) v(x)}{x}
  -
  \Int[\Omega]{v(x)}{x} \\
  & =
  u^\prime \pbraces{\frac{1}{2} +} v\pbraces{\frac{1}{2} -}
  -
  \Int[0][\frac{1}{2}]{u^\primeprime(x) v(x)}{x}
  -
  2 u^\prime \pbraces{\frac{1}{2} -} v\pbraces{\frac{1}{2} +}
  -
  \Int[\frac{1}{2}][1]{2 u^\primeprime(x) v(x)}{x}
  -
  \Int[\Omega]{v(x)}{x}
\end{align*}

\begin{align*}
  \implies
  u^\prime \pbraces{\frac{1}{2} +} v\pbraces{\frac{1}{2} -}
  -
  \Int[0][\frac{1}{2}]{u^\primeprime(x) v(x)}{x}
  -
  2 u^\prime \pbraces{\frac{1}{2} -} v\pbraces{\frac{1}{2} +}
  -
  \Int[\frac{1}{2}][1]{2 u^\primeprime(x) v(x)}{x}
  \stackrel{!}{=}
  \Int[\Omega]{v(x)}{x}
\end{align*}

Laut Satz 5.9 (Einbettungssatz von Sobolev), ist $v$ stetig.

\begin{align*}
  \implies
  v \in C(\Omega)
\end{align*}

\includegraphicsunboxed{PDEs/PDEs_-_Satz_5-9_(Einbettungssatz_von_Sobolev).png}

Laut Satz 5.6 (Spur von Sobolevfunktionen) und Satz 5.7 (Charakterisierung von $H_0^1$-Funktionen), muss $v \in H_0^1(\Omega) \cap C(\Omega)$ am Rand $\partial \Omega = \Bbraces{0, 1}$ verschwinden.

\begin{align*}
  \implies
  v(0) = v(1) = 0
\end{align*}

\includegraphicsunboxed{PDEs/PDEs_-_Satz_5-6_(Spur_von_Sobolevfunktionen).png}
\includegraphicsunboxed{PDEs/PDEs_-_Satz_5-7_(Charakterisierung_von_H_0-1-Funktionen).png}

Das müsste auch für $u$ gelten.
$u$ setzen wir daher an als quadratischen Spline mit Stützstellen $0, \frac{1}{2}, 1$.

\begin{align*}
  u(x)
  & :=
  \begin{cases}
    c_1 x^2 + c_2 x + c_3, & 0 \leq x < \frac{1}{2}, \\
    c_4 x^2 + c_5 x + c_6, & \frac{1}{2} < x \leq 1
  \end{cases} \\
  \implies
  u^\prime(x)
  & =
  \begin{cases}
    2 c_1 x + c_2, & 0 \leq x < \frac{1}{2}, \\
    2 c_4 x + c_5, & \frac{1}{2} < x \leq 1
  \end{cases} \\
  \implies
  u^\primeprime(x)
  & =
  \begin{cases}
    2 c_1, & 0 \leq x < \frac{1}{2}, \\
    2 c_4, & \frac{1}{2} < x \leq 1
  \end{cases}
\end{align*}

Aus unserer oberern Umformulierung können wir nun folgende Bedingungen ziehen.

\begin{align*}
  \implies
  \begin{cases}
    u^\prime \pbraces{\frac{1}{2} -} = 2 u^\prime \pbraces{\frac{1}{2} +}
    & \implies 2 (2 c_1 \pbraces{\frac{1}{2}} + c_2) = 2 c_4 \pbraces{\frac{1}{2}} + c_5 \\
    u^\primeprime(x)
    =
    \begin{cases}
      -1,           & 0 \leq x < \frac{1}{2}, \\
      -\frac{1}{2}, & \frac{1}{2} < x \leq 1
    \end{cases}
    & \implies 2 c_1 = -1, \quad 2 c_4 = -\frac{1}{2} \\
    u\pbraces{\frac{1}{2} -} = u\pbraces{\frac{1}{2} +}
    & \implies c_1 \pbraces{\frac{1}{2}}^2 + c_2 \pbraces{\frac{1}{2}} + c_3 = c_4 \pbraces{\frac{1}{2}}^2 + c_5 \pbraces{\frac{1}{2}} + c_6 \\
    u(0) = 0
    & \implies c_3 = 0 \\
    u(1) = 0
    & \implies c_4 + c_5 + c_6 = 0
    \end{cases}
\end{align*}

Das können wir auch schön als Lineares Gleichungssystem schreiben und lösen.

\begin{align*}
  \begin{pmatrix}
    2 & 2 & 0 & -1 & -1 & 0 \\
    2 & 0 & 0 &  0 &  0 & 0 \\
    0 & 0 & 0 &  2 &  0 & 0 \\
    \frac{1}{4} & \frac{1}{2} & 1 & -\frac{1}{4} & -\frac{1}{2} & -1 \\
    0 & 0 & 1 &  0 &  0 & 0 \\
    0 & 0 & 0 &  1 &  1 & 1
  \end{pmatrix}
  c
  =
  \begin{pmatrix}
    0 \\ -1 \\ -\frac{1}{2} \\ 0 \\ 0 \\ 0
  \end{pmatrix}
  \implies
  c
  =
  \begin{pmatrix}
    -\frac{1}{2} \\ \frac{11}{24} \\ 0 \\ -\frac{1}{4} \\ \frac{1}{6} \\ \frac{1}{12}
  \end{pmatrix}
\end{align*}

Damit erhalten wir unseren Kandidaten für $u$.

\begin{align*}
  u(x)
  =
  \begin{cases}
    -\frac{1}{2} x^2 + \frac{11}{24} x,              & 0 \leq x < \frac{1}{2}, \\
    -\frac{1}{4} x^2 + \frac{1}{6} x + \frac{1}{12}, & \frac{1}{2} < x \leq 1
  \end{cases}
\end{align*}

Von diesem müssen wir noch zeigen, dass $u \in H_0^1(\Omega)$.

\begin{enumerate}[label = \arabic*.]

  \item Schritt (\enquote{$H^1$}):

  \begin{enumerate}[label = (\roman*)]

    \item $u$ ist stetig, also auch $u^2$.
    $\overline{\Omega}$ ist kompakt.
    Daher ist $u^2$ integrierbar, d.h. $u$ quadratisch integrierbar auf $\Omega$.

    \begin{align*}
      u \in C(\Omega) & \implies u^2 \in C(\Omega) \\
      \overline{\Omega} ~\text{kompakt}~ & \implies u^2 \in L^1(\Omega) \implies u \in L^2(\Omega)
    \end{align*}

    \item Wir zeigen zunächst, dass die distributionelle Ableitung $u^\prime$ von $u$ wie folgt aussieht.

    \begin{align*}
      u^\prime(x)
      \stackrel{!}{=}
      \begin{cases}
        -x + \frac{11}{24},           & 0 \leq x < \frac{1}{2}, \\
        -\frac{1}{2} x + \frac{1}{6}, & \frac{1}{2} < x \leq 1
      \end{cases}
    \end{align*}

    Dazu rechnen wir, via partieller Integration, $\Forall \varphi \in \mathcal{D}(\Omega):$

    \begin{align*}
      \abraces{u^\prime, \varphi}
      & :=
      -\abraces{u, \varphi^\prime} \\
      & =
      -\Int[0][\frac{1}{2}]{\pbraces{-\frac{1}{2} x^2 + \frac{11}{24} x} \varphi^\prime(x)}{x} \\
      & \quad
      -\Int[\frac{1}{2}][1]{\pbraces{-\frac{1}{4} x^2 + \frac{1}{6} x + \frac{1}{12}} \varphi^\prime(x)}{x} \\
      & \stackrel
      {
        \mathrm{PI}
      }{=}
      \Int[0][\frac{1}{2}]{\pbraces{-x + \frac{11}{24}} \varphi(x)}{x}
      +
      \frac{1}{2} x^2 \varphi(x) \Big |_{x=0}^{\frac{1}{2}}
      -
      \frac{11}{24} x \varphi(x) \Big |_{x=0}^{\frac{1}{2}} \\
      & \quad
      \Int[\frac{1}{2}][1]{\pbraces{-\frac{1}{2} x + \frac{1}{6}} \varphi(x)}{x}
      +
      \frac{1}{4} x \varphi(x) \Big |_{x=\frac{1}{2}}^1
      -
      \frac{1}{6} x \varphi(x) \Big |_{x=\frac{1}{2}}^1
      -
      \frac{1}{12} \varphi(x) \Big |_{x=\frac{1}{2}}^1 \\
      & =
      \frac{1}{2} \pbraces{\frac{1}{2}}^2 \varphi \pbraces{\frac{1}{2}} - \frac{11}{24} \pbraces{\frac{1}{2}} \varphi \pbraces{\frac{1}{2}}
      -
      \frac{1}{2} \pbraces{\frac{1}{2}}^2 \varphi \pbraces{\frac{1}{2}} + \frac{1}{6} \pbraces{\frac{1}{2}} \varphi \pbraces{\frac{1}{2}} + \frac{1}{12} \varphi \pbraces{\frac{1}{2}} \\
      & \quad
      +
      \Int[0][1]{u^\prime(x) \varphi(x)}{x} \\
      & =
      \underbrace
      {
        \pbraces
        {
          \frac{1}{8} - \frac{11}{48} - \frac{1}{16} + \frac{1}{12} + \frac{1}{12}
        }
      }_0
      \varphi \pbraces{\frac{1}{2}}
      +
      \Int[0][1]{u^\prime(x) \varphi(x)}{x} \\
      & =
      \Int[0][1]{u^\prime(x) \varphi(x)}{x}
    \end{align*}

    $u^\prime$ hat nur eine Unstetigkeitsstelle (in $\frac{1}{2}$).
    Daher können wir analog zu (ii) argumentieren, dass $u^\prime \in L^2(\Omega)$.

    \begin{align*}
      \implies
      u^\prime \in L^2(\Omega)
    \end{align*}

  \end{enumerate}

  \begin{align*}
    \implies
    u, u^\prime \in L^2(\Omega)
    \implies
    u \in H^1(\Omega)
  \end{align*}

  \item Schritt (\enquote{$H_0$}):

  $u$ ist stetig und verschwindet am Rand.
  Laut Satz 5.7 (Charakterisierung von $H_0^1$-Funktionen), sind wir fertig.

  \begin{align*}
    u \in C(\Omega)
    \implies
    T(u) = u |_{\partial \Omega} = 0
    \implies
    u \in H_0^1(\Omega)
  \end{align*}

\end{enumerate}

\end{solution}

% --------------------------------------------------------------------------------
