% --------------------------------------------------------------------------------

\begin{exercise}

Sei $\Omega = (0,1)$. Lösen Sie das folgende Problem: Gesucht ist ein $u \in H_0^1(\Omega)$ mit
\begin{align*}
  \int_\Omega a(x)u^{\prime}(x)v^{\prime}(x) - v(x) dx = 0 \text{ für alle } v \in H_0^1(\Omega),
\end{align*}
wobei $a(x) = \1_{(0,1/2]}(x) + 2\cdot\1_{(1/2,1)}(x)$.
\end{exercise}

% --------------------------------------------------------------------------------

\begin{solution}
Motivation: Für $u \in C^1(\Omega) \cap H^2(\Omega)$ gilt
\begin{align*}
  \int_\Omega a(x)u^{\prime}(x)v^{\prime}(x) &= \int_\Omega v(x) dx \\
  \iff \int_0^{1/2} u^{\prime}(x)v^{\prime}(x)dx + 2\int_{1/2}^1 u^{\prime}(x)v^{\prime}(x) dx &= \int_0^1 v(x) dx \\
  \iff u^{\prime}(1/2)v(1/2) - u^{\prime}(1/2)v(1/2) -
  2\int_{1/2}^1 u^{\primeprime}(x)v(x) dx \int_0^{1/2}u^{\primeprime}(x)v(x) &= \int_0^1 v(x) dx \\
  \iff - 2\int_{1/2}^1 u^{\primeprime}(x)v(x) dx - \int_0^{1/2}u^{\primeprime}(x)v(x) &= \int_0^1 v(x) dx \\
  \iff \int_0^1 v(x)(a(x)u^{\primeprime}(x) + 1) dx = 0
\end{align*}
Ansatz: Suche $u \in C^1(\Omega) \subset H_1(\Omega)$, sodass $u^{\primeprime}(x)a(x) + 1 = 0$ punktweise, sowie $u(0) = u(1) = 0$.
\begin{align*}
  u(x) = -\int\int \frac{1}{a(z)} dz dy  &=
  -\int_0^x\int_0^y \1_{(0,1/2]}(x) + \frac{1}{2}\cdot\1_{(1/2,1)}(x) dz dy  \\
  &= -\int_0^x y - \frac{2y-1}{4}\1_{(1/2,1)}(y) + C dy \\
  &= -\frac{x^2}{2} + \frac{x^2-x}{4}\1_{(1/2,1)}(x) + Cx + D
\end{align*}
Damit $u \in H_0^1(\Omega)$, muss noch $u(0) = u(1) = 0$ erreicht werden
\begin{align*}
  0 &\stackrel{!}{=} u(1) = -\frac{1}{2} - \frac{1-1}{4} + C + D \iff C + D = \frac{1}{2} \\
  0 &\stackrel{!}{=} u(0) = D \iff D = 0 \iff C = \frac{1}{2} \\
\end{align*}
Unser Lösungskandidat $u \in C^1(\Omega)$ ist also
\begin{align*}
  u(x) &= -\frac{1}{4}(2x^2-2x- (x^2-x)\1_{(1/2,1)}(x)) \\
  u^{\prime}(x) &= -\frac{1}{4}(4x -2 - (2x-1)\1_{(1/2,1)}(x))
\end{align*}
Berechne nun die distributionelle Ableitung von $u^{\prime}(x)$:
\begin{align*}
  \langle u^{\primeprime}, \phi \rangle &= \langle u^{\prime}, -\phi^{\prime}\rangle
  = -\int_0^1 u^{\prime}(x)\phi^{\prime}(x) dx = \frac{1}{4}\left(\int_0^{1/2}(4x-2)\phi^{\prime}(x) dx +
  \int_{1/2}^1(4x-2-2x+1)\phi^{\prime}(x) dx\right) \\
  &= \frac{1}{4}\left([(4x-2)\phi(x)]_{x=0}^{1/2} + [(2x-1)\phi(x)]_{x=0}^{1/2} -
  4\int_0^{1/2}\phi(x) - 2\int_{1/2}^1\phi(x)dx\right) \\
  &= -\int_0^1 \frac{1}{a(x)}\phi(x) dx.
\end{align*}
Also gilt auch für die distributionelle Ableitung $a(x)u^{\primeprime}(x) + 1 = 0$
fast überall und somit für alle $v \in H_0^1(\Omega)$
\begin{align*}
  \int_0^1 v(x)(a(x)u^{\primeprime}(x) + 1) dx &= 0 \\
  \iff - \int_0^{1/2}u^{\primeprime}(x)v(x) -2\int_{1/2}^1 u^{\primeprime}(x)v(x) dx
  &= \int_0^1 v(x) dx \\
   \iff \underbrace{u^{\prime}(1/2)v(1/2) - u^{\prime}(1/2)v(1/2)}_{=0} - \int_0^{1/2}u^{\primeprime}(x)v(x)
  -2 \int_{1/2}^1 u^{\primeprime}(x)v(x) dx  &=
  \int_0^1 v(x) dx \\
  \stackrel{PI}{\iff} \int_0^{1/2} u^{\prime}(x)v^{\prime}(x)dx +
  2\int_{1/2}^1 u^{\prime}(x)v^{\prime}(x) dx &= \int_0^1 v(x) dx \\
  \iff \int_\Omega a(x)u^{\prime}(x)v^{\prime}(x) &= \int_\Omega v(x) dx
\end{align*}
Bei der partiellen Integration verwenden wir die stetige Differenzierbarkeit von $u$,
sowie den Einbettungssatz von Sobolev für die Stetigkeit von
$v \in H^1(\Omega) \subset C(\overline{\Omega})$.
\end{solution}

% --------------------------------------------------------------------------------
