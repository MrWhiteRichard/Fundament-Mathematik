% -------------------------------------------------------------------------------- %

\begin{exercise}

Sei $\Omega \subset \R^n$ ein beschränktes Gebiet mit $\partial\Omega \in C^2$.
Zeigen Sie, dass die \textit{biharmonische Gleichung}

\begin{align*}
  \Delta^2 u = f ~\text{in}~ \Omega,
  \quad
  u = \nabla u \cdot \nu = 0 ~\text{auf}~ \partial \Omega
\end{align*}

eine schwache Lösung besitzt, falls $f \in L^2(\Omega)$.
Hierbei ist $\nu$ der äußere Normaleneinheitsvektor auf $\partial\Omega$.
Anleitung:

\begin{enumerate}[label = (\alph*)]

  \item Zeigen Sie zunächst, dass die schwache Formulierung des obigen Randwertproblems

  \begin{align*}
    \Int[\Omega]{\Delta u \Delta v}{x} = \Int[\Omega]{f v}{x}
    ~\text{für alle}~ v \in H_0^2(\Omega)
  \end{align*}

  lautet.

  \item Zeigen Sie, dass $a(u, v) = \Int[\Omega]{\Delta u \Delta v}{x}$ eine stetige und koerzive Bilinearform ist.
  Verwenden Sie hierbei folgende Ungleichung:
  Für alle $u \in H_0^2(\Omega)$ gilt $\norm[H^2(\Omega)]{u} \leq C \norm[L^2(\Omega)]{\Delta u}$.

  \item Zeigen Sie die Existenz einer schwachen Lösung des Randwertproblems.

\end{enumerate}

\end{exercise}

% -------------------------------------------------------------------------------- %

\begin{solution}

\phantom{}

\begin{enumerate}[label = (\alph*)]

  \item Wir zeigen zunächst, dass für $v \in H_0^2(\Omega)$ auch $\nabla v \in H_0^1(\Omega)$.
  Aus der Definition von $H_0^2(\Omega) = \overline{C_0^{\infty}(\Omega)}^{\norm[H^2(\Omega)]{\cdot}}$ folgt, dass wir $H_0^2(\Omega)$-Funktionen als $\norm[H^2(\Omega)]{\cdot}$-Grenzwert von $C_0^\infty(\Omega)$-Funktionen darstellen können.

  \begin{align*}
    v \in H_0^2(\Omega) = \overline{C_0^{\infty}(\Omega)}^{\norm[H^2(\Omega)]{\cdot}}
    \implies
    \Exists (v_n)_{n \in \N} \subset C_0^{\infty}(\Omega):
    v_n \xrightarrow[n \to \infty]{H^2(\Omega)} v
  \end{align*}

  \begin{align*}
    \implies
    \norm[H^2(\Omega)]{v - v_n}
    =
    \pbraces
    {
      \sum_{|\alpha| \leq 2}
      \Int[\Omega]{\abs{D^\alpha(v - v_n)}^2}{x}
    }^{1/2}
    \xrightarrow{n \to \infty}
    0
  \end{align*}

  \begin{multline*}
    \implies
    \norm[H^1(\Omega)]{\nabla v - \nabla v_n}
    =
    \norm[H^1(\Omega)]{\nabla(v - v_n)}
    =
    \pbraces
    {
      \sum_{|\alpha| \leq 1}
      \Int[\Omega]{\abs{D^\alpha \nabla(v - v_n)}^2}{x}
    }^{1/2} \\
    \leq
    \pbraces
    {
      \sum_{|\alpha| \leq 2}
      \Int[\Omega]{\abs{D^\alpha(v - v_n)}^2}{x}
    }^{1/2}
    =
    \norm[H^2(\Omega)]{v - v_n}
    \xrightarrow{n \to \infty}
    0
  \end{multline*}

  \begin{align*}
    \implies
    \nabla v \in \overline{C_0^{\infty}(\Omega)}^{\norm[H^1(\Omega)]{\cdot}} = H_0^1(\Omega)
  \end{align*}

  Laut Satz 5.7 (Charakterisierung von $H_0^1$-Funktionen) verschwindet der Trace von $v$.
  Das gilt nun aber auch für $\nabla v$.

  \begin{align*}
    \implies
    T(\nabla v) = 0,
    \quad
    T(v) = 0
  \end{align*}

  \includegraphicsunboxed{PDEs/PDEs_-_Satz_5-7_(Charakterisierung_von_H_0-1-Funktionen).png}

  Wir multiplizieren die Differentialgleichung mit $v \in H_0^2(\Omega)$, integrieren über $\Omega$ und integrieren zweimal partiell:

  \begin{align*}
    \Delta^2 u = f
    \implies
    (\Delta^2 u) v = f v
  \end{align*}

  \begin{align*}
    \implies
    F(v)
    & :=
    \Int[\Omega]{f v}{x}
    =
    \Int[\Omega]{(\Delta^2 u) v}{x}
    =
    \Int[\Omega]{(\Div \nabla \Delta u) v}{x} \\
    & \stackrel
    {
      \mathrm{Gauß}
    }{=}
    -\Int[\Omega]{\nabla \Delta u \cdot \nabla v}{x}
    +
    \Int[\partial \Omega]{(\nabla (\Delta u) \cdot \nu) \underbrace{T(v)}_0}{x} \\
    & \stackrel
    {
      \mathrm{Gauß}
    }{=}
    \Int[\Omega]{\Delta u \Div \nabla v}{x}
    -
    \Int[\partial \Omega]{(\underbrace{T(\nabla v)}_0 \cdot \nu) \Delta u}{s}
    =
    \Int[\Omega]{\Delta u \Delta v}{x}
    =:
    a(u, v)
  \end{align*}

  \item

  \begin{itemize}
    \item Stetigkeit:
    
    $|\cdot|$ bezeichnet hier (unter Anderem) die Frobenius-Norm.
    $\Hess$ liefert die Hesse-Matrix.

    \begin{align*}
      \abs{\Delta v}^2
      =
      \abs
      {
        \sum_{i=1}^n
        \pderivative[2]{x_i} u
      }^2
      \stackrel
      {
        \mathrm{CSB}
      }{\leq}
      \pbraces
      {
        \sum_{i=1}^n
        1^2
      }
      \pbraces
      {
        \sum_{i=1}^n
        \abs{\pderivative[2]{x_i} u}^2
      }
      \leq
      n
      \sum_{i,j=1}^n
      \abs
      {
        \frac{\partial^2}{\partial x_i \partial x_j} u
      }^2
      =
      n \abs{\Hess u}^2
    \end{align*}

    \begin{align*}
      \implies
      \abs{a(u, v)}
      & =
      \abs{\Int[\Omega]{\Delta u \Delta v}{x}}
      \leq
      \Int[\Omega]{\abs{\Delta u \Delta v}}{x}
      \stackrel
      {
        \mathrm{CSB}
      }{\leq}
      \pbraces
      {
        \Int[\Omega]{\abs{\Delta u}^2}{x}
      }^{1/2}
      \pbraces
      {
        \Int[\Omega]{\abs{\Delta v}^2}{x}
      }^{1/2} \\
      & \leq
      \pbraces
      {
        \Int[\Omega]{n \abs{\Hess u}^2}{x}
      }^{1/2}
      \pbraces
      {
        \Int[\Omega]{n \abs{\Hess v}^2}{x}
      }^{1/2} \\
      & =
      n
      \pbraces
      {
        \Int[\Omega]{\abs{\Hess u}^2}{x}
      }^{1/2}
      \pbraces
      {
        \Int[\Omega]{\abs{\Hess v}^2}{x}
      }^{1/2} \\
      & =
      n \norm[L^2(\Omega)]{\Hess u} \norm[L^2(\Omega)]{\Hess v}
      \leq
      n \norm[H^2(\Omega)]{u} \norm[H^2(\Omega)]{v}
    \end{align*}

    \item Koerzivität:

    \begin{align*}
      a(u, u)
      =
      \Int[\Omega]{\Delta u \Delta u}{x}
      =
      \Int[\Omega]{\abs{\Delta u}^2}{x}
      =
      \norm[L^2(\Omega)]{\Delta u}^2
      \geq
      C^{-2} \norm[H^2(\Omega)]{u}^2
    \end{align*}

  \end{itemize}

  \item Laut Konstruktion, ist $H_0^2$ ist mit $(\cdot, \cdot)_{H^2(\Omega)}$ ein Hilberbraum.
  In (b) haben wir gezeigt, dass die Bilinearform $a$ stetig und koerziv ist.
  $F \in H^{-2}(\Omega)$ ist ein lineares und stetiges Funktional.

  \begin{align*}
    \abs{F(v)}
    =
    \abs{\Int[\Omega]{f v}{x}}
    \leq
    \Int[\Omega]{\abs{f v}}{x}
    =
    (\norm[L^1(\Omega)]{f v})
    \stackrel
    {
      \mathrm{CSB}
    }{\leq}
    \norm[L^2(\Omega)]{f} \norm[L^2(\Omega)]{v}
    \leq
    \norm[L^2(\Omega)]{f} \norm[H^2(\Omega)]{v}
  \end{align*}

  Laut dem Lemma von Lax-Milgram existiert nun genau ein $u \in H_0^2(\Omega)$, sodass $a(u, v) = F(v)$ für alle $v \in H_0^2(\Omega)$.

  \begin{align*}
      \implies
      \ExistsOnlyOne u \in H_0^2(\Omega):
      \Forall v \in H_0^2(\Omega):
      a(u, v) = F(v)
  \end{align*}

\end{enumerate}

\end{solution}

% -------------------------------------------------------------------------------- %
