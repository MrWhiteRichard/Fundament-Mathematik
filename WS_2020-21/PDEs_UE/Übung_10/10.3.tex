% --------------------------------------------------------------------------------

\begin{exercise}
\textit{(Smoothing properties of the heat operator)} \\
Let $\Omega=(0,2 \pi)$ and consider the initial value problem (IVP)
\begin{align*}
  \left\{\begin{array}{ll}
  u_{t}=u_{x x} & \text { for } (x, t) \in \Omega \times(0, \infty), \\
  u(x, t=0)=u_{0} & \text { for } x \in \Omega, \\
  u(x=0, t)=u(x=2 \pi, t) & \text { for } t>0,
  \end{array}\right.
\end{align*}
with $u_{0} \in L^{2}(\Omega) .$ The set $\left\{\phi_{n}(x)=e^{\mathrm{in} x}: n \in \mathbb{Z}\right\}$ is a complete orthogonal system in the complex Hilbert space $L^{2}(\Omega),$ corresponding to the eigenfunctions of the operator $L u=-u_{x x}$ with periodic boundary conditions. We now consider the operator
\begin{align*}
  e^{-L t}: L^{2}(\Omega) \mapsto L^{2}(\Omega), \quad
  v \mapsto \sum_{n \in \mathbb{Z}} e^{-n^{2} t}\left\langle v,
  \frac{\phi_{n}}{\left\|\phi_{n}\right\|}\right\rangle_{L^{2}}
  \frac{\phi_{n}}{\left\|\phi_{n}\right\|}.
\end{align*}
In what sense is $u(\cdot, t)=e^{-L t} u_{0}$ a solution of IVP? Furthermore, show that:
\begin{enumerate}[label = (\roman*)]
  \item For $t \geq 0$ it holds that $\left\|e^{-L t}\right\|_{L^{2} \rightarrow L^{2}}=1$.
  \item For $t>0$ it holds that
  $\left\|e^{-L t}\right\|_{L^{2} \rightarrow H_{p e r}^{1}} \leq C\left(1+t^{-1 / 2}\right)$
  for some constant $C>0$.
  \item For $t>0$ it holds that $\left\|e^{-L t}\right\|_{L^{2} \rightarrow H_{\text {per}}^{k}} \leq C_{k}\left(1+t^{-a_{k}}\right)$ for all $k \in \mathbb{N}$ and $a_{k}, C_{k}>0$.
\end{enumerate}
Notice that here $\left\|e^{-L t}\right\|_{X \rightarrow Y}$ denotes the operator-norm of $e^{-L t}$ as an operator from $X$ to $Y$.
\end{exercise}

% --------------------------------------------------------------------------------

\begin{solution}
Betrachte die Funktion $v(x,t) = u(x,t) - u(0,t)$. Dann gilt
\begin{align*}
  \left\{\begin{array}{ll}
  v_{t}(x,t) - v_{xx}(x,t) = u_t(x,t) - u_{xx}(x,t) - u_t(0,t) = - u_t(0,t) & \text { for } (x, t) \in \Omega \times(0, \infty), \\
  v(x, 0) = u(x,0) - u(0,0) =: v_{0} & \text { for } x \in \Omega, \\
  v(0, t) = v(2 \pi, t) = u(0,t) - u(0,t) = 0 & \text { for } t>0,
  \end{array}\right.
\end{align*}
$L(u) = -u_{xx}$
\begin{align*}
  u(x,0) &= \sum_{n \in \Z}\exp(-0n^2)
  \left\langle u_0, \frac{\phi_n}{\|\phi_n\|}\right\rangle \frac{\phi_n}{\|\phi_n\|}
  = \sum_{n \in \Z}\left\langle u_0, \frac{\phi_n}{\|\phi_n\|}\right\rangle
  \frac{\phi_n}{\|\phi_n\|} = u_0(x). \\
  u(0,t) &=
\end{align*}
\begin{enumerate}[label = (\roman*)]
  \item Fall $t = 0$: Sei $u \in L^2(\Omega)$
  \begin{align*}
    \|\exp(-L0)u\|_{L^2(\Omega)}^2 &=
    \left\|\sum_{n\in\Z}\exp(-0n^2)
    \left\langle u, \frac{\phi_n}{\|\phi_n\|}\right\rangle \frac{\phi_n}{\|\phi_n\|}
    \right\|_{L^2(\Omega)}^2
    = \sum_{n\in\Z}
    \left|\left\langle u, \frac{\phi_n}{\|\phi_n\|}\right\rangle\right|^2
    \frac{\|\phi_n\|^2}{\|\phi_n\|^2}\\
    &= \|u\|_{L^2(\Omega)}^2
  \end{align*}
  Fall $t > 0$: Sei $u \in L^2(\Omega)$:
  \begin{align*}
    \|\exp(-Lt)u\|_{L^2(\Omega)}^2 &=
    \left\|\sum_{n\in\Z}\exp(-tn^2)
    \left\langle u, \frac{\phi_n}{\|\phi_n\|}\right\rangle \frac{\phi_n}{\|\phi_n\|}
    \right\|_{L^2(\Omega)}^2
    \leq \sum_{n\in\Z}\left\|
    \left\langle u, \frac{\phi_n}{\|\phi_n\|}\right\rangle \frac{\phi_n}{\|\phi_n\|}
    \right\|_{L^2(\Omega)}^2 = \|u\|_{L^2(\Omega)}.
  \end{align*}
  Damit erhalten wir $\left\|e^{-L t}\right\|_{L^{2}} \leq 1$. Für die andere
  Richtung erhalten wir für $u(x) = \phi_0(x) \equiv 1$
  \begin{align*}
    \|\exp(-Lt)u\|_{L^2(\Omega)}^2 &=
    \left\|\sum_{n\in\Z}\exp(-tn^2)
    \left\langle \phi_0, \frac{\phi_n}{\|\phi_n\|}\right\rangle \frac{\phi_n}{\|\phi_n\|}
    \right\|_{L^2(\Omega)}^2
    = \left\|\left\langle \phi_0,\frac{\phi_0}{\|\phi_0\|}\right\rangle \frac{\phi_0}{\|\phi_0\|}
    \right\|_{L^2(\Omega)}^2 \\
    &= \|u\|_{L^2(\Omega)}.
  \end{align*}
  \item
  Wir verwenden $\exp(-x) \leq \frac{1}{\sqrt{x}}$:
  \begin{align*}
    \|\exp(-Lt)u\|_{H_{per}^1}^2 &= \|\exp(-Lt)u\|_{L^2(\Omega)}^2 + \|(\exp(-Lt)u)_x\|_{L^2(\Omega)}^2
    \\
    &= \|\exp(-Lt)u\|_{L^2(\Omega)}^2 + \left\|\sum_{n\in\Z}\exp(-tn^2)
    \left\langle u, \frac{\phi_n}{\|\phi_n\|}\right\rangle \frac{in\phi_n}{\|\phi_n\|}
    \right\|_{L^2(\Omega)}^2 \\
    &= \|\exp(-Lt)u\|_{L^2(\Omega)}^2 + \sum_{n\in\Z}|n|\exp(-tn^2)\left|
    \left\langle u, \frac{\phi_n}{\|\phi_n\|}\right\rangle\right|^2 \\
    &\leq \|\exp(-Lt)u\|_{L^2(\Omega)}^2 + \sum_{n\in\Z}|n|\frac{1}{\sqrt{t}|n|}\left|
    \left\langle u, \frac{\phi_n}{\|\phi_n\|}\right\rangle\right|^2 \\
    &\leq \left(1 + \frac{1}{\sqrt{t}}\right)\|u\|_{L^2(\Omega)}
  \end{align*}
  \item   Wir verwenden $\exp(-x) \leq x^{-k/2}$:
  \begin{align*}
    \|(\exp(-Lt)u)^{(k)}\|_{L^2(\Omega)}^2 &=  \left\|\sum_{n\in\Z}\exp(-tn^2)
    \left\langle u, \frac{\phi_n}{\|\phi_n\|}\right\rangle \frac{(in)^k\phi_n}{\|\phi_n\|}
    \right\|_{L^2(\Omega)}^2 \\
    &= \sum_{n\in\Z}|n^k|\exp(-tn^2)\left|
    \left\langle u, \frac{\phi_n}{\|\phi_n\|}\right\rangle\right|^2 \\
    &\leq \sum_{n\in\Z}|t^{-k/2}|\left|
    \left\langle u, \frac{\phi_n}{\|\phi_n\|}\right\rangle\right|^2 \\
    &= t^{-k/2}\|u\|_{L^2(\Omega)}
  \end{align*}
  Damit erhalten wir für $t < 1$
  \begin{align*}
    \|\exp(-Lt)u\|_{H_{per}^k}^2 &= \sum_{i=0}^k \|(\exp(-Lt)u)^{(i)}\|_{L^2(\Omega)}^2
    \leq \sum_{i=0}^k t^{-i/2} \|u\|_{L^2(\Omega)} \\
    &= \sum_{i=0}^k \left(\frac{1}{\sqrt{t}}\right)^i \|u\|_{L^2(\Omega)}
    = \frac{t^{-(k+1)/2} - 1}{t^{-1/2} - 1}\|u\|_{L^2(\Omega)} \\
    &= \frac{(1- t^{(k+1)/2})t^{1/2}}{(1-t^{1/2})t^{(k+1)/2}}\|u\|_{L^2(\Omega)}
    = \frac{(1- t^{(k+1)/2})}{(1-t^{1/2})t^{k/2}}\|u\|_{L^2(\Omega)} \\
    &\leq C_kt^{-k/2}\|u\|_{L^2(\Omega)}
  \end{align*}
  Dabei gilt die letzte Abschätzung aufgrund
  \begin{align*}
    \lim_{t \to 1 } \frac{(1- t^{(k+1)/2})}{(1-t^{1/2})} =
    \lim_{t \to 1 }\frac{-(k+1)/2t^{(k-1)/2}}{-\frac{1}{2\sqrt{t}}} = \lim_{t\to 1}(k+1)t^{k/2} =
    k + 1,
  \end{align*}
  also ist die Funktion $f: t \mapsto \frac{(1- t^{(k+1)/2})}{(1-t^{1/2})}$
  auf dem Kompaktum $[0,1]$ durch eine Konstante $C_k$ beschränkt. \\
  Für $t \geq 1$ gilt
  \begin{align*}
    \|\exp(-Lt)u\|_{H_{per}^k}^2 &= \sum_{i=0}^k \left(\frac{1}{\sqrt{t}}\right)^i \|u\|_{L^2(\Omega)}
    \leq \sum_{i=0}^k \|u\|_{L^2(\Omega)} = (k + 1)\|u\|_{L^2(\Omega)}.
  \end{align*}
  Insgesamt erhalten wir also
  \begin{align*}
    \|\exp(-Lt)u\|_{H_{per}^k}^2 \leq \max(C_k, k + 1)(1 + t^{-k/2})\|u\|_{L^2(\Omega)}.
  \end{align*}
\end{enumerate}

\end{solution}

% --------------------------------------------------------------------------------
