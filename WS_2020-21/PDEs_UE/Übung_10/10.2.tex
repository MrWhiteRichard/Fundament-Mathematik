% --------------------------------------------------------------------------------

\begin{exercise}
\textit{(Periodic Sobolev Spaces)} \\
Let $\Omega=(0,2 \pi)$ and consider the
complete orthonormal system of $L^{2}(\Omega)$ given by
\begin{align*}
  \left\{C_{0}=\frac{1}{\sqrt{2 \pi}}, C_{n}(x)=\frac{1}{\sqrt{\pi}} \cos (n x), S_{n}(x)=\frac{1}{\sqrt{\pi}} \sin (n x) \,\Bigg|\, n \in \mathbb{N}\right\}.
\end{align*}
\begin{enumerate}[label = (\roman*)]
  \item Show that for $k \in \mathbb{N}$ the space
  \begin{align*}
      H_{per}^{k}(\Omega):=\left\{f \in H^{k}(\Omega) \mid f^{(j)}(0)=f^{(j)}(2 \pi) \text { for } j=0, \ldots, k-1\right\}
  \end{align*}
  is a well-defined Hilbert space.
  \item Show that $f \in H_{per}^{1}(\Omega)$ if and only if
  \begin{align*}
      f=\sum_{m=1}^{\infty} a_{m} S_{m}+\sum_{m=0}^{\infty} b_{m} C_{m} \quad \text { with } \quad
      \sum_{m=1}^{\infty} m^{2}\left(\left|a_{m}\right|^{2}+\left|b_{m}\right|^{2}\right)<\infty.
  \end{align*}
  In this case, $f$ can be differentiated \glqq term-wise\grqq.
  \item For $n \in \mathbb{N}$ consider the projection

  \begin{align*}
    P_{n}&: H_{p e r}^{k}(\Omega) \rightarrow H_{p e r}^{k}(\Omega) \\
    f&=\sum_{m=1}^{\infty} a_{m} S_{m}+\sum_{m=0}^{\infty} b_{m} C_{m}
    \mapsto P_{n} f=\sum_{m=1}^{n} a_{m} S_{m}+\sum_{m=0}^{n} b_{m} C_{m}.
  \end{align*}

  Show that for $f \in H_{per}^{k}(\Omega)$ it holds that
  \begin{align*}
  \left\|f-P_{n} f\right\|_{L^{2}(\Omega)} \leq \frac{1}{(n+1)^{k}}\left\|f^{(k)}\right\|_{L^{2}(\Omega)}.
  \end{align*}

\end{enumerate}
\end{exercise}

% --------------------------------------------------------------------------------

\begin{solution}

\phantom{}

\begin{enumerate}[label = (\roman*)]

  \item Offensichtlich ist $H_\mathrm{per}^k(\Omega)$ ein linearer Unterraum von $H^k(\Omega)$.
  Für die Abgeschlssenheit bzgl. $\norm[H^k(\Omega)]{\cdot}$ verwenden wir ...
  
  \includegraphicsunboxed{PDEs/PDEs_-_Satz_5-9_(Einbettungssatz_von_Sobolev).png}

  \begin{align*}
    \implies
    H^k(\Omega) \hookrightarrow C^0(\Omega)
  \end{align*}

  Sei $(f_n)_{n \in \N}$ eine Folge aus $H_\mathrm{per}^k(\Omega)$, die bzgl. $\norm[H^k(\Omega)]{\cdot}$ gegen ein $f$ konvergiert.

  \begin{align*}
    (f_n)_{n \in \N} \in H_\mathrm{per}^k(\Omega)^\N:
    \quad
    f_n \xrightarrow[n \to \infty]{H^k(\Omega)} f
  \end{align*}

  \begin{multline*}
    \implies
    \abs{f(2 \pi) - f(0)}
    \leq
    \abs{f(2 \pi) - f_n(2 \pi)} + \underbrace{\abs{f_n(2 \pi) - f_n(0)}}_0 + \abs{f_n(0) - f(0)} \\
    \leq
    2 \norm[L^\infty(\Omega)]{f - f_n}
    \leq
    2 C_\mathrm{Sobolev} \underbrace{\norm[H^k(\Omega)]{f - f_n}}_{\xrightarrow[n \to \infty]{} 0}
    \xrightarrow[]{n \to \infty}
    0
  \end{multline*}

  \item 

  \begin{enumerate}

    \item Inklusion (\Quote{$\subseteq$}):
    
    Sei $f \in H_\mathrm{per}^1(\Omega) \subseteq L^2(\Omega)$, dann können wir $f$ bzgl. der oberen ONB darstellen.

    \begin{align*}
      \implies
      f
      =
      \sum_{m=1}^\infty (f, S_m) S_m + \sum_{m=0}^\infty (f, C_m) C_m
    \end{align*}

    Die folgende Darstellung von $f^\prime$, so wie in der Angabe postuliert, wird weiter unten dann noch (rigoros) nachgerechnet.
    Die Fourier-Koeffizienten sind eindeutig.

    \begin{align*}
      \implies
      \sum_{m=0}^\infty (f^\prime, C_m) C_m + \sum_{m=1}^\infty (f^\prime, S_m) S_m
      =
      f^\prime
      =
      \sum_{m=1}^\infty (f, S_m) m C_m + \sum_{m=1}^\infty (f, C_m) (-m) S_m
    \end{align*}

    Mit der Parceval-Gleichung erhalten wir schließlich noch die $\ell^2$-Bedingung.

    \begin{align*}
      \sum_{m=1}^\infty m^2 \pbraces{\abs{(f, C_m)}^2 + \abs{(f, S_m)}^2}
      \stackrel
      {
        \mathrm{P}
      }{=}
      \norm[L^2(\Omega)]{f^\prime}^2
      <
      \infty
    \end{align*}

    \item Inklusion (\Quote{$\supseteq$}):
    
    \begin{align*}
      f
      :=
      \sum_{m=1}^\infty a_m S_m + \sum_{m=0}^\infty b_m C_m
    \end{align*}

    \begin{itemize}

      \item \Quote{$_\mathrm{per}$}:
      
      \dots gilt Summanden-weise, also auch insgesamt.

      \item \Quote{$^1$}:
      
      \begin{align*}
        \norm[L^2(\Omega)]{f}
        & =
        \norm[L^2(\Omega)]
        {
          \sum_{m=1}^\infty a_m S_m + \sum_{m=0}^\infty b_m C_m
        } \\
        & \leq
        \sum_{m=1}^\infty \frac{m}{m} \abs{a_m} \underbrace{\norm[L^2(\Omega)]{S_m}}_1
        +
        \sum_{m=1}^\infty \frac{m}{m} \abs{b_m} \underbrace{\norm[L^2(\Omega)]{C_m}}_1
        +
        \abs{b_0} \underbrace{\norm[L^2(\Omega)]{C_0}}_1 \\
        & \stackrel
        {
          \mathrm{CSB}
        }{\leq}
        \underbrace
        {
          \sqrt
          {
            \sum_{m=1}^\infty \frac{1}{m^2}
          }
        }_{
          \pi / \sqrt{6}
        }
        \underbrace
        {
          \sqrt
          {
            \sum_{m=1}^\infty m^2 \abs{a_m}^2
          }
        }_{
          < \infty
        }
        +
        \underbrace
        {
          \sqrt
          {
            \sum_{m=1}^\infty \frac{1}{m^2}
          }
        }_{
          \pi / \sqrt{6}
        }
        \underbrace
        {
          \sqrt
          {
            \sum_{m=1}^\infty m^2 \abs{b_m}^2
          }
        }_{
          < \infty
        }
        +
        \abs{b_0}        <
        \infty
      \end{align*}

      Wir werden im Folgenden den Satz 6.8 verwenden.

      \includegraphicsboxed{PDEs/PDEs_-_Lemma_6-8.png}

      \begin{align*}
        \sum_{m=1}^\infty m^2 \pbraces{\abs{a_m}^2 + \abs{b_m}^2} < \infty
        & \stackrel
        {
          \mathrm{6.8}
        }{\iff}
        \sum_{m=1}^N a_m S_m^\prime + \sum_{m=0}^N b_m C_m^\prime \\
        & =
        \sum_{m=1}^N a_m m C_m + \sum_{m=0}^N b_m (-m) S_m \\
        & \xrightarrow[n \to \infty]{\norm[L^2(\Omega)]{\cdot}}
        \sum_{m=1}^\infty a_m m C_m + \sum_{m=0}^\infty b_m (-m) S_m
        =
        f
      \end{align*}

      Diese Tatsache können wir verwenden, um eine $\lim$-$\int$-Vertauschung zu rechtfertigen.
      Nachdem der $L^2(\Omega)$ abgeschlossen, müssen wir nur noch (heiß angekündigt) zeigen ... $\Forall \varphi \in \mathcal{D}(\Omega):$

      \begin{align*}
        \abraces{f^\prime, \varphi}
        & =
        - \abraces{f, \varphi^\prime}
        =
        - \Int[0][2 \pi]
        {
          \pbraces
          {
            \sum_{m=1}^\infty a_m S_m + \sum_{m=0}^\infty b_m C_m
          }
          \varphi^\prime
        }{x} \\
        & =
        \sum_{m=1}^\infty - \Int[0][2 \pi]{a_m S_m \varphi^\prime}{x}
        +
        \sum_{m=0}^\infty \Int[0][2 \pi]{b_m C_m \varphi^\prime}{x} \\
        & \stackrel
        {
          \mathrm{PI}
        }{=}
        \sum_{m=1}^\infty \Int[0][2 \pi]{a_m m C_m \varphi}{x}
        +
        \sum_{m=0}^\infty \Int[0][2 \pi]{b_m (-m S_m) \varphi}{x} \\
        & =
        \Int[0][2 \pi]
        {
          \Big (
            \underbrace
            {
              \sum_{m=1}^\infty a_m m C_m
              +
              \sum_{m=0}^\infty b_m (-m S_m)
            }_{
              \in L^2(\Omega) \subseteq L^1_\mathrm{lok}
            }
          \Big )
          \varphi
        }{x} \\
        & =
        \abraces
        {
          \sum_{m=1}^\infty a_m m C_m
          +
          \sum_{m=0}^\infty b_m (-m S_m),
          \varphi
        }
      \end{align*}

      Dabei ist $f \in L^2(\Omega) \in L^1_\mathrm{lok}$, eine reguläre Distribution.
      Die erste $\lim$-$\int$-Vertauschung folgt via unbedingter Konvergenz.
      Bei der Partiellen Integration fallen die Randterme weg, weil $\varphi$ als Testfunktion am Rand verschwindet.

    \end{itemize}

  \end{enumerate}

  \item ToDo

\end{enumerate}

\end{solution}

% --------------------------------------------------------------------------------

\begin{solution}

\phantom{}

\begin{enumerate}[label = (\roman*)]
  \item Wir zeigen, dass die Funktionale
  \begin{align*}
    K_j: H^k(\Omega) \to \R;\quad f \mapsto f^{(j)}(0) - f^{(j)}(2\pi), \quad j = 0, \dots, k - 1
  \end{align*}
  stetig sind und
  \begin{align*}
    H_{per}^k(\Omega) = \bigcap_{j=0}^{k-1}\ker(K_j)
  \end{align*}
  somit ein abgeschlossener Unterraum von $H^k(\Omega)$ und daher wieder ein
  Hilbertraum ist. \\
  Zur Wohldefiniertheit von $K_j:$ Da $k - 1/2 > k -1$ wird mit dem Einbettungssatz
  von Sobolev $H^k(\Omega) \hookrightarrow C^{k-1}(\Omega)$ stetig eingebettet und
  die punktweisen Auswertungen sind somit sinnvoll definiert.
  \begin{align*}
    |K_j(f)| = |f^{(j)}(2\pi) - f^{(j)}(0)| = \left|\int_0^{2\pi}f^{j+1}(x) dx \right|
    \stackrel{CSU}{\leq} \sqrt{2\pi}\|f^{j+1}\|_{L^2(0,2\pi)} \leq \sqrt{2\pi}\|f\|_{H^k(0,2\pi)},
    \quad j = 0,\dots,k-1.
  \end{align*}
  \item Sei $f \in H_{per}^1(\Omega) \subset L^2(\Omega)$, dann gilt
  \begin{align*}
    f=\sum_{m=1}^{\infty} (f,S_m) S_{m}+\sum_{m=0}^{\infty} (f,C_m) C_{m}
  \end{align*}
  \begin{align*}
    a_m := (f,S_m)_{L^2} &= \frac{1}{\sqrt{\pi}}\int_0^{2\pi} f(x)\sin(mx) dx
    = \frac{1}{\sqrt{\pi}}\left(\underbrace{\left[-\frac{1}{m}f(x)\cos(mx)\right]_0^{2\pi}}_{=0}
    + \frac{1}{m}\int_0^{2\pi}f^{\prime}(x)\cos(mx) dx \right) \\
    &= \frac{1}{\sqrt{\pi}}\left(\frac{1}{m}\int_0^{2\pi}f^{\prime}(x)\cos(mx) dx \right)
    \leq \frac{1}{m\sqrt{\pi}}\|f^{\prime}\|_{L^2(0,2\pi)}\|\cos(mx)\|_{L^2(0,2\pi)} \\
    &= \frac{1}{m\sqrt{\pi}}\|f^{\prime}\|_{L^2(0,2\pi)}\sqrt{\frac{1}{m}\int_0^{2m\pi}\cos^2(u) du}
    = \frac{1}{m\sqrt{\pi}}\|f^{\prime}\|_{L^2(0,2\pi)}\sqrt{\frac{1}{2m}(2m\pi + \sin(2m\pi)\cos(2m\pi))} \\
    &= \frac{1}{m\sqrt{\pi}}\|f^{\prime}\|_{L^2(0,2\pi)}\sqrt{\pi} =
    \frac{1}{m}\|f^{\prime}\|_{L^2(0,2\pi)}
  \end{align*}
  \begin{align*}
    b_m := (f,C_m)_{L^2} &= \frac{1}{\sqrt{\pi}}\int_0^{2\pi} f(x)\cos(mx) dx
    = \frac{1}{\sqrt{\pi}}\left(\underbrace{\left[\frac{1}{m}f(x)\sin(mx)\right]_0^{2\pi}}_{=0}
    - \frac{1}{m}\int_0^{2\pi}f^{\prime}(x)\sin(mx) dx \right) \\
    &\leq \frac{1}{m}\|f^{\prime}\|_{L^2(0,\pi)}
  \end{align*}
\end{enumerate}

\end{solution}

% --------------------------------------------------------------------------------
