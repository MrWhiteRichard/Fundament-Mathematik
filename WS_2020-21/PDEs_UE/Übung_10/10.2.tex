% --------------------------------------------------------------------------------

\begin{exercise}
\textit{(Periodic Sobolev Spaces)} \\
Let $\Omega=(0,2 \pi)$ and consider the
complete orthonormal system of $L^{2}(\Omega)$ given by
\begin{align*}
  \left\{C_{0}=\frac{1}{\sqrt{2 \pi}}, C_{n}(x)=\frac{1}{\sqrt{\pi}} \cos (n x), S_{n}(x)=\frac{1}{\sqrt{\pi}} \sin (n x) \,\Bigg|\, n \in \mathbb{N}\right\}.
\end{align*}
\begin{enumerate}[label = (\roman*)]
  \item Show that for $k \in \mathbb{N}$ the space
  \begin{align*}
      H_{per}^{k}(\Omega):=\left\{f \in H^{k}(\Omega) \mid f^{(j)}(0)=f^{(j)}(2 \pi) \text { for } j=0, \ldots, k-1\right\}
  \end{align*}
  is a well-defined Hilbert space.
  \item Show that $f \in H_{per}^{1}(\Omega)$ if and only if
  \begin{align*}
      f=\sum_{m=1}^{\infty} a_{m} S_{m}+\sum_{m=0}^{\infty} b_{m} C_{m} \quad \text { with } \quad
      \sum_{m=1}^{\infty} m^{2}\left(\left|a_{m}\right|^{2}+\left|b_{m}\right|^{2}\right)<\infty.
  \end{align*}
  In this case, $f$ can be differentiated \glqq term-wise\grqq.
  \item For $n \in \mathbb{N}$ consider the projection

  \begin{align*}
    P_{n}&: H_{p e r}^{k}(\Omega) \rightarrow H_{p e r}^{k}(\Omega) \\
    f&=\sum_{m=1}^{\infty} a_{m} S_{m}+\sum_{m=0}^{\infty} b_{m} C_{m}
    \mapsto P_{n} f=\sum_{m=1}^{n} a_{m} S_{m}+\sum_{m=0}^{n} b_{m} C_{m}.
  \end{align*}

  Show that for $f \in H_{per}^{k}(\Omega)$ it holds that
  \begin{align*}
  \left\|f-P_{n} f\right\|_{L^{2}(\Omega)} \leq \frac{1}{(n+1)^{k}}\left\|f^{(k)}\right\|_{L^{2}(\Omega)}.
  \end{align*}

\end{enumerate}
\end{exercise}

% --------------------------------------------------------------------------------

\begin{solution}

\phantom{}

\begin{enumerate}[label = (\roman*)]
  \item Wir zeigen, dass die Funktionale
  \begin{align*}
    K_j: H^k(\Omega) \to \R;\quad f \mapsto f^{(j)}(0) - f^{(j)}(2\pi), \quad j = 0, \dots, k - 1
  \end{align*}
  stetig sind und
  \begin{align*}
    H_{per}^k(\Omega) = \bigcap_{j=0}^{k-1}\ker(K_j)
  \end{align*}
  somit ein abgeschlossener Unterraum von $H^k(\Omega)$ und daher wieder ein
  Hilbertraum ist. \\
  Zur Wohldefiniertheit von $K_j:$ Da $k - 1/2 > k -1$ wird mit dem Einbettungssatz
  von Sobolev $H^k(\Omega) \hookrightarrow C^{k-1}(\Omega)$ stetig eingebettet und
  die punktweisen Auswertungen sind somit sinnvoll definiert.
  \begin{align*}
    |K_j(f)| = |f^{(j)}(2\pi) - f^{(j)}(0)| = \left|\int_0^{2\pi}f^{j+1}(x) dx \right|
    \stackrel{CSU}{\leq} \sqrt{2\pi}\|f^{j+1}\|_{L^2(0,2\pi)} \leq \sqrt{2\pi}\|f\|_{H^k(0,2\pi)},
    \quad j = 0,\dots,k-1.
  \end{align*}
  \item Sei $f \in H_{per}^1(\Omega) \subset L^2(\Omega)$, dann gilt
  \begin{align*}
    f=\sum_{m=1}^{\infty} (f,S_m) S_{m}+\sum_{m=0}^{\infty} (f,C_m) C_{m}
  \end{align*}
  \begin{align*}
    a_m := (f,S_m)_{L^2} &= \frac{1}{\sqrt{\pi}}\int_0^{2\pi} f(x)\sin(mx) dx
    = \frac{1}{\sqrt{\pi}}\left(\underbrace{\left[-\frac{1}{m}f(x)\cos(mx)\right]_0^{2\pi}}_{=0}
    + \frac{1}{m}\int_0^{2\pi}f^{\prime}(x)\cos(mx) dx \right) \\
    &= \frac{1}{\sqrt{\pi}}\left(\frac{1}{m}\int_0^{2\pi}f^{\prime}(x)\cos(mx) dx \right)
    \leq \frac{1}{m\sqrt{\pi}}\|f^{\prime}\|_{L^2(0,2\pi)}\|\cos(mx)\|_{L^2(0,2\pi)} \\
    &= \frac{1}{m\sqrt{\pi}}\|f^{\prime}\|_{L^2(0,2\pi)}\sqrt{\frac{1}{m}\int_0^{2m\pi}\cos^2(u) du}
    = \frac{1}{m\sqrt{\pi}}\|f^{\prime}\|_{L^2(0,2\pi)}\sqrt{\frac{1}{2m}(2m\pi + \sin(2m\pi)\cos(2m\pi))} \\
    &= \frac{1}{m\sqrt{\pi}}\|f^{\prime}\|_{L^2(0,2\pi)}\sqrt{\pi} =
    \frac{1}{m}\|f^{\prime}\|_{L^2(0,2\pi)}
  \end{align*}
  \begin{align*}
    b_m := (f,C_m)_{L^2} &= \frac{1}{\sqrt{\pi}}\int_0^{2\pi} f(x)\cos(mx) dx
    = \frac{1}{\sqrt{\pi}}\left(\underbrace{\left[\frac{1}{m}f(x)\sin(mx)\right]_0^{2\pi}}_{=0}
    - \frac{1}{m}\int_0^{2\pi}f^{\prime}(x)\sin(mx) dx \right) \\
    &\leq \frac{1}{m}\|f^{\prime}\|_{L^2(0,\pi)}
  \end{align*}
\end{enumerate}

\end{solution}

% --------------------------------------------------------------------------------
