% --------------------------------------------------------------------------------

\begin{exercise}

\textit{(Exponential decay to equilibrium for the Fokker-Planck equation)} \\
We consider smooth solutions of the Fokker-Planck equation
\begin{align}\label{fokker-planck}
  \partial_{t} u=\nabla \cdot(\nabla u+u \nabla V), \quad t>0,
  \quad u(0)=u_{0} \quad \text { in } \mathbb{R}^{d},
\end{align}
where we assume that $V$ satisfies the Bakry-Emery condition
$\nabla^{2} V \geq \lambda \, \mathrm{Id}$ for some $\lambda>0$. Furthermore, assume that $\phi \in C^{4}((0, \infty))$ is convex with $\phi(1)=0,$ and $1 / \phi^{\prime \prime}$ is welldefined and concave. Examples of admissible functions $\phi$ are $\phi(s)=s(\log (s)-1)+1$ and $\phi(s)=s^{\alpha}-1$ for $1<\alpha \leq 2$.
\begin{enumerate}[label = (\roman*)]
  \item Compute $u_{\infty},$ a stationary solution of \eqref{fokker-planck} that is strictly positive
  $\left(u_{\infty}>0\right)$ and has unit mass $\left(\int_{\R^d} u_{\infty}=1\right)$.
  \item Define the relative entropy with respect to $\phi$ as
  \begin{align*}
    H_{\phi}[u]=\int_{\mathbb{R}^{d}} \phi\left(\frac{u}{u_{\infty}}\right) u_{\infty} \,\mathrm{d} x.
  \end{align*}
  Notice that, setting $\rho=\frac{u}{u_{\infty}},$ we have that
  $\partial_{t} u=\nabla \cdot\left(u_{\infty} \nabla \rho\right) .$
  Using this, show that the \textit{entropy production}, $-\frac{d}{d t} H_{\phi}[u],$ is non-negative.
  \item Show that
  \begin{align*}
    \nabla \partial_{t} \rho &= \nabla \Delta \rho-\nabla^{2} \rho \nabla V-\nabla^{2} V \nabla \rho, \\
    \nabla \rho \cdot \nabla \Delta \rho &= \nabla \cdot\left(\nabla^{2} \rho \nabla \rho\right)
    -\left|\nabla^{2} \rho\right|^{2},
  \end{align*}
  where $\left|\nabla^{2} \rho\right|^{2}=
  \sum_{i, j=1, \ldots, n}\left|\partial_{i} \partial_{j} \rho\right|^{2}$.
  Using these identities and the expression for $\frac{d}{d t} H_{\phi}[u]$
  that you have obtained in (i), show that
  \begin{align*}
    \frac{d^{2}}{d t^{2}} H_{\phi}[u] &\geq
    \int_{\mathbb{R}^{d}}\left(\phi^{\primeprime \primeprime}(\rho)|\nabla \rho|^{4}+4
    \phi^{\primeprimeprime}(\rho) \nabla \rho^{T} \nabla^{2} \rho \nabla \rho+2
    \phi^{\prime \prime}(\rho)\left|\nabla^{2} \rho\right|^{2}\right) u_{\infty} \mathrm{d} x
    -2 \lambda \frac{d}{d t} H_{\phi}[u] \\
    &= 2 \int_{\mathbb{R}^{d}} \phi^{\prime \prime}(\rho)\left|\nabla^{2}
    \rho+\frac{\phi^{\prime \prime \prime}(\rho)}{\phi^{\prime \prime}(\rho)}
    \nabla \rho \otimes \nabla \rho\right|^{2} u_{\infty} \mathrm{d} x
    +\int_{\mathbb{R}^{d}}\left(\phi^{\primeprime \primeprime}(\rho)-2
    \frac{\phi^{\primeprimeprime}(\rho)^{2}}{\phi^{\prime \prime}(\rho)}\right)|\nabla \rho|^{4}
    u_{\infty} \mathrm{d} x
    -2 \lambda \frac{d}{d t} H_{\phi}[u].
  \end{align*}
  \item Using the concavity of $1 / \phi^{\prime \prime}$ and convexity of $\phi,$
  argue that the result of (iii) yields that
  \begin{align}\label{eq}
    \frac{d^{2}}{d t^{2}} H_{\phi}[u] \geq-2 \lambda \frac{d}{d t} H_{\phi}[u], \quad t>0
  \end{align}
  \item Argue that integrating \eqref{eq} on the interval $(s, \infty)$ yields that
  \begin{align*}
    \frac{d}{d t} H_{\phi}[u(s)] \leq-2 \lambda H_{\phi}[u(s)], \quad s \geq 0
  \end{align*}
  \textit{Hint:} For this use Gronwall's lemma applied to \eqref{eq} and you may use
  (without proof) that \\
   $\lim _{t \rightarrow \infty} H_{\phi}[u(t)]=0$.
  \item Using (without proof) that
  \begin{align*}
    \left\|u(t)-u_{\infty}\right\|_{L^{1}\left(R^{d}\right)}
    \leq \frac{2}{\phi^{\prime \prime}(1)} H_{\phi}[u(t)],
  \end{align*}

  which follows from the Csiszár-Kullback-Pinsker inequality, show that
  \begin{align*}
    \left\|u(t)-u_{\infty}\right\|_{L^{1}\left(\mathrm{R}^{d}\right)}^{2}
    \leq \frac{1}{\phi^{\prime \prime}(1)} H_{\phi}\left[u_{0}\right] e^{-2 \lambda t}.
  \end{align*}
\end{enumerate}
\end{exercise}

% --------------------------------------------------------------------------------

\begin{solution}

\phantom{}
\begin{enumerate}[label = (\roman*)]
  \item Schreiben wir die Differentialgleichung zuerst schöner darauf

  \begin{align*}
    \partial_t u
    =
    \Delta u + \nabla u \cdot \nabla V + u \Delta V
  \end{align*}

  Bilden wir die Fourier-Transformierte:

  \begin{align*}
    0
    &=
    F[\Div(\nabla u + u \nabla V)](k)
    =
    \sum_{j = 1}^d F[\partial_{x_j,x_j}u](k) +  F[\partial_{x_j}(u \partial_{x_j}V)](k)
    =
    -|k^2| \hat{u} + \sum_{j = 1}^d i k_j \widehat{u \partial_{x_j}V} \\
    &=
    -|k^2| \hat{u} + \sum_{j = 1}^d i k_j
    \Int[\R^d]{
      u \partial_{x_j}V e^{-ik \cdot x}
    }{x}
  \end{align*}
\end{enumerate}

\end{solution}

% --------------------------------------------------------------------------------
