% --------------------------------------------------------------------------------

\begin{exercise}

\phantom{}

\begin{enumerate}[label = (\roman*)]

    \item Lösen Sie das Randwertproblem für die Laplacegleichung in ebenen Polarkoordinaten
    
    \begin{align*}
        (\Delta u)(r, \varphi)
        & =
        0 ~\text{für}~ r < R, \\
        u(R, \varphi)
        & =
        f(\varphi) ~\text{für alle}~ \varphi,
    \end{align*}

    wobei

    \begin{align*}
        \Delta u
        =
        u_{rr} + \frac{1}{r} u + \frac{1}{r^2} u_{\varphi \varphi}
    \end{align*}

    der Laplaceoperator in Polarkoordinaten, $R > 0$ eine positive Konstante und $f$ eine stückweise stetig differenzierbare $2 \pi$-periodische Funktion ist?

    \item Wie sieht die Lösung konkret im Fall
    
    \begin{align*}
        f(\varphi)
        =
        \begin{cases}
            0,  \varphi = 0, \pi \\
            1,  0 < \varphi < \pi \\
            -1, \pi < \varphi < 2 \pi
        \end{cases}
    \end{align*}

    mit $R = 1$ aus?

\end{enumerate}

\textit{Hinweis:}
Verwenden Sie einen Separationsansatz.
Betrachten Sie dazu zunächst Einzellösungen $u_n$ der Gestalt $u_n(r, \varphi) = v_n(r) \cdot w_n(\varphi)$ (mit $w_n$ $2 \pi$-periodisch), welche die Differentialgleichung erfüllen und insbesondere $C^2$ im Nullpunkt sind.
Die gesuchte Gesamtlösung ergibt sich dann als Summe über die Einzellösungen $u_n$ mit geeigneten Koeffizienten.
Falls Sie dabei auf die homogene eulersche Differentialgleichung 2. Ordnung stoßen, verwenden Sie Aufgabe 6 von Blatt 1 oder schlagen sie in einer beliebigen Quelle ein Fundamentalsystem von Lösungen nach.

\end{exercise}

% --------------------------------------------------------------------------------

\begin{solution}

ToDo!

\end{solution}

% --------------------------------------------------------------------------------
