% --------------------------------------------------------------------------------

\begin{exercise}

Sei $\Omega \subset \R^n$ ein beschränktes Gebiet mit $\partial\Omega \in C^2$.
Zeigen Sie, dass die \textit{biharmonische Gleichung}
\begin{align*}
  \Delta^2 u = f \text{ in } \Omega, \quad u = \nabla u \cdot \nu = 0 \text{ auf } \partial\Omega
\end{align*}
eine schwache Lösung besitzt, falls $f \in L^2(\Omega)$. Hierbei ist $\nu$ der
äußere Normaleneinheitsvektor auf $\partial\Omega$. Anleitung:
\begin{enumerate}[label = (\alph*)]
  \item Zeigen Sie zunächst, dass die schwache Formulierung des obigen Randwertproblems
  \begin{align*}
    \int_\Omega \Delta u \Delta v dx = \int_\Omega f v dx \text{ für alle } v \in H_0^2(\Omega)
  \end{align*}
  lautet.
  \item Zeigen Sie, dass $a(u,v) = \int_\Omega \Delta u \Delta v dx$ eine stetige
  und koerzive Bilinearform ist. Verwenden Sie hierbei folgende Ungleichung:
  Für alle $u \in H_0^2(\Omega)$ gilt $\|u\|_{H^2(\Omega)} \leq C\|\Delta u\|_{L^2(\Omega)}$.
  \item Zeigen Sie die Existenz einer schwachen Lösung des Randwertproblems.
\end{enumerate}
\end{exercise}

% --------------------------------------------------------------------------------

\begin{solution}

\phantom{}

\end{solution}

% --------------------------------------------------------------------------------
