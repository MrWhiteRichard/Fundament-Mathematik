% --------------------------------------------------------------------------------

\begin{exercise}

Sei $\Omega \subset \R^n$ ein beschränktes Gebiet mit $\partial\Omega \in C^2$.
Zeigen Sie, dass die \textit{biharmonische Gleichung}

\begin{align*}
  \Delta^2 u = f ~\text{in}~ \Omega,
  \quad
  u = \nabla u \cdot \nu = 0 ~\text{auf}~ \partial \Omega
\end{align*}

eine schwache Lösung besitzt, falls $f \in L^2(\Omega)$.
Hierbei ist $\nu$ der äußere Normaleneinheitsvektor auf $\partial\Omega$.
Anleitung:

\begin{enumerate}[label = (\alph*)]

  \item Zeigen Sie zunächst, dass die schwache Formulierung des obigen Randwertproblems

  \begin{align*}
    \Int[\Omega]{\Delta u \Delta v}{x} = \Int[\Omega]{f v}{x}
    ~\text{für alle}~ v \in H_0^2(\Omega)
  \end{align*}

  lautet.

  \item Zeigen Sie, dass $a(u, v) = \Int[\Omega]{\Delta u \Delta v}{x}$ eine stetige und koerzive Bilinearform ist.
  Verwenden Sie hierbei folgende Ungleichung:
  Für alle $u \in H_0^2(\Omega)$ gilt $\norm[H^2(\Omega)]{u} \leq C \norm[L^2(\Omega)]{\Delta u}$.

  \item Zeigen Sie die Existenz einer schwachen Lösung des Randwertproblems.

\end{enumerate}

\end{exercise}

% --------------------------------------------------------------------------------

\begin{solution}

\phantom{}

\begin{enumerate}[label = (\alph*)]

  \item Laut Satz 5.4 (Charakterisierng von Sobolevfunktionen) folgt aus $v \in H_0^2(\Omega)$, dass $\nabla v \in H_0^1(\Omega)$.
  
  \begin{align*}
    v \in H_0^2(\Omega)
    \implies
    \nabla v \in H_0^1(\Omega)
  \end{align*}

  Wir multiplizieren die Differentialgleichung mit $v \in H^1(\Omega)$, integrieren über $\Omega$ und integrieren zweimal partiell:

  \begin{align*}
    \Delta^2 u = f
    \implies
    (\Delta u) v = f v
  \end{align*}

  \begin{align*}
    \implies
    F(v)
    & :=
    \Int[\Omega]{f v}{x}
    =
    \Int[\Omega]{(\Delta^2 u) v}{x}
    =
    \Int[\Omega]{(\Div \nabla \Delta u) v}{x} \\
    & \stackrel
    {
      \mathrm{Gauß}
    }{=}
    -\Int[\Omega]{\nabla \Delta u \cdot \nabla v}{x}
    +
    \Int[\partial \Omega]{\underbrace{(\nabla u \cdot \nu)}_0 v}{x} \\
    & \stackrel
    {
      \mathrm{Gauß}
    }{=}
    \Int[\Omega]{\Delta u \Div \nabla v}{x}
    -
    \Int[\partial \Omega]{(\underbrace{\nabla v}_0 \cdot \nu) \Delta u}{s}
    =
    \Int[\Omega]{\Delta u \Delta v}{x}
    =:
    a(u, v)
  \end{align*}

  \item

  \begin{itemize}
    \item Stetigkeit:
    
    \begin{align*}
      \abs{\Delta v}^2
      =
      \abs
      {
        \sum_{i=1}^n
        \pderivative[2]{x_i} u
      }^2
      \stackrel
      {
        \mathrm{CSB}
      }{\leq}
      \pbraces
      {
        \sum_{i=1}^n
        1^2
      }
      \pbraces
      {
        \sum_{i=1}^n
        \abs{\pderivative[2]{x_i} u}^2
      }
      \leq
      n
      \sum_{i,j=1}^n
      \abs
      {
        \frac{\partial^2}{\partial x_i \partial x_j} u
      }^2
      =
      n \abs{\Hess u}^2
    \end{align*}
    
    \begin{align*}
      \implies
      \abs{a(u, v)}
      & =
      \abs{\Int[\Omega]{\Delta u \Delta v}{x}}
      \leq
      \Int[\Omega]{\abs{\Delta u \Delta v}}{x}
      \stackrel
      {
        \mathrm{CSB}
      }{\leq}
      \pbraces
      {
        \Int[\Omega]{\abs{\Delta u}^2}{x}
      }^{1/2}
      \pbraces
      {
        \Int[\Omega]{\abs{\Delta v}^2}{x}
      }^{1/2} \\
      & \leq
      \pbraces
      {
        \Int[\Omega]{n \abs{\Hess u}^2}{x}
      }^{1/2}
      \pbraces
      {
        \Int[\Omega]{n \abs{\Hess v}^2}{x}
      }^{1/2} \\
      & =
      n
      \pbraces
      {
        \Int[\Omega]{\abs{\Hess u}^2}{x}
      }^{1/2}
      \pbraces
      {
        \Int[\Omega]{\abs{\Hess v}^2}{x}
      }^{1/2} \\
      & =
      n \norm[L^2(\Omega)]{\Hess u} \norm[L^2(\Omega)]{\Hess v}
      \leq
      n \norm[H^1(\Omega)]{u} \norm[H^1(\Omega)]{v}
    \end{align*}

    \item Koerzivität:
    
    \begin{align*}
      a(u, u)
      =
      \Int[\Omega]{\Delta u \Delta u}{x}
      =
      \Int[\Omega]{\abs{\Delta u}^2}{x}
      =
      \norm[L^2(\Omega)]{\Delta u}^2
      \geq
      C^{-2} \norm[H^2(\Omega)]{u}^2
    \end{align*}

  \end{itemize}

  \item Laut Konstruktion, ist $H_0^2$ ist mit $(\cdot, \cdot)_{H^2(\Omega)}$ ein Hilberbraum.
  In (b) haben wir gezeigt, dass die Bilinearform $a$ stetig und koerziv ist.
  $F \in H^{-1}(\Omega)$ ist ein lineares und stetiges Funktional.

  \begin{align*}
    \abs{F(v)}
    =
    \abs{\Int[\Omega]{f v}{x}}
    \leq
    \Int[\Omega]{\abs{f v}}{x}
    =
    (\norm[L^1(\Omega)]{f v})
    \stackrel
    {
      \mathrm{CSB}
    }{\leq}
    \norm[L^2(\Omega)]{f} \norm[L^2(\Omega)]{v}
    \leq
    \norm[L^2(\Omega)]{f} \norm[H^2(\Omega)]{v}
  \end{align*}

  Laut dem Lemma von Lax-Milgram existiert nun genau ein $u \in H_0^2(\Omega)$, sodass $a(u, v) = F(v)$ für alle $v \in H_0^2(\Omega)$.

  \begin{align*}
      \implies
      \ExistsOnlyOne u \in H_0^2(\Omega):
      \Forall v \in H_0^2(\Omega):
      a(u, v) = F(v)
  \end{align*}

\end{enumerate}

\end{solution}

% --------------------------------------------------------------------------------
