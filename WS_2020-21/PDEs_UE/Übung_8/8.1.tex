% --------------------------------------------------------------------------------

\begin{exercise}

Sei $\Omega \subseteq \R^n$ ein beschränktes Gebiet mit $\partial\Omega \in C^1$.
Betrachten Sie die Poisson-Gleichung mit Neumann-Randbedingungen
\begin{align}
\begin{cases}
  -\Delta u = f & \text{in } \Omega, \\
  \nabla u \cdot \nu = 0 & \text{auf } \Omega,
\end{cases} \label{neumann}
\end{align}
wobei $f \in L^2(\Omega)$.
\begin{enumerate}[label = \alph*)]
  \item Bestimmen Sie die schwache Formulierung für das Randwertproblem \eqref{neumann}
  \item Beweisen Sie die Existenz und Eindeutigkeit einer schwachen Lösung $u \in V$,
  wobei $V := \{v \in H^1(\Omega): \int_\Omega v(x)dx = 0\}$, für das RWP \eqref{neumann}. \\
  \textit{Hinweis:} Poincaré-Ungleichung aus Aufgabe 4 von letzter Woche.
  \item Diskutieren Sie die Existenz und Eindeutigkeit von schwachen und klassischen
  Lösungen von \eqref{neumann}, falls $\int_\Omega f(x)dx \neq 0$.
\end{enumerate}
\end{exercise}

% --------------------------------------------------------------------------------

\begin{solution}

\phantom{}
\begin{enumerate}[label = \alph*)]
  \item Wir multiplizieren die Differentialgleichung mit $v \in H^1(\Omega)$
  und integrieren über $\Omega$
  \begin{align*}
    \int_\Omega -\Delta u v dx = \int_\Omega fv dx \iff
    \int_\Omega \nabla u \nabla v dx = \int_{\partial\Omega}(\nabla u \nu)v ds + \int_\Omega fv dx
  \end{align*}
  Wir definieren
  \begin{align*}
    a(u,v) &:= \int_\Omega \nabla u \nabla v dx \\
    F(v) &:= \int_{\partial \Omega} \underbrace{(\nabla u \nu)}_{=0}v ds + \int_\Omega fv dx
    = \int_\Omega fv dx.
  \end{align*}
  Dann lautet die schwache Formulierung: Finde $u \in H^1(\Omega)$, sodass
  \begin{align*}
    a(u,v) = F(v) \quad \text{für alle } v \in H^1(\Omega).
  \end{align*}
  \item Wir wollen wieder das Lemma von Lax-Milgram auf den Hilbertraum $V$ anwenden. Dafür prüfen wir
  die Voraussetzungen an

  \begin{itemize}
    \item Stetigkeit von $a$:
    \begin{align*}
      |a(u,v)| \leq \|\nabla u\|_{L^2(\Omega)}\|\nabla v\|_{L^2(\Omega)}
      \leq \|u\|_{H^1(\Omega)}\|v\|_{H^1(\Omega)}
    \end{align*}
    \item Koerzivität von $a$:
    \begin{align*}
      a(u,u) = \int_\Omega \nabla u \nabla u dx = \|\nabla u\|_{L^2(\Omega)}^2
      \geq \frac{1}{C^2+1}(\|u\|_{L^2(\Omega)}^2+ \|\nabla u\|_{L^2(\Omega)}^2)
      = \frac{1}{C^2+1}\|u\|_{H^1(\Omega)}^2
    \end{align*}
    \item Stetigkeit von $F$:
    \begin{align*}
      |F(v)| \leq \|f\|_{L^2(\Omega)}\|v\|_{L^2(\Omega)} \leq \|f\|_{L^2(\Omega)}\|v\|_{H^1(\Omega)}
    \end{align*}
  \end{itemize}
  Mit dem Lemma von Lax-Milgram existiert nun genau ein $u \in V$, sodass
  \begin{align*}
    a(u,v) = F(v) \quad \text{für alle } v \in V.
  \end{align*}
  Damit $u$ auch eine schwache Lösung ist, müssen wir diese Gleichheit jetzt noch
  für alle $w \in H^1(\Omega)$ zeigen. Mit $\overline{w} := \frac{1}{|\Omega|}\int_\Omega w(x) dx$ gilt
  \begin{align*}
    w \in H^1(\Omega) &\implies w - \overline{w} \in V \\
    a(u,w) &= a(u,w) - \int_\Omega \nabla u \underbrace{\nabla \overline{w}}_{=0}
    = a(u,w-\overline{w}) = F(w-\overline{w}) = F(w) - \overline{w}\underbrace{\int_\Omega f}_{=0} dx = F(w)
  \end{align*}
  \item Für $\int_\Omega f(x) dx \neq 0$ und eine schwache Lösung $u \in H^1(\Omega)$ folgt mit
  dem Satz von Gauß für Sobolev-Funktionen
  \begin{align*}
    0 \neq \int_\Omega f(x) dx = -\int_\Omega \Delta u = \int_{\partial \Omega} \nabla u \cdot \nu ds
    = \int_{\partial \Omega} 0 ds = 0.
  \end{align*}
ein Widerspruch! Also kann es in dem Fall keine schwachen und insbesondere keine klassischen Lösungen geben.
\end{enumerate}
\end{solution}

% --------------------------------------------------------------------------------
