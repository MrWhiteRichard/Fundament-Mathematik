% --------------------------------------------------------------------------------

\begin{exercise}

Betrachten Sie den Differentialoperator $Lu = u_{tt} - c^2u_{xx}$ für $c \in \R_+$
und $(x,t) \in \R^2$.
\begin{enumerate}[label = (\roman*)]
  \item Zeigen Sie, dass
  \begin{align*}
    G(x,t) = \frac{1}{2c}H(t)[H(x+ct) - H(x-ct)]
  \end{align*}
  eine Fundamentallösung von $L$ mit Pol in $(x,t) = (0,0)$ ist, wobei $H$
  die Heaviside-Funktion ist.
  \item Bestimmen Sie eine Fundamentallösung von $L$ mit Pol in $(x,t) = (\xi,\tau)$.
  \item Berechnen Sie $\partial_tG(x,t)$.
\end{enumerate}
\end{exercise}

% --------------------------------------------------------------------------------

\begin{solution}

\phantom{}
\begin{itemize}
    \item[(i)] \begin{align*}
    \langle LG, \phi \rangle = \langle G, L\phi \rangle &=
    \int_{\R^2} G(x, t) L\phi(x, t) \\
     &= \int_{\R} \int_{\R} \frac{1}{2c} \underbrace{H(t)}_{\mathbbm{1}_{(0, \infty)}(t)}
    \left[ \underbrace{H(x+ct) - H(x-ct)}_{\mathbbm{1}_{(-ct, ct)}(x)}\right]
    \left( \phi_{tt}(x, t) - c^2 \phi_{xx}(x, t) \right) dt dx \\
    &= \frac{1}{2c} \left(
    \int_\R \int_{|x|/c}^\infty \phi_{tt}(x, t) \mathrm{~d}t \mathrm{~d}x
    - c^2 \int_0^\infty \int_{-ct}^{ct} \phi_{xx}(x, t) \mathrm{~d}x \mathrm{~d}t \right)\\
    &= \frac{1}{2c} \left( \int_\R - \phi_t(x, |x|/c)
    - c^2 \int_0^\infty \phi_x(ct, t)  - \phi_x(-ct, t) \mathrm{~d}x\mathrm{~d}t \right)\\
    &= \frac{1}{2c}\left(- \int_\R \phi_t(x, |x|/c) \mathrm{~d}x
    - c^2 \int_0^\infty \phi_x(ct, t) \mathrm{~d}t + c^2
    \int_0^\infty \phi_x(-ct, t) \mathrm{~d}t\right)\\
    &\stackrel{(u = x/c)}{=~} - \frac{1}{2c} \left(c \int_0^\infty \phi_t(cu, u) \mathrm{~d}u + c
    \int_0^\infty \phi_t(-cu, u) \mathrm{~d}u + c^2 \int_0^\infty \phi_x(ct, t) \mathrm{~d}t
    - c^2\int_0^\infty \phi_x(-ct, t) \mathrm{~d}t\right) \\
    &= - \frac{1}{2} \left( \int_0^\infty c \phi_x(cu, u) + \phi_t(cu, u) \mathrm{~d}u +
    \int_0^\infty - c \phi_x(-cu, u) + \phi_t(-cu, u) \mathrm{~d}u \right) \\
    &\stackrel{(\ast)}{~=~} - \frac{1}{2}
    \left(\phi(cu, u) \big|_{u=0}^\infty + \phi(-cu, u) \big|_{u=0}^\infty \right) = \phi(0).
\end{align*}

Die Gleichheit ($\ast$) folgt hierbei aus der Kettenregel:
\begin{align*}
    \frac{d}{du} \left( u \mapsto \phi(cu, u) \right) = \nabla\phi\begin{pmatrix}cu\\u\end{pmatrix} \cdot \begin{pmatrix}c\\1\end{pmatrix} = c \phi_x(cu, u) + \phi_t (cu, u), \\
    \frac{d}{du} \left( u \mapsto \phi(-cu, u) \right) = \nabla\phi\begin{pmatrix}-cu\\u\end{pmatrix} \cdot \begin{pmatrix}-c\\1\end{pmatrix} = - c \phi_x(cu, u) + \phi_t (cu, u).
\end{align*}

\item[(ii)] Weil $L$ ein Differentialoperator mit konstanten Koeffizienten ist, erhalten wir gemäß Satz 3.21 eine Fundamentallösung im Pol $\xi = (\xi_1, \xi_2)$ durch
\begin{align*}
(t, x) \mapsto G(x - \xi_1, t - \xi_2).
\end{align*}

\item[(iii)]
\begin{align*}
    \langle \partial_t G, \phi \rangle = - \langle G, \partial_t \phi \rangle =
    - \int_{\R^2} G(x, t) \partial_t \phi(x, t) \mathrm{~d}(x, t) =
    - \frac{1}{2c} \int_\R \int_{|x|/c}^\infty \partial_t \phi(x, t)    \mathrm{~d}t \mathrm{~d}x = \\
    \frac{1}{2c} \int_\R \phi(x, |x|/c) \mathrm{~d}x =
    \frac{1}{2c} \left( \int_0^\infty \phi(x, x/c) \mathrm{~d}x \int_0^\infty \phi(-x, x/c) \mathrm{~d}x \right) \stackrel{(u = x/c)}{~=~} \\
    \frac{1}{2} \int_0^\infty \phi(cu, u) + \phi(-cu, u) \mathrm{~d}u.
\end{align*}

\end{itemize}

\end{solution}

% --------------------------------------------------------------------------------
