% --------------------------------------------------------------------------------

\begin{exercise}

Betrachten Sie den Differentialoperator $Lu = u_{tt} - c^2u_{xx}$ für $c \in \R_+$
und $(x,t) \in \R^2$.
\begin{enumerate}[label = (\roman*)]
  \item Zeigen Sie, dass
  \begin{align*}
    G(x,t) = \frac{1}{2c}H(t)[H(x+ct) - H(x-ct)]
  \end{align*}
  eine Fundamentallösung von $L$ mit Pol in $(x,t) = (0,0)$ ist, wobei $H$
  die Heaviside-Funktion ist.
  \item Bestimmen Sie eine Fundamentallösung von $L$ mit Pol in $(x,t) = (\xi,\tau)$.
  \item Berechnen Sie $\partial_tG(x,t)$.
\end{enumerate}
\end{exercise}

% --------------------------------------------------------------------------------

\begin{solution}

\phantom{}
\begin{enumerate}[label = (\roman*)]
  \item
  \begin{align*}
    \langle L(G(x,t)), \phi(x,t)\rangle
    &= \langle G(x,t), \phi_{tt}(x,t) - c^2\phi_{xx}(x,t)\rangle \\
    &= \int_{\R^2}\frac{1}{2c}H(t)[H(x+ct) - H(x-ct)](\phi_{tt}(x,t) - c^2\phi_{xx}(x,t))d(x,t) \\
    &= \frac{1}{2c}\int_{0}^{\infty}\int_{\R}[H(x+ct) - H(x-ct)](\phi_{tt}(x,t) - c^2\phi_{xx}(x,t)) dx dt \\
    &=  \frac{1}{2c}\int_{0}^{\infty}\int_{-ct}^{ct}c^2\phi_{xx}(x,t) - \phi_{tt}(x,t) dx  dt \\
    &=  \frac{1}{2c}\left(\int_{0}^{\infty}c^2(\phi_{x}(ct,t) - \phi_x(-ct,t))dt -
     \int_{\R}\int_{\R}\1_{[0,\infty]}(t)\1_{[-ct,ct]}(x) \phi_{tt}(x,t) dt dx\right) \\
    &=  \frac{1}{2c}\left(\int_{0}^{\infty}c^2(\phi_{x}(ct,t) - \phi_x(-ct,t))dt -
    \int_{\R}\int_{\R}\1_{[0,\infty]}(t)\1_{[-\frac{c}{x},\frac{c}{x}]}(t) \phi_{tt}(x,t) dt dx\right) \\
    &=  \frac{1}{2c}\left(\int_{0}^{\infty}c^2(\phi_{x}(ct,t) - \phi_x(-ct,t))dt -
    \int_{\R}\int_0^{c/|x|} \phi_{tt}(x,t) dt dx\right) \\
    &=  \frac{1}{2c}\left(\int_{0}^{\infty}c^2(\phi_{x}(ct,t) - \phi_x(-ct,t))dt -
    \int_{\R}\phi_t(x,\frac{c}{|x|}) - \phi_t(x,0) dx\right) \\
  \end{align*}
\end{enumerate}

\end{solution}

% --------------------------------------------------------------------------------
