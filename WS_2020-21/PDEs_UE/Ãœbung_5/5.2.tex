% --------------------------------------------------------------------------------

\begin{exercise}

Bestimmen Sie für das Randwertproblem
\begin{align*}
  -\frac{d}{dx}\left(k(x)\frac{du}{dx}\right) &= f(x), \quad 0 < x < 1 \\
  u(0) = u(1) &= 0
\end{align*}
mit $k(x) > 0$ für $0 \leq x \leq 1$ eine Funktion $g(x,y)$ (genannt
\textit{greensche Funktion}) sodass
\begin{align*}
  u(x) = \int_0^1g(x,y)f(y)dy
\end{align*}
das Randwertproblem löst. Gehen Sie dazu wie folgt vor:
\begin{enumerate}[label = (\roman*)]
  \item Multiplizieren Sie die Differentialgleichung mit einer Funktion $K(x)$,
  welche das homogene Problem
  \begin{align*}
    -\frac{d}{dx}\left(k(x)\frac{du}{dx}\right) = 0
  \end{align*}
  löst und die Randbedingung bei $x = 1$ erfüllt.
  \item Integrieren Sie die erhaltene Gleichung von $x = 0$ bis $x = 1$, um einen
  Ausdruck für $u^{\prime}(0)$ zu erhalten.
  \item Leiten Sie daraus eine Formel für $u^{\prime}(x)$ und dann $u(x)$ her.
  \item Bringen Sie die Formel für $u(x)$ auf die gewünschte Form.
\end{enumerate}
Überprüfen Sie am Ende, dass
\begin{align*}
  -\frac{d}{dy}\left(k(y)\frac{dg}{dy}(x,y)\right) &= \delta(x-y) \\
  g(x,0) = g(x,1) &= 0
\end{align*}
gilt. \\
\textit{Hinweis:} Die Methode ist im Spezialfall $k(x) = 1$ mit $K(x) = 1 - x$ etwas einfacher.
\end{exercise}

% --------------------------------------------------------------------------------

\begin{solution}

\phantom{}
\begin{enumerate}[label = (\roman*)]
  \item Lösen wir also das homogene Problem mit Randbedingung $u(1) = 0$:
  \begin{align*}
    -\frac{d}{dx}\left(k(x)\frac{du}{dx}\right) = 0
    \iff -k(x)u^{\prime}(x) = C
    \iff u(x) = C\int_x^1 \frac{1}{k(y)} dy
  \end{align*}
  Setzen wir $C = 1$ erhalten wir eine Lösung $K(x) = \int_x^1 \frac{1}{k(y)}dy$. \\
  Rechnen wir mal mal:
  \begin{align*}
    -(k(x)u^{\prime}(x))^{\prime}K(x) = f(x)K(x)
  \end{align*}
  \item
  Integrieren wir mal:
  \begin{align*}
    -\int_0^1(k(x)u^{\prime}(x))^{\prime}K(x)dx
    &= \int_0^1f(x)K(x)dx \\
    \iff -[k(x)u^{\prime}(x)K(x)]_{x=0}^1 + \int_0^1k(x)u^{\prime}(x)K^{\prime}(x)
    &= \int_0^1f(x)K(x)dx \\
    \iff k(0)u^{\prime}(0)K(0) - \underbrace{k(1)u^{\prime}(1)K(1)}_{=0} - \int_0^1u^{\prime}(x)
    &= \int_0^1f(x)K(x)dx \\
    \iff k(0)u^{\prime}(0)K(0) + \underbrace{(u(0) - u(1))}_{=0}
    &= \int_0^1f(x)K(x)dx \\
    \iff u^{\prime}(0) &= \frac{\int_0^1f(x)K(x)dx}{k(0)K(0)}
  \end{align*}
  \item
  \begin{align*}
    -(k(x)u^{\prime}(x))^{\prime} = f(x) &\iff -k(x)u^{\prime}(x) = \int_{0}^xf(x)dx - k(0)u^{\prime}(0)  \\
    &\iff u^{\prime}(x)
    = \frac{k(0)u^{\prime}(0) - \int_0^x f(y)dy}{k(x)} \\
    &\iff u(x) = \int_0^x \frac{k(0)u^{\prime}(0) - \int_0^y f(z) dz}{k(y)}dy \\
    &\iff u(x) =  k(0)u^{\prime}(0)\int_0^x \frac{1}{k(y)} - \int_0^y \frac{f(z)}{k(y)} dz dy \\
    &\iff u(x) =  k(0)u^{\prime}(0)(K(0) - K(x)) - \int_0^x\int_0^y \frac{f(z)}{k(y)} dz dy \\
    &\iff u(x) =  \frac{\int_0^1f(y)K(y)dy}{K(0)}(K(0) - K(x)) - \int_0^x\int_0^y \frac{f(z)}{k(y)} dz dy \\
    &\iff u(x) = \int_0^1\frac{f(z)K(z)}{K(0)}(K(0) - K(x))dz - \int_0^1\int_0^1\1_{[0,x]}(y) \1_{[0,y]}(z) \frac{f(z)}{k(y)}dy dz\\
    &\iff u(x) = \int_0^1\frac{f(z)K(z)}{K(0)}(K(0) - K(x)) - f(z)\int_0^1\1_{[0,x]}(y) \1_{[z,1]}(y) \frac{1}{k(y)}dy dz\\
    &\iff u(x) = \int_0^1\frac{f(z)K(z)}{K(0)}(K(0) - K(x)) - f(z)\1_{[0,x]}(z)(K(z) - K(x)) dz\\
    &\iff u(x) = \int_0^1f(z)\left[\frac{K(z)}{K(0)}(K(0) - K(x)) - \1_{[0,x]}(z)(K(z) - K(x))\right] dz\\
  \end{align*}
  Also erhalten wir mit
  \begin{align*}
    g(x,z) = \frac{K(z)}{K(0)}(K(0) - K(x)) - \1_{[0,x]}(z)(K(z) - K(x))
  \end{align*}
  die lange gesuchte greensche Funktion.
  \item Zuerst die Randbedingungen, die sind einfacher (und wir brauchen sie für die erste Gleichheit noch):
  \begin{align*}
    g(x,0) &= \frac{K(0)}{K(0)}(K(0) - K(x)) - \1_{[0,x]}(0)(K(0) - K(x))
    = (K(0) - K(x)) - (K(0) - K(x)) = 0,\\
    g(x,1) &= \frac{K(1)}{K(0)}(K(0) - K(x)) - \1_{[0,x]}(1)(K(1) - K(x)) = 0.
  \end{align*}
  Wie ist die erste Gleichheit zu verstehen? Ich kann für festes $x$
  zeigen, dass $\delta_x$ rauskommt...
  \begin{align*}
    -\left\langle\frac{d}{dy}\left(k(y)\frac{dg}{dy}(x,y)\right), \phi(y) \right\rangle &=
    -\left\langle g(x,y), (k(y)\phi^{\prime}(y))^{\prime} \right\rangle \\
    &= -\int_0^1 g(x,y)(k(y)\phi^{\prime}(y))^{\prime} dy \\
    &= -[g(x,y)k(y)\phi^{\prime}(y)]_{y=0}^1 + \int_0^1k(y)\phi^{\prime}(y)g_y(x,y)dy \\
    &= \int_0^1k(y)\phi^{\prime}(y)g_y(x,y)dy \\
    &= \int_0^xk(y)\phi^{\prime}(y)\left(\frac{K(x) - K(0)}{K(0)k(y)} + \frac{1}{k(y)}\right)dy
    + \int_x^1k(y)\phi^{\prime}(y)\left(\frac{K(x) - K(0)}{K(0)k(y)}\right)dy\\
    &= \int_0^x\phi^{\prime}(y)\frac{K(x)}{K(0)}dy + \int_x^1\phi^{\prime}(y)\frac{K(x) - K(0)}{K(0)} \\
    &= \frac{K(x)}{K(0)}\phi(x) - \phi(0) + \phi(1)\frac{K(x) - K(0)}{K(0)} - \phi(x)\frac{K(x) - K(0)}{K(0)}= \phi(x).
  \end{align*}
  Zum Schluss verwenden wir die kuriose Tatsache, dass $\phi(0) = \phi(1) = 0$,
  da $\phi$ eine Testfunktion auf $[0,1]$ ist.
\end{enumerate}

\end{solution}

% --------------------------------------------------------------------------------
