% --------------------------------------------------------------------------------

\begin{exercise}

Bestimmen Sie mit der Spiegelungsmethode eine greensche Funktion für das
Dirichlet-Problem für $\Delta$ auf dem Keil
\begin{align*}
  \Omega := \{(x,y) \in \R^2: 0 < x \text{ und } 0 < y < x\}.
\end{align*}

\end{exercise}

% --------------------------------------------------------------------------------

\begin{solution}

Betrachte die Fundamentallösung von $\Delta$ mit Pol $(\xi,\eta) \in \Omega$.
Nun ergänzen wir diese Fundamentallösung mit insgesamt sieben Spiegelpolen,
mit alternierenden Vorzeichen. Die Pole lauten also wie folgt:

\begin{align*}
  +(\xi,\eta), \quad& -(\eta,\xi) \\
  +(-\eta,\xi), \quad& -(-\xi,\eta) \\
  +(-\xi,-\eta), \quad& -(-\eta,-\xi) \\
  +(\eta,-\xi), \quad& -(\xi,-\eta)
\end{align*}

Dass das so auch funktioniert lässt sich, so wie in Aufgabe 3, mit Matrizen veranschaulichen.

\begin{align*}
  \underbrace
  {
    \begin{pmatrix}
      NW & N & NO \\
      W  &   & O  \\
      SW & S & SO
    \end{pmatrix}
  }_{
    \text{keine}
  }
  & -
  \underbrace
  {
    \begin{pmatrix}
      SW & S & SO \\
      W  &   & O  \\
      NW & N & NO
    \end{pmatrix}
  }_{
    x ~\text{Achse}
  }
  -
  \underbrace
  {
    \begin{pmatrix}
      NO & N & NW \\
      O  &   & W  \\
      SO & S & SW
    \end{pmatrix}
  }_{
    y ~\text{Achse}
  }
  +
  \underbrace
  {
    \begin{pmatrix}
      SO & S & SW \\
      O  &   & W  \\
      NO & N & NW
    \end{pmatrix}
  }_{
    x \& y ~\text{Achse}
  } \\
  & -
  \underbrace
  {
    \begin{pmatrix}
      SO & O & NO \\
      S  &   & N  \\
      SW & W & NW
    \end{pmatrix}
  }_{
    \text{Neben-Diagonale}
  }
  +
  \underbrace
  {
    \begin{pmatrix}
      SW & W & NW \\
      S  &   & N  \\
      SO & O & NO
    \end{pmatrix}
  }_{
    \text{Neben-Diagonale}~
    \&~
    x ~\text{Achse}
  }
  +
  \underbrace
  {
    \begin{pmatrix}
      NO & O & SO \\
      N  &   & S  \\
      NW & W & SW
    \end{pmatrix}
  }_{
    \text{Neben-Diagonale}~
    \&~
    y ~\text{Achse}
  } \\
  & -
  \underbrace
  {
    \begin{pmatrix}
      NW & W & SW \\
      N  &   & S  \\
      NO & O & SO
    \end{pmatrix}
  }_{
    \text{Neben-Diagonale}~
    \&~
    x \& y ~\text{Achse}
  }
  =
  0
\end{align*}

Die Summe aller dieser Fundamentallösung hat dann genau einen Pol in unserem Keil und lautet

\begin{align*}
  G(x,y,\xi,\eta) = \frac{1}{4\pi}(
  \ln((x-\xi)^2 + (y-\eta)^2) -
  \ln((x-\eta)^2 + (y - \xi)^2) +
  \ln((x + \eta)^2 + (y - \xi)^2) -
  \ln((x+ \xi)^2 + (y - \eta)^2) + \\
  \ln((x + \xi)^2 + (y + \eta)^2) -
  \ln((x + \eta)^2 + (y + \xi)^2) +
  \ln((x - \eta)^2 + \ln(y + \xi)^2) -
  \ln((x - \xi)^2 + (y + \eta)^2))
\end{align*}
Klarerweise ist diese Funktion eine Fundamentallösung in unserem Gebiet $\Omega$.
Für die Randbedingungen an $\partial \Omega = \{(x,y) \in \R_+^2: y = 0 \lor x = y\}$. \\
Fall 1: $y = 0$
\begin{align*}
  G(x,0,\xi,\eta) &= \frac{1}{4\pi}(
  \ln((x-\xi)^2 + \eta^2) -
  \ln((x-\eta)^2 + \xi^2) +
  \ln((x + \eta)^2 + \xi^2) -
  \ln((x+ \xi)^2 + \eta^2) + \\
  &\ln((x + \xi)^2 + \eta^2) -
  \ln((x + \eta)^2 + \xi^2) +
  \ln((x - \eta)^2 + \xi^2) -
  \ln((x - \xi)^2 + \eta^2)) \\
  &= \frac{1}{4\pi}(
  \underbrace{\ln((x-\xi)^2 + \eta^2) -
  \ln((x - \xi)^2 + \eta^2)}_{=0} -
  \underbrace{\ln((x-\eta)^2 + \xi^2) +
  \ln((x - \eta)^2 + \xi^2)}_{=0} + \\
  &\underbrace{\ln((x + \eta)^2 + \xi^2) -
  \ln((x + \eta)^2 + \xi^2)}{=0} -
  \underbrace{\ln((x+ \xi)^2 + \eta^2) +
  \ln((x + \xi)^2 + \eta^2)}_{=0})
\end{align*}
Fall 2: $x = y$
\begin{align*}
  G(x,x,\xi,\eta) &= \frac{1}{4\pi}(
  \ln((x-\xi)^2 + (x-\eta)^2) -
  \ln((x-\eta)^2 + (x - \xi)^2) +
  \ln((x + \eta)^2 + (x - \xi)^2) -
  \ln((x+ \xi)^2 + (x - \eta)^2) + \\
  &\ln((x + \xi)^2 + (x + \eta)^2) -
  \ln((x + \eta)^2 + (x + \xi)^2) +
  \ln((x - \eta)^2 + \ln(x + \xi)^2) -
  \ln((x - \xi)^2 + (x + \eta)^2)) \\
  &= \frac{1}{4\pi}(
  \underbrace{\ln((x-\xi)^2 + (x-\eta)^2) -
  \ln((x-\eta)^2 + (x - \xi)^2)}_{=0} +
  \underbrace{\ln((x + \eta)^2 + (x - \xi)^2) -
  \ln((x+ \xi)^2 + (x - \eta)^2)}_{=0} + \\
  &\underbrace{\ln((x + \xi)^2 + (x + \eta)^2) -
  \ln((x + \eta)^2 + (x + \xi)^2)}_{=0} +
  \underbrace{\ln((x - \eta)^2 + \ln(x + \xi)^2) -
  \ln((x - \xi)^2 + (x + \eta)^2)}_{=0}).
\end{align*}

\end{solution}

% --------------------------------------------------------------------------------
