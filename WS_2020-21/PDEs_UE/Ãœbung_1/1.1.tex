% -------------------------------------------------------------------------------- %

\begin{exercise}

Berechnen Sie die Operatornorm von

\begin{align*}
    T: \ell^2 \to \ell^1,
    \quad
    Tx = (x_1, x_2 / 2, x_3 / 3, x_4 / 4, \ldots)..
\end{align*}

\end{exercise}

% -------------------------------------------------------------------------------- %

\begin{solution}

Wir zeigen, dass die Operatornorm $\norm{T} = \pi / \sqrt{6}$ ist.

Sei dazu $x \in \ell^2$.
Laut der Cauchy-Schwarz-Ungleichung gilt folgende Abschätzung.

\begin{align*}
    \norm[1]{Tx}^2
    =
    \left(\sum_{n=1}^\infty
    \vbraces
    {
        \frac{x_n}{n}
    }\right)^2
    =
    \left(\sum_{n=1}^\infty
    \vbraces{x_n}
    \vbraces
    {
        \frac{1}{n}
    }\right)^2
    \stackrel{\text{CSB}}{\leq}
    \underbrace
    {
        \pbraces
        {
            \sum_{n=1}^\infty
            x_n^2
        }
    }_{
        \norm[1]{x}^2
    }
    \underbrace
    {
        \pbraces
        {
            \sum_{n=1}^\infty
            \frac{1}{n^2}
        }
    }_{
        \pi^2 / 6
    }
\end{align*}

Dadurch erhalten wir eine Abschätzung nach oben für die Operatornorm von $T$.

\begin{align*}
    \norm{T}
    :=
    \sup
    \Bbraces
    {
        \frac
        {
            \norm[1]{Tx}
        }{
            \norm[2]{x}
        }:
        x \in \ell^2 \setminus \Bbraces{0}
    }
    \leq
    \pi / \sqrt{6}
\end{align*}

Für die andere Ungleichung, betrachten wir die Folge $x := (1/n)_{n \in \N} \in \ell^2$.

\begin{align*}
    \implies
    Tx = (1/n^2)_{n \in \N} \in \ell^1
    \implies
    \norm{T}
    \geq
    \frac
        {
            \norm[1]{Tx}
        }{
            \norm[2]{x}
        }
    =
    \pi / \sqrt{6}
\end{align*}

\end{solution}

% -------------------------------------------------------------------------------- %
