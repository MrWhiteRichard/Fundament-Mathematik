% --------------------------------------------------------------------------------

\begin{exercise}

Sei $\Omega \subset \R^n$ offen und $u \in C^2(\Omega)$.
Zeigen Sie:

\begin{enumerate}[label = (\roman*)]
    \item Wenn $\phi \in C^\infty(\R)$ konvex und $u$ harmonisch ist, dann ist $v = \phi(u)$ subharmonisch, d.h. $\Delta v \geq 0$.
    \item Ist $u \in C^3(\Omega)$ harmonisch, so ist $v = |\nabla u|^2$ subharmonisch.
\end{enumerate}

\end{exercise}


% --------------------------------------------------------------------------------

\begin{solution}
  \begin{itemize}

  \item[(i)] Durch Anwenden von Ketten- und Produktregel erhalten wir
  \begin{align}
      \frac{\partial^2 v}{\partial^2 x_i} = \frac{\partial v}{\partial x_i} \left(\frac{\partial v}{\partial x_i}\right) = \frac{\partial v}{\partial x_i} \left(\phi^\prime(u(x)) \frac{\partial u}{\partial x_i}\right) = \phi^{\prime\prime}(u(x)) \left(\frac{\partial u}{\partial x_i}\right)^2 + \phi^\prime(u(x)) \frac{\partial^2 u}{\partial^2 x_i}.
  \end{align}
  Aus der Konvexität von $\phi$ und der Harmonität von $u$ folgt daraus
  \begin{align}
      \Delta v(x) = \underbrace{\phi^{\prime\prime}(u(x))}_{\geq 0, \text{~weil $\phi$ konvex}} \sum_{i=1}^n \left(\frac{\partial u}{\partial x_i}\right)^2 + \phi^\prime(u(x)) \underbrace{\sum_{i=1}^n \frac{\partial^2 u}{\partial^2 x_i}}_{= 0} \geq 0.
  \end{align}
  \end{itemize}

\end{solution}

% --------------------------------------------------------------------------------
