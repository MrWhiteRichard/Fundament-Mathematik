% -------------------------------------------------------------------------------- %

\begin{exercise}

Zeigen Sie mit Hilfe der vorherigen Aufgabe:
ist $u \in H^1(\Omega)$, dann sind auch $u^+ = \max \Bbraces{u, 0}$, $u^- = - \min \Bbraces{u, 0}$ und $|u|$ in $H^1(\Omega)$.
(Hinweis: die ersten beiden Aussagen ergeben sich direkt aus der dritten.)

\end{exercise}

% -------------------------------------------------------------------------------- %

\begin{solution}
Wir zeigen nach dem Hinweis zuerst $|u| \in H^1(\Omega)$ für $u \in H^1(\Omega)$. Dazu approximieren wir
die Betragsfunktion mit $n \in \N$ bel. durch

\begin{align*}
  f_n(x)
  :=
  \begin{cases}
    p_n(x) \quad , |x| < \frac{1}{n} \\
    |x|  \quad , |x| \geq \frac{1}{n}
  \end{cases}
\end{align*}

wobei $p_n$ das Hermite-Interpolationspolynom ist, sodass also $p_n(\pm \frac{1}{n}) = \frac{1}{n}$ und
$p_n^\prime(\pm \frac{1}{n}) = \pm 1$ gilt. Explizit ist

\begin{align*}
  p_n(x)
  =
  -\frac{n^3}{2}x^4 + \frac{3n}{2}x^2
  =
  x^2(-\frac{n^3}{2}x^2 + \frac{3n}{2})
\end{align*}

Es gilt also $f_n \in C^1(\R), f_n(0) = 0,\sup_{y \in \R}|f_n^\prime(y)|<\infty$ und damit können wir nach
Aufgabe $7$ schließen, dass $\forall n \in \N: f_n \circ u \in H^1(\Omega)$ und $\partial_i (f_n \circ u) =
f_n^\prime \circ u \partial_i u$. Nun gilt es

\begin{align*}
  \norm[H^1(\Omega)]{f_n \circ u - f \circ u}
  \xrightarrow{n \to \infty}
  0
\end{align*}

also die Konvergenz von $f_n \circ u$ gegen $|u|$ in $H^1(\Omega)$ zu zeigen. Dazu zeigen wir
die Konvergenz der Funktionenfolge als auch deren Ableitungen in $L^2(\Omega)$.
Um später majorisierte Konvergenz anwenden zu können, zeigen wir zuerst noch $\forall x \in [-\frac{1}{n},\frac{1}{n}]: |x| \geq p_n(x)$. Dazu sei $g_n(x):= x - p_n(x)$, explizit

\begin{align*}
  g_n(x)
  =
  x(1 + \frac{n^3}{2}x^3 - \frac{3n}{2}) \\
  g_n(x) = 0
  \Leftrightarrow
  x = 0 \lor x = \frac{1}{n} \lor x = - \frac{2}{n}
\end{align*}

Zwischen den Nullstellen gibt es keine Vorzeichenwechsel, deswegen schließen wir mit

\begin{align*}
  g_n(\frac{1}{2n})
  =
  \frac{1}{2n}(1 + \frac{1}{16} - \frac{3}{4})
  =
  \frac{1}{2n} \frac{5}{16}
  >
  0
\end{align*}

dass $\forall x \in [0, \frac{1}{n}]: g_n(x) \geq 0$, also $x \geq p_n(x)$. Weil die $p_n$ gerade Funktionen sind, gilt dann schließlich $\forall x \in [-\frac{1}{n},\frac{1}{n}]: |x| \geq p_n(x)$.
Nun zeigen wir die Konvergenz in $L^2(\Omega)$:

\begin{align*}
  \lim_{n \to \infty} \norm[L^2(\Omega)]{f \circ u - f_n \circ u}^2
  =
  \lim_{n \to \infty} \Int[\Omega]{f(u(x)) - f_n(u(x))^2}{x}
  =
  \lim_{n \to \infty} \Int[\Omega]{\1_{
    [-\frac{1}{n}, \frac{1}{n}]
    }(u(x)) \Big(|u(x)|-p_n(u(x))\Big)^2
    }{x} \\
  =
  \Int[\Omega]{\underbrace{\lim_{n \to \infty}\1_{
    [-\frac{1}{n}, \frac{1}{n}]
    }(u(x))}_{0} \Big(|u(x)|-p_n(u(x))\Big)^2}{x}
  =
  0
\end{align*}

Wobei wir unsere integrierbare Majorante mit $|\1_{
  [-\frac{1}{n}, \frac{1}{n}]
  }(u(x)) \Big(|u(x)|-p_n(u(x))\Big)^2| \leq 4|u(x)|^2$
  wobei wir die zuvor gezeigte Abschätzung der $p_n$ verwenden. Nun zeigen wir die Konvergenz der
  Ableitungen, dabei zeigen wir, dass die Folge der Ableitungen eine Cauchy-Folge in $L^2(\Omega)$ ist, womit
  wir erhalten, dass $f_n \circ u$ eine Cauchy-Folge in $H^1(\Omega)$ ist.

  Seien also $n,m \in \N, n < m$, $\epsilon > 0$ beliebig.

  \begin{align*}
    \norm[L^2(\Omega)]{\partial_i (f_n \circ u) - \partial_i (f_m \circ u)}^2
    =
    \norm[L^2(\Omega)]{(f_n^\prime \circ u - f_m^\prime \circ u) \partial_i u}^2
    =
    \Int[\Omega]{\Big(f_n^\prime(u(x)) - f_m^\prime(u(x))\Big)^2 (\partial_i u(x))^2}{x} \\
    =
    \Int[\Omega]{\1_{
      [-\frac{1}{n}, \frac{1}{n}]
      }(u(x)) \Big(p_n^\prime(u(x))-f_m^\prime(u(x))\Big)^2 (\partial_i u(x))^2
    }{x}
    =
    \cdots
  \end{align*}

Wir machen eine Fallunterscheidung für $p_n^\prime - f_m^\prime$:
\begin{enumerate}[label = \textbf{Fall \roman*:}]
  \item $\frac{1}{m} \leq y \leq \frac{1}{n}$: Hier gilt

    \begin{align*}
      f_m^\prime(y) - p_n^\prime(y)
      =
      1 - 3ny + 2n^3 y^3
      \leq
      1 + 2n^3 y^3
      \leq
      3 \\
      p_n^\prime(y) - f_m^\prime(y)
      =
      3ny - 2n^3 y^3 - 1
      \leq
      3ny
      \leq
      3
    \end{align*}

  \item $0 \leq y < \frac{1}{m}$: Hier gilt

    \begin{align*}
      p_m^\prime(y) - p_n^\prime(y)
      =
      2y^3(n^3-m^3) + 3y(m-n)
      \leq
      3 \frac{m-n}{m}
      \leq
      3 (1 - \frac{n}{m})
      \leq
      3 \\
      p_n^\prime(y) - p_m^\prime(y)
      =
      2y^3(m^3-n^3) + 3y(n-m)
      \leq
      2y^3(m^3-n^3)
      \leq
      3
    \end{align*}
\end{enumerate}

Wir können also $|p_n^\prime - f_m^\prime|$ in jedem Fall mit $3$ abschätzen (da die Ableitungen ungerade Funktionen sind) und erhalten so insgesamt

\begin{align*}
  \cdots
  \leq
  9 \Int[\Omega]{\1_{
    [-\frac{1}{n}, \frac{1}{n}]
    }(u(x)) (\partial_i u(x))^2}{x}
    <
    \epsilon \quad \text{für}~ n ~\text{hinreichend groß}
\end{align*}
Diese letzte Ungleichung erhalten wir widerum durch majorisierte Konvergenz, da sich der
Integrand durch $(\partial_i(u(x)))^2$ abschätzen lässt.

Da wir nun wissen, dass $f_n \circ u$ eine Cauchy-Folge in $H^1(\Omega)$ ist und wir schon
$f_n \circ u \rightarrow f \circ u$ in $L^2(\Omega)$ gezeigt haben, folgern wir $f_n \circ u \rightarrow f \circ u$ in $H^1(\Omega)$.

Um noch die ersten beiden Aussagen zu zeigen schreiben wir

\begin{align*}
  \max\{u,0\}
  &=
  \frac{u + 0 + |u - 0|}{2} \in H^1(\Omega) \\
  -\min\{u, 0\}
  &=
  -\frac{u + 0 - |u - 0|}{2} \in H^1(\Omega)
\end{align*}
\end{solution}

% -------------------------------------------------------------------------------- %
