% --------------------------------------------------------------------------------

\begin{exercise}

Beweisen Sie mithilfe der Ungleichung von Cauchy-Schwarz:

\begin{align*}
    \sqrt{x (3x + y)} + \sqrt{y (3y + z)} + \sqrt{z (3z + x)}
    \leq
    2 (x + y + z)
\end{align*}

\end{exercise}

% --------------------------------------------------------------------------------

\begin{solution}

Bezeichne $(\cdot, \cdot)$ das euklidische Skalarprodukt im $\R^3$.
Der Übersicht halber definieren wir noch $a, b \in \R^3$.

\begin{align*}
    a & := \pbraces
    {
        \sqrt{x (3x + y)},
        \sqrt{y (3y + z)},
        \sqrt{z (3z + x)}
    }^T, \\
    b & := (1, 1, 1)^T
\end{align*}

\begin{align*}
    \implies
    \text{lhs}
    =
    (a, b)
    \stackrel{\text{CSB}}{\leq}
    \norm[2]{a} \norm[2]{b}
    & =
    (
        \underbrace
        {1^2 + 1^2 + 1^2}_{3}
    )^{1/2}
    (
        \underbrace{x (3x + y)}_{3x^2 + xy}
        +
        \underbrace{y (3y + z)}_{3y^2 + yz}
        +
        \underbrace{z (3z + x)}_{3z^2 + zx}
    )^{1/2} \\
    & \stackrel{!}{\leq}
    \sqrt{3} \sqrt{3}
    (
        \underbrace
        {x^2 + 2xy + y^2 + 2yz + z^2 + 2zx}_{(x + y + z)^2}
    )^{1/2}
    \stackrel{!}{=}
    \text{rhs}
\end{align*}

Bei dem ersten \Quote{!}, verlangen wir, dass $xy, yz, zx \geq 0$;
Bei dem zweiten, gehen wir davon aus, dass in der Angabe ein $3$er statt einem $2$er stehen sollte.

\end{solution}

% --------------------------------------------------------------------------------
