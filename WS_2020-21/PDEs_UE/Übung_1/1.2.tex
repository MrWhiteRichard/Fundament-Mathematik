% --------------------------------------------------------------------------------

\begin{exercise}

Sei $(\mathcal{P}([0, 1]), \norm[\infty]{\cdot})$ der normierte Vektorraum aller Polynome $p: [0, 1] \to \K$.
Ist die Abbildung $D: \mathcal{P}([0, 1]) \to \mathcal{P}([0, 1]), Dp = p^\prime$ beschränkt?
Wenn ja, berechnen Sie $\norm{D}$.

\end{exercise}

% --------------------------------------------------------------------------------

\begin{solution}

Wir zeigen, dass der Operator $D$ auf $\mathcal{P}([0, 1])$ unbeschränkt ist.

Dazu, definieren wir die folgenden Polynome.

\begin{align*}
    p_0(x) & := 1 \\
    p_1(x) & := 1/2 + x/2 \\
    p_2(x) & := 1/3 + x/3 + x^2 / 3 \\
    \vdots & \\
    p_n(x) & := \sum_{i=0}^n \frac{x^i}{n+1}, \quad n \in \N_0
\end{align*}

Für $n \in \N_0$ berechnen wir die Norm von $p_n$ ...

\begin{align*}
    \norm[\infty]{p_n}
    :=
    \sup_{x \in [0, 1]} |p_n(x)|
    =
    \sum_{i=0}^n \frac{1}{n+1} = 1
\end{align*}

... und dessen Ableitung $p_n^\prime(x) = \sum_{i=1}^n \frac{i x^i}{n+1}$ ...

\begin{align*}
    \norm[\infty]{p_n^\prime}
    :=
    \sup_{x \in [0, 1]} |p_n^\prime(x)|
    =
    \sum_{i=1}^n \frac{i}{n+1}
    =
    \frac{1}{n+1} \frac{n (n+1)}{2} = n/2.
\end{align*}

Damit, kann $\norm{D}$ beliebig groß werden.

\begin{align*}
    \implies
    \norm{D}
    :=
    \sup
    \Bbraces
    {
        \frac
        {
            \norm[\infty]{Dp}
        }{
            \norm[\infty]{p}
        }:
        p \in \mathcal{P}([0, 1]) \setminus \Bbraces{0}
    }
    \geq
    \frac{\norm[\infty]{p_n^\prime}}{\norm[\infty]{p_n}}
    =
    n/2 \xrightarrow{n \to \infty} \infty
\end{align*}

\end{solution}

% --------------------------------------------------------------------------------
