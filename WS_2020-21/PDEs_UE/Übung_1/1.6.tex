% --------------------------------------------------------------------------------

\begin{exercise}

Lösen Sie die homogene eulersche Differentialgleichung

\begin{align*}
    a_2 x^2 y^\primeprime(x) + a_1 x y^\prime(x) + a_0 y(x) = 0
\end{align*}

mit $a_2, a_1, a_0 \in \R$ durch die Transformation $z(t) = y(e^t)$.

\end{exercise}

% --------------------------------------------------------------------------------

\begin{solution}

Wir wollen eine ODE lösen, in der $z, z^\prime, z^\primeprime$ vorkommen.
Man kann aber nicht einfach bei den Ableitungen $e^t$ für $x$ substituieren.
Das liegt daran, dass $y^\prime = y^\prime$, wir wollen aber eigentlich ($z$) nach $t$ ableiten.
Aushilfe bietet die Kettenregel ...

\begin{align*}
    z(t)
    =
    y(e^t)
    & \iff
    y(e^t)
    =
    z(t) \\
    z^\prime(t)
    =
    y^\prime(e^t) e^t
    & \iff
    y^\prime(e^t)
    =
    z^\prime(t) e^{-t} \\
    z^\primeprime(t)
    =
    \derivative{t}
    \bbraces
    {
        y^\prime(e^t) e^t
    }
    =
    \underbrace
    {
        \derivative{t}
        \bbraces
        {
            y^\prime(e^t)
        }
    }_{
        y^\primeprime(e^t) e^t
    }
    e^t
    +
    \underbrace
    {
        y^\prime(e^t)
    }_{
        z^\prime(t) e^{-t}
    }
    e^t
    & \iff
    y^\primeprime(e^t)
    =
    (
        z^\primeprime(t)
        -
        z^\prime(t)
    )
    e^{-2t}
\end{align*}

JETZT können wir endlich substituieren.

\begin{align*}
    \text{rhs}
    =
    a_2 (e^t)^2 y^\primeprime(e^t)
    +
    a_1 e^t y^\prime(e^t)
    +
    a_0 y(e^t)
    & =
    a_2 e^{2t}
    (
        z^\primeprime(t)
        -
        z^\prime(t)
    )
    e^{-2t}
    +
    a_1 e^t
    z^\prime(t)
    e^{-t}
    +
    a_0
    z(t) \\
    & =
    a_2 z^\primeprime(t) + (a_1 - a_2) z^\prime(t) + a_0 z(t)
\end{align*}

Diese ODE lässt sich nun leicht lösen.

\begin{align*}
    Z := (z, z^\prime)^T
    \implies
    Z^\prime(t)
    =
    \underbrace
    {
        \begin{pmatrix}
            0                & 1 \\
            -\frac{a_0}{a_2} & \frac{a_2 - a_1}{a_2}
        \end{pmatrix}
    }_{=: A}
    Z(t)
    \implies
    Z(t) = e^{tA}
\end{align*}

Wenn man das $z$ gefunden hat, muss man nur noch rücksubstituieren.

\begin{align*}
    z(t) = y(e^t)
    \iff
    z(\ln{x}) = y(x)
\end{align*}

\end{solution}

% --------------------------------------------------------------------------------
