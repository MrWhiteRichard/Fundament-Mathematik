% -------------------------------------------------------------------------------- %

\begin{exercise}

Zeigen Sie, dass folgendes Problem nur die triviale Lösung $u = 0$ hat:

\begin{align*}
    \pderivative[2][u]{x} + \pderivative[2][u]{y} & = u^3 ~\text{für}~ x^2 + y^2 < 1, \\
    \quad
    u & = 0 ~\text{für}~ x^2 + y^2 = 1.
\end{align*}

\end{exercise}

% -------------------------------------------------------------------------------- %

\begin{solution}
	Sei $u \in C^2(B_1(0)) \cap C^0(\overline{B_1(0)})$ eine Lösung des Problems.
  Dann ist auch $v := u^2$ in dieser Menge und es gilt
	\begin{align*}
	v_{xx} = \pderivative[][]{x}(2 u u_x) = 2 u_x^2 + 2 u u_{xx}
	\end{align*}
	und analog für $v_{yy}$ also gilt in $B_1(0)$ die Gleichung
	\begin{align*}
	\Delta v = 2 (u_x^2 + u_y^2) + 2 u \Delta u = 2(u_x^2 + u_y^2) + 2 u^4 \geq 0.
	\end{align*}
	und weiters gilt $v \geq 0$ sowie auf $\partial B_1(0)$ die Gleichung $v = 0$. Nach dem Maximumsprinzip nimmt $v$ das Maximum am Rand an, deshalb muss $v = 0$ auf ganz $B_1(0)$ gelten, also ist $u^2 = v = 0$ und daher $u = 0$.

\end{solution}

% -------------------------------------------------------------------------------- %
