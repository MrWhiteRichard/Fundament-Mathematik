% --------------------------------------------------------------------------------

\begin{exercise}

Zeigen Sie mit Hilfe der vorherigen Aufgabe:
ist $u \in H^1(\Omega)$, dann sind auch $u^+ = \max \Bbraces{u, 0}$, $u^- = - \min \Bbraces{u, 0}$ und $|u|$ in $H^1(\Omega)$.
(Hinweis: die ersten beiden Aussagen ergeben sich direkt aus der dritten.)

\end{exercise}

% --------------------------------------------------------------------------------

\begin{solution}

\begin{align*}
  f(x) &:= |x| \in C(\R)
\end{align*}
Sei $\eta_\delta$ der Standard-Mollifier. Dann ist $f\ast \eta_\delta \in C^{\infty}(\R)$
und es gilt $(f\ast\eta_\delta) \circ u \in H^1(\Omega)$, falls $f\ast\eta_\delta(0) = 0$.
Berechnen wir die Ableitung von $f\ast\eta_\delta:$
\begin{align*}
  |(f\ast\eta_\delta)^{\prime}(x)| = |f^{\prime}\ast\eta_\delta(x)| = |\sgn\ast\eta_\delta(x)| \leq 1.
\end{align*}
Schaun wir uns $f\ast\eta_\delta(0)$ an:
\begin{align*}
  f\ast\eta_\delta(0) = \int_{\R}f(-y)\eta_\delta(y) dy \neq 0.
\end{align*}
Zu zeigen:
\begin{align*}
  \|(f\ast\eta_\delta) \circ u - f\circ u\|_{H^1(U)} \xrightarrow{\delta \to 0} 0.
\end{align*}
Ana3: $\|f- f\ast\eta_\delta\|_{L^p(\Omega)} \to 0$.
\end{solution}

% --------------------------------------------------------------------------------
