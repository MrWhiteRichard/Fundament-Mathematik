% -------------------------------------------------------------------------------- %

\begin{exercise}

Zeigen Sie, dass die Aussage von Beispiel 1. schon für $u \in C^1(\Omega)$ gilt und $u$ dann sogar in $C^\infty(\Omega)$ ist. \\
\textit{Hinweis:}
Schreiben Sie die Mittelwerteigenschaft als $u(x) = u \ast v$ für eine geeignete Funktion $v$;
versuchen Sie dann $v$ durch eine besser geeignete Funktion $\phi \in C^2(\R^n)$ zu ersetzen.

\end{exercise}

% -------------------------------------------------------------------------------- %

\begin{solution}



	Wir wählen eine radialsymmetrische Testfunktion $\varphi$ mit den Eigenschaften
	\begin{align*}
	\supp \varphi \subseteq \overline{B_1(0)}, \quad \varphi \geq 0, \quad \Int[\R^n]{\varphi}{\lambda^n} = 1.
	\end{align*}
	Die skalierte Standardtestfunktion
  \begin{align*}
    \varphi(x) = \begin{cases}
      \left(\int_{B_1(0)}\exp\left(\frac{1}{|y|^2-1}\right) dy\right)
      \exp\left(\frac{1}{|x|^2-1}\right), & x \in B_1(0) \\
      0, & \text{sonst}
    \end{cases}
  \end{align*} bietet sich dafür an. Weiters definieren wir für alle $\varepsilon \in \R^+$ die Funktion $\varphi_\varepsilon (x) := \varepsilon^{-n} \varphi\pbraces{\frac{1}{\varepsilon}x}$.
  Definiere die offene Menge $\Omega_{\delta} := \{x \in \Omega: \dist(x,\partial\Omega) > \delta\}$.
  Dann folgt für $x \in \Omega_{\delta}$ und beliebiges $\delta > 0$ wegen $B_{\delta}(x) \subset \Omega$
	\begin{align*}
	(u \ast \varphi_\delta)(x) &= \Int[B_\delta(x)]{\varphi_\delta(x - y) u(y) }{\lambda^n}
  = \Int[0][\delta]{\Int[\partial B_\rho(x)]{\varphi_\delta(x - y) u(y) }{H^{n-1}(y)}}{\rho} \\
	 &= \Int[0][\delta]{\delta^{-n} \varphi\pbraces{\frac{\rho}{\delta}}
   \Int[\partial B_\rho(x)]{ u(y) }{H^{n-1}(y)}}{\rho} = \delta^{-n}
   \Int[0][\delta]{\varphi\pbraces{\frac{\rho}{\delta}}S_n \rho^{n-1} u(x)}{\rho} \\
	 &\stackrel{\eta = \rho/\delta}{=} \delta^{-n} S_n u(x) \Int[0][1]{\varphi(\eta) (\eta \delta)^{n-1} \delta }{\eta}
   = u(x) S_n  \Int[0][1]{\varphi(\eta) \eta^{n-1}  }{\eta} \\
	 &= u(x) \Int[0][1]{\int_{\partial B_{\eta}(0)}\varphi(z)  dz}{\eta}
   = u(x) \Int[B_1(0)]{\varphi(z)}{\lambda^n(z)} = u(x)
	\end{align*}
  Mit Lemma 3.14 erhalten wir also $u \in C^{\infty}(\Omega_{\delta})$
  und da Stetigkeit eine lokale Eigenschaft ist, folgt mit der Tatsache, dass
  $\bigcup_{\delta > 0}\Omega_\delta = \Omega$ schon $u \in C^{\infty}(\Omega)$.
  Also können wir Aufgabe 1 anwenden.
  Meines Erachtens nach sind die gegebenen Voraussetzungen sogar unnötig stark.
  Für $u \in L^1(\Omega)$ sollten alle Schritte auch gelten.
\end{solution}

% -------------------------------------------------------------------------------- %
