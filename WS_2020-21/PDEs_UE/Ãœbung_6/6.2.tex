% --------------------------------------------------------------------------------

\begin{exercise}

Zeigen Sie, dass die Aussage von Beispiel 1. schon für $u \in C^1(\Omega)$ gilt und $u$ dann sogar in $C^\infty(\Omega)$ ist. \\
\textit{Hinweis:}
schreiben sie die Mittelwerteigenschaft als $u(x) = u \ast v$ für eine geeignete Funktion $v$;
versuchen Sie dann $v$ durch eine besser geeignete Funktion $\phi \in C^2(\Omega)$ zu ersetzen.

\end{exercise}

% --------------------------------------------------------------------------------

\begin{solution}

Wähle zunächst $v(x) = -\frac{1}{S_nR^{n-1}}\1_{B_R(0)}(x)$. Dann folgt
\begin{align*}
  u*v(x) = \int_{\R^n}u(x-y)v(y) dy = -\frac{1}{S_nR^{n-1}}\int_{B_R(0)}u(x-y) dy = \frac{1}{S_nR^{n-1}}\int_{B_R(x)}u(y) dy = u(x).
\end{align*}
Jetzt versuchen wir $v$ durch $C^{\infty}$-Funktionen zu approximieren,
wobei $(\eta_{\epsilon})$ ein radialsymmetrischer Mollifier ist:
\begin{align*}
  v_{\epsilon} := v*\eta_{\epsilon}
\end{align*}
Wir wissen $\lim_{\epsilon  \to 0}\|v*\eta_{\epsilon} - v\|_{L^p} = 0$
\end{solution}

% --------------------------------------------------------------------------------
