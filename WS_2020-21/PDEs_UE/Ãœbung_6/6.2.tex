% --------------------------------------------------------------------------------

\begin{exercise}

Zeigen Sie, dass die Aussage von Beispiel 1. schon für $u \in C^1(\Omega)$ gilt und $u$ dann sogar in $C^\infty(\Omega)$ ist. \\
\textit{Hinweis:}
schreiben sie die Mittelwerteigenschaft als $u(x) = u \ast v$ für eine geeignete Funktion $v$;
versuchen Sie dann $v$ durch eine besser geeignete Funktion $\phi \in C^2(\Omega)$ zu ersetzen.

\end{exercise}

% --------------------------------------------------------------------------------

\begin{solution}



	Wir wählen eine Testfunktion $\varphi$ mit den Eigenschaften
	\begin{align*}
	\supp \varphi \subseteq \overline{B_1(0)}, \quad \varphi \geq 0, \quad \Int[\R^n]{\varphi}{\lambda^n} = 1, \quad |x| = |y| \Rightarrow \varphi(x) = \varphi(y).
	\end{align*}
	Die skalierte Standardtestfunktion bietet sich dafür an. Weiters definieren wir für alle $\varepsilon \in \R^+$ die Funktion $\varphi_\varepsilon (x) := \varepsilon^{-n} \varphi\pbraces{\frac{1}{\varepsilon}x}$. Gegeben sei eine beliebige kompakte Menge $K \subseteq \Omega$. Wählen wir nun $0 < \delta < \dist(\partial \Omega, K)$ so gilt für jedes $x \in K$
	\begin{align*}
	(u \ast \varphi_\delta)(x) &= \Int[B_\delta(x)]{\varphi_\delta(x - y) u(y) }{\lambda^n} = \Int[0][\delta]{\Int[\partial B_\rho(x)]{\varphi_\delta(x - y) u(y) }{H^{n-1}(y)}}{\rho} \\
	 &= \Int[0][\delta]{\delta^{-n} \tilde{\varphi}\pbraces{\frac{\rho}{\delta}}\Int[\partial B_\rho(x)]{ u(y) }{H^{n-1}(y)}}{\rho} = \delta^{-n} \Int[0][\delta]{\tilde{\varphi}\pbraces{\frac{\rho}{\delta}}S_n \rho^{n-1} u(x)}{\rho} \\
	 &= \delta^{-n} S_n u(x) \Int[0][1]{\tilde{\varphi}(\eta) (\eta \delta)^{n-1} \delta }{\eta} = u(x) S_n  \Int[0][1]{\tilde{\varphi}(\eta) \eta^{n-1} \delta }{\eta} \\
	 &= u(x) \Int[B_1(0)]{\varphi(z)}{\lambda^n(z)} = u(x)
	\end{align*}
\end{solution}

% --------------------------------------------------------------------------------
