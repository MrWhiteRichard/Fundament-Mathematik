% -------------------------------------------------------------------------------- %

\begin{exercise}

Sei $U \subseteq \R^n$ offen und beschränkt und $u \in H^1(U)$.
Angenommen, $f \in C^1(\R)$ und $\sup_{y \in \R} |f^\prime(y)| < \infty$. Zeigen Sie, dass $f \circ u \in H^1(U)$ und $\partial_i(f \circ u) = f^\prime(u) \partial_i u$ gilt.
Zeigen Sie dasselbe Resultat für unbeschränkte $U$ unter der zusätzlichen Voraussetzung $f(0) = 0$.

\end{exercise}

% -------------------------------------------------------------------------------- %

\begin{solution}
Wir wissen zweierlei Dinge:
\begin{align*}
  H^1(U) &:= \overline{\{u \in C^{\infty}(U); \|u\|_{H^1(U)} < \infty\}} \\
  u \in H^1(U) &\iff u \in L^2(U) \land \partial_i u \in L^2(U), \quad i=1,\dots,n.
\end{align*}
Definitionsgemäß können wir also $u$ in der $\|\cdot\|_{H^1(U)}$-Norm durch
$C^{\infty}(U)$-Funktionen $u_n$ mit $\|u_n\|_{H^1(U)} < \infty$ approximieren.
Wir zeigen zuerst $f \circ u \in L^2(\Omega)$. Da $f \in C^1(\R)$ und die Ableitung
beschränlt ist, ist $f$ auch Lipschitz-stetig mit Lipschitz-Konstante $L := \sup_{y \in \R} |f^\prime(y)|$ (folgt aus MWS) und es folgt
\begin{align*}
  \|f\circ u - f\circ u_n\|_{L^2(U)}^2 &= \int_U (f \circ u - f\circ u_n)^2 dx
  = \int_U (f(u(x))- f(u_n(x)))^2 dx \\
  &\leq \int_U L^2(u(x) - u_n(x))^2 dx = L^2\|u - u_n\|_{L^2(U)}^2
  \leq L^2\|u - u_n\|_{H^1(U)}^2  \xrightarrow{n \to \infty} 0.
\end{align*}
Fall 1: $U$ unbeschränkt, aber dafür $f(0) = 0$:
\begin{align*}
  \|f\circ u_n\|_{L^2(U)}^2 = \int_U f(u_n(x))^2dx = \int_U [f(u_n(x)) - f(0)]^2 dx
  \leq \int_U L^2u_n(x)^2 = L^2\|u_n\|_{L^2(U)} < \infty
\end{align*}
Fall 2: $U$ beschränkt: (für beschränkte Mengen gilt $L^2(U) \subset L^1(U)$)
\begin{align*}
  \|f\circ u_n\|_{L^2(U)}^2 &= \int_U f(u_n(x))^2dx = \int_U (f(u_n(x)) - f(0))^2 dx - \int_U f(0)^2 dx + 2f(0) \int_U f(u_n(x)) dx \\
  &\leq \int_U L^2 |u_n(x)|^2 dx - \lambda^n(U)f(0)^2 + 2f(0) \int_U f(u_n(x)) -f(0) + f(0) dx \\
  &\leq L^2\|u_n\|_{L^2(U)} + \lambda^n(U)f(0)^2 + 2f(0) L \int_U u_n(x)dx \\
  &\leq L^2\|u_n\|_{H^1(U)} + \lambda^n(U)f(0)^2 + 2f(0) L \|u_n\|_{L^1(U)} < \infty
\end{align*}
Da alle $f\circ u_n \in L^2(U)$ ist aufgrund der Abgeschlossenheit von $L^2(U)$
auch $f\circ u \in L^2(U)$. \\
Jetzt müssen wir noch die distributionellen Ableitungen von $f\circ u$ nachprüfen:
Sei dazu $\phi \in \mathcal{D}(U)$ beliebig.
\begin{align*}
  \langle \partial_i(f\circ u_n), \phi \rangle &= - \langle f\circ u_n, \partial_i \phi \rangle \\
  &= -\int_U f(u_n(x)) \partial_i\phi(x) dx = \int_U \partial_i f(u_n(x)) \phi(x) dx
  - \underbrace{\int_{\partial U} f(u_n(x))\phi(x) dx}_{=0} \\
  &= \int_U f^\prime(u_n(x)) \partial_i u_n(x) \phi(x) dx
  = \langle f^{\prime}(u_n)\partial_i u_n, \phi \rangle
\end{align*}
Dichtheitsargumente:
\begin{align*}
  \langle \partial_i(f\circ u_n) - \partial_i(f\circ u), \phi \rangle
  = \int_U[f(u_n(x)) - f(u(x))]\partial_i\phi(x) dx \leq
  \|f\circ u_n - f\circ u\|_{L^2(U)}\|\partial_i \phi\|_{L^2(U)} \xrightarrow{n \to \infty} 0.
\end{align*}
Damit folgt
\begin{align*}
  \langle \partial_i (f \circ u), \phi \rangle
  = \lim_{n \to \infty}\langle \partial_i (f \circ u_n), \phi \rangle
  = \lim_{n \to \infty}\langle f^{\prime}(u_n)\partial_i u_n, \phi \rangle.
\end{align*}
Berechnen wir nun den letzten Grenzwert
\begin{align*}
  \langle f^{\prime}(u_n)\partial_i u_n - f^{\prime}(u)\partial_i u, \phi \rangle &=
  \int_U [f^{\prime}(u_n(x))\partial_i u_n(x) - f^{\prime}(u(x))\partial_i u(x)] \phi(x) dx \\
  &= \int_U [(f^{\prime}(u_n(x)) -f^{\prime}(u(x)))\partial_i u_n(x)
  + (\partial_i u_n(x) - \partial_i u(x))f^{\prime}(u(x))]\phi(x) dx \\
  &\leq \|(f^{\prime}(u_n) - f^{\prime}(u))\phi(x)\|_{L^2(U)}
  \underbrace{\|\partial_i u_n(x)\|_{L^2(U)}}_{< \infty}
   + \underbrace{\|\partial_i u_n - \partial_i u\|_{L^2(U)}\|f^{\prime}(u)\phi\|_{L^2(U)}}_{\to 0}
\end{align*}
Mit majorisierter Konvergenz erhalten wir aufgrund
$(f^{\prime}(u_n(x) - f^{\prime}(u(x))^2 < 4\sup_{y \in \R}f^{\prime}(y)^2$
\begin{align*}
  \lim_{n \to \infty}\int_U (f^{\prime}(u_n(x)) - f^{\prime}(u(x)))^2\phi(x)^2 dx
  &= \int_{\supp(\phi)} \phi(x)^2\lim_{n \to \infty}(f^{\prime}(u_n(x)) - f^{\prime}(u(x)))^2 dx \\
  &= \int_{\supp(\phi)}\phi(x)^2(f^{\prime}(u(x)) - f^{\prime}(u(x)))^2 dx = 0.
\end{align*}
Insgesamt erhalten wir also $\langle \partial_i (f \circ u), \phi \rangle = f^{\prime}(u)\partial_i u$.
Jetzt gilt es noch zu überprüfen, dass diese distributionellen Ableitungen wieder
in $L^2(U)$ liegen:
\begin{align*}
  \int_{U}(f^{\prime}(u(x))\partial_i u(x))^2 dx
  \leq \sup_{y \in \R}f^{\prime}(y)^2\|\partial_i u\|_{L^2(U)}^2
  \leq \sup_{y \in \R}f^{\prime}(y)^2\|u\|_{H^1(U)}^2 < \infty.
\end{align*}
Damit ist auch $f \circ u \in H^1(U)$ und das Beispiel ist gelöst.
\end{solution}

% -------------------------------------------------------------------------------- %
