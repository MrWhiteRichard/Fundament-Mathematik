% --------------------------------------------------------------------------------

\begin{exercise}

Zeigen Sie:
Ist $w$ harmonisch auf einer offenen Menge $\Omega \subseteq \R^n$, $r > 0$, $x \in \Omega$ sodass $\overline{B_r(x)} \subseteq \Omega$, dann gilt für alle $i = 1, \dots, n:$

\begin{align*}
    |\partial_i w|
    \leq
    \frac{n}{r}
    \norm[L^\infty(B_r(x))]{w}.
\end{align*}

Zeigen Sie weiters, dass für jeden Multiindex $\alpha$ mit $|\alpha| = k$ gilt:

\begin{align*}
    |\partial^\alpha w(x)|
    \leq
    \pbraces
    {
        \frac{kn}{r}
    }^k
    \norm[L^\infty(B_r(x))]{w}.
\end{align*}

\end{exercise}

% --------------------------------------------------------------------------------

\begin{solution}

Mit dem Satz von Schwarz erhalten wir, dass auch $\partial_i w$ harmonisch ist.
Somit folgt mit der Mittelwerteigenschaft, sowie dem Satz vor dem Satz von Gauß, dass

\begin{align*}
  |\partial_iw(x)|
  &= \left|\frac{n}{S_nr^n}\int_{B_{r}(x)}\partial_iw(y) dy\right| \\
  &= \left|\frac{n}{S_nr^n}\int_{\partial B_{r}(x)}w(y)\nu_i(y) dS(y)\right| \\
  &= \left|\frac{n}{S_nr^n}\int_{\partial B_{r}(x)}w(y)\frac{y_i}{|y|} dS(y)\right| \\
  &\leq \left|\frac{n}{S_nr^n}S_nr^{n-1}\|w\|_{L^\infty(\partial B_{r}(x))}\right| \\
  &= \frac{n}{r}\|w\|_{L^\infty(\partial B_{r}(x))} \leq \frac{n}{r}\|w\|_{L^\infty(B_{r}(x))}.
\end{align*}

Sei $\alpha$ nun ein beliebiger Multiindex der Ordnung $k$.
Wähle $i$ und $\beta$, sodass $\partial^\alpha w = \partial_i(\partial^\beta w)$.
($\beta$ hat Ordnung $k-1$.)
Da $\partial^\alpha w$ wieder harmonisch ist folgt

\begin{align*}
  |\partial^\alpha w(x)|
  &= \left|\frac{k^nn}{S_nr^n}\int_{B_{r/k}(x)}\partial^\alpha w(y) dy\right| \\
  &= \left|\frac{k^nn}{S_nr^n}\int_{B_{r/k}(x)}\partial_i(\partial^\beta w(y)) dy\right| \\
  &= \left|\frac{k^nn}{S_nr^n}\int_{\partial B_{r/k}(x)}\partial^\beta w(y)\frac{y_i}{|y|} dy\right| \\
  &\leq \left|\frac{k^nn}{S_nr^n}\frac{S_nr^{n-1}}{k^{n-1}}
  \|\partial^\beta w\|_{L^\infty(\partial B_{r/k}(x))}\right| \\
  &= \frac{kn}{r}\|\partial^\beta w\|_{L^\infty(\partial B_{r/k}(x))}
  \leq \frac{kn}{r}\|\partial^\beta w\|_{L^\infty( B_{r/k}(x))}.
\end{align*}

Jetzt gilt es die Prozedur zu iterieren.
Für $y \in B_{r/k}(x)$ gilt $B_{r/k}(y) \subset B_{2r/k}(x) \subset \Omega$.
Analog zu obiger Rechnunge erhalten wir also Folgendes.
($\gamma$ sei dabei ein Multiindex mit Ordnung $k-2$.)

\begin{align*}
  \implies
  |\partial^\beta w(y)|
  =
  \cdots
  \leq
  \frac{kn}{r}
  \norm
  [
    L^\infty(B_{r/k}(y))
  ]
  {\partial^\gamma w}
  \leq
  \frac{kn}{r}
  \norm
  [
    L^\infty(B_{2r/k}(x))
  ]
  {\partial^\gamma w}
\end{align*}

Weil $y$ beliebig war, gilt die Abschätzung auch mit linker Seite $\norm[L^\infty( B_{r/k}(x))]{\partial^\beta w}$.
Nach der $k$-ten Iteration erhalten wir damit die Behauptung.

\end{solution}

% --------------------------------------------------------------------------------
