% --------------------------------------------------------------------------------

\begin{exercise}

Sei $\Omega \subset \R^n$ offen und $u \in C^2(\Omega)$.
Zeigen Sie:

\begin{enumerate}[label = (\roman*)]
    \item Wenn $\phi \in C^\infty(\R)$ konvex und $u$ harmonisch ist, dann ist $v = \phi(u)$ subharmonisch, d.h. $\Delta v \geq 0$.
    \item Ist $u \in C^3(\Omega)$ harmonisch, so ist $v = |\nabla u|^2$ subharmonisch.
\end{enumerate}

\end{exercise}


% --------------------------------------------------------------------------------

\begin{solution}
  \begin{itemize}

  \item[(i)] Durch Anwenden von Ketten- und Produktregel erhalten wir
  \begin{align}
      \frac{\partial^2 v}{\partial^2 x_i} = \frac{\partial v}{\partial x_i} \left(\frac{\partial v}{\partial x_i}\right) = \frac{\partial v}{\partial x_i} \left(\phi^\prime(u(x)) \frac{\partial u}{\partial x_i}\right) = \phi^{\prime\prime}(u(x)) \left(\frac{\partial u}{\partial x_i}\right)^2 + \phi^\prime(u(x)) \frac{\partial^2 u}{\partial^2 x_i}.
  \end{align}
  Aus der Konvexität von $\phi$ und der Harmonität von $u$ folgt daraus
  \begin{align}
      \Delta v(x) = \underbrace{\phi^{\prime\prime}(u(x))}_{\geq 0, \text{~weil $\phi$ konvex}} \sum_{i=1}^n \left(\frac{\partial u}{\partial x_i}\right)^2 + \phi^\prime(u(x)) \underbrace{\sum_{i=1}^n \frac{\partial^2 u}{\partial^2 x_i}}_{= 0} \geq 0.
  \end{align}
  \item[(ii)] Nach Definition ist $|\nabla u|^2 = \sum_{i=1}^n \left(\frac{\partial u}{\partial x_i}\right)^2$. Wir berechnen
\begin{align}
    \frac{\partial^2}{\partial^2 x_j} \left(\frac{\partial u}{\partial x_i}\right)^2 (x) = \frac{\partial}{\partial x_j} \left(2 \frac{\partial u}{\partial x_i} \cdot \frac{\partial^2 u}{\partial x_j \partial x_i} \right)(x) = \\
    2 \left(\frac{\partial^2 u}{\partial x_j \partial x_i}\right)^2 (x) + 2\frac{\partial u}{\partial x_i} \cdot \frac{\partial^3 u}{\partial^2 x_j \partial x_i} (x).
\end{align}
Durch Summation und aus dem Satz von Schwarz erhalten wir
\begin{align}
    \frac{\partial^2}{\partial^2 x_j} |\nabla u|^2 (x) = 2 \sum_{i=1}^n \left[ \left(\frac{\partial^2 u}{\partial x_j \partial x_i} \right)^2(x) + \frac{\partial u}{\partial x_i}(x) \cdot \frac{\partial}{\partial x_i} \left(\frac{\partial^2 u}{\partial^2 x_j}\right)(x) \right].
\end{align}
Abermaliges Summieren liefert
\begin{align}
    \frac{1}{2} \Delta v (x) = \sum_{j=1}^n \sum_{i=1}^n \left(\frac{\partial^2 u}{\partial x_j \partial x_i} \right)^2(x) +
    \sum_{i=1}^n \frac{\partial u}{\partial x_i}(x) \cdot \frac{\partial}{\partial x_i} \left( \underbrace{\sum_{j=1}^n \frac{\partial^2 u}{\partial^2 x_j}(x)}_{= 0} \right) \geq 0.
\end{align}
  \end{itemize}

\end{solution}

% --------------------------------------------------------------------------------
