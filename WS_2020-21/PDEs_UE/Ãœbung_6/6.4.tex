% --------------------------------------------------------------------------------

\begin{exercise}

Es sei $(a, b) \subset \R$ ein beschränktes offenes Intervall und $L := \derivative[2]{x} + g \derivative{x}$, mit $g: (a, b) \to \R$ eine beschränkte Funktion.
Zeigen Sie, dass für eine Funktion $u \in C^2((a, b)) \cap C([a, b])$ folgende Implikationen gelten.

\begin{enumerate}[label = (\roman*)]
    \item $Lu > 0$ in $(a, b)$ $\implies$ $u$ kann ihr Maximum nicht in $(a, b)$ annehmen.
    \item $Lu \geq 0$ in $(a, b)$ $\implies$ $u$ kann ihr Maximum nicht in $(a, b)$ annehmen, außer $u$ ist konstant.
\end{enumerate}

Überlegen Sie, ob (i) und (ii) ihre Gültigkeit behalten, wenn $g$ nur in jedem abgeschlossenen Intervall $[a^\prime, b^\prime] \subset (a, b)$, aber nicht unbedingt in $[a, b]$ beschränkt ist.

\end{exercise}

% --------------------------------------------------------------------------------

\begin{solution}
	\phantom{}
	\begin{enumerate}[label = (\roman*)]
		\item Wir führen den Beweis durch Kontraposition. Sei $x_0$ das Maximum von $u$, also $\forall x \in (a, b): u(x) \leq u(x_0)$. Damit gilt $u^\prime(x_0) = 0$ und $u^{\prime\prime}(x_0) \leq 0$ und daher
		\begin{align*}
		0 < u^{\prime\prime}(x_0) + g(x_0) u^\prime(x_0) = u^{\prime\prime}(x_0) \leq 0. \quad \lightning
		\end{align*}
		\item Wir führen einen Beweis durch Widerspruch. Wir wählen eine Stelle $x_0 \in (a,b)$ an der $u$ das Maximum annimmt, also $\forall x \in (a,b): u(x) \leq u(x_0)$. Weiters wissen wir, da $u$ nicht konstant ist, dass es ein $\eta \in (a,b)$ gibt mit $u(\eta) < u(x_0)$. Ohne Beschränkung der Allgemeinheit sei $\eta > x_0$. Nun definieren wir
		\begin{align*}
		x_1 := \sup\Bbraces{y \in [x_0, b) \mid \forall z \in [x_0, y]: u(z) = u(x_0)} < b.
		\end{align*}

    ---------------------------------alte Version--------------------------------------------------------\\
		Als nächstes wählen wir $0 < \delta_0 < b - x_1$. Nach Definition von $x_1$ finden wir ein $x_2 \in (x_1,x_1 + \delta_0)$ mit $u(x_2) < u(x_1)$ und danach mit dem Mittelwertsatz ein $x_3 \in (x_1, x_2]$ mit $u^\prime(x_3) < 0$. Da $x_1$ Maximum von $u$ ist gilt $u^\prime(x_1) = 0$ und $u^{\prime\prime}(x_1) \leq 0$ daher finden wir wieder mit dem Mittelwertsatz ein $x_4 \in (x_1, x_3]$ mit $u^{\prime\prime}(x_4) < 0$. Als nächstes wählen wir $\delta_1 \in (0, \delta_0)$ mit $u^{\prime\prime}(x_4) < - \delta_1$.  Sei
		\begin{align*}
		M := \sup\Bbraces{|g(y)|: y \in [x_1,x_1 + \delta_1]}.
		\end{align*}
		Nun wählen wir $\delta_2 \in (0, \delta_1)$ mit $\delta_2 < \frac{1}{2M}$. Nach Voraussetzung gilt $u^{\prime\prime}(x_1) = u^{\prime\prime}(x_1) + g(x_1) u^\prime(x_1) \geq 0$, also $u^{\prime\prime}(x_1) = 0$. Nach dem Zwischenwersatz können wir daher
		\begin{align*}
		x_5 := \inf\Bbraces{z \in (x_1, x_1 + \delta_1): u^{\prime\prime}(z) = -\delta_2}
		\end{align*}
		definieren. Da die Menge wegen der Stetigkeit von $u^{\prime\prime}$ abgeschlossen ist handelt es sich tatsächlich um ein Minimum. Noch ein letzetes Mal verwenden wir den Mittelwertsatz und erhalten ein $x_6 \in (x_1, x_5)$ mit
		\begin{align*}
		u^\prime(x_5) = u^\prime(x_1) + u^{\prime\prime}(x_6) (x_5 - x_1) = u^{\prime\prime}(x_6) (x_5 - x_1)
		\end{align*}
		Wir bemerken hier noch, dass $u^{\prime\prime}(x_6) < 0$ nach Entstehung aus dem Mittelwertsatz. Wäre $u^{\prime\prime}(x_6) < -\delta_2$ so gäbe es nach dem Zwischenwertsatz noch ein $x_7 < x_6 < x_5$ mit $u^{\prime\prime}(x_7) = -\delta_2$, das kann nach Definition von $x_5$ nicht sein. Also gilt
		\begin{align*}
		0 &\leq u^{\prime\prime}(x_5) + g(x_5) u^\prime (x_5) = u^{\prime\prime}(x_5) + g(x_5) u^{\prime\prime}(x_6)(x_5 - x_1) \\
		&\leq u^{\prime\prime}(x_5) + M |u^{\prime\prime}(x_6)| \frac{1}{2M} \leq -\delta_2 + \frac{\delta_2}{2} = -\frac{\delta_2}{2} < 0
		\end{align*}
		Ein Widerspruch!\\
    ---------------------------------neue Version--------------------------------------------------------\\
    Als nächstes wählen wir $0 < \delta_0 < b - x_1$.
    \begin{align*}
    M := \sup\Bbraces{|g(y)|: y \in [x_1,x_1 + \delta_0]}.
    \end{align*}
    Nun wähle $\delta_1 < \frac{1}{2M}$.
    Nach Definition von $x_1$ finden wir ein $x_2 \in (x_1,x_1 + \delta_1)$ mit $u(x_2) < u(x_1)$ und danach mit dem Mittelwertsatz ein $x_3 \in (x_1, x_2]$ mit $u^\prime(x_3) < 0$.
    Da $x_1$ Maximum von $u$ ist gilt $u^\prime(x_1) = 0$ und $u^{\prime\prime}(x_1) \leq 0$ daher
    finden wir wieder mit dem Mittelwertsatz ein $x_4 \in (x_1, x_3]$ und ein $n \in \N$ mit $u^{\prime\prime}(x_4) < -\frac{1}{n}$.
    \begin{align*}
      x_5 := \min \left\{x \in (x_1,x_1+\delta_1): u^{\primeprime}(x) = -\frac{1}{n}\right\}
    \end{align*}
    Wieder mit dem Mittelwertsatz existiert ein $x_6 \in (x_1,x_5)$
    \begin{align*}
    u^\prime(x_6) = u^\prime(x_1) + u^{\prime\prime}(x_6) (x_5 - x_1) = u^{\prime\prime}(x_6) (x_5 - x_1)
    \end{align*}
    Also gilt
    \begin{align*}
    0 &\leq u^{\prime\prime}(x_5) + g(x_5) u^\prime (x_5)
    = u^{\prime\prime}(x_5) + g(x_5) u^{\prime\prime}(x_6)(x_5 - x_1) \\
    &\leq -\frac{1}{n} + M\frac{1}{n}\frac{1}{2M} = -\frac{1}{n}\left(1 - \frac{1}{2}\right) < 0.
    \end{align*}
    Ein Widerspruch!
	\end{enumerate}


\end{solution}

% --------------------------------------------------------------------------------
