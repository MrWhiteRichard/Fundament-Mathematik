% --------------------------------------------------------------------------------

\begin{exercise}
\textit{(Smoothing properties of the heat operator)} \\
Let $\Omega=(0,2 \pi)$ and consider the initial value problem (IVP)
\begin{align*}
  \left\{\begin{array}{ll}
  u_{t}=u_{x x} & \text { for } (x, t) \in \Omega \times(0, \infty), \\
  u(x, t=0)=u_{0} & \text { for } x \in \Omega, \\
  u(x=0, t)=u(x=2 \pi, t) & \text { for } t>0,
  \end{array}\right.
\end{align*}
with $u_{0} \in L^{2}(\Omega) .$ The set $\left\{\phi_{n}(x)=e^{\mathrm{in} x}: n \in \mathbb{Z}\right\}$ is a complete orthogonal system in the complex Hilbert space $L^{2}(\Omega),$ corresponding to the eigenfunctions of the operator $L u=-u_{x x}$ with periodic boundary conditions. We now consider the operator
\begin{align*}
  e^{-L t}: L^{2}(\Omega) \mapsto L^{2}(\Omega), \quad
  v \mapsto \sum_{n \in \mathbb{Z}} e^{-n^{2} t}\left\langle v,
  \frac{\phi_{n}}{\left\|\phi_{n}\right\|}\right\rangle_{L^{2}}
  \frac{\phi_{n}}{\left\|\phi_{n}\right\|}.
\end{align*}
In what sense is $u(\cdot, t)=e^{-L t} u_{0}$ a solution of IVP? Furthermore, show that:
\begin{enumerate}[label = (\roman*)]
  \item For $t \geq 0$ it holds that $\left\|e^{-L t}\right\|_{L^{2} \rightarrow L^{2}}=1$.
  \item For $t>0$ it holds that
  $\left\|e^{-L t}\right\|_{L^{2} \rightarrow H_{p e r}^{1}} \leq C\left(1+t^{-1 / 2}\right)$
  for some constant $C>0$.
  \item For $t>0$ it holds that $\left\|e^{-L t}\right\|_{L^{2} \rightarrow H_{\text {per}}^{k}} \leq C_{k}\left(1+t^{-a_{k}}\right)$ for all $k \in \mathbb{N}$ and $a_{k}, C_{k}>0$.
\end{enumerate}
Notice that here $\left\|e^{-L t}\right\|_{X \rightarrow Y}$ denotes the operator-norm of $e^{-L t}$ as an operator from $X$ to $Y$.
\end{exercise}

% --------------------------------------------------------------------------------

\begin{solution}

\phantom{}

\end{solution}

% --------------------------------------------------------------------------------
