% --------------------------------------------------------------------------------

\begin{exercise}

\textit{(Galerkin method for the Poisson equation)} \\
Let $\Omega \subset \mathbb{R}^{n}$ be a bounded, open set with smooth boundary.
For $f \in L^{2}(\Omega)$ construct a solution of the Poisson equation
\begin{align}\label{poisson}
  \begin{array}{llll}-\Delta u=f & \text { in } \Omega, & u=0 & \text { on } \partial \Omega\end{array}
\end{align}
using a Galerkin method. To do this, let $\left\{\phi_{k}\right\}$ for $k \in \mathbb{N}$ denote the eigenfunctions of the Laplacian with homogeneous Dirichlet boundary data on $\Omega .$ Then prove that for any $m \in \mathbb{N}$ there exists
\begin{align*}
  u_{m}=\sum_{k=1}^{m} \mathbf{d}_{m}^{k} \phi_{k}, \quad \text { for } \mathbf{d}_{m}^{k} \in \mathbb{R}
\end{align*}

that satisfies
\begin{align*}
  \int_{\Omega} \nabla u_{m} \cdot \nabla \phi_{k} \mathrm{~d}
  x=\int_{\Omega} f \phi_{k}, \quad \text { for } k=1, \ldots, m.
\end{align*}

To finish, show that the sequence $\left\{u_{m}\right\}_{m \in \mathbb{N}}$
converges weakly in $H_{0}^{1}(\Omega)$ to a weak solution of \eqref{poisson}.

\end{exercise}

% --------------------------------------------------------------------------------

\begin{solution}

	Wir wissen bereits, dass \eqref{poisson} eine eindeutige schwache Lösung besitzt und können deshalb den Operator 
	\begin{align*}
	K: L^2(\Omega) \to L^2(\Omega): f \mapsto u
	\end{align*}
	mit jener eindeutigen Lösung $u$ betrachten. Wir wissen bereits aus der Vorlesung, dass $K$ ein injektiver, kompakter, selbstadjungierter Operator. Wegen der Injektivität von $K$ können wir den inversen Operator 
	\begin{align*}
	L: K\pbraces{L^2(\Omega)} \to L^2(\Omega)
	\end{align*}
	betrachten.Wir interpretieren den Ausdruck \textit{smooth boundary} aus der Angabe als $\partial \Omega \in C^2$, womit dann mit Lemma 6.14 die Gleichheit $D(L) = H^2(\Omega) \cap H_0^1(\Omega)$ gilt. Daraus schließen wir sofort $-\Delta = L$. Den Ausdruck \textit{the eigenfunctions} aus der Angabe interpretieren wir so, dass es sich um eine Orthonormalbasis handelt. Alle diese Eigenfunktionen sind dann, wie wir aus der Vorlesung wissen, auch schon Eigenfunktionen von $K$ und es gilt \begin{align*}
	K(f) = \sum_{k \in \N} \mu_k \left\langle f, \phi_k \right\rangle \phi_k
	\end{align*}
	wobei $\mu_k$ jeweils Eigenwert bezüglich $K$ von $\varphi_k$ sein soll. Wir können die $\mu_k$ so ordnen, dass sie eine monotone Nullfolge bilden. 
\end{solution}

% --------------------------------------------------------------------------------
