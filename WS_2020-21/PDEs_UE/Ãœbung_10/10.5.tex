% -------------------------------------------------------------------------------- %

\begin{exercise}

\textit{(Exponential decay to equilibrium for the Fokker-Planck equation)} \\
We consider smooth solutions of the Fokker-Planck equation
\begin{align}\label{fokker-planck}
  \partial_{t} u=\nabla \cdot(\nabla u+u \nabla V), \quad t>0,
  \quad u(0)=u_{0} \quad \text { in } \mathbb{R}^{d},
\end{align}
where we assume that $V$ satisfies the Bakry-Emery condition
$\nabla^{2} V \geq \lambda \, \mathrm{Id}$ for some $\lambda>0$. Furthermore, assume that $\phi \in C^{4}((0, \infty))$ is convex with $\phi(1)=0,$ and $1 / \phi^{\prime \prime}$ is welldefined and concave. Examples of admissible functions $\phi$ are $\phi(s)=s(\log (s)-1)+1$ and $\phi(s)=s^{\alpha}-1$ for $1<\alpha \leq 2$.
\begin{enumerate}[label = (\roman*)]
  \item Compute $u_{\infty},$ a stationary solution of \eqref{fokker-planck} that is strictly positive
  $\left(u_{\infty}>0\right)$ and has unit mass $\left(\int_{\R^d} u_{\infty}=1\right)$.
  \item Define the relative entropy with respect to $\phi$ as
  \begin{align*}
    H_{\phi}[u]=\int_{\mathbb{R}^{d}} \phi\left(\frac{u}{u_{\infty}}\right) u_{\infty} \,\mathrm{d} x.
  \end{align*}
  Notice that, setting $\rho=\frac{u}{u_{\infty}},$ we have that
  $\partial_{t} u=\nabla \cdot\left(u_{\infty} \nabla \rho\right) .$
  Using this, show that the \textit{entropy production}, $-\frac{d}{d t} H_{\phi}[u],$ is non-negative.
  \item Show that
  \begin{align*}
    \nabla \partial_{t} \rho &= \nabla \Delta \rho-\nabla^{2} \rho \nabla V-\nabla^{2} V \nabla \rho, \\
    \nabla \rho \cdot \nabla \Delta \rho &= \nabla \cdot\left(\nabla^{2} \rho \nabla \rho\right)
    -\left|\nabla^{2} \rho\right|^{2},
  \end{align*}
  where $\left|\nabla^{2} \rho\right|^{2}=
  \sum_{i, j=1, \ldots, n}\left|\partial_{i} \partial_{j} \rho\right|^{2}$.
  Using these identities and the expression for $\frac{d}{d t} H_{\phi}[u]$
  that you have obtained in (i), show that
  \begin{align*}
    \frac{d^{2}}{d t^{2}} H_{\phi}[u] &\geq
    \int_{\mathbb{R}^{d}}\left(\phi^{\primeprime \primeprime}(\rho)|\nabla \rho|^{4}+4
    \phi^{\primeprimeprime}(\rho) \nabla \rho^{T} \nabla^{2} \rho \nabla \rho+2
    \phi^{\prime \prime}(\rho)\left|\nabla^{2} \rho\right|^{2}\right) u_{\infty} \mathrm{d} x
    -2 \lambda \frac{d}{d t} H_{\phi}[u] \\
    &= 2 \int_{\mathbb{R}^{d}} \phi^{\prime \prime}(\rho)\left|\nabla^{2}
    \rho+\frac{\phi^{\prime \prime \prime}(\rho)}{\phi^{\prime \prime}(\rho)}
    \nabla \rho \otimes \nabla \rho\right|^{2} u_{\infty} \mathrm{d} x
    +\int_{\mathbb{R}^{d}}\left(\phi^{\primeprime \primeprime}(\rho)-2
    \frac{\phi^{\primeprimeprime}(\rho)^{2}}{\phi^{\prime \prime}(\rho)}\right)|\nabla \rho|^{4}
    u_{\infty} \mathrm{d} x
    -2 \lambda \frac{d}{d t} H_{\phi}[u].
  \end{align*}
  \item Using the concavity of $1 / \phi^{\prime \prime}$ and convexity of $\phi,$
  argue that the result of (iii) yields that
  \begin{align}\label{eq}
    \frac{d^{2}}{d t^{2}} H_{\phi}[u] \geq-2 \lambda \frac{d}{d t} H_{\phi}[u], \quad t>0
  \end{align}
  \item Argue that integrating \eqref{eq} on the interval $(s, \infty)$ yields that
  \begin{align*}
    \frac{d}{d t} H_{\phi}[u(s)] \leq-2 \lambda H_{\phi}[u(s)], \quad s \geq 0
  \end{align*}
  \textit{Hint:} For this use Gronwall's lemma applied to \eqref{eq} and you may use
  (without proof) that \\
   $\lim _{t \rightarrow \infty} H_{\phi}[u(t)]=0$.
  \item Using (without proof) that
  \begin{align*}
    \left\|u(t)-u_{\infty}\right\|_{L^{1}\left(R^{d}\right)}
    \leq \frac{2}{\phi^{\prime \prime}(1)} H_{\phi}[u(t)],
  \end{align*}

  which follows from the Csiszár-Kullback-Pinsker inequality, show that
  \begin{align*}
    \left\|u(t)-u_{\infty}\right\|_{L^{1}\left(\mathrm{R}^{d}\right)}^{2}
    \leq \frac{1}{\phi^{\prime \prime}(1)} H_{\phi}\left[u_{0}\right] e^{-2 \lambda t}.
  \end{align*}
\end{enumerate}
\end{exercise}

% -------------------------------------------------------------------------------- %

\begin{solution}

\phantom{}
\begin{enumerate}[label = (\roman*)]
  \item Schreiben wir die Differentialgleichung zuerst schöner darauf

  \begin{align*}
    \partial_t u
    =
    \Delta u + \nabla u \cdot \nabla V + u \Delta V
  \end{align*}

  Die stationäre Gleichung wird sicher schon erfüllt, falls $\nabla u_\infty + u_\infty \nabla V = 0$. Aus dem eindimensionalen Fall motiviert, da $u^\prime + V^\prime u = 0$ von $u = ce^{-V}$ gelöst wird, definieren wir

  \begin{align*}
    u_\infty(x)
    &:=
    c \exp(-V(x)) \\
    \frac{\partial}{\partial x_i} u_\infty
    &=
    c \frac{\partial}{\partial x_i} (\exp \circ -V)
    =
    - c \exp(-V) \frac{\partial}{\partial x_i} V
    \quad
    \implies
    \nabla u_\infty
    =
    -u_\infty \nabla V
  \end{align*}

  Nun wollen wir $c := \frac{1}{\Int[\R^d]{ \exp(-V(s))}{x}}$, so bleibt noch zu zeigen, dass

  \begin{align*}
    \Int[\R^d] {e^{-V(s)}}{x} < \infty
  \end{align*}

  Das gilt, da man aus $\nabla^2V \geq \lambda \id$ mit dem mehrdimensionalen Taylor folgern kann, dass $V \geq C_1 + C_2|x| + \lambda |x|^2$ mit $C_1, C_2 \in \R$.
  \item
  \begin{align*}
    \nabla \cdot (u_{\infty} \nabla \rho) = \nabla \cdot (u_\infty \frac{\nabla u u_{\infty} -
    u \nabla u_{\infty}}{u_{\infty}^2}) = \nabla \cdot (\nabla u - \frac{u \nabla u_{\infty}}{u_{\infty}})
    = \nabla \cdot (\nabla u + \frac{u u_{\infty} \nabla V}{u_{\infty}})
    = \nabla \cdot (\nabla u + u \nabla V) = \partial_t u
  \end{align*}
  Unter der Annahme, dass $H_\phi: C^2_1((0,\infty) \times \R^d) \rightarrow \R$ rechnen wir ganz rigoros

  \begin{align*}
    -\frac{d}{dt} H_\phi[u]
    &=
    -\Int[\R^d]{ \frac{d}{dt}(\phi(\rho)u_\infty)}{x}
    =
    -\Int[\R^d]{u_\infty \phi^\prime(\rho)\frac{1}{u_\infty} \underbrace{\Div(u_\infty \nabla \rho)}_{\partial_t u}}{x} \\
    &=
    -\lim_{r \rightarrow \infty} \Int[B_r(0)]{\phi^\prime(\rho) \Div(u_\infty \nabla \rho)}{x}
    \stackrel{Gauss}{=}
    \lim_{r \rightarrow \infty} \Int[B_r(0)]{\nabla(\phi^\prime(\rho)) u_\infty \nabla \rho}{x}
    -
    \Int[\partial B_r(0)]{\phi^\prime u_\infty (\nabla \rho \cdot \nu)}{s} \\
    &=
    \lim_{r \rightarrow \infty} \Int[B_r(0)]{\phi^\primeprime(\rho) u_\infty \nabla \rho \cdot \nabla \rho}{x}
    =
    \Int[\R^d]{\phi^\primeprime(\rho) |\nabla \rho|^2 u_\infty}{x}
  \end{align*}
  Das Randintegral verschwindet, da wir aus $u_\infty > 0$ und $\Int[\R^d]{u_\infty}{x} = 1$ schließen, dass für $\epsilon > 0$ ein $r > 0$ existiert, sodass $u_\infty(|x|) < \epsilon$ für $|x| \geq r$. Für die Vertauschung von Integral und Ableitung müsste man noch eine geeignete Majorante finden.

  \item Die ersten beiden Identitäten rechnen wir nach, wobei wir bei der ersten $\nabla u_\infty = - u_\infty \nabla V$ verwenden

  \begin{align}\label{first}
    \nabla \partial_t \rho
    =
    \nabla \frac{\partial_t u}{u_\infty}
    =
    \nabla \frac{\Div(u_\infty \nabla \rho)}{u_\infty}
    =
    \nabla(\frac{u_\infty \Delta \rho + \nabla u_\infty \cdot \nabla \rho}{u_\infty})
    =
    \nabla \Delta \rho - \nabla (\nabla V \cdot \nabla \rho)
    =
    \nabla \Delta \rho - \nabla^2 \rho \nabla V - \nabla^2 V \nabla \rho
    \end{align}

    \begin{align}\label{second}
    \nabla \cdot (\nabla^2 \rho \nabla \rho)
    =
    \Div(\nabla^2 \rho \nabla \rho)
    =
    \sum_{i=1}^d \partial_i \sum_{k=1}^d \partial_k \rho \partial_i \partial_k \rho
    =
    \sum_{i=1}^d \sum_{k=1}^d \partial_k \rho \partial_k \partial_i^2  \rho + (\partial_k \partial_i \rho)^2
    =
    \nabla \rho \cdot \nabla \Delta \rho + |\nabla^2 \rho|^2
  \end{align}

  Nun zu der Abschätzung, wobei wir erstmal wieder Integral und Ableitung einfach Vertauschen und die Darstellung aus (ii) verwenden. Die Abschätzung

  \begin{align*}
    \nabla \rho \cdot \nabla^2 V \nabla \rho
    \geq
    \lambda |\nabla \rho|^2
  \end{align*}

  erhalten wir aus der Bakry-Emery condition von V. Wir wollen

  \begin{align*}
    \frac{d^2}{d t^2} H_{\phi}[u] &\geq
    \int_{\R^d}(
    \underbrace{
    \phi^{\primeprime \primeprime}(\rho)|\nabla \rho|^4
    }_{1}
    +
    \underbrace{
    4\phi^\primeprimeprime(\rho) \nabla \rho^{T} \nabla^2 \rho \nabla \rho
    }_{2}
    +
    \underbrace{
    2\phi^{\prime \prime}(\rho)|\nabla^{2} \rho|^2
    }_{3}
    ) u_{\infty} \mathrm{d} x
    -\underbrace{
    2 \lambda \frac{d}{d t} H_{\phi}[u]
    }_{4}
    \end{align*}

  \begin{align*}
    \frac{d^2}{dt^2}H_\phi[u]
    &=
    -\Int[\R^d]{\frac{d}{dt}(\phi^\primeprime(\rho) |\nabla \rho|^2) u_\infty}{x}
    =
    -\Int[\R^d]{(\phi^\primeprimeprime \partial_t\rho |\nabla \rho|^2 + 2\phi^\primeprime \nabla \rho \cdot \nabla \partial_t \rho) u_\infty}{x} \\
    &\stackrel{\ref{first}}{=}
    -\Int[\R^d]{
    (\phi^\primeprimeprime \partial_t\rho |\nabla \rho|^2 +2 \phi^\primeprime \nabla \rho \cdot (\nabla \Delta \rho - \nabla^2 \rho \nabla V - \nabla^2V \nabla \rho)u_\infty
    }{x} \\
    &\stackrel{\ref{second}}{=}
    -\Int[\R^d]{
    \Big(\phi^\primeprimeprime
    \partial_t\rho
    |\nabla \rho|^2
    +
    2 \phi^\primeprime(
    \Div(\nabla^2 \rho \nabla \rho) - |\nabla^2 \rho|^2
    -
    \nabla \rho \cdot \nabla^2 \rho \nabla V
    -
    \nabla \rho \cdot\nabla^2V \nabla \rho)\Big)u_\infty
    }{x}\\
    &\geq
    -\Int[\R^d]{
    \Big(\phi^\primeprimeprime
    \partial_t\rho
    |\nabla \rho|^2
    +
    2 \phi^\primeprime(
    \Div(\nabla^2 \rho \nabla \rho) - \underbrace{|\nabla^2 \rho|^2}_{3}
    -
    \nabla \rho \cdot \nabla^2 \rho \nabla V)\Big)u_\infty}{x}
    \underbrace{-2 \lambda \frac{d}{dt} H_\phi[u]}_{4}
    \rightsquigarrow
  \end{align*}

  An dieser Stelle lassen wir die Terme, die schon direkt mit jenen aus der Angabe übereinstimmen der Einfachheithalber weg. Eine kleine Nebenrechnung erlauben wir uns dabei noch

  \begin{align}
    \partial_i |\nabla \rho|^2
    =
    \partial_i \sum_{j=1}^d (\partial_j \rho)^2
    =
    \sum_{j=1}^d 2 \partial_j \rho \partial_i \partial_j \rho
    \\ \label{third}
    \nabla |\nabla \rho|^2 \cdot \nabla \rho
    =
    2\sum_{i,j=1}^d \partial_j \rho \partial_i \partial_j \rho \partial_i \rho
    =
    2 \nabla \rho \cdot \nabla^2\rho \nabla \rho
  \end{align}

  \begin{align*}
    &\rightsquigarrow
    -\Int[\R^d]{
    \Big(\phi^\primeprimeprime(\rho)
    \partial_t\rho
    |\nabla \rho|^2
    +
    2 \phi^\primeprime(
    \Div(\nabla^2 \rho \nabla \rho)
    -
    \nabla \rho \cdot \nabla^2 \rho \nabla V)\Big)u_\infty}{x} \\
    &=
    - \lim_{r \rightarrow \infty}
    \Int[B_r(0)]{
    \phi^\primeprimeprime(\rho) \Div(u_\infty \nabla \rho) |\nabla \rho|^2
    }{x}
    -
    \Int[\R^d]{2 \phi^\primeprime(
    \Div(\nabla^2 \rho \nabla \rho)
    -
    \nabla \rho \cdot \nabla^2 \rho \nabla V)u_\infty}{x} \\
    &\stackrel{Gauss}{=}
    \lim_{r \rightarrow \infty}
    \Int[B_r(0)]{
    \nabla (\phi^\primeprimeprime(\rho)|\nabla \rho|^2) u_\infty \nabla \rho
    }{x}
    -
    \Int[\R^d]{2 \phi^\primeprime(
    \Div(\nabla^2 \rho \nabla \rho)
    -
    \nabla \rho \cdot \nabla^2 \rho \nabla V)u_\infty}{x} \\
    &=
    \Int[\R^d]{
    (\phi^{(4)}(\rho)\nabla \rho |\nabla \rho|^2+ \phi^{(3)}\nabla |\nabla \rho|^2) u_\infty \nabla \rho
    }{x}
    -
    \Int[\R^d]{2 \phi^\primeprime(
    \Div(\nabla^2 \rho \nabla \rho)
    -
    \nabla \rho \cdot \nabla^2 \rho \nabla V)u_\infty}{x} \\
    &=
    \Int[\R^d]{
    (\underbrace{\phi^{(4)}(\rho)|\nabla \rho|^4}_{1} + \phi^{(3)}\nabla |\nabla \rho|^2 \cdot \nabla \rho) u_\infty
    }{x}
    -
    \Int[\R^d]{2 \phi^\primeprime(
    \Div(\nabla^2 \rho \nabla \rho)
    -
    \nabla \rho \cdot \nabla^2 \rho \nabla V)u_\infty}{x} \\
    &\stackrel{\ref{third}}{\rightsquigarrow}
    \Int[\R^d]{
    \phi^{(3)}2\nabla \rho \cdot \nabla^2 \rho \nabla \rho~ u_\infty
    }{x}
    -
    \Int[\R^d]{2 \phi^\primeprime(
    \Div(\nabla^2 \rho \nabla \rho)
    -
    \nabla \rho \cdot \nabla^2 \rho \nabla V)u_\infty}{x} \\
    &=
    \Int[\R^d]{
    2\phi^{(3)}\nabla \rho \cdot \nabla^2 \rho \nabla \rho~ u_\infty
    }{x}
    -
    \lim_{r \rightarrow \infty}
    \Int[B_r(0)]{2 \phi^\primeprime
    \Div(\nabla^2 \rho \nabla \rho) u_\infty}{x}
    +
    \Int[\R^d]{2\phi^\primeprime \nabla \rho \cdot \nabla^2 \rho \nabla V~u_\infty}{x} \\
    &\stackrel{Gauss}{=}
    \Int[\R^d]{
    2\phi^{(3)}\nabla \rho \cdot \nabla^2 \rho \nabla \rho~ u_\infty
    }{x}
    +
    \lim_{r \rightarrow \infty}
    \Int[B_r(0)]{2 \nabla (\phi^\primeprime u_\infty) \cdot
    \nabla^2 \rho \nabla \rho }{x}
    +
    \Int[\R^d]{2\phi^\primeprime \nabla \rho \cdot \nabla^2 \rho \nabla V~u_\infty}{x} \\
    &=
    \Int[\R^d]{
    2\phi^{(3)}\nabla \rho \cdot \nabla^2 \rho \nabla \rho~ u_\infty
    }{x}
    +
    \Int[\R^d]{2 (\phi^{(3)} \nabla \rho  u_\infty + \phi^\primeprime \nabla u_\infty)
    \nabla^2 \rho \nabla \rho}{x}
    +
    \Int[\R^d]{2\phi^\primeprime \nabla \rho \cdot \nabla^2 \rho \nabla V~u_\infty}{x}\\
    &=
    \Int[\R^d]{
    2\phi^{(3)}\nabla \rho \cdot \nabla^2 \rho \nabla \rho~ u_\infty
    }{x}
    +
    \Int[\R^d]{2 \phi^{(3)} \nabla \rho  \cdot \nabla^2 \rho \nabla \rho u_\infty}{x}
    -
    \Int[\R^d]{\phi^\primeprime u_\infty \nabla V \cdot \nabla^2 \rho \nabla \rho}{x}\\
    &+
    \Int[\R^d]{2\phi^\primeprime \nabla \rho \cdot \nabla^2 \rho \nabla V~u_\infty}{x}
    =
    \Int[\R^d]{\underbrace{4 \phi^\primeprimeprime(\rho) \nabla \rho^T \nabla^2 \rho \nabla \rho}_{2}}{x}
  \end{align*}

  Damit haben wir schließlich alle Teile gesammelt.

  Für die letzte Gleichheit zeigen wir:
  \begin{align*}
  &4\phi^{\primeprimeprime}(\rho) \nabla \rho^{T} \nabla^{2} \rho \nabla \rho+2
  \phi^{\prime \prime}(\rho)\left|\nabla^{2} \rho\right|^{2} \\
  &= 4\phi^{\primeprimeprime}(\rho)\sum_{j,k=1}^n(\partial_j\partial_k\rho)(\partial_j\rho\partial_k\rho) +
  2\phi^{\primeprime}\sum_{j,k=1}^n |\partial_j\partial_k\rho|^2 \\
  &= 2\phi^{\primeprime}\sum_{j,k=1}^n 2\frac{\phi^{\primeprimeprime}(\rho)}{\phi^{\primeprime}(\rho)}
  (\partial_j\partial_k\rho)(\partial_j\rho\partial_k\rho)+(\partial_j\partial_k\rho)^2 \\
  &= 2\phi^{\primeprime}\sum_{j,k=1}^n \left(\frac{\phi^{\primeprimeprime}(\rho)}{\phi^{\primeprime}(\rho)}
  (\partial_j\rho\partial_k\rho)+(\partial_j\partial_k\rho)\right)^2 - \frac{\phi^{\primeprimeprime}(\rho)^2}{\phi^{\primeprime}(\rho)^2}
  (\partial_j\rho\partial_k\rho)^2\\
  &= 2\phi^{\primeprime}\left|\nabla^2\rho +
  \frac{\phi^{\primeprimeprime}(\rho)}{\phi^{\primeprime}(\rho)} \nabla\rho \otimes
  \nabla \rho \right|^2 -
   2\frac{\phi^{\primeprimeprime}(\rho)^2}{\phi^{\primeprime}(\rho)}
  \sum_{j,k=1}^n(\partial_j\rho\partial_k\rho)^2\\
  &= 2\phi^{\primeprime}\left|\nabla^2\rho +
  \frac{\phi^{\primeprimeprime}(\rho)}{\phi^{\primeprime}(\rho)} \nabla\rho \otimes
  \nabla \rho \right|^2 -
   2\frac{\phi^{\primeprimeprime}(\rho)^2}{\phi^{\primeprime}(\rho)}
  |\nabla\rho|^4\\
  \end{align*}
  \item $\phi$ ist konvex, daher ist $\phi^{\primeprime}(\rho) \geq 0$ und
  somit das erste Integral positiv. Wir rechnen
  \begin{align*}
  \left(\frac{1}{\phi^{\primeprime}(\rho)}\right)^{\primeprime} &=
  \left(\frac{-\phi^{\primeprimeprime}(\rho)}{\phi^{\primeprime}(\rho)^2}\right)^{\prime} =
  \frac{-\phi^{\primeprime\primeprime}(\rho)(\phi^{\primeprime}(\rho))^2 +
  2(\phi^{\primeprimeprime}(\rho))^2\phi^{\primeprime}(\rho)}{\phi^{\primeprime}(\rho)^4} \\
  &= -\frac{1}{(\phi^{\primeprime}(\rho))^2}
  \left(\phi^{\primeprime\primeprime}(\rho) -
  2\frac{\phi^{\primeprimeprime}(\rho)^2}{\phi^{\primeprime}(\rho)}\right)
  \end{align*}
  Aus der Konkavität von $\frac{1}{\phi^{\primeprime}}$ folgt
  \begin{align*}
  \left(\phi^{\primeprime\primeprime}(\rho) -
  2\frac{\phi^{\primeprimeprime}(\rho)^2}{\phi^{\primeprime}(\rho)}\right) =
  -\underbrace{\left(\frac{1}{\phi^{\primeprime}(\rho)}\right)^{\primeprime}}
  _{\geq 0}\underbrace{\phi^{\primeprime}(\rho)^2}_{\geq 0} \geq 0
  \end{align*}
  und damit die Positivität des zweiten Integrals.
  \item Wir integrieren \eqref{eq} über $(s,\infty)$:
  \begin{align*}
    \int_s^{\infty}  \frac{d^{2}}{d t^{2}} H_{\phi}[u] dt &\geq
    \int_s^{\infty} -2 \lambda \frac{d}{d t} H_{\phi}[u] ds \\
    \iff \lim_{t \to \infty} \frac{d}{d t} H_{\phi}[u(t)] -\frac{d}{d t} H_{\phi}[u(s)]
    &\geq -2\lambda(\underbrace{\lim_{t \to \infty} H_{\phi}[u(t)]}_{=0} - H_{\phi}[u(s)]) \\
    \iff \lim_{t \to \infty} \frac{d}{d t} H_{\phi}[u(t)] -\frac{d}{d t} H_{\phi}[u(s)]
    &\geq 2\lambda H_{\phi}[u(s)] \\
  \end{align*}
  Mit dem Gronwall-Lemma angewandt auf \eqref{eq}, sowie (ii) erhalten wir
  \begin{align*}
    0 < -\frac{d}{d t} H_{\phi}[u(t)] \leq -\frac{d}{d t} H_{\phi}[u(s)]\exp(-2\lambda (t-s))
    \xrightarrow{t \to \infty} 0
  \end{align*}
  und somit
  \begin{align*}
  -\frac{d}{d t} H_{\phi}[u(s)]
  &\geq 2\lambda H_{\phi}[u(s)] \iff \frac{d}{d t} H_{\phi}[u(s)] \leq -2\lambda H_{\phi}[u(s)].
  \end{align*}
  \item Wir verwenden die Ungleichung
  \begin{align*}
  \left\|u(t)-u_{\infty}\right\|_{L^{1}\left(R^{d}\right)}^2
  \leq \frac{2}{\phi^{\prime \prime}(1)} H_{\phi}[u(t)]
  \end{align*}
  und erhalten mittels Gronwall
  \begin{align*}
    H_{\phi}[u(t)] \leq H_{\phi}[u(0)]\exp(-2\lambda t)
  \end{align*}
  und damit
  \begin{align*}
  \left\|u(t)-u_{\infty}\right\|_{L^{1}\left(R^{d}\right)}^2
  \leq \frac{2}{\phi^{\prime \prime}(1)} H_{\phi}[u_0]\exp(-2\lambda t).
  \end{align*}
\end{enumerate}

\end{solution}

% -------------------------------------------------------------------------------- %
