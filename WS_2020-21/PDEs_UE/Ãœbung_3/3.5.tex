% --------------------------------------------------------------------------------

\begin{exercise}

Eine Distribution $u \in \mathcal{D}^\prime(\Omega)$ heißt von endlicher Ordnung wenn es ein $m \in \N_0$ gibt sodass man fur alle kompakten Mengen $K \subset \Omega$ eine Konstante $C > 0$ finden kann, sodass für alle $\varphi \in \mathcal{D}(K)$ gilt:

\begin{align*}
    |\abraces{u, \varphi}|
    \leq
    C \norm[C^m(K)]{\varphi}.
\end{align*}

In diesem Fall heißt $m$ die Ordnung von $u$.
Gibt es kein solches $m$, sagt man, dass $u$ unendliche Ordnung hat.

Bestimmen Sie die Ordnung folgender Distributionen ($\Omega \subset \R^n$ offen):

\begin{enumerate}[label = (\roman*)]
    \item $\delta \in \mathcal{D}^\prime(\R)$,
    \item $f \in L^1_{\text{loc}}(\Omega)$,
    \item $\varphi \mapsto \partial^\alpha \varphi(x_0)$ für $\alpha \in \N_0^n$, $x_0 \in \Omega$,
    \item $\varphi \mapsto \sum_{j \in \N} \partial^{\alpha_j} \varphi(x_j)$ für eine Folge $(x_j)_j$ in $\Omega$ ohne Häufungspunkt und $\alpha_j \in \N_0^n$.
\end{enumerate}

\end{exercise}

% --------------------------------------------------------------------------------

\begin{solution}

ToDo!

\end{solution}

% --------------------------------------------------------------------------------
