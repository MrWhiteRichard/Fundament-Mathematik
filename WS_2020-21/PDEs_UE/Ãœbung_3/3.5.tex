% --------------------------------------------------------------------------------

\begin{exercise}

Eine Distribution $u \in \mathcal{D}^\prime(\Omega)$ heißt von endlicher Ordnung wenn es ein $m \in \N_0$ gibt sodass man für alle kompakten Mengen $K \subset \Omega$ eine Konstante $C > 0$ finden kann, sodass für alle $\varphi \in \mathcal{D}(K)$ gilt:

\begin{align*}
    |\abraces{u, \varphi}|
    \leq
    C \norm[C^m(K)]{\varphi}.
\end{align*}

In diesem Fall heißt $m$ die Ordnung von $u$.
Gibt es kein solches $m$, sagt man, dass $u$ unendliche Ordnung hat.

Bestimmen Sie die Ordnung folgender Distributionen ($\Omega \subset \R^n$ offen):

\begin{enumerate}[label = (\roman*)]
    \item $\delta \in \mathcal{D}^\prime(\R)$,
    \item $f \in L^1_{\text{loc}}(\Omega)$,
    \item $\varphi \mapsto \partial^\alpha \varphi(x_0)$ für $\alpha \in \N_0^n$, $x_0 \in \Omega$,
    \item $\varphi \mapsto \sum_{j \in \N} \partial^{\alpha_j} \varphi(x_j)$ für eine Folge $(x_j)_j$ in $\Omega$ ohne Häufungspunkt und $\alpha_j \in \N_0^n$.
\end{enumerate}

\end{exercise}

% --------------------------------------------------------------------------------

\begin{solution}

\begin{enumerate}[label = (\roman*)]
  \item Wir behaupten, die Ordnung ist $0$. Sei dazu $K \subset \Omega$ eine beliebige kompakte Menge, $C=1$, $\varphi \in \mathcal{D}(K)$:

  \begin{align*}
    |\abraces{\delta, \varphi}|
    = |\varphi(0)|
    \leq
    \norm[C(K)]{\varphi}
  \end{align*}

  Die Ungleichung ist für $0 \in K$ klar, ansonsten ist $0 \notin \supp(\varphi)$ und die Ungleichung trivialerweise erfüllt.


  \item Wir behaupten, die Ordnung ist $0$. Sei dazu wieder $K \subset \Omega$ eine beliebige kompakte Menge, $\varphi \in \mathcal{D}(K)$:

  \begin{align*}
      |\abraces{f, \varphi}|
      =
      |\int_K f(x) \varphi(x) dx|
      \leq
      \int_K |f(x)| |\varphi(x)| dx
      \leq
      \norm[C(K)]{\varphi} \norm[L^1 (K)]{f|_K}
  \end{align*}

  Mit $C = \norm[L^1 (K)]{f|_K}$ als Konstante (in $\varphi$) haben wir die Behauptung also gezeigt.

  \item Wir behaupten, die Ordnung ist kleiner (eventuell gleich) $|\alpha|$.
  Erinnern wir uns zuerst an die Defintion der Norm.

  \begin{align*}
    \norm[C^k (K)]{\phi}
    =
    \sum_{|\alpha| \leq k} \sup_{x \in K} |D^\alpha \phi(x)|
  \end{align*}

  Sei nun $K \subset \Omega$ eine beliebige kompakte Menge, $C = 1$, $\varphi \in \mathcal{D}(K)$:

  \begin{align*}
    |\abraces{\varphi \mapsto \partial^\alpha \varphi(x_0), \varphi}|
    =
    |\partial^\alpha \varphi(x_0)|
    \leq
    \norm[C^{|\alpha|}(K)]{\varphi}
  \end{align*}

  Wie in $(i)$ gilt die Ungleichung sowohl für $x_0 \in K$ als auch $x_0 \notin K$.

  \item Hier unterscheiden wir
  \begin{enumerate}[label = \arabic*.]
    \item Fall: $\max |\alpha_j| < \infty$. Dann wählen wir $m = \max |\alpha_j|$.
    Für kompaktes $K \subset \Omega$ gilt, da die Folge $x_j$ in $\Omega$ keinen Häufungspunkt hat

    \begin{align*}
      C_K := |\Bbraces{j \in \N : x_j \in K}| < \infty
    \end{align*}

    Das $C_K$ sei nun also unsere Konstante. Dann gilt für $\varphi \in \mathcal{D}(K)$

    \begin{align*}
      |\abraces{\varphi \mapsto \sum_{j \in \N} \partial^{\alpha_j} \varphi(x_j), \varphi}|
      =
      |\sum_{x_j \in \K} \partial^{\alpha_j} \varphi(x_j)|
      \leq
      \sum_{x_j \in \K} |\partial^{\alpha_j} \varphi(x_j)|
      \leq
      C_K \norm[C^m(K)]{\varphi}
    \end{align*}

    \item Fall: Nun gibt es also $j \in \N$ sodass $|\alpha_j| = \infty$. Sei nun
    $m \in \N_0$ beliebig. Wählen wir nun eine kompakte Menge $K_m$ sodass es ein
    $j_0 \in \N: x_{j_0} \in K \land |\alpha_{j_0}| > m$ gibt. Es soll von allen Folgengliedern
    nur $x_{j_0} \in K$ (solange keine doppelt vorkommen ist das auch möglich). Dann gilt für
    $\varphi \in \mathcal{D}(K)$

    \begin{align*}
      |\sum_{j \in \N} \partial^{\alpha_j} \varphi(x_j)|
      =
      |\partial^{\alpha_{j_0}} \varphi(x_{j_0})|
    \end{align*}

    und damit, wenn wir in $(iii)$ noch Gleichheit für die Ordnung zeigen können,
    ist die Ordnung größer $m$.

  \end{enumerate}
\end{enumerate}

\end{solution}

% --------------------------------------------------------------------------------
