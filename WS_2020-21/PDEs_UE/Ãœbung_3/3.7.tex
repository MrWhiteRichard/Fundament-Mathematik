% --------------------------------------------------------------------------------

\begin{exercise}

Der \textit{Träger} eine Distribution $T \in \mathcal{D}^\prime(\Omega)$ ist das Komplement der größten offenen Menge, auf der $T$ verschwindet:

\begin{align*}
    \supp{T}
    =
    \Omega
    \setminus
    \bigcup
    \Bbraces
    {
        U \subseteq \Omega ~\text{offen}~ |~
        T ~\text{verschwindet auf}~ U
    }.
\end{align*}

Zeigen Sie:

\begin{enumerate}[label = (\roman*)]
    \item Für $f \in C(\Omega)$ ist der distributionelle Träger gleich dem üblichen Träger der Funktion $f$.
    \item Ist $T \in \mathcal{D}^\prime(\R^n)$ eine Distribution von endlicher Ordnung $m$ und $\psi \in \mathcal{D}(\R^n)$ eine Testfunktion deren Ableitung $\partial^\alpha \psi$ für $|\alpha| \leq m$ verschwindet, dann ist $\abraces{T, \psi} = 0$.
    \item Gilt $\supp{T} \cap \supp{\varphi} = 0$, dann folgt $\abraces{T, \varphi} = 0$.
    \item Gilt $fT = 0$ für $T \in \mathcal{D}^\prime(\Omega)$ und $f \in C^\infty(\Omega)$, dann folgt $\supp{T} \subseteq \Bbraces{x: f(x) = 0}$.
\end{enumerate}

\end{exercise}

% --------------------------------------------------------------------------------

\begin{solution}
\phantom{}
\begin{enumerate}[label = (\roman*)]
	\item Sei $f \in C(\Omega)$. Wir definieren die Mengen
	\begin{align*}
	A := \overline{\{x \in \Omega \mid f(x) \neq 0 \}}, \qquad B := \bigcup \left\{U \subseteq \Omega \mid \forall \phi \in \mathcal{D}(\Omega): \pbraces{\supp(\phi) \subseteq U \Rightarrow \Int[\Omega][]{f\phi}{\lambda^n} = 0} \right\}
	\end{align*}
	und wollen $A = \Omega \setminus B$ zeigen.
	\begin{enumerate}
		\item[$\subseteq$:] Sei $x \in A$ und $V \subseteq \Omega$ eine beliebige offene Umgebung von $x$. Nach Definition von $A$ gibt es ein $y \in V$ mit $f(y) \neq 0$, o.B.d.A $f(y) > 0$. Wegen der Stetigkeit von $f$ finden wir nun eine Umgebung $W$ von $y$ mit $f(W) > 0$. Nun wählen wir ein $0 \neq \phi \in \mathcal{D}(\Omega)$ mit $\phi \geq 0$ und $\supp(\phi) \subseteq W$. Nun ist aber $\Int[\Omega][]{f\phi}{\lambda^n} > 0$, also ist $x \notin B$ und damit ist $x \in \Omega \setminus B$.
		\item[$\supseteq$:] Sei $x \in A^c$, also gibt es eine Umgebung $V$ von $x$ mit $f(V) = 0$. Für alle $\phi \in \mathcal{D}(\Omega)$ mit $\supp(\phi) \subseteq U$ gilt also $\Int[\Omega][]{f\phi}{\lambda^n} = 0$ also $V \subseteq B$ und damit $x \in B$.
	\end{enumerate}
	\item Definieren wir die kompakte Menge $K := \supp(\psi)$, so gilt $\vbraces{\abraces{T, \psi}} \leq C \norm[C^m(K)]{\psi} = 0$
	\item Nach Definition von supp $T$ gilt also
\begin{align}
    \text{supp~}\phi \subseteq \bigcup \left\{U \subseteq \Omega \text{~offen} \mid \forall \psi \in \mathcal{D}(\Omega): \text{~supp~}\psi \subseteq U \implies \langle T, \psi\rangle = 0 \right\}.
\end{align}
Die rechte Menge ist kompakt, die linke eine Vereinigung offener Mengen. Nach dem Satz von Heine-Borel existiert eine endliche Teilüberdeckung:
\begin{align}
     \text{supp~}\phi \subseteq \bigcup_{i = 0}^n U_i,
\end{align}
wobei für alle $\psi$ mit supp $\psi \subseteq U_i$ gilt: $\langle T, \psi\rangle = 0.$ \textbf{braucht man das überhaupt?}
Wir wählen eine dieser Überdeckung untergeordnete, lokal endliche $C^\infty$-Zerlegung der Eins $(\zeta_k)_{k=0}^\infty$. Wegen der lokalen Endlichkeit und der Linearität von $T$ gilt
\begin{align}
    \langle T, \phi \rangle = \langle T, \sum_{k \in \N} \zeta_k \phi \rangle = \sum_{k \in \N} \langle T, \zeta_k \phi \rangle = \sum_{k \in \N} 0 = 0.
\end{align}
	\item Seien $T \in \mathcal{D}^\prime(\Omega)$ und $f \in C^\infty(\Omega)$ so, dass $fT = 0$. Sei $x \in \Omega$ so, dass $f(x) \neq 0$, o.B.d.A $f(x) > 0$. Wegen der Stetigkeit von $f$ finden wir eine Umgebung $U$ von $x$ mit $f(U) > 0$. Für ein beliebiges gegebenes $\phi \in \mathcal{D}(\Omega)$ mit $\supp(\phi) \subseteq U$ definieren wir $\psi := \frac{\phi}{f}$ und erhalten
	\begin{align*}
	\abraces{T, \phi} = \abraces{T, f\psi} = \abraces{fT, \psi} = 0
	\end{align*}
	und damit ist $x \notin \supp(T)$.
\end{enumerate}

\end{solution}

% --------------------------------------------------------------------------------
