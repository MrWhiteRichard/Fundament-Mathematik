% --------------------------------------------------------------------------------

\begin{exercise}

Der \textit{Träger} eine Distribution $T \in \mathcal{D}^\prime(\Omega)$ ist das Komplement der größten offenen Menge, auf der $T$ verschwindet:

\begin{align*}
    \supp{T}
    =
    \Omega
    \setminus
    \bigcup
    \Bbraces
    {
        U \subseteq \Omega ~\text{offen}~ |~
        T ~\text{verschwindet auf}~ U
    }.
\end{align*}

Zeigen Sie:

\begin{enumerate}[label = (\roman*)]
    \item Für $f \in C(\Omega)$ ist der distributionelle Träger gleich dem üblichen Träger der Funktion $f$.
    \item Ist $T \in \mathcal{D}^\prime(\R^n)$ eine Distribution von endlicher Ordnung $m$ und $\psi \in \mathcal{D}(\R^n)$ eine Testfunktion deren Ableitung $\partial^\alpha \psi$ für $|\alpha| \leq m$ verschwindet, dann ist $\abraces{T, \psi} = 0$.
    \item Gilt $\supp{T} \cap \supp{\varphi} = 0$, dann folgt $\abraces{T, \varphi} = 0$.
    \item Gilt $fT = 0$ für $T \in \mathcal{D}^\prime(\Omega)$ und $f \in C^\infty(\Omega)$, dann folgt $\supp{T} \subseteq \Bbraces{x: f(x) = 0}$.
\end{enumerate}

\end{exercise}

% --------------------------------------------------------------------------------

\begin{solution}

ToDo!

\end{solution}

% --------------------------------------------------------------------------------
