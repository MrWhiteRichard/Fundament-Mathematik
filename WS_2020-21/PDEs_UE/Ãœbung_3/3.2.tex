% --------------------------------------------------------------------------------

\begin{exercise}

Bestimmen Sie folgende Grenzwerte in $\mathcal{D}^\prime(\R)$:

\begin{enumerate}[label = (\roman*)]
    \item $\lim_{\lambda \to \infty} \sin{\lambda x}$,
    \item $\lim_{\lambda \to \infty} \frac{\sin{\lambda x}}{x}$, \\
\end{enumerate}

und zeigen Sie, dass

\begin{enumerate}[label = (\roman*)]
    \setcounter{enumi}{2}
    \item $\lim_{a \to 0+} \frac{a}{x^2 + a^2} = \pi \delta$.
\end{enumerate}


\end{exercise}

% --------------------------------------------------------------------------------

\begin{solution}
\phantom{}
\begin{enumerate}[label = (\roman*)]
	\item Sei $\phi \in \mathcal{D}(\R)$ beliebig. Es gilt
	\begin{align*}
	\left|\abraces{\sin(\lambda x), \phi}\right| = \left|\frac{1}{\lambda}\abraces{\cos(\lambda x), \phi^\prime}\right| \leq \frac{1}{|\lambda|} \int_\R \left|\cos(\lambda x) \varphi^\prime(x)\right| dx \leq \frac{1}{|\lambda|} \int_\R \left|\varphi^\prime(x)\right| dx \rightarrow 0
	\end{align*}
	für $\lambda$ gegen $\infty$.
	\item Sei $\phi \in \mathcal{D}(\R)$ beliebig und sei $\supp(\phi)\subseteq [a, b]$.
	\item Mit dem Satz über die majorisierte Konvergenz gilt
	\begin{align*}
	\lim_{a \to 0+}\abraces{\arctan\pbraces{\frac{x}{a}}, \phi} = \lim_{a \to 0+} \int_\R \arctan\pbraces{\frac{x}{a}} \phi(x) dx = \int_\R \lim_{a \to 0+} \arctan\pbraces{\frac{x}{a}} \phi(x) dx = \pi \abraces{H - \frac{1}{2}, \phi},
	\end{align*}
	wobei $H$ die Heaviside-Funktion ist und nach Beispiel 3.12 die Gleichheit $H^\prime = \delta$ gilt. Mit Lemma 3.13 erhalten wir
	\begin{align*}
	\frac{a}{x^2 + a^2} = \pbraces{\arctan\pbraces{\frac{x}{a}}}^\prime \to \pbraces{\pi\pbraces{H - \frac{1}{2}}}^\prime = \pi \delta
	\end{align*}
\end{enumerate}

\end{solution}

% --------------------------------------------------------------------------------
