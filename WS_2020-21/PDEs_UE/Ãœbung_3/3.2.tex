% --------------------------------------------------------------------------------

\begin{exercise}

Bestimmen Sie folgende Grenzwerte in $\mathcal{D}^\prime(\R)$:

\begin{enumerate}[label = (\roman*)]
    \item $\lim_{\lambda \to \infty} \sin{\lambda x}$,
    \item $\lim_{\lambda \to \infty} \frac{\sin{\lambda x}}{x}$, \\
\end{enumerate}

und zeigen Sie, dass

\begin{enumerate}[label = (\roman*)]
    \setcounter{enumi}{2}
    \item $\lim_{a \to 0+} \frac{a}{x^2 + a^2} = \pi \delta$.
\end{enumerate}


\end{exercise}

% --------------------------------------------------------------------------------

\begin{solution}
\phantom{}
\begin{enumerate}[label = (\roman*)]
	\item Sei $\phi \in \mathcal{D}(\R)$ beliebig. Es gilt
	\begin{align*}
	\left|\abraces{\sin(\lambda x), \phi}\right| = \left|\frac{1}{\lambda}\abraces{\cos(\lambda x), \phi^\prime}\right| \leq \frac{1}{|\lambda|} \int_\R \left|\cos(\lambda x) \varphi^\prime(x)\right| dx \leq \frac{1}{|\lambda|} \int_\R \left|\varphi^\prime(x)\right| dx \rightarrow 0
	\end{align*}
	für $\lambda$ gegen $\infty$.
	\item Sei $\phi \in \mathcal{D}(\R)$ beliebig und $\text{supp~}\phi \subseteq [-a,a]$. Es gilt

\begin{align}
    \left\langle \frac{\sin(\lambda x)}{x}, \phi\right\rangle =
    \int_\R \frac{\sin(\lambda x)}{x} \phi(x) \text{~d}x=
    \underbrace{\int_{|x| > a} \frac{\sin(\lambda x)}{x} \phi(x) \text{~d}x}_{= 0} +
    \int_{|x| < a} \frac{\sin(\lambda x)}{x} \phi(x) \text{~d}x = \\
    \int_{|x| < a} \frac{\sin(\lambda x)}{x} (\phi(x) - \phi(0)) \text{~d}x + \phi(0) \int_{|x| < a} \frac{\sin(\lambda x)}{x} \text{~d}x.
\end{align}

Wir definieren $\psi(x) := \frac{\phi(x) - \phi(0)}{x}$ und erkennen, dass wir $\psi$ mit $\psi(0) := \phi^\prime(0)$ stetig fortsetzen können. Wir wenden auf das linke Integral partielle Integration und die Dreiecksungleichung an und sehen, dass es für $\lambda \rightarrow \infty$ gegen Null geht:
\begin{align}
\left|\int_{|x| < a} \sin(\lambda x) ~\psi(x) \text{~d}x\right|
=
\left|-\frac{1}{\lambda} \cos(\lambda x) ~ \psi(x) \Big|_{t=-a}^a \right| +
\left|\frac{1}{\lambda} \int_{|x| < a} \cos(\lambda x) ~\psi^\prime(x) \text{~d}x
\right| \\
\leq \frac{1}{\lambda} (2 \|\psi\| + 2a \|\psi^\prime\|)
\stackrel{\lambda \rightarrow \infty}{\longrightarrow} 0.
\end{align}
Für das zweite Integral substituieren wir $x \backslash \frac{x}{\lambda}$ und verwenden wir unser Wissen aus der Analysis:
\begin{align}
    \lim_{\lambda \rightarrow \infty} \int_{-\lambda}^{\lambda} \frac{\sin{(\lambda x)}}{x} \text{~d}x = \pi
\end{align}
und somit
\begin{align}
\left\langle \frac{\sin(\lambda x)}{x}, \phi\right\rangle = \pi\delta.
\end{align}
	\item Mit dem Satz über die majorisierte Konvergenz gilt
	\begin{align*}
	\lim_{a \to 0+}\abraces{\arctan\pbraces{\frac{x}{a}}, \phi} = \lim_{a \to 0+} \int_\R \arctan\pbraces{\frac{x}{a}} \phi(x) dx = \int_\R \lim_{a \to 0+} \arctan\pbraces{\frac{x}{a}} \phi(x) dx = \pi \abraces{H - \frac{1}{2}, \phi},
	\end{align*}
	wobei $H$ die Heaviside-Funktion ist und nach Beispiel 3.12 die Gleichheit $H^\prime = \delta$ gilt. Mit Lemma 3.13 erhalten wir
	\begin{align*}
	\frac{a}{x^2 + a^2} = \pbraces{\arctan\pbraces{\frac{x}{a}}}^\prime \to \pbraces{\pi\pbraces{H - \frac{1}{2}}}^\prime = \pi \delta
	\end{align*}
\end{enumerate}

\end{solution}

% --------------------------------------------------------------------------------
