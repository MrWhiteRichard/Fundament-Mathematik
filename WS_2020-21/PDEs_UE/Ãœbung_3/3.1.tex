% --------------------------------------------------------------------------------

\begin{exercise}

Zeigen Sie:

\begin{enumerate}[label = (\roman*)]

    \item Ist $f \in C^\infty(\R^n)$ und $y \in \R^n$, dann gibt es Funktionen $f_i \in C^\infty(\R^n)$ sodass

    \begin{align*}
        f(x) = f(y) + \sum_{i=1}^n (x_i - y_i) f_i(x).
    \end{align*}

    \item Gilt $xT = 0$ für eine Distribution $T \in \mathcal{D}^\prime(\R)$, so ist $T = c \delta$ für eine Konstante $c$.
    \item $u \in \mathcal{D}^\prime(\R)$ mit $u^\prime = 0$ impliziert $u = ~\text{const}$.
    \item Für jedes $f \in C^\infty(\R)$ existieren Konstanten $c_0, c_1$ sodass

    \begin{align*}
        f \delta^\prime
        =
        c_0 \delta
        +
        c_1 \delta^\prime.
    \end{align*}

\end{enumerate}

\end{exercise}

% --------------------------------------------------------------------------------

\begin{solution}
\phantom{}
\begin{enumerate}[label = (\roman*)]
	\item Sei also $f \in C^\infty(\R^n)$ und $y \in \R^n$. Wir verwenden die Taylorsche Formel, siehe dazu Kaltenbäck Satz 10.2.10, und erhalten
	\begin{align*}
	f(x) = f(y) + \int_0^1 df((1-t)y + tx)(x - y) dt = f(y) + \sum_{i = 1}^n (x_i - y_i) \int_0^1 \pderivative[][f]{y_i} ((1-t)y + tx) dt.
	\end{align*}
	Mit der Definition
	\begin{align*}
	f_i(x) := \int_0^1  \pderivative[][f]{y_i} ((1-t)y + tx) dt
	\end{align*}
	erhalten wir die gewünschte Gestalt, die $f_i$ sind auch $C^\infty$ weil man eine beliebige partielle Ableitung des Integranden auf dem Intervall $[0,1]$ mit dem Supremum majorisieren kann also den Differentialoperator mit dem Integral vertauschen darf. Vergleiche dazu Kusolitsch Korollar 9.37.
	\item Sei $T \in \mathcal{D}^\prime(\R)$ mit $xT = 0$ und $\varphi \in \mathcal{D}(\R)$. Gemäß Hinweis wählen wir ein $\chi \in \mathcal{D}(\R)$ mit $\forall x \in \supp{\varphi}: \chi(x) = 1$ (gibt es sowas?
  Ja, betrachte die Faltung von $\1_A$ mit $A \subset \R$ kompakt, sodass $\dist(\supp \varphi, A^C) > \delta$ mit einer bekannten Testfunktion $g$ mit $\supp g \subset [-\delta, \delta]$.
  Da $f \in L^1(\R)$ und $g \in C^{\infty}(\R)$ ist auch $f*g \in C^{\infty}(\R)$ und es gilt $f*g \equiv 1$ auf $\supp \varphi$.) Wir berechnen
	\begin{align*}
	\abraces{T,\varphi} = \abraces{T, \chi \varphi} \stackrel{(i)}{=} \abraces{T, \chi (\varphi(0) + x\varphi_1)} = \varphi(0) \abraces{T, \chi} + \underbrace{\abraces{xT, \chi \varphi_1}}_{=0}.
	\end{align*}
	Nun hängt das $\chi$ allerdings noch von $\varphi$ ab, wir erhalten obige Gleichheit allerdings für alle $\chi \in \mathcal{D}(\R)$ mit $\forall x \in \supp{\varphi}: \chi(x) = 1$ Für ein $\psi \in \mathcal{D}(\R)$ mit $\psi(0) \neq 0$ gilt also für alle entsprechenden $\chi$ die Gleichheit $c := \frac{\abraces{T,\psi}}{\psi(0)} = \abraces{T, \chi}$. Für eine beliebige weitere Funktion $\hat{\psi}$ können wir nun das $\chi$ so wählen, dass es nicht nur am Träger von $\hat{\psi}$ sondern auch am Träger von $\psi$ den Wert $1$ annimmt. Für solch ein $\chi$ kennen wir aber schon den Wert $\abraces{T, \chi} = c$. 
	\item Sei $u \in \mathcal{D}^\prime(\R)$ mit $u^\prime = 0$. Wir erinnern uns an den Beweis von Blümlingers Prop. 6.14. Sei $\Psi_0$ eine Testfunktion die $\int_\R \Psi_0 d\lambda = 1$ erfüllt. Wir definieren für beliebiges $\phi \in \mathcal{D}(\R)$ die Funktion $\zeta := \phi - \int_\R \phi d\lambda \Psi_0$, welche $\int_\R \zeta d\lambda = 0$ erfüllt. Die Funktion $\theta(x) := \int_{-\infty}^{x} \zeta d\lambda$ ist aus $\mathcal{D}(\R)$ und es gilt $\theta^\prime = \zeta$. So erhalten wir die Darstellung $\phi = \theta^\prime + \int_\R \phi d\lambda \Psi_0$. So können wir berechnen
	\begin{align*}
	\abraces{u, \phi} = \abraces{u, \theta^\prime} + \abraces{u, \int_\R \phi d\lambda \Psi_0} = \abraces{u^\prime, \theta} + \int_\R \phi d\lambda \abraces{u, \Psi_0} = \abraces{\abraces{u, \Psi_0}, \phi}
	\end{align*}
	\item Sei $f \in C^\infty(\R)$. Es gilt
	\begin{align*}
	\abraces{f \delta^\prime, \varphi} = \abraces{\delta^\prime, f\varphi} = - (f\varphi)^\prime (0) = -f(0) \varphi^\prime(0) - f^\prime(0) \varphi(0) = f(0) \abraces{\delta^\prime, \phi} - f^\prime(0) \abraces{\delta, \varphi}
	\end{align*}
\end{enumerate}

\end{solution}

% --------------------------------------------------------------------------------
