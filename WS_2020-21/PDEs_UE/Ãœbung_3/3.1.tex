% --------------------------------------------------------------------------------

\begin{exercise}

Zeigen Sie:

\begin{enumerate}[label = (\roman*)]

    \item Ist $f \in C^\infty(\R^n)$ und $y \in \R^n$, dann gibt es Funktionen $f_i \in C^\infty(\R^n)$ sodass
    
    \begin{align*}
        f(x) = f(y) + \sum_{i=1}^n (x_i - y_i) f_i(x).
    \end{align*}

    \item Gilt $xT = 0$ für eine Distribution $T \in \mathcal{D}^\prime(\R)$, so ist $T = c \delta$ für eine Konstante $c$.
    \item $u \in \mathcal{D}^\prime(\R)$ mit $u^\prime = 0$ impliziert $u = ~\text{const}$.
    \item Für jedes $f \in C^\infty(\R)$ existieren Konstanten $c_0, c_1$ sodass
    
    \begin{align*}
        f \delta^\prime
        =
        c_0 \delta
        +
        c_1 \delta^\prime.
    \end{align*}

\end{enumerate}

\end{exercise}

% --------------------------------------------------------------------------------

\begin{solution}

ToDo!

\end{solution}

% --------------------------------------------------------------------------------
