% --------------------------------------------------------------------------------

\begin{exercise}

Eine Distribution heißt \textit{positiv} wenn $\abraces{u, \varphi} \geq 0$ für alle $\varphi \geq 0$ gilt.

\begin{enumerate}[label = (\roman*)]
    \item Zeigen Sie, dass jede positive Distribution Ordnung $0$ hat.
    \item Zeigen Sie, dass folgende Distribution $T \in \mathcal{D}^\prime(\R)$ nicht positiv ist:
    
    \begin{align*}
        \abraces{T, \varphi}
        =
        \Int[-\infty][-1]
        {
            \frac{\varphi(x)}{|x|}
        }{x}
        +
        \Int[1][\infty]
        {
            \frac{\varphi(x)}{|x|}
        }{x}
        +
        \Int[-1][1]
        {
            \frac{\varphi(x) - \varphi(0)}{|x|}
        }{x}.
    \end{align*}

\end{enumerate}

\end{exercise}

% --------------------------------------------------------------------------------

\begin{solution}

\phantom{}

\begin{enumerate}[label = (\roman*)]

    \item Seien $K \subseteq \Omega$ kompakt und $u \in \mathcal{D}(K)^\prime$ positiv sowie $\varphi \in \mathcal{D}(K)$.
    Laut der Konstruktion aus Aufgabe 1, $\Exists \chi \in \mathcal{D}(\Omega):$

    \begin{align*}
        \Forall x \in K: \chi(x) = 1,
        \quad
        \chi \geq 0.
    \end{align*}
    
    Laut der Definition der Supremumsnorm gilt $\phi := \varphi - \chi \norm[C(K)]{\varphi} \leq 0$.
    Wegen der Linearität von $u$, folgt aus $\phi \leq 0$ auch $\abraces{u, \phi} \leq 0$.

    \begin{align*}
        \implies
        \abraces{u, \phi} \leq 0
        \iff
        \abraces{u, \varphi} \leq \abraces{u, \chi} \norm[C(K)]{\varphi}
	\end{align*}

    \item Wir wählen die Funktion $\varphi$ welche, wie wir aus der Vorlesung wissen, aus $\mathcal{D}(\R)$ ist und $\varphi \geq 0$ erfüllt.

    \begin{align*}
        \varphi(x)
        =
        \begin{cases}
            \exp\pbraces{\frac{1}{x^2 - 1}}, & \vbraces{x} < 1, \\
            0,                               & \vbraces{x} \geq 1
        \end{cases}
	\end{align*}

    Man beachte, dass $\supp{\varphi} \subseteq [-1, 1]$.
    Außerdem nimmt die Funktion im Punkt $0$ ihr Maximum an.

    \begin{align*}
    \implies
    \Forall x \in [-1, 1]:
    \varphi(x) \leq \varphi(0)
    \implies
	\abraces{T, \varphi} = \Int[-1][1]
	{
		\frac{\varphi(x) - \varphi(0)}{|x|}
	}{x} < 0 
	\end{align*}

    Damit $T$ nicht positiv.

\end{enumerate}

\end{solution}

% --------------------------------------------------------------------------------
