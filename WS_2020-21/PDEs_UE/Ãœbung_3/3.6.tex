% --------------------------------------------------------------------------------

\begin{exercise}

Eine Distribution heißt \textit{positiv} wenn $\abraces{u, \varphi} \geq 0$ für alle $\varphi \geq 0$ gilt.

\begin{enumerate}[label = (\roman*)]
    \item Zeigen Sie, dass jede positive Distribution Ordnung $0$ hat.
    \item Zeigen Sie, dass folgende Distribution $T \in \mathcal{D}^\prime(\R)$ nicht positiv ist:
    
    \begin{align*}
        \abraces{T, \varphi}
        =
        \Int[-\infty][-1]
        {
            \frac{\varphi(x)}{|x|}
        }{x}
        +
        \Int[1][\infty]
        {
            \frac{\varphi(x)}{|x|}
        }{x}
        +
        \Int[-1][1]
        {
            \frac{\varphi(x) - \varphi(0)}{|x|}
        }{x}.
    \end{align*}

\end{enumerate}

\end{exercise}

% --------------------------------------------------------------------------------

\begin{solution}
\phantom{}
\begin{enumerate}[label = (\roman*)]
	\item Seien $K \subseteq \Omega$ und $\phi \in \mathcal{D}(K)$ beliebig.
	\item Wir wählen die Funktion 
	\begin{align*}
	\phi(x) =
	\begin{cases}
	\exp\pbraces{\frac{1}{x^2 - 1}} &, \vbraces{x} < 1 \\
	0 &, \vbraces{x} \geq 1,
	\end{cases}
	\end{align*}
	welche, wie wir aus der Vorlesung wissen, aus $\mathcal{D}(\R)$ ist und $\phi \geq 0$ erfüllt. Außerdem nimmt die Funktion im Punkt $0$ ihr Maximum an. Deshalb ist sicher
	\begin{align*}
	\abraces{T, \varphi} = \Int[-1][1]
	{
		\frac{\varphi(x) - \varphi(0)}{|x|}
	}{x} < 0 
	\end{align*}
	und damit $T$ nicht positiv.
\end{enumerate}

\end{solution}

% --------------------------------------------------------------------------------
