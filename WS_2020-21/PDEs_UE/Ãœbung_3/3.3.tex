% --------------------------------------------------------------------------------

\begin{exercise}

\phantom{}

\begin{enumerate}[label = (\roman*)]
    \item Zeigen Sie, dass die Funktion
    
    \begin{align*}
        f(x)
        =
        \begin{cases}
            \ln{|x|} & x \neq 0 \\
            0        & x = 0
        \end{cases}
    \end{align*}

    eine reguläre Distribution definiert, die punktweise Ableitung

    \begin{align*}
        f^\prime(x)
        =
        \begin{cases}
            \frac{1}{x}        & x \neq 0 \\
            \text{undefiniert} & x = 0
        \end{cases}
    \end{align*}

    jedoch nicht.

    \item Es bezeichne $\pv{(\frac{1}{x})}$ die Distribution
    
    \begin{align*}
        \langle \pv{\pbraces{\frac{1}{x}}}, \varphi \rangle
        =
        \lim_{\varepsilon \to 0+}
        \pbraces
        {
            \Int[-\infty][-\varepsilon]
            {
                \frac{\varphi(x)}{x}
            }{x}
            +
            \Int[\varepsilon][\infty]
            {
                \frac{\varphi(x)}{x}
            }{x}
        }
        =
        \lim_{\varepsilon \to 0+}
        \Int[|x| > \varepsilon]
        {
            \frac{\varphi(x)}{x}
        }{x}.
    \end{align*}

    Zeigen Sie, dass $\abraces{\pv{(\frac{1}{x})}, \varphi} = \Int[0][\infty]{\frac{\varphi(x) - \varphi(-x)}{x}}{x}$.

    \item Überprüfen Sie, dass $(\ln{|x|})^\prime = \pv{(\frac{1}{x})}$ in $\mathcal{D}^\prime(\R)$ gilt.

\end{enumerate}

\end{exercise}

% --------------------------------------------------------------------------------

\begin{solution}

\phantom{}

\begin{enumerate}[label = (\roman*)]

    \item Gegeben sei eine beliebige kompakte Menge $K \subseteq \R$.
    Wir wählen $a \in \R^+$ mit $K \subseteq [-a, a]$.
    Laut der Regel von L'Hospital, gilt

    \begin{align*}
        \lim_{\epsilon \to 0+}
        \ln{(\epsilon)} \epsilon = 0.
    \end{align*}

    \begin{align*}
        \implies
        \Int[K]{|f(x)|}{x}
        & =
        \Int[-a][a]{|\ln{|x||}}{x}
        =
        2 \Int[0][a]{|\ln{x}|}{x}
        =
        2 \pbraces
        {
            \Int[0][1]{-\ln{x}}{x}
            +
            \Int[1][a]{\ln{x}}{x}
        } \\
        & =
        2 \pbraces
        {
            -
            \lim_{\epsilon \to 0+}
            (\ln{(x)} x - x) \Big|_\epsilon^1
            +
            (\ln{(x)} x - x) \Big|_1^a
        } \\
        & =
        2
        \Bigg (
            -
            \underbrace
            {
                (\ln{(1)} 1 - 1)
            }_{< \infty}
            +
            \underbrace
            {
                \lim_{\epsilon \to 0+}
                (\ln{(\epsilon)} \epsilon - \epsilon)
            }_0
            +
            \underbrace
            {
                (\ln{(a)} a - a)
            }_{< \infty}
            -
            \underbrace
            {
                (\ln{(1)} 1 - 1)
            }_{< \infty}
        \Bigg ) < \infty
	\end{align*}
	Betrachte hingegen 
	\begin{align*}
	\int_0^1 \vbraces{f^\prime(x)} dx = \int_0^1 \frac{1}{x} dx = \log(1) - \lim_{\epsilon \to 0+} \log(\epsilon) = \infty.
	\end{align*}
	\item Wir berechnen
	\begin{align*}
	\abraces{\pv\pbraces{\frac{1}{x}}, \varphi} &= \lim_{\varepsilon \to 0+}
	\pbraces
	{
		\Int[-\infty][-\varepsilon]
		{
			\frac{\varphi(t)}{t}
		}{t}
		+
		\Int[\varepsilon][\infty]
		{
			\frac{\varphi(x)}{x}
		}{x}
	} \\
	&= \lim_{\varepsilon \to 0+}
	\pbraces
	{
	-\Int[\varepsilon][\infty]
	{
		\frac{\varphi(-x)}{x}
	}{x}
	+
	\Int[\varepsilon][\infty]
	{
		\frac{\varphi(x)}{x}
	}{x}
	} = \Int[0][\infty]{\frac{\varphi(x) - \varphi(-x)}{x}}{x}
	\end{align*}
	\item 
	\begin{align*}
	\abraces{\pbraces{\log|x|}^\prime, \varphi} &= -\abraces{\log|x|, \varphi^\prime} = -\Int[\R][]{\log|x| \varphi^\prime(x)}{x} = -\Int[0][\infty]{\log|x| (\varphi^\prime(x) + \varphi^\prime(-x))}{x} \\
	&= -\Int[0][\infty]{\log|x| (\varphi(x) - \varphi(-x))^\prime}{x} = \Int[0][\infty]{\frac{\varphi(x) - \varphi(-x)}{x}}{x} = \abraces{\pv\pbraces{\frac{1}{x}}, \varphi}
	\end{align*}
\end{enumerate}

\end{solution}

% --------------------------------------------------------------------------------
