% --------------------------------------------------------------------------------

\begin{exercise}

Zeigen Sie, dass

\begin{align*}
    \lim_{\varepsilon \to 0+}
    \frac{1}{x - i \varepsilon}
    =
    \pv
    {
        \pbraces
        {
            \frac{1}{x}
        }
    }
    +
    \pi i \delta.
\end{align*}

\end{exercise}

% --------------------------------------------------------------------------------

\begin{solution}

Wir beginnen mit der folgenden Rechnung.

\begin{align*}
    \frac{1}{x - i\varepsilon}
    =
    \frac
    {
        x + i \varepsilon
    }{
        (x - i\varepsilon)
        \overline
        {
            (x - i\varepsilon)
        }
    }
    =
    \frac
    {
        x + i \varepsilon
    }{
        |x - i \varepsilon|^2
    }
    =
    \frac{x + i \varepsilon}{x^2 + \varepsilon^2}
    =
    \frac{x}{x^2 + \varepsilon^2} + i\frac{\varepsilon}{x^2 + \varepsilon^2}.
\end{align*}

Gegen was der zweite Summand konvergiert wissen wir schon aus der zweiten Aufgabe.
Kümmern wir uns also um den Ersten.
Sei dazu $\phi \in \mathcal{D}(\R)$.

\begin{align*}
    f_\varepsilon:
    [a,b] \to \R:
    x \mapsto \log\pbraces{\sqrt{x^2 + \varepsilon^2}} \phi(x),
    \quad
    f:
    [a,b] \to \R:
    x \mapsto \log{|x|} \phi(x)
\end{align*}

Seien $a \in \R^-$, $b \in \R^+$, sodass $\supp{\phi} \subseteq [a, b]$.
Wir können also mit dem Satz von der dominierten Konvergenz

die folgende Vertauschung des Grenzwerts mit dem Integral rechtfertigen.
Auf den Intervallen $[a,-\delta], [\delta, b]$ ist für $\varepsilon^2 < \delta/2$
die Funktion
\begin{align*}
  g(x) = \max\left\{\left|\log\pbraces{\sqrt{a^2 + \varepsilon^2}}\right|, \left|\log\pbraces{\sqrt{b^2 + \varepsilon^2}}\right|, \left|\log\pbraces{\sqrt{\delta/2}}\right|\right\}|\phi(x)|
\end{align*} eine
integrierbare Majorante.
Während für das Intervall $[-\delta,\delta]$ für $ \delta^2 + \varepsilon^2 < 1$ die Ungleichung
\begin{align*}
  \left|\log\pbraces{\sqrt{x^2 + \varepsilon^2}}\right| \leq \left|\log(|x|)\right|
\end{align*}
gilt, also können wir als integrierbare Majorante $g(x) = |\log{|x|}\phi(x)|$ nehmen.
\begin{align*}
    \lim_{\varepsilon \to 0+}
    \abraces{\log \pbraces{\sqrt{x^2 + \varepsilon^2}}, \phi}
    & =
    \lim_{\varepsilon \to 0+}
    \Int[\R]{\log\pbraces{\sqrt{x^2 + \varepsilon^2}} \phi(x)}{x}
    =
    \lim_{\varepsilon \to 0+}
    \Int[a][b]{\log\pbraces{\sqrt{x^2 + \varepsilon^2}} \phi(x)}{x} \\
    & =
    \lim_{\varepsilon \to 0+}
    \pbraces
    {
        \Int[a][-\delta]{f_\varepsilon(x)\phi(x)}{x}
        +
        \Int[-\delta][\delta]{f_\varepsilon(x)\phi(x)}{x}
        +
        \Int[\delta][b]{f_\varepsilon(x) \phi(x)}{x}
    } \\
    & =
    \Int[\R][]{\log|x| \phi(x)}{x}
    =
    \abraces{\log|x|, \phi}.
\end{align*}

Mit Lemma 3.13 folgt die gewünschte Gleichheit.

\begin{align*}
    \frac{x}{x^2 + \varepsilon^2}
    =
    \pderivative{x}
    \bbraces
    {
        \log
        \pbraces
        {
            \sqrt{x^2 + \varepsilon^2}
        }
    }
    \xrightarrow{\varepsilon \to 0+}
    [\log{|x|}]^\prime
    =
    \pv \pbraces{\frac{1}{x}}
\end{align*}

\end{solution}

% --------------------------------------------------------------------------------
