% --------------------------------------------------------------------------------

\begin{exercise}

Zeigen Sie, dass

\begin{align*}
    \lim_{\varepsilon \to 0+}
    \frac{1}{x - i \varepsilon}
    =
    \pv
    {
        \pbraces
        {
            \frac{1}{x}
        }
    }
    +
    \pi i \delta.
\end{align*}

\end{exercise}

% --------------------------------------------------------------------------------

\begin{solution}

Wir beginnen mit der Rechnung 
\begin{align*}
\frac{1}{x - i\varepsilon} = \frac{x + i \varepsilon}{x^2 + \varepsilon^2} = \frac{x}{x^2 + \varepsilon^2} + i\frac{\varepsilon}{x^2 + \varepsilon^2}.
\end{align*}
Gegen was der zweite Summand konvergiert wissen wir schon aus der zweiten Aufgabe. K"ummern wir uns also um den Ersten. Könnten wir beispielsweise für ein beliebiges $\phi \in \mathcal{D}(\R)$ zeigen, dass (evtl Satz von der monotonen Konvergenz?)
\begin{align*}
\lim_{\varepsilon \to 0+} \abraces{\log\pbraces{\sqrt{x^2 + \varepsilon^2}}, \phi}&= \lim_{\varepsilon \to 0+} \Int[\R][]{\log\pbraces{\sqrt{x^2 + \varepsilon^2}} \phi(x)}{x} = \Int[\R][]{\lim_{\varepsilon \to 0+}\log\pbraces{\sqrt{x^2 + \varepsilon^2}} \phi(x)}{x} = \\
&= \Int[\R][]{\log|x| \phi(x)}{x} = \abraces{\log|x|, \phi},
\end{align*}
so würde mit Lemma 3.13 schon 
\begin{align*}
\frac{x}{x^2 + \varepsilon^2} = \pbraces{\log\pbraces{\sqrt{x^2 + \varepsilon^2}}}^\prime \to \pbraces{\log|x|} = \pv\pbraces{\frac{1}{x}}
\end{align*}
folgen.
\end{solution}

% --------------------------------------------------------------------------------
