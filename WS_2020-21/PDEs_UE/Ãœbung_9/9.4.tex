% -------------------------------------------------------------------------------- %

\begin{exercise}
Betrachten Sie die eindimensionale Wärmeleitungsgleichung mit gemischten Randbedingungen
\begin{align*}
  \begin{cases}
    u_t - u_{xx} = 0 & \text{für } x \in (0,\pi),\ t > 0, \\
    u(0,t) = 0 & \text{für } t > 0, \\
    u_x(\pi,t) = 0 & \text{für } t > 0, \\
    u(x,0) = u_0(x) & \text{für } x \in (0,\pi),
  \end{cases}
\end{align*}
wobei $u_0 \in L^2(0,\pi)$.
\begin{enumerate}[label = (\roman*)]
  \item Bestimmen Sie ein vollständiges Orthonormalsystem $(\phi_n)_{n \in \N} \subset L^2(0,\pi)$
  mit $\phi^{\primeprime}_n = \lambda_n\phi_n$ in $(0,\pi)$ mit Randbedingungen
  $\phi_n(0) = \phi_n^{\prime}(\pi) = 0.$
  \item Konstruieren Sie aus $(\phi_n)_{n \in \N}$ eine Lösungsformel für das
  obige parabolische Problem.
  \item Welche Abklingrate (für $t \to \infty$) hat die Wärmeenergie
  $E(t) := \int_0^{\pi} u(x,t) dx$ für eine Lösung $u$?
\end{enumerate}



\end{exercise}

% -------------------------------------------------------------------------------- %

\begin{solution}
\begin{enumerate}[label = (\roman*)]
  \item Wir betrachten zunächst Lösungen der Differentialgleichung

  \begin{align*}
    \phi_n
    =
    c_1 e^{\sqrt{\lambda_n}x} + c_2 e^{-\sqrt{\lambda_n}x}
  \end{align*}

  Fall 1: $\lambda_n > 0$: \\
  Dann hat die Lösung folgende Form:

  \begin{align*}
    \phi_n(x)
    =
    c_1 \cosh(\sqrt{\lambda_n}x) + c_2 \sinh(\sqrt{\lambda_n}x)
  \end{align*}

  Diese kann die Randbedingungen jedoch nur im trivialen Fall $c_1, c_2 = 0$ erfüllen, da

  \begin{align*}
    &\phi_n(0) = c_1 \stackrel{!}{=} 0\\
    \implies
    &\phi^\prime(\pi) = c_2 \sqrt{\lambda_n} \cosh({\sqrt{\lambda_n}x}) \stackrel{!}{=} 0
  \end{align*}

  Nun ist $\cosh$ jedoch eine strikt positive Funktion, also muss auch $c_2 = 0$.

  Fall 2: $\lambda_n = 0$: \\
  Hier ist unsere Differentialgleichung nur $\phi_n^\primeprime = 0$, hat also die Form $c_1 x + c_2$. Auch hier erhalten wir $\phi_n \equiv 0$.

  Fall 3: $\lambda_n < 0$: \\
  Lösungen sind gegeben durch

  \begin{align*}
    \phi_n(x) = c_1 \sin(\sqrt{|\lambda_n|}x) + c_2 \cos(\sqrt{|\lambda_n|}x)
  \end{align*}

  Für die RB setzen wir ein
  \begin{align*}
    &\phi_n(0) = c_2 \stackrel{!}{=}0 \\
    \implies
    &\phi_n^\prime(\pi) = c_2 \sqrt{|\lambda_n|} \cos({\sqrt{|\lambda_n|}\pi}) \stackrel{!}{=} 0
    \iff
    \sqrt{|\lambda_n|}\pi = \frac{\pi}{2} + k\pi, \quad k \in \Z
  \end{align*}

  Wir haben also zunächst mit der Wahl $k := n-1$ eine Darstellung von $\phi_n$, mit einer noch zu bestimmenden Konstante c, als

  \begin{align*}
    \phi_n(x) = c \sin\left(\frac{2n-1}{2}x\right)
  \end{align*}

  Zweimaligens differenzieren liefert

  \begin{align*}
    \phi_n^\primeprime(x) = \underbrace{-\left(\frac{2n-1}{2}\right)^2}_{\lambda_n} c\sin\left(\frac{2n-1}{2}x\right)
  \end{align*}

  Da $\norm[L^2((0,\pi))]{\sin(\frac{2n-1}{2}x)} = \sqrt{\frac{\pi}{2}}$ wählen wir $c = \sqrt{\frac{2}{\pi}}$ um $\norm[L^2((0,\pi))]{\phi_n} = 1$ zu erhalten. \\

  Wir haben also ein Orthonormalsystem gefunden, von dem wir noch zeigen wollen, dass es auch vollständig ist.
  Aus der Analysis wissen wir, dass $\{\frac{1}{\sqrt{\pi}} \cos(nx): n \in \N\} \cup \{\frac{1}{\sqrt{\pi}} \sin(nx): n \in \N^{+}\}$ eine vollständiges Orthonormalsystem vom $L^2(-\pi, \pi)$ ist.
  Betrachten wir also eine beliebige Funktion $f \in L^2(0, \pi)$ und setzen wir diese ungerade fort auf $(- \pi, \pi)$, wissen wir, dass wir das neue $\tilde{f}$ darstellen können als

  \begin{align*}
    \tilde{f} = \sum_{n=0}^{\infty} (\tilde{f}, \frac{1}{\sqrt{\pi}} \cos(nx))_{L^2(- \pi,\pi)} \frac{1}{\sqrt{\pi}} \cos(nx) + \sum_{n=1}^{\infty} (\tilde{f}, \frac{1}{\sqrt{\pi}} \sin(nx))_{L^2(-\pi, \pi)} \frac{1}{\sqrt{\pi}} \sin(nx)
  \end{align*}

  Die cosinus-Teile fallen weg, da $\tilde{f}$ ungerade ist.

  Aus der oberen Gleichheit folgt also, dass für unser $f \in L^2(0, \pi)$

  \begin{align*}
    f = \sum_{n=1}^{\infty} ({f}, \sqrt{\frac{2}{\pi}} \sin(nx))_{L^2(0, \pi)} \sqrt{\frac{2}{\pi}} \sin(nx)
  \end{align*}

  und somit $\{\sqrt{\frac{2}{\pi}} \sin(nx): n \in \N^{+}\}$ eine Orthonormalbasis ist. Daraus kann man vermutlich zeigen, dass auch unser Orthonormalsystem $\{\sqrt{\frac{2}{\pi}} \sin(\frac{2n-1}{2}x): n \in \N^{+}\}$ vollständig ist - darauf müssen wir an dieser Stelle leider verzichten.

  \item Da wir ein vollständiges Orthonormalsystem haben, konvergiert die folgende Reihe unbedingt und damit absolut.

  \begin{align*}
    \sum_{n=1}^\infty (u_0, \phi_n)_{L^2} \phi_n = u_0
  \end{align*}

  Folgende Funktion ist also wohldefiniert.

  \begin{align*}
    u(x,t)
    :=
    \sum_{n=1}^\infty e^{\lambda_n t} (u_0, \phi_n)_{L^2} \phi_n
    \leq
    \sum_{n=1}^\infty
    \underbrace
    {
      \abs{e^{\lambda_n t}}
    }_{\leq 1}
    \abs
    {
      (u_0, \phi_n)_{L^2} \phi_n
    }
    < \infty
  \end{align*}

  Diese Funktion löst die PDE, dazu setzen wir ein

  \begin{align*}
    u_t - u_{xx} &=\sum_{n=1}^\infty \lambda_n e^{\lambda_n t} (u_0, \phi_n)_{L^2} \phi_n - \sum_{n=1}^\infty \lambda_n e^{\lambda_n t} (u_0, \phi_n)_{L^2} \phi_n= 0 \\
    u(0,t) &= \sum_{n=1}^\infty e^{\lambda_n t} (u_0, \phi_n)_{L^2} \sqrt{\frac{2}{\pi}} \sin(\sqrt{-\lambda_k} 0) = 0 \\
    u_x(\pi, t) &= \sum_{n=1}^\infty \lambda_n e^{\lambda_n t} (u_0, \phi_n)_{L^2} \sqrt{\frac{2}{\pi}} \cos(\sqrt{-\lambda_k} \pi)= 0 \\
    u(x, 0) &= \sum_{n=1}^\infty (u_0, \phi_n)_{L^2} \phi_n(x) = u_0(x)
  \end{align*}

  \item Unter der Annahme, dass mit der Abklingrate eine Ähnliche Ungleichung wie in $(6.15)$ im Skript gemeint ist, schätzen wir ab (wobei $\lambda_1$ der betragsmäßig kleinste (d.h. größte) EW ist)

  \begin{align*}
    \norm[L^2]{u(\cdot,t)}^2
    & =
    \norm[L^2]
    {
      \sum_{n=1}^\infty e^{\lambda_n t} (u_0, \phi_n)_{L^2} \phi_n
    }^2 \\
    & \stackrel{\text{Pythagoras}}{=}
    \sum_{n=1}^\infty
    \norm[L^2]
    {
      e^{\lambda_n t} (u_0, \phi_n)_{L^2} \phi_n
    }^2 \\
    & =
    \Int[0][\pi]{\sum_{n=1}^\infty e^{2\lambda_n t} (u_0, \phi_n)_{L^2}^2 \phi_n^2}{x} \\
    & \leq
    e^{2\lambda_1 t} \sum_{n=1}^\infty (u_0, \phi_n)_{L^2}^2 \underbrace{\norm[L^2(0, \pi)]{\phi_n}^2}_1
    \stackrel
    {
      \mathrm{Parceval}
    }{=}
    e^{2\lambda_1 t} \norm[L^2]{u_0}^2
  \end{align*}

  Insgesamt erhalten wir eine Abschätzung

  \begin{align*}
    |E(t)|
    \leq
    \Int[0][\pi]{|u(x,t)|}{x}
    \leq
    \sqrt{\pi} \norm[L^2]{u(\cdot,t)}
    \leq
    \sqrt{\pi} e^{\lambda_1 t}\norm[L^2]{u_0}
  \end{align*}
  \end{enumerate}
\end{solution}
