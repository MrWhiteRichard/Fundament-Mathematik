% -------------------------------------------------------------------------------- %

\begin{exercise}

Betrachten Sie die Differentialgleichung
\begin{align*}
  \Delta^2 u +  u = f \quad \text{in } \R^n
\end{align*}
mit $f \in L^2(\R^n)$ und $\Delta^2 u = \Delta(\Delta u)$.
\begin{enumerate}[label = (\roman*)]
  \item Bestimmen Sie formal für eine Lösung $u$ eine Darstellung der
  Fourier-Transformierten $\hat{u}$ in Abhängigkeit von $\hat{f}$.
  \item Zeigen Sie $\hat{u} \in L^1(\R^n) \cap L^2(\R^n)$ für $n \leq 7$. \\
  \textit{Hinweis:} Sie können verwenden, dass $\hat{f} \in L^2(\R^n)$.
\end{enumerate}

\end{exercise}

% -------------------------------------------------------------------------------- %

\begin{solution}

\phantom{}\begin{itemize}
    \item[(i)] Es gilt $\Delta^2 u = \sum_{i, j=0}^n \frac{\partial^4 u}{\partial x_i^2 \partial x_j^2}.$ Weiters stellen wir fest, dass
    \begin{align*}
     \sum_{i, j=0}^n k_i^2 k_j^2 =
     \sum_{i = 0}^n k_i^2 \left(\sum_{j = 0}^n k_j^2\right) = \sum_{i = 0}^n k_i^2 ~|k|^2 = |k|^4.
    \end{align*}
    Wir wenden nun auf beide Seiten der Differentialgleichung die Fouriertransformation an und erhalten mit der gerade gemachten Hilfsrechnung
    \begin{align*}
     \mathcal{F}\left(\sum_{i, j=0}^n \frac{\partial^4 u}{\partial x_i^2 \partial x_j^2}\right)(k) + \hat u(k) &= \hat f(k) \\
     \left(\sum_{i, j=0}^n k_i^2 k_j^2\right) \hat u(k) + \hat u(k) &= \hat f(k) \\
     \left(|k|^4 + 1\right)\hat u(k) &= \hat f(k) \\
     \hat u(k) &= \frac{\hat f(k)}{|k|^4 + 1}.
    \end{align*}
    \item[(ii)] Aus Teil (i) und der Ungleichung von Cauchy-Schwarz-Bunjakowski folgt
    \begin{align*}
        \int_{\R^n} \left| \hat u \right| \mathrm{d}k = \int_{\R^n} \left| \frac{\hat f(k)}{|k|^4 + 1} \right| \mathrm{d}k \leq \|f\|_{L^2(\R^n)} \left\|\frac{1}{|k|^4 + 1}\right\|_{L^2(\R^n)}.
    \end{align*}
    Für die Abschätzung des zweiten Faktors ist die Tatsache hilfreich, dass die Funktion $\frac{1}{|k|^4 + 1}$ radialsymmetrisch ist. Deshalb können wir die Koflächenformel anwenden und erhalten
    \begin{align*}
        \int_{\R^n} \left| \frac{1}{|k|^4 + 1} \right|^2 ~\mathrm{d}k &=
        \int_{\R^n} \frac{1}{|k|^8 + 2|k|^4 + 1} ~\mathrm{d}k\\ &\leq \int_{\R^n} \frac{1}{|k|^8 + 1} ~\mathrm{d}k\\
        &= \int_0^\infty \int_{\partial B_r(0)} \frac{1}{r^8 + 1} ~\mathrm{d}\mathcal{H}^{n-1} ~\mathrm{d}r\\
        &= \int_0^\infty \mathcal{H}^{n-1}\left(\partial B_r(0)\right) \frac{1}{r^8 + 1} ~\mathrm{d}r\\
        &= S_n \int_0^\infty \frac{r^{n-1}}{r^8 + 1}
        ~\mathrm{d}r\\
        &\leq S_n \left(\int_0^1 1~
        \mathrm{d}r + \int_1^\infty \frac{1}{r^2}
        ~\mathrm{d}r\right) < \infty;
    \end{align*}
    also gilt $\hat u \in L^1(\R^n)$.

    $\hat u \in L^2(\R^n)$ erhalten wir durch eine wesentlich simplere Überlegung:
    \begin{align*}
        \int_{\R^n} \left| \hat u \right|^2 \mathrm{d}k = \int_{\R^n} \left| \frac{\hat f(k)}{|k|^4 + 1} \right|^2 \mathrm{d}k \leq \int_{\R^n} \left| \hat f(k) \right|^2 \mathrm{d}k < \infty.
    \end{align*}
\end{itemize}

\end{solution}

% -------------------------------------------------------------------------------- %
