% --------------------------------------------------------------------------------

\begin{exercise}

Sei $\Omega$ eine beschränkte Teilmenge von $\R^n$ mit $C^{\infty}$-Rand und $\1_\Omega$ ihre
Indikatorfunktion. Zeigen Sie
\begin{align*}
  \langle \Delta \1_\Omega, \varphi \rangle
  = \int_{\partial\Omega} \frac{\partial \varphi}{\partial \nu} ds,
\end{align*}
wobei $\nu$ der äußere Normaleneinheitsvektor auf $\partial\Omega$ ist.

\end{exercise}

% --------------------------------------------------------------------------------

\begin{solution}

Klarerweise ist $\1_\Omega$ eine $L^1_{\mathrm{loc}}$-Funktion, somit ist der Ausdruck
auf der linken Seite wohldefiniert und es gilt (da der Laplace-Operator formal selbstadjungiert ist)
\begin{align*}
  \langle \Delta \1_\Omega, \varphi \rangle
  = \langle  \1_\Omega, \Delta \varphi \rangle
  = \int_{\R^n}\1_\Omega \Delta \varphi dx
  = \int_{\Omega} \Delta \varphi dx
  = \int_{\Omega} \mathrm{div} (\nabla \varphi) dx.
\end{align*}
Mit dem Integralsatz von Gauß erhalten wir nun
\begin{align*}
  \int_{\Omega} \mathrm{div} (\nabla \varphi) dx
  = \int_{\partial\Omega} \nabla \varphi \nu ds
  = \int_{\partial\Omega} \frac{\partial \varphi}{\partial \nu} ds.
\end{align*}
\end{solution}

% --------------------------------------------------------------------------------
