% --------------------------------------------------------------------------------

\begin{exercise}

Zeigen Sie: Die Funktion
\begin{align*}
  u(x,y) = (8\pi)^{-1}(x^2 + y^2)\ln\sqrt{x^2+y^2}, \quad (x,y) \neq (0,0)
\end{align*}
ist eine Fundamentallösung von $\Delta^2$ mit Pol in $(0,0)$ im $\R^2$, wobei
\begin{align*}
  \Delta^2u = \Delta(\Delta u) = \sum_{i,j = 1}^2 u_{x_ix_ix_jx_j}.
\end{align*}

\end{exercise}

% --------------------------------------------------------------------------------

\begin{solution}

Wir zeigen also für beliebiges $\phi \in \mathcal{D}(\R^2)$:

\begin{align*}
  \langle \Delta^2u, \phi \rangle
  :=
  \langle u, \Delta^2 \phi \rangle
  \stackrel{!}{=}
  \phi(0)
\end{align*}

Wir machen einen Ansatz wie im Skript: Sei $\Omega_\varepsilon = \R^2 \setminus \overline{B_\varepsilon(0)}$, dann soll

\begin{align*}
  \phi(0)
  \stackrel{!}{=}
  \lim_{\varepsilon \rightarrow 0}\int_{\Omega_\varepsilon} u \Delta^2 \phi dx
\end{align*}

Nun integrieren wir viermal partiell gemäß dem Satz von Gauß:

\begin{align*}
  \int_{\Omega_\varepsilon} u \Delta^2 \phi dx
  =
  \int_{\Omega_\varepsilon} u \text{div}\underbrace{\nabla \Delta \phi}_F dx
  =
  -\int_{\Omega_\varepsilon} \underbrace{\nabla u}_{F} \cdot \nabla \underbrace{\Delta \phi}_{u} dx
  +
  \int_{\partial \Omega_\varepsilon} u(\nabla \Delta \phi \cdot \nu) ds
  = \\
  \int_{\Omega_\varepsilon} \underbrace{\text{div}\nabla u}_{\Delta u} \underbrace{\Delta \phi}_{\text{div} \nabla \phi} dx
  -
  \int_{\partial \Omega_\varepsilon} \Delta \phi (\nabla u \cdot \nu) ds
  +
  \int_{\partial \Omega_\varepsilon} u(\nabla \Delta \phi \cdot \nu) ds
  = \\
  -\int_{\Omega_\varepsilon} \nabla \Delta u \cdot \nabla \phi dx
  +
  \int_{\partial \Omega_\varepsilon} \Delta u (\nabla \phi \cdot \nu) ds
  -
  \int_{\partial \Omega_\varepsilon} \Delta \phi (\nabla u \cdot \nu) ds
  +
  \int_{\partial \Omega_\varepsilon} u(\nabla \Delta \phi \cdot \nu) ds
  = \\
  \underbrace{\int_{\Omega_\varepsilon} \Delta^2 u \phi dx}_{0}
  -
  \int_{\partial \Omega_\varepsilon} \phi (\nabla \Delta u \cdot \nu) ds
  +
  \int_{\partial \Omega_\varepsilon} \Delta u (\nabla \phi \cdot \nu) ds
  -
  \int_{\partial \Omega_\varepsilon} \Delta \phi (\nabla u \cdot \nu) ds
  +
  \int_{\partial \Omega_\varepsilon} u(\nabla \Delta \phi \cdot \nu) ds
\end{align*}

Die Integrale über den Rand sehen wir uns einzeln an. Dabei ist der Normalenvektor an
$\partial \Omega_\varepsilon$ gegeben durch

\begin{align*}
  \nu(x,y)
  =
  -\frac{1}{\varepsilon}
  \left(\begin{array}{c}
    x \\
    y
  \end{array}\right)
\end{align*}

Für das erste wenden wir den Mittelwertsatz der Integralrechnung an: Es existiert
ein $x_\varepsilon \in \partial B_\varepsilon(0)$, sodass

\begin{align*}
  -\int_{\partial \Omega_\varepsilon} \phi (\nabla \Delta u \cdot \nu) ds
  =
  -\int_{\partial \Omega_\varepsilon} \phi
  \begin{pmatrix} x \\ y \end{pmatrix}
  \frac{1}{2\pi (x^2+y^2)} \cdot
  - \begin{pmatrix} x \\ y \end{pmatrix}
  \frac{1}{\varepsilon} ds
  =
  \phi(x_\varepsilon) \frac{1}{2\pi \varepsilon} \int_{\partial \Omega_\varepsilon} 1 ds
  =
  \phi(x_\varepsilon)
  \stackrel{\varepsilon \rightarrow 0}{\longrightarrow}
  \phi(0)
\end{align*}

Das zweite transformieren wir in Polarkoordinaten, also $x = r\cos(\varphi), y = r\sin(\varphi)$.
\end{solution}

% --------------------------------------------------------------------------------
