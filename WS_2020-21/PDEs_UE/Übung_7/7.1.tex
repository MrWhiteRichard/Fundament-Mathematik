% --------------------------------------------------------------------------------

\begin{exercise}

  Zeigen Sie, dass
  \begin{align*}
      u(x, y) = \ln\left(\ln \frac{1}{\sqrt{x^2 + y^2}}\right) \in H^1(B_{1/2}(0)).
  \end{align*}

\end{exercise}

% --------------------------------------------------------------------------------

\begin{solution}
Zunächst einmal wollen wir zeigen, dass $u(x, y) \in H^1(B_{1/2}(0))$.
\begin{align*}
  \int_{B_{1/2}(0)}u^2 dx = \int_0^{1/2}\int_0^{2\pi}\left(\ln\left(\ln \frac{1}{\sqrt{x^2 + y^2}}\right)\right)^2 r d\varphi dr = 2\pi \int_0^{1/2}\left(\ln\left(\ln \frac{1}{r}\right)\right)^2 r  dr
\end{align*}

Um zu zeigen, dass der Integrand stetig auf $[0,1]$ ist (und somit das Integral existiert), müssen wir noch die Stetigkeit im Punkt $0$ zeigen. Dafür verwenden wir die Regel von de l'Hospital:

\begin{align*}
  & \lim_{r \to 0}\frac{\left(\ln(-\ln r )\right)^2}{\frac{1}{r}} \\
  =& \lim_{r \to 0}\frac{2\left(\ln(-\ln r)\right) \cdot \frac{1}{-\ln r}\cdot (-\frac{1}{r})}{-\frac{1}{r^2}} = \lim_{r \to 0} -2 \cdot \frac{\left(\ln\left(-\ln r \right)\right)}{\frac{1}{r} \cdot \ln r} \\
  =& -2 \cdot \lim_{r \to 0} \frac{\frac{1}{-\ln r} \cdot -\frac{1}{r}}{\frac{1}{r} \cdot \frac{1}{r} - \frac{1}{r^2} \ln r} = -2 \cdot \lim_{r \to 0} \frac{r}{\ln r (1 - \ln r)} = 0
\end{align*}

Wir berechnen nun zuerst naiv die punktweise Ableitung
\begin{align*}
  \frac{\partial}{\partial x} u(x,y) &= -\frac{2}{\ln(x^2+y^2)}
  \frac{\partial}{\partial x}(-\frac{1}{2}\ln(x^2+y^2)) =
  \frac{1}{\ln(x^2+y^2)}\frac{1}{x^2+y^2}\frac{\partial}{\partial x}(x^2+y^2) \\
  &= \frac{2x}{\ln(x^2+y^2)(x^2+y^2)} \\
  \frac{\partial}{\partial y} u(x,y) &= \frac{2y}{\ln(x^2+y^2)(x^2+y^2)} \\
  |\nabla u(x,y)|^2 &= 4\frac{x^2+y^2}{\ln(x^2+y^2)^2(x^2+y^2)^2} = \frac{4}{\ln(x^2+y^2)^2(x^2+y^2)}
\end{align*}
und integrieren sie über $B_{1/2}(0)$
\begin{align*}
  \int_{B_{1/2}(0)}|\nabla u|^2 dx = \int_0^{1/2}\int_0^{2\pi}\frac{4}{r\ln(r^2)^2} d\varphi dr
= 2\pi\int_0^{1/2}\frac{1}{r\ln(r)^2} dr
\stackrel{u = \ln(r)}{=} 8\pi \int_{-\infty}^{-\ln(2)} \frac{1}{u^2} du
  = 2\pi[-\frac{1}{u}]_{-\infty}^{-\ln(2)} = \frac{8\pi}{\ln(2)}.
\end{align*}
Jetzt müssen wir noch nachweisen, dass die distributionelle Ableitung
tatsächlich mit der punktweisen Ableitung übereinstimmt.
Wir berechnen also für $\Omega_\epsilon := B_{1/2}(0)\backslash B_\epsilon(0)$
\begin{align*}
  \int_{B_{1/2}(0)}\frac{2x}{\ln(x^2+y^2)(x^2+y^2)}\phi dxdy &=
  \int_{B_\epsilon(0)}\frac{2x}{\ln(x^2+y^2)(x^2+y^2)}\phi dxdy
  + \int_{\Omega_\epsilon}\frac{2x}{\ln(x^2+y^2)(x^2+y^2)}\phi dxdy \\
  &= \int_{B_\epsilon(0)}\frac{2x}{\ln(x^2+y^2)(x^2+y^2)}\phi dxdy
  - \int_{\Omega_\epsilon}u(x,y)\phi_x dxdy +
  \int_{\partial \Omega_\epsilon}u(x,y)\phi_x ds
\end{align*}
Das Randintegral lässt sich schreiben  als
\begin{align*}
  \int_{\partial \Omega_\epsilon}u(x,y)\phi_x ds
  = \ln(-\ln(\epsilon))\int_{\partial B_\epsilon(0)}\phi(x) ds = \ln(-\ln(\epsilon))\phi(x_\epsilon)\epsilon S_n
  = \phi(x_\epsilon)S_n\frac{\ln(-\ln(\epsilon))}{\frac{1}{\epsilon}}.
\end{align*}
Mit der Regel von l'Hospital erhalten wir
\begin{align*}
  \lim_{\epsilon \to 0}\frac{\ln(-\ln(\epsilon))}{\frac{1}{\epsilon}}
  = \lim_{\epsilon \to 0}\frac{\frac{1}{\epsilon\ln(\epsilon)}}{-\frac{1}{\epsilon^2}}
  = \lim_{\epsilon \to 0}-\frac{\epsilon}{\ln(\epsilon)} = 0.
\end{align*}
Schließlich schätzen wir noch das erste Integral ab
\begin{align*}
  \left|\int_{B_\epsilon(0)}\frac{2x}{\ln(x^2+y^2)(x^2+y^2)}\phi dxdy\right|
  &\leq -\int_{B_\epsilon(0)}\frac{2\sqrt{x^2 +y^2}}{\ln(x^2+y^2)(x^2+y^2)}\phi dxdy
  = -2\int_0^\epsilon\int_0^{2\pi}\frac{r^2}{\ln(r^2)r^2} d\phi dr \\
  &= -4\pi\int_0^\epsilon \frac{1}{2\ln(r)} dr \leq -2\pi \frac{\epsilon}{\ln(\epsilon)}
  \xrightarrow{\epsilon \to 0} 0.
\end{align*}
Insgesamt erhalten wir also
\begin{align*}
\int_{B_{1/2}(0)}\frac{2x}{\ln(x^2+y^2)(x^2+y^2)}\phi dxdy &=
\lim_{\epsilon \to 0}\left(\int_{B_\epsilon(0)}\frac{2x}{\ln(x^2+y^2)(x^2+y^2)}\phi dxdy
+ \int_{\Omega_\epsilon}\frac{2x}{\ln(x^2+y^2)(x^2+y^2)}\phi dxdy\right) \\
&= \lim_{\epsilon \to 0}- \int_{\Omega_\epsilon}u(x,y)\phi_x dxdy
= -\int_{B_{1/2}(0)}u(x,y)\phi_x dxdy
\end{align*}
Damit ist die punktweise Ableitung tatsächlich auch die schwache Ableitung.
\end{solution}

% --------------------------------------------------------------------------------
