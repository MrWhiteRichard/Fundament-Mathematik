% --------------------------------------------------------------------------------

\begin{exercise}

\phantom{}

Seien $T > 0$ und $\Omega \subseteq \R^n$ ein beschränktes Gebiet mit glattem Rand $\partial \Omega$, wobei $G := \Omega \times (0, T]$ und $\Gamma := (\Omega \times \{0\}) \cup (\partial \Omega \times [0, T))$.
Betrachten Sie die Differentialoperatoren

\begin{align*}
  L_1 u
  =
  -\sum_{i, j=1}^n
  a_{ij}(x, t)
  \frac{\partial^2}{\partial x_i \partial x_j}
  +
  \sum_{i=1}^n
  b_i(x, t)
  \frac{\partial u}{\partial x_i}
\end{align*}

und $L_2 u = L_1 u + c(x, t) u$, für eine symmetrische und gleichmäßig elliptische Matrix $A = (a_{ij}(x, t)) \in \R^{n \times n}$ mit $a_{ij} \in C(\overline{G})$, einen Vektor $b = (b_i(x, t)) \in \R^n$ und $c \in C(\overline{G})$.
Zeigen Sie:

\begin{enumerate}[label = (\roman*)]

  \item Für $u \in C_1^2(G) \cap C(\overline{G})$ mit $u_t + L_2u \leq 0$ in $G$ und $u \leq 0$ auf $\Gamma$ gilt $u \leq 0$ in $G$.

  \textbf{Hinweis:}
  \textit
  {
    Beachten Sie, dass $c$ negative Werte annehmen darf.
    Welche Differentialgleichung erfüllt $v = \exp(\lambda t) u$?
  }
  
  \item Für $u, v \in C_1^2(G) \cap C(\overline{G})$ und eine stetige differenzierbare Funktion $f = f(x, t, u)$ mit

  \begin{align*}
    u_t + L_1 u + f(x, t, u)
    \leq
    v_t + L_1 v + f(x, t, v)
    \quad
    \text{in}~ G ~\text{und}~ u \leq v ~\text{auf}~ \Gamma
  \end{align*}

  gilt $u \leq v$ in $G$.
  
  \item Für eine stetige differenzierbare Funktion $f = f(x, t, u)$ gilt, dass das Anfangsrandwertproblem für die Differentialgleichung 

  \begin{align*}
	  \begin{cases}
	  	u_t + L_1 u + f(x, t, u) = 0 & ~\text{in}~ G, \\
	  	u(\cdot, 0) = u_0            & ~\text{in}~ \Omega, \\
	  	u = g                        & ~\text{auf}~ \partial \Omega \times (0, T),
	  \end{cases}
  \end{align*}

  höchstens eine klassische Lösung haben kann. 

\end{enumerate}

\end{exercise}

% --------------------------------------------------------------------------------

\begin{solution}

\phantom{}

\begin{enumerate}[label = (\roman*)]

  \item Man erinnere sich an die Definition vom
  
  \begin{align*}
    C_1^2(G)
    :=
    \Bbraces
    {
      u:
      G \to \R:
      u, u_t, u_{x_i}, u_{x_i x_j} \in C^0(G)
      \Forall i, j
    }.
  \end{align*}

  Wir wollen Satz 6.31 (Schwaches Maximumprinzip für $c \geq 0$) auf $v$ anwenden.

  \begin{tcolorbox}[standard jigsaw, opacityback = 0]
    \centering
    \includegraphics
    [width = 0.75 \textwidth]
    {PDEs/PDEs_-_Satz_6-31-1_(Schwaches_Maximumprinzip_für_c_geq_0).png} \\
    \includegraphics
    [width = 0.75 \textwidth]
    {PDEs/PDEs_-_Satz_6-31-2_(Schwaches_Maximumprinzip_für_c_geq_0).png}
  \end{tcolorbox}

  \begin{multline*}
    v_t + L_1 v
    =
    (\exp(\lambda t) u)_t + L_1 (\exp(\lambda t) u)
    =
    \lambda \exp(\lambda t) u + \exp(\lambda t) u_t
    +
    \exp(\lambda t) L_1 u \\
    =
    \lambda \exp(\lambda t) u + \exp(\lambda t) u_t
    +
    \exp(\lambda t) (L_2 u - c(x, t) u)
    =
    \exp(\lambda t) (u_t +  L_2 u)
    -
    \exp(\lambda t) (c(x, t) - \lambda) u
  \end{multline*}

  Wähle also $\lambda < -\norm[\infty]{c}$.

  \begin{align*}
    \implies
    \tilde{c}
    :=
    \exp(\lambda t)
    \underbrace
    {
      (c(x, t) - \lambda)
    }_{
      > 0
    }
    >
    0
    \implies
    v_t + L_1 v + \tilde{c} u
    =
    \exp(\lambda t)
    \underbrace
    {
      (u_t +  L_2 u)
    }_{
      \leq 0
    }
    \leq
    0
  \end{align*}

  Nun können wir Satz 6.31 (Schwaches Maximumprinzip für $c \geq 0$) auf $v$ und $\tilde{L} := L_1 + \tilde{c}$ anwenden.

  \begin{align*}
    \implies
    \sup_{(x, t) \in G} v(x, t)
    \stackrel{\mathrm{MP}}{\leq}
    \max \Bbraces{0, \sup_{(x, t) \in \Gamma} v(x,t)}
    =
    0
  \end{align*}

  Nachdem $u$ und $v$ dasselbe Vorzeichen haben, folgt die Behauptung.

  \item Betrachte die Funktion $w := u - v \in C_1^2(G) \cap C(\overline{G})$ und erhalten mit dem Mittelwertsatz der Differentialrechnung, dass $\Forall (x, t) \in G: \Exists \xi_{x, t} \in [\min \Bbraces{u(x, t), v(x, t)}, \max \Bbraces{u(x, t), v(x, t)}]:$

  \begin{align*}
    w_t
    =
    u_t - v_t
    \leq
    L_1 v - L_1 u + f(x, t, v) - f(x, t, u)
    \stackrel
    {
      \mathrm{MWS}
    }{=}
    L_1 \underbrace{(v - u)}_{-w} + \partial_u f(x, t, \xi_{x, t}) \underbrace{(v - u)}_{-w}.
  \end{align*}

  \begin{align*}
    c(x, t) := \partial_u f(x, t,\xi_{x, t}),
    \quad
    L_2 := L_1 + c
    & \implies
    w_t + L_2 w \leq 0 ~\text{in}~ G, \\
    u \leq v ~\text{auf}~ \Gamma
    & \implies
    w = u - v \leq 0 ~\text{auf}~ \Gamma
  \end{align*}

  Hier wurde noch das Problem erkannt, dass $c$ möglicherweise nicht stetig ist.
  Wenn wir das behoben haben, dann gilt nach Punkt (i), dass

  \begin{align*}
    \implies
    u - v = w \leq 0 ~\text{in}~ G
    \implies
    u \leq v ~\text{in}~ G.
  \end{align*}

  \item Wenn wir zwei Lösungen $u, v$ des Problems betrachten, so gilt 

  \begin{align*}
    \implies &
    \begin{cases}
      u_t + L_1 u + f(x, t, u) = v_t + L_1 v + f(x, t, v), & \text{in}~ G, \\
      u = u_0 = v,                                         & \text{in}~ \Omega, \\
      u = g = v,                                           & \text{auf}~ \partial \Omega \times (0, T),
    \end{cases} \\
    \implies &
    u = v ~\text{auf}~ \Gamma.
  \end{align*}

  Mit Punkt (ii) folgt $u \leq v$ und $v \leq u$ also $u = v$. 

\end{enumerate}

\end{solution}

% --------------------------------------------------------------------------------
