% --------------------------------------------------------------------------------

\begin{exercise}

Seien $\Omega \subset \R^n$ ein beschränktes Gebiet, $g, u_0 \in L^2(\Omega)$ und
$\alpha > 0$. Zeigen Sie, dass das semilineare Problem
\begin{align*}
  \begin{cases}
    u_t - \Delta(u^\alpha) = g(x) & \text{in } \Omega \times (0,\infty), \\
    u = 0 & \text{auf } \partial\Omega \times (0,\infty), \\
    u(\cdot,0) = u_0 & \text{in } \Omega,
  \end{cases}
\end{align*}
höchstens eine nichtnegative Lösung $u \in C^2(\overline{\Omega} \times (0,\infty))$
besitzt. \\
\textit{Hinweis:} Betrachten Sie zwei Lösungen $u$ und $v$ des obigen ARWP
und benutzen Sie die Lösung $w(t)$ von
\begin{align*}
  \Delta w(t) = u(t) - v(t) \quad \text{in } \Omega, \quad
  w(t) = 0 \quad \text{auf } \partial\Omega
\end{align*}
als Testfunktion.
\end{exercise}

% --------------------------------------------------------------------------------

\begin{solution}

\phantom{}
Seien $u$ und $v$ zwei nichtnegative Lösungen. Nach Satz 5.20 hat das Poisson-Problem
\begin{align}\label{poisson}
\begin{cases}
\Delta w(t) = u(t) - v(t) &\text{in~} \Omega,\\
w(t) = 0 &\text{auf~} \partial\Omega
\end{cases}
\end{align}
eine eindeutige schwache Lösung $w(t) \in H_0^1(\Omega),$ das heißt für alle $\varphi \in H_0^1(\Omega)$ gilt
\begin{align}\label{schwach}
    \int_\Omega (u-v)(t) \varphi \mathrm{~d}x = -\int_\Omega \nabla w(t) \cdot \nabla \varphi \mathrm{~d}x.
\end{align}

Wir können $w$ nun als Testfunktion für die schwache Formulierung unseres ARW-Problems verwenden: Es gilt
\begin{align*}
    \int_0^t \int_\Omega gw \mathrm{~d}x \mathrm{~d}s
    &= \int_0^t \int_\Omega \left(u_t - \Delta(u^\alpha)\right) w \mathrm{~d}x \mathrm{~d}s \\
    &= \int_0^t \left(\int_\Omega u_t w \mathrm{~d}x - \int_\Omega \mathrm{div}\nabla(u^\alpha) w \mathrm{~d}x\right) \mathrm{d}s\\
    &= \int_0^t \left(\int_\Omega u_t w \mathrm{~d}x + \int_\Omega \nabla(u^\alpha) \cdot \nabla w \mathrm{~d}x - \int_{\partial\Omega} \underbrace{w}_{=~0}(\nabla(u^\alpha)\cdot\nu) \mathrm{~d}\mathcal{H}^{n-1}\right) \mathrm{d}s\\
    &= \int_0^t \int_\Omega u_t w + \nabla(u^\alpha) \cdot \nabla w \mathrm{~d}x \mathrm{~d}s.
\end{align*}

Dieselbe Gleichheit gilt natürlich auch für $v.$ Wenn wir beide Gleichungen voneinander subtrahieren, erhalten wir
\begin{align*}
    0 &= \int_0^t \int_\Omega u_t w + \nabla(u^\alpha) \cdot \nabla w \mathrm{~d}x \mathrm{~d}s - \int_0^t \int_\Omega v_t w + \nabla(v^\alpha) \cdot \nabla w \mathrm{~d}x \mathrm{~d}s\\
    &= \int_0^t \left(\int_\Omega (u-v)_t w \mathrm{~d}x - \int_\Omega \nabla(u^\alpha - v^\alpha) \cdot \nabla w \mathrm{~d}x\right) \mathrm{d}s\\
    %&= \int_0^t \left(- \int_\Omega \nabla w_t \cdot \nabla w \mathrm{~d}x - \int_\Omega (u^\alpha - v^\alpha) \Delta w \mathrm{~d}x\right) \mathrm{d}s\\
    &= - \int_0^t \left(\int_\Omega \nabla w_t \cdot \nabla w \mathrm{~d}x - \int_\Omega (u^\alpha - v^\alpha) (u-v) \mathrm{~d}x\right)\mathrm{d}s.
\end{align*}

Das dabei auftretende Randintegral fällt weg, weil $w$ auf $\partial\Omega$ verschwindet. Weiters haben wir \eqref{schwach} auf $\varphi = u^\alpha - v^\alpha$ angewandt.

Weil für $\alpha > 0$ die Abbildung $x \mapsto x^\alpha$ monoton steigend ist, gilt stets $(u^\alpha - v^\alpha) (u-v) \geq 0.$ Des weiteren ist $w \equiv 0$ für $t = 0$ die eindeutig bestimmte schwache Lösung von \eqref{poisson}, womit $\nabla w(\cdot, 0) = 0$ ist. Nun gilt
\begin{align*}
    0 &\geq - \int_0^t \int_\Omega (u^\alpha - v^\alpha) (u-v) \mathrm{~d}x \mathrm{~d}s\\
    &= \int_0^t \int_\Omega \nabla w_t \cdot \nabla w \mathrm{~d}x \mathrm{~d}s\\
    &\stackrel{(\ast)}{=} \int_0^t \frac{1}{2} ~\frac{\mathrm{d}}{\mathrm{d}s} \left\| \nabla w(\cdot, s)\right\|_{L^2} \mathrm{d}s\\
    &= \frac{1}{2} \left\| \nabla w(\cdot, t)\right\|_{L^2} - \frac{1}{2} \| \underbrace{\nabla w(\cdot, 0)}_{=0}\|_{L^2}\\
    &= \frac{1}{2} \left\| \nabla w(\cdot, t)\right\|_{L^2};
\end{align*}

$w(\cdot, t)$ ist also für alle $t > 0$ eine konstante Funktion. Damit gilt $0 = \Delta w(t) = u(t) - v(t),$ was zu beweisen war.

Zeigen wir noch die Gleichheit ($\ast$): Für ein $u \in C^\infty(\Omega)$ ist
\begin{align*}
    \Int[t_0][t]{u_t u}{x}
    & \stackrel
    {
      \mathrm{PI}
    }{=}
    u^2(x, t) - u^2(x, t_0) - \Int[t_0][t]{u_t u}{x} \\
    \implies
    \|u(t_0)\|^2_{L^2} + 2 \int_{t_0}^{t} (u_t(s), u(s))_{L^2} \mathrm{~d}s
    &= \|u(t_0)\|^2_{L^2} + \int_\Omega 2 \int_{t_0}^{t} u_t u \mathrm{~d}s \mathrm{~d}x\\
    &= \|u(t_0)\|^2_{L^2} + \int_\Omega u^2(x, t) - u^2(x, t_0) \mathrm{~d}x\\
    &= \| u(t)\|_{L^2}.
\end{align*}
Wenn wir nun beide Seiten nach $t$ differenzieren, erhalten wir
\begin{align*}
    \frac{\mathrm{d}}{\mathrm{d}t} \|u(t)\|_{L^2} = 2 \left(u_t(t), u(t)\right)_{L^2}.
\end{align*}
Aus der Dichtheit von $C^\infty(\Omega)$ in $L^2(\Omega)$ folgt die gewünschte Gleichheit für alle $u \in L^2(\Omega),$ insbesondere also für $\nabla w(\cdot, s).$
\end{solution}

% --------------------------------------------------------------------------------
