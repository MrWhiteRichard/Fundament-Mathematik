% --------------------------------------------------------------------------------

\begin{exercise}

Sei $\Omega \subset \R^n$ ein beschränktes Gebiet mit $\partial \Omega \in C^1$ und sei $u$ eine klassische Lösung des parabolischen Problems

\begin{align*}
  \begin{cases}
    u_t - \Delta u + c u = 0 & \text{in}~ \Omega \times (0, \infty), \\
    u = 0                    & \text{auf}~ \partial \Omega \times (0, \infty), \\
    u(\cdot, 0) = u_0        & \text{in}~ \Omega,
  \end{cases}
\end{align*}

wobei $u_0 \geq 0$ in $\Omega$ und $c \in L^\infty(\Omega)$.
Zeigen Sie:
$u \geq 0$ in $\Omega \times (0, \infty)$.
Achtung:
$c$ ist also nicht notwendig nichtnegativ!

\textit{Hinweis:}
Welche Differentialgleichung löst $v = \exp(\lambda t) u$?

\end{exercise}

% --------------------------------------------------------------------------------

\begin{solution}

Seien $T > 0$, $G_T = \Omega \times (0, T)$ ein beschränkter Zylinder und $\Gamma_T$ dessen Mantel $\cup$ Grundfläche.

Gemäß Hinweis wählen wir die Substitution $v := \exp(\lambda t)$.
Weil $u$ eine klassische Lösung ist, gilt $\Delta u = u_t + c u$ in $\Omega \times (0, \infty)$.

\begin{align*}
  \implies
  v_t
  & =
  (\exp(\lambda t) u)_t
  =
  \exp (\lambda t) u_t
  +
  \lambda \exp(\lambda t) u
  =
  \exp (\lambda t) u_t + \lambda v, \\
  \Delta v
  & =
  \Delta (\exp(\lambda t) u)
  =
  \exp(\lambda t) \Delta u
  =
  \exp(\lambda t) (u_t + c u)
  =
  \exp(\lambda t) u_t + \exp(\lambda t) c u
  =
  \exp(\lambda t) u_t + c v
\end{align*}

Fügen wir beides zusammen erhalten wir folgende PDE.
(Die RWBs und AWBs interessieren uns dabei nicht.)

\begin{align*}
  v_t - \Delta v + (c - \lambda) v = 0
\end{align*}

Darauf wollen wir Satz 6.31 (Schwaches Maximumprinzip für $c \geq 0$) anwenden.

\begin{tcolorbox}[standard jigsaw, opacityback = 0]
  \centering
  \includegraphics
  [width = 0.75 \textwidth]
  {PDEs/PDEs_-_Satz_6-31-1_(Schwaches_Maximumprinzip_für_c_geq_0).png} \\
  \includegraphics
  [width = 0.75 \textwidth]
  {PDEs/PDEs_-_Satz_6-31-2_(Schwaches_Maximumprinzip_für_c_geq_0).png}
\end{tcolorbox}

Wählen wir $\lambda := - \norm[L^\infty(\Omega)]{c}$, so ist $(c - \lambda) \geq 0$.
Da $u$ klassische Lösung ist, muss $v \in C_{1}^{2}(G_T) \cap C^0(\overline{G_T})$.
Wir können also tatsächlich das Schwache Minimumsprinzip auf $v$ anwenden.

\begin{align*}
  \implies
  & v = \exp(\lambda t) 0 = 0 ~\text{auf Mantel von}~ G_T, \\
  & v = \exp(\lambda t) u_0 \geq 0 ~\text{auf Boden von}~ G_T \\
  \implies
  & v \geq 0 ~\text{auf}~ \Gamma_T \\
  \stackrel{\mathrm{MP}}{\implies}
  & v \geq \inf_{(x, t) \in G_T} v \geq \min \Bbraces{0, \inf_{(x, t) \in \Gamma_T}} = 0 ~\text{auf}~ G_T \\
  \implies
  & u = \exp(-\lambda t) v \geq 0 ~\text{auf}~ G_T
\end{align*}

$T$ kann nun beliebig groß sein.

\begin{align*}
  \implies u \geq 0 ~\text{auf}~ \Omega \times (0, \infty)
\end{align*}

\end{solution}

% --------------------------------------------------------------------------------
