% --------------------------------------------------------------------------------

\begin{exercise}

\phantom{}
	Betrachten Sie die skalare Reaktions-Diffusionsgleichung
	\begin{align*}
		u_t = \Delta u + \lambda u - u^3 \quad \text{für } (x,t) \in \Omega \times (0, \infty)
	\end{align*}
	für $u(x,t) \in \R$ auf einem beschränkten Gebiet $\Omega \subseteq \R^n$ mit glattem Rand $\partial \Omega$ und einem negativen Parameter $\lambda$.
	\begin{enumerate}[label = (\roman*)]
		\item Bestimmen Sie die räumlich homogenen Lösungen $u = u(t)$ und untersuchen Sie deren asymptotisches Verhalten für $t \to \infty$. \newline
		\textit{Hinweis:} Die räumlich homogenen Lösungen erfüllen eine gewöhnliche DGl. Bestimmen Sie die Stationärzustände dieser DGl und deren Stabilität.

		\item Betrachten Sie das ARWP mit der Randbedingung
		\begin{align*}
			u(x,t) = 0 \quad \text{für } (x,t) \in \partial \Omega \times (0, \infty),
		\end{align*}
		und beschränkten Anfangsdaten
		\begin{align*}
			m \leq u(x,0) \leq M \quad \text{für }  x \in \Omega, t \geq 0,
		\end{align*}
		Zeigen Sie, dass klassische Lösungen $u(x,t)$ des ARWP und die räumlich homogenen Lösungen $\underline{u}(t)$ bzw. $\overline{u}(t)$ von dem ARWP mit Anfangsbedingungen $\underline{u}(0) = \min\{0, m\}$ bzw. $\overline{u}(0) = \max\{0, M\}$ die Ungleichung
		\begin{align*}
			\underline{u}(t) \leq u(x,t) \leq \overline{u}(t) \quad \text{für } x \in \Omega, t \geq 0,
		\end{align*}
		erfüllen.

		\item Was können Sie aus diesen Ungleichungen für das zeitlich asymptotische Verhalten von klassischen Lösungen $u(x,t)$ des ARWP schließen?
	\end{enumerate}
\end{exercise}

% --------------------------------------------------------------------------------

\begin{solution}

\phantom{}
	\begin{enumerate}[label = (\roman*)]
		\item Für eine räumlich homogene Lösung $u$ gilt $\Delta u = 0$ also
		\begin{align*}
			u_t = \lambda u - u^3.
		\end{align*}
		Diese ODE ist separabel, um eine Lösung zu erhalten berechnen wir also eine Stammfunktion
		\begin{align*}
			\Int{\frac{1}{u(\lambda - u^2)}}{u} &\stackrel{u^2 = v}{=} \Int{\frac{1}{2 v(\lambda - v)}}{v} = \frac{1}{2} \pbraces{\Int{\frac{1}{\lambda v}}{v} + \Int{\frac{1}{\lambda (\lambda - v)}}{v}} \\
			 &= \frac{1}{2\lambda} \pbraces{\ln(v) - \ln(v - \lambda)} = \frac{1}{2 \lambda} \ln\pbraces{\frac{u^2}{u^2 - \lambda}}
		\end{align*}
		Nun lösen wir
		\begin{align*}
			\frac{1}{2 \lambda} \ln\pbraces{\frac{u^2}{u^2 - \lambda}} = t + \tilde{C}
		\end{align*}
		nach $u$ auf und erhalten
		\begin{align*}
			u(t) = \pm \frac{\sqrt{- \lambda} \exp(\lambda t)}{\sqrt{\exp(2\lambda t) - C^{-1}}}
		\end{align*}
		mit $C := \exp (2 \lambda \tilde{C}) > 0$. Sind das alle Lösungen? \newline
		Wir fragen uns auch welche Ruhelagen das System hat und erkennen
		\begin{align*}
			\lambda u - u^3 = - u (u - \sqrt{\lambda}) (u + \sqrt{\lambda}).
		\end{align*}
		Da $\lambda < 0$, also $\sqrt{\lambda} \in \C \setminus \R$, gibt es also nur die Ruhelage Null.
		
		\includegraphicsboxed{ODEs/ODEs - Satz 5.8.png}

		Wegen
    dem Prinzip der linearisierten Stabilität und
    \begin{align*}
			\partial_u (\lambda u - u^3)\mid_{u = 0} = (\lambda - 3 u^2)\mid_{u = 0} = \lambda < 0
		\end{align*}
		ist diese Ruhelage sogar asymptotisch stabil. Es gilt sogar ($\lambda < 0$)
    \begin{align*}
	  \lim_{t \to \infty} u(t)
	  =
	  \pm \lim_{t \to \infty} \frac{\sqrt{- \lambda} \exp(\lambda t)}{\sqrt{\exp(2\lambda t) - C^{-1}}} = 0.
    \end{align*}
    Somit konvergiert die homogene Lösung für alle Startwerte gegen die
    asymptotische Ruhelage.
		\item Für ein beliebiges $T > 0$ gilt
		\begin{align*}
			\underline{u}_t - \Delta \underline{u} - \lambda \underline{u} + \underline{u}^3 = 0 = u_t - \Delta u - \lambda u + u^3 = 0 = \overline{u}_t - \Delta \overline{u} - \lambda \overline{u} + \overline{u}^3 \quad \text{in } \Omega \times (0, T].
		\end{align*}
		Außerdem gilt
		\begin{align*}
			\underline{u}(0) = \min\{0, m\} \leq m \leq u(x, 0) \leq M \leq \max\{0, M\} \leq \overline{u}(0) \quad \text{für jedes } x \in \Omega.
		\end{align*}
		Schließlich gilt
		\begin{align*}
			\underline{u}(0) = \min\{0, m\} \leq 0 \leq \max\{0, M\} = \overline{u}(0)
		\end{align*}
		$0$ ist eine Ruhelage entsprechender ODE also auch eine Lösung.
		Da sich die Lösungen ($\underline{u}$, $0$ und $\overline{u}$) nicht schneiden dürfen gilt schon
		\begin{align*}
			\underline{u}(t) \leq 0 = u(x, t) = 0 \leq \overline{u}(t) \quad \text{für jedes } (x,t) \in \partial\Omega \times [0, T)
		\end{align*}
		Nach der vorherigen Aufgabe 1 (ii) gilt ($L_1 := -\Delta$ und $f(x, t, u) := u^3 - \lambda u$)
		\begin{align*}
			\underline{u}(t) \leq u(x,t) \leq \overline{u}(t) \quad \text{für jedes } (x,t) \in \Omega \times (0, T]
		\end{align*}
		und da $T > 0$ beliebig war folgt die Aussage.
		\item Wir erkennen, dass eine klassische Lösung $u$, wegen des Einschluss-Satzes und der letzten Ungleichung aus (ii), für $t \to \infty$ gegen $0$ konvergiert.
	\end{enumerate}

\end{solution}

% --------------------------------------------------------------------------------
