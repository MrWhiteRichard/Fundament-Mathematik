% --------------------------------------------------------------------------------

\begin{exercise}

Sei $u(x,t)$ eine beschränkte Lösung des Cauchyproblems für die Wärmeleitungsgleichung
\begin{align*}
  \begin{cases}
    u_t = a^2u_{xx} & \text{für } t > 0 \text{ und } x \in \R \\
    u(x,0) = \varphi(x) & \text{für } x \in \R,
  \end{cases}
\end{align*}
wobei $a > 0$ ist und $\varphi \in C(\R)$
\begin{align*}
  \lim_{x \to \infty} \varphi(x) = b, \quad \lim_{x \to -\infty} \varphi(x) = c
\end{align*}
erfüllt. Berechnen Sie den Grenzwert von $u(x,t),\ x \in \R$ für $t \to \infty$.
\end{exercise}

% --------------------------------------------------------------------------------

\begin{solution}

Sei $u$ eine Lösung der Wärmeleitungsgleichung mit $u(\cdot,t), u_t(\cdot,t) \in L^1(\R)$. \\
Wir gehen analog zum Skript vor und erhalten
\begin{align*}
0 = \mathscr{F}(u_t - a^2u_{xx}) = \hat{u_t} + a^2|k|^2\hat{u}
\end{align*}
Die Lösung dieser homogenen Differentialgleichung mit Anfangsbedingung $\hat{u}(k,0) = \hat{\varphi}(k)$ lautet
\begin{align*}
  \hat{u}(k,t) = \hat{\varphi}(k)\exp(-(ak)^2t)
\end{align*}
Mit der Inversionsformel erhalten wir mit $w = \mathscr{F}^{-1}(\exp(-(ak)^2t))$
\begin{align*}
  u(x,t) &= \mathscr{F}^{-1}(\hat{u})(x,t) = \frac{1}{2\pi}\int_\R \hat{u}(k,t)\exp(ikx)dk
  = \frac{1}{2\pi}\int_\R \hat{\varphi}(k)\exp(-(ak)^2t)\exp(ikx)dk \\
  &= \frac{1}{2\pi}\int_\R \hat{\varphi}(k)\hat{w}(k,t)\exp(ikx)dk
  = \frac{1}{2\pi}\int_\R \widehat{\varphi \ast w}(k,t)\exp(ikx)dk
  = \widehat{\varphi \ast w}(x,t).
\end{align*}
Nun berechnen wir $w$: Mit
\begin{align*}
  \exp\left(-\frac{x^2}{2}\right) = \mathscr{F}^{-1}\left(\sqrt{2\pi}\exp\left(-\frac{y^2}{2}\right)\right)
  = \frac{1}{2\pi}\int_\R \left(\sqrt{2\pi}\exp\left(-\frac{y^2}{2}\right)\right)\exp(iyx) dy
\end{align*}

erhalten wir
\begin{align*}
  w(x,t) &= \frac{1}{2\pi}\int_\R \exp(-(ak)^2t)\exp(ikx) dk
  = \frac{1}{2\pi}\frac{1}{\sqrt{2t}a}\int_\R \exp\left(\frac{-y^2}{2}\right)
  \exp\left(\frac{iyx}{\sqrt{2t}a}\right) dy \\
  &= \frac{1}{2\pi}\frac{1}{\sqrt{4\pi t}a}\int_\R \sqrt{2\pi}\exp\left(\frac{-y^2}{2}\right)
  \exp\left(\frac{iyx}{\sqrt{2t}a}\right) dy
  = \frac{1}{\sqrt{4\pi} \sqrt{t}a}\exp\left(\frac{-k^2}{2}\right)|_{k = x/(a \sqrt{2t})} \\
  &= \frac{1}{\sqrt{4\pi} \sqrt{t}a}\exp\left(\frac{-x^2}{4a^2 t}\right)
\end{align*}
Also erhalten wir (mit der Substitution $z = \frac{y}{a \sqrt{2t}}$)
\begin{align*}
  u(x,t) &= \varphi \ast w (x,t) =
  \frac{1}{\sqrt{4\pi} \sqrt{t}a} \int_\R \exp\left(\frac{-y^2}{4a^2 t}\right)\varphi(x-y) dy \\
  &= \frac{a \sqrt{2t}}{\sqrt{4\pi} \sqrt{t}a} \int_\R \exp\left(\frac{-z^2}{2}\right)\varphi(x-z(a \sqrt{2t})) dz \\
  &= \frac{\sqrt{2}}{\sqrt{4\pi}} \left(\int_{\R^{+}} \exp\left(\frac{-z^2}{2}\right)\varphi(x-z(a \sqrt{2t})) dz + \int_{\R^{-}} \exp\left(\frac{-z^2}{2}\right)\varphi(x-z(a \sqrt{2t})) dz \right)
\end{align*}
Nun können wir den Grenzwert ausrechnen (mithilfe Majorisierter Konvergenz): \\
Da $\phi \in C(\R)$ und die Grenzwerte im Unendlichen endlich sind, ist auch
$M := \sup_{x\in\R}|\phi(x)| < \infty$ und $g(z) := M\exp\left(\frac{-z^2}{2}\right)$
eine integrierbare Majorante.
\begin{align*}
  \lim_{t \to \infty} u(x,t) &= \lim_{t \to \infty}
  \frac{\sqrt{2}}{\sqrt{4\pi}} \left(\int_{\R^{+}} \exp\left(\frac{-z^2}{2}\right)\varphi(x-z(a \sqrt{2t})) dz + \int_{\R^{-}} \exp\left(\frac{-z^2}{2}\right)\varphi(x-z(a \sqrt{2t})) dz \right) \\
  &= \frac{\sqrt{2}}{\sqrt{4\pi}} \left(c\int_{\R^{+}} \exp\left(\frac{-z^2}{2}\right) dz + b\int_{\R^{-}} \exp\left(\frac{-z^2}{2}\right) dz \right) \\
  &= \sqrt{\frac{\pi}{2}} \frac{\sqrt{2}}{\sqrt{4\pi}} (b+c) = \frac{1}{2} (b+c)
\end{align*}
\end{solution}

% --------------------------------------------------------------------------------
