% --------------------------------------------------------------------------------

\begin{exercise}

Betrachten Sie das Anfangswertproblem für die \textit{freie Schrödingergleichung}
\begin{align*}
  \begin{cases}
  i\frac{\partial u}{\partial t} = -\Delta u & \text{in } \Omega = \R^n \times [0,\infty), \\
  u(x,0) = f(x) & \text{für } x \in \R^n,
  \end{cases}
\end{align*}
wobei $f \in L^1(\R^n) \cap L^2(\R^n)$ und $i$ die imaginäre Einheit ist.
\begin{enumerate}[label = (\roman*)]
  \item Bestimmen Sie für eine Lösung $u$ des AWP mit $u(t) \in L^1(\R^n)$
  für alle $t \geq 0$ eine Darstellung mittels Fourier-Transformation.
  \item Zeigen Sie, dass $\|u(\cdot,t)\|_{L^2(\Omega)} = \|f\|_{L^2(\Omega)}$ für $t > 0$.
\end{enumerate}

\end{exercise}

% --------------------------------------------------------------------------------

\begin{solution}
	\phantom{}
	\begin{enumerate}[label = (\roman*)]
		\item Wir betrachten das Fouriertransformierte Anfangswertproblem
			\begin{align*}
			i \widehat{u}_t = \widehat{- \Delta u} = - \sum_{i = 1}^n \widehat{\pderivative[2][]{x_i}u} = \sum_{i = 1}^n x_i^2 \widehat{u} = |x|^2 \widehat{u} \\
			\widehat{u}(k,0) = \widehat{f}(x).
			\end{align*}
			Eine Lösung davon ist gegeben durch
			\begin{align*}
			\widehat{u}(k,t) = \widehat{f}(k) \mathrm{e}^{-i|k|^2t}
			\end{align*}
			und mit der Definition $\widehat{w}(k,t) := \mathrm{e}^{-i|k|^2t}$ erhalten wir
			\begin{align*}
			u(x,t) = (2\pi)^{-n} \Int[\R][]{\widehat{f}(k) \widehat{w}(k,t) \mathrm{e}^{i x \cdot k}}{k} = (2\pi)^{-n} \Int[\R][]{\widehat{f \ast w}(k, t) \mathrm{e}^{i x \cdot k}}{k} = (f \ast w)(x,t)
			\end{align*}
			Nun ist die Frage, was ist $w$?
		\item
	\end{enumerate}

\end{solution}

% --------------------------------------------------------------------------------
