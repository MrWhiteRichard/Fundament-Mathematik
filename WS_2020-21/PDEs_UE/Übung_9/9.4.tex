% --------------------------------------------------------------------------------

\begin{exercise}
Betrachten Sie die eindimensionale Wärmeleitungsgleichung mit gemischten Randbedingungen
\begin{align*}
  \begin{cases}
    u_t - u_{xx} = 0 & \text{für } x \in (0,\pi),\ t > 0, \\
    u(0,t) = 0 & \text{für } t > 0, \\
    u_x(\pi,t) = 0 & \text{für } t > 0, \\
    u(x,0) = u_0(x) & \text{für } x \in (0,\pi),
  \end{cases}
\end{align*}
wobei $u_0 \in L^2(0,\pi)$.
\begin{enumerate}[label = (\roman*)]
  \item Bestimmen Sie ein vollständiges Orthonormalsystem $(\phi_n)_{n \in \N} \subset L^2(0,\pi)$
  mit $\phi^{\primeprime}_n = \lambda_n\phi_n$ in $(0,\pi)$ mit Randbedingungen
  $\phi_n(0) = \phi_n^{\prime}(\pi) = 0.$
  \item Konstruieren Sie aus $(\phi_n)_{n \in \N}$ eine Lösungsformel für das
  obige parabolische Problem.
  \item Welche Abklingrate (für $t \to \infty$) hat die Wärmeenergie
  $E(t) := \int_0^{\pi} u(x,t) dx$ für eine Lösung $u$?
\end{enumerate}



\end{exercise}

% --------------------------------------------------------------------------------

\begin{solution}

\phantom{}
\begin{enumerate}
  \item Betrachte die allgemeine Lösung für $\phi_n$:
  \begin{align*}
    \phi_n(x) = C_1\exp(\sqrt{-\lambda_n}x) + C_2\exp(-\sqrt{-\lambda_n}x)
  \end{align*}
  Durch Einsetzen in die Randbedingung bei $x = 0$ erhalten wir:
  \begin{align*}
    0 \stackrel{!}{=} C_1 + C_2 \implies C_2 = -C_1.
  \end{align*}
  Für die zweite Randbedingung wählen wir zur Vereinfachung $\lambda_n = n^2, n \in \N$
  und erhalten
  \begin{align}
    0 \stackrel{!}{=} \phi^{\prime}_n(\pi) = [C(\exp(inx) - \exp(-inx)]^{\prime}
    = Cn(\underbrace{-\sin(nx) + \sin(-nx)}_{=0} + i\underbrace{(\cos(nx) - \cos(-nx))}_{=0}) = 0.
  \end{align}
  Damit lautet unser vollständiges Orthonormalsystem
  \begin{align*}
    \phi_n(x) = \frac{\exp(inx) - \exp(-inx)}{\|\exp(inx) - \exp(-inx)\|_{L^2(0,\pi)}}
    = \frac{2i\sin(nx)}{\sqrt{2\pi}}
  \end{align*}
  \item Definiere
  \begin{align*}
    u(x,t) := \sum_{k=1}^{\infty}\exp(-k^2t)(u_0,\phi_k)_{L^2(0,\pi)}\phi_k(x)
  \end{align*}
  Damit gilt
  \begin{align*}
    u(0,t) &= \sum_{k=1}^{\infty}\exp(-k^2t)(u_0,\phi_k)_{L^2(0,\pi)}\phi_k(0) = 0. \\
    u(x,0) &= \sum_{k=1}^{\infty}(u_0,\phi_k)_{L^2(0,\pi)}\phi_k(x) = u_0(x). \\
    u_x(\pi,t) &= \partial_x\left(\sum_{k=1}^{\infty}\exp(-k^2t)(u_0,\phi_k)_{L^2(0,\pi)}\phi_k(x)\right)\Bigg|_{x=\pi} \stackrel{!}{=} 0.
  \end{align*}
  Für die dritte Gleichung müssen wir also zeigen, dass
  \begin{align*}
    0 &= \lim_{h \to 0^+}\left\|\frac{1}{h}(u(\pi,t) - u(\pi - h,t))\right\|_{L^2(0,\pi)}^2
    = \frac{1}{h^2}\int_0^{\pi}(u(\pi,t) -  u(\pi -h,t))^2 dt \\
    &= \frac{1}{h^2}\int_0^{\pi}\left(\sum_{k=1}^{\infty}\exp(-k^2t)(u_0,\phi_k)_{L^2(0,\pi)}\phi_k(\pi) -  \sum_{k=1}^{\infty}\exp(-k^2t)(u_0,\phi_k)_{L^2(0,\pi)}\phi_k(\pi - h)\right)^2 dt \\
    &= \frac{i}{2\pi h^2}\int_0^{\pi}\left(\sum_{k=1}^{\infty}
    \exp(-k^2t)(u_0,\phi_k)_{L^2(0,\pi)}(\sin(n\pi) - \sin(n(\pi-h)))\right)^2 dt \\
    &= \frac{i}{2\pi h^2}\int_0^{\pi}\left(\sum_{k=1}^{\infty}
    \exp(-k^2t)(u_0,\phi_k)_{L^2(0,\pi)}\sin(n(\pi-h))\right)^2 dt \\
  \end{align*}
\end{enumerate}
\end{solution}

% --------------------------------------------------------------------------------
