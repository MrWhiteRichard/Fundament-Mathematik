% --------------------------------------------------------------------------------

\begin{exercise}
Betrachten Sie die eindimensionale Wärmeleitungsgleichung mit gemischten Randbedingungen
\begin{align*}
  \begin{cases}
    u_t - u_{xx} = 0 & \text{für } x \in (0,\pi),\ t > 0, \\
    u(0,t) = 0 & \text{für } t > 0, \\
    u_x(\pi,t) = 0 & \text{für } t > 0, \\
    u(x,0) = u_0(x) & \text{für } x \in (0,\pi),
  \end{cases}
\end{align*}
wobei $u_0 \in L^2(0,\pi)$.
\begin{enumerate}[label = (\roman*)]
  \item Bestimmen Sie ein vollständiges Orthonormalsystem $(\phi_n)_{n \in \N} \subset L^2(0,\pi)$
  mit $\phi^{\primeprime}_n = \lambda_n\phi_n$ in $(0,\pi)$ mit Randbedingungen
  $\phi_n(0) = \phi_n^{\prime}(\pi) = 0.$
  \item Konstruieren Sie aus $(\phi_n)_{n \in \N}$ eine Lösungsformel für das
  obige parabolische Problem.
  \item Welche Abklingrate (für $t \to \infty$) hat die Wärmeenergie
  $E(t) := \int_0^{\pi} u(x,t) dx$ für eine Lösung $u$?
\end{enumerate}



\end{exercise}

% --------------------------------------------------------------------------------

\begin{solution}

\phantom{}
\begin{enumerate}
  \item Betrachte die allgemeine Lösung für $\phi_n$:
  \begin{align*}
    \phi_n(x) = C_1\exp(\sqrt{-\lambda_n}x) + C_2\exp(-\sqrt{-\lambda_n}x)
  \end{align*}
  Durch Einsetzen in die Randbedingung bei $x = 0$ erhalten wir:
  \begin{align*}
    0 \stackrel{!}{=} C_1 + C_2 \implies C_2 = -C_1.
  \end{align*}
  Für die zweite Randbedingung wählen wir zur Vereinfachung $\lambda_n = n^2, n \in \N$
  und erhalten
  \begin{align}
    0 \stackrel{!}{=} \phi^{\prime}_n(\pi) = [C(\exp(inx) - \exp(-inx)]^{\prime}
    = Cn(\underbrace{-\sin(nx) + \sin(-nx)}_{=0} + i\underbrace{(\cos(nx) - \cos(-nx))}_{=0}) = 0.
  \end{align}
  Damit lautet unser vollständiges Orthonormalsystem
  \begin{align*}
    \phi_n(x) = \frac{\exp(inx) - \exp(-inx)}{\|\exp(inx) - \exp(-inx)\|_{L^2(0,\pi)}}
    = \frac{2i\sin(nx)}{\sqrt{2\pi}}
  \end{align*}
  \item Definiere
  \begin{align*}
    u(x,t) := \sum_{k=1}^{\infty}\exp(-k^2t)(u_0,\phi_k)_{L^2(0,\pi)}\phi_k(x)
  \end{align*}
  Damit gilt
  \begin{align*}
    u(0,t) &= \sum_{k=1}^{\infty}\exp(-k^2t)(u_0,\phi_k)_{L^2(0,\pi)}\phi_k(0) = 0. \\
    u(x,0) &= \sum_{k=1}^{\infty}(u_0,\phi_k)_{L^2(0,\pi)}\phi_k(x) = u_0(x). \\
    u_x(\pi,t) &= \partial_x\left(\sum_{k=1}^{\infty}\exp(-k^2t)(u_0,\phi_k)_{L^2(0,\pi)}\phi_k(x)\right)\Bigg|_{x=\pi} \stackrel{!}{=} 0.
  \end{align*}
  Für die dritte Gleichung müssen wir also zeigen, dass
  \begin{align*}
    0 &= \lim_{h \to 0^+}\left\|\frac{1}{h}(u(\pi,t) - u(\pi - h,t))\right\|_{L^2(0,\pi)}^2
    = \frac{1}{h^2}\int_0^{\pi}(u(\pi,t) -  u(\pi -h,t))^2 dt \\
    &= \frac{1}{h^2}\int_0^{\pi}\left(\sum_{k=1}^{\infty}\exp(-k^2t)(u_0,\phi_k)_{L^2(0,\pi)}\phi_k(\pi) -  \sum_{k=1}^{\infty}\exp(-k^2t)(u_0,\phi_k)_{L^2(0,\pi)}\phi_k(\pi - h)\right)^2 dt \\
    &= \frac{i}{2\pi h^2}\int_0^{\pi}\left(\sum_{k=1}^{\infty}
    \exp(-k^2t)(u_0,\phi_k)_{L^2(0,\pi)}(\sin(n\pi) - \sin(n(\pi-h)))\right)^2 dt \\
    &= \frac{i}{2\pi h^2}\int_0^{\pi}\left(\sum_{k=1}^{\infty}
    \exp(-k^2t)(u_0,\phi_k)_{L^2(0,\pi)}\sin(n(\pi-h))\right)^2 dt \\
  \end{align*}
\end{enumerate}
\end{solution}

% --------------------------------------------------------------------------------

\begin{solution}
\begin{enumerate}[label = (\roman*)]
  \item Wir betrachten zunächst Lösungen der Differentialgleichung

  \begin{align*}
    \phi_n
    =
    c_1 e^{\sqrt{\lambda_n}x} + c_2 e^{-\sqrt{\lambda_n}x}
  \end{align*}

  Fall 1: $\lambda_n > 0$: \\
  Dann hat die Lösung folgende Form:

  \begin{align*}
    \phi_n(x)
    =
    c_1 \cosh(\sqrt{\lambda_n}x) + c_2 \sinh(\sqrt{\lambda_n}x)
  \end{align*}

  Diese kann die Randbedingungen jedoch nur im trivialen Fall $c_1, c_2 = 0$ erfüllen, da

  \begin{align*}
    &\phi_n(0) = c_1 \stackrel{!}{=} 0\\
    \implies
    &\phi^\prime(\pi) = c_2 \sqrt{\lambda_n} \cosh({\sqrt{\lambda_n}x}) \stackrel{!}{=} 0
  \end{align*}

  Nun ist $\cosh$ jedoch eine strikt positive Funktion, also muss auch $c_2 = 0$.

  Fall 2: $\lambda_n = 0$: \\
  Hier ist unsere Differentialgleichung nur $\phi_n^\primeprime = 0$, hat also die Form $c_1 x + c_2$. Auch hier erhalten wir $\phi_n \equiv 0$.

  Fall 3: $\lambda_n < 0$: \\
  Lösungen sind gegeben durch

  \begin{align*}
    \phi_n(x) = c_1 \sin(\sqrt{|\lambda_n|}x) + c_2 \cos(\sqrt{|\lambda_n|}x)
  \end{align*}

  Für die RB setzen wir ein
  \begin{align*}
    &\phi_n(0) = c_2 \stackrel{!}{=}0 \\
    \implies
    &\phi_n^\prime(\pi) = c_2 \sqrt{|\lambda_n|} \cos({\sqrt{|\lambda_n|}\pi}) \stackrel{!}{=} 0
    \iff
    \sqrt{|\lambda_n|}\pi = \frac{\pi}{2} + k\pi, \quad k \in \Z
  \end{align*}

  Wir haben also zunächst mit der Wahl $k := n-1$ eine Darstellung von $\phi_n$, mit einer noch zu bestimmenden Konstante c, als

  \begin{align*}
    \phi_n(x) = c \sin(\frac{2n-1}{2}x)
  \end{align*}

  Zweimaligens differenzieren liefert

  \begin{align*}
    \phi_n^\primeprime(x) = c \underbrace{-(\frac{2n-1}{2})^2}_{\lambda_n} \sin(\frac{2n-1}{2}x)
  \end{align*}

  Da $\norm[L^2((0,\pi))]{\sin(\frac{2n-1}{2}x)} = \sqrt{\frac{\pi}{2}}$ wählen wir $c = \sqrt{\frac{2}{\pi}}$ um $\norm[L^2((0,\pi))]{\phi_n} = 1$ zu erhalten. \\
  Noch offen $\rightarrow$ Vollständigkeit.
  \item Da wir ein vollständiges Orthonormalsystem haben, konvergiert die Reihe

  \begin{align*}
    u(x,t)
    :=
    \sum_{n=1}^\infty e^{\lambda_n t} (u_0, \phi_n)_{L^2} \sqrt{\frac{2}{\pi}} \sin(\sqrt{-\lambda_k} x)
  \end{align*}

  unbedingt. Diese Funktion löst die PDE, dazu setzen wir ein

  \begin{align*}
    u_t - u_{xx} &=\sum_{n=1}^\infty \lambda_n e^{\lambda_n t} (u_0, \phi_n)_{L^2} \sqrt{\frac{2}{\pi}} \sin(\sqrt{-\lambda_k} x) - \sum_{n=1}^\infty \lambda_n e^{\lambda_n t} (u_0, \phi_n)_{L^2} \sqrt{\frac{2}{\pi}} \sin(\sqrt{-\lambda_k} x)= 0 \\
    u(0,t) &= \sum_{n=1}^\infty e^{\lambda_n t} (u_0, \phi_n)_{L^2} \sqrt{\frac{2}{\pi}} \sin(\sqrt{-\lambda_k} 0) = 0 \\
    u_x(\pi, t) &= \sum_{n=1}^\infty \lambda_n e^{\lambda_n t} (u_0, \phi_n)_{L^2} \sqrt{\frac{2}{\pi}} \cos(\sqrt{-\lambda_k} \pi)= 0 \\
    u(x, 0) &= \sum_{n=1}^\infty (u_0, \phi_n)_{L^2} \phi_n(x) = u_0(x)
  \end{align*}

  \item Unter der Annahme, dass mit der Abklingrate eine Ähnliche Ungleichung wie in $(6.15)$ im Skript gemeint ist, schätzen wir ab (wobei $\lambda_1$ der kleinste EW ist)

  \begin{align*}
    \norm[L^2]{u(\cdot,t)}
    =
    \Int[0][\pi]{u(x,t)^2}{x}
    \stackrel{\text{Pythagoras}}{=}
    \Int[0][\pi]{\sum_{n=1}^\infty e^{2\lambda_n t} (u_0, \phi_n)_{L^2}^2 \frac{2}{\pi} \sin(\sqrt{-\lambda_k} x)^2}{x}
    \leq
    e^{2\lambda_1 t} \norm[L^2]{u_0}^2
  \end{align*}

  Insgesamt erhalten wir eine Abschätzung

  \begin{align*}
    |E(t)|
    \leq
    \Int[0][\pi]{|u(x,t)|}{x}
    \leq
    \sqrt{\pi} \norm[L^2]{u(\cdot,t)}
    \leq
    \sqrt{\pi} e^{\lambda_1 t}\norm[L^2]{u_0}
  \end{align*}
  \end{enumerate}
\end{solution}
