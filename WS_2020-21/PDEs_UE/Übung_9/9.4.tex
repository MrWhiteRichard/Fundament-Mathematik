% --------------------------------------------------------------------------------

\begin{exercise}
Betrachten Sie die eindimensionale Wärmeleitungsgleichung mit gemischten Randbedingungen
\begin{align*}
  \begin{cases}
    u_t - u_{xx} = 0 & \text{für } x \in (0,\pi),\ t > 0, \\
    u(0,t) = 0 & \text{für } t > 0, \\
    u_x(\pi,t) = 0 & \text{für } t > 0, \\
    u(x,0) = u_0(x) & \text{für } x \in (0,\pi),
  \end{cases}
\end{align*}
wobei $u_0 \in L^2(0,\pi)$.
\begin{enumerate}[label = (\roman*)]
  \item Bestimmen Sie ein vollständiges Orthonormalsystem $(\phi_n)_{n \in \N} \subset L^2(0,\pi)$
  mit $\phi^{\primeprime}_n = \lambda_n\phi_n$ in $(0,\pi)$ mit Randbedingungen
  $\phi_n(0) = \phi_n^{\prime}(\pi) = 0.$
  \item Konstruieren Sie aus $(\phi_n)_{n \in \N}$ eine Lösungsformel für das
  obige parabolische Problem.
  \item Welche Abklingrate (für $t \to \infty$) hat die Wärmeenergie
  $E(t) := \int_0^{\pi} u(x,t) dx$ für eine Lösung $u$?
\end{enumerate}



\end{exercise}

% --------------------------------------------------------------------------------

\begin{solution}

\phantom{}

\end{solution}

% --------------------------------------------------------------------------------
