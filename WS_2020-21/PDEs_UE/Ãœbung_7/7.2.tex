% --------------------------------------------------------------------------------

\begin{exercise}

  Sei $\Omega \subseteq \R^n$ ein beschränktes Gebiet mit $\partial\Omega \in C^1.$ Zeigen Sie:
  \begin{itemize}
      \item[(a)] Sind $u \in H^k(\Omega) (k \in \N)$ und $v \in C^\infty(\overline{\Omega}),$ so folgt $uv \in H^k(\Omega).$
       \item[(b)] Sind $u \in H^1(\Omega)$ und $v \in C_0^\infty(\overline{\Omega}),$ so folgt $uv \in H^1_0(\Omega).$
  \end{itemize}

\end{exercise}

% --------------------------------------------------------------------------------

\begin{solution}

\begin{itemize}
  \item[a)] Wir führen den Beweis mittels Induktion nach $k$:
  \begin{itemize}
    \item[$k = 0:$] $\|uv\|_{H^0(\Omega)}^2 = \int_\Omega (uv)^2 dx \leq
    \left(\max_{x \in \overline{\Omega}}v(x)\right)^2 \|u\|_{H^0(\Omega)}^2 < \infty$
    \item[$k \rightsquigarrow k+1:$] Sei $u \in H^{k+1}(\Omega), v \in C^{\infty}(\overline{\Omega})$
    beliebig.
    \begin{align*}
      \int_\Omega uvD^i\phi dx &= \int_\Omega u D^i(v\phi)dx  - \int_\Omega u(D^i v)\phi dx \\
      &= -\int_\Omega D^i(u)v\phi dx - \int_\Omega u(D^iv)\phi dx
      = -\int_\Omega (D^i(u)v + uD^iv)\phi dx
    \end{align*}
    Also erhalten wir $D^i(uv) = D^i(u)v + uD^i(v)$. Für $|\alpha| = k + 1$ beliebig
    folgt somit wegen \\
    $H^{k+1}(\Omega) \subset H^k(\Omega)$
    \begin{align*}
      \int_\Omega (D^{\alpha}(uv))^2 dx = \int_\Omega (D^\beta D^i (uv))^2 dx
    = \int_\Omega \overbrace{(D^\beta(\underbrace{D^i(u)}_{\in H^k(\Omega)}v)}^{\in L^2(\Omega)}
      + \overbrace{D^\beta(\underbrace{u}_{\in H^k(\Omega)}D^i(v))}^{\in L^2(\Omega)})^2 dx < \infty
    \end{align*}
  \end{itemize}
  \item[(b)] Wir wählen eine Folge $(u_n)_{n \in \N}$ aus $C^\infty(\Omega)$ mit $\lim_{n \rightarrow \infty} \|u_n - u\|_{H^1(\Omega)} = 0.$

Zuerst berechnen wir für festes $i \leq n$
\begin{align*}
    &\int_\Omega (D^i\left((u_n-u) v)\right)^2 \mathrm{~d}\lambda^n\\
    &= \int_{\Omega} (D^i(u_n-u)v + (u_n-u)D^iv)^2 \mathrm{~d}\lambda^n\\
    &\leq \|v\|^2_\infty \|u_n - u\|^2_{H^1(\Omega)} + 2 \int_{\Omega} v (D^iv) (u_n-u) D^i(u_n - u) \mathrm{~d}\lambda^n + \|D^iv\|^2_\infty \|u_n - u\|^2_{H^1(\Omega)}\\
    &\leq \underbrace{\|u_n - u\|^2_{H^1(\Omega)}}_{\xrightarrow[]{n \rightarrow \infty}{} 0} \left(\|v^2\|_\infty + \|D^iv\|^2_\infty\right) + 2\|vD^iv\|^2_\infty \underbrace{\|u_n-u\|_{L^2(\Omega)}}_{\xrightarrow[]{n \rightarrow \infty}{} 0} \underbrace{\|D^i(u_n-u)\|_{L^2(\Omega)}}_{\xrightarrow[]{n \rightarrow \infty}{} 0} \xrightarrow[]{n \rightarrow \infty}{} 0.
\end{align*}
Deshalb gilt
\begin{align*}
    \| u_n v - uv \|^2_{H^1(\Omega)}
    &= \| (u_n-u) v\|^2_{H^1(\Omega)}\\
    &= \int_\Omega((u_n-v) v)^2 \mathrm{~d}\lambda^n
    + \sum_{i = 1}^n \int_\Omega (D^i\left((u_n-u) v)\right)^2 \mathrm{~d}\lambda^n\\
    &\leq \|v\|^2_\infty \underbrace{\|u_n - u\|^2_{H^1(\Omega)}}_{\xrightarrow[]{n \rightarrow \infty}{} 0} + \sum_{i = 1}^n \underbrace{\int_{\Omega} (D^i\left((u_n-u) v)\right)^2 \mathrm{~d}\lambda^n}_{\xrightarrow[]{n \rightarrow \infty}{} 0} \xrightarrow[]{n \rightarrow \infty}{} 0.
\end{align*}

Als Grenzwert der Folge der $C_0^\infty$-Funktionen $u_n v$ liegt $uv$ damit in $H_0^1(\Omega).$
\end{itemize}

\end{solution}

% --------------------------------------------------------------------------------
