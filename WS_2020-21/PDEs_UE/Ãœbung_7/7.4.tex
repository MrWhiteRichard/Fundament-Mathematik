% --------------------------------------------------------------------------------

\begin{exercise}

  Zeigen Sie die \textit{2. poincarésche Ungleichung:} Sei $\Omega \subseteq \R^n$ ein beschränktes Gebiet mit $\partial\Omega \in C^1.$ Dann existiert eine Konstante $C > 0,$ sodass für alle $u \in H^1(\Omega)$
  \begin{align*}
      \| u - \overline{u} \|_{L^2(\Omega)} \leq C \| \nabla u \|_{L^2(\Omega)}
  \end{align*}
  gilt, wobei $\overline{u} := \frac{1}{|\Omega|} \int_\Omega u(x) \mathrm{~d}x$.

  \textit{Hinweis:} Sie dürfen folgende Aussage ohne Beweis verwenden: Sei $\Omega \subseteq \R^n$ ein beschränktes Gebiet mit $\partial\Omega \in C^1$ und $u \in H^1(\Omega)$
  mit $\| \nabla u \|_{L^2(\Omega)} = 0.$ Dann ist $u$ eine konstante Funktion.

\end{exercise}

% --------------------------------------------------------------------------------

\begin{solution}

Wir führen einen Widerspruchsbeweis. Angenommen, es gäbe keine solche Konstante,
dann finden wir eine Folge $(u_n)_{n \in \N} \subset H^1(\Omega)$ mit
\begin{align*}
  \|u_n - \overline{u_n}\|_{L^2(\Omega)} \geq n\|\nabla u_n\|_{L^2(\Omega)}
\end{align*}
Definiere nun $v_n := \frac{u_n - \overline{u_n}}{\|u_n - \overline{u_n}\|_{L^2(\Omega)}}$.
Dann folgt
\begin{align*}
  \|v_n\|_{L^2(\Omega)} &= 1, \\
  \qquad \|\nabla v_n\|_{L^2(\Omega)} &\leq \frac{1}{n}, \\
  \int_\Omega v_n &= \frac{1}{\|u_n -\overline{u_n}\|_{L^2(\Omega)}}
  \left(\int_{\Omega}u_n - \frac{1}{|\Omega|}u_n dx dy\right)
  = \frac{1}{\|u_n -\overline{u_n}\|_{L^2(\Omega)}}
  \left(\int_{\Omega}u_n dy - \int_\Omega u_n dx\right) = 0.
\end{align*}
Insbesondere ist $(v_n)_{n \in \N}$ eine beschränkte Folge in $H^1(\Omega)$.
Aus dem Satz von Rellich-Kondrachov erhalten wir vermöge der kompakten Einbettung
$H^1(\Omega) \hookrightarrow H^0(\Omega) = L^2(\Omega)$ eine konvergente Teilfolge
$(v_{n_k})_{k\in \N}$ in $L^2(\Omega)$. Für den Grenzwert $v := \lim_{k \to \infty} v_{n_k}$ gilt
\begin{align*}
  \|v\|_{L^2(\Omega)} = 1, \quad \int_\Omega v = 0.
\end{align*}
Außerdem ist $(v_{n_k})_{k \in \N}$ sogar eine Cauchy-Folge in $H^1(\Omega)$
und konvergiert daher in $H^1(\Omega)$ gegen den selben Grenzwert.
Aus der Stetigkeit von $\|\nabla(\cdot)\|_{L^2(\Omega)}$ bezüglich der $H^1(\Omega)$-Norm
folgt also
\begin{align*}
  \|\nabla v\| = \lim_{k \to \infty}\|\nabla v_{n_k}\|_{L^2(\Omega)} = 0.
\end{align*}
Laut Hinweis dürfen wir nun behaupten, dass $v$ bereits konstant sein muss.
Aus $\int_\Omega v = 0$ folgt damit sogar $v \equiv 0$, im Widerspruch zu $\|v\|_{L^2(\Omega)} = 1$.
\end{solution}

% --------------------------------------------------------------------------------
