% -------------------------------------------------------------------------------- %

\begin{exercise}

  Sei $\Omega = \{(x, y) \in \R^2: 0 < x < 1, x^{1/5} < y < 1\}.$
  \begin{itemize}
      \item[(a)] Finden Sie eine Funktion $u \in H^2(\Omega),$ sodass $u \not\in C^0(\overline{\Omega}).$ \textit{Hinweis:} $u(x, y) = y^\alpha.$
      \item[(b)] Warum ist das kein Widerspruch zur stetigen Einbettung von $H^2(\Omega)$ in $C^0(\Omega)$ in zweidimensionalen Gebieten?
  \end{itemize}

\end{exercise}

% -------------------------------------------------------------------------------- %

\begin{solution}
\phantom{}
\begin{itemize}
    \item[(a)] Wähle $\alpha = -\frac{1}{2}$, dann ist $u(x,y) = y^{\alpha}$ auf der $y$-Achse
    sicher nicht stetig bis zum Rand und somit nicht in $C^0(\overline{\Omega})$.
    \begin{align*}
      \|u\|_{L^2(\Omega)} &= \int_0^1\int_{x^{1/5}}^1 (y^{-1/2})^2 dy dx
      = -\int_0^1 \ln(x^{1/5}) dx = -\frac{1}{5}[x\ln(x) - x]_0^1 = \frac{1}{5} \\
      \left\|\frac{\partial}{\partial y}u\right\|_{L^2(\Omega)} &=
      \int_0^1\int_{x^{1/5}}^1 \left(-\frac{1}{2}y^{-3/2}\right)^2 dy dx
      = \frac{1}{4}\int_0^1\left[-\frac{1}{2}y^{-2}\right]_{x^{1/5}}^1 dx \\
      &= -\frac{1}{8}\int_0^1 1 - x^{-2/5} dx = -\frac{1}{8}\left(1 - \frac{5}{3}\right) = \frac{1}{12} \\
      \left\|\frac{\partial^2}{\partial y^2}u(x,y)\right\| &=
      \int_0^1\int_{x^{1/5}}^1 \left(\frac{3}{4}y^{-5/2}\right)^2 dy dx
      = \frac{9}{16}\int_0^1 \left[-\frac{1}{4}y^{-4}\right]_{x^{1/5}}^1 dx \\
      &= -\frac{9}{64}\int_0^1 1 - x^{-4/5} dx = -\frac{9}{64}(1 - 5) = \frac{9}{16}
    \end{align*}
    Da die $x$-Ableitungen wegfallen, ist $u \in H^2(\Omega)$.
    \item[b)]

    \begin{figure}[h!]
      \centering
      \includegraphicsboxed{PDEs/PDEs_-_Satz_5-9_(Einbettungssatz_von_Sobolev).png}
      \caption{\cite{PDEs}}
      \label{fig:Satz 5.9}
    \end{figure}

    In unserem Fall ist die Bedingung $k - n/2 = 2 - 1 > 0 = m$ erfüllt, also
    muss die Bedingung an $\partial \Omega$ verletzt sein.
    In der Tat kann man zeigen, dass der Rand sich in keiner Umgebung von $(0,0)$
    durch eine Lipschitz-stetige Funktion darstellen lassen kann, da
    \begin{align*}
      \lim_{x \to 0^+}\frac{x^{1/5}}{x} = x^{-4/5} = +\infty.
    \end{align*}
\end{itemize}


\end{solution}

% -------------------------------------------------------------------------------- %
