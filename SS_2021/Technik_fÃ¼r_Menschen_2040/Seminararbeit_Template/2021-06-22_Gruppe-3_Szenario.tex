%\documentclass[twoside, a4paper, DIV=11, bibliography=totocnumbered]{scrreprt}
\documentclass[twoside, a4paper, DIV=11, open=any, bibliography=totoc]{scrbook}

\usepackage{url}
\usepackage{hyperref}
\usepackage{graphicx}
\usepackage[all]{hypcap}
\usepackage[utf8]{inputenc}
\usepackage[ngerman]{babel}
\usepackage[skipabove=\baselineskip,skipbelow=\baselineskip,outermargin=30pt,innermargin=30pt]{mdframed}

\usepackage{mdframed}

\usepackage{blindtext} % remove this one, only for template/demonstration

\newcommand{\Quote}[1]{\glqq #1\grqq{}}

\hypersetup
{
 pdfauthor={Vorname Name},
 pdftitle={Seminararbeit ... },
 pdfkeywords={...},
 colorlinks=true,
 citecolor=black,
 filecolor=black,
 linkcolor=black,
 urlcolor=black
}

\KOMAoption{headings}{normal}

\tolerance=4000
\emergencystretch=20pt

\begin{document}

\chapter{Szenario} \label{chap:szenario}

\section{Szenario-Gruppe} \label{sec:szengruppe}

Das Szenario wurde in Gruppe 3 durchgeführt. Mitglieder dieser Gruppe waren:

\begin{itemize}
    \item 11916463 – Sophie, Hinterholzer
    \item 11802325 – Christoph, Neumayr
    \item 11819578 – Florian, Schager
    \item 11806459 – Daniel, Teubl
\end{itemize}

\section{Annahmen} \label{sec:szenannahmen}

Die Klimakrise bleibt ein dominierendes Thema unserer zukünftigen Gesellschaft.

\section{Kontext} \label{sec:szenkontext}

Unsere Szenarien handeln über die Familie Schmidt. Diese besteht aus:
\begin{itemize}
    \item Heinz (Opa 72)
    \item Julia (Mutter 50)
    \item Max (Schüler 14)
    \item Mia (Studentin 21)
\end{itemize}
Sie leben alle gemeinsam in einer Wohnung in Wien. 
Die Szenarien beschäftigen sich hauptsächlich mit positiven (Utopie) 
negativen (Dystopie) Veränderungen in Bezug auf das Klima und die Umwelt,
sowie mit den Maßnahmen und Reaktionen der Gesellschaft.


\section{Dystopie} \label{sec:szendystopie}

Der Tag fängt an wie jeder andere Tag auch. Es gibt, wie sonst auch immer,
Haferflocken mit Mandelmilch zum Frühstück und Opa liest wieder
seine Nachrichten. \Quote{Weißt du, Max}, fängt Opa wieder an zu reden, 
\Quote{früher hatten
wir noch ganz andere
Probleme. Da mussten wir uns noch selbst mit den wichtigen Fragen herumschlagen und
Eigeninitiative besitzen.
Da war es nicht so einfach, herauszufinden was man mal werden will. Wir mussten
selbst herumprobieren und
uns fragen, was wirklich zu uns passt und nicht wie ihr heute durch Computer direkt
in eure Leidenschaften
eingeteilt werdet und dann nur noch das Lernen. Wusstet ihr, dass ich früher mal
Französisch sprechen konnte?}

Diese Geschichte erzählt Opa besonders gerne.
\Quote{Früher war die Welt noch schwer, die
Jugend von Heute ist so
verwöhnt, bla bla bla ...}

Wenn er wüsste, wie es draußen wirklich ist und sich nicht von seiner
populistischen Propaganda berieseln lassen würde.
Nach seiner Lungenkrebsdiagnose darf er nämlich nicht mehr an die frische Luft,
wobei ja man von frischer Luft nicht mehr wirklich sprechen kann.
Der Smog der sich jeden Tag über die Stadt niedersenkt ist bereits für
junge, gesunde Erwachsene gesundheitsschädlich, für kranke Alte ist diese
Belastung demnach absolut lebensbedrohlich.

In Geschichte haben wir gelernt, dass es nicht normal ist, dass man in ummauerten
Gebieten wohnen muss, damit
man sicher draußen sein kann. Wir haben auch gelernt, dass es früher über 1 Mrd.
Menschen in Afrika gegeben hat.
Dass diese Menschen jetzt dort nicht mehr überleben können, sei die Schuld von
Indien und China.
Was ich nicht ganz
verstehe ist, wie sie es früher gemacht haben. Wie soll man bitte draußen sicher
sein, wenn man nicht in einem
reichen Viertel wohnt? Unsere Bezirkswache hat so schon viel zu viel zu tun, doch
ohne die Mauer wäre alles
unmöglich zu regeln. Ich beneide die Menschen draußen vor der Mauer manchmal.
Sie sind zwar wohl nicht so reich wie wir, doch können sie wenigstens frei
herumlaufen und was soll
man denen schon antun. Sie haben ja nichts, das man stehlen könnte, oder für das
man sie entführen würde.

Naja, ab in die Schule. 
Heute ist im Wetterbericht wieder Maskenstufe III angekündigt worden,
so schlimm war es schon lange nicht mehr. Den Gestank unter diesen Masken hält man kaum
aus, aber was tut man nicht alles für seine Lunge.
Heute ist ja wieder einer dieser \Quote{Zukunftstests}, wie es
die Lehrkräfte immer nennen.
Dass diese nicht funktionieren, weiß eigentlich jeder, doch trotzdem werden sie
durchgezogen und alle, die
von denen nicht betroffen sind finden sie super. Anscheinend hatte ich bisher immer
eine ausgeprägte Leidenschaft für Mathe, was hieß, dass ich 10 Stunden pro Woche in
dem Fach verbringen musste,
welches mir am wenigsten Spaß machte. Hoffentlich wird es dieses mal Sport, aber
bei dem Test kann man gar nicht
sagen, was schlussendlich rauskommt. Eine Freundin von mir hat mal versucht nicht
ehrlich zu antworten, um
mit mir in einer Klasse zu sein... Dass das nicht gut ausging könnt ihr euch ja
denken. Mittlerweile kennt
sie sich zwar super in Bergbau aus, aber wer braucht das heutzutage schon noch?
Wird doch eh alles recycled und
nicht neu abgebaut... Aber das Programm sei die Zukunft haben sie uns vorgepredigt.
Ihr werdet alle glücklich,
haben sie gesagt. Dass es nicht funktioniert, das übersieht jeder gerne. Wir sind
ja auch nur verwöhnte
Jugendliche, die nicht wissen, was Leiden wirklich ist und was im Leben wirklich wichtig
ist.
Ich freue mich, wenn der Tag endlich vorbei ist.

\vspace{10pt}

Beim Frühstück checke ich ich schnell mein Tablet, um die neuesten Wetterinformationen
einzuholen. Für heute ist seit langem mal wieder Maskenstufe III vorgeschrieben, 
so schlimm war die Luftverschmutzung erst ein einziges Mal im letzten Sommer.
Hoffentlich haben wir die Spezialmasken noch irgendwo zuhause verstaut,
ansonsten sind wir die nächsten Tage hier eingeschlossen. In der untersten Lade
der alten Kommode beim Vorzimmerspiegel werde ich schließlich fündig, gerade noch
rechtzeitig um Max die Maske für den Schulweg mitzugeben.
Ein weiterer Blick auf die vorgeschriebene Gesundheits-App verrät mir, dass
es heute wieder Linsen mit Reis gibt. Das ist jetzt schon das dritte Mal in den
letzten zehn Tagen! Am Anfang fanden wir alle die neuen Essenspläne noch toll,
viel gesünder würden wir uns ernähren, abwechslungsreicher und wir müssten uns
gar keine Gedanken mehr darüber machen, was wir morgen kochen wollen.
Inzwischen glaube ich aber, dass das ganze ein ziemlicher Schuss in den Ofen war.
Jede Woche der selbe, langweilige Fraß und viel gesünder 
fühl ich mich auch nicht als zuvor.

In dem Moment schaut mir Papa über die Schulter.
\Quote{Schon wieder Linsen mit Reis}, schimpft er. 
\Quote{Jetzt verbieten sie mir schon meine
Wohnung zu verlassen und nicht mal mehr das eigene Essen kann man sich aussuchen!
Julia, lange halt ich das nicht mehr aus! Kannst du nicht mal wieder bei unserem
alten Freund Peter im 18. vorbeischaun? Der alte Hase hat sicher noch irgendwo ein
gutes Stück Fleisch gebunkert, so wie ich ihn kenne.}

Papa, du weißt ja, dass mir das selber auch auf den Keks geht. Aber zahlt sich
das aus, dafür in die äußeren Bezirke zu fahren? Du weißt ja selbst, wie es
da zugeht. Anscheinend seien letzte Woche 15 Menschen bei einer
Demonstration getötet worden und sie hätten es fast geschafft, bis in die
Innenstadt vorzudringen. Wenn das passiert, 
dann werde ich sicherlich wegziehen, denn ohne diese Mauer zwischen uns
und denen, kann und will ich nicht leben. 

Wien wird es eh nicht mehr lange schaffen, denn 32 Millionen Menschen
sind einfach zu viel.
Was haben die ganzen Leute sich gedacht, als sie hierher gekommen sind? Klar sind
die Lebensbedingungen anderswo
nicht besser, aber wieso genau hier? Vor der ersten Klimawelle hatten wir ja sogar
noch überlegt aufs Land zu
ziehen, nicht zuletzt dem Opa zuliebe, aber glücklicherweise kam es nicht so weit... 
Sonst hätten sie uns auch sicherlich ausgeplündert.
Wenn es nach mir gegangen wäre, dann hätten diese ganzen Leute in Afrika bleiben
können und alles wäre super.
Dass wir alle Schuld am Klimawandel sind, halte ich für ausgeschlossen. Ich habe
mal gelesen, dass China und
Indien für 98\% aller Umweltschäden verantwortlich sind - wieso sollen wir also an
den Auswirkungen leiden?
Glücklicherweise wurde das mit den Flüchtlingsbezirken gut umgesetzt. Als ob ich
neben denen wohnen will?
Ich habe gehört, dass letztens wieder ein Virus im 26. Bezirk ausgebrochen sei und
dabei 400 Menschen gestorben
sind. Da sind sie sicher selbst Schuld, denn das ihnen zugeteilte Wasser
verschwenden die immer extrem.
Bald soll eh wieder eine Umstrukturierung stattfinden. Hoffentlich wird Wien, dann
wieder ein Stück sicherer.
Habe gehört, sie wollen härtere Strafen für Diebstahl und Armut einführen. Endlich,
dann werden diese ganzen
Leute sicher nach Deutschland weiterziehen. Mir wurde mal gesagt, dass es dort noch
viel schlimmer zugeht.
Die Städte seien komplett überlaufen und keinerlei Sicherheitsmaßnahmen wurden
ergriffen. Es sei nicht
ungewöhnlich, dass man einfach auf offener Straße ausgeraubt wird... Puh, bin ich
froh, dass unsere Regierung
das richtig geregelt hat. Jetzt aber los, ich will meine U-Bahn nicht verpassen.
Immer diese Leidensbeklagungen von Opa... Das geht mir mittlerweile richtig auf die
Nerven. Er versteht einfach
nichts. Kein bisschen Mitgefühl bekommt er von mir, denn seine Generation ist ja
Schuld, dass wir jetzt hier
in dieser Lage feststecken. Hätten sie direkter gehandelt und nicht weiter mit
diesen Umweltverschmutzern
geredet, dann wäre es nie so weit gekommen. Über 98\% der Verschmutzungen ist ja nur
aus den zwei Ländern gekommen,
wieso hat man sie dann nicht direkt verurteilt? Zu egoistisch waren sie und hatten
keinerlei Rücksicht auf
andere Menschen und die Welt allgemein. Zum Glück studiere ich Umweltressourcen,
denn sonst würde ich ja auch
nur blind dem folgen, was die Regierung so verzapft. Als ob ich wirklich glauben
würde, dass die Flüchtlinge
schuld an der Vergiftung der Donau sind... So ein populistisches Gedankengut kommt
bei meinem Opa natürlich
super an. Ich versuche ihm zwar zu erklären, dass das alles nur von natürlichem
Algenwachstum kommt, aber
das ist für ihn "zu weit hergeholt" und er fängt wieder mit seinen Anfeindungen an.
Es stimmt zwar, dass die meisten Straftaten in den äußeren Bezirken verübt werden
und diese Leute einfach
den Drang dazu haben, Dinge kaputtzumachen und alles zu verschmutzen, aber ich
denke, dass die Straftaten auch
mit einer Einmischung von Indien zu tun hat. Ich habe mal gehört, dass sehr viel
Unruhe durch die Propaganda
von Außen entsteht und wir endlich eine Lösung für diese Einmischung finden müssen.

\section{Utopie} \label{sec:szenutopie}

Wir schreiben den 22.06.2040.
Heute ist ein schöner Tag! 
Die Nachrichten lesen sich mit Genuss, denn die Wissenschaft hat endlich 
einen Weg gefunden das überschüssige CO2 in der Atmosphäre zu speichern 
um den Klimawandel zu verlangsamen. Der Herr Opa Schmidt freut sich 
und nippt an seinem Kaffee. Er denkt über seine Vergangenheit nach 
und ist sehr froh darüber, dass sich die Welt zusammengetan hat um 
Lösungen zur Bekämpfung der rasenden Klimaveränderung umzusetzen. 
Denn schon damals als Opa Schmidt ein junger Student war hatten die 
Leute gute Ideen und wollten sich für die Natur und das Wohl der damit 
so eng verknüpften Menschen einsetzen. Aber wie bekanntlich jeder weiß, 
dauern großflächige Veränderungen mindestens 20 Jahre in ihrer Umsetzung. 
Opa Schmidt hatte damals befürchtet, dass die Menschen in der Frage der Klimakrise 
zu langsam handeln, doch wie er nun in seiner Zeitung lesen kann, 
ist dem nicht so und wir sind einer grauenvollen Dystopie gerade noch entwischt. 
Er stellt sich vor was er heute wohl in seiner Zeitung lesen würde wenn alles 
anders gekommen wäre…Angefangen vom Artensterben bis hin zu Klimaflüchtlingen und 
Zusammenbrüchen von Gesellschaften und dem Wirtschaftssystem...


Max Schmidt sitzt in der Schule. Er folgt gespannt dem Geographie Unterricht 
und hört wie sein Lehrer zu sprechen beginnt. Der Lehrer beginnt mit dem Thema 
Wüstenbildung und gesunder Vegetation. Als Beispiel nennt er hier den erfolgreichen 
Stopp der Ausbreitung der Sahara. Die Bewohner dieses afrikanischen Bereiches 
konnten durch gezielte Setzung von Bäumen die Weiterbildung der Wüste stoppen. 
Somit gelang ihnen etwas, dass ihnen niemand zugetraut hätte! Max freut sich, 
denn was wäre mit all den prächtigen Wildtieren in der afrikanischen Savanne, 
wenn das Leben dort durch Dürre und heiße Flächen vertrieben worden wäre. 
Er denkt natürlich auch an die Bewohner Afrikas und ist von deren 
Einsatz beeindruckt. Es ist wohl alles möglich, wenn wir Menschen 
alle an einem Strang ziehen, sagt der Geographie Lehrer und nennt hierbei 
auch die überwundene Klimakrise.


Da die gesamten inneren Bezirke seit vielen Jahren autofrei sind kann Julia entspannt 
mittig den Ring entlang radeln. Auf ihrem Heimweg bleibt sie noch schnell am Naschmarkt 
stehen. Dort will sie Erdäpfel und frische Kräuter bei ihrem Lieblingsstand kaufen. 
Ihre selbst angebauten Erdäpfel an den Fassaden sind leider noch nicht bereit zur Ernte, 
aber dies ist ja kein Problem da nur noch biologischer Anbau 
von Obst und Gemüse erlaubt ist. Zum Glück meldet sich ihre digitale Brille 
(Augmented Reality), dass im Kühlschrank keine Milch und kein Fleisch mehr sind. 
Diese besorgt sie ebenfalls noch schnell.


Mia sitzt gerade in der Universität in der Vorlesung und lehnt sich zurück. 
Sie bekommt eine Benachrichtigung, dass sie aufgrund ihrer guten universitären 
Leistungen und ihrer täglichen Anwesenheit an der Universität an einem 
Studienaustauschprogramm teilnehmen darf. Das Gute daran ist, 
dass sie durch ihre vorbildliche Leistung keine Kosten auf sie zukommen werden. 
Ihre Mutter wird sich freuen, wenn sie erfährt, was für eine tolle Chance sie nun hat. 
Derzeit läuft es echt gut. Auch die Kosten der Wohnung ihrer Familie werden 
ab nächstem Monat von der Stadt übernommen. Das ist nur möglich, da sie seit fast 
fünf Jahren kein Auto mehr als Familie nutzen.

\section{Konsequenzen} \label{sec:szenkonsequenzen}

Um eine derartige Dystopie zu verhindern, sollten wir als Gesellschaft 
schleunigst Maßnahmen treffen,
um umweltfreundliche Lebensbedingungen in der Zukunft gewährleisten zu können.

\section{Fragen}

Staatliche Maßnahmen / Vorschriften zur Sicherung der Volksgesundheit:

\begin{itemize}
    \item Könnt ihr euch vorstellen, dass in näherer Zukunft durch Vorschrift
    oder einfach aus Notwendigkeit zum Selbstschutz aus Luftverschmutzungsgründen 
    das Tragen von Schutzmasken auch in europäischen Städten zur Norm wird,
    wie es beispielsweise in chinesischen Städten bereits schon vor der Corona-Krise praktiziert wurde?
    \item Weiters, könnt ihr euch eine verpflichtende Gesundheitscheck-App, wie in unserem Szenario, vorstellen?
    Was haltet ihr von (finanziellen) Boni/Strafen für das (Nicht)-Einhalten der Ratschläge der App?
    Vielleicht eine Krankenversicherung mit Bonus-/Malus-System à la Autoversicherung?
    Was passiert dann mit Menschen mit schlechter genetischer Disposition, z.B. Erbkrankheiten?
    \item Denkt ihr die Kosten unseres Pension- \& Gesundheitssystem werden auch in einer gealterten Gesellschaft in 2040
    noch tragbar sein, oder was muss verändert werden, um das zu erreichen?
\end{itemize}

\end{document}