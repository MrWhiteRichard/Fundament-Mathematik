%\documentclass[twoside, a4paper, DIV=11, bibliography=totocnumbered]{scrreprt}
\documentclass[twoside, a4paper, DIV=11, open=any, bibliography=totoc]{scrbook}

\usepackage{url}
\usepackage{hyperref}
\usepackage{graphicx}
\usepackage[all]{hypcap}
\usepackage[utf8]{inputenc}
\usepackage[ngerman]{babel}
\usepackage[skipabove=\baselineskip,skipbelow=\baselineskip,outermargin=30pt,innermargin=30pt]{mdframed}

\usepackage{mdframed}

\usepackage{blindtext} % remove this one, only for template/demonstration

\newcommand{\Quote}[1]{\glqq #1\grqq{}}

\hypersetup
{
 pdfauthor={Vorname Name},
 pdftitle={Seminararbeit ... },
 pdfkeywords={...},
 colorlinks=true,
 citecolor=black,
 filecolor=black,
 linkcolor=black,
 urlcolor=black
}

\KOMAoption{headings}{normal}

\tolerance=4000
\emergencystretch=20pt

\begin{document}



\section*{Dystopie} \label{sec:szendystopie}

Der Tag fängt an wie jeder andere Tag auch. Es gibt, wie sonst auch immer,
Frühstück und Opa liest wieder
seine Nachrichten. "Wisst ihr Kinder", fängt Opa wieder an zu reden, "früher hatten
wir noch ganz andere
Probleme... Da mussten wir uns noch mit den wichtigen Fragen herumschlagen und
Eigeninitiative besitzen.
Da war es nicht so einfach, herauszufinden was man mal werden will. Wir mussten
selbst herumprobieren und
uns fragen, was wirklich zu uns passt und nicht wie ihr heute durch Computer direkt
in eure Leidenschaften
eingeteilt werdet und dann nur noch das Lernen. Wusstet ihr, dass ich früher mal
Französisch sprechen konnte?"
Diese Geschichte erzählt Opa besonders gerne.
"Früher war die Welt noch schwer, die
Jugend von Heute ist so
verwöhnt, bla bla bla ..." Wenn er wüsste, wie es draußen wirklich ist und sich
nicht von seiner
populistischen Propaganda berieseln lassen würde.
Nach seiner Lungenkrebsdiagnose darf er nämlich nicht mehr an die frische Luft,
wobei ja man von frischer Luft nicht mehr wirklich sprechen kann.
Der Smog der sich jeden Tag über die Stadt niedersenkt ist bereits für
junge, gesunde Erwachsene gesundheitsschädlich, für kranke Alte ist diese
Belastung demnach absolut lebensbedrohlich.
In Geschichte haben wir gelernt, dass es nicht normal ist, dass man in ummauerten
Gebieten wohnen muss, damit
man sicher draußen sein kann. Wir haben auch gelernt, dass es früher über 1 Mrd.
Menschen in Afrika gegeben hat.
Dass diese Menschen jetzt dort nicht mehr überleben können, sei die Schuld von
Indien und China.
Was ich nicht ganz
verstehe ist, wie sie es früher gemacht haben. Wie soll man bitte draußen sicher
sein, wenn man nicht in einem
reichen Viertel wohnt? Unsere Bezirkswache hat so schon viel zu viel zu tun, doch
ohne die Mauer wäre alles
unmöglich zu regeln. Ich beneide die Menschen draußen vor der Mauer manchmal.
Sie sind zwar wohl nicht so reich wie wir, doch können sie wenigstens frei
herumlaufen und was soll
man denen schon antun. Sie haben ja nichts, das man stehlen könnte, oder für das
man sie entführen würde.
Naja, ab in die Schule. 
Heute ist im Wetterbericht wieder Maskenstufe III angekündigt worden,
so schlimm war es schon lange nicht mehr. Den Gestank unter diesen Masken hält man kaum
aus, aber was macht man nicht alles für seine Lunge.
Heute ist ja wieder einer dieser "Zukunftstests", wie es
die Lehrkräfte immer nennen.
Dass diese nicht funktionieren, weiß eigentlich jeder, doch trotzdem werden sie
durchgezogen und alle, die
von denen nicht betroffen sind finden sie super. Anscheinend hatte ich bisher immer
eine ausgeprägte Leidenschaft für Mathe, was hieß, dass ich 10 Stunden pro Woche in
dem Fach verbringen musste,
welches mir am wenigsten Spaß machte. Hoffentlich wird es dieses mal Sport, aber
bei dem Test kann man gar nicht
sagen, was schlussendlich rauskommt. Eine Freundin von mir hat mal versucht nicht
ehrlich zu antworten, um
mit mir in einer Klasse zu sein... Dass das nicht gut ausging könnt ihr euch ja
denken. Mittlerweile kennt
sie sich zwar super in Bergbau aus, aber wer braucht das heutzutage schon noch?
Wird doch eh alles recycled und
nicht neu abgebaut... Aber das Programm sei die Zukunft haben sie uns vorgepredigt.
Ihr werdet alle glücklich,
haben sie gesagt. Dass es nicht funktioniert, das übersieht jeder gerne. Wir sind
ja auch nur verwöhnte
Jugendliche, die nicht wissen, was Leiden wirklich ist und was wichtig im Leben
ist.
Ich freue mich, wenn der Tag vorbei ist.


Beim Frühstück checke ich ich schnell mein Tablet, um die neuesten Wetterinformationen
einzuholen. Für heute ist seit langem mal wieder Maskenstufe III vorgeschrieben, 
so schlimm war die Luftverschmutzung erst einmal im letzten Sommer.
Hoffentlich haben wir die Spezialmasken noch irgendwo zuhause verstaut,
ansonsten sind wir die nächsten Tage hier eingeschlossen. In der dritten Lage
der alten Kommode beim Vorzimmerspiegel wird sie schließlich fündig, gerade noch
rechtzeitig um Max die Maske für den Schulweg mitzugeben.
Ein weiterer Blick auf die vorgeschriebene Gesundheitsapp verrät mir, dass
es heute wieder Linsen mit Reis gibt. Das ist jetzt schon das dritte Mal in den
letzten zehn Tagen! Am Anfang fanden wir alle die neuen Essenspläne noch toll,
viel gesünder würden wir uns ernähren, abwechslungsreicher und wir müssten uns
gar keine Gedanken mehr darüber machen, was wir morgen kochen wollen.
Inzwischen glaube ich aber, dass das ganze ein ziemlicher Schuss in den Ofen war.
Jede Woche der selbe, langweilige Fraß und viel gesünder 
fühl ich mich auch nicht als zuvor.

In dem Moment schaut mir Papa über die Schulter.
Schon wieder Linsen mit Reis, schimpft er. Jetzt verbieten sie mir schon meine
Wohnung zu verlassen und nicht mal mehr das eigene Essen kann man sich aussuchen!
Julia, lange halt ich das nicht mehr aus! Kannst du nicht mal wieder bei unserem
alten Freund im 18. vorbeischaun? Der alte Hase hat sicher noch irgendwo ein
gutes Stück Fleisch gebunkert, so wie ich ihn kenne.

Papa, du weißt ja, dass mir das selber auch auf den Keks geht. Aber zahlt sich
das aus, dafür in die äußeren Bezirke zu fahren? Du weißt ja selbst, wie es
da zugeht. Anscheinend seien letzte Woche 15 Menschen bei einer
Demonstration getötet worden und sie hätten es fast geschafft, bis in die
Innenstadt vorzudringen. Wenn das
passiert, dann werde ich sicherlich wegziehen, denn ohne diese Mauer zwischen uns
und denen, kann und will ich
nicht leben. 

Wien wird es eh nicht mehr lange schaffen, denn 32 Millionen Menschen
sind einfach zu viel.
Was haben die ganzen Leute sich gedacht, als sie hierher gekommen sind? Klar sind
die Lebensbedingungen anderswo
nicht besser, aber wieso genau hier? Vor der ersten Klimawelle hatten wir ja sogar
noch überlegt aufs Land zu
ziehen, aber glücklicherweise kam es nicht so weit... Sonst hätten sie uns auch
sicherlich ausgeplündert.
Wenn es nach mir gegangen wäre, dann hätten diese ganzen Leute in Afrika bleiben
können und alles wäre super.
Dass wir alle Schuld am Klimawandel sind, halte ich für ausgeschlossen. Ich habe
mal gelesen, dass China und
Indien für 98\% aller Umweltschäden verantwortlich sind - wieso sollen wir also an
den Auswirkungen leiden?
Glücklicherweise wurde das mit den Flüchtlingsbezirken gut umgesetzt... Als ob ich
neben denen wohnen will?
Ich habe gehört, dass letztens wieder ein Virus im 26. Bezirk ausgebrochen sei und
dabei 400 Menschen gestorben
sind. Da sind sie sicher selbst Schuld, denn das ihnen zugeteilte Wasser
verschwenden die immer extrem.
Bald soll eh wieder eine Umstrukturierung stattfinden. Hoffentlich wird Wien, dann
wieder ein Stück sicherer.
Habe gehört, sie wollen härtere Strafen für Diebstahl und Armut einführen. Endlich,
dann werden diese ganzen
Leute sicher nach Deutschland weiterziehen. Mir wurde mal gesagt, dass es dort noch
viel schlimmer zugeht.
Die Städte seien komplett überlaufen und keinerlei Sicherheitsmaßnahmen wurden
ergriffen. Es sei nicht
ungewöhnlich, dass man einfach auf offener Straße ausgeraubt wird... Puh, bin ich
froh, dass unsere Regierung
das richtig geregelt hat. Jetzt aber los, ich will meine U-Bahn nicht verpassen.
Immer diese Leidensbeklagungen von Opa... Das geht mir mittlerweile richtig auf die
Nerven. Er versteht einfach
nichts. Kein bisschen Mitgefühl bekommt er von mir, denn seine Generation ist ja
Schuld, dass wir jetzt hier
in dieser Lage feststecken. Hätten sie direkter gehandelt und nicht weiter mit
diesen Umweltverschmutzern
geredet, dann wäre es nie so weit gekommen. Über 98\% der Verschmutzungen ist ja nur
aus den zwei Ländern gekommen,
wieso hat man sie dann nicht direkt verurteilt? Zu egoistisch waren sie und hatten
keinerlei Rücksicht auf
andere Menschen und die Welt allgemein. Zum Glück studiere ich Umweltressourcen,
denn sonst würde ich ja auch
nur blind dem folgen, was die Regierung so verzapft. Als ob ich wirklich glauben
würde, dass die Flüchtlinge
schuld an der Vergiftung der Donau sind... So ein populistisches Gedankengut kommt
bei meinem Opa natürlich
super an. Ich versuche ihm zwar zu erklären, dass das alles nur von natürlichem
Algenwachstum kommt, aber
das ist für ihn "zu weit hergeholt" und er fängt wieder mit seinen befeindlichenden
Beschimpfungen an.
Es stimmt zwar, dass die meisten Straftaten in den äußeren Bezirken verübt werden
und diese Leute einfach
den Drang dazu haben, Dinge kaputtzumachen und alles zu verschmutzen, aber ich
denke, dass die Straftaten auch
mit einer Einmischung von Indien zu tun hat. Ich habe mal gehört, dass sehr viel
Unruhe durch die Propaganda
von Außen entsteht und wir endlich eine Lösung für diese Einmischung finden müssen.

\end{document}