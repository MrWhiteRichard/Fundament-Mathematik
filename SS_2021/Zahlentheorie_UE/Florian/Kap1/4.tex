% -------------------------------------------------------------------------------- %

\begin{exercise}

Zeigen Sie, dass für alle $n \geq 1$ gilt:
$(n! + 1, (n+1)! + 1) = 1$.

\end{exercise}

% -------------------------------------------------------------------------------- %

\begin{solution}

Um eine erste Idee zur Lösung zu erhalten, betrachten wir
zuerst einige Beispiele für kleine $n$:

\begin{align*}
  n &= 1: (2, 3) = 1 \\
  n &= 2: (3, 7) = 1 \\
  n &= 3: (7, 25) = (7, 5 \cdot 5) = 1 \\
  n &= 4: (25, 121) = (5 \cdot 5, 11 \cdot 11) = 1 \\
  n &= 5: (121, 721) = (11 \cdot 11, 7 \cdot 103) = 1
\end{align*}

Sei nun angenommen, dass $d := (n! + 1, (n+1)! + 1)  > 1$.

Da $(n+1)! + 1 - (n! + 1) = n!\cdot n$ folgt $d | n! \cdot n$.

Nun unterscheiden wir zwei Fälle:

\begin{itemize}
  \item[$(d,n) = 1$:] In dem Fall erhalten wir direkt $d | n!$ und 
  mit $d | (n! + 1 - n!) = 1$ einen Widerspruch!
  \item[$(d,n) > 1$:] In diesem Fall folgt aus $(d,n) | d$ und $d | n! + 1$, 
  sowie $(d,n) | n$ und $n | n!$,
  dass $(d,n) | (n! + 1 - n!) = 1$ mit Widerspruch zu $(d,n) > 1$!
\end{itemize}

\end{solution}

% -------------------------------------------------------------------------------- %
