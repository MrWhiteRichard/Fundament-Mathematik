% -------------------------------------------------------------------------------- %

\begin{exercise}

Zeigen Sie, dass

\begin{enumerate}
  \item $p^2 + 2$ keine Primzahl ist für $p > 3$ prim,
  \item $n^4 + 4^n$ keine Primzahl ist für $n > 1$.
\end{enumerate}

\end{exercise}

% -------------------------------------------------------------------------------- %

\begin{solution}

\phantom{}

\begin{enumerate}
  \item Wir unterscheiden zwei Fälle:
  \begin{itemize}
    \item[$p \equiv 1 \mod{3}$:]
    Dann gilt
    \begin{align*}
      p^2 + 2 \equiv 1 \cdot 1 + 2 \equiv 0 \mod{3}
    \end{align*}
    und damit $3 | p^2 + 2$ und $p^2 + 2 \notin \P$.
    \item[$p \equiv 2 \mod{3}$:]
    Dann gilt
    \begin{align*}
      p^2 + 2 \equiv 2 \cdot 2 + 2 \equiv 0 \mod{3}
    \end{align*}
    und wieder $3 | p^2 + 2$.
  \end{itemize}
  \item 
  Für gerade $n$ klarerweise auch $4^n + n^4$ gerade und sicher keine Primzahl.

  Daher müssen wir nur noch ungerade $n$ betrachten.


  Sei $n = 2k - 1$ mit $k \geq 1$:

  \begin{align*}
    n^4 + 4^n &= (n^2 + 2^n)^2 - 2n^2\cdot2^n = (n^2 + 2^n)^2 - (2^k n)^2 \\
    &= (n^2 + 2^n - 2^k n)\underbrace{(n^2 + 2^n + 2^k n)}_{>1}.
  \end{align*}

  Es bleibt also noch zu zeigen, dass der erste Faktor nicht 1 sein kann.
  Für $n \leq 2^{k-1}$ gilt

  \begin{align*}
    n^2 + 2^n - 2^k n \geq n^2 > 1.
  \end{align*}

  Für $n \geq 2^k$ gilt

  \begin{align*}
    n^2 + 2^n - 2^k n = 2^n + n(n - 2^k) \geq 2^n > 1.
  \end{align*}

  Für $2^{k-1} < n < 2^k$ gilt

  \begin{align*}
    n^2 + 2^n - 2^k n = 2^n + n(n - 2^k) > 2^n + (2^k-1)(2^{k-1} - 2^k)
    = 2^n + 2^n - 2^{n+1} + 2^k - 2^{k-1} = 2^k - 2^{k-1} \geq 1.
  \end{align*}

\end{enumerate}


\end{solution}

% -------------------------------------------------------------------------------- %
