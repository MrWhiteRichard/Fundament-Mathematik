% -------------------------------------------------------------------------------- %

\begin{exercise}

Lösen Sie mit Hilfe der letzten Aufgabe:
\begin{enumerate}[label = (\alph*)]
  \item Auf wieviel Nullen endet $(169!)$?
  \item $\sqrt[n]{n!} \leq \prod_{p|n}p^{\frac{1}{p-1}}$.
\end{enumerate}

\end{exercise}

% -------------------------------------------------------------------------------- %

\begin{solution}

\phantom{}

\begin{enumerate}[label = (\alph*)]
  \item Wir zählen also, wie oft der Faktor $10$ in $(169!)$ vorkommt,
  was genau dem Minimum der Anzahl der Zweier- und Fünfer-Faktoren entspricht.
  \begin{align*}
    \sum_{k \geq 1} \left\lfloor \frac{169}{2^k}\right\rfloor
    &= 84 + 42 + 21 + 10 + 5 + 2 + 1 = 165 \\
    \sum_{k \geq 1} \left\lfloor \frac{169}{5^k}\right\rfloor
    &= 33 + 6 + 1 = 40.
  \end{align*}
  Also endet $(169!)$ auf $40$ Nullen.
  \item
  \begin{align*}
    \sqrt[n]{n!} \leq \prod_{p|n}p^{\frac{1}{p-1}}
    \iff n! \leq \prod_{p|n}p^{\frac{n}{p-1}}
  \end{align*}
  Es gilt zusätzlich mit der geometrischen Summenformel
  \begin{align*}
    \sum_{k \geq 1} \left\lfloor \frac{n}{p^k}\right\rfloor
    \leq n\sum_{k \geq 1} \frac{1}{p^k}
    = n\frac{1/p}{1 - 1/p} = \frac{n}{p - 1}
  \end{align*}
  und damit mit der vorigen Aufgabe
  \begin{align*}
    n! = \prod_{p|n}p^{\sum_{k \geq 1} \left\lfloor \frac{n}{p^k}\right\rfloor}
    \leq \prod_{p|n}p^{\frac{n}{p - 1}}.
  \end{align*}
\end{enumerate}

\end{solution}

% -------------------------------------------------------------------------------- %
