% -------------------------------------------------------------------------------- %

\begin{exercise}[Legendre]

Zeigen Sie, dass die Zahl $n!$ den Primfaktor $p$ genau

\begin{align*}
  \sum_{k \geq 1} \left\lfloor \frac{n}{p^k}\right\rfloor
\end{align*}
Mal enthält.
\end{exercise}

% -------------------------------------------------------------------------------- %

\begin{solution}

Die Zahl $n!$ enthält den Primfaktor $p$ genau $\sum_{i=1}^n \nu_p(i)$ Mal.
Die Zahlen bis $n$, welche den Primfaktor $p$ mindestens einmal enthalten sind genau
$p,2p,\dots,\left\lfloor \frac{n}{p}\right\rfloor p$.
Allgemein gibt es demnach genau $\left\lfloor \frac{n}{p^k}\right\rfloor$ Zahlen bis $n$,
welche den Primfaktor $p$ mindestens $k$ Mal enthalten.
Summieren wir diese Zahlen auf, zählen wir also genau die Vielfachheit
des Primfaktors $p$ ab.

Damit gilt

\begin{align*}
  \sum_{i=1}^n \nu_p(i) = \sum_{k \geq 1}\sum_{\substack{i=1 \\ p^k | i}}^n 1
  = \sum_{k \geq 1} \left\lfloor \frac{n}{p^k}\right\rfloor.
\end{align*}

\end{solution}

% -------------------------------------------------------------------------------- %
