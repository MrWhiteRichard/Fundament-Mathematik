% -------------------------------------------------------------------------------- %

\begin{exercise}

Berechnen Sie $\left(\frac{70}{97}\right), \left(\frac{-14}{83}\right)$
und $\left(\frac{55}{89}\right)$.

\end{exercise}

% -------------------------------------------------------------------------------- %

\begin{solution}

\begin{align*}
    \left(\frac{70}{97}\right) &= 
    \left(\frac{2}{97}\right)\left(\frac{5}{97}\right)\left(\frac{7}{97}\right) \\
    &= \left(\frac{97}{5}\right)\left(\frac{97}{7}\right)
    = \left(\frac{2}{5}\right)\left(\frac{-1}{7}\right)
    = (-1)(-1) = 1, \\
    \left(\frac{-14}{83}\right) &= 
    \left(\frac{2}{83}\right)\left(\frac{-7}{83}\right) \\
    &= (-1)\left(\frac{-1}{83}\right)(-1)\left(\frac{83}{7}\right) \\
    &= -\left(\frac{2}{7}\right) = 1, \\
    \left(\frac{55}{89}\right) &= \left(\frac{5}{89}\right)\left(\frac{11}{89}\right) \\
    &= \left(\frac{89}{5}\right)\left(\frac{89}{11}\right) \\
    &= \left(\frac{-1}{5}\right)\left(\frac{1}{11}\right) = 1.
\end{align*}

\end{solution}

% -------------------------------------------------------------------------------- %
