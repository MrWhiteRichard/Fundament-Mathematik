% -------------------------------------------------------------------------------- %

\begin{exercise}

Zeigen Sie, dass es unendlich viele Primzahlen der Form $4k+1$ gibt.

\end{exercise}

% -------------------------------------------------------------------------------- %

\begin{solution}

Sei wieder angenommen, dass die Menge endlich wäre, also

\begin{align*}
    \{p \in \P: p \equiv 1 \mod{4}\} = \{p_1,\dots,p_k\},
\end{align*}

wobei $p_k$ die größte Primzahl der From $4k+1$ bezeichnet.

Nun betrachte die Zahl $n := (p_k!)^2 + 1$. Jeder Primfaktor dieser Zahl ist
größer als $p_k$. Sei $q$ der kleinste Primfaktor von $n$. Dann folgt

\begin{align*}
    (p_k!)^2 \equiv -1 \mod{q}
\end{align*}

und da $q \nmid (p_k!)$ laut dem kleinen Fermat $(p_k!)^{q-1} \equiv 1 \mod(q)$.

Daraus schließen wir

\begin{align*}
    1 \equiv (p_k!)^{q-1} = ((p_k!)^2)^{\frac{q-1}{2}} \equiv (-1)^{\frac{q-1}{2}} \mod{q}
\end{align*}

und da $q > 2$ gilt sogar Gleichheit. 
Schließlich erhalten wir mit $\frac{q-1}{2} = 2k \iff q = 4k + 1$ einen Widerspruch
dazu, dass $p_k$ die größte Primzahl der Form $4k+1$ ist.

\end{solution}

% -------------------------------------------------------------------------------- %
