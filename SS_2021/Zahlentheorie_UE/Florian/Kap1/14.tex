% -------------------------------------------------------------------------------- %

\begin{exercise}

Sei $\alpha \beta = \gamma^n$, wobei $\alpha, \beta, \gamma \in \Z[i]$
und $\alpha$ und $\beta$ relativ prim zueinander sind.

Zeigen Sie, dass Gaußsche Zahlen $\varepsilon \in \{\pm 1, \pm i\}$ und
$\delta \in \Z[i]$ existieren, sodass $\alpha = \varepsilon \delta^n$.

\end{exercise}

% -------------------------------------------------------------------------------- %

\begin{solution}

Sei $\varepsilon_1 \prod_{i=1}^n p_i^{e_i}$ die Primfaktorzerlegung 
von $\gamma$ in $\Z[i]$ und damit

\begin{align*}
    \alpha \beta = \varepsilon_1^n \prod_{i=1}^n p_i^{n \cdot e_i}.
\end{align*}

Nachdem $\alpha$ und $\beta$ relativ prim sind, gilt für alle $p_i, i = 1,\dots,n$
entweder $p_i | \alpha$ oder $p_i | \beta$. 

Definiere 

\begin{align*}
    q_i := \begin{cases}
        p_i, & \text{falls } p_i | \alpha \\
        1, & \text{falls } p_i | \beta.
    \end{cases}
\end{align*}

Dann folgt 

\begin{align*}
    \alpha = \varepsilon \prod_{i=1}^n q_i^{n e_i}
    = \varepsilon \left(\underbrace{\prod_{i=1}^n q_i^{e_i}}_{:=\delta}\right)^n.
\end{align*}


\end{solution}

% -------------------------------------------------------------------------------- %
