% -------------------------------------------------------------------------------- %

\begin{exercise}

Zeigen Sie, dass $1 + \frac{1}{2} + \frac{1}{3} + \dots + \frac{1}{n}$ für $n > 1$
keine ganze Zahl ist.

\end{exercise}

% -------------------------------------------------------------------------------- %

\begin{solution}
Wir definieren $f(n) := [1,\dots,n]$ und erhalten
\begin{align*}
  \sum_{i=1}^n \frac{1}{i} = \frac{\sum_{i=1}^n \frac{f(n)}{i}}{f(n)}
\end{align*}
Wir zeigen, dass die Summe im Zähler eine ungerade Zahl sein muss und 
da der Nenner gerade ist, somit der
Bruch keine ganze Zahl sein kann.

Da $f(n)$ das kleinste gemeinsame Vielfache von
$1,\dots,n$ ist, sind alle Summanden im Zähler ganze Zahlen. Definiere $k := \lfloor \log_2(n) \rfloor$.
Dann ist $\frac{f(n)}{2^k}$ der einzige ungerade Summand in der Summe, da

\begin{align*}
  \nu_2(f(n)) = \max_{i=1}^n \nu(i) =  \lfloor \log_2(n) \rfloor
\end{align*}

 und die Summe
muss insgesamt ungerade sein. Alle anderen Summanden sind gerade, da
 $2^k$ die größte Zweierpotenz kleiner $n$ ist und somit $\nu_2(i) < k$
für alle $i = 1,\dots,2^k-1,2^k+1,\dots,n$.

\end{solution}

% -------------------------------------------------------------------------------- %
