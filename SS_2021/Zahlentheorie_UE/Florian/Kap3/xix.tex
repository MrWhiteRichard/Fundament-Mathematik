% -------------------------------------------------------------------------------- %

\begin{exercise}

Es seien $a$ und $b$ zwei teilerfremde natürliche Zahlen und $m,n,r,s \in \Z$,
sodass $ms - nr = \pm 1$. Berechnen Sie $(ma + nb, ra + sb)$.

\end{exercise}

% -------------------------------------------------------------------------------- %

\begin{solution}
Gelte o.b.d.A. $ms - nr = 1$.

Wir berechnen 

\begin{align*}
    -r(ma + nb) + m(ra + sb) = b(ms - nr) = b \\
    s(ma + nb) - n(ra + sb) = a(ms - nr) = a.
\end{align*}

Da $(a,b) = 1$ existieren $k, l \in \Z$ mit $ka + lb = 1$.

Damit erhalten wir

\begin{align*}
    1 &= ka + lb = ks(ma + nb) - kn(ra + sb) - lr(ma + nb) + lm(ra + sb) \\
    &= \underbrace{(ks - lr)}_{\in \Z}(ma + nb) + 
        \underbrace{(lm - kn)}_{\in \Z}(ra + sb),
\end{align*}

woraus $(ma + nb, ra + sb) = 1$ folgt.

\end{solution}

% -------------------------------------------------------------------------------- %
