% -------------------------------------------------------------------------------- %

\begin{exercise}

Komet A ist alle 5 Jahre von der Erde aus sichtbar und wurde das letzte
Mal vor einem Jahr gesehen. Komet B ist alle 8 Jahre sichtbar und wurde
das letzte Mal vor 2 Jahren gesehen. Komet C ist alle 11 Jahre sichtbar
und wurde das letzte Mal vor 8 Jahren gesehen. In wie vielen Jahren
können alle Kometen frühestens gleichzeitig gesehen werden?
Und in wie vielen Jahren darauf das nächste Mal?

\end{exercise}

% -------------------------------------------------------------------------------- %

\begin{solution}

Die Lösung des unteren Systems an Kongruenzen entspricht
genau der vergangenen Zeit zwischen der letzten gleichzeitigen Sichtung
und jetzt. Nachdem die Moduln paarweise teilerfremd sind, ist der chinesische
Restsatz anwendbar, welcher eine Lösung alle $5\cdot 8 \cdot 11 = 440$
Jahre liefert.

\begin{align*}
    x \equiv 1 \mod{5} \\
    x \equiv 2 \mod{8} \\
    x \equiv 8 \mod{11} \\
\end{align*}

Für den chinesischen Restsatz lösen wir vorerst

\begin{align*}
    88b_1 &\equiv 1 \mod{5} \iff b_1 \equiv 2 \mod{5} \\
    55b_2 &\equiv 1 \mod{8} \iff b_2 \equiv 7 \mod{8} \\
    40b_3 &\equiv 1 \mod{11} \iff b_3 \equiv 8 \mod{11}
\end{align*}

und erhalten 

\begin{align*}
    x_0 \equiv 2 \cdot 1\cdot 88 + 7 \cdot 2 \cdot 55 + 8 \cdot 8 \cdot 40
    \equiv 426 \mod{440}.
\end{align*}

Also findet die nächste gleichzeitige Sichtung in $14$ Jahren statt
und alle zukünftigen gleichzeitigen Sichtungen in $14 + 440k$ Jahren.

\end{solution}

% -------------------------------------------------------------------------------- %
