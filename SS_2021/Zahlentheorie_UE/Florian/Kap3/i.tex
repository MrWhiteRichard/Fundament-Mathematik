% -------------------------------------------------------------------------------- %

\begin{exercise}

Bestimmen Sie alle Lösungen $x, y > 0$ des Gleichungssystems

\begin{align*}
    \begin{cases}
        x + y = 5432 \\
        \kgV(x,y) = 223020.
    \end{cases}
\end{align*}

\end{exercise}

% -------------------------------------------------------------------------------- %

\begin{solution}

\begin{align*}
    223020 &= 2^2\cdot 3^3 \cdot 5 \cdot 7 \cdot 59 \\
    5432 &= 2^2 \cdot 7 (2\cdot 97).
\end{align*}

Daraus schließen wir, dass die Faktoren $3^3,5,59$ nur entweder in $x$
oder $y$ enthalten sein können, da die Summe diese Faktoren nicht teilt,
während die anderen Primfaktoren $2^2,7$ von $223020$ sowohl $x$ als auch $y$ teilen müssen.


Also haben wir das reduzierte Problem

\begin{align*}
    x &= 2^2\cdot 7 n \\
    y &= 2^2\cdot 7 m \\
    n + m &= 194 \\
    n\cdot m &= 7965 \\
    (n,m) &= 1.
\end{align*}

Die einzigen Lösungskandidaten (bis auf Vertauschung von $n$ und $m$) sind dann
$(n = 59, m = 3^3\cdot 5 = 135), (n = 59\cdot 3 = 177, m = 3^2 \cdot 5 = 45)$,
wovon nur ersterer $n+m = 194$ erfüllt, während jede andere Aufteilung der Primfaktoren
klarerweise die Summenbedingung verletzt.

Damit sind die einzigen Lösungen $(x,y) = (1652,3780)$ und $(x,y) = (3780,1652)$.
\end{solution}

% -------------------------------------------------------------------------------- %
