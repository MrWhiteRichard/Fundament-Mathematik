% -------------------------------------------------------------------------------- %

\begin{exercise}

Zeigen Sie, dass für alle $a,b,c > 0$

\begin{align*}
    [a,b,c] = \frac{abc(a,b,c)}{(a,b)(b,c)(c,a)}.
\end{align*}

\end{exercise}

% -------------------------------------------------------------------------------- %

\begin{solution}

Wir nutzen die Tatsache, dass wir $[a,b,c]$ durch

\begin{align*}
    \forall p \in \P: \nu_p([a,b,c]) = \max\{ \nu_p(a), \nu_p(b), \nu_p(c)\}
\end{align*}

charakterisieren können.

Sei dafür $p \in \P, p | abc$ beliebig. Gelte o.b.d.A $\nu_p(a) \leq \nu_p(b) \leq \nu_p(c)$.
Dann folgt

\begin{align*}
    \nu_p\left(\frac{abc(a,b,c)}{(a,b)(b,c)(c,a)}\right)
    &= \nu_p(a) + \nu_p(b) + \nu_p(c) + \min\{\nu_p(a), \nu_p(b), \nu_p(c)\} \\
    &- \min\{\nu_p(a), \nu_p(b)\} - \min\{\nu_p(a), \nu_p(c)\}
    - \min\{\nu_p(b), \nu_p(c)\} \\
    &= \nu_p(a) + \nu_p(b) + \nu_p(c) + \nu_p(a) - \nu_p(a) - \nu_p(a) - \nu_p(b) \\
    &= \nu_p(c) = \max\{\nu_p(a), \nu_p(b), \nu_p(c)\} = \nu_p([a,b,c]).
\end{align*}

Damit folgt mit der eindeutigen Primfaktorzerlegung $[a,b,c] = \frac{abc(a,b,c)}{(a,b)(b,c)(c,a)}$.

\end{solution}

% -------------------------------------------------------------------------------- %
