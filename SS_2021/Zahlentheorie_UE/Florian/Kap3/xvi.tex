% -------------------------------------------------------------------------------- %

\begin{exercise}

Es sei $p$ eine ungerade Primzahl. Berechnen Sie $\left( \frac{\frac{p+1}{2}}{p} \right)$
und $\left(\frac{\frac{p-1}{2}}{p}\right)$.

\end{exercise}

% -------------------------------------------------------------------------------- %

\begin{solution}

Die Kongruenz $x^2 \equiv \frac{p+1}{2} \mod{p}$ ist genau dann
lösbar, wenn $(2x)^2 \equiv 2 \mod{p}$ lösbar ist. 

Also gilt nach dem zweiten Ergänzungssatz

\begin{align*}
    \left( \frac{\frac{p+1}{2}}{p} \right)
    &= \left(\frac{2}{p}\right)
    = \begin{cases}
        1, & p \equiv \pm 1 \mod{8} \\
        -1, & p \equiv \pm 3 \mod{8}
    \end{cases}.
\end{align*}

Für den zweiten Teil berechnen wir mit beiden Ergänzungssätzen

\begin{align*}
    \left(\frac{\frac{p-1}{2}}{p}\right) =
    \left(\frac{-1}{p}\right)\left( \frac{\frac{p+1}{2}}{p} \right) =
    (-1)^{\frac{p-1}{2}}(-1)^{\frac{p^2-1}{8}}
    = \begin{cases}
        1, & p \equiv 1 \mod{8} \\
        1, & p \equiv 3 \mod{8} \\
        -1, & p \equiv 5 \mod{8} \\
        -1, & p \equiv 7 \mod{8}
    \end{cases}.
\end{align*}

\end{solution}

% -------------------------------------------------------------------------------- %
