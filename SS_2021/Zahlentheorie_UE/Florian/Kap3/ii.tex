% -------------------------------------------------------------------------------- %

\begin{exercise}

Auf wieviel Nuller endet $\frac{500!}{200!}$?

\end{exercise}

% -------------------------------------------------------------------------------- %

\begin{solution}

Wir benutzen wieder Aufgabe 5, um die Anzahl der 5er- und 2er-Faktoren
zu zählen:

\begin{align*}
    \nu_5\left(\frac{500!}{200!}\right)
    &= \sum_{k\geq 1} \left\lfloor \frac{500}{5^k} \right\rfloor -
    \sum_{k\geq 1} \left\lfloor \frac{200}{5^k} \right\rfloor \\
    &= 100 + 20 + 4 - (40 + 8 + 1) = 124 - 49 = 75 \\
    \nu_2\left(\frac{500!}{200!}\right)
    &= \sum_{k\geq 1} \left\lfloor \frac{500}{2^k} \right\rfloor -
    \sum_{k\geq 1} \left\lfloor \frac{200}{2^k} \right\rfloor \\
    &= 250 + 125 + 62 + 31 + 15 + 7 + 3 + 1
    - (100 + 50 + 25 + 12 + 6 + 3 + 1)
    = 495 - 197 = 297.
\end{align*}

Also kommt der Faktor 10 genau 75 Mal in $\frac{500!}{200!}$ vor.

\end{solution}

% -------------------------------------------------------------------------------- %
