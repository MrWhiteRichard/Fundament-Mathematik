% -------------------------------------------------------------------------------- %

\begin{exercise}

Es seien $a$ und $b$ zwei teilerfremde natürliche Zahlen.
Berechnen Sie $(a^3 - b^3, a^2 - b^2)$.

\end{exercise}

% -------------------------------------------------------------------------------- %

\begin{solution}

Wir berechnen $a^3-b^3 = (a-b)(a^2 + ab + b^2)$ und $(a^2 - b^2) = (a-b)(a+b)$.

Um den ggT von $a^2+ab+b^2$ und $a+b$ zu finden berechnen wir vorerst
\begin{align*}
    a^2 + ab + b^2 - b(a+b) &= a^2 \\
    a^2 + ab + b^2 - a(a+b) &= b^2.
\end{align*}

Wegen der Teilerfremdheit von $a$ und $b$ 
existieren $k,l \in \Z$ mit $ka + lb = 1$. Damit erhalten wir

\begin{align*}
    (ka + l b)^2 &= k^2a^2 + l^2b^2 + 2kalb = 1 \\
    k^2(1 + 2lb)a^2 + l^2(1 + 2ka)b^2 &= k^2a^2 + l^2b^2 + 2kalb(ka + l b)
    = 1
\end{align*}

und schließlich

\begin{align*}
    1 &= k^2(1 + 2lb)[a^2 + ab + b^2 - b(a+b)] + l^2(1 + 2ka)[a^2 + ab + b^2 - a(a+b)] \\
    &=  [k^2(1 + 2lb) + l^2(1 + 2ka)](a^2 + ab + b^2) - [b k^2(1 + 2lb) + al^2(1 + 2ka)](a+b).
\end{align*}

Also folgt $(a^2 + ab + b^2, a + b) = 1$ und schließlich $(a^3 - b^3, a^2 - b^2) = a - b$.

\end{solution}

% -------------------------------------------------------------------------------- %
