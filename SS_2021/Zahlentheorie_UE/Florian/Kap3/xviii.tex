% -------------------------------------------------------------------------------- %

\begin{exercise}

Es seien $p$ und $q$ Primzahlen, sodass $p = 2q + 1$ und $m$ sei eine
natürliche Zahl, welche $1 \leq m \leq p - 2$ erfüllt.
Zeigen Sie, dass $m$ eine Primitivwurzel modulo $p$ ist, genau dann wenn
$m$ ein quadratischer Nichtrest modulo $p$ ist.

\end{exercise}

% -------------------------------------------------------------------------------- %

\begin{solution}

Sei $m$ keine Primitivwurzel. Die Ordnung von $m$ muss ein Teiler von $p - 1 = 2q$ 
sein, die einzigen Kandidaten sind daher $1,2,q$.

Für $\ord(m) = 1$ folgt sofort $m \equiv 1 \mod{p}$, was sicher ein quadratischer Rest modulo $p$ ist.

Für $\ord(m) = 2$ gilt $m^2 - 1 \equiv 0 \mod{p}$.
Das Polynom vom zweiten Grad über dem Körper $\Z_p$ kann maximal 2 Nullstellen
haben; diese sind gegeben durch $m \equiv \pm 1 \mod{p}$.
Dabei ist $m \equiv 1 \mod{p}$ nicht möglich, da sonst $\ord(m) = 1$ gelten würde.
Der Fall $m \equiv -1 \mod{p}$ ist wegen der Voraussetzung 
$1 \leq m \leq p - 2$ ebenso ausgeschlossen.

Es bleibt nur noch der Fall $\ord(m) = q$.
Hierbei wähle eine Primitivwurzel $g$ modulo $p$.
Dann existiert ein $k \in \Z^+$, sodass $g^k \equiv m$ und

\begin{align*}
    1 \equiv m^q \equiv g^{kq} \mod{p}.
\end{align*}

Damit folgt $2q | kq$ und daher $2|k$. Also ist $k/2$ eine ganze Zahl und
$x_0:= g^{k/2}$ erfüllt $x_0^2 \equiv g^k \equiv m \mod{p}$.

Sei umgekehrt $m$ eine Primitivwurzel und angenommen, dass 
ein $x_0$ existiert mit $x_0^2 \equiv m \mod{p}$.
Wir finden ein $k$, sodass $x_0 \equiv m^k$. Damit gilt

\begin{align*}
    m \equiv x_0^2 \equiv m^{2k} \mod{p}
\end{align*}

und weiters $m^{2k - 1} \equiv 1 \mod{p}$.

Daraus folgt $p - 1 | 2k - 1$. Das ist allerdings ein Widerspruch, da $p - 1 = 2q$
gerade ist und $2k - 1$ ungerade ist.

\end{solution}

% -------------------------------------------------------------------------------- %
