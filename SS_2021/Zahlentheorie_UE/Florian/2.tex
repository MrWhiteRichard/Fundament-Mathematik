% --------------------------------------------------------------------------------

\begin{exercise}

Zeigen Sie: Sind $a, b \in \Z^+$, sodass $a|b^2, b^2 | a^3, a^3 | b^4, \dots$ gilt,
dann folgt $a = b$.

\end{exercise}

% --------------------------------------------------------------------------------

\begin{solution}

Sei $a \neq b$. Dann existiert ein $p \in \P$, sodass $\nu_p(a) \neq \nu_p(b)$.
Für den Fall $\nu_p(a) > \nu_p(b)$ finden wir ein $n \in 2\N + 1$, sodass
\begin{align*}
  \nu_p(a) \frac{n}{n+1} > \nu_p(b)
  \iff \nu_p(a^n) > \nu_p(b^{n+1})
  \iff a^n \nmid b^{n+1}
\end{align*}
Sonst gilt $\nu_p(b) > \nu_p(a)$ und wir finden analog ein $n \in 2\N$
und führen obige Rechnung genauso durch.
\end{solution}

% --------------------------------------------------------------------------------
