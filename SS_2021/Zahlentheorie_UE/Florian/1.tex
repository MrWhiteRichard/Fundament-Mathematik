% -------------------------------------------------------------------------------- %

\begin{exercise}

Zeigen Sie, dass

\begin{enumerate}
  \item $p^2 + 2$ keine Primzahl ist für $p > 3$ prim,
  \item $n^4 + 4^n$ keine Primzahl ist für $n > 1$.
\end{enumerate}

\end{exercise}

% -------------------------------------------------------------------------------- %

\begin{solution}

\phantom{}

\begin{enumerate}
  \item Wir unterscheiden zwei Fälle:
  \begin{itemize}
    \item[$p \equiv 1 (3)$:]
    Dann gilt
    \begin{align*}
      p^2 + 2 \equiv 1 \cdot 1 + 2 \equiv 0 (3)
    \end{align*}
    und damit $3 | p^2 + 2$ und $p^2 + 2 \notin \P$.
    \item[$p \equiv 2 (3)$:]
    Dann gilt
    \begin{align*}
      p^2 + 2 \equiv 2 \cdot 2 + 2 \equiv 0 (3)
    \end{align*}
    und wieder $p^2 + 2 \equiv 0 (3)$.
  \end{itemize}
  \item Wir bemerken zuerst
  \begin{align*}
    4^n \equiv \begin{cases}
      1(5), & 2\mid n \\
      4(5), & 2\nmid n
    \end{cases}.
  \end{align*}
  Wir zeigen die Aussage mit Induktion: $4^1 \equiv 4(5)$
  und $4^2 = 16 \equiv 1(5)$. \\
  Für den Induktionsschritt unterscheiden wir zwei Fälle:
  Für $n$ gerade gilt:
  \begin{align*}
    4^n = 4^{n-1}\cdot 4 \equiv 4\cdot 4 \equiv 1 (5).
  \end{align*}
  Für $n$ ungerade gilt:
  \begin{align*}
    4^n = 4^{n-1}\cdot 4 \equiv 1\cdot 4 \equiv 4 (5).
  \end{align*}
  Weiters ist für gerade $n$ klarerweise auch $4^n + n^4$
  gerade und sicher keine Primzahl.
  Damit müssen wir nur noch ungerade $n$ betrachten.

  Wir unterschieden fünf Fälle:
  \begin{itemize}
    \item[$n \equiv 0 (5)$:] Für $n = 5$ gilt
     \begin{align*}
       5^4 + 4^5 = 1649 = 17\cdot97 \notin \P.
     \end{align*}
     Für $n > 5$ ist $2|n$ und somit auch $2|n^4 + 4^n$.
    \item[$n \equiv 1 (5)$:] Dann gilt
     \begin{align*}
       n^4 \equiv 1\cdot1\cdot1\cdot1 \equiv 1 (5)
     \end{align*}
     und da $n$ ungerade ist, gilt
     \begin{align*}
       n^4 + 4^n \equiv 1 + 4 \equiv 0 (5).
     \end{align*}
  \item[$n \equiv 2 (5)$:] Dann gilt
    \begin{align*}
      n^4 + 4^n \equiv 2^4 + 4 = 20 \equiv 0 (5).
    \end{align*}
  \item[$n \equiv 3 (5)$:] Dann gilt
    \begin{align*}
      n^4 + 4^n \equiv 3^4 + 4 = 85 \equiv 0 (5).
    \end{align*}
  \item[$n \equiv 4(5)$:] Dann gilt
    \begin{align*}
      n^4 + 4^n \equiv 4^4 + 4 = 260 \equiv 0 (5).
    \end{align*}

  \end{itemize}

\end{enumerate}


\end{solution}

% -------------------------------------------------------------------------------- %
