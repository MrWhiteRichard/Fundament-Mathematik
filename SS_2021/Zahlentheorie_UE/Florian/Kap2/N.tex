% -------------------------------------------------------------------------------- %

\begin{exercise}

Finden Sie die kleinste positive ganze Zahl $x$, sodass $x$ geteilt durch
$10,9,\dots,2$ die Reste $9,8,\dots,1$ hat.

\end{exercise}

% -------------------------------------------------------------------------------- %

\begin{solution}

Wir lösen also folgendes System an Kongruenzen:
        
\begin{align*}
    x &\equiv -1 \mod{10} \\
    &\vdots \\
    x &\equiv -1 \mod{2}.
\end{align*}

Zuerst bemerken wir, dass wir dieses Problem auf ein System von
Kongruenzen mit Moduln aus den höchsten auftretenden Primpotenzen 
reduzieren können, da für $n = \prod_{i=1}^n p_i^{e_i}$

\begin{align*}
    (\forall i = 1,\dots,n: x \equiv -1 \mod{p_i^{e_i}}) 
    \implies (\forall i = 1,\dots,n: p_i^{e_i} | (x + 1))
    \implies n | (x+1)
\end{align*}
gilt.

Leicht nachprüfbar löst $n = [2,3,4,5,6,7,8,9,10] - 1 = 2519$ die gegebenen Kongruenzen.
Nachdem wir das Problem auf Kongruenzen mit teilerfremden Moduln reduziert haben,
ist der chinesische Restsatz anwendbar, welcher uns eindeutige Lösbarkeit modulo $[2,3,4,5,6,7,8,9,10]$
liefert. Demnach kann es also keine kleinere ganze, positive Lösung geben.



\end{solution}

% -------------------------------------------------------------------------------- %
