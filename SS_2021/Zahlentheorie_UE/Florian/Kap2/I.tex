% -------------------------------------------------------------------------------- %

\begin{exercise}

Finden Sie die kleinste positive ganze Zahl, sodass $x \equiv 5 \mod{12}, 
x \equiv 17 \mod{20}$ und $x \equiv 23 \mod{42}$.

\end{exercise}

% -------------------------------------------------------------------------------- %

\begin{solution}

Da $(12,20) = 4 > 1$ können wir den chinesischen Restsatz nicht direkt anwenden.
Wir können aber manuell die Moduln in ihre Primfaktoren aufteilen und erhalten das System

\begin{align*}
    x &\equiv 2 \mod{3}, \\
    x &\equiv 1 \mod{4}, \\
    x &\equiv 1 \mod{4}, \\
    x &\equiv 2 \mod{5}, \\
    x &\equiv 1 \mod{2}, \\
    x &\equiv 2 \mod{3}, \\
    x &\equiv 2 \mod{7}.
\end{align*}

Nach Entfernen der redundanten Kongruenzen erhalten wir ein
System auf das sich der chinesische Restsatz anwenden lässt:

\begin{align*}
    x &\equiv 2 \mod{3}, \\
    x &\equiv 1 \mod{4}, \\
    x &\equiv 2 \mod{5}, \\
    x &\equiv 2 \mod{7}.
\end{align*}

Wir lösen also

\begin{align*}
    140b_1 &\equiv 1 \mod{3} \iff b_1 \equiv 2 \mod{3}, \\
    105b_2 &\equiv 1 \mod{4} \iff b_2 \equiv 1 \mod{4}, \\
    84b_3 &\equiv 1 \mod{5} \iff b_3 \equiv 4 \mod{5}, \\
    60b_4 &\equiv 1 \mod{7} \iff b_4 \equiv 2 \mod{7}.
\end{align*}

Damit erhalten wir

\begin{align*}
    x \equiv 2\cdot 2 \cdot 140 + 1 \cdot 1 \cdot 105 + 
    4 \cdot 2 \cdot 84 + 2 \cdot 2 \cdot 60 \equiv 317 \mod{420}.
\end{align*}

Nachdem die Lösung modulo 420 eindeutig ist, muss also 317 die kleinste
positive Lösung sein.

\end{solution}

% -------------------------------------------------------------------------------- %
