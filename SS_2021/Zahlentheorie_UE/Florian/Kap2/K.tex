% -------------------------------------------------------------------------------- %

\begin{exercise}

Lösen Sie die folgenden quadratischen Kongruenzen:

\begin{enumerate}[label = (\alph*)]
    \item $x^2 + 5x + 3 \equiv 0 \mod{11}$.
    \item $x^2 + 7x + 4 \equiv 0 \mod{10}$.
    \item $2x^2 + 3x + 7 \equiv 0 \mod{12}$.
\end{enumerate}

\end{exercise}

% -------------------------------------------------------------------------------- %

\begin{solution}

\phantom{}

\begin{enumerate}[label = (\alph*)]
    \item Die Kongruenz ist genau dann lösbar, wenn

    \begin{align*}
        (2x + 5)^2 \equiv 25 - 12 \equiv 2 \mod{11} \\
        (2x + 5)^2 \equiv 25 - 12 \equiv 2 \mod{4}
    \end{align*}

    lösbar sind. Da $11 \equiv 3 \mod{8}$ ist laut dem zweiten
    Ergänzungssatz die Kongruenz nicht lösbar.

    \item Wieder ergänzen wir auf ein vollständiges Quadrat
    und faktorisieren den Modul:

    \begin{align*}
        (2x^2 + 7)^2 \equiv 49 - 16 \equiv 3 \mod{5} \\
        (2x^2 + 7)^2 \equiv 49 - 16 \equiv 5 \mod{8} \\
    \end{align*}

    Da $\left(\frac{3}{5}\right) = \left(\frac{2}{3}\right) = -1$
    ist auch diese Kongruenz nicht lösbar.

    \item 

    \begin{align*}
        2x^2 + 1 \equiv 0 \mod{3} \iff x \equiv \pm 1 \mod{3} \\
        2x^2 + 3x + 3 \equiv 0 \mod{4} \iff x \equiv 1 \mod{4}.
    \end{align*}

    Also erhalten wir die beiden Lösungen $x \equiv 1 \mod{12}$ und
    $x \equiv 5 \mod{12}$.
\end{enumerate}

\end{solution}

% -------------------------------------------------------------------------------- %
