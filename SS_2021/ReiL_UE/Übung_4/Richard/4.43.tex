% -------------------------------------------------------------------------------- %

\begin{exercise}[Implementation Task: $n$-step Algorithm; Exercise 7.2]

With an $n$-step method, the value estimates \textit{do} change from step to step, so an algorithm that used the sum of TD errors (see previous exercise) in place of the error in \eqref{eq:7.2} would atually be a slightly different algorithm.
Would it be a better algorithm of a worse one?
Devise and programm a small experiment to answer this question empirically.

\end{exercise}

% -------------------------------------------------------------------------------- %

\begin{solution}

\phantom{}

\end{solution}

% -------------------------------------------------------------------------------- %
