% --------------------------------------------------------------------------------

\begin{exercise}[Exercise 3.16]

Now consider adding a constant $c$ to all the rewards in an episodic task, such as maze running.
Would this have any effect, or would it leave the task unchanged as in the continuing task above?
Why or why not?
Give an example.    

\end{exercise}

% --------------------------------------------------------------------------------

\begin{solution}

Since we are adding a constant $c \neq 0$, the notion of an absorbing state, that generates only rewards of zero, is not applicable any more.
Furthermore, we should differentiate between episodes by virtue of a (secondary) index $i$.
This has to be taken into account when working with

\begin{align*}
    v_{c, i} \doteq c \gamma^{t+1} \frac{1 - \gamma^{T_i - t - 1}}{1 - \gamma},
\end{align*}

\begin{align*}
    G_{t, i}^\text{new}
    =
    \sum_{k=t+1}^{T_i}
        \gamma^{k-t-1} R_{k, i}^\text{new}
    =
    \sum_{k=t+1}^{T_i}
        \gamma^{k-t-1} (R_{k, i}^\text{old} + c)
    =
    \sum_{k=t+1}^{T_i}
        \gamma^{k-t-1} R_{k, i}^\text{old}
    +
    c \gamma^{t+1}
    \sum_{k=0}^{T_i - t - 1}
        \gamma^k
    =
    G_t^\text{old} + v_{c, i},
\end{align*}

and

\begin{align*}
    v_\pi^\text{new}(s)
    \stackrel{!}{=}
    \E[G_{t, i}^\text{new} \mid S_{t, i} = s]
    =
    \E[G_{t, i}^\text{old} + v_{c, i} \mid S_{t, i} = s]
    =
    \E[G_{t, i}^\text{old} \mid S_{t, i} = s] + v_{c, i}
    =
    v_\pi^\text{old}(s) + v_{c, i}.
\end{align*}

Now, it may occur that for the $i$-th and $j$-th expisode, $T_i \neq T_j$.
In that case we run into an issue because of

\begin{align*}
    v_\pi^\text{new}(s)
    \stackrel{!}{=}
    v_\pi^\text{old} + v_{c, i}
    \neq
    v_\pi^\text{old} + v_{c, j}
    \stackrel{!}{=}
    v_\pi^\text{new}(s).
\end{align*}

\end{solution}

% --------------------------------------------------------------------------------
