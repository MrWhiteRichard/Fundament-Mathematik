% --------------------------------------------------------------------------------

\begin{exercise}[Exercise 3.15]

In the gridworld example, rewards are positive for goals, negative for running into the edge of the world, and zero the rest of the time.
Are the signs of these rewards important, or only the intervals between them?
Prove, using \eqref{eq:2.15}, that adding a constant $c$ to all the rewards adds a constant, $v_c$, to the values of all states, and thus does not affect the relative values of any states under any policies.
What is $v_c$ in terms of $c$ and $\gamma$?

\begin{align} \label{eq:2.15}
    G_t
    \doteq
    R_{t+1} + \gamma R_{t+2} + \gamma^2 R_{t+3} + \cdots
    =
    \sum_{k=0}^\infty
        \gamma^k R_{t+k+1}
\end{align}

\end{exercise}

% --------------------------------------------------------------------------------

\begin{solution}

The signs of the rewards are irrelevant insofar that, if the rewards are altered, as long as the relative distances are perserved they were only shifted by a common constant $c$.
As shown below, this property (existence of a universal shifting constant $v_c$) gets passed on to state value functions (under an arbitrary policy $\pi$).

Let the index $\text{new}$ denote the (by $c$) shifted version of an object and write $\text{old}$ for the unshifted one.
There are $2$ possiblities of proving the desired result.

\begin{enumerate}[label = \arabic*.]

    \item Possiblity (following the specification):
    
    \begin{align*}
        v_c \doteq \frac{c}{1 - \gamma}
    \end{align*}
    
    \begin{align*}
        G_t^\text{new}
        \stackrel
        {
            \eqref{eq:2.15}
        }{=}
        \sum_{k=0}^\infty
            \gamma^k R_{t+k+1}^\text{new}
        =
        \sum_{k=0}^\infty
            \gamma^k (R_{t+k+1}^\text{old} + c)
        =
        \sum_{k=0}^\infty
            \gamma^k R_{t+k+1}^\text{old}
        +
        c
        \sum_{k=0}^\infty
            \gamma^k
        =
        G_t^\text{old} + v_c
    \end{align*}

    \begin{align*}
        v_\pi^\text{new}
        \doteq
        \E_\pi[G_t^\text{new} \mid S_t = s]
        =
        \E_\pi[G_t^\text{old} + v_c \mid S_t = s]
        =
        \E_\pi[G_t^\text{old} \mid S_t = s] + v_c
        =
        v_\pi^\text{old} + v_c
    \end{align*}

    \item Possiblity (using the Bellman equations):
    
    Borrowing from Differential Equations, we first make the Ansatz $v_\pi^\text{new}(s^\prime) = v_\pi^\text{old}(s^\prime) + v_c$.

    \begin{align*}
        v_\pi^\text{new}
        & =
        \sum_{a \in \mathcal A(s)}
            \pi(a \mid s)
            \sum_{\substack{s^\prime \in \mathcal S \\ r \in \mathcal R^\text{new}}}
                p(s^\prime, r \mid s, a)
                [r + \gamma v_\pi^\text{new}(s^\prime)] \\
        & =
        \sum_{a \in \mathcal A(s)}
            \pi(a \mid s)
            \sum_{\substack{s^\prime \in \mathcal S \\ r \in \mathcal R^\text{old}}}
                p(s^\prime, r \mid s, a)
                [(r + c) + \gamma (v_\pi^\text{old}(s^\prime) + v_c)] \\
        & =
        \sum_{a \in \mathcal A(s)}
            \pi(a \mid s)
            \sum_{\substack{s^\prime \in \mathcal S \\ r \in \mathcal R^\text{old}}}
                p(s^\prime, r \mid s, a)
                [(r + \gamma v_\pi^\text{old}(s^\prime)) + (c + \gamma v_c)] \\
        & =
        \sum_{a \in \mathcal A(s)}
            \pi(a \mid s)
            \Bigg (
                \sum_{\substack{s^\prime \in \mathcal S \\ r \in \mathcal R^\text{old}}}
                    p(s^\prime, r \mid s, a)
                    (r + \gamma v_\pi^\text{old}(s^\prime))
                +
                (c + \gamma v_c)
                \underbrace
                {
                    \sum_{\substack{s^\prime \in \mathcal S \\ r \in \mathcal R^\text{old}}}
                        p(s^\prime, r \mid s, a)
                }_1
            \Bigg ) \\
        & =
        \sum_{a \in \mathcal A(s)}
            \pi(a \mid s)
            \sum_{\substack{s^\prime \in \mathcal S \\ r \in \mathcal R^\text{old}}}
                p(s^\prime, r \mid s, a)
                (r + \gamma v_\pi^\text{old}(s^\prime))
        +
        (c + \gamma v_c)
        \underbrace
        {
            \sum_{a \in \mathcal A(s)}
                \pi(a \mid s)
        }_1 \\
        & \stackrel{!}{=}
        v_\pi^\text{old}(s) + v_c
    \end{align*}

    \begin{align*}
        \impliedby
        v_c \stackrel{!}{=} c + \gamma v_c
        \iff
        c = v_c - \gamma v_c = (1 - \gamma) v_c
        \iff
        \frac{c}{1 - \gamma} = v_c
    \end{align*}

    Since the oinear system of Bellman equations is uniquely solvable and we have guessed the solution correctly, it is in fact this unique solution.

\end{enumerate}

\end{solution}

% --------------------------------------------------------------------------------
