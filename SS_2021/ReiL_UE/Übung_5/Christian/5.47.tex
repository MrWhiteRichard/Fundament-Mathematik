% -------------------------------------------------------------------------------- %

\begin{exercise}

Show that tabular methods such as presented in Part I of this book are a special case of linear function approximation.
What would the feature vectors be?

\end{exercise}

% -------------------------------------------------------------------------------- %

\begin{solution}

To see this we note that the dimension $d$ of the problem is exactly $|\mathcal{S}|$ and write the subscripts in terms of states. If we then choose the feature vectors to be

\begin{align*}
  x_{S}(s) =
  \mathds{1}_{S}(s), \quad S \in \mathcal{S}
\end{align*}

the approximate function $\hat{v}$ is just

\begin{align*}
  \hat{v}(s,\textbf{w})
  =
  w_s
\end{align*}

 so we can identify the approximate function at a state with the weight of the according state. The general SGD update for linear function approximation reads

\begin{align*}
  \textbf{w}_{t+1}
  =
  \textbf{w}_t + \alpha \bbraces{U_t - \hat{v}(S_t,\textbf{w}_t)}\textbf{x}(S_t).
\end{align*}

With our choice of feature vectors the weight vector thus fulfills

\begin{align*}
  w_{S,t+1}
  =
  \begin{cases}
    w_{S,t} & \text{if } S \neq S_t \\
    w_{S,t} + \alpha\bbraces{U_t - w_{S,t}} & \text{if } S = S_t
  \end{cases}
\end{align*}

For special choices of $U_t$ this is just the same update as we did in the tabular case, e.g. $U_t = G_t$ leads to the Monte-Carlo update and $U_t = R_{t+1} + \gamma w_{S_{t+1}}$ leads to a TD(0) update.
\end{solution}

% -------------------------------------------------------------------------------- %
