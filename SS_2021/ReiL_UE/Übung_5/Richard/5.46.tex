% ---------------------------------------------------------------- %

\begin{exercise}[Autonomous Orchard]

An AI controlled orchard needs to decide when to harvest its trees.
To do this it measures the concentration of three chemicals in the air.
Each day the orchard can choose to wait or harvest.
Waiting costs one credit in operating costs while a harvest ends the process.
Once a crop is harvested, packaged and sold, the orchard is told the profit or loss of that harvest.
Most experts agree that the function mapping the chemical concentrations to the profit is linear with some error.
The orchard has several samples of the profits from other harvests:

\phantom{}

\resizebox{\textwidth}{!}
{
    \begin{math}
        \begin{aligned}
            \begin{array}{|c|c|c|c|}
                \hline
                \text{Concentration of}~ A ~\text{(ppm)} &
                \text{Concentration of}~ B ~\text{(ppm)} &
                \text{Concentration of}~ C ~\text{(ppm)} &
                \text{Profit/Rewrad (credits)} \\ \hline
                4  & 7  & 1  & 3   \\ \hline
                10 & 6  & 0  & -15 \\ \hline
                20 & 1  & 15 & 5   \\ \hline
                4  & 19 & 3  & 21  \\ \hline
            \end{array}
        \end{aligned}
    \end{math}
}

\phantom{}

Begin to approximate (by hand) the function that maps the state feature vector to $Q(\text{state}, \text{harvest})$ using an MC goal.
Do a gradient decent step on each ssample.
A sensible learning rate would be around $0.01$, but feel free to try any value.

\end{exercise}

% ---------------------------------------------------------------- %

\begin{solution}

\phantom{}

\end{solution}

% ---------------------------------------------------------------- %