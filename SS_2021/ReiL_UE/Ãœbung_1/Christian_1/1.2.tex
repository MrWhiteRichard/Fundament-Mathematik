\begin{exercise}
Consider a $k$-armed bandit problem with $k = 4$ actions, denoted $1$, $2$, $3$, and $4$.
Consider applying to this problem a bandit algorithm using $\epsilon$-greedy action selection, sample-average action-value estimation, and initial estimates of $Q_1(a) = 0$, for all $a$.
Suppose the initial sequence of actions and rewards is $A_1 = 1$, $R_1 = 1$, $A_2 = 2$, $R_2 = 1$, $A_3 = 2$, $R_3 = 2$, $A_4 = 2$, $R_4 = 2$, $A_5 = 3$, $R_5 = 0$.
On some of these time steps the $\epsilon$ case may have occured, causing an action to be selected at random.
On with time steps did this definitely occur?
On with time steps could this possibly have occured?
\end{exercise}

\begin{solution}
The first action is taken arbitrarily, since $\forall a: Q_1(a)=0$. Here we can not differentiate if the $\epsilon$ case occured or not. We take action $1$ and update $Q$:

\begin{align*}
  Q_2(1) = 1
  \qquad
  Q_2(a) = 0, \quad a \in \{2,3,4\}
\end{align*}

Then we are definitely in the $\epsilon$-case. The $\argmax_a Q_2(a) = 1$ but we take action $2$. Again we update $Q$:

\begin{align*}
  Q_3(1) = 1
  \quad
  Q_3(2) = 1
  \quad
  Q_3(3) = 0
  \quad
  Q_3(4) = 0
\end{align*}

On step three it is likely that we took a greedy action, though we can obviously not guarantee it, since we take action $2$ and $\argmax_a Q_3(a) = \{1,2\}.$ Again we update $Q$:

\begin{align*}
  Q_4(1) = 1
  \quad
  Q_4(2) = \frac{3}{2}
  \quad
  Q_4(3) = 0
  \quad
  Q_4(4) = 0
\end{align*}

On step four it again is likely, though not guaranteed, that we take the greedy action. Again update $Q$:

\begin{align*}
  Q_5(1) = 1
  \quad
  Q_5(2) = \frac{5}{3}
  \quad
  Q_5(3) = 0
  \quad
  Q_5(4) = 0
\end{align*}

On step five we definitely do exploration, since $\argmax_a Q(a) = 2$ but we take action 3. Since we get a reward of 0, it holds that $Q_6(a) = Q_5(a)$ for all actions $a$.
\end{solution}
