% -------------------------------------------------------------------------------- %

\begin{exercise}[Exercise 2.6 Mysterious Spikes]

The results shown in Figure 2.3 (below) should be quite reliable because they are
averages over 2000 individual, randomly chosen 10-armed bandit tasks. Why, then,
are there oscillations and spikes in the early part of the curve for the optimistic
method? In other words, what might make this method perform particularly better or
worse, on average, on particular early steps?
\includegraphicsboxed{figure_2.3.png}

\end{exercise}

% -------------------------------------------------------------------------------- %

\begin{solution}

Assuming that the optimistic initialization is significantly above the mean values
of the 10-armed testbed, then the optimistic method would start with trying all 10
actions once first, thus significantly decreasing the action-value estimates for each action.
After these initial ten steps, the method continues with the action, which yielded
the highest reward in the first try. However, most likely, the action-value estimation
for this action is still too high and gets thus decreased further with increased exploration,
therefore opening the door for the other possible actions again.

An illustration of that fact is given by the table below.

\begin{center}
\begin{tabular}{ |c||c|c|c|c|c|c|c|c|c|c| }
 \hline
 \multicolumn{11}{|c|}{Action-value estimates over time} \\
 \hline
 Time step & $Q(A_1)$ & $Q(A_2)$ & $Q(A_3)$ & $Q(A_4)$ & $Q(A_5)$ & $Q(A_6)$ & $Q(A_7)$ & $Q(A_8)$ & $Q(A_9)$ & $Q(A_10)$ \\
 \hline
 1 & \hl{5} & \hl{5} & \hl{5} & \hl{5} & \hl{5} & \hl{5} & \hl{5} & \hl{5} & \hl{5} & \hl{5} \\
 2 & \hl{5} & \hl{5} & \hl{5} & 3.4 & \hl{5} & \hl{5} & \hl{5} & \hl{5} & \hl{5} & \hl{5} \\
 3 & \hl{5} & \hl{5} & \hl{5} & 3.4 & \hl{5} & \hl{5} & \hl{5} & \hl{5} & \hl{5} & 2.7 \\
 4 & \hl{5} & \hl{5} & \hl{5} & 3.4 & \hl{5} & \hl{5} & \hl{5} & \hl{5} & 4.1 & 2.7 \\
 5 & \hl{5} & \hl{5} & 3.8 & 3.4 & \hl{5} & \hl{5} & \hl{5} & \hl{5} & 4.1 & 2.7 \\
 6 & \hl{5} & \hl{5} & 3.8 & 3.4 & 4.2 & \hl{5} & \hl{5} & \hl{5} & 4.1 & 2.7 \\
 7 & 3.9 & \hl{5} & 3.8 & 3.4 & 4.2 & \hl{5} & \hl{5} & \hl{5} & 4.1 & 2.7 \\
 8 & 3.9 & \hl{5} & 3.8 & 3.4 & 4.2 & \hl{5} & 3.3 & \hl{5} & 4.1 & 2.7 \\
 9 & 3.9 & 4.4 & 3.8 & 3.4 & 4.2 & \hl{5} & 3.3 & \hl{5} & 4.1 & 2.7 \\
 10 & 3.9 & 4.4 & 3.8 & 3.4 & 4.2 & \hl{5} & 3.3 & 3.7 & 4.1 & 2.7 \\
 11 & 3.9 & \hl{4.4} & 3.8 & 3.4 & 4.2 & 3.6 & 3.3 & 3.7 & 4.1 & 2.7 \\
 \hline
\end{tabular}
\end{center}


\end{solution}

% -------------------------------------------------------------------------------- %
