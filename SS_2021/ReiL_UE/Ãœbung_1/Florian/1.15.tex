% -------------------------------------------------------------------------------- %

\begin{exercise}[Exercise 3.15]

In the gridworld example, rewards are positive for goals, negative for running
into the edge of the world, and zero the rest of the time. Are the signs of these
rewards important, or only the intervals between them?
Prove, using (3.8), that adding a constant $c$ to all the rewards adds a constant, $v_c$,
to the values of all states, and thus does not affect the relative values of any
states under any policies. What is $v_c$ in terms of $c$ and $\gamma$?

\end{exercise}

% -------------------------------------------------------------------------------- %

\begin{solution}

Of course the signs are important, since inverting them would also invert all
state-values, which would render the previously worst action in each state to be the best one.

Denote by $G_t^*, R_t^*$ and $v_\pi^*$ the modified rewards, returns and value function by adding the constant $c$ to every reward.

\begin{align*}
  v_\pi^*(s) &= \E[G_t^* | S_t = s] = \E\left[\sum_{k=0}^{\infty}\gamma^kR_{t+k+1}^* | S_t = s\right]
  = \E\left[\sum_{k=0}^{\infty}\gamma^k(R_{t+k+1} + c) | S_t = s\right] \\
  &= \E\left[\sum_{k=0}^{\infty}\gamma^kR_{t+k+1} | S_t = s\right] + \frac{c}{1 - \gamma}
  = v_\pi(s) + \frac{c}{1 - \gamma}.
\end{align*}

Therefore we obtain $v_c = c/(1-\gamma)$.

\end{solution}

% -------------------------------------------------------------------------------- %
