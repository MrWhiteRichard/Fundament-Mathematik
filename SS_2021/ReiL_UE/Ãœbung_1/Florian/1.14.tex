% -------------------------------------------------------------------------------- %

\begin{exercise}[Exercise 3.14]

The Bellman equation (3.14) must hold for each state for the value function $v_\pi$
shown in Figure 3.2 (right) of Example 3.5. Show numerically that this equation
holds for the center state, valued at +0.7, with respect to its four neighboring states,
valued at +2.3, +0.4, -0.4 and +0.7. (These numbers are accurate only to one decimal place.)

\end{exercise}

% -------------------------------------------------------------------------------- %

\begin{solution}

The Bellman equation states

\begin{align*}
  v_\pi(s) = \sum_a \pi(a|s) \sum_{s',r} p(s',r|s,a)[r + \gamma v_\pi(s')].
\end{align*}

In our case, we can confirm that

\begin{align*}
  0.7 &\approx v_\pi(s) = \sum_a \pi(a|s) \sum_{s',r} p(s',r|s,a)[r + \gamma v_\pi(s')] \\
  &\approx \frac{1}{4}[0.9\cdot 2.3 + 0.9 \cdot 0.4 - 0.9 \cdot 0.4 + 0.9 \cdot 0.7]
  = 0.675.
\end{align*}

\end{solution}

% -------------------------------------------------------------------------------- %
