% -------------------------------------------------------------------------------- %

\begin{exercise}[Exercise 3.6]

Suppose you treated pole-balancing as an episodic task but also used discounting, with all rewards zero except for $-1$ upon failure.
What then would the return be at each time?
How does this return differ from that in the discounted, continuing formulation of this task?

\end{exercise}

% -------------------------------------------------------------------------------- %

\begin{solution}

Failure always occures at timestep $T$, followed by the start of the next episode.
Therefore, the specification  of the reward translates to $R_\tau = -\delta_{T \tau}$.

\begin{align*}
    \implies
    G_t
    & \doteq
    R_{t+1} + \gamma R_{t+2} + \gamma^2 R_{t+3} + \cdots + \gamma^{T-t-1} R_T \\
    & =
    \sum_{\tau = t+1}^T
        \gamma^{\tau - t - 1} R_\tau \\
    & =
    -\gamma^{T-t-1}
\end{align*}

In the continuing formulation, multiple $\gamma$-powers are subtracted, depending on which timesteps failure occures at.

\end{solution}

% -------------------------------------------------------------------------------- %
