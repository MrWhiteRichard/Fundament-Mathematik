\begin{exercise}
In the gridworld example, rewards are positive for goals, negative for running into the edge of the world, and zero the rest of the time.
Are the signs of these rewards important, or only the intervals between them?
Prove, using \eqref{eq:2.15}, that adding a constant $c$ to all the rewards adds a constant, $v_c$, to the values of all states, and thus does not affect the relative values of any states under any policies.
What is $v_c$ in terms of $c$ and $\gamma$?

\begin{align} \label{eq:2.15}
    G_t
    \doteq
    R_{t+1} + \gamma R_{t+2} + \gamma^2 R_{t+3} + \cdots
    =
    \sum_{k=0}^\infty
        \gamma^k R_{t+k+1}
\end{align}

\end{exercise}

\begin{solution}
In the following we write $ v_\pi^+$ for the state value where we added a constant and $v_\pi$ for the state value function where we have not added the constant. By use the definition of $v_\pi$ as well as basic properties of the expectation we get

\begin{align*}
  v_\pi^+(s)
  &=
  \E_\pi
  \bigg[
  G_t^+ \big| S_t = s
  \bigg]
  =
  \E_\pi
  \bigg[
  \sum_{k=0}^\infty \gamma^k (R_{t+k+1} + c) \big| S_t = s
  \bigg] \\
  &=
  \E_\pi
  \bigg[
  \sum_{k=0}^\infty \gamma^k R_{t+k+1} + \sum_{k=0}^\infty \gamma^k c \big| S_t = s
  \bigg]
  =
  \E_\pi
  \bigg[
  \sum_{k=0}^\infty \gamma^k R_{t+k+1} \big| S_t = s
  \bigg]
  +
  \E_\pi
  \bigg[
  \frac{c}{1-\gamma} \big| S_t = s
  \bigg] \\
  &=
  v_\pi(s) + \frac{c}{1-\gamma} \qquad \forall s \in \mathcal{S}.
\end{align*}

So we get

\begin{align*}
  v_\pi^+(s) = v_\pi(s) + v_c, \qquad \forall s \in \mathcal{S}
\end{align*}

with $v_c = \frac{c}{1-\gamma}$.
\end{solution}
