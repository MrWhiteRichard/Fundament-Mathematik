% -------------------------------------------------------------------------------- %

\begin{exercise}[Exercise 8.1]

The non-planning method looks particularly poor in Figure 8.3 because it is
a one-step method; a method using multi-step bootstrapping would do better.
Do you think one of the multi-step bootstrapping methods from Chapter 7
could do as well as the Dyna method? Explain why or why not.

\end{exercise}

% -------------------------------------------------------------------------------- %

\begin{solution}

As seen in the previous exercise, multi-step bootstrapping
definitively constitutes a big improvement from the simple one-step method,
however the results show that they cannot quite achieve the top performances
of the Dyna methods. 

The big difference between the two methods is that in every step of the episode
the multi-step method updates one action-value by looking $n$ steps ahead, whilst
the planning method updates $n$ action-values by looking one step ahead respectively.
Therefore the multi-step method obtains more accurate estimations for a small sample
of state-action pairs, while the planning method obtains a less accurate, but far broader
estimation over the state-action space, which allows it to find the optimal solution faster,
since the precise values of the individual state-action pairs are not necessarily needed to find the
optimal policy in this case.

\end{solution}

% -------------------------------------------------------------------------------- %
