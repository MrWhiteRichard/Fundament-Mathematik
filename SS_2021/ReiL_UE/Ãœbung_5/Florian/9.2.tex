% -------------------------------------------------------------------------------- %

\begin{exercise}[Exercise 9.1]

Show that tabular methods such as presented in Part I of this book are
a special case of linear function approximation. What would the feature
vectors be?

\end{exercise}

% -------------------------------------------------------------------------------- %

\begin{solution}

We can interpret the tabular methods as a special case of linear function
approximation if we choose the dimension of the feature vector to be
the total number of states. Then we determine an (arbitrary) ordering
of the state space $\mathcal{S} = \{s_1,\dots,s_n\}$, and identify
each state $s_i$ with the $i$-th unit vector

$x(s_i) = (\underbrace{0,\dots}_{i - 1},1,\dots,0)$.

Then our linear function approximation $\hat{v}$ just reads

\begin{align*}
    \hat{v}(\textbf{w}, \textbf{x}(s_j)) = \sum_{i=1}^n x_i(s_j)w_j
    = w_j.
\end{align*}

Hence every state has its own weight, which does not influence other state values.
\end{solution}

% -------------------------------------------------------------------------------- %
