% ---------------------------------------------------------------- %

\begin{exercise}[Exercise 9.1]

Show that tabular methods such as presented in Part I of this book are a special case of linear function approximation.
What would the feature vectors be?

\end{exercise}

% ---------------------------------------------------------------- %

\begin{solution}

\begin{align} \label{eq:9.8} \tag{9.8}
    \hat v(s, \mathbf w)
    \doteq
    \mathbf w^\top \mathbf x(s)
    \doteq
    \sum_{i=1}^d w_i x_i(s)
\end{align}

W.l.o.g assume that $\mathcal S = \Bbraces{1, \dots, d}$.
Let $x_i(s) = \1_{i=s}$ for $s \in \mathcal S$.
This yields

\begin{align*}
    \hat v(s, \mathbf w) \stackrel{\eqref{eq:9.8}}{=} w_s
    \quad
    \text{and}
    \quad
    \nabla \hat v(s, \mathbf w) = \mathbf x(s) = \mathbf e_s,
\end{align*}

where $\mathbf e_s \in \R^d$ is the $s$-th canonical basis vector.

\begin{align} \label{eq:7.2} \tag{7.2}
    V_{t+n}(S_t)
    \doteq
    V_{t+n-1}(S_t)
    +
    \alpha [G_{t:t+n} - V_{t+n-1}(S_t)],
    \quad
    0 \leq t < T
\end{align}

\begin{align} \label{eq:9.15} \tag{9.15}
    \mathbf w_{t+n}
    \doteq
    \mathbf w_{t+n-1}
    +
    \alpha [G_{t:t+n} - \hat v(S_t, \mathbf w_{t+n-1})] \nabla \hat v(S_t, \mathbf w_{t+n-1}),
    \quad
    0 \leq t < T
\end{align}

Let $V_t(s) = w_{t, s}$, then \eqref{eq:7.2} describes the update of the $s$-th component of \eqref{eq:9.15}, because the respective component of $\mathbf e_s$ is $1$.
All other components of \eqref{eq:9.15} are not really updated (i.e. they inherit the previous value), since the respective component of $\mathbf e_s$ is $0$.

\end{solution}

% ---------------------------------------------------------------- %