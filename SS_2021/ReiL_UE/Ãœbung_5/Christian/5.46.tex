% -------------------------------------------------------------------------------- %

\begin{exercise}

An AI controlled orchard needs to decide when to harvest its trees.
To do this it measures the concentration of three chemicals in the air.
Each day the orchard can choose to wait or harvest.
Waiting costs one credit in operating costs while a harvest ends the process.
Once a crop is harvested, packaged and sold, the orchard is told the profit or loss of that harvest.
Most experts agree that the function mapping the chemical concentrations to the profit is linear with some error.
The orchard has several samples of the profits from other harvests:

\phantom{}

\resizebox{\textwidth}{!}
{
    \begin{tabular}{|c|c|c|c|}
        \hline
        Concentration of $A$ (ppm) &
        Concentration of $B$ (ppm) &
        Concentration of $C$ (ppm) &
        Profit/Rewrad (credits)    \\ \hline
        $4$  & $7$  & $1$  & $3$   \\ \hline
        $10$ & $6$  & $0$  & $-15$ \\ \hline
        $20$ & $1$  & $15$ & $5$   \\ \hline
        $4$  & $19$ & $3$  & $21$  \\ \hline
    \end{tabular}
}

\phantom{}

Begin to approximate (by hand) the function that maps the state feature vector to $Q(\text{state}, \text{harvest})$ using an MC goal.
Do a gradient decent step on each sample.
A sensible learning rate would be around $0.01$, but feel free to try any value.


\end{exercise}

% -------------------------------------------------------------------------------- %

\begin{solution}

We start our with $\textbf{w}_0 = (0,0,0)^\top$, choose the parameter values $\gamma = 0.9 , \alpha = 0.01$ and note that the MC goal is $G_t$, which we will now calculate:

\begin{align*}
  G_0
  &=
  3 - 0.9\cdot 15 + 0.9^2\cdot 5 + 0.9^3 \cdot 21 = 8.859\\
  G_1
  &=
  -15 + 0.9 \cdot 5 + 0.9^2 \cdot 21 = 15.369 \\
  G_2
  &=
  5 + 0.9\cdot 21 = 23.9 \\
  G_3
  &=
  21
\end{align*}

Our approximate function is

\begin{align*}
  \hat{q}(s,a,\textbf{w})
  =
  \sum_{i \in \{A,B,C\}} w_i \cdot x_i(s,a)
\end{align*}

where $\textbf{x}(S,A)$ is our feature vector. We assume that the above table gives us $\textbf{x}(s,\text{harvest})$ for $s \in \{S_1,\dots,S_4\}$. Our update rule for the weights is

\begin{align*}
  \textbf{w}_{t+1}
  =
  \textbf{w}_t + \alpha \bbraces{G_t - \hat{q}(S_t,A_t,\textbf{w}_t)}\textbf{x}(S_t,A_t)
\end{align*}

With all this information we can now happily begin to calculate the weights and in turn update our approximate function.

\begin{align*}
  \textbf{w}_1
  =
  \begin{pmatrix}
    0 \\
    0 \\
    0
  \end{pmatrix}
  +
  0.01\bbraces{8.859 - 0}
  \begin{pmatrix}
    4 \\
    7 \\
    1
  \end{pmatrix}
  =
  \begin{pmatrix}
    0.35436 \\
    0.62013 \\
    0.08859
  \end{pmatrix}
\end{align*}

\begin{align*}
  \textbf{w}_2
  =
  \begin{pmatrix}
    0.35436 \\
    0.62013 \\
    0.08859
  \end{pmatrix}
  +
  0.01\bbraces{15.369 - 7.26438}
  \begin{pmatrix}
    10 \\
    6 \\
    0
  \end{pmatrix}
  =
  \begin{pmatrix}
    1.164822 \\
    1.1064072 \\
    0.08859
  \end{pmatrix}
\end{align*}

\begin{align*}
  \textbf{w}_3
  =
  \begin{pmatrix}
    1.164822 \\
    1.1064072 \\
    0.08859
  \end{pmatrix}
  + 0.01\bbraces{23.9 - 25.7316972}
  \begin{pmatrix}
    20 \\
    1 \\
    15
  \end{pmatrix}
  =
  \begin{pmatrix}
    0.79848256 \\
    1.088090228 \\
    -0.18616458
  \end{pmatrix}
\end{align*}

\begin{align*}
  \textbf{w}_4
  =
  \begin{pmatrix}
    0.79848256 \\
    1.088090228 \\
    -0.18616458
  \end{pmatrix}
  + 0.01\bbraces{21 - 23.30915083}
  \begin{pmatrix}
    4 \\
    19 \\
    3
  \end{pmatrix}
  =
  \begin{pmatrix}
    0.7061165268 \\
    0.6493515703 \\
    -0.2554391049
  \end{pmatrix}
\end{align*}

If we really wanted to we could now calculate the approximate function for the states with this updated weight vector but that is just more tedious calculations.
\end{solution}

% -------------------------------------------------------------------------------- %
