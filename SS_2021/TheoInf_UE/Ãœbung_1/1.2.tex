% --------------------------------------------------------------------------------

\begin{exercise}

Verwenden Sie den Algorithmus aus der Vorlesung, um den NFA

\begin{center}
    \begin{tikzpicture}[
        ->,
        node distance = 2cm,
        initial text = $ $,
        initial distance = 1.5cm
    ]

        \node [state, initial]                   (q_0) {$q_0$};
        \node [state, right of = q_0]            (q_1) {$q_1$};
        \node [state, below of = q_0, accepting] (q_2) {$q_2$};

        \draw
        (q_0) edge [bend left]  node [above] {$b$} (q_1)
        (q_1) edge [loop right] node         {$b$} (q_1)
        (q_1) edge [bend left]  node [below] {$a$} (q_0)
        (q_0) edge              node [left]  {$b$} (q_2);

        % example from
        % https://www3.nd.edu/~kogge/courses/cse30151-fa17/Public/other/tikz_tutorial.pdf

        % \node [state, initial, accepting] (1) {$1$};
        % \node [state, below left of = 1]  (2) {$2$};
        % \node [state, right of = 2]       (3) {$3$};

        % \draw
        % (1) edge [above]                          node {$b$}        (2)
        % (1) edge [below, bend right, left = 0.3]  node {$\epsilon$} (3)
        % (2) edge [loop left]                      node {$a$}        (2)
        % (2) edge [below]                          node {$a, b$}     (3)
        % (3) edge [above, bend right, right = 0.3] node {$a$}        (1);
 
    \end{tikzpicture}
\end{center}

in einen DFA umzuwandeln.

\end{exercise}

% --------------------------------------------------------------------------------

\begin{solution}

Wir verwenden die Potenzmengenkonstruktion aus dem Beweis von Satz 1.1.
Wir konstruieren dazu aber bloß jene Zustände, die vom Startzustand $q_0$ erreichbar sind.

Wir erstellen eine Tabelle mit dem Startzustand und den Symbolen des Alphabets.
Der Eintrag in der $i$-ten Zeile und der $j$-ten Spalte soll die Menge von Zuständen des NFA enthalten die von einem Zustand aus der Beschriftung der $i$-ten Zeile mit dem Symbol das die $j$-Spalte beschriftet erreichbar sind.
Falls dieses Prozess einen neuen Zustand des DFA erzeugt, so wird dieser als neue, leere, Zeile der Tabelle hinzugefugt.
Diese Tabelle ist saturiert wenn jeder Zustand des DFA der in der Tabelle vorkommt auch als Beschriftung einer Zeile vorkommt.
Sobald die Tabelle saturiert ist, ist die Konstruktion des NFA vollständig.

\begin{align*}
    \begin{array}{c|cc}
                           & a             & b                  \\ \hline
        \Bbraces{q_0}      & \emptyset     & \Bbraces{q_1, q_2} \\
        \emptyset          & \emptyset     & \emptyset          \\
        \Bbraces{q_1, q_2} & \Bbraces{q_0} & \Bbraces{q_1}      \\
        \Bbraces{q_1}      & \Bbraces{q_0} & \Bbraces{q_1}
    \end{array}
\end{align*}

Auf diese Weise haben wir also den folgenden DFA konstruiert, der per constructionem äquivalent zu dem NFA aus der Angabe.

\begin{center}
    \begin{tikzpicture}[
        ->,
        node distance = 2cm,
        initial text = $ $,
        initial distance = 1.5cm
    ]

        \node [state, initial]                   (q_0)  {$q_0$};
        \node [state, right of = q_0]            (q_)   {$q_\emptyset$};
        \node [state, below of = q_0, accepting] (q_12) {$q_{12}$};
        \node [state, left of = q_12]            (q_1)  {$q_1$};

        \draw
        (q_0)  edge                    node [above]      {$a$} (q_)
        (q_)   edge [loop above]       node [above]      {$a$} (q_)
        (q_)   edge [loop below]       node [below]      {$b$} (q_)
        (q_0)  edge                    node [left]       {$b$} (q_12)
        (q_12) edge [bend right]       node [right]      {$a$} (q_0)
        (q_12) edge                    node [below]      {$b$} (q_1)
        (q_1)  edge                    node [above left] {$a$} (q_0);

    \end{tikzpicture}
\end{center}

\end{solution}

% --------------------------------------------------------------------------------
