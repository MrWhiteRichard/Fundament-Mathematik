% --------------------------------------------------------------------------------

\begin{exercise}

Für $k \geq 1$ sei $L_k = \Bbraces{w \in \Bbraces{a, b}^\ast \mid n_a(v) - n_b(v)| \leq k ~\text{für jedes Präfix}~ v ~\text{von}~ w}$.
Ist $L_k$ eine reguläre Sprache?

\end{exercise}

% --------------------------------------------------------------------------------

\begin{solution}

Wir finden einen expliziten NFA $D_k$, sodass $\operatorname L(D_k) = L_k$.

\begin{center}
    \begin{tikzpicture}[
        ->,
        node distance = 2cm,
        initial text = $ $,
        initial distance = 1.5cm,
        initial above
    ]
    
        \node [state, accepting]                             (1) {$q_{-k}$};
        \node [state, accepting, right of = 1, draw = white] (2) {$\cdots$};
        \node [state, accepting, right of = 2]               (3) {$q_{-1}$};
        \node [state, accepting, right of = 3, initial]      (4) {$q_0$};
        \node [state, accepting, right of = 4]               (5) {$q_1$};
        \node [state, accepting, right of = 5, draw = white] (6) {$\cdots$};
        \node [state, accepting, right of = 6]               (7) {$q_k$};

        \draw
        (4) edge [bend left] node [above] {$a$} (5)
        (5) edge [bend left] node [above] {$a$} (6)
        (6) edge [bend left] node [above] {$a$} (7)
        (7) edge [bend left] node [below] {$b$} (6)
        (6) edge [bend left] node [below] {$b$} (5)
        (5) edge [bend left] node [below] {$b$} (4)
        (4) edge [bend left] node [below] {$b$} (3)
        (3) edge [bend left] node [below] {$b$} (2)
        (2) edge [bend left] node [below] {$b$} (1)
        (1) edge [bend left] node [above] {$a$} (2)
        (2) edge [bend left] node [above] {$a$} (3)
        (3) edge [bend left] node [above] {$a$} (4);
    
    \end{tikzpicture}
\end{center}

Für einen Zustand $q_\ell$, $\ell \in \Bbraces{0, \pm 1, \dots, \pm k}$, und einem Pfad $w$ von $q_0$ nach $q_\ell$ gilt $n_a(w) - n_b(w) = \ell$.
Das zeigt $\operatorname L(D_k) \subseteq L_k$.

Für die umgekehrte Inklusion sei $w \in L_k$.
Wir verwenden Induktion nach $|w|$.

Für $|w| = 0$, muss $w = \epsilon$, also das leere Wort sein.
Diese ist durch unseren Automaten offenbar unmittelbar (im Start-Zustand) erreichbar.
Ansonsten gibt es ein $c \in \Bbraces{a, b}$ und $v \in \Bbraces{a, b}^{|w| - 1}$, sodass $w = v c$.
Laut Induktionshypothese, ist $v$ erreichbar und führt zu einem Zustand $q_\ell$, $\ell \in \Bbraces{0, \pm 1, \dots, \pm k}$.

\begin{align*}
    \sigma := (n_a(w) - n_b(w)) - (n_a(v) - n_b(v)) = \sigma_a - \sigma_b,
    \quad
    \sigma_a := n_a(w) - n_a(v),
    \quad
    \sigma_b := n_b(w) - n_b(v)
\end{align*}

Für $c = a$ gilt $\sigma = 1 - 0$ und $w$ führt nach $q_{\ell + 1}$.
Für $c = b$ gilt $\sigma = 0 - 1$ und $w$ führt nach $q_{\ell - 1}$.
$w$ ist also erreichbar und führt zum Zustand $q_{\ell + \sigma}$.

\end{solution}

% --------------------------------------------------------------------------------
