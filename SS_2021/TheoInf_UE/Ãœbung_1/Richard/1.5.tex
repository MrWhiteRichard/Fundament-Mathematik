% --------------------------------------------------------------------------------

\begin{exercise}

Ein NFA $\abraces{Q, A, \Delta, q_0, F}$ heißt \textit{gekürzt} falls es für alle $q \in Q \setminus \Bbraces{q_0}$ Wörter $u, v \in A^\ast$ und einen Zustand $q_\mathrm{f} \in F$ gibt so dass $(q_0, u, q) \in \Delta$ und $(q, v, q_\mathrm{f}) \in \Delta$.
Zeigen Sie dass eine Sprache $L$ regulär ist genau dann wenn ein gekürzter Automat existiert der $L$ akzeptiert.

\end{exercise}

% --------------------------------------------------------------------------------

\begin{solution}

Die Rückrichtung ist klar.
Für die Hinrichtung bemerkt man, dass gekürzte Zustände $q \in Q \setminus \Bbraces{q_0}$ zum NFA $N = \abraces{Q, A, \Delta, q_0, F}$ mit $L = \operatorname L(N)$, d.h.

\begin{align*}
    \Exists u, v \in A^\ast, \Exists q_\mathrm{f} \in F:
        (q_0, u, q), (q, v, q_\mathrm{f}) \in \Delta,
\end{align*}

genau jene sind, durch die ein Wort $w \in L$ führt.
Sei nämlich $q$ ein solcher Zustand.
Dann gilt $(q_0, w, q_\mathrm{f}) \in \Delta$.
Das Wort $w := uv \in L$ führt durch $q$.
Die Umkehrung ist klar.

Der gekürzte \Quote{Teil-Automat} mit den gekürzten Zuständen aus $Q$ (und dem Anfangszustand $q_0$) reicht also aus, um alle Worte $w \in L$ zu generieren.

\end{solution}

% --------------------------------------------------------------------------------
