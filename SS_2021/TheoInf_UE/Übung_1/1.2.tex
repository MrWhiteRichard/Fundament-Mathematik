% --------------------------------------------------------------------------------

\begin{exercise}

Verwenden Sie den Algorithmus aus der Vorlesung, um den NFA

\begin{center}
    \begin{tikzpicture}[
        ->,
        node distance = 2cm,
        initial text = $ $,
        initial distance = 1.5cm
    ]

        \node [state, initial]                   (q_0) {$q_0$};
        \node [state, right of = q_0]            (q_1) {$q_1$};
        \node [state, below of = q_0, accepting] (q_2) {$q_2$};

        \draw
        (q_0) edge [bend left]  node [above] {$b$} (q_1)
        (q_1) edge [loop right] node         {$b$} (q_1)
        (q_1) edge [bend left]  node [below] {$a$} (q_0)
        (q_0) edge              node [left]  {$b$} (q_2);

        % example from
        % https://www3.nd.edu/~kogge/courses/cse30151-fa17/Public/other/tikz_tutorial.pdf

        % \node [state, initial, accepting] (1) {$1$};
        % \node [state, below left of = 1]  (2) {$2$};
        % \node [state, right of = 2]       (3) {$3$};

        % \draw
        % (1) edge [above]                          node {$b$}        (2)
        % (1) edge [below, bend right, left = 0.3]  node {$\epsilon$} (3)
        % (2) edge [loop left]                      node {$a$}        (2)
        % (2) edge [below]                          node {$a, b$}     (3)
        % (3) edge [above, bend right, right = 0.3] node {$a$}        (1);
 
    \end{tikzpicture}
\end{center}

in einen DFA umzuwandeln.

\end{exercise}

% --------------------------------------------------------------------------------

\begin{solution}

ToDo!

\end{solution}

% --------------------------------------------------------------------------------
