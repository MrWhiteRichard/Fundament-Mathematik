% -------------------------------------------------------------------------------- %

\begin{exercise}

Zeigen Sie dass die folgenden Relationen und Funktionen primitiv rekursiv sind:

\begin{enumerate}

    \item Teilbarkeit:
    $(x, y) \mapsto \begin{cases} 1 & \text{falls}~ x ~\text{ein Teiler von}~ y ~\text{ist} \\ 0 & \text{sonst} \end{cases}$

    \item Teilerfremdheit:
    $(x, y) \mapsto \begin{cases} 1 & \text{falls}~ \ggT(x, y) = 1 \\ 0 & \text{sonst} \end{cases}$

    \item Die Eulersche $\varphi$-Funktion:
    $\varphi: \N \to \N, n \mapsto |\Bbraces{i \in \N \mid 1 \leq i \leq n, \ggT(i, n) = 1}|$.

\end{enumerate}

\end{exercise}

% -------------------------------------------------------------------------------- %

\begin{solution}

\phantom{}

\begin{enumerate}[label = (\alph*)]

    \item

    \begin{align*}
        x \mid y
        =
        \pbraces
        {
            \exists z \leq y
                \chi_=(x \cdot z, y)
        }
    \end{align*}

    \item Sei $\chi_\neq := 1 \dotdiv \chi_=$.

    \begin{align*}
        \chi_=(\ggT(x, y), 1)
        =
        \pbraces
        {
            \neg
            \exists z \leq x + y
            \pbraces
            {
                z \mid x
                \land
                z \mid y
                \land
                \chi_\neq(z, 1)
            }
        }
    \end{align*}

    \item Es gilt für alle $n \in \N:$

    \begin{align*}
        \varphi(n)
        & =
        |\Bbraces{i \in \N \mid 1 \leq i \leq n, \ggT(i, n) = 1}| \\
        & =
        \sum_{i=1}^n
            |\Bbraces{i \mid \ggT(i, n) = 1}| \\
        & =
        \sum_{i=1}^n
            \chi_=(\ggT(i, n), 1).
    \end{align*}
    
\end{enumerate}

\end{solution}

% -------------------------------------------------------------------------------- %
