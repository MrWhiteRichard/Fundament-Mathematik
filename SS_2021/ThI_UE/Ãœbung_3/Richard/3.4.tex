% -------------------------------------------------------------------------------- %

\begin{exercise}

Geben Sie eine Turingmaschine an, die eine Duplikation durchführt, d.h. die das Eingabeband

\begin{center}

    \turingtape{1}{5}
    {
        {
            $\triangleright$,
            $x_1$,
            \dots,
            $x_n$,
            \textvisiblespace
        }
    }

\end{center}    

für $x_1, \dots, x_n \in \Bbraces{0, 1}$ in das Ausgabeband

\begin{center}

    \turingtape{1}{9}
    {
        {
            $\triangleright$,
            $x_1$,
            \dots,
            $x_n$,
            \textvisiblespace,
            $x_1$,
            \dots,
            $x_n$,
            \textvisiblespace
        }
    }

\end{center}    

transformiert.

\end{exercise}

% -------------------------------------------------------------------------------- %

\begin{solution}

Eine Turingmaschine, die die Instrukitonen aus dem folgenden Diagramm befolgt, leistet das Gewünschte.

\resizebox{\textwidth}{!}
{
    \begin{tikzpicture}[
        ->,
        node distance = 3cm,
        initial text = $ $
    ]
    
        % -------------------------------- %
        % nodes

        \node [state, initial] (s) {$s$};
        \node [state, right of = s] (q_0) {$q_0$};
        
        \node [state, above right of = q_0] (q_m1) {$q_{-1}$};
        \node [state, right of = q_m1]      (q_m2) {$q_{-2}$};
        \node [state, right of = q_m2]      (q_m3) {$q_{-3}$};
        \node [state, right of = q_m3]      (q_m4) {$q_{-4}$};
        
        \node [state, below right of = q_0] (q_p1) {$q_1$};
        \node [state, right of = q_p1]      (q_p2) {$q_2$};
        \node [state, right of = q_p2]      (q_p3) {$q_3$};
        \node [state, right of = q_p3]      (q_p4) {$q_4$};
        
        \node [state, below right of = q_m4] (q_0_clone) {$q_0$};
        \node [state, right of = q_0_clone, accepting] (finish) {fertig};
        
        % -------------------------------- %
        % edges
        
        \draw (s) edge node [above] {$\triangleright : \rightarrow$} (q_0);
        
        \draw
        (q_0)  edge              node [above, rotate =  45] {$0 : ($ \textvisiblespace $,  \rightarrow)$} (q_m1)
        (q_m1) edge [loop above] node                       {$0, 1: \rightarrow$}                         (q_m1)
        (q_m1) edge              node [above]               {\textvisiblespace $: \rightarrow$}           (q_m2)
        (q_m2) edge [loop above] node                       {$0, 1: \rightarrow$}                         (q_m2)
        (q_m2) edge              node [above]               {\textvisiblespace $: (0, \leftarrow)$}       (q_m3)
        (q_m3) edge [loop above] node                       {$0, 1: \leftarrow$}                          (q_m3)
        (q_m3) edge              node [above]               {\textvisiblespace $: \leftarrow$}            (q_m4)
        (q_m4) edge [loop above] node                       {$0, 1: \leftarrow$}                          (q_m4)
        (q_m4) edge              node [above, rotate = -45]  {\textvisiblespace $: (0, \rightarrow)$}     (q_0_clone);

        \draw
        (q_0)  edge              node [below, rotate = -45] {$1 : ($ \textvisiblespace $,  \rightarrow)$} (q_p1)
        (q_p1) edge [loop below] node                       {$0, 1: \rightarrow$}                         (q_p1)
        (q_p1) edge              node [below]               {\textvisiblespace $: \rightarrow$}           (q_p2)
        (q_p2) edge [loop below] node                       {$0, 1: \rightarrow$}                         (q_p2)
        (q_p2) edge              node [below]               {\textvisiblespace $: (1, \leftarrow)$}       (q_p3)
        (q_p3) edge [loop below] node                       {$0, 1: \leftarrow$}                          (q_p3)
        (q_p3) edge              node [below]               {\textvisiblespace $: \leftarrow$}            (q_p4)
        (q_p4) edge [loop below] node                       {$0, 1: \leftarrow$}                          (q_p4)
        (q_p4) edge              node [below, rotate =  45]  {\textvisiblespace $: (1, \rightarrow)$}     (q_0_clone);

        \draw (q_0_clone) edge node [above] {\textvisiblespace $: -$} (finish);
    
        % -------------------------------- %

    \end{tikzpicture}
}

Der Ästhetik halber kommt $q_0$ doppelt vor.
Beide Vorkommnisse werden identifiziert und die Mengen der jeweiligen Inputs bzw. Outputs vereinigt.

Die Zustände sind wie folgt zu verstehen.

\begin{itemize}
    
    \item $q_0$
    \dots
    Wenn ein \textvisiblespace vorliegt, terminiere;
    sonst, kopiere das vorliegende Bit, $0$ oder $1$, und setze den Platzhalter \textvisiblespace.

    \item $q_{\pm 1}$
    \dots
    Trage den Bit bis zur Barriere, dem nächsten \textvisiblespace.

    \item $q_{\pm 2}$
    \dots
    Setze den Bit beim nächst-rechten \textvisiblespace ein.

    \item $q_{\pm 3}$
    \dots
    Geh zur Barriere zurück.

    \item $q_{\pm 4}$
    \dots
    Geh zum Platzhalter zurück.

\end{itemize}

\end{solution}

% -------------------------------------------------------------------------------- %
