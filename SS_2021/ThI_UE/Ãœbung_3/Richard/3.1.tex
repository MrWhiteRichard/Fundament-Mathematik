% -------------------------------------------------------------------------------- %

\begin{exercise}

Zeigen Sie dass die folgenden Funktionen primitiv rekursiv sind \footnote{Sie dürfen dafür annehmen dass für alle $n, k \in \N$ die konstante Funktion $c_k^n: \N^n \to \N, (x_1, \dots, x_n) \mapsto k$ primitiv rekursiv ist.}:

\begin{enumerate}[label = (\alph*)]

    \item Multiplikation $(x, y) \mapsto x \cdot y$

    \item Abgeschnittener Vorgänger $x \mapsto \operatorname p(x) = \begin{cases} 0 & \text{falls}~ x = 0 \\ x - 1 & \text{falls}~ x > 0 \end{cases}$

    \item Abgeschnittene Subtraktion $(x, y) \mapsto x \dotdiv y = \begin{cases} 0 & \text{falls}~ x \leq y \\ x - y & \text{falls}~ x > y \end{cases}$

    \item Die charakteristische Funktion von kleiner-gleich $(x, y) \mapsto \chi_\leq(x, y) = \begin{cases} 1 & \text{falls}~ x \leq y \\ 0 & \text{falls}~ x > y \end{cases}$

    \item Die charakteristische Funktion der Gleichheit $(x, y) \mapsto \chi_=(x, y) = \begin{cases} 1 & \text{falls}~ x = y \\ 0 & \text{falls}~ x \neq y \end{cases}$

    \item Falls $g, f_0, f_1, \dots, f_n: \N^k \to \N$ primitiv rekursiv sin, dann ist auch
    
    \begin{align*}
        h:
            \N^k \to \N:
            \overline x
            \mapsto
            \begin{cases}
                f_1(\overline x)     & \text{falls}~ g(\overline x) = 1     \\
                f_0(\overline x)     & \text{falls}~ g(\overline x) = 0     \\
                \vdots               &                                      \\
                f_{n-1}(\overline x) & \text{falls}~ g(\overline x) = {n-1} \\
                f_n(\overline x)     & \text{falls}~ g(\overline x) \geq n
            \end{cases}
    \end{align*}

    primitiv rekursiv.

\end{enumerate}

\end{exercise}

% -------------------------------------------------------------------------------- %

\begin{solution}

\phantom{}

\begin{enumerate}[label = (\alph*)]

    \item $f := \Cn[\operatorname P_1^1, 0]$ ist, als Verknüpfung von Basisfunktion, primitiv rekursiv.

    \begin{align*}
        f(x) := 0,, \quad \text{für}~ x \in \N.
    \end{align*}

    Es gilt für alle $x \in \N:$

    \begin{align*}
        x \cdot 0 = 0 = f(x).
    \end{align*}

    Aus Beispiel 3.1 wissen wir schon, dass die Addition $(x, y) \mapsto x + y$ zweier natürlicher Zahlen primitiv rekursiv ist.

    Wir können $\cdot$ also als primitive Rekursion der primitiv Rekursiven
    $g := \Cn[+, \operatorname P_1^1, \operatorname P_1^3]$ ist daher, als Verknüpfung von primitiv rekursiven Funktionen, primitiv rekursiv.

    \begin{align*}
        g(x, y, z) := z + x,, \quad \text{für}~ x, y, z \in \N.
    \end{align*}

    Es gilt für alle $x, y \in \N:$
    
    \begin{align*}
        x \cdot (y + 1) = x \cdot y + x = g(x, y, x \cdot y).
    \end{align*}

    Schließlich ist $\cdot = \Pr[f, g]$, als primitive Rekursion von primitiv rekursiven Funktionen, primitiv rekursiv.

    \item $f := 0$ ist, als Basisfunktion, primitiv rekursiv.
    Es gilt

    \begin{align*}
        \operatorname p(0) = 0 = f.
    \end{align*}

    $g = \operatorname P_1^2$ ist, als Basisfunktion, primitiv rekursiv.

    \begin{align*}
        g(y, z) = y, \quad \text{für}~ y, z \in \N.
    \end{align*}

    Es gilt für alle $y \in \N:$

    \begin{align*}
        \operatorname p(y + 1) = y = g(y, z).
    \end{align*}

    Schließlich ist $\operatorname p = \Pr[f, g]$, als primitive Rekursion von primitiv rekursiven (Basis-)Funktionen, primitiv rekursiv.

    \item $f := \operatorname P_1^1$ ist, als Basisfunktion, primitiv rekursiv.
    
    \begin{align*}
        f(x) = x, \quad \text{für}~ x \in \N.
    \end{align*}
    
    Es gilt für alle $x \in \N:$

    \begin{align*}
        x \dotdiv 0 = x = f(x)
    \end{align*}
    
    $g = \Cn[\operatorname p, \operatorname P_3^3]$ ist, als Verknüpfung von primitiv rekursiven Funktionen, primitiv rekursiv.

    \begin{align*}
        g(x, y, z) = \operatorname p(z), \quad \text{für}~ x, y, z \in \N.
    \end{align*}

    Wir machen eine Fallunterscheidung, um zu zeigen, dass für alle $x, y \in \N:$

    \begin{align*}
        x \dotdiv (y + 1) = \operatorname p(x \dotdiv y) = g(x, y, x \dotdiv y), \quad \text{für}~ x, y \in \N.
    \end{align*}

    Seien also $x, y \in \N$.

    \begin{enumerate}[label = \arabic*.]

        \item Fall ($x \leq y + 1$):
        
        \begin{align*}
            x \dotdiv (y + 1)
            =
            0
            =
            \operatorname p(0)
            =
            \operatorname p(x \dotdiv y)
        \end{align*}

        \item Fall ($x > y + 1$):
            
        \begin{align*}
            x \dotdiv (y + 1)
            =
            x - (y + 1)
            =
            \underbrace{x - y}_{> 0} - 1
            =
            \underbrace{x \dotdiv y}_{> 0} - 1
            =
            \operatorname p(x \dotdiv y)
        \end{align*}    

    \end{enumerate}

    Schließlich ist $\operatorname \dotdiv = \Pr[f, g]$, als primitive Rekursion von primitiv rekursiven Funktionen, primitiv rekursiv.

\end{enumerate}

Es sei angemerkt, dass man mit Projektionen die Argumente immer umstrukturieren kann;
z.b.

\begin{align*}
    (a, b, c, x, y, z) \rightsquigarrow (z, x, x, c, a, z)
\end{align*}

durch

\begin{align*}
    (\operatorname P_6^6, \operatorname P_4^6, \operatorname P_4^6, \operatorname P_3^6, \operatorname P_1^6, \operatorname P_6^6).
\end{align*}

Nachdem die Projektionen allesamt primitiv rekursive (Basis-)Funktionen sind, ist diese Umstrukturierung auch \enquote{primitiv rekursiv}.

\begin{enumerate}[label = (\alph*), start = 4]

    \item Es gilt gilt für alle $x, y \in \N:$
    

    \begin{align*}
        x \lor y
        & :=
        \max(x, y) \\
        & =
        \begin{cases}
            x, & x \geq y, \\
            y, & x < y
        \end{cases} \\
        & =
        x + (y \dotdiv x),
    \end{align*}

    \begin{align*}
        x \land y
        & :=
        \min(x, y) \\
        & =
        x + y - \max(x, y),
    \end{align*}

    und

    \begin{align*}
        \chi_\leq(x, y)
        =
        \min(1, (y + 1) \dotdiv x).
    \end{align*}

    Die $\lor$-$\land$-Notation verwenden wir später noch.

    \item Es gilt für alle $x, y \in \N:$

    \begin{align*}
        \chi_=(x, y)
        =
        \chi_\leq(x, y) \cdot \chi_\leq(y, x).
    \end{align*}

    \item Mittels Induktion nach $n$ sieht man, dass für alle $\overline x \in \N^k:$
    
    \begin{align*}
        h(\overline x)
        =
        \sum_{i=0}^{n-1}
            f_i(\overline x)
            \cdot
            \chi_=(0, g(\overline x))
        +
        f_n(\overline x)
        \cdot
        \chi_\leq(n, g(\overline x))
    \end{align*}

    und $h$, als endliche Verknüpfung primitiv rekursiver Funktionen, primitiv ist.

\end{enumerate}

\end{solution}

% -------------------------------------------------------------------------------- %
