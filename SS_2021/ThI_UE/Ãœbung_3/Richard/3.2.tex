% -------------------------------------------------------------------------------- %

\begin{exercise}

Sei $f: \N^{k+1} \to \N$ primitiv rekursiv.
Zeigen Sie dass die folgenden Funktionen ebenfalls primitiv rekursiv sind:

\begin{enumerate}[label = (\alph*)]
    \item $(\overline x, z) \mapsto \sum_{y=0}^z f(\overline x, y)$
    \item $(\overline x, z) \mapsto \prod_{y=0}^z f(\overline x, y)$
\end{enumerate}

Sei nun $f: \N^{k+1} \to \Bbraces{0, 1}$ primitiv rekursiv.
Zeigen Sie dass die folgenden Funktionen ebenfalls primitiv rekursiv sind:

\begin{enumerate}[label = (\alph*), start = 3]
    \item $(\overline x, z) \mapsto \forall y \leq z f(\overline x, y) = \begin{cases} 1 & \text{falls für alle}~ y \in \Bbraces{0, \dots, z} ~\text{gilt:}~ f(\overline x, y) = 1 \\ 0 & \text{falls ein}~ y \in \Bbraces{0, \dots, z} ~\text{existiert so dass:}~ f(\overline x, z) = 0 \end{cases}$
    \item $(\overline x, z) \mapsto \exists y \leq z f(\overline x, y) = \begin{cases} 1 & \text{falls ein}~ y \in \Bbraces{0, \dots, z} ~\text{existiert so dass}~ f(\overline x, z) = 1 \\ 0 & \text{falls für alle}~ y \in \Bbraces{0, \dots, z} ~\text{gilt:}~ f(\overline x, y) = 0 \end{cases}$
\end{enumerate}

\end{exercise}

% -------------------------------------------------------------------------------- %

\begin{solution}

\phantom{}

\begin{enumerate}[label = (\alph*)]

    \item Wir schreiben (nur) hier $\varphi$ statt $f$, um das Symbol \enquote{$f$} zu reserveren.
    Seien

    \begin{align*}
        h(\overline x, y)
        :=
        \sum_{\eta = 0}^y
            \varphi(\overline x, \eta),
        \quad
        \text{für}~
        \overline x \in \N^k,
        y \in \N
    \end{align*}

    und

    \begin{align*}
        f & := \Cn[\varphi, \operatorname P_1^k, \dots, \operatorname P_k^k, 0], \\
        g & := \Cn[+, \operatorname P_{k+2}, \Cn[\varphi, \operatorname P_1^{k+2}, \dots, \operatorname P_k^{k+2}, \Cn[+, \operatorname P_{k+1}^{k+2}, 1]]],
    \end{align*}

    d.h. für $\overline x \in \R^k$, und $y, z \in \R$ ist

    \begin{align*}
        f(\overline x)
        & :=
        \varphi(\operatorname P_1^k(\overline x), \dots, \operatorname P_k^k(\overline x), 0) \\
        & =
        \varphi(\overline x, 0), \\
        g(\overline x, y, z)
        & :=
        \operatorname P_{k+2}^{k+2}(\overline x, y, z)
        +
        \varphi(\operatorname P_1^{k+2}(\overline x, y, z), \dots, \operatorname P_k^{k+2}(\overline x, y, z), \operatorname P_{k+1}^{k+2}(\overline x, y, z) + 1) \\
        & =
        z + \varphi(\overline x, y + 1).
    \end{align*}

    Dann gilt

    \begin{align*}
        h(\overline x, 0)
        & :=
        \sum_{\eta = 0}^0
            \varphi(\overline x, \eta) \\
        & =
        \varphi(\overline x, 0) \\
        & =
        f(\overline x), \\
        h(\overline x, y + 1)
        & =
        \sum_{\eta = 0}^{y+1}
            \varphi(\overline x, \eta) \\
        & =
        \sum_{\eta = 0}^y
            \varphi(\overline x, \eta)
        +
        \varphi(\overline x, y + 1) \\
        & =
        h(\overline x, y) + \varphi(y + 1) \\
        & =
        g(\overline x, y, h(\overline x, y)).
    \end{align*}

    \item Das sieht man analog zu (a), nur mit \enquote{$\prod$} statt \enquote{$\sum$}, und \enquote{$\cdot$} statt \enquote{$+$}.
    
    \item

    \begin{align*}
        \pbraces
        {
            \forall y \leq z
                f(\overline x, y)
        }
        =
        \prod_{y=1}^z
            f(\overline x, y).
    \end{align*}

    \item Wir treiben den Spaß weiter mit $\neg g := 1 \divdot g$, für $g: \N^{k+1} \to \Bbraces{0, 1}$.
    
    \begin{align*}
        \pbraces
        {
            \exists y \leq z
                f(\overline x, y)
        }
        =
        \pbraces
        {
            \neg
            \forall y \leq z
                \neg f(\overline x, y)
        }
    \end{align*}

\end{enumerate}

\end{solution}

% -------------------------------------------------------------------------------- %
