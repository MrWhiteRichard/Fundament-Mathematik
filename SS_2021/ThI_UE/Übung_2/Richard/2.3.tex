% --------------------------------------------------------------------------------

\begin{exercise}

Eine kontextfreie Grammatik $G = \abraces{N, T, P, S}$ heißt \textit{linear} wenn jede Produktion von der Form $A \to u B v$ oder $A \to u$ ist wobei $B \in N$, $u, v \in T^\ast$.
Eine Sprache $L$ heißt \text{linear} falls eine lineare Grammatik $G$ existiert, mit $\operatorname L(G) = L$.
Beweisen Sie den folgenden Schleifensatz (pumping lemma) für lineare Sprachen: \\

\textbf{Satz.}
Sei $L$ eine lineare Sprache.
Dann gibt es ein $n \in \N$ so dass jedes $w \in L$ mit $|w| \geq n$ geschrieben werden kann als $w = v_1 v_2 v_3 v_4 v_5$ so dass

\begin{enumerate}
    \item $v_2 v_4 \neq \varepsilon$,
    \item $|v_1 v_2 v_4 v_5| \leq n$, und
    \item für alle $k \geq 0$ ist auch $v_1 v_2^k v_3 v_4^k v_5 \in L$.
\end{enumerate}

\textit{Hinweis: Entfernen Sie zunächst alle Umbenennungen aus $G$. Setzen Sie $n = m_{|N| + 1}$ wobei \\ $m_k = \max \Bbraces{|u| + |v| \mid S \implies_g^{\leq k} u A v}$.}

\end{exercise}

% --------------------------------------------------------------------------------

\begin{solution}

ToDo!

\end{solution}

% --------------------------------------------------------------------------------
