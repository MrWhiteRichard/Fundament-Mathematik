% -------------------------------------------------------------------------------- %

\begin{exercise}

Sei $L = \Bbraces{w c^n \mid w \in \Bbraces{a, b}^\ast, n_a(w) = n ~\text{oder}~ n_b(w) = n}$.
Geben Sie eine kontextfreie Grammatik an die $L$ erzeugt.

\end{exercise}

% -------------------------------------------------------------------------------- %

\begin{solution}

Die fehlende Quantisierung für $n$ lässt 2 zulässige Interpretationen von $L$ zu.

\begin{enumerate}

    \item Interpretation ($L = L_n := \Bbraces{w c^n \mid w \in \Bbraces{a, b}^\ast, n_a(w) = n ~\text{oder}~ n_b(w) = n}$, $n \in \N$):
    
    An diese habe ich ursprünglich gedacht.

    Zwecks besseren Verständnisses, listen wir die Elemente von $L$ für verschiedene $n \in \N$ auf.
    Dabei bezeichnet $x \cdot x = x^m$ für irgendein beliebiges $m \in \N$.
    Wenn der Ausdruck mehrmals vorkommt, dürfen die $m$ auch unterschiedlich sein.
    
    \begin{itemize}
    
        \item $n = 0$:
        
        \begin{align*}
            a \cdots a \\
            b \cdots b \\
        \end{align*}
    
        \item $n = 1$:
        
        \begin{align*}
            (a \cdots a) b (a \cdots a) c \\
            (b \cdots b) a (b \cdots b) c \\
        \end{align*}
    
        \item $n = 2$:
        
        \begin{align*}
            (a \cdots a) b (a \cdots a) b (a \cdots a) c^2 \\
            (b \cdots b) a (b \cdots b) a (b \cdots b) c^2 \\
        \end{align*}
    
        \item $n \in \N$:
        
        \begin{align*}
            \underbrace
            {
                ((a \cdots a) b)
                \cdots
                ((a \cdots a) b)
            }_{
                \text{$n$-mal}
            }
            (a \cdots a)
            c^n \\
            \underbrace
            {
                ((b \cdots b) a)
                \cdots
                ((b \cdots b) a)
            }_{
                \text{$n$-mal}
            }
            (b \cdots b)
            c^n \\
        \end{align*}
    \end{itemize}
    
    Wir wollen nun all diese Wörter systematisch aufbauen, mit einem endlichem Regelwerk $P$.
    Sei dabei aber o.B.d.A. $n \neq 0$.
    
    \begin{align*}
        P
        =
        \begin{cases}
            S \to X^n C^n \mid Y^n C^n, \\
            X \to X B \mid B X \mid A, \\
            Y \to Y A \mid A Y \mid B, \\
            A \to a, \\
            B \to b, \\
            C \to c            
        \end{cases}
    \end{align*}
    
    Die kontextfreie Grammatik $G = \abraces{N, T, P, S}$ erzeugt $L$.
    
    \begin{align*}
        N = \Bbraces{S, X, Y, A, B, C},
        \quad
        T = \Bbraces{a, b, c}
    \end{align*}

    \item Interpretation ($L = \bigcup_{n \in \N} L_n$):
    
    Man wäre möglicherweise nun dazu geneigt, Satz 2.2 zu verwenden; d.h. die Abschlusseigenschaft kontextfreier Sprachen unter Vereinigung.
    Diese wurde aber nur für endliche Vereinigungen bewiesen.
    Wir gehen also zu Fuß mit $G = \abraces{N, T, P, S}$, wobei

    \begin{align*}
        N = \Bbraces{S, A, B},
        \quad
        T = \Bbraces{a, b},
        \quad
        P
        =
        \begin{cases}
            S \to A \mid B, \\
            A \to a A c \mid b A \mid \varepsilon, \\
            B \to b B c \mid a B \mid \varepsilon.
        \end{cases}
    \end{align*}

\end{enumerate}

Die missverständliche Formulierung der Angabe kostet zwar etwas Zeit, wirft aber gleichzeitig die Frage auf, wann unendliche Vereinigeungen (oder vielleicht sogar Grenzwerte) von kontextfreien Sprachen wieder kontextfrei sind.

\end{solution}

% -------------------------------------------------------------------------------- %
