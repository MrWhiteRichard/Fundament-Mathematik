% -------------------------------------------------------------------------------- %

\begin{exercise}

Eine Klausel ist eine Menge von Literalen.
Ein Literal ist ein Atom oder ein negiertes Atom.
Seien $C$ und $D$ Klauseln und $p$ ein Atom so dass $p \in C$ und $p \in D$.
Dann wird $E = (C \setminus \Bbraces{p}) \cup (D \setminus \Bbraces{\neg p})$ als \textit{Resolvente} von $C$ und $D$ bezeichnet.
Die \textit{Resolutionsregel} erlaubt die Ableitung von $E$ aus $C$ und $D$.
Die leere Klausel $\emptyset$ entspricht, als leere Disjunktion, der Aussage \enquote{falsch}.
Eine Klauselmenge $\mathcal C$ ist unerfüllbar genau dann wenn sich aus $\mathcal C$ mittels Resolution die leere Klausel ableiten lässt.
Wir betrachten das folgende Entscheidungsproblem:

\begin{center}
    \begin{hetzlbox}[title = 2SAT]
        
        Eingabe:
        Klauselmenge $\mathcal C$ in der jede Klausel höchstens $2$ Literale enthält
    
        Frage:
        Ist $\mathcal C$ erfüllbar?
    
    \end{hetzlbox}   
\end{center}

Zeigen Sie dass $\mathrm{2SAT} \in \mathbf P$ ist.

\end{exercise}

% -------------------------------------------------------------------------------- %

\begin{solution}

\phantom{}

\end{solution}

% -------------------------------------------------------------------------------- %
