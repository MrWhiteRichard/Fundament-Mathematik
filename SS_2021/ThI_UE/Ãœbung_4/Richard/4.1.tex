% -------------------------------------------------------------------------------- %

\begin{exercise}

Eine Klausel ist eine Menge von Literalen.
Ein Literal ist ein Atom oder ein negiertes Atom.
Seien $C$ und $D$ Klauseln und $p$ ein Atom so dass $p \in C$ und $p \in D$.
Dann wird $E = (C \setminus \Bbraces{p}) \cup (D \setminus \Bbraces{\neg p})$ als \textit{Resolvente} von $C$ und $D$ bezeichnet.
Die \textit{Resolutionsregel} erlaubt die Ableitung von $E$ aus $C$ und $D$.
Die leere Klausel $\emptyset$ entspricht, als leere Disjunktion, der Aussage \enquote{falsch}.
Eine Klauselmenge $\mathcal C$ ist unerfüllbar genau dann wenn sich aus $\mathcal C$ mittels Resolution die leere Klausel ableiten lässt.
Wir betrachten das folgende Entscheidungsproblem:

\begin{center}
    \begin{hetzlbox}[title = 2SAT]
        
        Eingabe:
        Klauselmenge $\mathcal C$ in der jede Klausel höchstens $2$ Literale enthält
    
        Frage:
        Ist $\mathcal C$ erfüllbar?
    
    \end{hetzlbox}   
\end{center}

Zeigen Sie dass $\mathrm{2SAT} \in \mathbf P$ ist.

\end{exercise}

% -------------------------------------------------------------------------------- %

\begin{solution}

Es gibt bloß folgende Formen der Resolution:

\begin{align}
    \emptyset & = \Res(\Bbraces{L}, \Bbraces{\overline L}), \label{eq:res_1} \tag{1} \\
    \Bbraces{L} & = \Res(\Bbraces{K}, \Bbraces{\overline K, L}), \label{eq:res_2.1} \tag{2.1} \\
    \Bbraces{L} & = \Res(\Bbraces{K, L}, \Bbraces{\overline K, L}), \label{eq:res_2.2} \tag{2.2} \\
    \Bbraces{L, K} & = \Res(\Bbraces{X, L}, \Bbraces{\overline X, K}); \label{eq:res_3} \tag{3}
\end{align}

Nachdem wir 2SAT und nicht 3SAT betrachten, gibt es keine

\begin{align}
    \Bbraces{L, K} & = \Res(\Bbraces{X, L, K}, \Bbraces{\overline X}), \label{eq:res_3.1} \tag{3.1} \\
    \Bbraces{L, K} & = \Res(\Bbraces{X, L, K}, \Bbraces{\overline X, K}), \label{eq:res_3.2} \tag{3.2} \\
    \Bbraces{L, K} & = \Res(\Bbraces{X, L, K}, \Bbraces{\overline X, K, L}). \label{eq:res_3.3} \tag{3.3} \\
\end{align}

Wenn wir \eqref{eq:res_1} anwenden, dann sind wir fertig.

Wenn wir \eqref{eq:res_2.1} anwenden müssen, dann könnte es sein, dass $\Bbraces{K}$ erst durch \eqref{eq:res_2.1} oder \eqref{eq:res_2.2} erzeugt werden muss.
Das werden wir auch tun.

\begin{enumerate}[label = \arabic*., start = 0]

    \item Das gleiche könnte ad hoc auch für $\Bbraces{\overline K, L}$ und \eqref{eq:res_3} gelten.
    $\Bbraces{\overline K, L}$ erhält man aber (nur) aus \eqref{eq:res_3}, d.h.

    \begin{align*}
        \Bbraces{\overline K, L}
        =
        \Res(\Bbraces{X, \overline K}, \Bbraces{\overline X, L}).
    \end{align*}

    Wir haben aber bereits $\Bbraces{K}$.
    Also können wir stattdessen auch mit \eqref{eq:res_2.1} arbeiten, d.h.

    \begin{align} \label{eq:res_bsp_0} \tag{4}
        \Bbraces{X, \overline K}
        \xrightarrow{\Res(\Bbraces{K}, \cdot)}
        \Bbraces{X}
        \xrightarrow{\Res(\cdot, \Bbraces{\overline X, L})}
        \Bbraces{L}.
    \end{align}

    \item Wenn $\Bbraces{X, \overline K}$ durch \eqref{eq:res_3} hervorgeht, dann sei

    \begin{align*}
        \Bbraces{X, \overline K}
        =
        \Res(\Bbraces{Y, X}, \Bbraces{\overline Y, \overline K}).        
    \end{align*}

    Wir haben aber bereits $\Bbraces{K}$.
    Also können wir stattdessen auch mit \eqref{eq:res_2.1} arbeiten, d.h.

    \begin{align*}
        \Bbraces{\overline Y, \overline K}
        \xrightarrow{\Res(\Bbraces{K}, \cdot)}
        \Bbraces{\overline Y}
        \xrightarrow{\Res(\cdot, \Bbraces{Y, X})}
        \Bbraces{X}.
    \end{align*}

    Von da aus können wir mit \eqref{eq:res_bsp_0} weitermachen.

    \item Wenn $\Bbraces{\overline X, L}$ durch \eqref{eq:res_3} hervorgeht, dann sei
    
    \begin{align*}
        \Bbraces{\overline X, L}
        =
        \Res(\Bbraces{Y, \overline X}, \Bbraces{\overline Y, L}).
    \end{align*}

    Wir haben aber bereits $\Bbraces{K}$.
    Wegen 1. haben wir damit auch $\Bbraces{X}$.
    Also können wir stattdessen auch mit \eqref{eq:res_2.1} arbeiten, d.h.

    \begin{align*}
        \Bbraces{Y, \overline X}
        \xrightarrow{\Res(\Bbraces{X}, \cdot)}
        \Bbraces{Y}
        \xrightarrow{\Res(\cdot, \Bbraces{\overline Y, L})}
        \Bbraces{L}.
    \end{align*}

    \item Falls $\Bbraces{Y, X}$, $\Bbraces{\overline Y, \overline K}$ oder $\Bbraces{Y, \overline X}$, $\Bbraces{\overline Y, L}$ durch \eqref{eq:res_3} hervorgeht, so können wir jeweils 1. oder 2. analog darauf anwenden.

\end{enumerate}

Wir können also \eqref{eq:res_3}, in gewisser Weise, via \eqref{eq:res_2.1}, umgehen.
An der Stelle noch zwei Bemerkungen.

\begin{itemize}

    \item Durch \eqref{eq:res_2.1} erzeugen wir ein Singleton.
    \eqref{eq:res_2.1} arbeitet mit einem Singleton.
    Es kann also sein, dass man \eqref{eq:res_2.1} mit einem Erzeugnis von \eqref{eq:res_2.1} anwenden muss.
    D.h. also, wir müssen \eqref{eq:res_2.1} laufend anwenden, solange es nur geht.

    \item Durch \eqref{eq:res_2.2} erzeugen wir auch ein Singleton.
    \eqref{eq:res_2.2} arbeitet aber nicht mit Singletons.
    Es kann also nicht sein, dass man \eqref{eq:res_2.2} it einem Erzeugnis von \eqref{eq:res_2.1} oder \eqref{eq:res_2.2} anwenden muss.
    D.h. also, wir müssen \eqref{eq:res_2.2} und \eqref{eq:res_3} bloß am Anfang ausschöpfen und uns dann erst um \eqref{eq:res_2.1} kümmern.

\end{itemize}

Nun zu unserem Algorithmus als Pseudocode.
Sei dazu $\mathcal C$ eine passende Klauselmenge, d.h. für alle $C \in \mathcal C$ gilt $|C| \leq 2$.
Wir verwenden die Notation $V(G)$, $E(G)$, und $w(G)$, für die Knoten- und Kanten-Menge, bzw. Gewichts-Funktion eines Graphen $G$.

\begin{algorithm}[H]
    $\mathcal C_\text{sing} := \Bbraces{C \in \mathcal C \mid |C| = 1}$ \\
    $V(G_0) := \Bbraces{C \in \mathcal C \mid |C| = 2}$ \\
    $E(G_0) := \Bbraces{\Bbraces{C, D} \mid C, D \in \mathcal C, C \neq D, C, D ~\text{resolvierbar}, \Res(C, D) \not \in V(G_0)}$ \\
    $w(G_0): E(G_0) \to \bigcup \mathcal C: e \mapsto \Bbraces{L \mid e ~\text{kann entlang}~ L ~\text{resolviert werden.}} \neq \emptyset$ \\
    $n := 0$ \\
    \While{$|E(G_n)| \neq 0$}
    {
        $G_{n+1} := G_n$ \\
        wähle $e \in E_{n+1}$ \\
        wähle $L \in w(G_{n+1})(e)$ und $w(G_{n+1})(e) := w(G_{n+1})(e) \setminus \Bbraces{L}$ \\
        \If{$|w(G_{n+1})(e)| = 0$}
        {
            $E(G_{n+1}) := E(G_{n+1}) \setminus \Bbraces{e}$ \\
        }
        $R := \Res_L(e)$ \\
        \eIf{$|R| = 1$}
        {
            $\mathcal C_\text{sing} := \mathcal C_\text{sing} \cup \Bbraces{R}$ \\
        }{
            $V(G_{n+1}) := V(G_{n+1}) \cup \Bbraces{R}$ \\
            $E_\text{new} := \Bbraces{\Bbraces{V, R} \mid v \in V(G_{n+1}), ~ v, r ~\text{resolvierbar}, ~ \Res(v, R) \not \in V(G_{n+1})}$ \\
            $E(G_{n+1}) := E(G_{n+1}) \cup E_\text{new}$ \\
            $w(G_{n+1}) := w(G_{n+1}) \cup \Bbraces{(e^\prime, \Bbraces{L^\prime \mid e ~\text{kann entlang}~ L^\prime ~\text{resolviert werden}}) \mid e^\prime \in E_\text{new}}$ \\
        }
        $n := n + 1$ \\
    }
    \caption
    {
        Ausschöpfen von \eqref{eq:res_2.2} und \eqref{eq:res_3}
    }
\end{algorithm}

Sei $n$ die Anzahl der Variablen in $\mathcal C$.
Es gibt nicht mehr als

\begin{align*}
    \binom{n}{2}
    =
    \frac{n!}{(n-2)! 2!}
    =
    \frac{n (n-1)}{2}
\end{align*}

Paarmengen auf $\operatorname L(\mathcal C)$.
Mehr Knoten können also nicht erzeugt werden.
Demnach können nicht mehr als doppelt so viele Kanten erzeugt werden.
In jeder Schleifen-Instanz wird eine Kante entfernt.
In jeder Schleifen-Instanz wird der ganze Graph (in polynomieller Zeit) durchlaufen.
Irgendwann wird es also keine Kanten mehr geben und die Schleife terminiert (in polynomieller Zeit).

\begin{algorithm}[H]
    $\mathcal C_\text{pair} := \Bbraces{C \in \mathcal C \mid |C| = 2} = \mathcal C \setminus \mathcal C_\text{sing}$ \\
    $\mathcal C_\text{sing}^\prime := \emptyset$ \\
    \While{$|\mathcal C_\text{sing}| \neq 0$}
    {
        wähle $C \in \mathcal C_\text{sing}$ und $\mathcal C_\text{sing} := \mathcal C_\text{sing} \setminus \Bbraces{C}$, $\mathcal C_\text{sing}^\prime := \mathcal C_\text{sing}^\prime \cup \Bbraces{C}$ \\
        \For{$D \in \mathcal C_\text{pair}$}
        {
            $\mathcal C_\text{sing} := \mathcal C_\text{sing} \cup \Bbraces{\Res_L(C, D) \mid \Bbraces{C, D} ~\text{kann entlang}~ L ~\text{resolviert werden}}$ \\
        }
    }
    \caption
    {
        Ausschöpfen von \eqref{eq:res_2.1}
    }
\end{algorithm}

Es gibt nur begrenzt viele Singletons die durch Resolution entstehen können.
$\mathcal C_\text{sing}$ wird immer leerer und es besteht eine immer geringere Chance, in einer while-Schleifen-Instanz, Singletons durch Resolution zu erhalten.
Auch die for-Schleife ist polynomiell beschränkt.

\begin{algorithm}[H]
    \For{$C \in \mathcal C_\text{sing}^\prime$}
    {
        \For{$D \in \mathcal C_\text{sing}^\prime \setminus \Bbraces{C}$}
        {
            \If{$\emptyset \in \Bbraces{\Res_L(C, D) \mid \Bbraces{C, D} ~\text{kann entlang}~ L ~\text{resolviert werden}}$}
            {
                \KwRet nein
            }
        }
    }
    \KwRet ja
    \caption
    {
        Ausschöpfen von \eqref{eq:res_1}
    }
\end{algorithm}

Die Laufzeit ist hier beschränkt durch $\Landau(|\mathcal C_\text{sing}|^2)$.

\end{solution}

% -------------------------------------------------------------------------------- %
