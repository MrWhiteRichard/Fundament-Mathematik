% -------------------------------------------------------------------------------- %

\begin{exercise}

Sei $G = (V, E)$ ein ungerichteter Graph.
Eine \textit{Knotenüberdeckung} von $G$ ist eine Menge $V^\prime \subseteq V$ so dass

\begin{align*}
    \Bbraces{u, v} \in E
    \implies
    u \in V^\prime ~\text{oder}~ v \in V^\prime.
\end{align*}

Das Knotenüberdeckungsproblem ist:

\begin{center}
    \begin{hetzlbox}[title = Knotenüberdeckung]
    
        Eingabe:
        ein endlicher ungerichteter Graph $G$, ein $k \in \N$
        
        Frage:
        Ist $\mathcal C$ erfüllbar?
    
    \end{hetzlbox}    
\end{center}

Zeigen Sie dass Knotenüberdecknung $\mathbf{NP}$-vollständig ist.

\begin{itshape}
    Hinweis:
    Reduzieren Sie $\mathrm{3SAT}$ auf Knotenüberdecknung indem Sie zur Darstellung eines Atoms $p$ den Baustein

    \begin{center}
        \begin{tikzpicture}[node distance = 2cm]

            \node [state]                   (not_p) {$\neg p$};
            \node [state, right of = not_p] (p)     {$p$};
        
            \draw (not_p) edge (p);
        
        \end{tikzpicture}
    \end{center}

    verwenden und zur Darstellung einer Klausel $l_1 \lor l_2 \lor l_3$ den Baustein

    \begin{center}
        \begin{tikzpicture}[node distance = 2cm]

            \node [state]                       (l_1) {$l_1$};
            \node [state, below left  of = l_1] (l_3) {$l_3$};
            \node [state, below right of = l_1] (l_2) {$l_2$};
        
            \draw (l_1) edge (l_2) (l_2) edge (l_3) (l_3) edge (l_1);
        
        \end{tikzpicture}
    \end{center}

\end{itshape}

\end{exercise}

% -------------------------------------------------------------------------------- %

\begin{solution}

Sei $\mathcal C$ eine Klauselmenge mit Klauseln $C \in \mathcal C$, sodass $|C| \leq 3$.
Betrachte den Graphen $G = (V, E)$ mit $V = V_\text{Atom} \cup V_\text{Klausel}$.
Für jedes Atom $p \in \operatorname L(\mathcal C)$ seien $(p, \_, \_)$, $(\neg p, \_, \_)$ durch Kanten aus $E$ verbundene (Atom-)Knoten aus $V_\text{Atom}$.
Für jede Klausel

\begin{itemize}
    \item $\Bbraces{L} = C \in \mathcal C$ seien $(L, C, 1), (L, C, 2), (L, C, 3)$
    \item $\Bbraces{L_1, L_2} = C \in \mathcal C$ seien $(L_1, C, 1), (L_2, C, 1), (L_2, C, 2)$ oder $(L_1, C, 1), (L_1, C, 2), (L_2, C, 1)$
    \item $\Bbraces{L_1, L_2, L_3} = C \in \mathcal C$ seien $(L_1, C, 1), (L_2, C, 1), (L_3, C, 1)$
\end{itemize}

durch Kanten aus $E$ verbundene (Klausel-)Knoten aus $V_\text{Klausel}$.
Es seien zwei (verschiedene) Knoten aus $V_\text{Atom}$ bzw. $V_\text{Klausel}$ ebenfalls durch eine Kante aus $E$ verbunden, wenn ihre ersten Komponenten gleich sind.
Sonst mögen $V$ und $E$ nichts Weiteres enthalten.

\begin{example*}

    Betrachte die Klauselmenge

    \begin{align*}
        \mathcal C
        :=
        \Bbraces{\Bbraces{\neg p_1}; \Bbraces{p_1, \neg p_2}; \Bbraces{p_1, p_2, \neg p_3}}.
    \end{align*}

    Diese wird erfüllt durch die Belegung

    \begin{align*}
        b:
        \begin{cases}
            p_1 \mapsto 0, \\
            p_2 \mapsto 0, \\
            p_3 \mapsto 0.
        \end{cases}
    \end{align*}

    Im folgenden Graphen sind die roten Kanten nicht in der Knotenüberdeckung enthalten, die grünen allerdings schon.

    \begin{center}
        \begin{tikzpicture}[node distance = 2cm]

            % Atom
            \node [state, color = red]                        (atom_p_1)     {$p_1$};
            \node [state, color = green, below of = atom_p_1] (atom_p_1_not) {$\neg p_1$};
            \draw (atom_p_1) edge (atom_p_1_not);
        
            % Klausel
            \node [state, color = green, right of       = atom_p_1_not]        (klausel_1_literal_2) {$\neg p_1$};
            \node [state, color = red,   above right of = klausel_1_literal_2] (klausel_1_literal_1) {$\neg p_1$};
            \node [state, color = green, below right of = klausel_1_literal_1] (klausel_1_literal_3) {$\neg p_1$};
            \draw (klausel_1_literal_1) edge (klausel_1_literal_2)
                  (klausel_1_literal_2) edge (klausel_1_literal_3)
                  (klausel_1_literal_3) edge (klausel_1_literal_1);
        
            % Atom
            \node [state, color = red,   below of = atom_p_1_not] (atom_p_2)     {$p_2$};
            \node [state, color = green, below of = atom_p_2]     (atom_p_2_not) {$\neg p_2$};
            \draw (atom_p_2) edge (atom_p_2_not);
        
            % Klausel
            \node [state, color = red,   right of       = atom_p_2_not]        (klausel_2_literal_2) {$\neg p_2$};
            \node [state, color = green, above right of = klausel_2_literal_2] (klausel_2_literal_1) {$p_1$};
            \node [state, color = green, below right of = klausel_2_literal_1] (klausel_2_literal_3) {$\neg p_2$};
            \draw (klausel_2_literal_1) edge (klausel_2_literal_2)
                  (klausel_2_literal_2) edge (klausel_2_literal_3)
                  (klausel_2_literal_3) edge (klausel_2_literal_1);
        
            % Atom
            \node [state, color = red,   below of = atom_p_2_not] (atom_p_3)     {$p_3$};
            \node [state, color = green, below of = atom_p_3]     (atom_p_3_not) {$\neg p_3$};
            \draw (atom_p_3) edge (atom_p_3_not);
        
            % Klausel
            \node [state, color = green, right of       = atom_p_3_not]        (klausel_3_literal_2) {$p_2$};
            \node [state, color = green, above right of = klausel_3_literal_2] (klausel_3_literal_1) {$p_1$};
            \node [state, color = red,   below right of = klausel_3_literal_1] (klausel_3_literal_3) {$\neg p_3$};
            \draw (klausel_3_literal_1) edge (klausel_3_literal_2)
                  (klausel_3_literal_2) edge (klausel_3_literal_3)
                  (klausel_3_literal_3) edge (klausel_3_literal_1);
                  
            % other edges
        
            \draw (klausel_1_literal_1) edge             (atom_p_1_not);
            \draw (klausel_1_literal_2) edge             (atom_p_1_not);
            \draw (klausel_1_literal_3) edge [bend left] (atom_p_1_not);
        
            \draw (klausel_2_literal_1) -- (2, -4.6) --  (atom_p_1);
            \draw (klausel_2_literal_2) edge             (atom_p_2_not);
            \draw (klausel_2_literal_3) edge [bend left] (atom_p_2_not);
        
            \draw (klausel_3_literal_1) -- (1, -8.6) -- (1, 0) -- (atom_p_1);
            \draw (klausel_3_literal_2) edge                      (atom_p_2);
            \draw (klausel_3_literal_3) edge [bend left]          (atom_p_3_not);
        
        \end{tikzpicture}
    \end{center}

\end{example*}

\enquote{$\implies$}:

Sei $\mathcal C$ erfüllbar und $b: \operatorname L(\mathcal C) \to \Bbraces{0, 1}$ eine Belegung mit $\hat b(\mathcal C) = 1$.
Weil alle Klauseln $C \in \mathcal C$ ja $\hat b(C) = 1$ erfüllen, also jeweils für mindestens ein $L \in C$ gilt $\hat b(L) = 1$, können wir eine Auswahlfunktion $f$ betrachten:

\begin{align*}
    f:
        \mathcal C \to \bigcup \mathcal C:
        C \mapsto f(C) \in \Bbraces{L \in C \mid \hat b(L) = 1} \neq \emptyset.
\end{align*}

Seien

\begin{align*}
    V_\text{Atom}^\prime
    & :=
    \Bbraces{(L, \_, \_) \in V_\text{Atom} \mid L \in \bigcup \mathcal C, \hat b(L) = 1}, \\
    V_\text{Klausel}^\prime
    & :=
    \Bbraces{v \in V_\text{Klausel} \mid C \in \mathcal C, v \neq (f(C), C, 1)},
\end{align*}

und $V^\prime := V_\text{Atom}^\prime \cup V_\text{Klausel}^\prime$.
Dann ist $V^\prime$ eine Knotenüberdeckung von $V$ mit $|V^\prime| \leq k = |\operatorname L(\mathcal C)| + 2 |\mathcal C|$.

\begin{enumerate}[label = \arabic*.]
    
    \item Sei $e = \Bbraces{v_1, v_2}$ eine Kante zwischen zwei Atom-Knoten $v_1 = (p, \_, \_), v_2 = (\neg p, \_, \_) \in V_\text{Atom}$.
    Wenn $b(p) = 1$, dann ist $v_1 \in V_\text{Atom}^\prime \subseteq V^\prime$, sonst ist $b(\neg p) = 1$ und $v_2 \in V_\text{Atom}^\prime \subseteq V^\prime$.

    \item Sei $e = \Bbraces{v_1, v_2}$ eine Kante zwischen zwei Klausel-Knoten $v_1, v_2 \in V_\text{Klausel}$.
    Es aknn nicht $v_1 \not \in V_\text{Klausel}^\prime$ und und $v_2 \not \in V_\text{Klausel}^\prime$, weil genau ein Knote aus dem Bauteil fehlt.
    Wende De Morgan an.

    \item Sei $e = \Bbraces{v_1, v_2}$ eine Kante zzwischen einem Atom-Knoten $v_1 = (L, \_, \_) \in V_\text{Atom}$ und Klausel-Knoten $v_2 = (L, C, i) \in V_\text{Klausel}$.
    Angenommen, $v_1, v_2 \not \in V^\prime$, dann gilt

    \begin{align*}
        v_1 \not \in V_\text{Atom},
        \quad
        \text{d.h.}
        \quad
        \hat b(L) = 0,
    \end{align*}

    und

    \begin{align*}
        v_2 \not \in V_\text{Klausel},
        \quad
        & \text{d.h.}
        \quad
        L = f(C), \quad i = 1, \\
        & \text{also}
        \quad
        \hat b(L) = \hat b(f(C)) = 1.
    \end{align*}

    Widerspruch!

\end{enumerate}

\enquote{$\impliedby$}:

Sei $V^\prime$ eine Knotenüberdeckung von $V$ mit $|V^\prime| \leq k = |\operatorname L(\mathcal C)| + 2 |\mathcal C|$.
Weil die Klausel-Knoten in den jeweiligen Klausel-Bauteilen alle verbunden sind, müssten jeweils $2$ davon in $V^\prime$ liegen, insgesmt also $2 |\mathcal C|$.
Weil die Atom-Knoten in den jeweiligen Klausel-Bauteilen alle verbunden sind, muss jeweils $1$ davon in $V^\prime$ liegen, insgesmt also $|\operatorname L(\mathcal C)|$.
Das sind alle; insgesamt insgesamt also $k = |V^\prime|$.

Sei

\begin{align*}
    b:
        \operatorname L(\mathcal C) \to \Bbraces{0, 1}:
        p
        \mapsto
        \begin{cases}
            1, & (p, \_, \_) \in V^\prime, \\
            0, & \text{sonst}.
        \end{cases}
\end{align*}

Sei $C \in \mathcal C$ eine beliebige Klausel und $v_2 := (L, C, i) \not \in V^\prime$, insbesondere also $L \in C$.
Laut Konstruktion von $G$, gibt es eine Kante $e = \Bbraces{v_1, v_2}$, mit $v_1 = \Bbraces{L, \_, \_} \in V_\text{Atom}$ ein Atom-Knoten ist.
Weil $V^\prime$ eine Knotenüberdeckung ist, $e = \Bbraces{v_1, v_2} \in E$, und $v_2 \not \in V^\prime$, muss $v_1 \in V^\prime$.
Per Definitionem von $b$, gilt daher

\begin{align*}
    1 = \hat b(L) \leq \hat b(C) \leq 1.
\end{align*}

Nachdem das also für alle Klauseln gilt, folgt

\begin{align*}
    \hat b(\mathcal C)
    =
    \bigwedge_{C \in \mathcal C}
        \underbrace{\hat b(C)}_1
    =
    1.
\end{align*}

Nun ist

\begin{itemize}
    \item laut Aufgabe 2, $L := \textsc{3Sat}$ $\mathbf{NP}$-vollständig,
    \item laut Satz 4.3, $L^\prime := \textsc{Knotenüberdecknung} \in \mathbf{NP}$, und
    \item laut dem soeben Gezeigten, $L$ auf $L^\prime$ polynomiell reduzierbar, i.Z. $L \leq_\mathrm{p} L^\prime$.
\end{itemize}

Laut Lemma 4.2, ist also auch $L^\prime$ $\mathbf{NP}$-vollständig.

\end{solution}

% -------------------------------------------------------------------------------- %
