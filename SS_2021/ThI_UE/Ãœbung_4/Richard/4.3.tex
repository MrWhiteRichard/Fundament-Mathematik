% -------------------------------------------------------------------------------- %

\begin{exercise}

Sei $G = (V, E)$ ein ungerichteter Graph.
Eine \textit{Knotenüberdeckung} von $G$ ist eine Menge $V^\prime \subseteq V$ so dass

\begin{align*}
    \Bbraces{u, v} \in E
    \implies
    u \in V^\prime ~\text{oder}~ v \in V^\prime.
\end{align*}

Das Knotenüberdeckungsproblem ist:

\begin{center}
    \begin{hetzlbox}[title = Knotenüberdeckung]
    
        Eingabe:
        ein endlicher ungerichteter Graph $G$, ein $k \in \N$
        
        Frage:
        Ist $\mathcal C$ erfüllbar?
    
    \end{hetzlbox}    
\end{center}

Zeigen Sie dass Knotenüberdecknung $\mathbf{NP}$-vollständig ist.

\begin{itshape}
    Hinweis:
    Reduzieren Sie $\mathrm{3SAT}$ auf Knotenüberdecknung indem Sie zur Darstellung eines Atoms $p$ den Baustein

    \begin{center}
        \begin{tikzpicture}[node distance = 2cm]

            \node [state]                   (not_p) {$\neg p$};
            \node [state, right of = not_p] (p)     {$p$};
        
            \draw (not_p) edge (p);
        
        \end{tikzpicture}
    \end{center}

    verwenden und zur Darstellung einer Klausel $l_1 \lor l_2 \lor l_3$ den Baustein

    \begin{center}
        \begin{tikzpicture}[node distance = 2cm]

            \node [state]                       (l_1) {$l_1$};
            \node [state, below left  of = l_1] (l_3) {$l_3$};
            \node [state, below right of = l_1] (l_2) {$l_2$};
        
            \draw (l_1) edge (l_2) (l_2) edge (l_3) (l_3) edge (l_1);
        
        \end{tikzpicture}
    \end{center}

\end{itshape}

\end{exercise}

% -------------------------------------------------------------------------------- %

\begin{solution}

\phantom{}

\end{solution}

% -------------------------------------------------------------------------------- %
