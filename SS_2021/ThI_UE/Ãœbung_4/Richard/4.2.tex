% -------------------------------------------------------------------------------- %

\begin{exercise}

Sei $\mathcal C$ eine Klauselmenge, sei $C \in \mathcal C$ eine Klausel und $L_1, L_2 \in C$.
Sei $C_0 = C \setminus \Bbraces{L_1, L_2}$.
Finden Sie, unter Zuhilfenahme eines neuen Atoms $p$, Klauseln $C_1, \dots, C_n$ die jeweils höchstens drei Literale enthalten so dass $\mathcal C$ erfüllbar ist genau dann wenn

\begin{align*}
    (\mathcal C \setminus \Bbraces{C})
    \cup
    \Bbraces{C_0 \cup \Bbraces{p}, C_1, \dots, C_n}
\end{align*}

erfüllbar ist.
Wir betrachten das folgende Entscheidungsproblem:

\begin{center}
    \begin{hetzlbox}[title = 3SAT]
        
        Eingabe:
        Klauselmenge $\mathcal C$ in der jede Klausel höchstens $3$ Literale enthält
    
        Frage:
        Ist $\mathcal C$ erfüllbar?
    
    \end{hetzlbox}    
\end{center}

Zeigen Sie dass $\mathrm{3SAT}$ $\mathbf{NP}$-vollständig ist.

\end{exercise}

% -------------------------------------------------------------------------------- %

\begin{solution}

Zunächst ein kleines \dots

\begin{lemma*}

    \phantom{}

    \begin{itemize}

        \item Wenn $C_1 \subseteq C_2 = C$ Klauseln sind, dann gilt, für alle Belegungen $b: \operatorname L(\Bbraces{C_1, C_2}) \to \Bbraces{0, 1}$, dass

        \begin{align*}
            \hat b(C_1) \leq \hat b(C_2)
            \quad
            \text{und}
            \quad
            \hat b(C) = 1
            \iff
            \Exists L \in C:
                \hat b(L) = 1.
        \end{align*}

        \item Wenn $\mathcal C_1 \subseteq \mathcal C_2 = \mathcal C$ Klauselmengen sind, dann gilt für alle Belegungen $b: \operatorname L(\mathcal C_1 \cup \mathcal C_2) \to \Bbraces{0, 1}$, dass

        \begin{align*}
            \hat b(\mathcal C_1) \geq \hat b(\mathcal C_2)
            \quad
            \text{und}
            \quad
            \hat b(\mathcal C) = 1
            \iff
            \Forall C \in \mathcal C:
                \hat b(C) = 1.
        \end{align*}

    \end{itemize}

\end{lemma*}

Seien $L_3, \dots, L_m$ die restlichen Literale aus $C$, d.h.

\begin{align*}
    C = \Bbraces{L_1, \dots, L_m}
    \supset
    C_0 = \Bbraces{L_3, \dots, L_m}.
\end{align*}

Sei weiters

\begin{align*}
    \mathcal C^\prime
    :=
    (\mathcal C \setminus \Bbraces{C})
    \cup
    \Bbraces{C_0 \cup \Bbraces{p}, C_1, \dots, C_n}.
\end{align*}

\enquote{$\impliedby$}:
Sei $\mathcal C^\prime$ erfüllbar.
Dann gibt es eine Belegung $b: \operatorname L(\mathcal C^\prime) \to \Bbraces{0, 1}$, sodass $\hat b(\mathcal C^\prime) = 1$.
Insbesondere gilt also

\begin{align*}
    \hat b(\mathcal C \setminus \Bbraces{C}) = 1,
    \quad
    \text{und}
    \quad
    \hat b(C_0 \cup \Bbraces{p}) = \hat b(C_1) = \dots = \hat b(C_n) = 1.
\end{align*}

\begin{enumerate}[label = \arabic*.]

    \item Fall ($b(p) = 0$):
    
    Angenommen, $\hat b(C_0) = 0$, dann wäre

    \begin{align*}
        1
        =
        \hat b(C_0 \cup \Bbraces{p})
        =
        \underbrace{\hat b(C_0)}_0
        \lor
        \underbrace{b(p)}_0
        =
        0.
    \end{align*}

    Also ist $\hat b(C_0) = 1$.
    Damit ist auch $\hat b(C) = 1$.

    \item Fall ($b(p) = 1$):
    
    Seien

    \begin{align*}
        C_1 := C^\prime \cup \Bbraces{\neg p},
        \quad
        C^\prime \subseteq C,
        \quad
        |C^\prime| = 1, 2.
    \end{align*}

    Dann gilt

    \begin{align*}
        1
        \geq
        \hat b(C)
        \geq
        \hat b(C^\prime)
        =
        \hat b(C^\prime)
        \lor \underbrace{\hat b(\neg p)}_0
        =
        \hat b(C_1)
        =
        1.
    \end{align*}

\end{enumerate}

Insgesamt bekommen wir also $\hat b(\mathcal C) = 1$. \\

\enquote{$\implies$}:
Sei nun $\mathcal C$ erfüllbar.
Dann gibt es eine Belegung $b: \operatorname L(\mathcal C) \to \Bbraces{0, 1}$, sodass $\hat b(\mathcal C) = 1$.
Insbesondere gilt also

\begin{align*}
    \hat b(\mathcal C \setminus \Bbraces{C}) = 1
    \quad
    \text{und}
    \quad
    \hat b(C) = 1.
\end{align*}

\begin{enumerate}[label = \arabic*.]

    \item Fall ($\hat b(C_0) = 0$):
    
    Damit $\hat b(\mathcal C^\prime) = 1$, muss

    \begin{align*}
        1
        \stackrel{!}{=}
        \hat b(C_0 \cup \Bbraces{p})
        =
        \underbrace{\hat b(C_0)}_0
        \lor
        \hat b(p)
        =
        b(p).
    \end{align*}

    Außerdem gilt $\hat b(\Bbraces{L_1, L_2}) = 1$, sonst wäre

    \begin{align*}
        1
        =
        \hat b(C)
        =
        \underbrace{\hat b(C_0)}_0
        \lor
        \underbrace{\hat b(\Bbraces{L_1, L_2})}_0
        =
        0.
    \end{align*}

    Damit $\hat b(C) = 1$, muss aber auch

    \begin{align*}
        1
        \stackrel{!}{=}
        \hat b(C_1)
        =
        \hat b(C^\prime)
        \lor
        \underbrace{\hat b(\neg p)}_0
        =
        \hat b(C^\prime).
    \end{align*}

    Wir wählen also $C^\prime := \Bbraces{L_1, L_2}$.

    \item Fall ($\hat b(C_0) = 1$):
    
    Wir können also getrost $b(p) := 0$ wählen und trotzdem gilt

    \begin{align*}
        \hat b(C_0 \cup \Bbraces{p})
        =
        \underbrace{\hat b(C_0)}_1
        \lor
        \underbrace{\hat b(p)}_0
        =
        1.
    \end{align*}

    Außerdem gilt

    \begin{align*}
        1
        \geq
        \hat b(C_1)
        =
        \hat b(C^\prime)
        \lor
        \underbrace{\hat b(\neg p)}_1
        \geq
        1.
    \end{align*}

\end{enumerate}

Insgesamt bekommen wir also

\begin{align*}
    1
    \geq
    \hat b(\mathcal C^\prime)
    =
    \underbrace{\hat b(\mathcal C \setminus \Bbraces{C})}_1
    \land
    \underbrace{\hat b(C_0 \cup \Bbraces{p})}_1
    \land
    \underbrace{\hat b(C_1)}_1
    =
    1.
\end{align*}

Fun Fact:
Für alle Belegungen $b: \operatorname L(\mathcal C^\prime) \to \Bbraces{0, 1}$ gilt

\begin{align*}
    \hat b(C_1)
    =
    \hat b(\Bbraces{L_1, L_2} \cup \Bbraces{p})
    =
    \hat b(\neg p \lor (L_1 \lor L_2))
    =
    \hat b(p \to (L_1 \lor L_2)),
\end{align*}

und

\begin{align*}
    \hat b(C_0 \cup \Bbraces{p})
    =
    \hat b(\Bbraces{L_3, \dots, L_m} \cup \Bbraces{p})
    =
    \hat b(\neg (\neg p) \lor (L_3 \lor \dots \lor L_m))
    =
    \hat b(\neg p \to (L_3 \lor \dots \lor L_m)).
\end{align*}

Nun können wir das soeben Gezeigte auf alle (endlichen) Klauseln $C$, einer beliebigen (endlichen) Klauselmenge $\mathcal C$, mit $|C| > 3$, anwenden.
In polynomieller Laufzeit können wir für $\mathcal C$ also eine erfüllbarkeitsäquivalente Klauselmenge $\mathcal C^\prime$ finden, sodass für alle $C \in \mathcal C^\prime$ gilt $|C| \leq 3$.

Nun ist

\begin{itemize}
    \item laut Satz 4.4, $\textsc{Sat} =: L$ $\mathbf{NP}$-vollständig,
    \item laut Satz 4.2, $L \in \mathbf{NP}$ und damit auch $L^\prime := \textsc{3Sat} \in \mathbf{NP}$, und
    \item laut dem soeben Gezeigten, $L$ auf $L^\prime$ polynomiell reduzierbar, i.Z. $L \leq_\mathrm{p} L^\prime$.
\end{itemize}

Laut Lemma 4.2, ist also auch $L^\prime$ $\mathbf{NP}$-vollständig.

\end{solution}

% -------------------------------------------------------------------------------- %
