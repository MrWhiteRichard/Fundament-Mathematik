% --------------------------------------------------------------------------------

\begin{exercise}

Wir sagen dass eine deterministische Turingmaschine $M = \abraces{Q, \delta, q_0}$ \textit{primitiv rekursive Laufzeit} hat galls eine primitiv rekursiv FUnktion $t: \N \to \N$ existiert so dass für alle $x \in \N$ eine Konfiguration $(\mathrm{fertig}, u, v)$ existiert sowie ein $k \leq t(x)$ so dass $(q_0, \triangleright, x) \xrightarrow{M^k} (\mathrm{fertig}, u, v)$.

Zeigen Sie dass eine Funktion $f: \N \to \N$ primitiv rekursiv ist genau dann wenn eine Turingmaschine existiert die $f$ in primitiv rekursiv Laufzeit berechnet.

\begin{itshape}

    Hinweis:
    Stützen Sie sich auf die Beweise  der Äquivalenz  der Begriffe Turing-berechenbar und partiell rekursiv.
    Sie dürfen die Aussage verwenden dass für alle primitiv rekursiven $g: \N^{k+1} \to \Bbraces{0, 1}$ auch die Funktion

    \begin{align*}
        (\overline x, z)
        \mapsto
        (\mu y \leq z) g(\overline x, y)
        =
        \begin{cases}
            \text{das kleinste $y \leq z$ so dass $g(\overline x, y) = 1$}
            & \text{falls so ein $y$ existiert} \\
            0
            & \text{sonst}
        \end{cases}
    \end{align*}

    primitiv rekursiv ist.

\end{itshape}

\end{exercise}

% --------------------------------------------------------------------------------

\begin{solution}

\phantom{}

\end{solution}

% --------------------------------------------------------------------------------
