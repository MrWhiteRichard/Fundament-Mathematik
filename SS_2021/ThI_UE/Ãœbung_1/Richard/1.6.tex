% --------------------------------------------------------------------------------

\begin{exercise}

Seien $A, B$ Alphabete und $\varphi: A^\ast \to B^\ast$ ein Monoidhomomorphismus.
Zeigen Sie dass für alle regulären $L \subseteq A^\ast$ auch $\varphi(L) = \Bbraces{\varphi(w) \mid w \in L}$ regulär ist.

\end{exercise}

% --------------------------------------------------------------------------------

\begin{solution}

Sei $N = \abraces{Q, A, \Delta, q_0, F}$ ein NFA für $L = \operatorname L(N)$.
Betrachte den NFA $\varphi(N) := \abraces{\varphi(Q), \varphi(A), \varphi(\Delta), q_0, F}$.
$\varphi(Q)$ und $\varphi(\Delta)$ werden im Folgenden noch festgelegt.
Für jedes Tripel $(q_\delta, w_\delta, q_\delta^\prime) = \delta \in \Delta \subseteq Q \times A \times Q$ seien

\begin{align*}
    n_\delta := |\varphi(w_\delta)|,
    \quad
    \text{und}
    \quad
    v_{\delta, 1}, \dots, v_{\delta, n_\delta} \in B:
        v_{\delta, 1} \cdots v_{\delta, n_\delta}
        =
        v_\delta
        :=
        \varphi(w_\delta)
        \in
        \varphi(A) \subseteq B^\ast.
\end{align*}

Falls $n_\delta > 1$, was ja wegen $\varphi(A) \not \subseteq B$ passieren kann, ist $\varphi(w_\delta)$ kein Buchstabe $\in B$ mehr, sondern ein \Quote{echtes} Wort $\in B^\ast$.
Wir fügen neue Zustände $q_{\delta, 2}, \dots, q_{\delta, n_{\delta}}$, wobei $q_{\delta, 1} := q_\delta \in Q$ und $q_{\delta, n_\delta + 1} := q_\delta^\prime \in Q$, zu $Q$ hinzu.
Sei $\varphi(Q) \supseteq Q$ die Menge all jener (inklusive neuer) Zustände.
Die neuen Zustände sollen dabei alle unterschiedlich sein, d.h.

\begin{align*}
    \Forall \delta_1, \delta_2 \in \Delta:
        \pbraces
        {
            n_{\delta_1}, n_{\delta_2} > 1
            \implies
            \Bbraces{q_{\delta_1, 2}, \dots, q_{\delta_1, n_{\delta_1}}}
            \cap
            \Bbraces{q_{\delta_2, 2}, \dots, q_{\delta_2, n_{\delta_2}}}
            =
            \emptyset
        }.
\end{align*}

Damit kommen wir schließlich zur Definition von

\begin{align*}
    \varphi(\Delta)
    :=
    \bigcup_{\delta \in \Delta}
        \Bbraces
        {
            (q_{\delta, 1}, v_{\delta, 1}, q_{\delta, 2}),
            \dots,
            (q_{\delta, n_\delta}, v_{\delta, 1}, q_{\delta, n_\delta + 1})
        }.
\end{align*}

Wir visualisieren einen Teil des Automaten $N$ und den zugehörigen Teil in $\varphi(N)$.
Man beachte, dass $q_{\delta, 1} = q_\delta \in Q$ und $q_{\delta, n_\delta + 1} = q_\delta^\prime \in Q$ (o.B.d.A. sind diese Zustände $\in Q \setminus F$) und $q_{\delta, 2}, \dots, q_{\delta, n_\delta} \in \varphi(Q) \setminus Q$, sowie $v_{\delta, 1} \cdots v_{\delta, n_\delta} = v_\delta = \varphi(w_\delta)$.

\begin{center}
    \begin{tikzpicture}[
        ->,
        node distance = 2cm,
        initial text = $ $,
        initial distance = 1.5cm,
        initial above
    ]

        \node [state,               draw = white] (1) {$\cdots$};
        \node [state, right of = 1]               (2) {$q_\delta$};
        \node [state, right of = 2]               (3) {$q_\delta^\prime$};
        \node [state, right of = 3, draw = white] (4) {$\cdots$};

        \draw (2) edge node [above] {$w_\delta$} (3);
    
    \end{tikzpicture}

    \vspace{1cm}

    $\stackrel{\varphi}{\mapsto}$

    \vspace{1cm}

    \begin{tikzpicture}[
        ->,
        node distance = 2cm,
        initial text = $ $,
        initial distance = 1.5cm,
        initial above
    ]

        \node [state,               draw = white] (1) {$\cdots$};
        \node [state, right of = 1]               (2) {$q_{\delta, 1}$};
        \node [state, right of = 2]               (3) {$q_{\delta, 2}$};
        \node [state, right of = 3, draw = white] (4) {$\cdots$};
        \node [state, right of = 4]               (5) {$q_{\delta, n_\delta}$};
        \node [state, right of = 5]               (6) {$q_{\delta, n_\delta + 1}$};
        \node [state, right of = 6, draw = white] (7) {$\cdots$};

        \draw (2) edge node [above] {$v_{\delta, 1}$}            (3);
        \draw (3) edge node [above] {$v_{\delta, 2}$}            (4);
        \draw (4) edge node [above] {$v_{\delta, n_\delta - 1}$} (5);
        \draw (5) edge node [above] {$v_{\delta, n_\delta}$}     (6);

        \draw (2) edge [bend right] node [below] {$v_\delta$} (6);
    
    \end{tikzpicture}
\end{center}

Wir behaupten, dass $\varphi(L) = \varphi(\operatorname L(N)) = \operatorname L(\varphi(N))$.
Aus der Definition 1.7 der erweiterten Übergangsrelation (hier als $\Delta^\ast$), dem oberen Bild (Smiley), und der Tatsache, dass

\begin{align*}
    \varphi(\epsilon) = \epsilon,
    \quad
    \Forall u, v \in A^\ast:
        \varphi(u v) = \varphi(u) \varphi(v),
\end{align*}

sieht man unmittelbar, dass

\begin{align*}
    \varphi(Q) \times \varphi(A^\ast) \times \varphi(Q)
    \supseteq
    \varphi(\Delta)^\ast
    \supseteq
    \varphi(\Delta^\ast)
    :=
    \Bbraces{(q_1, \varphi(w), q_2): (q_1, w, q_2) \in \Delta^\ast}.
\end{align*}

Sei $v \in \varphi(\operatorname L(N))$ und $w \in \operatorname L(N)$ mit $\varphi(w) = v$.

\begin{align*}
    & \implies
    \Exists q \in F: (q_0, w, q) \in \Delta^\ast \\
    & \implies
    (q_0, v, q) = (q_0, \varphi(w), q) \in \varphi(\Delta^\ast) \subseteq \varphi(\Delta)^\ast \\
    & \implies
    v \in \operatorname L(\varphi(N))
\end{align*}

Sei nun $v \in \operatorname L(\varphi(N))$ und $q \in F$ sodass $(q_0, v, q) \in \varphi(\Delta)^\ast$.
Seien $q_1, \dots, q_n$, mit $q_n = q \in F$, Zustände $\in Q$, sodass

\begin{align*}
    \Exists v_1, \dots, v_n \in \varphi(A) \subseteq B^\ast:
        (q_0, v_1, q_1), \dots, (q_{n-1}, v_n, q_n)
        \in
        \varphi(\Delta^\ast),
        \quad
        v_1 \cdots v_n = v.
\end{align*}

Seien $w_1, \dots, w_n \in A$, sodass

\begin{align*}
    \varphi(w_1) = v_1, \dots, \varphi(w_n) = v_n,
    \quad
    (q_0, w_1, q_1), \dots, (q_{n-1}, w_n, q_n) \in \Delta \subseteq Q \times A \times Q.
\end{align*}

Weil $q_n = q \in F$ und $(q_0, w, q) \in \Delta^\ast$ erhalten wir, dass $w := w_1 \cdots w_n \in \operatorname L(N)$.

\begin{align*}
    \implies
    v
    =
    v_1 \cdots v_n
    =
    \varphi(w_1) \cdots \varphi(w_n)
    =
    \varphi(w_1 \cdots w_n)
    =
    \varphi(w)
    \in
    \varphi(\operatorname L(N))
\end{align*}

\end{solution}

% --------------------------------------------------------------------------------
