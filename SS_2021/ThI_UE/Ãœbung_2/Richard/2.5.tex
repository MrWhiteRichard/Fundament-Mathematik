% --------------------------------------------------------------------------------

\begin{exercise}

Sei $G = \abraces{\Bbraces{S}, \Bbraces{a, b, c, +, \cdot}, P, S}$ wobei $P =$

\begin{align*}
    S \to S + S \mid S \cdot S \mid a \mid b \mid c.
\end{align*}

Zeigen Sie dass $G$ mehrdeutig ist.
Ist $\operatorname L(G)$ inhärent mehrdeutig?

\end{exercise}

% --------------------------------------------------------------------------------

\begin{solution}

Betrachte die beiden Ableitungsbäume für $a \cdot b + c$.

\begin{center}
    \begin{tikzpicture}
        \node {$S$}
            child
            {
                node {$S$}
                child
                {
                    node {$S$}
                    child
                    {
                        node {$a$}
                    }
                }
                child
                {
                    node {$\cdot$}
                }
                child
                {
                    node {$S$}
                    child
                    {
                        node {$b$}
                    }
                }
            }
            child
            {
                node {$+$}
            }
            child
            {
                node {$S$}
                child
                {
                    node {$c$}
                }
            };
    \end{tikzpicture}
\end{center}

\begin{center}
    \begin{tikzpicture}
        \node {$S$}
            child
            {
                node {$S$}
                child
                {
                    node {$a$}
                }
            }
            child
            {
                node {$\cdot$}
            }
            child
            {
                node {$S$}
                child
                {
                    node {$S$}
                    child
                    {
                        node {$b$}
                    }
                }
                child
                {
                    node {$+$}
                }
                child
                {
                    node {$S$}
                    child
                    {
                        node {$c$}
                    }
                }
            };
    \end{tikzpicture}
\end{center}

Offenbar gilt

\begin{align*}
    \operatorname L(G)
    =
    L
    :=
    \Bbraces
    {
        \pbraces
        {
            \prod_{i=1}^n
                x_i y_i
        }
        x_{n+1}
        \mid
        n \in \N, \,
        x_1, \dots, x_{n+1} \in \Bbraces{a, b, c}, \,
        y_1, \dots, y_n \in \Bbraces{+, \cdot}
    }.
\end{align*}

Folgender NFA zeigt, dass $L$ regulär ist.

\begin{center}
    \begin{tikzpicture}[
        ->,
        node distance = 2cm,
        initial text = $ $,
        initial distance = 1.5cm,
        initial above
    ]
    
        \node [state, initial] (1) {};
        \node [state, accepting, right of = 1] (2) {};
        \node [state, right of = 2] (3) {};
        
        \draw
        (1) edge             node [above] {$a, b, c$}  (2)
        (2) edge [bend left] node [above] {$+, \cdot$} (3)
        (3) edge [bend left] node [below] {$a, b, c$}  (2);
    
    \end{tikzpicture}
\end{center}

Laut Aufgabe 6, kann $L$ daher auch nicht inhärent mehrdeutig sein.

\end{solution}

% --------------------------------------------------------------------------------
