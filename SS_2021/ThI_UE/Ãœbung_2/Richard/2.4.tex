% --------------------------------------------------------------------------------

\begin{exercise}

Zeigen Sie dass die regulären Sprachen strikt in den linearen Sprachen enthalten sind und die linearen Sprachens trikt in den kontextfreien. \\

\textit{Hinweis: $\Bbraces{a^i b^i c^j d^j \mid i, j \in \N}$.}

\end{exercise}

% --------------------------------------------------------------------------------

\begin{solution}

Die erste Behauptung folgt aus Satz 2.1 und Beispiel 2.5.

Betrachte nun die Sprache $L := \Bbraces{a^i b^i c^j d^j \mid i, j \in \N}$.
Diese ist, wegen Satz 2.2, als Quadrat der Sprache aus Beispiel 2.5, kontextfrei.
Angenommen, sie wäre linear.
Sei $n$ so wie im Satz (Schleifensatz (pumping lemma) für lineare Sprachen) der Aufgabe 3.

Das Wort $a^n b^n c^n d^n \in L$ lässt sich nun leicht aus $L$ \blockquote{rauspumpen}.
Wegen der 1. und 3. Bedingung, kann man überhaupt \blockquote{pumpen};
und wegen der 2. passiert das nur lokal in einer, oder zwei angrenzenden Sequenzen desselben Symbols.

Wir haben dabei übrigens auch gezeigt, dass das Produkt zweier linearen Sprachen im Allgemeinen nicht linear ist.

\end{solution}

% --------------------------------------------------------------------------------
