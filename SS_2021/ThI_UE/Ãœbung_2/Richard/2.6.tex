% -------------------------------------------------------------------------------- %

\begin{exercise}

Zeigen Sie dass keine inhärent mehrdeutige reguläre Sprache existiert. \\

\textit{Hinweis: Transformieren Sie einen DFA in eine Grammatik.}

\end{exercise}

% -------------------------------------------------------------------------------- %

\begin{solution}

Sei $L \subseteq A^\ast$ eine reguläre Sprache und $D = \abraces{Q, A, \delta, q_0, F}$ ein DFA für $L$, d.h. $\operatorname L(D) = L$.
Definiere die Grammatik $G = \abraces{N, T, P, S}$ durch $N = Q$, $T = A$,

\begin{align*}
    P = \Bbraces{q \to a p \mid \delta(q, a) = p} \cup \Bbraces{q \to \varepsilon \mid q \in F},
\end{align*}

und $S = q_0$.
Dann behaupten wir, dass $\operatorname L(G) = \operatorname L(D) = L$.
Das wurde im Beweis von Satz 2.1 bereits getan (für $D$ als NFA).
(Darüber hinaus, wurde gezeigt, dass $G$ strikt rechtslinear ist.)
Wir zeigen, dass $G$ nun auch eindeutig ist, d.h. für alle $w \in L$ genau ein Ableitungsbaum existiert.

Sei $w = w_1 \cdots w_n$ mit $w_1, \dots, w_n \in A$ und

\begin{align*}
    \delta(q_0, w_1) =: q_1, \,
    \delta(q_1, w_2) =: q_2, \,
    \dots, \,
    \delta(q_{n-1}, w_n) =: q_n.
\end{align*}

Weil $D$ ein DFA ist, ist $\delta$ eine Funktion, und $q_1, \dots, q_{n-1} \in Q$, $q_n \in F$ damit eindeutig.
Laut Definition von $P$ gilt nun

\begin{align*}
    (q_0 \to w_1 q_1),
    (q_1 \to w_2 q_2),
    \dots,
    (q_{n-1} \to w_n q_n),
    (q_n \to \varepsilon)
    \in
    P.
\end{align*}

Das können wir auch in Form einer Ableitung von $w$ schreiben.

\begin{align*}
    S
    \Longrightarrow_G
    w_1 q_1
    \Longrightarrow_G
    w_1 w_2 q_2
    \Longrightarrow_G
    \cdots
    \Longrightarrow_G
    w_1 \cdots w_n q_n
    \Longrightarrow_G
    w_1 \cdots w_n = w
\end{align*}

Man sieht nun relativ leicht, dass diese Ableitung durch die Struktur von $D$ und den Aufbau von $w$ prädestiniert ist.

\begin{enumerate}[label = \arabic*., start = 0]

    \item Sie muss jedenfalls, an der $0$-ten Stelle, mit $q_0 = S$ anfangen.

    \item Wenn $w = \varepsilon$, dann steht an der $1$-ten Stelle $\varepsilon$ und wir sind fertig.
    Weil $G$ strikt rechtslinear ist, steht sonst an der $1$-ten Stelle auf der linken Seite $w_1$.
    Laut der Definition von $P$, muss auf der rechten Seite $q_1 := \delta(q_0, w_1)$ stehen.
    
    \item Wenn $w = w_1$, dann steht an der $2$-ten Stelle $w_1$ und wir sind fertig.
    Weil $G$ strikt rechtslinear ist, steht sonst an der $2$-ten Stelle auf der linken Seite $w_1 w_2$.
    Laut der Definition von $P$, muss auf der rechten Seite $q_2 := \delta(q_1, w_2)$ stehen.

    \item [\dots]

\end{enumerate}

Dieser Algorithmus terminiert, weil $n < \infty$.
Er liefert eine eindeutige Ableitung von $w$.

\end{solution}

% -------------------------------------------------------------------------------- %
