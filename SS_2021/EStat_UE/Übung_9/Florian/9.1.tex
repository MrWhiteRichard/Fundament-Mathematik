% -------------------------------------------------------------------------------- %

\begin{exercise}[\textbf{The GLRT for the normal variance - simple hypotheses}]

Derive the generalized likelihood ratio test (GLRT) for the normal variance:
Assume $X_1,\dots,X_n$ are i.i.d. $\mathcal{N}(\mu,\sigma^2)$, where both $\mu$ and $\sigma$
are unknown. We want to test

\begin{align*}
  H_0: \sigma^2 = \sigma_0^2 \quad \text{vs.} \quad H_1: \sigma^2 \neq \sigma_0^2.
\end{align*}

\end{exercise}

% -------------------------------------------------------------------------------- %

\begin{solution}

  \phantom{}

\end{solution}

% -------------------------------------------------------------------------------- %
