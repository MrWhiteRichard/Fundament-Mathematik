% -------------------------------------------------------------------------------- %

\begin{exercise}[Cramér-Rao lower bound - Simulation]

In Homework 7 Exercise 2 a density $f(x \mid \theta) = \theta x^{\theta - 1}$ for $0 < x < 1$ and $\theta > 0$ was given.
The goal was to find a sutable function $g$ of the parameter $\theta$ such taht there exists an unbiased estimator of $g(\theta)$ which attains the Cramér-Rao lower bound.

A unbiased statistic which attains the Cramßer-Rao lower bound is for $g(\theta) = \frac{1}{\theta}$ given by

\begin{align*}
    S_n(X_1, \dots, X_n)
    =
    -\frac{1}{n}
    \sum_{i=1}^n
        \ln(X_i).
\end{align*}

Implement the following steps in \texttt R:

\begin{enumerate}

    \item Write pdf \texttt{dhw}, cdf \texttt{phw}, quantile \texttt{qhw} and random sampling function \texttt{rhw} for the above distribution parameterized by $\theta$ (see for example \texttt{?runif}, \texttt{?rnorm}).

    \begin{itshape}
        Hint:
        Given an strict monotone continuous cdf $F$, then $F^{-1}(U)$ is distributed with cdf $F$ for $U \sim \mathit U(0, 1)$.
    \end{itshape}

    \item Fix an arbitrary $\theta$ and perform a simulation with growing sample size $n = 500, 1000, 1500, \dots, 10000$ each  with $100$ replications for the estimation of $g(\theta)$ with the statistic $S_n$.

    \item Create a scatter plot of all the estimates over teh sample size, add the sample mean and standard deviation aggregated over the sample size to the plot.
    Finally, add the theoretical mean an standard deviation of the statistic $S_n$.

\end{enumerate}

\end{exercise}

% -------------------------------------------------------------------------------- %

\begin{solution}

\phantom{}

\begin{enumerate}[label = (\alph*)]

    \item For $x \in \R$, we have

    \begin{align*}
        f_\theta(x \mid \theta)
        =
        \begin{cases}
            0,                     & x \leq 0,  \\
            \theta x^{\theta - 1}, & 0 < x < 1, \\
            0,                     & 1 \leq x,  \\
        \end{cases}
    \end{align*}

    and

    \begin{align*}
        F_\theta(x \mid \theta)
        =
        \inf f(x \mid \theta)
        =
        \begin{cases}
            0,        & x \leq 0,  \\
            x^\theta, & 0 < x < 1, \\
            1,        & 1 \leq  x.   \\
        \end{cases}
    \end{align*}
    
    For $x, p \in \R$, we have

    \begin{align*}
        x^\theta = F_\theta(x) = p
        \iff
        x = p^\frac{1}{\theta} = F_\theta^{-1}(p).
    \end{align*}

    \item Recall, from Homework 7, Exercise 2, that
    
    \begin{align*}
        \E S_n = \frac{1}{\theta},
        \quad
        \Var S_n = \frac{1}{n \theta^2}.
    \end{align*}

\end{enumerate}

\lstinputlisting{8.1.R}

\end{solution}

% -------------------------------------------------------------------------------- %
