% -------------------------------------------------------------------------------- %

\begin{exercise}[\textbf{Cramér-Rao lower bound - Simulation}]

In Homework 7 Exercise 2 a density $f(x|\theta) = \theta x^{\theta - 1}$
for $0 < x < 1$ and $\theta > 0$ was given. The goal was to find a suitable
function $g$ of the parameter $\theta$ such that there exists an unbiased
estimator of $g(\theta)$ which attains the Cramér-Rao lower bound.

An unbiased statistic which attains the Cramér-Rao lower bound is for
$g(\theta) = 1/\theta$ given by

\begin{align*}
  S_n(X_1,\dots,X_n) = - \frac{1}{n}\sum_{i=1}^n \ln(X_i).
\end{align*}

Implement the following steps in \texttt{R}:

\begin{enumerate}[label = (\alph*)]
  \item Write pdf \texttt{dhw}, cdf \texttt{phw}, quantile \texttt{qhw}
  and random sampling function \texttt{rhw} for the above distribution
  parametrized by $\theta$ (see for example \texttt{?runif}, \texttt{?rnorm}).

  \textit{Hint:} Given an strict monotone continuous cdf $F$, then $F^{-1}(U)$
  is distributed with cdf $F$ for $U \sim U(0,1)$.

  \item Fix an arbitrary $\theta$ and perform a simulation with growing
  sample size $n = 500, 1000, 1500, \dots, 10000$ each with 100 replications
  for the estimation of $g(\theta)$ with the statistic $S_n$.
  
  \item Create a scatter plot of all the estimates over the sample size,
  add the sample mean and standard deviation aggregated over
  the sample size to the plot. Finally, add the theoretical mean and
  standard deviation of the statistic $S_n$.
\end{enumerate}

\end{exercise}

% -------------------------------------------------------------------------------- %

\begin{solution}

  \phantom{}

  \begin{enumerate}[label = (\alph*)]
    \item
    Our pdf reads $f(x|\theta) = \theta x^{\theta - 1}$. Integrating over x yields the pdf
    \begin{align*}
      F(x|\theta) &= \int_0^x f(y|\theta) dy = \theta \int_0^x y^{\theta - 1} dy \\
      &=  [y^{\theta}]_0^{x} = x^{\theta}.
    \end{align*}
    Next we need to calculate the quantile function

    \begin{align*}
      p \stackrel{!}{=} F(x|\theta) \iff
      p = x^{\theta} \iff
      p^{1/\theta} = x.
    \end{align*}

    Therefore $q(p|\theta) = F^{-1}(p|\theta) = p^{1/\theta}$.

    Let $X$ be uniformly distributed on the interval $(0,1)$. Then

    \begin{align*}
      f_{X^{1/\theta}} = f_X(x^{\theta})|\theta x^{\theta - 1}| = f(x|\theta).
    \end{align*}

    Therefore we can transform random samples from a uniform $(0,1)$ distribution
    with $g(x) = x^{1/\theta}$ to obtain random samples for our custom distribution.
  \end{enumerate}

\end{solution}

% -------------------------------------------------------------------------------- %
