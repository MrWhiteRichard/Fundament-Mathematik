% -------------------------------------------------------------------------------- %

\begin{exercise}[\textbf{Comparing groups}]

    Health professionals warn that transmission of infectious diseases may occur during the
    traditional handshake greeting. Two alternative methods of greeting (popularized in sports)
    are the high five and the first bump. Researchers compared the hygiene of these alternative
    greetings in a designed study and reported the results in the American Journal of Infection
    Control (Aug. 2014). A sterile-gloved hand was dipped into a culture of bacteria, then made
    contact for three seconds with another sterile-gloved hand via either a handshake, high five,
    or fist bump. The researchers then counted the number of bacteria present on the second,
    recipient, gloved hand. This experiment was replicated five times for each contact method.
    Simulated data (recorded as a percentage relative to the mean of the handshake), based on
    information provided by the journal article, are provided in the table.

    \begin{center}
        \begin{tabular}{c c c c c c}
            Handshake: & 131 & 74 & 129 & 96 & 92 \\
            High five: & 44 & 70 & 69 & 43 & 53 \\
            Fist bump: & 15 & 14 & 21 & 29 & 21
        \end{tabular}
    \end{center}


    \begin{enumerate}[label = (\alph*)]
        \item The researchers reported that more bacteria were transferred
        during a handshake compared with a high five. Use a 95\% confidence interval to support this statement statistically.
        \item The researchers also reported that the fist bump gave a lower transmission
        of bacteria than the high five. Use a 95\% confidence interval to support this statement statistically.
        \item Based on the results, parts (a) and (b), which greeting method would you
        recommend as being the most hygienic?
    \end{enumerate}

\end{exercise}

% -------------------------------------------------------------------------------- %

\begin{solution}

\phantom{}

\begin{enumerate}[label = (\alph*)]
    \item
    \item
    \item None of the above.
\end{enumerate}

\end{solution}

% -------------------------------------------------------------------------------- %
