% -------------------------------------------------------------------------------- %

\begin{exercise}[Boxplot and quantiles]

Two novel randomized algorithms ($A$ and $B$) are to be compared regarding their runtimes.
Both algorithms were executed $n$ times.
The runtimes (in seconds) are stored in the file \texttt{algorithms.Rdata}

\begin{enumerate}[label = (\alph*)]

    \item Set the working directory and load the data using \texttt{load()}.
    Create a boxplot to compare the running times.
    Color the boxes and add proper notations (axes notations, title etc.).
    More info via \texttt{?boxplot}

    \item Comment on the following statements / questions only using the graphic
    
    \begin{enumerate}[label = (\alph*)]
        \item The first quartile of the times in $A$ was about?
        \item the interquartile range of the times in $B$ is about trice the interquartile range of $A$
        \item Is $n = 100$?
        \item More than half of the running times in $B$ were faster than $3/4$ of the running times in $A$.
        \item At least $50 \%$ in $A$ were faster than the $25 \%$ slowest in $B$.
        \item At least $60 \%$ in $A$ were faster than the $25 \%$ slowest in $B$.
    \end{enumerate}

    \item Regarding the runtimes
    
    \begin{align*}
        23.7, 13.7, 7.6, 9.0, 44.3, 3.5, 2.2, 34.2
    \end{align*}

    which are a subset of $B$, find all empirical (a) medians, (b) first quartiles and (c) $2/3$-qua \underline n tiles (not using \texttt R).

\end{enumerate}

\end{exercise}

% -------------------------------------------------------------------------------- %

\begin{solution}

\phantom{}

\end{solution}

% -------------------------------------------------------------------------------- %
