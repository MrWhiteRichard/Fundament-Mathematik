% --------------------------------------------------------------------------------

\begin{exercise}[\textbf{Basketball free throws}]

Two professional basketball players, Tom and John, each throw ten free throws with
a basketball. Tom makes 80\% of the free throws he tries, while John makes 85\%
of the free throws he tries.

\begin{enumerate}[label = (\alph*)]
  \item What is the probability that the number of free throws that Tom will make
  is exactly 7?
  \item What is the probability that the number of free throws that John will make
  is at least 8?
  \item Player who achieves the highest score wins the game. It is assumed that
  the two players do not influence each other when throwing. What is the probability
  that neither Tom or John will win the game?
\end{enumerate}
\textit{Hint:} Use \texttt{R}-function \texttt{dbinom()} to calculate the probability
mass functions.

\end{exercise}

% --------------------------------------------------------------------------------

\begin{solution}

\phantom{}

\begin{enumerate}[label = (\alph*)]
  \item $\P(X_{\text{Tom}} = 7) = 105 \cdot 0.8^{7}\cdot0.2^{3}$.
  \item $\P(X_{\text{John}} \geq 8) =
  45\cdot0.85^{8}\cdot0.15^{2} + 10\cdot 0.85^{9}\cdot 0.15 + 0.85^{10}$.
  \item
  \begin{align*}
    \P(\text{draw}) = \sum_{i=0}^{10} \P(X_{\text{Tom}} = i) \cdot \P(X_{\text{John}} = i)
  \end{align*}
  The command \texttt{sum(dbinom(0:10, 10, 0.8)*dbinom(0:10, 10, 0.75))} in \texttt{R}
  yields a probability of $0.2063486$.
\end{enumerate}

\end{solution}

% --------------------------------------------------------------------------------
