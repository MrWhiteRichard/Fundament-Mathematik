% -------------------------------------------------------------------------------- %

\begin{exercise}[\textbf{Drug company}]

Manufacturing and selling drugs that claim to reduce an individual's cholesterol
level is big business. A company would like to market their drug to women if their
cholesterol is in the top 15\%. Assume the cholesterol levels of adult American
women can be described by a Normal model with a mean of 188mg/dL and a standard
deviation of 24 mg/dL.

\begin{enumerate}[label = (\alph*)]
  \item Use \texttt{R} to draw and label the Normal model.
  \item What percent of adult women do you expect to have cholesterol levels over
  200 mg/dL?
  \item What percent of adult women do you expect to have cholesterol levels between
  150 mg/dL and 170 mg/dL?
  \item Calculate the interquartile range of the cholesterol levels. Recall, the
  interquartile range is the difference between upper and lower quartile, i.e.

  \begin{align*}
    IQR = x_{0.75} - x_{0.25}.
  \end{align*}
  \item Above what value are the highest 15\% of women's cholesterol levels?
\end{enumerate}

\textit{Hint:} If using \texttt{R} for all computations the following commands
\texttt{pnorm(), qnorm()} and \texttt{dnorm()} are useful. Otherwise values from
Table of standard Normal distribution should be used.
\end{exercise}

% -------------------------------------------------------------------------------- %

\begin{solution}

\phantom{}

\end{solution}

% -------------------------------------------------------------------------------- %
