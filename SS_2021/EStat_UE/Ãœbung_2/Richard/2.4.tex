% --------------------------------------------------------------------------------

\begin{exercise}[Hurricane insurance]

An insurance company needs to asses the risk associated with providing hurricane insurance.
During $22$ years from $1990$ through $2011$, Florida was hit by $27$ major hurricanes (level $3$ and above).
The insurance company assumed a Poisson distribution for modeling the number of hurricanes.

\begin{enumerate}[label = (\alph*)]

    \item If hurricanes are independent and the mean has not changed, what is the probability of having a year in Florida with each of the following?

    \begin{enumerate}[label = (\arabic*)]
        \item No hits.
        \item Exactly one hit.
        \item More than two its.
    \end{enumerate}

    \item Use \texttt R to estimate the number of hurricane hits that will occur with the probability $99.5 \%$.

    \textit{Hint:}
    One of the following \texttt R-commands:
    \texttt{dpois()}, \texttt{ppois()}, \texttt{qpois()}, \texttt{rpois()} is applicable.

\end{enumerate}

\end{exercise}

% --------------------------------------------------------------------------------

\begin{solution}

\phantom{}

\begin{enumerate}[label = (\alph*)]

    \item Florida was hit by $\lambda := \frac{27}{22}$ hurricanes per year on average (from $1990$ through $2011$).
    Let $X$ denote teh number of hurricane hits per year.

    \begin{align}
        P(X = 0)
        =
        \frac{\lambda^0}{0!} e^{-\lambda}
        =
        e^{-\frac{27}{22}}
        \approx
        0.293090827 \\
        P(X = 1)
        =
        \frac{\lambda^1}{1!} e^{-\lambda}
        =
        \frac{27}{22} e^{-\frac{27}{22}}
        \approx
        0.359702379 \\
        \sum_{k=3}^\infty P(X = k)
        =
        1 - \sum_{k=0}^2 P(X = k)
        =
        1 - \pbraces
        {
            \frac{(27 / 22)^2}{2} + 27 / 22 + 1
        }
        e^{-\frac{27}{22}}
        \approx
        0.126480334
    \end{align}

    \item In the \Quote{help}-section of \texttt{qpois} it says:
    \Quote{The quantile is right continuous: \texttt{qpois(p, lambda)} is the smallest integer $x$ such that $P(X \leq x) \geq p$.}

    The following code yields $5$.

    \lstset{style = fundament}
    \lstinputlisting{2.4.r}

\end{enumerate}

\end{solution}

% --------------------------------------------------------------------------------
