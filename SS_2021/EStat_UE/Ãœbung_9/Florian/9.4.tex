% -------------------------------------------------------------------------------- %

\begin{exercise}[\textbf{Most powerful test for the normal variance - $\mu$ is known}]

  Let $X_1,\dots,X_n$ be i.i.d. $\mathcal{N}(\mu,\sigma^2)$, where $\mu$ is known.
  
  
  \begin{enumerate}[label = (\alph*)]
    \item Find an MP test at level $\alpha$ for testing two simple hypotheses
    
    \begin{align*}
      H_0: \sigma^2 = \sigma_0^2 \quad \text{vs.} \quad H_1: \sigma^2 = \sigma_1^2, \sigma_1 > \sigma_0.
    \end{align*}

    \item Show that the MP test is a UMP test for testing

    \begin{align*}
      H_0: \sigma^2 \leq \sigma_0^2 \quad \text{vs.} \quad H_1: \sigma^2 > \sigma_0^2.
    \end{align*}
 
    \textit{Hint:} $\sum_i (X_i - \mu)^2 \sim \sigma^2\chi^2(n)$.
  \end{enumerate}
  
  \end{exercise}
  
  % -------------------------------------------------------------------------------- %
  
  \begin{solution}
  
  \phantom{}

  \begin{enumerate}[label = (\alph*)]
    \item The likelihood ratio reads

    \begin{align*}
      \lambda(\textbf{x}) &
      = \frac{L(\sigma_1^2, \textbf{x})}{L(\sigma_0^2, \textbf{x})} \\
      &= \frac{\sigma_0^n}{\sigma_1^n}
      \exp\left(-\left( \frac{1}{2\sigma_1^2} - \frac{1}{2\sigma_0^2}\right) \sum_{i=1}^n (x_i - \mu)^2\right).
    \end{align*}

    Since $\lambda(\textbf{x})$ is a non-decreasing function of
    $T(\textbf{x}) = \sum_{i=1}^n (x_i - \mu)^2$ our test rejects $H_0$ if

    \begin{align*}
      T(\textbf{x}) \geq C,
    \end{align*}

    where $\alpha = \P_{\sigma_0^2}(T(\textbf{X}) \geq C)$.

    Since $(X_i - \mu) \sim \mathcal{N}(0,\sigma^2)$ we have
    $(X_i - \mu)^2 \sim \sigma^2 \chi^2(1)$ and 
    $\sum_{i=1}^n (X_i - \mu)^2 \sim \sigma^2 \chi^2(n)$.

    Hence our critical value reads $C = \frac{1}{\sigma^2}\chi_{\alpha}^2(n)$.

    \item Since $\lambda(x)$ is monotonously non-decreasing in $T(\textbf{x})$ 
    for $\sigma_1^2 > \sigma_0^2$ the theorem from the lecture proves that 
    above test is indeed an UMP for testing this hypothesis.
  \end{enumerate}
  
  \end{solution}
  
  % -------------------------------------------------------------------------------- %
  