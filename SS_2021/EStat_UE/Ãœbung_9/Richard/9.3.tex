% -------------------------------------------------------------------------------- %

\begin{exercise}[Most powerful test 2]

Let $X_1, \dots, X_n$ be iid from a distribution with density

\begin{align*}
    f_\theta(x) = \frac{x}{\theta} e^{-\frac{x^2}{2 \theta}},
    x \geq 0,
    \theta > 0.
\end{align*}

\begin{enumerate}[label = (\alph*)]

    \item Derive the MP test at level $\alpha$ for testing two simple hypoheses
    
    \begin{align*}
        H_0: \theta = \theta_0
        \quad
        \textit{vs}
        \quad
        H_1: \theta = \theta_1, \theta_1 > \theta_0.
    \end{align*}

    \item Is there a uniformly most powerful (UMP) test at level $\alpha$ for testing the one-sided composite hypothesis
    
    \begin{align*}
        H_0: \theta = \theta_0
        \quad
        \textit{vs}
        \quad
        H_1: \theta > \theta_0
    \end{align*}

    What is its power function?

    \textit{Hint:}
    Show $X_i^2 \sim \exp(1 / 2 \theta)$, so that $\sum_i X_i^2 \sim \theta_{\chi^2}(2 n)$.

\end{enumerate}

\end{exercise}

% -------------------------------------------------------------------------------- %

\begin{solution}

First, we show the hint.

\begin{align*}
    F_{X_i^2}(x)
    & =
    P(X_i^2 \leq x) \\
    & =
    P(-\sqrt x \leq X_i \leq \sqrt x) \\
    & =
    \int_{-\sqrt x}^{\sqrt x}
        f_{X_i}(\xi)
        ~ \mathrm d \xi \\
    & =
    \int_{-\sqrt x}^{\sqrt x}
        \frac{\xi}{2 \theta}
        e^{-\frac{\xi^2}{2 \theta}}
        ~ \mathrm d \xi \\
    & =
    \frac{1}{2}
    \pbraces
    {
        \int_0^{\sqrt x}
            \frac{\xi}{\theta}
            e^{-\frac{\xi^2}{2 \theta}}
            ~ \mathrm d \xi
        -
        \int_0^{\sqrt{-x}}
            \frac{\xi}{\theta}
            e^{-\frac{\xi^2}{2 \theta}}
            ~ \mathrm d \xi
    } \\
    & \stackrel{!}{=}
    \frac{2}{2}
    \int_0^x
        \frac{\xi}{\theta}
        e^{-\frac{\eta}{2 \theta}}
        \frac{1}{2 \xi}
        ~ \mathrm d \eta \\
    & =
    \int_0^x
        \frac{1}{2 \theta}
        e^{-\frac{\eta}{2 \theta}}
        ~ \mathrm d \eta \\
    & \sim
    \exp \pbraces{\frac{1}{2 \theta}}
\end{align*}

For \enquote ! we used

\begin{align*}
    \eta = \xi^2
    \implies
    \frac{\mathrm d \eta}{\mathrm d \xi} = 2 \xi
    \implies
    \mathrm d \xi = \frac{1}{2 \xi} \mathrm d \eta.
\end{align*}

\begin{align*}
    T(\mathbf X)
    & =
    \sum_{i=1}^n
        X_i \\
    & \sim
    \sum_{i=1}^n
        \exp \pbraces{\frac{1}{2 \theta}} \\
    & \stackrel{!}{=}
    \operatorname{Gamma} \pbraces{n, \frac{1}{2 \theta}} \\
    & \sim
    \int_{-\infty}^x
        \frac
        {
            \pbraces{\frac{1}{2 \theta}}^n
        }{
            \Gamma(n)
        }
        \xi^{n-1}
        e^{-\frac{1}{2 \theta} \xi}
        ~ \mathrm d \xi \\
    & \stackrel{!!}{=}
    \int_{-\infty}^{x / \theta}
        \frac{1}{\theta^n}
        \frac{1 / 2^n}{\Gamma(n)}
        (\theta \eta)^{n-1}
        e^{-\frac{\eta}{\theta}}
        \theta
        ~ \mathrm d \eta \\
    & =
    \int_{-\infty}^{x / \theta}
        \frac{1 / 2^n}{\Gamma(n)}
        \eta^{n-1}
        e^{-\frac{\eta}{\theta}}
        ~ \mathrm d \eta \\
    & =
    F(T(\mathbf X) \leq x / \theta) \\
    & =
    F(\theta T(\mathbf X) \leq x) \\
    & \sim
    \theta \operatorname{Gamma} \pbraces{\frac{2 n}{2}, \frac{1}{2}} \\
    & \stackrel{!}{=}
    \theta \chi^2(2 n)
\end{align*}

For \enquote ! we used \cite[section 10, 11]{brand}.
For \enquote{!!}, we used

\begin{align*}
    \eta = \frac{\xi}{\theta}
    \implies
    \frac{\mathrm d \xi}{\mathrm d \eta} = \frac{1}{\theta}
    \implies
    \mathrm \eta = \theta ~ \mathrm d \xi.
\end{align*}

\begin{enumerate}[label = (\alph*)]

    \item We apply the Neyman-Pearson Lemma from \cite[lecture 10, slide 18]{EStat}.
    
    \begin{align*}
        \lambda(\mathbf x)
        & =
        \frac
        {
            L(\theta_1 \mid \mathbf x)
        }{
            L(\theta_0 \mid \mathbf x)
        } \\
        & =
        \frac
        {
            f_{\theta_1}(\mathbf x)
        }{
            f_{\theta_0}(\mathbf x)
        } \\
        & =
        \frac
        {
            \prod_{i=1}^n
                \frac{x_i}{\theta_1}
                e^{-\frac{x_i^2}{2 \theta_1}}
        }{
            \prod_{i=1}^n
                \frac{x_i}{\theta_0}
                e^{-\frac{x_i^2}{2 \theta_0}}
        } \\
        & =
        \pbraces{\frac{\theta_0}{\theta_1}}^n
        \exp \sum_{i=1}^n \frac{x_i^2}{2 \theta_0} - \frac{x_i^2}{2 \theta_1} \\
        & =
        \pbraces{\frac{\theta_0}{\theta_1}}^n
        \exp \sum_{i=1}^n \frac{(\theta_1 - \theta_0) x_i^2}{\theta_0 \theta_1} \\
        & =
        \pbraces{\frac{\theta_0}{\theta_1}}^n
        \exp \frac{(\theta_1 - \theta_0) T(x)}{\theta_0 \theta_1}
    \end{align*}

    \begin{align*}
        \Omega_1
        :=
        \Bbraces{\mathbf x: \lambda(\mathbf x) \geq C}
    \end{align*}

    \begin{align*}
        \lambda(\mathbf x) \geq C
        \iff
        T(\mathbf x) \geq \frac{\theta_0 \theta_1}{\theta_1 - \theta_0} \ln \pbraces{\pbraces{\frac{\theta_1}{\theta_0}}^n C} =: C^\star
    \end{align*}

    \begin{align*}
        \implies &
        \alpha = P_{\theta_0}(\lambda(\mathbf X) \geq C) = P_{\theta_0}(T(\mathbf X \geq C^\star)) = 1 - P_{\theta_0}(T(\mathbf X \leq C^\star)) \\
        \implies &
        1 - \alpha = P_{\theta_0}(T(\mathbf X) / \theta_0 \leq C^\star / \theta_0) \stackrel{!}{=} F_{\chi^2(2 n)}(C^\star / \theta_0) \\
        \implies &
        C^\star / \theta_0 = \chi^2_{1 - \alpha}(2 n) \\
        \implies &
        C^\star = \theta_0 \chi^2_{1 - \alpha}(2 n)
    \end{align*}

    For \enquote ! we used

    \begin{align*}
        P_\theta(T(\mathbf X) / \theta \leq x)
        & =
        \int_\bbraces{T(\mathbf X) / \theta \leq x}
            ~ \mathrm d P_\theta \\
        & =
        \int_{(T(\mathbf X) / \theta)^{-1}((-\infty, x])}
            1 \circ (T(\mathbf X) / \theta)
            ~ \mathrm d P_\theta \\
        & \stackrel{!}{=}
        \int_{-\infty}^x
            1
            ~ \mathrm d P_\theta (T(\mathbf X) / \theta)^{-1} \\
        & \stackrel{!!}{=}
        \int_{-\infty}^x
            f_{\chi^2(2 n)}(\xi)
            ~ \mathrm d \xi \\
        & =
        F_{\chi^2(2 n)}(x).
    \end{align*}

    For \enquote ! we used \cite[Satz 9.62]{kusolitsch2014maß}.
    For \enquote{!!} we used \cite[Satz 9.49]{kusolitsch2014maß}.

    \item We apply the Theorem on \cite[lecture 10, slide 33]{EStat}, so yes!
    Indeed, $\lambda(\mathbf x)$ is monotone in $T(\mathbf x)$.
    Furthermore,

    \begin{align*}
        \pi(\theta)
        & =
        P_\theta(\text{reject}~ H_0) \\
        & =
        P_\theta(\mathbf X \in \Omega_1) \\
        & =
        P_\theta(T(\mathbf X) \geq C^\star) \\
        & =
        1 - P_\theta(T(\mathbf X) / \theta \leq \theta_0 \chi_{1 - \alpha}^2(2 n) / \theta) \\
        & =
        1 - F_{\chi^2(2 n)}(\theta_0 \chi_{1 - \alpha}^2(2 n) / \theta).
    \end{align*}

    Indeed,

    \begin{align*}
        \pi(\theta_0)
        & =
        1 - F_{\chi^2(2 n)}(\chi_{1 - \alpha}^2(2 n)) \\
        & =
        1 - F_{\chi^2(2 n)}(F_{\chi^2(2 n)}^{-1}(1 - \alpha)) \\
        & =
        1 - (1 - \alpha) \\
        & =
        \alpha.
    \end{align*}

\end{enumerate}

\end{solution}

% -------------------------------------------------------------------------------- %
