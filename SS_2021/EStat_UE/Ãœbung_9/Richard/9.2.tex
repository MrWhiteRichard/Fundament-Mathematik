% -------------------------------------------------------------------------------- %

\begin{exercise}[Most powerful test 1]

Let $X_1, \dots, X_n$ be iid $\mathrm{Uniform}(0, \theta)$.

\begin{enumerate}[label = (\alph*)]

    \item Derive the most powerful (MP) test at level $\alpha$ for testing
    
    \begin{align*}
        H_0: \theta = \theta_0
        \quad
        \textit{vs}
        \quad
        H_1: \theta = \theta_1, \theta_1 > \theta_0.
    \end{align*}

    \item Calculate the power of the MP test.

\end{enumerate}

\end{exercise}

% -------------------------------------------------------------------------------- %

\begin{solution}

\phantom{}

\begin{enumerate}[label = (\alph*)]

    \item First, we decide, how to choose the rejection region $\Omega_1$ (and acceptance region $\Omega_0$).
    If $\mathbf X \in \Omega_1$, then we should reject $H_0$.
    According to \cite[lecture 6, slide 69]{EStat}, the MLE for $\theta$ is

    \begin{align*}
        T(\mathbf X) := X_{(n)}.
    \end{align*}

    If $T(\mathbf X)$ is large, then there is a higher chance of $\theta \approx T(\mathbf X) \approx \theta_1$.
    Then, we should favour $H_1$ and reject $H_0$.
    In order to aquire an appropriate $\alpha$, we have to specify exactly, where $T(\mathbf X)$ should be, in order to reject $H_0$.
    We make the Ansatz

    \begin{align*}
        \Omega_1
        :=
        \Bbraces{\mathbf x: T(\mathbf x) > a}
        =
        T^{-1}((a, \infty)).
    \end{align*}

    This leads to the condition

    \begin{align*}
        \alpha
        \stackrel{!}{=}
        P_{\theta_0}(\mathbf X \in \Omega_1)
        =
        P_{\theta_0}(T(\mathbf X) > a)
        =
        1 - P_{\theta_0}(X_{(n)} \leq a)
        =
        1 - \prod_{i=1}^n P_{\theta_0}(X_i \leq a)
        =
        1 - \pbraces{\frac{a}{\theta_0}}^n,
    \end{align*}

    and hence,

    \begin{align*}
        a = \theta_0 (1 - \alpha)^{1/n}.
    \end{align*}

    Now, only the question remains, whether our test is the most powerful one.
    In order to prove this, we will first calculate the power \dots

    \item The power of this test is now defined as

    \begin{align*}
        \pi
        & =
        P_{\theta_1}(\mathbf X \in \Omega_1) \\
        & =
        P_{\theta_1}(T(\mathbf X) > a) \\
        & =
        P_{\theta_1}(X_{(n)} > \theta_0 (1 - \alpha)^{1/n}) \\
        & =
        1 - P_{\theta_1}(X_{(n)} \leq \theta_0 (1 - \alpha)^{1/n}) \\
        & =
        1 - \prod_{i =1}^n P_{\theta_1}(X_i \leq \theta_0 (1 - \alpha)^{1/n}) \\
        & =
        1 - \pbraces{\frac{1}{\theta_1} \theta_0 (1 - \alpha)^{1/n}}^n \\
        & =
        1 - (1 - \alpha) \pbraces{\frac{\theta_0}{\theta_1}}^n.
    \end{align*}

    Now, any other test at level $\alpha$ with rejection region $\Omega_1^\prime$, has power

    \begin{align*}
        \pi^\prime
        & =
        P_{\theta_1}(\mathbf X \in \Omega_1^\prime) \\
        & =
        \int_{\Omega_1^\prime}
            \frac{1}{\theta_1^n}
            \mathbf 1_{(0, \theta_1)^n}(\mathbf x)
            ~ \mathrm d \mathbf x \\
        & =
        \int_{\Omega_1^\prime \cap (0, \theta_0)^n}
            \frac{1}{\theta_1^n}
            \mathbf 1_{(0, \theta_1)^n}(\mathbf x)
            ~ \mathrm d \mathbf x
        +
        \int_{\Omega_1^\prime \setminus (0, \theta_0)^n}
            \frac{1}{\theta_1^n}
            \mathbf 1_{(0, \theta_1)^n}(\mathbf x)
            ~ \mathrm d \mathbf x \\
        & \stackrel{!}{\leq}
        \frac{\theta_0^n}{\theta_1^n} \alpha
        +
        1 - \frac{\theta_0^n}{\theta_1^n} \\
        & =
        \pi.
    \end{align*}

    For \enquote{!} we used

    \begin{align*}
        \int_{\Omega_1^\prime \cap (0, \theta_0)^n}
            \frac{1}{\theta_1^n}
            \mathbf 1_{(0, \theta_1)^n}(\mathbf x)
            ~ \mathrm d \mathbf x
        & =
        \frac{\theta_0^n}{\theta_1^n}
        \int_{\Omega_1^\prime}
            \frac{1}{\theta_0^n}
            \mathbf 1_{(0, \theta_0)^n}(\mathbf x)
            ~ \mathrm d \mathbf x \\
        & =
        \frac{\theta_0^n}{\theta_1^n}
        P_{\theta_0}(\mathbf X \in \Omega_1^\prime) \\
        & =
        \frac{\theta_0^n}{\theta_1^n} \alpha,
    \end{align*}

    and

    \begin{align*}
        \int_{\Omega_1^\prime \setminus (0, \theta_0)^n}
            \frac{1}{\theta_1^n}
            \mathbf 1_{(0, \theta_1)^n}(\mathbf x)
            ~ \mathrm d \mathbf x
        & \leq
        \int_{(0, \infty)^n \setminus (0, \theta_0)^n}
            \frac{1}{\theta_1^n}
            \mathbf 1_{(0, \theta_1)^n}(\mathbf x)
            ~ \mathrm d \mathbf x \\
        & =
        1
        - 
        \int_{(0, \theta_0)^n}
            \frac{1}{\theta_1^n}
            \mathbf 1_{(0, \theta_1)^n}(\mathbf x)
            ~ \mathrm d \mathbf x \\
        & =
        1
        - 
        \frac{1}{\theta_1^n}
        \int
            \mathbf 1_{(0, \theta_1)^n \cap (0, \theta_0)^n}(\mathbf x)
            ~ \mathrm d \mathbf x \\
        & =
        1
        - 
        \frac{1}{\theta_1^n}
        \lambda^n((0, \theta_0)^n) \\
        & =
        1 - \frac{\theta_0^n}{\theta_1^n}.
    \end{align*}

\end{enumerate}

\end{solution}

% -------------------------------------------------------------------------------- %
