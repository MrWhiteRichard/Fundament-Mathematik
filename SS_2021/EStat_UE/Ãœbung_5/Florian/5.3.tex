% -------------------------------------------------------------------------------- %

\begin{exercise}[Simulations]

\phantom{}

\begin{enumerate}[label = (\alph*)]
  \item By applying the \texttt{R}-function \texttt{replicate()} generate a sample
  $X_1,\dots,X_{10}$ of size 10 from an exponential distribution with a rate parameter 0.2
  and sum up its elements. DO this sum 10 000 times and make a histogram of the simulation.
  Can you say something about the shape of the distribution?
  \item Use \texttt{R} to simulate 50 tosses of a fair coin (0 and 1). We call
  a \textit{run} a sequence of all 1's or all 0's. Estimate the average length of the
  longest run in 10 000 trails and report the result.

  \textit{Hint:} Use the commands \texttt{rbinom} and \texttt{rle}. The command
  \texttt{rle()} stands for run length encoding. For example,

  \begin{align*}
    \texttt{rle(rbinom(5, 1, 0.5))\$ lengths}
  \end{align*}

  is a vector of the lengths of all the different runs in trial of 5 flips of a fair coin.
\end{enumerate}

\end{exercise}

% -------------------------------------------------------------------------------- %

\begin{solution}

\phantom{}

\end{solution}

% -------------------------------------------------------------------------------- %
