% --------------------------------------------------------------------------------

\begin{exercise}

\phantom{}

\begin{enumerate}[label = (\alph*)]

    \item Delta method
    
    Let $X_1, \dots, X_n$ be i.i.d. from normal distribution with unknown mean $\mu$ and known variance $\sigma^2$.
    Let $\bar X = \frac{1}{n} \sum_{i=1}^n X_i$.
    Find the limiting distribution of $\sqrt n \pbraces{\bar X^3 - c}$ for an appropriate constant $c$.

    \item Logit transformation
    
    Let $X_n \sim \mathit{bin}(n, p)$.
    Consider the logit transformation, defined by

    \begin{align*}
        \mathit{logit}(y)
        =
        \ln \frac{y}{1 - y},
        \quad
        0 < y < 1.
    \end{align*}

    Determine the approximate distribution of $\mathit{logit} \pbraces{\frac{X_n}{n}}$.

\end{enumerate}

\end{exercise}

% --------------------------------------------------------------------------------

\begin{solution}

\phantom{}

\begin{enumerate}[label = (\alph*)]

    \item Due to the CLT, we have \cite[Lecture 4, Slide 59]{EStat}
    
    \begin{align*}
        \sqrt n (\bar X - \mu) \xrightarrow[n \to \infty]{\mathrm d} \mathcal N(0, \sigma^2).
    \end{align*}

    Let $g(x) := x^3$, then

    \begin{gather*}
        g(\bar X) = \bar X ^3, \quad g(\mu) = \mu^3 =: c, \\
        g^\prime(x) = 3 x^2, \quad \sigma g^\prime(\mu) = 3 \mu^2 \sigma =: s.
    \end{gather*}

    The lemma from \cite*[Lecture 5, Slide 10]{EStat} then implies

    \begin{align*}
        \sqrt n (\bar X^3 - c)
        =
        \sqrt n (g(\bar X) - g(\mu))
        \xrightarrow[n \to \infty]{\mathrm d}
        \mathcal N(0, \sigma^2 (g^\prime(\mu))^2)
        =
        \mathcal N(0, s^2).
    \end{align*}

    Therefore, we get the estimate

    \begin{align*}
        \hat X^3
        \approx
        \mathcal N \pbraces{\mu^3, \frac{9 \mu^4 \sigma^2}{n}}.
    \end{align*}

    \item

    \begin{enumerate}[label = \arabic*.]

        \item Step (CLT):
        
        Let $Y, Y_1, \dots, Y_n \sim \mathit{ber}(p)$ be i.i.d. and $X_n = \sum_{i=1}^n Y_i$.

        \begin{align*}
            \bar Y_n := \frac{1}{n} \sum_{i=1}^n Y_i = \frac{X_n}{n}
        \end{align*}

        Applying the CLT obtains

        \begin{align*}
            Z_n
            :=
            \sqrt n \frac{\bar Y_n - \E Y}{\sqrt{\Var Y}}
            \xrightarrow[n \to \infty]{\mathrm d}
            Z
            \sim
            \mathcal N(0, 1).
        \end{align*}

        Recall that

        \begin{align*}
            \E Y = p,
            \quad
            \Var Y = p (1 - p).
        \end{align*}

        Slutsky's Theorem \cite*[Lecture 4, Slide 70]{EStat} implies

        \begin{align*}
            \sqrt n \pbraces{\frac{X_n}{n} - p}
            =
            \sqrt{\Var Y} Z_n
            \xrightarrow[n \to \infty]{\mathrm d}
            \sqrt{\Var Y} Z
            \sim
            \mathcal N(0, \Var Y)
            =
            \mathcal N(0, p (1 - p)).
        \end{align*}

        \item Step (Delta Method):
        
        We want to apply it to $\mathit{logit}$.
        For that, we first calulate

        \begin{align*}
            \mathit{logit}^\prime(y)
            & =
            \frac{1}{\frac{y}{1 - y}} \frac{-1}{(1 - y)^2} (-1) \\
            & =
            \frac{1}{y (1 - y)},
        \end{align*}

        with the help of

        \begin{align*}
            \frac{y}{1 - y}
            & =
            -\frac{y - 1 + 1}{y - 1} \\
            & =
            \frac{1}{1 - y} - 1.
        \end{align*}

        The lemma from \cite*[Lecture 5, Slide 10]{EStat} then implies

        \begin{multline*}
            \sqrt n \pbraces{\mathit{logit} \pbraces{\frac{X_n}{n}} - \ln \frac{p}{1 - p}}
            =
            \sqrt n \pbraces{\mathit{logit} \pbraces{\frac{X_n}{n}} - \mathit{logit}(p)} \\
            \xrightarrow[n \to \infty]{\mathrm d}
            \mathcal N(0, p (1 - p) \mathit{logit}^\prime(p)^2)
            =
            \mathcal N \pbraces{0, \frac{1}{p (1 - p)}}.
        \end{multline*}

        Therefore, we get the estimate

        \begin{align*}
            \mathit{logit} \pbraces{\frac{X_n}{n}}
            \approx
            \mathcal N \pbraces{\mathit{logit}(p), \frac{1}{p (1 - p)}}.
        \end{align*}

    \end{enumerate}

\end{enumerate}

\end{solution}

% --------------------------------------------------------------------------------
