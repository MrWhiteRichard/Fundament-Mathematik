% --------------------------------------------------------------------------------

\begin{exercise}[Simulations]

\phantom{}

\begin{enumerate}[label = (\alph*)]

    \item By applying the \texttt R-function \texttt{replicate()} generate a sample $X_1, \dots, X_{10}$ of size $10$ from an exponential distribution with a rate parameter $0.2$ and sum up its elements.
    Do this sum $10 000$ times and make a histogram of the simulation.
    Can you say something about the shape of distribution?

    \item Use \texttt R to simulate $50$ tosses of a fair coin ($0$ and $1$).
    We call a \textit{run} a sequence of all $1$'s or all $0$'s.
    Estimate the average length of the longest run in $10 000$ trials and report the result.

    \textit{Hint:}
    Use the commands \texttt{rbinom} and \texttt{rle}.
    The command \texttt{rle()} stands for run length encoding.
    For example,

    \verb|rle(rbinom(5, 1, 0.5))$lengths|

    is a vector of the lengths of all the differnt runs in trial of $5$ filps of a fair coin.

\end{enumerate}

\end{exercise}

% --------------------------------------------------------------------------------

\begin{solution}

\phantom{}

\begin{enumerate}[label = (\alph*)]

    \item \phantom{}
    \lstinputlisting{5.3.a.r}

    \item \phantom{}
    \lstinputlisting{5.3.b.r}

\end{enumerate}

\end{solution}

% --------------------------------------------------------------------------------
