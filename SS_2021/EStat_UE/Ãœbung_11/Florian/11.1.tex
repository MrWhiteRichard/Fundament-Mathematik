% -------------------------------------------------------------------------------- %

\begin{exercise}[\textbf{Test power in the $z$-test}]

Let $X_1,\dots,X_n$ be i.i.d. random variables with $X_1 \sim \mathcal{N}(\mu,\sigma^2)$,
and $H_0: \mu = \mu_0$.

\begin{enumerate}[label = (\alph*)]
    \item Compute the test power of the left-sided $z$-test.
    Express it through cdf of the $\mathcal{N}(0,1)$ distribution, depending
    on $\mu_0,\mu,\sigma,n$ and the significance level $\alpha$.
    \item Comment on the impact of $\mu_0, \mu, \sigma, n$ and $\alpha$ on the test power.
\end{enumerate}

\end{exercise}

% -------------------------------------------------------------------------------- %

\begin{solution}

\phantom{}

\begin{enumerate}[label = (\alph*)]
    \item The alternative hypothesis reads $H_1: \mu > \mu_0$.
    
    We calculate the test power

    \begin{align*}
        \pi(\mu_0, \mu, \sigma, n, \alpha) 
        &= \P(\text{accept } H_1 | H_1 \text{ is true}) \\
        &= \P_{\mu}\left(\frac{\bar{X}_n - \mu_0}{\sigma/\sqrt{n}} > z_\alpha\right) \\
        &= \P_{\mu}\left(\frac{\bar{X}_n - \mu}{\sigma/\sqrt{n}} > z_\alpha - \frac{\mu - \mu_0}{\sigma/ \sqrt{n}}\right) \\
        &= 1 - \Phi\left(z_\alpha - \frac{\mu - \mu_0}{\sigma/ \sqrt{n}}\right) \\
        &= \Phi\left(\frac{\mu - \mu_0}{\sigma/ \sqrt{n}} - z_\alpha\right)
    \end{align*}

    \item The test power is monotonically increasing in $(\mu - \mu_0)$ and in $n$,
    but decreasing in $\sigma$ and $\alpha$.

    Hence the larger the actual effect size $\frac{\mu - \mu_0}{\sigma}$ 
    and the larger the sample size $n$ the more powerful the test,
    whilst a lower significance level hinders the power of the test.
\end{enumerate}

\end{solution}

% -------------------------------------------------------------------------------- %
