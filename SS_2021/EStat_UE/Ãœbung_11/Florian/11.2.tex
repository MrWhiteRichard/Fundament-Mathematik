% -------------------------------------------------------------------------------- %

\begin{exercise}[\textbf{Shock absorbers}]

    A manufacturer of automobile shock absorbers was interested in comparing the durability of
    its shocks with that of the shocks produced by its biggest competitor. To make the comparison,
    one of the manufacturer’s and one of the competitor’s shocks were randomly selected
    and installed on the rear wheel of each of six cars. After the cars had been driven 20000
    miles, the strength of each test shock was measured, coded, and recorded. Results are shown
    in the table.

    \begin{center}
        \begin{tabular}{c | c c } 
        \textbf{Car number} & \textbf{Manufacturer's shock} & \textbf{Competitor's shock} \\  
        \hline
        1 & 8.8 & 8.4 \\
        2 & 10.5 & 10.1 \\
        3 & 12.5 & 12.0 \\
        4 & 9.7 & 9.3 \\
        5 & 9.6 & 9.0 \\
        6 & 13.2 & 13.0 \\
        \hline
       \end{tabular}
       
    \end{center}

    Do these data present sufficient evidence to conclude that there is 
    a difference in the mean strength of the two types of shocks after
    20,000 miles of use?

\end{exercise}

% -------------------------------------------------------------------------------- %

\begin{solution}

We mindlessly choose our significance level $\alpha = 0.05$.

Let $\mu_0$ and $\mu_0$ denote the average test shocks of the two competitors.
We want to test $H_0: \mu_0 - \mu_1 = 0$ versus $H_1: \mu_0 - \mu_1 \neq 0$.

We consider the population of differences, define $\mu_d := \mu_0 - \mu_1$ and denote by $\bar{d}$ the sample mean
and by $s_d$ the sample standard deviation. Our test statistic then reads

\begin{align*}
    t = \frac{\bar{d} - \mu_d}{s_d/\sqrt{n}} \approx_{H_0} t_{n-1}
\end{align*}

Plugging in the given data, we obtain

\begin{align*}
    \bar{d} &= \frac{0.4 + 0.4 + 0.5 + 0.4 + 0.6 + 0.2}{6} = \frac{5}{12}, \\
    s_d &\approx 0.132916
\end{align*}

We reject the null if

\begin{align*}
    |\bar{d}| > \frac{s_d}{\sqrt{n}}t_{\alpha/2, 5} 
    \approx \frac{0.132916}{\sqrt{6}} \cdot 2.570582 \approx 0.139487.
\end{align*}

Since $5/12 \approx 0.4167 > 0.139487$ we conclude that 
there is a difference in the mean strength of the two types of shocks.

We can also compute the $p$-value:

\begin{align*}
    p &= \P(|t| \geq \frac{5/12}{0.132916 / \sqrt{6}}) 
    = 2\left(1 - F_t\left(\frac{5/12}{0.132916 / \sqrt{6}}\right)\right) \\
    &\approx 0.0005970639.
\end{align*}


\end{solution}

% -------------------------------------------------------------------------------- %
