% -------------------------------------------------------------------------------- %

\begin{exercise}[\textbf{Mechanics}]

In order to compare the means of two populations, independent random samples
of 400 observations are selected from each population, with the following results:

\begin{center}
    \begin{tabular}{c c}
        Sample 1 & Sample 2 \\
        $\bar{x}_1 = 5,275$ & $\bar{x}_2 = 5,240$ \\
        $s_1 = 150$ & $s_2 = 200$ \\
    \end{tabular}
\end{center}

\begin{enumerate}[label = (\alph*)]
    \item Use a 95\% confidence interval to estimate the difference between
    the population means $(\mu_1 - \mu_2)$. Interpret the confidence interval.
    \item Test the null hypothesis $H_0: (\mu_1 - \mu_2) = 0$ versus the
    alternative hypothesis $H_1: (\mu_1 - \mu_2) \neq 0$. Give the $p$-value of the test, and interpret the result.
    \item Suppose the test in the previous part were conducted with the alternative hypothesis 
    $H_1: (\mu - \mu_2) > 0$. How would your answer change?
    \item Test the null hypothesis $H_0: (\mu_1 - \mu_2) = 25$ versus the alternative $H_1: (\mu_1 - \mu_2) \neq 25$.
    Give the $p$-value, and interpret the result. Compare your answer with that
    obtained from the test conducted in part (b).
    \item What assumptions are necessary to ensure the validity of the inferential procedures applied in parts (a)-(d)?
\end{enumerate}

\end{exercise}

% -------------------------------------------------------------------------------- %

\begin{solution}

\phantom{}

\end{solution}

% -------------------------------------------------------------------------------- %
