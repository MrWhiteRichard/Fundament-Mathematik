% ---------------------------------------------------------------- %

\begin{exercise}[Shock absorbers]

A manufacturer of automobile shock absorbers was interested in comparing the durability of its shocks with that of the shocks produced by its biggest competitor.
To make the comparison, one of the manufacturer's and one of the competitor's shocks were randomly selected and installed on the rear wheel of each of six cars.
After the cars had been driven $20 000$ miles, the strength of each test shock was measured, coded, and recorded.
Results are shown in the table.

\begin{align*}
    \begin{array}{c|cc}
        \textbf{Car number}
        &
        \textbf{Manufacturer's shock}
        &
        \textbf{Competitor's shock} \\
        \hline
        1 & 8.8  & 8.4  \\
        2 & 10.5 & 10.1 \\
        3 & 12.5 & 12.0 \\
        4 & 9.7  & 9.3  \\
        5 & 9.6  & 9.0  \\
        6 & 13.2 & 13.0 \\
        \hline
    \end{array}
\end{align*}

Do these data present sufficient evidence to conclude there is a difference in the mean strength of the two types of shocks after $20 000$ miles of use?

\end{exercise}

% ---------------------------------------------------------------- %

\begin{solution}

Let $\mu_\mathrm{M}$ and $\mu_\mathrm{C}$ denote the average shock of the the manufacturer and competitor respectively.
$\mu = \mu_\mathrm{M} - \mu_\mathrm{C}$ is then the \enquote{difference in the mean strenth}.
Moreover, let $X_{\mathrm M, i}, X_{\mathrm C, i}$, be the respective shock samples and $X_i = X_{\mathrm M, i} - X_{\mathrm C, i}$ their differences, for $i = 1, \dots, 6$.

One concludes whether \enquote{there is a difference in the mean strenth} or not, by testing the Two-Tailed Test

\begin{align*}
    H_0: \mu = 0
    \quad
    \textit{vs}
    \quad
    H_1: \mu \neq 0.
\end{align*}

We use the test statistic

\begin{align*}
    T = \frac{\bar X_n - \mu}{S_n / \sqrt n} \sim t(n-1),
\end{align*}

with the sample average

\begin{align*}
    \bar X_n = \frac{1}{n} \sum_{i=1}^n X_i,
\end{align*}

and sample standard deviation

\begin{align*}
    S_n = \frac{1}{n-1} \sum_{i=1}^n (\bar X_n - X_i)^2.
\end{align*}

For our data, we get

\begin{align*}
    n = 6,
    \quad
    \bar x = \frac{5}{12}
    \quad
    \text{and}
    \quad
    s \approx 0.132916,
\end{align*}

and hence, under the $H_0$, the test statistic gets instanciated with

\begin{align*}
    t = \frac{\bar x - \mu}{s / \sqrt n} \approx \frac{\frac{5}{12} - 0}{0.132916 / \sqrt 6}
\end{align*}

According to \cite[lecture 9, slide 35]{EStat}, we can compute the

\begin{align*}
    \text{$p$-value}
    =
    2 P_0(T \geq |t|)
    \approx
    2 (1 - F_T(t))
    \approx
    0.0005970639.
\end{align*}

Since the $p$-value is pretty small, we should probably reject the $H_0$, i.e. conclude that there is indeed \enquote{a difference in the mean strenth}.

\end{solution}

% ---------------------------------------------------------------- %