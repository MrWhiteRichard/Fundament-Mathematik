% ---------------------------------------------------------------- %

\begin{exercise}[Shock absorbers]

A manufacturer of automobile shock absorbers was interested in comparing the durability of its shocks with that of the shocks produced by its biggest competitor.
To make the comparison, one of the manufacturer's and one of the competitor's shocks were randomly selected and installed on the rear wheel of each of six cars.
After the cars had been driven $20 000$ miles, the strength of each test shock was measured, coded, and recorded.
Results are shown in the table.

\begin{align*}
    \begin{array}{c|cc}
        \textbf{Car number}
        &
        \textbf{Manufacturer's shock}
        &
        \textbf{Competitor's shock} \\
        \hline
        1 & 8.8  & 8.4  \\
        2 & 10.5 & 10.1 \\
        3 & 12.5 & 12.0 \\
        4 & 9.7  & 9.3  \\
        5 & 9.6  & 9.0  \\
        6 & 13.2 & 13.0 \\
        \hline
    \end{array}
\end{align*}

Do these data present sufficient evidence to conclude there is a difference in the mean strength of the two types of shocks after $20 000$ miles of use?

\end{exercise}

% ---------------------------------------------------------------- %

\begin{solution}

\phantom{}

\end{solution}

% ---------------------------------------------------------------- %