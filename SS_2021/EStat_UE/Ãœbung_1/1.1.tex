% --------------------------------------------------------------------------------

\begin{exercise}[Card game]

A deck of $52$ cards has $13$ ranks ($2$, $3$, $4$, $5$, $6$, $7$, $8$, $9$, $10$, $J$, $Q$, $K$, $A$) an $4$ suits ($\textcolor{red}{\heartsuit}$, $\spadesuit$, $\textcolor{red}{\diamondsuit}$, $\clubsuit$).
Three cards are drawn randomly without replacement from a deck of $52$ cards.

\begin{enumerate}[label = (\alph*)]
    \item What ist the probability that all cards are in the same suit?
    \item What ist the probability that all cards have the same rank?
    \item What ist the probability that the three cards contain exactly one pair (a pair means two cards with the same rank from two different suits)?
\end{enumerate}

\end{exercise}

% --------------------------------------------------------------------------------

\begin{solution}

Let $X_i$ denote the card that gets chosen in the $i$-th step and

\begin{align*}
    S := \Bbraces{\textcolor{red}{\heartsuit}, \spadesuit, \textcolor{red}{\diamondsuit}, \clubsuit},
    \quad
    R := \Bbraces{2, 3, 4, 5, 6, 7, 8, 9, 10, J, Q, K, A}
\end{align*}

\begin{enumerate}[label = (\alph*)]

    \item

    \begin{align*}
        &
        P(\operatorname{suit} X_1 = \cdots = \operatorname{suit} X_3) \\
        & =
        \sum_{s \in S} P(\operatorname{suit} X_1 = \cdots = \operatorname{suit} X_3 = s) \\
        & =
        \sum_{s \in S} P(\operatorname{suit} X_1 = s) P(\operatorname{suit} X_2 = s ~|~ \operatorname{suit} X_1 = s) P(\operatorname{suit} X_3 = s ~|~ \operatorname{suit} X_1 = \operatorname{suit} X_2 = s) \\
        & =
        |S| \frac{13}{52} \frac{12}{51} \frac{11}{50}
    \end{align*}

    \item

    \begin{align*}
        &
        P(\operatorname{rank} X_1 = \dots = \operatorname{rank} X_3) \\
        & =
        \sum_{r \in R} P(\operatorname{rank} X_1 = \cdots = \operatorname{rank} X_3 = r) \\
        & =
        \sum_{r \in R} P(\operatorname{rank} X_1 = r) P(\operatorname{rank} X_2 = r ~|~ \operatorname{rank} X_1 = r) P(\operatorname{rank} X_3 = r ~|~ \operatorname{rank} X_1 = \operatorname{rank} X_2 = r) \\
        & =
        |R| \frac{4}{52} \frac{3}{51} \frac{2}{50}
    \end{align*}

    \item

    \begin{align*}
        &
        P((X_1, \dots, X_3) ~\text{contains exactly one pair}) \\
        & =
        P(|\Bbraces{\operatorname{rank} X_1, \dots, \operatorname{rank} X_3}| = 2) \\
        & =
        P(\operatorname{rank} X_1 = \operatorname{rank} X_2 \neq \operatorname{rank} X_3) \\
        & +
        P(\operatorname{rank} X_2 = \operatorname{rank} X_3 \neq \operatorname{rank} X_1) \\
        & +
        P(\operatorname{rank} X_3 = \operatorname{rank} X_1 \neq \operatorname{rank} X_2) \\
        & =
        \sum_{r_1 \in R} \sum_{r_2 \in R \setminus \Bbraces{r_1}}
            P(\operatorname{rank} X_1 = \operatorname{rank} X_2 = r_1, \operatorname{rank} X_3 = r_2) \\
        & +
        \sum_{r_1 \in R} \sum_{r_2 \in R \setminus \Bbraces{r_1}}
            P(\operatorname{rank} X_2 = \operatorname{rank} X_3 = r_2, \operatorname{rank} X_1 = r_1) \\
        & +
        \sum_{r_1 \in R} \sum_{r_2 \in R \setminus \Bbraces{r_1}}
            P(\operatorname{rank} X_3 = \operatorname{rank} X_1 = r_1, \operatorname{rank} X_2 = r_2) \\
        & =
        \sum_{r_1 \in R} \sum_{r_2 \in R \setminus \Bbraces{r_1}} \\
            & P(\operatorname{rank} X_1 = r_1) P(\operatorname{rank} X_2 = r_1 ~|~ \operatorname{rank} X_1 = r_1) P(\operatorname{rank} X_3 = r_2 ~|~ \operatorname{rank} X_1 = r_1, \operatorname{rank} X_2 = r_1)
            + \\
            & P(\operatorname{rank} X_1 = r_1) P(\operatorname{rank} X_2 = r_2 ~|~ \operatorname{rank} X_1 = r_1) P(\operatorname{rank} X_3 = r_2 ~|~ \operatorname{rank} X_1 = r_1, \operatorname{rank} X_2 = r_2)
            + \\
            & P(\operatorname{rank} X_1 = r_2) P(\operatorname{rank} X_2 = r_2 ~|~ \operatorname{rank} X_1 = r_1) P(\operatorname{rank} X_3 = r_2 ~|~ \operatorname{rank} X_1 = r_1, \operatorname{rank} X_2 = r_2) \\
        & =
        |R| (|R| - 1)
        \pbraces
        {
            \frac{4}{52} \frac{3}{51} \frac{12 \cdot 4}{50}
            +
            \frac{4}{52} \frac{12 \cdot 4}{51} \frac{12 \cdot 4 - 1}{50}
            +
            \frac{4}{52} \frac{12 \cdot 4}{51} \frac{3}{50}
        }
    \end{align*}

\end{enumerate}

\end{solution}

% --------------------------------------------------------------------------------
