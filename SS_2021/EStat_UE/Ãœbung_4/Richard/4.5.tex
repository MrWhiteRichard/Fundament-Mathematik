% --------------------------------------------------------------------------------

\begin{exercise}[Central Limit Theorem]

Let $\bar X_1$ and $\bar X_2$ be the means of two independent samples of size $n$ from the same population with variance $\sigma^2$.
Use the Central limit theorem to find a value for $n$ so that

\begin{align*}
    P(|\bar X_1 - \bar X_2| < \frac{\sigma}{50}) \approx 0.99.
\end{align*}

Justify your calculations.

\end{exercise}

% --------------------------------------------------------------------------------

\begin{solution}

Let $X_{1, 1} \sim \cdots \sim X_{n, 1} \sim X_{1, 2} \sim \cdots \sim X_{n, 2}$ be the samples of draws $1$ and $2$, respectively, i.e.

\begin{align*}
    \bar X_1 =: \bar X_{n, 1} = \frac{1}{n} \sum_{i=1}^n X_{i, 1},
    \quad
    \bar X_2 =: \bar X_{n, 2} = \frac{1}{n} \sum_{i=1}^n X_{i, 2},
\end{align*}

with common mean $\mu$ and variance $\sigma^2$.
Furthermore, let

\begin{align*}
    X_i := X_{i, 1} - X_{i, 2},
    \quad
    i = 1, \dots, n,
\end{align*}

and

\begin{align*}
    \bar X_n
    & :=
    \bar X_{n, 1} - \bar X_{n, 2} \\
    & =
    \frac{1}{n} \sum_{i=1}^n X_{i, 1}
    -
    \frac{1}{n} \sum_{i=1}^n X_{i, 2} \\
    & =
    \frac{1}{n} \sum_{i=1}^n X_{i, 1} - X_{i, 2} \\
    & =
    \frac{1}{n} \sum_{i=1}^n X_i.
\end{align*}

For $i = 1, \dots, n$ we calculate the first and second moments.

\begin{align*}
    E(X_i) & = E(X_{i, 1} - X_{i, 2}) = E(X_{i, 1}) - E(X_{i, 2}) = \mu - \mu = 0 \\
    V(X_i) & = V(X_{i, 1} + (-X_{i, 2})) = V(X_{i, 1}) + V(-X_{i, 2}) + 2 \operatorname{Cov}(X_{i, 1}), -X_{i, 2}) = 2 \sigma^2 - 2 \operatorname{Cov}(X_{i, 1}, X_{i, 2})
\end{align*}

However, in order to apply the Central Limit Theorem, we should have

\begin{align*}
    V(X_i) = \sigma^2 > 0
    \quad
    \text{i.e.}
    \quad
    \operatorname{Cov}(X_{i, 1}, X_{i, 2}) = \frac{\sigma^2}{2}.
\end{align*}

Let us assume the former.
Due to the Central Limit Theorem \cite[Lecture 4, Slide 61]{EStat}, there exists a $Z \sim \mathcal N(0, 1)$ such that

\begin{align*}
    Y_n
    :=
    \frac{\bar X_n - 0}{\sigma / \sqrt n}
    \xrightarrow[n \to \infty]{\text d}
    Z,
    \quad
    & \text{i.e.}
    \quad
    \forall x ~\text{continuity point of $F_Z = \Phi$}:
        F_{Y_n}(x) \xrightarrow{n \to \infty} F_Z(x) = \Phi(x), \\
    & \text{i.e.}
    \quad
    F_{Y_n} \xrightarrow[n \to \infty]{\text{point wise}} \Phi.
\end{align*}

Note that $\Phi$ is continuous.
Therefore, all points are continuity points of $\Phi$.

\begin{align*}
    0.99
    & \approx
    P \pbraces{|\bar X_1 - \bar X_2| < \frac{\sigma}{50}} \\
    & =
    P \pbraces{|Y_n| < \frac{\sqrt n}{50}} \\
    & =
    P \pbraces{-\frac{\sqrt n}{50} < Y_n < \frac{\sqrt n}{50}} \\
    & =
    F_{Y_n} \pbraces{\frac{\sqrt n}{50}} - F_{Y_n} \pbraces{-\frac{\sqrt n}{50}} \\
    & \stackrel
    {
        \text{CLT}
    }{\approx}
    \Phi \pbraces{\frac{\sqrt n}{50}} - \Phi \pbraces{-\frac{\sqrt n}{50}} \\
    & =
    \Phi \pbraces{\frac{\sqrt n}{50}} - \pbraces{1 - \Phi \pbraces{\frac{\sqrt n}{50}}} \\
    & =
    2 \Phi \pbraces{\frac{\sqrt n}{50}} - 1
\end{align*}

\begin{align*}
    \iff
    \Phi \pbraces{\frac{\sqrt n}{50}}
    \approx
    \frac{0.99 + 1}{2}
    =
    0.995
\end{align*}

Recall the table of $\mathcal N(0, 1)$-distribution \cite[Lecture 2, Slide 44]{EStat}.

\begin{align*}
    \iff
    \frac{\sqrt n}{50} \approx \Phi^{-1}(0.995) \in (2.57, 2.58)
    \iff
    n
    \approx
    \begin{cases}
        (50 \cdot 2.57)^2 = 16 512.25 \\
        (50 \cdot 2.58)^2 = 16 641
    \end{cases}
\end{align*}

\end{solution}

% --------------------------------------------------------------------------------
