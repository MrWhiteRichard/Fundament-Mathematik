% --------------------------------------------------------------------------------

\begin{exercise}[Real roots]

Let $A$, $B$ and $C$ be independent random variables, uniformly distributed on $(0, 1)$.

\begin{enumerate}[label = (\alph*)]

    \item What is the probability that the quadratic equation $A x^2 + B x + C = 0$ has real roots?

    \item Consider the following code in R.
    What does it do and how is it related to your solution in part (a)?

    \begin{lstlisting}
        n=10000
        a=runif(n)
        b=runif(n)
        c=runif(n)
        sum(b^2>4*a*c)/n
    \end{lstlisting}

\end{enumerate}

\textit{Hint:}
In HW2/ex. 3(b) we showed that if $X$ has uniform $(0, 1)$ distribution then $-\log X$ has exponential distribution $\exp(1)$.
In an analogue way, one an prove that $-s \log X \sim \exp(\frac{1}{s})$ for any $s > 0$.
Also, in HW4/ex. 2(b) we proved that the sum of two independent exponential distributions is a gamma distribution.
Namely, if $X \sim \exp(1)$ and $Y \sim \exp(1)$ are independent then $X + Y = \operatorname{Gamma}(2, 1)$.

\end{exercise}

% --------------------------------------------------------------------------------

\begin{solution}

\phantom{}

\begin{enumerate}[label = (\alph*)]

    \item Recall the solution formula for quadratic equations.
    
    \begin{align*}
        A x^2 + B x + C = 0
        \iff
        \frac
        {
            -B \pm \sqrt{B^2 - 4 A C}
        }{
            2 A
        }
    \end{align*}
    
    The solution(s) is $\in \R$ if and only if the discriminant $B^2 - 4 A C \geq 0$.
    We start our main calculation.

    \begin{align*}
        P(\exists x \in \R: A x^2 + B x + C = 0)
        & =
        P(4 A C - B^2 \leq 0) \\
        & =
        F_X(0)
    \end{align*}

    Let

    \begin{align*}
        X & = 4 A C - B^2, \\
        Y & = A, \\
        Z & = B,
    \end{align*}
    
    with inverse mapping
    
    \begin{align*}
        A & = Y, \\
        B & = Z, \\
        C & = \frac{X + Z^2}{4 Y}.
    \end{align*}
    
    The Jacobian is
    
    \begin{align*}
        J(x, y, z)
        =
        \begin{vmatrix}
            \pderivative[][a]{x} & \pderivative[][a]{y} & \pderivative[][a]{z} \\
            \pderivative[][b]{x} & \pderivative[][b]{y} & \pderivative[][b]{z} \\
            \pderivative[][c]{x} & \pderivative[][c]{y} & \pderivative[][c]{z} \\
        \end{vmatrix}
        =
        \begin{vmatrix}
            0 & 1 & 0 \\
            0 & 0 & 1 \\
            \frac{1}{4 y} & -\frac{x + z^2}{4 y^2} & \frac{z}{2 y}
        \end{vmatrix}
        =
        -\frac{1}{4 y}.
    \end{align*}
    
    By the change-of-variable formula \cite[Lecture 3, Slide 32]{EStat}, the joint density $X$, $Y$ and $Z$ is
    
    \begin{align*}
        f_{X, Y, Z}(x, y, z)
        & =
        f_{A, B, C}
        \pbraces
        {
            y, z, \frac{x + z^2}{4 y}
        } \\
        & =
        f_A(y) f_B(z) f_C \pbraces{\frac{x + z^2}{4 y}} \frac{1}{4 |y|} \\
        & =
        \mathbf 1_{(0, 1)}(y)
        \mathbf 1_{(0, 1)}(z)
        \mathbf 1_{(0, 1)}
        \pbraces{\frac{x + z^2}{4 y}}
        \frac{1}{4 y},
    \end{align*}
    
    by the independence of $A$, $B$, and $C$.
    We continue our main calculation.

    \begin{align*}
        & =
        P(X \leq 0) \\
        & =
        \Int[-\infty][0]
        {
            f_X(x)
        }{x} \\
        & =
        \Int[-\infty][0]
        {
            \Int[-\infty][\infty]
            {
                \Int[-\infty][\infty]
                {
                    f_{X, Y, Z}(x, y, z)
                }{z}
            }{y}
        }{x} \\
        & =
        \Int[-\infty][0]
        {
            \Int[-\infty][\infty]
            {
                \Int[-\infty][\infty]
                {
                    \mathbf 1_{(0, 1)}(y)
                    \mathbf 1_{(0, 1)}(z)
                    \mathbf 1_{(0, 1)}
                    \pbraces{\frac{x + z^2}{4 y}}
                    \frac{1}{4 y}
                }{z}
            }{y}
        }{x} \\
        & =
        \Int[-\infty][\infty]
        {
            \Int[-\infty][\infty]
            {
                \mathbf 1_{(0, 1)}(y)
                \mathbf 1_{(0, 1)}(z)
                \Int[-\infty][\infty]
                {
                    \mathbf 1_{(-\infty, 0)}(x)
                    \mathbf 1_{(0, 1)}
                    \pbraces
                    {
                        \frac{x + z^2}{4 y}
                    }
                }{x}
                \frac{1}{4 y}
            }{z}
        }{y}
    \end{align*}

    Now let $0 < y < 1$ and $0 < z < 1$.

    \begin{align*}
        &
        0 < \frac{x + z^2}{4 y} < 1, \quad x < 0 \\
        & \iff
        -z^2 < x < \min \Bbraces{4 y - z^2, 0}
        =
        \mathbf 1_{(-\infty, 0)}(4 y - z^2) (4 y - z^2)
        \stackrel{!}{=}
        \mathbf 1_{(2 \sqrt y, \infty)}(z) (4 y - z^2)
    \end{align*}

    For \enquote{!} we used

    \begin{align*}
        4 y - z^2 < 0
        \iff
        2 \sqrt y < z.
    \end{align*}

    We continue our main calculation.

    \begin{align*}
        & =
        \Int[-\infty][\infty]
        {
            \Int[-\infty][\infty]
            {
                \mathbf 1_{(0, 1)}(y)
                \mathbf 1_{(0, 1)}(z)
                (
                    \mathbf 1_{(2 \sqrt y, \infty)}(z) (4 y - z^2)
                    -
                    (-z^2)
                )
                \frac{1}{4 y}
            }{z}
        }{y} \\
        & =
        \Int[-\infty][\infty]
        {
            \Int[-\infty][\infty]
            {
                \mathbf 1_{(0, 1)}(y)
                \mathbf 1_{(0, 1)}(z)
                \mathbf 1_{(2 \sqrt y, \infty)}(z) (4 y - z^2)
            }{z}
            \frac{1}{4 y}
        }{y}
        +
        \Int[0][1]
        {
            \Int[0][1]
            {
                z^2
            }{z}
            \frac{1}{4 y}
        }{y}
    \end{align*}

    \begin{align*}
        0 < y < 1, \quad 0 < z < 1, \quad 2 \sqrt y
        \iff
        0 < y < \frac{1}{4}, \quad 2 \sqrt y < z < 1
    \end{align*}

    We continue our main calculation.

    \begin{align*}
        & =
        \Int[0][\frac{1}{4}]
        {
            \Int[2 \sqrt y][1]
            {
                4 y - z^2
            }{z}
            \frac{1}{4 y}
        }{y}
        +
        \Int[0][1]
        {
            \frac{1}{3}
            \frac{1}{4 y}
        }{y} \\
        & =
        \Int[0][\frac{1}{4}]
        {
            \Int[2 \sqrt y][1]
            {
                \frac{4 y}{4 y}
            }{z}
            -
            \Int[2 \sqrt y][1]
            {
                z^2
            }{z}
            \frac{1}{4 y}
        }{y}
        +
        \frac{1}{3 \cdot 4}
        \Int[0][1]
        {
            \frac{1}{y}
        }{y} \\
        & =
        \Int[0][\frac{1}{4}]
        {
            1 - 2 \sqrt y
            -
            \frac{1}{3 \cdot 4 y}
            (1 - 2 \sqrt y)^3
        }{y}
        +
        \frac{1}{12}
        \Int[0][1]
        {
            \frac{1}{y}
        }{y} \\
        & =
        \Int[0][\frac{1}{4}]{}{y}
        -
        2
        \Int[0][\frac{1}{4}]
        {
            \sqrt y
        }{y}
        -
        \frac{1}{12}
        \Int[0][\frac{1}{4}]
        {
            \frac{1}{4}
        }{y}
        +
        \frac{2^3}{12}
        \Int[0][\frac{1}{4}]
        {
            \frac{\sqrt y^3}{y}
        }{y}
        +
        \frac{1}{12}
        \Int[0][1]
        {
            \frac{1}{y}
        }{y} \\
        & =
        \frac{1}{4}
        -
        2 \frac{2}{3} y^{3/2} \Big |_{y=0}^\frac{1}{4}
        +
        \frac{1}{12}
        \Int[\frac{1}{4}][1]
        {
            \frac{1}{y}
        }{y}
        +
        \frac{8}{12}
        \Int[0][\frac{1}{4}]
        {
            \sqrt y
        }{y} \\
        & =
        \frac{1}{4}
        -
        \frac{4}{3}
        \pbraces{\frac{1}{4}}^{3 / 2}
        +
        \frac{1}{12}
        \ln y \Big |_{y = \frac{1}{4}}^1
        +
        \frac{2}{3}
        \frac{2}{3}
        y^{3/2} \Big |_{y=0}^\frac{1}{4} \\
        & =
        \frac{1}{4}
        -
        \frac{4}{3}
        \frac{1}{2^3}
        -
        \frac{1}{12}
        \ln \frac{1}{4}
        +
        \frac{4}{9}
        \pbraces{\frac{1}{4}}^{3/2} \\
        & =
        \frac{1}{4}
        -
        \frac{1}{6}
        +
        \frac{1}{12}
        \ln 4
        +
        \frac{4}{9}
        \frac{1}{2^3} \\
        & =
        \frac{1}{4}
        -
        \frac{3}{18}
        +
        \frac{1}{12}
        \ln 4
        +
        \frac{1}{18} \\
        & =
        \frac{1}{4}
        -
        \frac{1}{9}
        +
        \frac{1}{12}
        \ln 4 \\
        & =
        \frac{9 - 4 + 3 \ln 4}{36} \\
        & =
        \frac{5 + 3 \ln 4}{36} \\
        & \approx
        0.254413419
    \end{align*}
    
    \item \verb|runif(n)| returns an array of length $n$ with uniformally chosen values in $(0, 1)$.
    \verb|b^2>4*a*c| returns an array of length $n$, where the \texttt{i}-th entry ($\texttt{i} = 1, \dots, n$) is \texttt{TRUE} if \verb|b[i]^2 < 4*a[i]*c[i]| and \texttt{FALSE} otherwise.
    \verb|sum(b^2>4*a*c)| returns the number of \texttt{TRUE}-entries in the array \verb|b^2>4*a*c|.
    
    Let $X_i$ be $1$ if the \texttt{i}-th entry in the array \verb|b^2>4*a*c| ist \texttt{TRUE} and $0$ otherwise (i.e. if it is \texttt{FALSE}) and $\mu := E(X_i)$.
    Due to the Weak Law of Large Numbers \cite[Lecture 4, Slide 52]{EStat},
    
    \begin{align*}
        \bar X_n \xrightarrow[n \to \infty]{\text p} \mu.
    \end{align*}
    
    Now, \verb|sum(b^2>4*a*c)/n| corresponds to $\bar X_n$.
    Therefore, this code snippet approximates
    
    \begin{align*}
        E(X_i)
        =
        P(B^2 \geq 4 A C)
        =
        P(\exists x \in \R: A x^2 + B x + C = 0).
    \end{align*}

\end{enumerate}

\end{solution}

% --------------------------------------------------------------------------------
