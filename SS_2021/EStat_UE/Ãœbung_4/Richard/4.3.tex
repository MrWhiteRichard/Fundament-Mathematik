% --------------------------------------------------------------------------------

\begin{exercise}[Real roots]

Let $A$, $B$ and $C$ be independent random variables, uniformly distributed on $(0, 1)$.

\begin{enumerate}[label = (\alph*)]

    \item What is the probability that the quadratic equation $A x^2 + B x + C = 0$ has real roots?

    \item Consider the following code in R.
    What does it do and how is it related to your solution in part (a)?

    \lstset{style = stackexchangeR}
    \begin{lstlisting}[language = R]
        n=10000
        a=runif(n)
        b=runif(n)
        c=runif(n)
        sum(b^2>4*a*c)/n
    \end{lstlisting}

\end{enumerate}

\textit{Hint:}
In HW2/ex. 3(b) we showed that if $X$ has uniform $(0, 1)$ distribution then $-\log X$ has exponential distribution $\exp(1)$.
In an analogue way, one an prove that $-s \log X \sim \exp(\frac{1}{s})$ for any $s > 0$.
Also, in HW4/ex. 2(b) we proved that the sum of two independent exponential distributions is a gamma distribution.
Namely, if $X \sim \exp(1)$ and $Y \sim \exp(1)$ are independent then $X + Y = \operatorname{Gamma}(2, 1)$.

\end{exercise}

% --------------------------------------------------------------------------------

\begin{solution}

\phantom{}

\begin{enumerate}[label = (\alph*)]

    \item

    \begin{enumerate}[label = \arabic*.]

        \item Step (Calculate $f_{B^2}$):
        
        Recall the corollary from \cite[Lecture 3, Slide 42]{EStat}.

        \begin{align*}
            f_{B^2}(x)
            & =
            \frac{1}{2 \sqrt x}
            (
                f_b(\sqrt x) + f_B(-\sqrt x)
            ) \\
            & =
            \frac{1}{2 \sqrt x}
            (
                \mathbf 1_{(0, 1)}(\sqrt x) + \mathbf 1_{(0, 1)}(-\sqrt x)
            ) \\
            & =
            \frac{1}{2 \sqrt x}
            \mathbf 1_{(0, 1)}(x)
        \end{align*}

        \item Step (Calculate $f_{A C}$):
        
        Let

        \begin{align*}
            u & = ac, \\
            v & = a,
        \end{align*}

        with inverse mapping

        \begin{align*}
            a & = v, \\
            c & = \frac{u}{v}.
        \end{align*}

        The Jacobian is

        \begin{align*}
            J(u, v)
            & =
            \vbraces
            {
                \begin{pmatrix}
                    \pderivative[][a]{u} & \pderivative[][a]{v} \\
                    \pderivative[][c]{u} & \pderivative[][c]{v} \\
                \end{pmatrix}
            } \\
            & =
            \vbraces
            {
                \begin{pmatrix}
                    0            & 1               \\
                    \nfrac{1}{v} & -\nfrac{u}{v^2} \\
                \end{pmatrix}
            } \\
            & =
            \frac{1}{v}.
        \end{align*}

        By the change-of-variable formula \cite[Lecture 3, Slide 32]{EStat}, the joint density $U$ and $V$ is

        \begin{align*}
            f_{U, V}(u, v)
            & =
            f_{A, C} \pbraces{v, \frac{u}{v}} |J(u, v)| \\
            & =
            f_A(v) f_C \pbraces{\frac{u}{v}} \frac{1}{|v|} \\
            & =
            \mathbf 1_{(0, 1)}(v) \mathbf 1_{(0, 1)} \pbraces{\frac{u}{v}} \frac{1}{|v|} \\
            & \stackrel{!}{=}
            \mathbf 1_{(0, 1)}(u) \mathbf 1_{(u, 1)}(v) \frac{1}{|v|}
        \end{align*}

        by the independence of $A$ and $C$.
        For \Quote{!} we used

        \begin{align*}
            0 < v, \frac{u}{v} < 1
            \iff
            0 < u < v < 1.
        \end{align*}

        We find the marginal of $U$ by integrating out $V$.

        \begin{align*}
            f_U(u)
            & =
            \Int[-\infty][\infty]
            {
                f_{U, V}(u, v)
            }{v} \\
            & =
            \Int[-\infty][\infty]
            {
                \mathbf 1_{(0, 1)}(u) \mathbf 1_{(u, 1)}(v) \frac{1}{|v|}
            }{v} \\
            & =
            \Int[-\infty][\infty]
            {
                \frac{1}{v}
            }{v}
            \,
            \mathbf 1_{(0, 1)}(u) \\
            & =
            \ln v \Big |_{v=u}^n \mathbf 1_{(0, 1)}(u) \\
            & =
            -\ln u \, \mathbf 1_{(0, 1)}(u)
        \end{align*}

        \item Step (\Quote{The Big Cheese}):
        
        Recall the solution formula for quadratic equations \dots

        \begin{align*}
            a x^2 + b x + c = 0
            \iff
            \frac
            {
                -b \pm \sqrt{b^2 - 4 a c}
            }{
                2 a
            }
        \end{align*}

        \begin{align*}
            P(\exists x \in \R: A x^2 + B x + C = 0)
            & =
            P(B^2 \geq 4 A C) \\
            & =
            \Int[-\infty][\infty]
            {
                P(B^2 = x \geq 4 A C)
            }{x} \\
            & =
            \Int[-\infty][\infty]
            {
                f_{B^2}(x)
                P(4 A C \leq x)
            }{x} \\
            & =
            \Int[-\infty][\infty]
            {
                f_{B^2}(x)
                P \pbraces{A C \leq \frac{x}{4}}
            }{x} \\
            & =
            \Int[-\infty][\infty]
            {
                f_{B^2}(x)
                \Int[-\infty][\frac{x}{4}]
                {
                    f_{A C}(y)
                }{y}
            }{x} \\
            & =
            \Int[-\infty][\infty]
            {
                \frac{1}{2 \sqrt x} \mathbf 1_{(0, 1)}(x)
                \Int[-\infty][\frac{x}{4}]
                {
                    -\ln y \, \mathbf 1_{(0, 1)}(y)
                }{y}
            }{x} \\
            & =
            -\frac{1}{2}
            \Int[0][1]
            {
                \frac{1}{2 \sqrt x}
                \Int[0][\frac{x}{4}]
                {
                    \ln y
                }{y}
            }{x} \\
            & =
            -\frac{1}{2}
            \Int[0][1]
            {
                \frac{1}{2 \sqrt x}
                y (\ln y - 1) \Big |_{y=0}^\frac{x}{4}
            }{x} \\
            & \stackrel{!}{=}
            -\frac{1}{2}
            \Int[0][1]
            {
                \frac{1}{2 \sqrt x}
                \frac{x}{4} \pbraces{\ln \frac{x}{4} - 1}
            }{x} \\
            & =
            -\frac{1}{8}
            \Int[0][1]
            {
                \sqrt x
                (
                    \ln x - \ln 4 - 1
                )
            }{x} \\
            & =
            -\frac{1}{8}
            \pbraces
            {
                \Int[0][1]
                {
                    \sqrt x \ln x
                }{x}
                -
                (\ln 4 + 1)
                \Int[0][1]
                {
                    \sqrt x
                }{x}
            } \\
            & =
            -\frac{1}{8}
            \pbraces
            {
                \frac{2}{9}
                x^\frac{3}{2}
                (
                    3 \ln x - 2
                )
                \Big |_{x=0}^1
                -
                (
                    \ln 4 + 1
                )
                \frac{2}{3}
                x^\frac{2}{3}
                \Big |_{x=0}^1
            } \\
            & =
            \stackrel{!!}{=}
            -\frac{1}{8}
            \pbraces
            {
                \frac{2}{9}
                (-2)
                -
                (
                    \ln 4 + 1
                )
                \frac{2}{3}
            } \\
            & =
            \frac{1}{8}
            \pbraces
            {
                \frac{2}{9}
                2
                +
                (
                    \ln 4 + 1
                )
                \frac{6}{9}
            } \\
            & =
            \frac
            {
                4 + 6 (\ln 4 + 1)
            }{
                72
            } \\
            & \approx
            0.254413419
        \end{align*}

        For \Quote{!} we used

        \begin{align*}
            \lim_{y \to 0}
                y \ln y
            & =
            \lim_{y \to 0}
                \frac{\ln}{1 / y} \\
            & \stackrel
            {
                \text{L'Hospital}
            }{=}
            \lim_{y \to 0}
                \frac{1 / y}{-1 / y^2} \\
            & =
            -\lim_{y \to 0}
                \frac{1}{1 / y} \\
            & =
            0.
        \end{align*}

        For \Quote{!!} we used

        \begin{align*}
            \lim_{x \to 0}
                x^\frac{3}{2} \ln x
            & =
            \lim_{x \to 0}
                \frac{\ln x}{1 / x^\frac{3}{2}} \\
            & \stackrel
            {
                \text{L'Hospital}
            }{=}
            \frac{2}{3}
            \lim_{x \to 0}
                \frac{1 / x}{1 / x^\frac{5}{2}} \\
            & =
            \frac{2}{3}
            \lim_{x \to 0}
                \frac{x^\frac{5}{2}}{x} \\
            & =
            0.
        \end{align*}

    \end{enumerate}

    \item \verb|runif(n)| returns an array of length $n$ with uniformally chosen values in $(0, 1)$.
    \verb|b^2>4*a*c| returns an array of length $n$, where the \texttt{i}-th entry ($\texttt{i} = 1, \dots, n$) is \texttt{TRUE} if \verb|b[i]^2 < 4*a[i]*c[i]| and \texttt{FALSE} otherwise.
    \verb|sum(b^2>4*a*c)| returns the number of \texttt{TRUE}-entries in the array \verb|b^2>4*a*c|.

    Let $X_i$ be $1$ if the \texttt{i}-th entry in the array \verb|b^2>4*a*c| ist \texttt{TRUE} and $0$ otherwise (i.e. if it is \texttt{FALSE}) and $\mu := E(X_i)$.
    Due to the Weak Law of Large Numbers \cite[Lecture 4, Slide 52]{EStat},

    \begin{align*}
        \bar X_n \xrightarrow[n \to \infty]{\text P} \mu.
    \end{align*}

    Now, \verb|sum(b^2>4*a*c)/n| corresponds to $\bar X_n$.
    Therefore, this code snippet approximates

    \begin{align*}
        E(X_i)
        =
        P(B^2 \geq 4 A C)
        =
        P(\exists x \in \R: A x^2 + B x + C = 0).
    \end{align*}

\end{enumerate}

\end{solution}

% --------------------------------------------------------------------------------
