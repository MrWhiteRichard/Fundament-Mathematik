% -------------------------------------------------------------------------------- %

\begin{exercise}[Sum of two independent distributions]

\phantom{}

\begin{enumerate}[label = (\alph*)]

    \item Let $X \sim \mathcal P(\lambda_1)$ and $Y \sim \mathcal P(\lambda_2)$ be two independent Poisson random variables.
    Show that

    \begin{align*}
        X + Y \sim \mathcal P(\lambda_1 + \lambda_2).
    \end{align*}

    \item Let $U$ and $V$ be two independent random variables with exponential distribution $\exp(\lambda)$.
    Show that

    \begin{align*}
        U + V
        & \sim
        \operatorname{Gamma}(2, \lambda) \quad \text{and} \\
        \min \Bbraces{U, V}
        & \sim
        \exp(2 \lambda).
    \end{align*}

    \textit{Hint}:
    It is useful to use moment generating functions.
    Recall the pdf of a random variable $X \sim \operatorname{Gamma}(\alpha, \beta)$ is

    \begin{align*}
        f(x)
        =
        \begin{cases}
            \frac{x^{\alpha - 1} \exp*{-\frac{x}{\beta}}}{\Gamma(\alpha) \beta^\alpha}
            & x > 0 \\
            0,
            & x \leq 0
        \end{cases},
    \end{align*}
    
    and its mgf is of the form $\pbraces{\frac{1}{1 - \beta t}}^\alpha$ for $t < \frac{1}{\beta}$.
    Particularly, the pdf of a random variable $X \sim \exp(\lambda) = \operatorname{Gamma}(1, \frac{1}{\lambda})$ is of the form

    \begin{align*}
        f(x)
        =
        \begin{cases}
            \lambda \exp*{-\lambda x},
            & x \geq 0 \\
            0,
            & x \leq 0
        \end{cases}.
    \end{align*}

\end{enumerate}

\end{exercise}

% -------------------------------------------------------------------------------- %

\begin{solution}

\phantom{}

\begin{enumerate}

    \item Recall the theorem from \cite[Lecture 3, Slide 35]{EStat}.
    
    \begin{align*}
        f_{X + Y}(k)
        & =
        \sum_{l \in \N}
            f_x(k - l) f_Y(l) \\
        & =
        \frac{k!}{k!}
        \sum_{l \in \N}
            \frac{\lambda_1^{k - l}}{(k - l)!}
            \exp*{-\lambda_1}
            \frac{\lambda_2^l}{l!}
            \exp*{-\lambda_2} \\
        & =
        \exp*{-(\lambda_1 + \lambda_2)}
        \frac{1}{k!}
        \sum_{l \in \N}
            \binom{k}{l}
            \lambda_1^{k - l}
            \lambda_2^l \\
        & =
        \frac{(\lambda_1 + \lambda_2)^k}{k!}
        \exp*{-(\lambda_1 + \lambda_2)}
    \end{align*}

    \item

    \begin{align*}
        f_{U + V}(x)
        & =
        \Int[-\infty][\infty]
        {
            f_U(x - y) f_V(y)
        }{y} \\
        & =
        \Int[-\infty][\infty]
        {
            \lambda
            \exp*{-\lambda (x - y)}
            \mathbf 1_{(0, \infty)}(x - y)
            \lambda
            \exp*{-\lambda y}
            \mathbf 1_{(0, \infty)}(y)
        }{y} \\
        & =
        \lambda^2
        \exp*{-\lambda x}
        \Int[-\infty][\infty]
        {
            \mathbf 1_{(-\infty, x)}(y)
            \mathbf 1_{(0, \infty)}
        }{y} \\
        & =
        \frac{\lambda^2}{1!}
        \exp*{-\lambda x}
        \Int[-\infty][\infty]
        {
            \mathbf 1_{(0, x)}(y)
        }{y}
        \mathbf 1_{(0, \infty)}(x) \\
        & =
        \frac{\lambda^2}{\Gamma(2)}
        x^{2 - 1}
        \exp*{-\lambda x}
        \mathbf 1_{(0, \infty)}(x)
    \end{align*}

    \begin{align*}
        P(\min \Bbraces{U, V} < x)
        & =
        1 - P(\min \Bbraces{U, V} \geq x) \\
        & =
        1 - P(U \geq x, V \geq x) \\
        & =
        1 - P(U \geq x) P(V \geq x) \\
        & =
        1 - \pbraces{\Int[x][\infty]{\lambda \exp*{-\lambda y} \mathbf 1_{0, \infty}(y)}{y}}^2 \\
        & =
        1 - \lambda^2 \pbraces{\Int[\max \Bbraces{0, x}][\infty]{\exp*{-\lambda y}}{y}}^2 \\
        & =
        1 - \lambda^2 \pbraces{\frac{1}{-\lambda} \exp*{-\lambda y} \Big |_{y = \max \Bbraces{0, x}}^\infty}^2 \\
        & =
        1 - (-\exp*{-\lambda \max \Bbraces{0, x}})^2 \\
        & =
        \begin{cases}
            1 - \exp*{-2 \lambda x},
            & x > 0, \\
            0,
            & x \leq 0
        \end{cases}
    \end{align*}

\end{enumerate}



\end{solution}

% -------------------------------------------------------------------------------- %
