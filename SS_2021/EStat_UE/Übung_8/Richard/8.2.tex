% -------------------------------------------------------------------------------- %

\begin{exercise}[Sufficient statistic and point estimator statistics]

Let $X_1, \dots, X_n$ be a random sample from a population with pdf

\begin{align*}
    f(x \mid \theta)
    =
    \begin{cases}
        \frac{\theta}{x^2}, & \theta \leq x \\
        0,                  & \text{otherwise}
    \end{cases}
\end{align*}

with unknown $\theta > 0$.
Use the Factorization theorem to obtain a sufficient statistic for $\theta$.

\end{exercise}

% -------------------------------------------------------------------------------- %

\begin{solution}

We apply the Fisher-Neyman Factorization theorem \cite[lecture 8, slide 13]{EStat}, with

\begin{align*}
    g(y \mid \theta) = \theta^n \mathbf 1_{(0, \infty)}(y),
    \quad
    h(x) = \pbraces{\prod_{i=1}^n x_i}^{-2},
    \quad
    T(x) = x_{(1)},
\end{align*}

and get

\begin{align*}
    f_{X_1, \dots, X_n}(x \mid \theta)
    & =
    \prod_{i=1}^n f_{X_i}(x_i \mid \theta) \\
    & =
    \prod_{i=1}^n \frac{\theta}{x_i^2} \mathbf 1_{(\theta, \infty)}(x_i) \\
    & =
    \theta^n \pbraces{\prod_{i=1}^n x_i}^{-2} \mathbf 1_{(\theta, \infty)} \pbraces{\min_{i=1}^n x_i} \\
    & =
    g(T(x)) h(x).
\end{align*}

\end{solution}

% -------------------------------------------------------------------------------- %
