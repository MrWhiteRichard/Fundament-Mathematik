% -------------------------------------------------------------------------------- %

\begin{exercise}[\textbf{Box of candles}]

There are blue and red candles in a box. Probability that a randomly chosen candle
is blue is $1/(1 + 2a)$, for $a > 0$. Based on a sample of sample size $n$, find
the maximum likelihood estimator (MLE) $\hat{a}$ of the parameter $a$.

\end{exercise}

% -------------------------------------------------------------------------------- %

\begin{solution}

Let the random variable $X$ denote the number of blue candles chosen after $n$ attempts.
Then $X \sim \text{Bin}(n, 1/(1 + 2a))$ and the likelihood reads

\begin{align*}
  L_n(a) = \binom{n}{x}\frac{1}{(1 + 2a)^x}\frac{(2a)^{n-x}}{(1 + 2a)^{n-x}}.
\end{align*}

Dropping the terms independent of $a$ we can simplify the expression to

\begin{align*}
  L_n(a) = \frac{(2a)^{n-x}}{(1 + 2a)^{n}}.
\end{align*}

To further simplify calculations, we consider the log-likelihood function

\begin{align*}
  \ell_n(a) = (n-x)\log(2a) - n\log(1 + 2a)
\end{align*}

For $x = 0$ the maximum is never attained, as the function
increases monotonously to the limit of 0 as $a$ goes to infinity.
For $x = n$ the maximum is attained at $a = 0$. \\
Calculating the derivative, we obtain for $0 < x < n$

\begin{align*}
  \ell'_n(a) = \frac{n-x}{a} - \frac{2n}{1 + 2a}
\end{align*}


Now we find the zeros of the gradient for $x \neq 0$:

\begin{align*}
  0 &\stackrel{!}{=} \ell'_n(a) = \frac{n-x}{a} - \frac{2n}{1 + 2a} \\
  \iff \frac{2n}{1 + 2a} &= \frac{n-x}{a} \\
  \iff 2na &= (n-x)(1 + 2a) \\
  \iff 2xa &= n-x \\
  \iff a &= \frac{n-x}{2x}.
\end{align*}

We check the second derivative

\begin{align*}
  \ell^{\primeprime}(a) &= \frac{x-n}{a^2} + \frac{4n}{(1 + 2a)^2}
\end{align*}

and plug in $\hat{a} := \frac{n-x}{2x}$

\begin{align*}
  \ell^{\primeprime}(\hat{a}) &= \frac{4x^2(x-n)}{(n-x)^2} + 
  \frac{4n}{(1 + (n-x)/x)^2} \\
  &= \frac{4x^2(x-n)}{(n-x)^2} + \frac{4x^2n}{n^2} \\
  &= \frac{4x^2}{(x-n)} + \frac{4x^2}{n} \\
  &= \frac{4x^3}{n(x-n)} < 0.
\end{align*}
Therefore $\hat{a} = \frac{n-x}{2x}$ must be our maximum likelihood estimator.
Like expected $1/(1+2\hat{a}) = \frac{x}{n}$.

\end{solution}

% -------------------------------------------------------------------------------- %
