% -------------------------------------------------------------------------------- %

\begin{exercise}[Method of moment estimator]

Let $X_1, \dots, X_n$ be a random sample from a population with pdf

\begin{align*}
    f(x)
    =
    \begin{cases}
        \frac{\theta x^{\theta - 1}}{3^\theta},
        & 0 < x < 3 \\
        0,
        & \text{otherwise}
    \end{cases}
\end{align*}

where $\theta \in \R^+$ is unknown parameter.

\begin{enumerate}[label = (\alph*)]
    \item Show that the method of moments estimator for $\theta$ is $T_n = \frac{\hat X}{3 - \hat X}$.
    \item Find the limiting distribution of $\frac{T_n - \theta}{\frac{1}{\sqrt n}}$ as $n \to \infty$.
\end{enumerate}

\end{exercise}

% -------------------------------------------------------------------------------- %

\begin{solution}

\phantom{}

\begin{enumerate}[label = (\alph*)]

    \item We first calculate the first moment.
    
    \begin{align*}
        \E X_1
        & =
        \Int[-\infty][\infty]
        {
            x f(x)
        }{x} \\
        & =
        \Int[0][3]
        {
            x \frac{\theta x^{\theta - 1}}{3^\theta}
        }{x} \\
        & =
        \frac{\theta}{3^\theta}
        \Int[0][3]
        {
            x^\theta
        }{x} \\
        & =
        \frac{\theta}{3^\theta}
        \frac{1}{\theta + 1}
        x^{\theta + 1} \Big |_{x=0}^3 \\
        & =
        \frac{\theta}{3^\theta}
        \frac{1}{\theta + 1}
        3^{\theta + 1} \\
        & =
        \frac{3 \theta}{\theta + 1}
    \end{align*}

    According to \cite[lecture 6, slide 32]{EStat}, the method of moments estimator is found by solving

    \begin{align*}
        &
        \frac{3 \hat \theta}{\hat \theta + 1}
        =
        \E_{\hat \theta} X_1
        \stackrel{!}{=}
        \frac{1}{n}
        \sum_{i=1}^n
            X_i
        =
        \bar X \\
        & \iff
        3 \hat \theta = \bar X (\hat \theta + 1) = \bar X \hat \theta + \bar X \\
        & \iff
        \bar X = 3 \hat \theta - \bar X \hat \theta = (3 - \bar X) \hat \theta \\
        & \iff
        \frac{\bar X}{3 - \bar X} = \hat \theta =: T_n.
    \end{align*}

    \item We want to apply the Delta Method \cite[lecture 5, slide 10]{EStat} to
    
    \begin{align*}
        &
        g(x) := \frac{x}{3 - x} \stackrel{!}{=} y \\
        & \iff
        x = y (3 - x) = 3 y - x y \\
        & \iff
        3 y = x + x y = (1 + \theta) y \\
        & \iff
        x = \frac{3 y }{1 + y} =: g^{-1}(y).
    \end{align*}

    Note, that

    \begin{align*}
        g(\bar X) = T_n,
        \quad
        g^{-1}(\theta) = \E X_1,
    \end{align*}

    and

    \begin{align*}
        g(x) = \frac{x}{3 - x} = -\frac{x - 3 + 3}{x - 3} = \frac{3}{3 - x} - 1,
        \quad
        g^\prime(x) = \frac{3}{(3 - x)^2}.
    \end{align*}

    The Central Limit Theorem \cite[lecture 4, slide 61]{EStat} yields

    \begin{align*}
        \sqrt n (\bar X - \E X_1)
        \xrightarrow[n \to \infty]{\text d}
        \mathcal N(0, \Var X_1).
    \end{align*}

    Furthermore, the Delta Method \cite[lecture 5, slide 10]{EStat} yields

    \begin{align*}
        \frac{T_n - \theta}{1 / \sqrt n}
        =
        \sqrt n (g(\bar X) - g(\E X_1))
        \xrightarrow[n \to \infty]{\text d}
        \mathcal N(0, \Var X_1 g^\prime(\E X_1)^2)
        =
        \mathcal N \pbraces{0, \frac{9 \Var X_1}{(3 - \E X_1)^4}}.
    \end{align*}

    For the sake of completness, one could further calculate

    \begin{align*}
        \E X_1^2
        & =
        \Int[-\infty][\infty]
        {
            x^2 f(x)
        }{x} \\
        & =
        \Int[0][3]
        {
            x^2 \frac{\theta x^{\theta - 1}}{3^\theta}
        }{x} \\
        & =
        \frac{\theta}{3^\theta}
        \Int[0][3]
        {
            x^{\theta + 1}
        }{x} \\
        & =
        \frac{\theta}{3^\theta}
        \frac{1}{\theta + 2}
        x^{\theta + 3} \Big |_{x=0}^3 \\
        & =
        \frac{\theta}{3^\theta}
        \frac{1}{\theta + 2}
        3^{\theta + 1} \\
        & =
        \frac{9 \theta}{\theta + 2},
    \end{align*}

    and thus

    \begin{align*}
        \Var X_1
        & =
        \E X_1^2 - \E^2 X_1 \\
        & =
        \frac{9 \theta}{\theta + 2}
        -
        \pbraces
        {
            \frac{3 \theta}{\theta + 1}
        }^2.
    \end{align*}

\end{enumerate}

\end{solution}

% -------------------------------------------------------------------------------- %
