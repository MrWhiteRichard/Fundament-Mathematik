% --------------------------------------------------------------------------------

\begin{exercise}[Continuous two-dimensional random variable]

The joint pdf of two random variables $X$ and $Y$ is defined by

\begin{align*}
    f(x, y)
    =
    \begin{cases}
        c (x + 2 y), & 0 < y < 1 ~\text{and}~ 0 < 2 \\
        0,           & \text{otherwise}
    \end{cases}.
\end{align*}

\begin{enumerate}[label = (\alph*)]
    \item Find the value of $c$ and the marginal distribution of $Y$.
    \item Find the joint cdf of $X$ and $Y$.
    \item Find the marginal distribution of $X$ and the pdf of $Z = \frac{9}{(X + 1)^2}$.
\end{enumerate}

\end{exercise}

% --------------------------------------------------------------------------------

\begin{solution}

\phantom{}

\begin{enumerate}[label = (\alph*)]

    \item

    \begin{multline*}
        1
        =
        \Int[\R^2]{f(x, y)}{(x, y)}
        =
        \Int[0][1]
        {
            \Int[0][1]
            {
                c (x + 2 y)
            }{x}
        }{y}
        =
        c
        \pbraces
        {
            \Int[0][2]{x}{x} \Int[0][1]{}{y}
            +
            2 \Int[0][2]{}{x} \Int[0][1]{y}{y}
        } \\
        =
        c
        \pbraces
        {
            \frac{2^2}{2} \cdot 1 + 2 \cdot 2 \cdot \frac{1}{2}
        }
        =
        4 c
    \end{multline*}

    \begin{align*}
        \implies c = \frac{1}{4}
    \end{align*}

    \begin{multline*}
        f_Y(y)
        =
        \Int[\R]{f(x, y)}{x}
        =
        \mathbf 1_{(0, 1)} \Int[0][2]{\frac{1}{4} (x + 2 y)}{x}
        =
        \mathbf 1_{(0, 1)}
        \frac{1}{4}
        \pbraces
        {
            \Int[0][2]{x}{x}
            +
            2 y \Int[0][2]{}{x}
        } \\
        =
        \mathbf 1_{(0, 1)}
        \frac{1}{4}
        \pbraces
        {
            \frac{2^2}{2}
            +
            2 y \cdot 2
        }
        =
        \mathbf 1_{(0, 1)}
        \pbraces{y + \frac{1}{2}}
    \end{multline*}

    \item

    \begin{align*}
        F_{X, Y}(x, y)
        & =
        \Int[-\infty][x]
        {
            \Int[-\infty][y]
            {
                f(\xi, \eta)
            }{\eta}
        }{\xi} \\
        & =
        \mathbf 1_{(0, \infty)^2}(x, y)
        \Int[0][\min \Bbraces{x, 2}]
        {
            \Int[0][\min \Bbraces{y, 1}]
            {
                \frac{1}{4}
                (\xi + 2 \eta)
            }{\eta}
        }{\xi} \\
        & =
        \mathbf 1_{(0, \infty)^2}(x, y)
        \frac{1}{4}
        \pbraces
        {
            \Int[0][\min \Bbraces{x, 2}]{\xi}{\xi}
            \Int[0][\min \Bbraces{y, 1}]{}{\eta}
            +
            2
            \Int[0][\min \Bbraces{x, 2}]{}{\xi}
            \Int[0][\min \Bbraces{y, 1}]{\eta}{\eta}
        } \\
        & =
        \mathbf 1_{(0, \infty)^2}(x, y)
        \frac{1}{4}
        \pbraces
        {
            \frac{\min \Bbraces{x, 2}^2}{2} \cdot y
            +
            2 \min \Bbraces{x, 2} \cdot \frac{\min \Bbraces{y, 1}^2}{2}
        } \\
        & =
        \mathbf 1_{(0, \infty)^2}
        \pbraces
        {
            \frac{\min \Bbraces{x, 2}^2 \min \Bbraces{y, 1}}{8}
            +
            \frac{\min \Bbraces{x, 2} \min \Bbraces{y, 1}^2}{4}
        }
    \end{align*}

    \item

    \begin{multline*}
        f_X(x)
        =
        \Int[\R]{f(x, y)}{y}
        =
        \mathbf 1_{(0, 2)}(x) \Int[0][1]{\frac{1}{4} (x + 2 y)}{y}
        =
        \mathbf 1_{(0, 2)}(x)
        \frac{1}{4}
        \pbraces
        {
            x \Int[0][1]{}{y}
            +
            2 \Int[0][1]{y}{y}
        } \\
        =
        \mathbf 1_{(0, 2)}(x) \frac{1}{4} \pbraces{x \cdot 1 + 2 \cdot \frac{1}{2}}
        =
        \mathbf 1_{(0, 2)}(x)
        \pbraces
        {
            \frac{x}{4}
            +
            \frac{1}{4}
        }
    \end{multline*}

    Recall the theorem from \cite[Lecture 3, Slide 38]{EStat}.

    \begin{align*}
        &
        g:
            (-\infty, -1) \cup (-1, \infty) \to (0, \infty), \,
            x \mapsto \frac{9}{(x + 1)^2} \stackrel{!}{=} y \\
        & \iff
        \frac{9}{y} = (x + 1)^2 \\
        & \iff
        \frac{3}{\sqrt y}
        =
        \begin{Bmatrix}
            +x + 1 \\
            -x + 1
        \end{Bmatrix}
        \iff
        \pm \pbraces{\frac{3}{\sqrt y} - 1} = x
    \end{align*}

    has two right inverses

    \begin{align*}
        h_+:
            (0, \infty) \to (-\infty, -1), \,
            y \mapsto \frac{3}{\sqrt y} - 1,
        \quad
        h_-:
            (0, \infty) \to (-1, \infty), \,
            y \mapsto -\frac{3}{\sqrt y} + 1
    \end{align*}

    with derivative(s)

    \begin{align*}
        h_\pm^\prime(y)
        =
        \mp \frac{1}{2} \frac{3}{y^{3 / 2}}
        =
        \mp \frac{3 / 2}{y^{3 / 2}}.
    \end{align*}

    \begin{align*}
        \implies
        f_Z(z)
        & =
        -
        f_X \pbraces{ \frac{3}{\sqrt z} - 1} \frac{3 / 2}{z^{3 / 2}}
        +
        f_X \pbraces{-\frac{3}{\sqrt z} + 1} \frac{3 / 2}{z^{3 / 2}} \\
        & \stackrel{!}{=}
        \pbraces
        {
            -
            \mathbf 1_{(1, 9)}(z)
            \pbraces
            {
                \frac{\frac{3}{\sqrt z} - 1}{4}
                +
                \frac{1}{4}
            }
            +
            \mathbf 1_{(9, \infty)}(z)
            \pbraces
            {
                \frac{-\frac{3}{\sqrt z} + 1}{4}
                +
                \frac{1}{4}
            }
        }
        \frac{3 / 2}{z^{3 / 2}} \\
        & =
        \pbraces
        {
            \mathbf 1_{(1, 9)}(z)
            \pbraces
            {
                \frac{-\frac{3}{\sqrt z} + 1}{4}
                -
                \frac{1}{4}
            }
            +
            \mathbf 1_{(9, \infty)}(z)
            \pbraces
            {
                \frac{-\frac{3}{\sqrt z} + 1}{4}
                +
                \frac{1}{4}
            }
        }
        \frac{3 / 2}{z^{3 / 2}} \\
        & =
        \pbraces
        {
            -
            \mathbf 1_{(1, 9)}(z)
            \frac{3 / 4}{\sqrt z}
            +
            \mathbf 1_{(9, \infty)}(z)
            \pbraces
            {
                -\frac{3 / 4}{\sqrt z}
                +
                \frac{1}{2}
            }
        }
        \frac{3 / 2}{z^{3 / 2}}
    \end{align*}

    For \enquote{$!$} we have used the following.

    \begin{align*}
        \frac{3}{\sqrt z} - 1 \in (0, 2)
        & \iff
        z \in (1, 9), \\
        -\frac{3}{\sqrt z} + 1 \in (0, 2)
        & \iff
        z \in (9, \infty)
    \end{align*}

\end{enumerate}

\end{solution}

% --------------------------------------------------------------------------------
