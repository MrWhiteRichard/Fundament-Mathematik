% --------------------------------------------------------------------------------

\begin{exercise}[\textbf{Rayleigh distribution}]

Let $X_1,\dots,X_n$ be a random sample with Rayleigh distribution

\begin{align*}
  f(x|\theta) = \begin{cases}
    \frac{x}{\theta^2}\exp\left(-\frac{x^2}{2\theta^2}\right), & x \geq 0 \\
    0, & x < 0
  \end{cases}
\end{align*}

where $\theta > 0$ is unknown.

\begin{enumerate}[label = (\alph*)]
  \item Find the method of moments estimator of $\theta$.
  \item Find the MLE of $\theta$ and its asymptotic variance.
\end{enumerate}

\textit{Hint:} Show that the first two moments are $\E[X] = \theta\sqrt{\pi/2}$
and $\E[X^2] = 2\theta^2$.
\end{exercise}

% --------------------------------------------------------------------------------

\begin{solution}

\phantom{}

\begin{enumerate}[label = (\alph*)]
  \item We use the fact that the second moment of a standard normal distribution is 1.
  \begin{align*}
    \E[X] &= \int_0^{\infty} \frac{x^2}{\theta^2}\exp\left(-\frac{x^2}{2\theta^2}\right) dx \\
    &= \frac{\theta}{2}\sqrt{2\pi}\int_\R u^2\frac{1}{\sqrt{2\pi}}\exp\left(-\frac{u^2}{2}\right) du \\
    &= \theta\sqrt{\pi/2}.
  \end{align*}

  For the method of moments we solve

  \begin{align*}
    \mu(\hat{\theta}) = \hat{\theta}\sqrt{\pi/2} \stackrel{!}{=} \bar{X}
    \iff \hat{\theta} = \sqrt{2/\pi}\bar{X}.
  \end{align*}

  \item We denote $x = (x_1,\dots,x_n)$. The joint distribution then reads
  
  \begin{align*}
    f_\theta(x) = \prod_{i=1}^n f(x_i|\theta) = 
    \prod_{i=1}^n\frac{x_i}{\theta^2}\exp\left(-\frac{x_i^2}{2\theta^2}\right)
  \end{align*}
  For the maximum likelihood estimation we consider the log-likelihood function
  (after dropping factors not containing $\theta$)

  \begin{align*}
    \ell(\theta|x) = \sum_{i=1}^n -2\log(\theta) - \frac{x_i^2}{2\theta^2}
    = -2n\log(\theta) - \frac{|x|^2}{2\theta^2}
  \end{align*}
  with derivative
  \begin{align}
    \ell'(\theta) = -\frac{2n}{\theta} + \frac{|x|^2}{\theta^3}.
  \end{align}
  Solving for zero yields
  \begin{align*}
    0 &\stackrel{!}{=} -\frac{2n}{\theta} + \frac{|x|^2}{\theta^3} \\
    \iff \frac{2n}{\theta} &= \frac{|x|^2}{\theta^3} \\
    \iff 2n\theta^2 &= |x|^2 \\
    \iff \theta &= \frac{|x|}{\sqrt{2n}} \\
  \end{align*}
  The second derivative reads
  \begin{align*}
    \ell^{\primeprime}(\theta) = \frac{2n}{\theta^2} - \frac{3|x|^2}{\theta^4}
  \end{align*}
  and plugging in $\hat{\theta} = \frac{|x|}{\sqrt{2n}}$ yields for $n \geq 1$

  \begin{align*}
    \ell^{\primeprime}(\hat{\theta}) = \frac{1 - 12n^2}{|x|^2} < 0.
  \end{align*}
  Therefore we have found the global maximizer of the log-likelihood function
  and our MLE reads $\hat{\theta}(X_1,\dots,X_n) = 
  \sqrt{\frac{\sum_{i=1}^n X_i^2}{2n}}$.
  For the asymptotic variance we calculate the second moment of the Rayleigh distribution.
  \begin{align*}
    \E[X^2] &= \int_0^{\infty} \frac{x^3}{\theta^2}\exp\left(-\frac{x^2}{2\theta^2}\right) dx \\
    &= \theta^2 \int_0^{\infty} u^3\exp\left(-\frac{u^2}{2}\right) du \\
    &= \theta^2\left( \left[-u^2\exp\left(-\frac{u^2}{2}\right)\right]_0^\infty
    + 2 \int_0^{\infty} u \exp\left(-\frac{u^2}{2}\right) du\right) \\
    &= 2\theta^2\left(\int_0^{\infty} u \exp\left(-\frac{u^2}{2}\right) du\right) \\
    &= 2\theta^2\left[-\exp\left(-\frac{u^2}{2}\right)\right]_0^\infty \\
    &= 2\theta^2.
  \end{align*}
  If we now define $\bar{Y_n} := \frac{1}{n}\sum_{i=1}^n X_i^2$, the
  Central Limit Theorem tells us
  \begin{align*}
    \sqrt{n}(\bar{Y_n} - 2\theta^2) \xrightarrow[d]{n \to \infty} 
    \mathcal{N}(0,\V(X_i^2)).
  \end{align*}

  Continuing with the Delta method ($g(x) = \sqrt{\frac{x}{2n}}$) we finally
  obtain with $\V(X_i) = 4\theta^4$

  \begin{align*}
    \hat{\theta} - \theta = g(\bar{Y_n}) - g(2\theta^2)
    \xrightarrow[d]{n \to \infty} g'(2\theta^2)\mathcal{N}(0,\V(X_i^2)/n)
    &= \frac{1}{4}\sqrt{\frac{1}{n\theta^2}}\mathcal{N}(0,\V(X_i^2)/n) \\
    &= \mathcal{N}\left(0,\frac{\V(X_i^2)}{16n\theta^2}\right)
    = \mathcal{N}\left(0,\frac{\theta^2}{4n}\right).
  \end{align*}

\end{enumerate}

\end{solution}

% --------------------------------------------------------------------------------
