% ---------------------------------------------------------------- %

\begin{exercise}[Comparing Two Populations 1]

A study of the differences in cognitive function between normal individuals and patients diagnosed with schizophrenia was published in the American Journal of Psychiatry (Apr. 2010).
The total time (in minutes) a subject spent on the Trail Making Test (a standard psychological test) was used as a measure of cognitive function.
The researchers theorize that the mean time on the Trail Making Test for schizophrenics will be larger than the corresponding mean for normal subjects.
The data for independent random samples of $41$ schizophrenics and $49$ normal individuals yielded the following results:

\begin{align*}
    \begin{array}{l|cc}
                                  & \text{Schizophrenia} & \text{Normal} \\ \hline
        \text{Sample size}        & 41                   & 49            \\
        \text{Standard deviation} & 62.24                & 16.34         \\
    \end{array}
\end{align*}

\begin{enumerate}[label = (\alph*)]
    
    \item Define the parameter of interest to the researchers.
    
    \item Set up the null and alternative hypothesis for testing the researchers’ theory.
    
    \item The researchers conducted the test, part (b), and reported a $p$-value of $.001$.
    What conclusions can you draw from this result?
    (Use $\alpha = 0.01$)
    
    \item Find a $99 \%$ confidence interval for the target parameter.
    Interpret the result.
    Does your conclusion agree with that of the previous part?

\end{enumerate}

\end{exercise}

% ---------------------------------------------------------------- %

\begin{solution}

\phantom{}

\end{solution}

% ---------------------------------------------------------------- %