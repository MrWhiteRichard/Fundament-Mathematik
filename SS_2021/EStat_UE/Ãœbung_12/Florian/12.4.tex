% -------------------------------------------------------------------------------- %

\begin{exercise}[\textbf{$\chi^2$-test for independence}]

100 students from major mathematics of three Viennese universities
were randomly chosen and asked which lecture, either a: calculus,
b: algebra, or c: probability, they enjoyed most.

The frequencies are given in the following table:

\begin{center}
    \begin{tabular}{c|c|c|c}
        & Uni A & Uni B & Uni C \\
        \hline
        calculus & 10 & 5 & 5 \\
        \hline
        algebra & 10 & 20 & 10 \\
        \hline
        probability & 20 & 5 & 
    \end{tabular}
\end{center}

Perform a $\chi^2$-test whether the preference for a lecture is independent
from the university, on a 5\% significance level.

\begin{enumerate}[label = (\alph*)]
    \item Only use the following table which gives the 95\%-quantile $q$
    of the $\chi^2$-distribution with $df$ degrees of freedom.

    \begin{center}
        \begin{tabular}{c||c|c|c|c|c|c|c|c|c}
            df & 1 & 2 & 3 & 4 & 5 & 6 & 7 & 8 & 9 \\
            \hline
            q & 3.84 & 5.99 & 7.81 & 9.49 & 11.07 & 12.59 & 14.07 & 15.51 & 16.92
        \end{tabular}
    \end{center}

    \item Solve the previous exercise using \texttt{R}.
\end{enumerate}

\end{exercise}

% -------------------------------------------------------------------------------- %

\begin{solution}

\phantom{}

\end{solution}

% -------------------------------------------------------------------------------- %
