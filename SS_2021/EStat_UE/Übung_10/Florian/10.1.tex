% -------------------------------------------------------------------------------- %

\begin{exercise}[\textbf{Exponential family}]

Show that the one-parameter exponential family has a monotone likelihood
ratio in a sufficient statistic $T(\textbf{X})$ if the natural parameter
$w(\theta)$ is a non-decreasing function in $\theta$.

\end{exercise}

% -------------------------------------------------------------------------------- %

\begin{solution}

A family of pdfs or pmfs is called an \textit{exponential family of distributions}
if it can be represented in the form

\begin{align*}
    f(x | \theta) = h(x)c(\theta)\exp\left(w(\theta) \cdot t(x)\right),
\end{align*}

where $h(x) \geq 0, c(\theta) \geq 0$.
Now let $\theta_1 < \theta_2$ and define $T(x) := t(x)$. Then we have

\begin{align*}
    \frac{f(T | \theta_2)}{f(T | \theta_1)}
    &= \frac{h(x)c(\theta_2)\exp\left(w(\theta_2)T\right)}{h(x)c(\theta_1)\exp\left(w(\theta_1)T\right)} \\
    &= \frac{c(\theta_2)}{c(\theta_1)}\exp\left([w(\theta_2) - w(\theta_1)]T\right).
\end{align*}

We calculate the derivative with respect to $T$:

\begin{align*}
    \frac{\mathrm{d}}{\mathrm{d} T}\left(\frac{f(T | \theta_2)}{f(T | \theta_1)}\right)
    &= \underbrace{[w(\theta_2) - w(\theta_1)]}_{\geq 0}
    \underbrace{\frac{c(\theta_2)}{c(\theta_1)}}_{\geq 0}
    \underbrace{\exp\left([w(\theta_2) - w(\theta_1)]T\right)}_{\geq 0} \geq 0.
\end{align*}

Therefore the likelihood ratio is non decreasing in $T(x)$ and it only remains to show that
$T(\textbf{X})$ constitutes indeed a sufficient statistic.

Since we can write

\begin{align*}
    f(x | \theta) = h(x) \cdot \underbrace{[c(\theta) \exp(w(\theta) \cdot T(x)))]}_{:= g(T(x) | \theta)},
\end{align*}

we see that $T(X)$ is in fact a sufficient statistic for $\theta$.

\end{solution}

% -------------------------------------------------------------------------------- %
