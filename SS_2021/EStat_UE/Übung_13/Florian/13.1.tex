% -------------------------------------------------------------------------------- %

\begin{exercise}[\textbf{Unemployment and a reduced workweek}]

In an effort to increase employment, France mandated in February 2000 that
all companies with 20 or more employees reduce the workweek to 35 hours.
The economic of the shortened workweek was analyzed in \textit{Economic Policy}
(2008). The researchers focused on several key variables such as hourly
wages, dual job holdings and level of unemployment. Assume that in the
year prior to the 35-hour workweek law, unemployment in France was 12\%.
Suppose that in a random sample of 500 French citizens (eligible workers)
taken several years after the law was enacted, 53 were unemployed.
Determine if the French unemployment rate dropped after the enactment
of the 35-hour workweek law.

\end{exercise}

% -------------------------------------------------------------------------------- %

\begin{solution}

We want to test the null hypothesis $H_0: \mu = 0.12$, where $\mu$
denotes the unemployment rate after the enactment of the 35-hour workweek law
against the alternative $H_1: \mu < 0.12$.

Let $X$ denote the random variable that counts the number of unemployed
people from a population of 300 samples. Because each individual person
has a chance of $\mu_0 = 0.12$ to be unemployed under the null we have
$X \sim_{H_0} \text{Bin}(500, 0.12)$.

Since $n \cdot p = \frac{500 \cdot 12}{100} = 60 > 15$
and $n \cdot (1 - p) = \frac{500 \cdot 88}{ 100} = 440 > 15$
we can use a normal approximation: $X \sim_{H_0} \mathcal{N}(60, 264/5)$.

Now we can use a regular one sided $z$-value test.
We calculate

\begin{align*}
    p-\text{value} = \P(X \leq 53) = \Phi\left(-7\sqrt{\frac{5}{264}}\right)
    \approx 0.1677
\end{align*}

At a standard significance level $\alpha = 0.05$ this means we cannot
reject the null that the unemployment rate stayed unchanged.
\end{solution}

% -------------------------------------------------------------------------------- %
