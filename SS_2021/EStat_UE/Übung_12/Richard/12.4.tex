% ---------------------------------------------------------------- %

\begin{exercise}[$\chi^2$-test for independence]

$100$ students from major mathematics of three Viennese universities were randomly chosen  asked which lecture, either
\begin{enumerate*}[label = \alph*:]
    \item calculus,
    \item algebra, or
    \item probability, they enjoyed most.
\end{enumerate*}
The frequencies are given in the following table:

\begin{align*}
    \begin{array}{c|ccc}
                           & \text{Uni} ~ A & \text{Uni} ~ B & \text{Uni} ~ C \\ \hline
        \text{calculus}    & 10             & 5              &  5             \\
        \text{algebra}     & 10             & 20             &  10            \\
        \text{probability} & 20             & 5              &                \\
    \end{array}
\end{align*}

Perform a $\chi^2$-test to test whether the preference for a lecture is independent from the university, on a $5 \%$ significance level.

\begin{enumerate}[label = (\alph*)]

    \item Only use the following table which gives the $95 \%$-quantile $q$ of the $\chi^2$-distribution with $df$ degrees of freedom.
    
    \begin{align*}
        \begin{array}{c||ccccccccc}
            df & 1    & 2    & 3    & 4    & 5     & 6     & 7     & 8     & 9     \\ \hline
            q  & 3.84 & 5.99 & 7.81 & 9.49 & 11.07 & 12.59 & 14.07 & 15.51 & 16.92 \\
        \end{array}
    \end{align*}

    \item Solve the previous exercise using \texttt R.

\end{enumerate}

\end{exercise}

% ---------------------------------------------------------------- %

\begin{solution}

\phantom{}

\end{solution}

% ---------------------------------------------------------------- %