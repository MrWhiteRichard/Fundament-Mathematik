% ---------------------------------------------------------------- %

\begin{exercise}[Missing Information]

An investigation of ethnic differences in reports of pain perception was presented at the annual meeting of the American Psychosomatic Society (Mar. 2001).
A sample of $55$ blacks and $159$ whites participated in the study.
Subjects rated (on a 13-point scale) the intensity and unpleasantness of pain felt when a bag of ice was placed on their foreheads for two minutes.
(Higher ratings correspond to higher pain intensity.)
A summary of the results is provided in the following table.

\begin{align*}
    \begin{array}{l|cc}
                                   & \text{Blacks} & \text{Whites} \\ \hline
        \text{Sample Size}         & 55            & 159           \\
        \text{Mean pain intensity} & 8.2           & 6.9           \\
    \end{array}
\end{align*}

\begin{enumerate}[label = (\alph*)]

    \item Why is it dangerous to draw a statistical inference from the summarized data?
    Explain.

    \item What values of the missing sample standard deviations would lead you to conclude (at $\alpha = 0.05$) that blacks, on average, have a higher pain intensity rating than whites?

    \item What values of the missing sample standard deviations would lead you to an inconclusive decision (at $\alpha = 0.05$) regarding whether blacks or whites have a higher mean intensity rating?

\end{enumerate}

\end{exercise}

% ---------------------------------------------------------------- %

\begin{solution}

\phantom{}

\begin{enumerate}[label = (\alph*)]

    \item It is dangerous to draw a statistical inference from the sumamrized data, becuase approimately $3/4$-ths of the samples were whites, while blacks represent only $1/4$-ths.
    Moreover, matters of skin color are inherently delicate.

    \item Our (one-tailed) test reads
    
    \begin{align*}
        H_0: \mu_1 - \mu_2 = 0
        \quad
        \textit{vs.}
        \quad
        H_1: \mu_2 - \mu_1 > 0,
    \end{align*}

    with test statistic

    \begin{align*}
        Z = \frac{\bar X_1 - \bar X_2}{\sqrt{S_1^2 / n_1 - S_2^2 / n_2}} \approx_{H_0} N(0, 1).
    \end{align*}

    We reject $H_0$, iff

    \begin{multline*}
        0.05
        =
        \alpha
        \stackrel{!}{>}
        \text{$p$-value}
        =
        P(Z > z \mid H_0)
        >
        P \pbraces{Z \geq \frac{\bar x_1 - \bar x_2}{\sqrt{s_1^2 / n_1 + s_2^2 / n_2}} \mid H_0} \\
        =
        1 - \Phi \pbraces{\frac{\bar x_1 - \bar x_2}{\sqrt{s_1^2 / n_1 + s_2^2 / n_2}}}
        =
        \Phi \pbraces{\frac{\bar x_2 - \bar x_1}{\sqrt{s_1^2 / n_1 + s_2^2 / n_2}}}
        =
        \Phi \pbraces{\frac{6.9 - 8.2}{\sqrt{s_1^2 / 55 + s_2^2 / 159}}},
    \end{multline*}

    iff

    \begin{align*}
        -1.64
        \approx
        \Phi^{-1}(0.05)
        >
        \frac{-1.3}{\sqrt{s_1^2 / 55 + s_2^2 / 159}}
    \end{align*}

    (approximately) iff

    \begin{align*}
        \sqrt{s_1^2 / 55 + s_2^2 / 159}
        <
        \frac{1.3}{1.64}.
    \end{align*}

    \item Our (two-tailed) test reads
    
    \begin{align*}
        H_0: \mu_1 - \mu_2 = 0
        \quad
        \textit{vs.}
        \quad
        H_1: \mu_2 - \mu_1 \neq 0,
    \end{align*}

    with test statistic

    \begin{align*}
        Z = \frac{\bar X_1 - \bar X_2}{\sqrt{S_1^2 / n_1 - S_2^2 / n_2}} \approx_{H_0} N(0, 1).
    \end{align*}

    We reject $H_0$, iff

    \begin{align*}
        0.05
        =
        \alpha
        \stackrel{!}{>}
        \text{$p$-value}
        =
        P(|Z| > |z| \mid H_0)
        =
        2 P(Z > |z| \mid H_0),
    \end{align*}

    iff

    \begin{multline*}
        0.025
        >
        P \pbraces{Z \geq \frac{\bar x_1 - \bar x_2}{\sqrt{s_1^2 / n_1 + s_2^2 / n_2}} \mid H_0}
        =
        1 - \Phi \pbraces{\frac{\bar x_1 - \bar x_2}{\sqrt{s_1^2 / n_1 + s_2^2 / n_2}}} \\
        =
        \Phi \pbraces{\frac{\bar x_2 - \bar x_1}{\sqrt{s_1^2 / n_1 + s_2^2 / n_2}}}
        =
        \Phi \pbraces{\frac{6.9 - 8.2}{\sqrt{s_1^2 / 55 + s_2^2 / 159}}},
    \end{multline*}

    iff

    \begin{align*}
        -1.96
        \approx
        \Phi^{-1}(0.025)
        >
        \frac{-1.3}{\sqrt{s_1^2 / 55 + s_2^2 / 159}}
    \end{align*}

    (approximately) iff

    \begin{align*}
        \sqrt{s_1^2 / 55 + s_2^2 / 159}
        <
        \frac{1.3}{1.96}.
    \end{align*}

\end{enumerate}

\end{solution}

% ---------------------------------------------------------------- %