% -------------------------------------------------------------------------------- %

\begin{exercise}[\textbf{Hypnosis}]

Some researchers claim that susceptibility to hypnosis can be acquired
or improved through training. To investigate this claim six subjects
were rated on a scale of 1-10 according to their initial susceptibility
to hypnosis and then given 4 weeks of training. Each subjects was rated
again after the training period. In the ratings below, higher numbers
represent greater susceptibility to hypnosis.

\begin{center}
    \begin{tabular}{c|c|c}
    \hline
    Subject & Before & After \\
    \hline
    1 & 10 & 18 \\
    2 & 16 & 19 \\
    3 & 7 & 11 \\
    4 & 4 & 3 \\
    5 & 7 & 5 \\
    6 & 2 & 3 \\
    \hline
    \end{tabular}
\end{center}

Specify and perform the appropriate hypothesis test. What potential issues
exist with this analysis?

\end{exercise}

% -------------------------------------------------------------------------------- %

\begin{solution}

Since we have a small sample size (6) of paired samples we use a $t$-test
with test statistic

\begin{align*}
    t = \frac{\bar{d}}{s_d/\sqrt{n}} \approx_{H_0} t(n - 1).
\end{align*}

Let $\mu_d = \mu_2 - \mu_1$ denote the mean of the population of differences.
Our null hypothesis reads $\mu_d = 0$.

We calculate the 95\%-confidence interval

\begin{align*}
    \bar{d} \pm t_{\alpha/2}(n-1) \frac{s_d}{\sqrt{n}}
    = 2.17 \pm 2.57 \frac{3.66}{\sqrt{6}}
    \approx 2.17 \pm 3.84.
\end{align*}

Since $0$ is contained within the range of the confidence interval we cannot
reject the null hypothesis at significance level $\alpha = 0.05$.
\end{solution}

% -------------------------------------------------------------------------------- %
