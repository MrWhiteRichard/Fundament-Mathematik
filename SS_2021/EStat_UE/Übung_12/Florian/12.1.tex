% -------------------------------------------------------------------------------- %

\begin{exercise}[\textbf{Comparing two populations 1}]

A Study of the differences in cognitive function between normal
individuals and patients diagnosed with schizophrenia was published
in the American Journal of Psychiatry (Apr. 2010). The total time
(in minutes) a subject spent on the Trail Making Test (a standard
psychological test) was used as a measure of cognitive function.
The researchers theorize that the mean time on the Trail Making Test
for schizophrenics will be larger than the corresponding mean for normal
subjects. The data for independent random samples of 41 schizophrenics
and 49 normal individuals yielded the following results:

\begin{center}
\begin{tabular}{c|cc}
    & Schizophrenia & Normal \\
    \hline
    Sample size & 41 & 49 \\
    Mean time & 104.23 & 62.24 \\
    Standard deviation & 62.24 & 16.34
\end{tabular}
\end{center}

\begin{enumerate}[label = (\alph*)]
    \item Define the parameter of interest to the researchers.
    \item Set up the null and alternative hypothesis for testing
    the researchers' theory.
    \item The researchers conducted the test, part (b), and reported a
    $p$-value of .001. What conclusions can you draw from this result?
    (Use $\alpha = 0.01$)
    \item Find a 99\% confidence interval for the target parameter.
    Interpret the result. Does your conclusion agree with that
    of the previous part?
\end{enumerate}

\end{exercise}

% -------------------------------------------------------------------------------- %

\begin{solution}

Since we work with large sample sizes ($n_1 = 41 \geq 30$ and $n_2 = 49 \geq 30$),
we can use a normal-based test for independent samples.

The parameter of interest are the sample means $\bar{X}_1 = 104.23, \bar{X}_2 = 62.24$,
as well as the sample standard deviations $s_{1} = 62.24, s_{2} = 16.34$.

Let $\mu_1, \mu_2$ denote the true means of the respective populations.
Our null hypothesis thus states that there is no difference in the means,
i.e. $\mu_1 - \mu_2 = 0$.
For the alternative, we use the one-sided hypothesis $\mu_1 - \mu_2 > 0$.

Our test statistic then reads

\begin{align*}
    Z = \frac{\bar{X}_1 - \bar{X}_2}{\sqrt{\frac{s_1^2}{n_1} + \frac{s_2^2}{n_2}}}
    \approx_{H_0} \mathcal{N}(0,1)
\end{align*}

Using $\alpha = 0.01$ as our significance level, we can safely reject the null, since $0.001 < \alpha$.


Plugging in the given values we obtain 

\begin{align*}
    \text{p-value} = \P\left(Z \geq \frac{41.99}{\sqrt{\frac{62.24^2}{41} + \frac{16.34^2}{49}}}\right)
    \approx 0.0000133 \neq .001 ?
\end{align*}

Furthermore, the 99\% confidence interval for our target parameter $\mu_1 - \mu_2$ reads

\begin{align*}
    41.99 \pm z_{\alpha/2} \sqrt{\frac{62.24^2}{41} + \frac{16.34^2}{49}}
    \approx 41.99 \pm 25.74957.
\end{align*}

Since $0$ is significantly lower than the lower bound of our confidence interval,
our conclusions agree with the previous part.
\end{solution}

% -------------------------------------------------------------------------------- %
