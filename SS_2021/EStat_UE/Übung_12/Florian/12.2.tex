% -------------------------------------------------------------------------------- %

\begin{exercise}[\textbf{Comparing two populations 2}]

Suppose you wish to compare a new method of teaching reading to slow
learners with the current standard method. You decide to base your
comparison on the results of a reading test given at the end of a learning
period of six months. Of a random sample of 22 slow learners, 10 are taught
by the new method and 12 are taught by the standard method.
All 22 children are taught by qualified instructors under similar
conditions for the designated six-month period. The results of the reading
test at the end of this period are given below.

\begin{enumerate}[label = (\alph*)]
    \item[] New Method: 80, 76, 70, 80, 66, 85, 79, 71, 81, 76.
    \item[] Standard Method: 79, 73, 72, 62, 76, 68, 70, 86, 75, 68, 73, 66.
    \item Use the data in the table to estimate the true mean difference
    between the test scores for the new method and the standard method.
    Use a 95\% confidence interval.
    \item Interpret the interval you found in the previous part.
    \item What assumptions must be made in order that the estimate be valid?
    Are they reasonably satisfied?
\end{enumerate}

\end{exercise}

% -------------------------------------------------------------------------------- %

\begin{solution}

We are given small sample sizes (10 and 12) of two independent populations
with sample means $\bar{X}_1 = 76.4$ and $\bar{X}_2 = 72.333$ and sample
variances $s_1^2 = 34.044$ and $s_2^2 = 40.242$.

The 95\%-confidence interval thus reads

\begin{align*}
    (76.4 - 72.333) \pm t_{\alpha/2}(\nu)\sqrt{\frac{s_1^2}{n_1} + \frac{s_2^2}{n_2}}
    \approx 4.067 \pm 5.427.
\end{align*}

with 

\begin{align*}
    \nu = \frac{\left(\frac{s_1^2}{n_1} +  \frac{s_2^2}{n_2}\right)^2}
            {\frac{(s_1^2/n_1)^2}{n_1 - 1} + \frac{(s_2^2/n_2)^2}{n_2 - 1}}
\end{align*}

Since $0$ is part of our confidence interval, we cannot reject the null
at significance level $\alpha = 0.05$.

Necessary assumptions are that the true mean and variance of the compared
populations need be finite, which seems to be satisfied.
Furthermore, we suppose that the population samples are normally distributed and independent.

If we cannot suppose a normal distribution, we can use the Wilcoxon Signed-Rank test.
For that purpose we first rank the observations:
\vspace{10pt}

\begin{center}
    
    \begin{tabular}{c|c||c|c}
        Standard Method & Rank & New Method & Rank \\
        \hline
        86 & 1 & 85 & 2 \\
        79 & 6.5 & 81 & 3 \\
        76 & 9 & 80 & 4.5 \\
        75 & 11 & 80 & 4.5 \\
        73 & 12.5 & 79 & 6.5 \\
        73 & 12.5 & 76 & 9 \\
        72 & 14 & 76 & 9 \\
        70 & 16.5 & 71 & 15 \\
        68 & 18.5 & 70 & 16.5 \\
        68 & 18.5 & 66 & 20.5 \\
        66 & 20.5 & & \\
        62 & 22 & &
    \end{tabular}

\end{center}

We define the random variable $W_n = \sum_{i=1}^{n_1+n_2}iZ_i$,
where $Z_i = 1$ iff the $i$-th ranked random variable is sampled from the $X$-population and $0$ otherwise.

Since we are dealing with large samples $n_1, n_2 \geq 0$, we can use a normal approximation to determine the rejection region:

\begin{align*}
    W_n \geq \frac{n_1(n+1)}{2} + 0.5 + z_{\alpha}\sqrt{\frac{n_1n_2(n+1)}{12}}
    \approx  140.4454.
\end{align*}

Plugging in the given data we obtain

\begin{align*}
    W_n = 1 + 6.5 + 9 + 11 + 12.5 + 12.5 + 14 + 16.5 + 18.5 + 18.5 + 20.5 + 22 
    = 162.5 > 140.4454,
\end{align*}

which means we can reject the null and conclude that the new method offers indeed an advantage.

\end{solution}

% -------------------------------------------------------------------------------- %
