% -------------------------------------------------------------------------------- %

\begin{exercise}[\textbf{Comparing two populations 2}]

Suppose you wish to compare a new method of teaching reading to slow
learners with the current standard method. You decide to base your
comparison on the results of a reading test given at the end of a learning
period of six months. Of a random sample of 22 slow learners, 10 are taught
by the new method and 12 are taught by the standard method.
All 22 children are taught by qualified instructors under similar
conditions for the designated six-month period. The results of the reading
test at the end of this period are given below.

\begin{align*}
    &\text{New Method:} 80, 76, 70, 80, 66, 85, 79, 71, 81, 76.
    &\text{Standard Method:} 79, 73, 72, 62, 76, 68, 70, 86, 75, 68, 73, 66.
\end{align*}

\begin{enumerate}[label = (\alph*)]
    \item Use the data in the table to estimate the true mean difference
    between the test scores for the new method and the standard method.
    Use a 95\% confidence interval.
    \item Interpret the interval you found in the previous part.
    \item What assumptions must be made in order that the estimate be valid?
    Are they reasonably satisfied?
\end{enumerate}

\end{exercise}

% -------------------------------------------------------------------------------- %

\begin{solution}

\phantom{}

\end{solution}

% -------------------------------------------------------------------------------- %
