\chapter{Graphen-Energie}

    In diesem Kapitel werden wir nur noch, so wie im zweiten Punkt von Definition \ref{def:graph}, ungewichtete ungerichtete Graphen ohne Mehrfach-Kanten oder Schleifen betrachten, d.h.

    \begin{align*}
        V & := V(G)                                            && \text{ist eine endliche Knoten-Menge,} \\
        E & := E(G) \subseteq \Bbraces{e \subseteq V: |e| = 2} && \text{ist eine (endliche) Kanten-Menge, und} \\
        G & := (V, E)                                          && \text{ist der Graph.}
    \end{align*}

    Außerdem benötigen wir nur noch das gewöhnliche charakteristische Polynom und Spektrum.
    Dementsprechend sind die Begriffe \enquote{Graph}, \enquote{charakteristisches Polynom} und \enquote{Spektrum} eingeschränkt zu verstehen.

    % -------------------------------------------------------------------------------------------------------------------------------- %

    \section{Chemische Graphen-Theorie und topologische Indizes}

        % Quelle: Introduction To Chemical Graph Theory, Seite ix
        Die \textit{Chemische Graphen-Theorie} beschäftigt sich mit der Analyse gewisser Moleküle als \textit{molekulare Graphen}.
        % Quelle: Introduction To Chemical Graph Theory, Seite 2
        Aus einem Molekül erhält man einen molekularen Graphen, indem man die Atome des Moleküls als Knoten und die Bindungen zwischen den Atomen als Kanten des molekularen Graphen auffasst.
        Dabei werden die Wasserstoff-Atome meist weggelassen (insbesondere auch in dieser Arbeit).
        Falls mehr als eine Bindung zwischen Atomen auftritt, wird trotzdem keine Mehrfach-Kante eingeführt.
        Darüber hinaus, sind die Kanten ungerichtet.
        Zur Veranschaulichung soll die folgende Abbildung \ref{fig:Biphenylen} dienen.

        % Inspiration: Graph Energy, Seite 12
        \begin{figure}[h!]
            \centering
            \subfloat[Molekül (Struktur-Formel)]
            {
                \chemfig
                {
                    H -[:30] C_5
                    * 6
                    (
                        - C_4 (-[:270] H) = C_3
                        * 4
                        (
                            - C_{11}
                            * 6
                            (
                                = C_{10} (-[:270] H) - C_9 (-[:330] H) = C_8 (-[:30] H) - C_7 (-[:90] H) = \phantom{C_{12}}
                            )
                            - C_{12} - \phantom{C_{10}}
                        )
                        - C_2 = C_1 (-[:90] H) - C_6 (-[:150] H) =
                    )
                }
            }
            \hspace*{1cm}
            \subfloat[molekularer Graph]
            {
                \chemfig
                {
                    5
                    * 6
                    (
                        - 4 - 3
                        * 4
                        (
                            - 11
                            * 6
                            (
                                - 10 - 9 - 8 - 7 - \phantom{12}
                            )
                            - 12 - \phantom{10}
                        )
                        - 2 - 1 - 6 -
                    )
                }
            }
            \hspace{0mm}
            \caption[Lol]
            {
                % Quelle: https://de.wikipedia.org/wiki/Biphenylen
                Biphenylen zählt zur Klasse der polycyclischen aromatischen Kohlenwasserstoffe\protect\footnotemark.
                Die Kohlenstoff-Atome des Moleküls bzw. Knoten des molekularen Graphen sind von $1$ bis $12$ nummeriert.
            }
            \label{fig:Biphenylen}
        \end{figure}

        \footnotetext{Kohlenwasserstoffe sind Moleküle, die nur aus Kohlenstoff- und Wasserstoff-Atomen bestehen.}

        % Quelle: Introduction To Chemical Graph Theory, Seite ix
        Nun spielen \textit{topologische (chemische) Indizes} eine zentrale Rolle in der Chemischen Graphen-Theorie.
        Das sind gewisse numerische Größen von (molekularen) Graphen.
        Diese Indizes können verwendet werden, um Aussagen über gewisse Eigenschaften bestimmter chemischer Substanzen (Molekülen) zu treffen.
        Beim Untersuchen eines topologischen Index stellt man sich of folgende Fragen:

        \begin{itemize}
            \item \enquote{Gibt es alternative Darstellungen zur Definition des Indexes?}
            \item \enquote{Wie kann man den Index abschätzen?}
            \item \enquote{Wie kann man den Index berechnen?}
            \item \enquote{Kann man mehr über den Index erfahren, indem man die Klasse der zugrunde liegenden Graphen einschränkt?}
            \item \enquote{Welche Graphen minimieren bzw. maximieren den Index?}
        \end{itemize}

        Drei der prominentesten topologischen Indizes sind in folgender Definition gegeben.

        % Quelle: Introduction To Chemical Graph Theory
        \begin{definition} \label{def:graph_indices}

            Sei $G = (V, E)$ ein Graph mit Spektrum $\sigma$.
            Dann lauten der \textit{Wiener Index}
            % Siedepunkt von Alkanen

            \begin{align*}
                W(G) := \sum_{\Bbraces{u, v} \subseteq V(G)} d(u, v),
            \end{align*}

            der \textit{Randić Index} von $G$ mit $\alpha \neq 0$
            % Siedepunkt, Bildungsenthalpie, chromatographische Retentionszeit,

            \begin{align*}
                R_\alpha(G)
                :=
                \sum_{\Bbraces{u, v} \in E(G)} (\deg(u) \deg(v))^\alpha
            \end{align*}

            und die \textit{(Graphen-)Energie}

            \begin{align*}
                \mathcal E(G)
                :=
                \sum_{\lambda \in \sigma(G)} |\lambda|.
            \end{align*}

            Falls keine Verwechslungsgefahr besteht, lassen wir $G$ gelegentlich auch weg, schreiben also $W$, $R_\alpha$ bzw. $\mathcal E$.

        \end{definition}

        \begin{remark}

            1947 hat Wiener mit seinem Index eine Formel aufgestellt, die den Siedepunkt von Alkanen gut beschreibt (vgl. cite{Wiener}).
            Der Randić Index kann nicht nur für den Siedepunkt, sondern auch die Bildungsenthalpie\footnote{Die Bildungsenthalpie ist jene Energiemenge, die bei der Bildung einer Verbindung aus ihren Elementen freigesetzt (exotherme Reaktion) oder verbraucht (endotherme Reaktion) wird.} und chromatographische Retentionszeit\footnote{Die chromatographische Retentionszeit ist jene Zeit, die ein Analyt für das Passieren der Säule benötigt. Dies entspricht der Zeitdifferenz zwischen Injektion und Detektion.} verwendet werden.
            Auf diese beiden Indizes werden wir aber nicht weiter eingehen.
            Stattdessen widmen wir uns nun voll und ganz der Graphen-Energie.

        \end{remark}

    % -------------------------------------------------------------------------------------------------------------------------------- %

    % Quelle: Graph Energy, 2.1 Hückel Molecular Orbital Theory
    \section{Motivation}

        Erich Hückel entwickelte in den 30ern eine Methode (Hückel molecular orbital (HMO) model), um Lösungen der Schrödinger-Gleichung
        
        \begin{align} \label{eq:Schrödinger}
            \hat H \Psi = \mathcal E \Psi
        \end{align}
        
        zu approximieren, für Konjugierte Kohlenwasserstoffe.
        Sie beschreibt das Verhalten der Elektronen in einem Moleküle und deren Energie.
        Dabei ist $\hat H$ ein gewisser \textit{Hamiltonscher Operator}, $\Psi$ die \textit{Wellenfunktion} und $\mathcal E$ die \textit{Energie} des Systems.
        \eqref{eq:Schrödinger} beschreibt ein schwieriges Eigenwert-Problem.

        Sei $G = (V, E)$ der molekulare Graph eines Kohlenwasserstoff-Moleküls mit $n = |V|$ Knoten.
        Im HMO-Modell wird ein Ansatz für $\Psi$ als Linearkombination von gewissen $n$ orthogonalen Basis-Funktionen gewählt und mit dem Koeffizienten-Vektor der Linearkombination identifiziert.
        Daraus ergibt sich, durch hier nicht ausgeführte Rechnungen,

        \begin{align} \label{eq:Schrödinger_lite}
            \mathbf H \Psi = \mathcal E \Psi,
        \end{align}

        wobei $\alpha$ das \textit{Coulomb-Integral}, $\beta$ das \textit{Resonanz-Integral} und $\mathbf H := \alpha \mathbf I + \beta \mathbf A$ die \textit{Hamiltonsche Matrix} sind.
        ($\mathbf A$ ist natürlich eine Adjazenz-Matrix von $G$.)
        \eqref{eq:Schrödinger_lite} zwar kein unbedingt triviales, aber dennoch leichteres, Eigenwert-Problem als \eqref{eq:Schrödinger}.
        Für Biphenylen aus Abbildung \ref{fig:Biphenylen} erhalten beispielsweise

        \resizebox{\linewidth}{!}
        {
            \begin{minipage}{\linewidth}
                \begin{align*}
                    \mathbf H
                    =
                    \alpha
                    \pbraces
                    {
                        \begin{array}{cccccccccccc}
                            1 & 0 & 0 & 0 & 0 & 0 & 0 & 0 & 0 & 0 & 0 & 0 \\
                            0 & 1 & 0 & 0 & 0 & 0 & 0 & 0 & 0 & 0 & 0 & 0 \\
                            0 & 0 & 1 & 0 & 0 & 0 & 0 & 0 & 0 & 0 & 0 & 0 \\
                            0 & 0 & 0 & 1 & 0 & 0 & 0 & 0 & 0 & 0 & 0 & 0 \\
                            0 & 0 & 0 & 0 & 1 & 0 & 0 & 0 & 0 & 0 & 0 & 0 \\
                            0 & 0 & 0 & 0 & 0 & 1 & 0 & 0 & 0 & 0 & 0 & 0 \\
                            0 & 0 & 0 & 0 & 0 & 0 & 1 & 0 & 0 & 0 & 0 & 0 \\
                            0 & 0 & 0 & 0 & 0 & 0 & 0 & 1 & 0 & 0 & 0 & 0 \\
                            0 & 0 & 0 & 0 & 0 & 0 & 0 & 0 & 1 & 0 & 0 & 0 \\
                            0 & 0 & 0 & 0 & 0 & 0 & 0 & 0 & 0 & 1 & 0 & 0 \\
                            0 & 0 & 0 & 0 & 0 & 0 & 0 & 0 & 0 & 0 & 1 & 0 \\
                            0 & 0 & 0 & 0 & 0 & 0 & 0 & 0 & 0 & 0 & 0 & 1 \\
                        \end{array}
                    }
                    +
                    \beta
                    \pbraces
                    {
                        \begin{array}{cccccccccccc}
                            0 & 1 & 0 & 0 & 0 & 1 & 0 & 0 & 0 & 0 & 0 & 0 \\
                            1 & 0 & 1 & 0 & 0 & 0 & 0 & 0 & 0 & 0 & 0 & 1 \\
                            0 & 1 & 0 & 1 & 0 & 0 & 0 & 0 & 0 & 0 & 1 & 0 \\
                            0 & 0 & 1 & 0 & 1 & 0 & 0 & 0 & 0 & 0 & 0 & 0 \\
                            0 & 0 & 0 & 1 & 0 & 1 & 0 & 0 & 0 & 0 & 0 & 0 \\
                            1 & 0 & 0 & 0 & 1 & 0 & 0 & 0 & 0 & 0 & 0 & 0 \\
                            0 & 0 & 0 & 0 & 0 & 0 & 0 & 1 & 0 & 0 & 0 & 1 \\
                            0 & 0 & 0 & 0 & 0 & 0 & 1 & 0 & 1 & 0 & 0 & 0 \\
                            0 & 0 & 0 & 0 & 0 & 0 & 0 & 1 & 0 & 1 & 0 & 0 \\
                            0 & 0 & 0 & 0 & 0 & 0 & 0 & 0 & 1 & 0 & 1 & 0 \\
                            0 & 0 & 1 & 0 & 0 & 0 & 0 & 0 & 0 & 1 & 0 & 1 \\
                            0 & 1 & 0 & 0 & 0 & 0 & 1 & 0 & 0 & 0 & 1 & 0 \\
                        \end{array}
                    }.
                \end{align*}            
            \end{minipage}
        }

        \phantom{}

        Seien $\lambda_i$ die Eigenwerte von $G$ und bezeichne mit $\mathcal E_i = \alpha + \beta \lambda_i$ die \textit{Energie-Level} der $\pi$-Elektronen, für $i = 1, \dots, n$, dann gilt

        \begin{align*}
            \det(\mathbf H - \mathcal E_i \mathbf I)
            =
            \det
            (
                (\alpha \mathbf I + \beta \mathbf A)
                -
                (\alpha + \beta \lambda_i) \mathbf I
            )
            =
            \det(\beta \mathbf A - \beta \lambda_i \mathbf I)
            =
            \beta^n \det(\mathbf A - \lambda_i \mathbf I)
            =
            0.    
        \end{align*}

        Die Energie-Level sind also die Eigenwerte von $\mathbf H$.
        Die Eigenvektoren $\Psi_i$ von $G$ bzw. $\mathbf H$ heißen \textit{molekulare Orbitale} der $\pi$-Elektronen.
        Die Anzahl der Elektronen in $\Psi_i$, laut der Physik, ist $g_i = 0, 1, 2$.
        In konjugierten\footnote{In konjugierten Kohlenwasserstoffen haben alle Kohlenstoff-Atome genau $3$ Nachbarn (Kohlenstoff- oder Wasserstoff-Atome).} Kohlenwasserstoffen ist die Anzahl der $\pi$-Elektronen insgesamt

        \begin{align*}
            \sum_{i=1}^n g_i = n.
        \end{align*}

        Die HMO Approximation der gesamten $\pi$-Elektronen-Energie lautet damit

        \begin{align} \label{eq:Energy_nontrivial}
            \mathcal E_\pi
            :=
            \sum_{i=1}^n g_i \mathcal E_i
            =
            \alpha n + \beta \sum_{i=1}^n g_i \lambda_i.
        \end{align}

        Seien $\mathcal E_i$ nichtfallend geordnet.
        In Molekülen wird ein Zustand minimaler Energie angestrebt.
        In den meisten interessanten chemischen Anwendungen wird das erzielt durch

        \begin{align*}
            g_i
            =
            \begin{cases}
                2, & \text{wenn} ~ \lambda_i > 0, \\
                0, & \text{wenn} ~ \lambda_i < 0.
            \end{cases}
        \end{align*}

        Laut Theorem ref gilt TODOTODOTODOTODOTODOTODOTODOTODOTODOTODOTODOTODOTODOTODOTODOTODOTODOTODOTODOTODOTODOTODOTODOTODOTODO

        \begin{align*}
            0
            =
            \tr \mathbf A
            =
            \sum_{i=1}^n \lambda_i
            =
            \sum_{\lambda_i > 0} \lambda_i
            -
            \sum_{\lambda_i < 0} (-\lambda_i).
        \end{align*}

        Wir erhalten den dritten Teil unsere Definition \ref{def:graph_indices} der Energie des Graphen $G$, als den nicht-trivialen Teil von \eqref{eq:Energy_nontrivial}:

        \begin{align*}
            \mathcal E
            :=
            \mathcal E(G)
            :=
            \sum_{i=1}^n |\lambda_i|
            =
            2 \sum_{\lambda_i > 0} \lambda_i
            =
            \sum_{i=1}^n g_i \lambda_i.    
        \end{align*}

    % -------------------------------------------------------------------------------------------------------------------------------- %

    \section{Die Coulson Integral-Formel}

        \begin{theorem}

            Sei $G = (V, E)$ ein Graph mit $n = |V|$ Knoten und charakteristischem Polynom $\chi(\lambda) := \chi(G, \lambda)$.
            Dann ist dessen Energie

            \begin{align}
                \mathcal E(G)
                & =
                \frac{1}{\pi}
                \int_0^\infty
                    \frac{1}{x^2}
                    \ln \pbraces{x^{2 n} \chi(i/x) \chi(-i/x)}
                    ~ \mathrm d x
                \label{eq:Coulson_1} \\
                & =
                \frac{1}{\pi}
                \int_0^\infty
                    \frac{1}{x^2}
                    \pbraces
                    {
                        n \ln x
                        +
                        \ln |\chi(i/x)|
                    }
                    ~ \mathrm d x
                \label{eq:Coulson_2} \\
                & =
                \frac{1}{\pi}
                \int_{-\infty}^{+\infty}
                    n - i x \frac{\chi^\prime(i x)}{\chi(i x)}
                    ~ \mathrm d x
                \label{eq:Coulson_3} \\
                & =
                \frac{1}{\pi}
                \int_{-\infty}^{+\infty}
                    n - x \frac{\mathrm d}{\mathrm d \lambda} \ln \chi(i x)
                    ~ \mathrm d x.
                \label{eq:Coulson_4}
            \end{align}

        \end{theorem}
            
        \begin{proof}[Beweis-Idee]

            Wir verwenden zunächst die Substitution $u = |a| x$, um zu zeigen, dass für alle $a \in \R$ gilt

            \begin{align} \label{eq:lemma}
                \int_0^\infty
                    \frac{1}{x^2}
                    \ln(1 + a^2 x^2)
                    ~ \mathrm d x
                =
                \pi |a|.
            \end{align}

            Wir stellen $\chi(\lambda)$ als Linearfaktor-Produkt dar.
            Daraus und \eqref{eq:lemma} folgt \eqref{eq:Coulson_1}.
            Daraus folgt wiederum \eqref{eq:Coulson_2}, wenn man beachtet, dass

            \begin{align*}
                \chi(-i/x) = \overline{\chi(i/x)}.
            \end{align*}

            Durch partielles Integrieren von \eqref{eq:Coulson_1} und der Substitution $u = -v$, erhalten wir \eqref{eq:Coulson_3}.
            Daraus folgt unmittelbar \eqref{eq:Coulson_4}

        \end{proof}

        \begin{theorem}

            Sei $G = (V, E)$ ein Graph mit $n = |V|$ Knoten und $m = |E|$ Kanten.
            Dann ist dessen Energie

            \begin{align*}
                \mathcal E(G)
                \leq
                \sqrt{2 n m}.                
            \end{align*}

        \end{theorem}

        \begin{proof}

            Seien $\lambda_1, \dots, \lambda_n$ die Eigenwerte von $G$.
            Dann gilt, laut der Ungleichung des Arithmetischen und Quadratischen Mittels, dass

            \begin{align*}
                \mathcal E(G)^2
                =
                \pbraces
                {
                    \sum_{i=1}^n
                        |\lambda_i|
                }^2
                \leq
                n \sum_{i=1}^n \lambda_i^2
                =
                n \tr \mathbf A(G)^2
                =
                2 n m.                
            \end{align*}

        \end{proof}
