\section{Definitionen und Notationen}

\begin{definition}

    Als \textit{Graph} bezeichnen wir ein Paar $G = (V, E)$, das aus einer Menge $V$ von \textit{Ecken} und einer Menge $E$ von \textit{Kanten} besteht.
    $V$ sei in dieser Arbeit stets endlich, $G$ also ein \textit{endlicher} Graph.
    $E$ kann drei verschiedene Typen von Elementen enthalten.

    \begin{enumerate}

        \item $E$ kann aus Paaren $(v, w)$, mit $v, w \in V$, bestehen.
        $G$ nennen wir dann \textit{gerichtet}.
        Kanten der Form $(v, v)$, wobei $v \in V$, sind dabei \textit{gerichtete Schleifen}.

        \item $E$ darf auch aus Multi-Paar-Mengen $[v, w]$, für $v, w \in V$ bestehen.
        In diesem Fall heißt $G$ \textit{ungerichtet}.
        \textit{Ungerichtete Schleifen} seien jetzt echte Multi-Mengen $[v, v]$ mit $v \in V$.

        \item Falls $G$ ein ungerichteter Graph ohne Schleifen sein soll, könne die Kanten auch Paar-Mengen $\Bbraces{v, w}$, mit $v, w \in V$, sein.

    \end{enumerate}

    $E$ kann eine Multi-Menge sein.
    $G$ ist dann ein \textit{Multi-Graph}.
    Falls, darüber hinaus, $E$ eine echte Multi-Menge ist, schreiben wir $V(G)$ und $E(G)$ für dessen Ecken bzw. Kanten.

\end{definition}

\begin{definition}

    Sei $G$ ein (Multi-)Graph.
    Zwei Ecken $v, w \in V(G)$ heißen \textit{adjazent}, wenn wie durch eine Kante $e \in E(G)$ verbunden werden, i.Z.

    \begin{align*}
        v, w ~\text{adjazent}~
        :\iff
        \Exists e \in E(G):
            \begin{cases}
                e = [v, w],                 & \text{wenn}~ G ~\text{ungerichtet}, \\
                e = (v, w) \lor e = (w, v), & \text{wenn}~ G ~\text{gerichtet}.
            \end{cases}
    \end{align*}

    Eine Ecke $v \in V(G)$ und eine Kante $e \in E(G)$ heißen \textit{inzident}, wenn die Ecke auf der Kante liegt, bzw. in sie hinein oder aus ihr herausführt, i.Z.

    \begin{align*}
        v, e ~\text{inzident}~
        :\iff
        \begin{cases}
            v \in e,                                        & \text{wenn}~ G ~\text{ungerichtet}, \\
            \Exists w \in V(G): e = (v, w) \lor e = (w, v), & \text{wenn}~ G ~\text{gerichtet}.
        \end{cases}
    \end{align*}

\end{definition}

\begin{definition}

    Sei $G$ ein ungerichteter (Multi-)Graph.
    Die \textit{Adjazenz} zwischen zwei Ecken $v, w \in V(G)$ ist die Vielfachheit der Kanten $e \in E(G)$, die diese Verbindet, i.Z.

    \begin{align*}
        \operatorname{adj}(v, w)
        :=
        |\Bbraces{e \in E(G): e = [v, w]}|.
    \end{align*}

    Der \textit{Grad} eines Knotens $v \in V(G)$ ist die Anzahl aller Kanten $e \in E(G)$, die auf ihm liegen, i.Z.

    \begin{align*}
        \deg v
        :=
        |\Bbraces{e \in E(G): v \in e}|.
    \end{align*}

    Die \textit{Inzidenz} eines Knotens $v \in V(G)$ und einer Kante $e \in E(G)$ ist

    \begin{align*}
        \operatorname{inc}(v, e)
        :=
        \begin{cases}
            1, & \text{wenn}~ v \in e, \\
            0, & \text{sonst}.
        \end{cases}
    \end{align*}

    Sei $G$ ein gerichteter (Multi-)Graph.
    Die \textit{Adjazenz} zwischen zwei Ecken $v, w \in V(G)$ ist die Vielfachheit der Kanten $e \in E(G)$, die vom ersten zum zweiten Ecken führt, i.Z.

    \begin{align*}
        \operatorname{adj}(v, w)
        :=
        |\Bbraces{e \in E(G): e = (v, w)}|.
    \end{align*}

    Der \textit{Eingangs-Grad} eines Knotens $v \in V(G)$ ist die Anzahl aller Kanten $e \in E(G)$, die in ihn hineinführen, i.Z.

    \begin{align*}
        \deg^- v
        :=
        |\Bbraces{e \in E(G): \Exists w \in V(G): (w, v) = e}|.
    \end{align*}

    Der \textit{Ausgangs-Grad} oder auch \textit{Grad} eines Knotens $v \in V(G)$ ist die Anzahl aller Kanten $e \in E(G)$, die aus ihm herausführen, i.Z.

    \begin{align*}
        \deg v
        :=
        \deg^- v
        :=
        |\Bbraces{e \in E(G): \Exists w \in V(G): (v, w) = e}|.
    \end{align*}

    Die \textit{Inzidenz} eines Knotens $v \in V(G)$ und einer Kante $e \in E(G)$ ist

    \begin{align*}
        \operatorname{inc}(v, e)
        :=
        \begin{cases}
             1, & \text{wenn}~ \Exists w \in V(G): (v, w) = e, \\
            -1, & \text{wenn}~ \Exists w \in V(G): (w, v) = e, \\
             0, & \text{sonst}.
        \end{cases}
    \end{align*}

    Unabhängig davon, ob $G$ gerichtet oder ungerichtet ist, sei die \text{Seidel-Adjazenz} zweier Ecken $v, w \in V(G)$

    \begin{align*}
        \operatorname{adj}^\prime(v, w)
        =
        \begin{cases}
             1, & \text{wenn}~ v \neq w, ~\text{und}~ v, w ~\text{adjazent}, \\
            -1, & \text{wenn}~ v \neq w, ~\text{und}~ v, w ~\text{nicht adjazent}, \\
             0, & \text{sonst}.
        \end{cases}
    \end{align*}

\end{definition}

\begin{definition} \label{def:listing}

    Sei $G$ ein (Multi-)Graph.
    Bezeichne $n = |V(G|$ die Anzahl der Ecken und $m = |E(G)|$ die Anzahl der Kanten von $G$.
    Seien $f: \Bbraces{1, \dots, n} \to V(G)$ und $g: \Bbraces{1, \dots, m} \to E(G)$ Bijektionen und $v_i = f(i)$ sowie $e_k = g(k)$, für $i = 1, \dots, n$ bzw. $k = 1, \dots, m$.
    $(v_1, \dots, v_n)$ und $(e_1, \dots, e_m)$ bezeichnen wir als Auflistungen von $V(G)$ bzw. $E(G)$.

\end{definition}

\begin{definition}

    Wir schließen an Definition \ref{def:listing} direkt an und nennen

    \begin{itemize}
        \item $\mathbf A(G, f) := (\operatorname{adj}(v_i, v_j))_{i, j = 1}^n$ eine \textit{Adjazenz-Matrix},
        \item $\mathbf D(G, f) := \diag (\deg v_i)_{i=1}^n$ eine \textit{Grad-Matrix},
        \item $\mathbf L(G, f) := \mathbf D(G, f) - \mathbf A(G, f)$ eine \textit{Laplace-Matrix},
        \item $\mathbf B(G, f, g) := (\operatorname{inc}(v_i, e_k))_{i, k = 1}^{n, m}$ eine \textit{Inzidenz-Matrix}, und
        \item $\mathbf S(G, f) := (\operatorname{adj}^\prime(v_i, v_j))_{i, j = 1}^n$ eine \textit{Seidel-Matrix}
    \end{itemize}

    von $G$ unter der(n) Abzählung(en) $f$ (und $g$).

    Die durch die Abzählungen indizierten Familien all jener genannten \textit{Graphen-Matrizen}, bezeichnen wir mit $\mathcal A(G)$, $\mathcal D(G)$, $\mathcal L(G)$, $\mathcal B(G)$, bzw. $\mathcal S(G)$.
    Gelegentlich werden wir diese aber als Mengen auffassen und auf die Indizierung verzichten.

\end{definition}

Wenn wir verschiedene Graphen-Matrizen betrachten, un nichts Zusätzliches gesagt wird, sollen jene, die zu selben Graphen gehören, auch derselben Auflistung entspringen.

\begin{remark}

    Offenbar sind Graphen durch ihre Adjazenz-Matrix eindeutig bestimmt.
    Je nachdem, wie man $V(G$ aufzählt, bekommt man aber möglicherweise, zum selben Graphen, verschiedene Adjazenz-Matrizen.
    Zwei Adjazenz-Matrizen $\mathbf A_1 := \mathbf A(G, f_1) \in \mathcal A(G)$ und $\mathbf A_2 :=  \mathbf A(G, f_2) \in \mathcal A(G)$ sind allerdings immer permutations-ähnlich, vermöge jener Permutations-Matrix $\mathbf P_\pi$, die zur Permutation $\pi$ gehört, die die eine Aufzählung $f_1$ in die andere Aufzählung $f_2$ überführt, i.Z.

    \begin{align*}
        \mathbf A_1 = \mathbf P_\pi^{-1} \mathbf A_2 \mathbf P_\pi,
        \quad
        \pi = f_2 \circ f_1^{-1} \in S_n.
    \end{align*}

    Analoges gilt auch für die übrigen Graphen-Matrizen.

\end{remark}

\begin{definition}
    Sei $G$ ein (Multi-)Graph, $\mathbf A \in \mathcal A(G)$, $\mathbf D \in \mathcal D(G)$, und $\mathbf S \in \mathcal S(G)$ beliebige Adjazenz-, Grad-, bzw. Seidel-Matrizen.
    Wir definieren die \textit{charakteristischen Polynome}

    \begin{gather*}
        P_G(\lambda) := \det(\lambda \mathbf I - \mathbf A),
        \quad
        Q_G(\lambda) := \frac{1}{\det \mathbf D} \det(\lambda \mathbf D - \mathbf A), \\
        R_G(\lambda) := \det(\lambda \mathbf I - \mathbf D - \mathbf A),
        \quad
        \text{und}
        \quad
        S_G(\lambda) := \det(\lambda \mathbf I - \mathbf S).
    \end{gather*}

    Bei der Definition von $Q_G$ setzen wir voraus, dass $\det D \neq 0$.
    Die Nullstellen-Mengen $\sigma_P(G)$, $\sigma_Q(G)$, $\sigma_R(G)$, bzw. $\sigma_S(G)$ nennen wir \textit{(Graphen-)Spektren} von $G$.
    $P_G$ und $\sigma_P(G)$ bekommen die Namen \textit{gewönliches} charakteristisches Polynom bzw. Spektrum;
    $P_S$ und $\sigma_S(G)$ die Namen \textit{charakteristisches Seidel-Polynom} bzw. \textit{Seidel-Spektrum}.

\end{definition}
