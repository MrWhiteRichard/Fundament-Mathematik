\chapter{Charakteristische Polynome} \label{chap:characteristic_polynomials}

    % Quelle: Spectra of Graphs - Theory an Application: The coefficints of P_G(lambda), Seite 31
    \section{Die Koeffizienten von $\chi_P(\lambda)$}

        \begin{theorem}

            Sei $G = (V, E)$ ein (Multi-)Graph mit (gewöhnlichem) charakteristischem Polynom

            \begin{align*}
                \chi_P(\lambda)
                =
                \lambda^n + a_1 \lambda^{n-1} + \cdots + a_n.
            \end{align*}

            Dann gilt für alle $i = 1, \dots, n$, dass

            \begin{align} \label{eq:characteristic_coefficints_directed}
                a_i = \sum_{L \in \mathcal L_i} (-1)^{p(L)},
            \end{align}

            wenn $G$ gerichtet und

            \begin{align} \label{eq:characteristic_coefficints_undirected}
                a_i = \sum_{U \in \mathcal U_i} (-1)^{p(U)} \cdot 2^{c(U)},
            \end{align}

            wenn $G$ ungerichtet ist.

        \end{theorem}

        \begin{proof}

            Laut Lemma \ref{lem:characteristic_coefficints}, lassen sich die Koeffizienten $a_i$, für $i = 1, \dots, n$, des gewöhnlichen charakteristischen Polynoms als $(-1)^i$ mal der Summe der $i$-Hauptminoren darstellen.
            Also erhalten wir

            \begin{align*}
                a_i
                =
                (-1)^i
                \sum_{\substack{I \subseteq \Bbraces{1, \dots, n} \\ |I| = i}}
                    \det \mathbf A_I.
            \end{align*}

            Laut Bemerkung \ref{rem:induced_sub_graphs}, sind die $i$-Hauptminoren die Adjazenz-Matrizen der induzierten Teil-Graphen $G_I$ von $G$ mit $i = n - |I|$ Knoten.
            Laut Satz \ref{thm:det} und Korollar \ref{cor:det}, kann man die $i$-Hauptminoren als Summe über lineare Teil-Graphen bzw. einfachen Teil-Figuren von $G$ schreiben.
            Wir erhalten weiters

            \begin{align*}
                =
                (-1)^i
                \sum_{\substack{I \subseteq \Bbraces{1, \dots, n} \\ |I| = i}}
                    \sum_{L \in \mathcal L_i(G_I)}
                        (-1)^{p(L) + i}
                =
                \sum_{L \in \mathcal L_i(G)} (-1)^{p(L)},
            \end{align*}

            bzw.

            \begin{align*}
                =
                (-1)^i
                \sum_{\substack{I \subseteq \Bbraces{1, \dots, n} \\ |I| = i}}
                    \sum_{U \in \mathcal U_i(G_I)}
                        (-1)^{p(L) + i} \cdot 2^{c(U)}
                =
                \sum_{U \in \mathcal U_i(G)} (-1)^{p(U)} \cdot 2^{c(U)}.
            \end{align*}

        \end{proof}

    % Quelle: Spectra of Graphs - Theory an Application: The coefficints of C_G(lambda), Seite 37
    \section{Die Koeffizienten von $\chi_L(\lambda)$}

    % Quelle: Spectra of Graphs - Theory an Application: The coefficints of Q_G(lambda), Seite 40
    \section{Die Koeffizienten von $\chi_Q(\lambda)$}

    % Quelle: Spectra of Graphs - Theory an Application: Reduction procedures for calculating the characteristic polynomial, Seite 59
    \section{Die Berechnung der charakteristischen Polynome}

        \subsection{Implementierung}

            Wir wollen nun einige der eben angedeuteten Algorithmen in konkretem Python-Code umsetzen.