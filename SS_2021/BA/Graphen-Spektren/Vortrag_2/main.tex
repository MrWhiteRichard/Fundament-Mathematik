\documentclass[aspectratio = 169]{beamer}

\mode<presentation>
{
    \usetheme[reversetitle, notitle, noauthor]{Wien} 
}

\usepackage{url}
\usepackage{graphicx}
\usepackage{appendixnumberbeamer} 

\graphicspath{{./}{./Figures/}}    

% Hier befinden sich Pakete, die wir beinahe immer benutzen ...

\usepackage[utf8]{inputenc}

% Sprach-Paket:
\usepackage[ngerman]{babel}

% damit's nicht so, wie beim Grill aussieht:
\usepackage{fullpage}

% Mathematik:
\usepackage{amsmath, amssymb, amsfonts, amsthm}
\usepackage{bbm}
\usepackage{mathtools, mathdots}

% Makros mit mehereren Default-Argumenten:
\usepackage{twoopt}

% Anführungszeichen (Makro \Quote{}):
\usepackage{babel}

% if's für Makros:
\usepackage{xifthen}
\usepackage{etoolbox}

% tikz ist kein Zeichenprogramm (doch!):
\usepackage{tikz}

% bessere Aufzählungen:
\usepackage{enumitem}

% (bessere) Umgebung für Bilder:
\usepackage{graphicx, subfig, float}

% Umgebung für Code:
\usepackage{listings}

% Farben:
\usepackage{xcolor}

% Umgebung für "plain text":
\usepackage{verbatim}

% Umgebung für mehrerer Spalten:
\usepackage{multicol}

% "nette" Brüche
\usepackage{nicefrac}

% Spaltentypen verschiedener Dicke
\usepackage{tabularx}
\usepackage{makecell}

% Für Vektoren
\usepackage{esvect}

% (Web-)Links
\usepackage{hyperref}

% Zitieren & Literatur-Verzeichnis
\usepackage[style = authoryear]{biblatex}
\usepackage{csquotes}

% so ähnlich wie mathbb
%\usepackage{mathds}

% Keine Ahnung, was das macht ...
\usepackage{booktabs}
\usepackage{ngerman}
\usepackage{placeins}

% special letters:

\newcommand{\N}{\mathbb{N}}
\newcommand{\Z}{\mathbb{Z}}
\newcommand{\Q}{\mathbb{Q}}
\newcommand{\R}{\mathbb{R}}
\newcommand{\C}{\mathbb{C}}
\newcommand{\K}{\mathbb{K}}
\newcommand{\T}{\mathbb{T}}
\newcommand{\E}{\mathbb{E}}
\newcommand{\V}{\mathbb{V}}
\renewcommand{\S}{\mathbb{S}}
\renewcommand{\P}{\mathbb{P}}
\newcommand{\1}{\mathbbm{1}}

% quantors:

\newcommand{\Forall}{\forall \,}
\newcommand{\Exists}{\exists \,}
\newcommand{\ExistsOnlyOne}{\exists! \,}
\newcommand{\nExists}{\nexists \,}
\newcommand{\ForAlmostAll}{\forall^\infty \,}

% MISC symbols:

\newcommand{\landau}{{\scriptstyle \mathcal{O}}}
\newcommand{\Landau}{\mathcal{O}}


\newcommand{\eps}{\mathrm{eps}}

% graphics in a box:

\newcommandtwoopt
{\includegraphicsboxed}[3][][]
{
  \begin{figure}[!h]
    \begin{boxedin}
      \ifthenelse{\isempty{#1}}
      {
        \begin{center}
          \includegraphics[width = 0.75 \textwidth]{#3}
          \label{fig:#2}
        \end{center}
      }{
        \begin{center}
          \includegraphics[width = 0.75 \textwidth]{#3}
          \caption{#1}
          \label{fig:#2}
        \end{center}
      }
    \end{boxedin}
  \end{figure}
}

% braces:

\newcommand{\pbraces}[1]{{\left  ( #1 \right  )}}
\newcommand{\bbraces}[1]{{\left  [ #1 \right  ]}}
\newcommand{\Bbraces}[1]{{\left \{ #1 \right \}}}
\newcommand{\vbraces}[1]{{\left  | #1 \right  |}}
\newcommand{\Vbraces}[1]{{\left \| #1 \right \|}}
\newcommand{\abraces}[1]{{\left \langle #1 \right \rangle}}
\newcommand{\round}[1]{\bbraces{#1}}

\newcommand
{\floorbraces}[1]
{{\left \lfloor #1 \right \rfloor}}

\newcommand
{\ceilbraces} [1]
{{\left \lceil  #1 \right \rceil }}

% special functions:

\newcommand{\norm}  [2][]{\Vbraces{#2}_{#1}}
\newcommand{\diam}  [2][]{\mathrm{diam}_{#1} \: #2}
\newcommand{\diag}  [1]{\mathrm{diag} \: #1}
\newcommand{\dist}  [1]{\mathrm{dist} \: #1}
\newcommand{\mean}  [1]{\mathrm{mean} \: #1}
\newcommand{\erf}   [1]{\mathrm{erf} \: #1}
\newcommand{\id}    [1]{\mathrm{id} \: #1}
\newcommand{\sgn}   [1]{\mathrm{sgn} \: #1}
\newcommand{\supp}  [1]{\mathrm{supp} \: #1}
\newcommand{\arsinh}[1]{\mathrm{arsinh} \: #1}
\newcommand{\arcosh}[1]{\mathrm{arcosh} \: #1}
\newcommand{\artanh}[1]{\mathrm{artanh} \: #1}
\newcommand{\card}  [1]{\mathrm{card} \: #1}
\newcommand{\Span}  [1]{\mathrm{span} \: #1}
\newcommand{\Aut}   [1]{\mathrm{Aut} \: #1}
\newcommand{\End}   [1]{\mathrm{End} \: #1}
\newcommand{\ggT}   [1]{\mathrm{ggT} \: #1}
\newcommand{\kgV}   [1]{\mathrm{kgV} \: #1}
\newcommand{\ord}   [1]{\mathrm{ord} \: #1}
\newcommand{\grad}  [1]{\mathrm{grad} \: #1}
\newcommand{\ran}   [1]{\mathrm{ran} \: #1}
\newcommand{\graph} [1]{\mathrm{graph} \: #1}
\newcommand{\Inv}   [1]{\mathrm{Inv} \: #1}
\newcommand{\pv}    [1]{\mathrm{pv} \: #1}
\newcommand{\GL}    [1]{\mathrm{GL} \: #1}
\newcommand{\Mod}{\mathrm{Mod} \:}
\newcommand{\Th}{\mathrm{Th} \:}
\newcommand{\Char}{\mathrm{char}}
\newcommand{\At}{\mathrm{At}}
\newcommand{\Ob}{\mathrm{Ob}}
\newcommand{\Hom}{\mathrm{Hom}}
\newcommand{\orthogonal}[3][]{#2 ~\bot_{#1}~ #3}
\newcommand{\Rang}{\mathrm{Rang}}
\newcommand{\NIL}{\mathrm{NIL}}
\newcommand{\Res}{\mathrm{Res}}
\newcommand{\lxor}{\dot \lor}
\newcommand{\Div}{\mathrm{div} \:}
\newcommand{\meas}{\mathrm{meas} \:}

% fractions:

\newcommand{\Frac}[2]{\frac{1}{#1} \pbraces{#2}}
\newcommand{\nfrac}[2]{\nicefrac{#1}{#2}}

% derivatives & integrals:

\newcommandtwoopt
{\Int}[4][][]
{\int_{#1}^{#2} #3 ~\mathrm{d} #4}

\newcommandtwoopt
{\derivative}[3][][]
{
  \frac
  {\mathrm{d}^{#1} #2}
  {\mathrm{d} #3^{#1}}
}

\newcommandtwoopt
{\pderivative}[3][][]
{
  \frac
  {\partial^{#1} #2}
  {\partial #3^{#1}}
}

\newcommand
{\primeprime}
{{\prime \prime}}

\newcommand
{\primeprimeprime}
{{\prime \prime \prime}}

% Text:

\newcommand{\Quote}[1]{\glqq #1\grqq{}}
\newcommand{\Text}[1]{{\text{#1}}}
\newcommand{\fastueberall}{\text{f.ü.}}
\newcommand{\fastsicher}{\text{f.s.}}

\theoremstyle{definition}

% unnumbered theorems
\newtheorem*{theorem*}    {Satz}
\newtheorem*{lemma*}      {Lemma}
\newtheorem*{corollary*}  {Korollar}
\newtheorem*{proposition*}{Proposition}
\newtheorem*{remark*}     {Bemerkung}
\newtheorem*{definition*} {Definition}
\newtheorem*{example*}    {Beispiel}
\newtheorem*{problem*}    {Problem}
\newtheorem*{algorithmus*}    {Algorithmus}
\newtheorem*{algorithmen*}    {Weiterführende Algorithmen}
\newtheorem*{anwendungen*}    {Anwendungen}

\renewcommand{\figurename}{Abbildung}
\renewcommand{\tablename} {Tabelle}



% To avoid a warning from the hyperref package:
\pdfstringdefDisableCommands{ %
    \def \translate{} %
}

\title[Graphen-Energie]{Graphen-Energie}
\subtitle{Bachelorarbeit aus Diskreter Mathematik}
\author[R. Weiss]{Richard Weiss}
\institute[TU Wien]{TU Wien, Vienna, Austria}     
\date{7. Juni 2021}

\begin{document}

% ---------------------------------------------------------------- %

\begin{frame}
    \titlepage
\end{frame}            

% ---------------------------------------------------------------- %

\begin{frame}{Wiederholung (1 / 2)}

    \begin{definition*}
    
        \begin{align*}
            V & := V(G)                                                    & \text{endliche Knoten-Menge}, \\
            E & := E(G) \in \Bbraces{E^\prime \subseteq V: |E^\prime| = 2} & \text{Kanten-Menge}, \\
            G & := (V, E)                                                  & \text{Graph}
        \end{align*}

    \end{definition*}    

\end{frame}

% ---------------------------------------------------------------- %

\begin{frame}{Wiederholung (2 / 2)}

    \begin{definition*}
    
        \begin{gather*}
            V = \Bbraces{v_1, \dots, v_n},
            \quad
            a_{i, j}
            :=
            \begin{cases}
                1, & \Bbraces{v_i, v_j} \in E, \\
                0, & \text{sonst},
            \end{cases} \\
            \mathbf A := \mathbf A(G) = (a_{i, j})_{i,j=1}^n ~\text{Adjazenz-Matrix},
        \end{gather*}
    
        \begin{align*}
            \chi(\lambda)
            & :=
            \chi(G, \lambda)
            :=
            \det(\mathbf A - \lambda \mathbf I)
            & \text{charakteristisches Polynom von $G$}, \\
            \sigma
            & :=
            \sigma(G)
            :=
            \Bbraces{\lambda \in \C: \chi(\lambda) = 0}
            \subset
            \R
            & \text{Spektrum von $G$}
        \end{align*}

    \end{definition*}    

\end{frame}

% ---------------------------------------------------------------- %

\begin{frame}{Chemische Graphen-Theorie und Graphen-Indizes}

    \begin{block}{}

        \begin{itemize}

            \item Chemische Graphen: Moleküle als Graphen darstellen:
            
            \begin{itemize}
                \item Knoten  \dots Atome,
                \item Kanten  \dots (Atom-)Bindungen,
                \item Graphen \dots Moleküle
            \end{itemize}

            \item (topologische) Graphen-Indizes: Numerische Eigneschaften von Graphen, z.B.:

            \begin{itemize}
    
                \item Wiener Index: Siedepunkt von Alkanen,

                \[
                    W(G) := \sum_{\Bbraces{u, v} \subseteq V(G)} d(u, v)
                \]

                \item Randic Index: Siedepunkt, Bildungsenthalpie, chromatographische Retentionszeit,

                % Bildungsenthalpie, bezeichnet diejenige Energiemenge, die bei der Bildung einer Verbindung aus ihren Elementen freigesetzt (exotherme Reaktion) oder verbraucht (endotherme Reaktion) wird.
                % Retentionszeit: Zeit, die ein Analyt für das Passieren der Säule benötigt. Dies entspricht der Zeitdifferenz zwischen Injektion und Detektion.

                \[
                    R_\alpha(G) := \sum_{\Bbraces{u, v} \in E(G)} (\deg(u) \deg(v))^\alpha,
                    \quad
                    \alpha \neq 0
                \]

                \item Energie \dots

                % \item Energie: Energie der $\pi$-Elektronen-Bindungen,

                % \[
                %     \mathcal E := \mathcal E(G) := \sum_{\lambda \in \sigma(G)} |\lambda|
                % \]

            \end{itemize}

        \end{itemize}

    \end{block}

\end{frame}

% ---------------------------------------------------------------- %

\begin{frame}{Motivation (1 / 4)}

    \begin{block}{}

        Erich Hückel entwickelte in den 30ern eine Methode (Hückel molecular orbital (HMO) model), um Lösungen der Schrödinger-Gleichung \eqref{eq:Schrödinger} zu approximieren, für Konjugierte Kohlenwasserstoffe.
        Sie beschreibt das Verhalten der Elektronen in einem Moleküle und deren Energie.

        \begin{align} \label{eq:Schrödinger}
            \hat H \Psi = \mathcal E \Psi
        \end{align}

        \begin{itemize}
            \item $\Psi$ \dots Wellenfunktion
            \item $\hat H$ \dots Hamiltonscher Operator des Systems
            \item $\mathcal E$ \dots Energie des Systems
        \end{itemize}

        \eqref{eq:Schrödinger} zu lösen ist ein Schwieriges Eigenwert-Problem!

    \end{block}

\end{frame}

% ---------------------------------------------------------------- %

\begin{frame}{Motivation (2 / 4)}

    \begin{block}{}

        Sei $G = (V, E)$ ein Kohlenwasserstoff-Molekül mit $n = |V|$ Atomen.
        Wir approximieren $\Psi$ durch eine Linearkombination aus $n$ orthogonalen Basis-Funktionen, sowie $\hat H$ durch $\mathbf H := \alpha \mathbf I + \beta \mathbf A$, und erhalten

        \begin{align} \label{eq:Schrödinger_lite}
            \mathbf H \Psi = \mathcal E \Psi.
        \end{align}

        \eqref{eq:Schrödinger_lite} zu lösen ist ein leichteres Eigenwert-Problem!

        Seien $\lambda_i$ die Eigenwerte von $G$ und $\mathcal E_i = \alpha + \beta \lambda_i$ die Energie-Level der $\pi$-Elektronen, für $i = 1, \dots, n$, dann gilt

        \[
            \det(\mathbf H - \mathcal E_i \mathbf I)
            =
            \beta^n \det(\mathbf A - \lambda_i \mathbf I)
            =
            0.
        \]

    \end{block}

\end{frame}

% ---------------------------------------------------------------- %

\begin{frame}{Motivation (3 / 4)}

    \begin{block}{}

        Die Eigenvektoren $\Psi_i$ von $G$ bzw. $\mathbf H$ entsprechen molekularen Orbitalen der $\pi$-Elektronen.
        Die Anzahl der Elektronen in $\Psi_i$ ist, laut Physik, $g_i \in \Bbraces{0, 1, 2}$.
        In Kohlenwasserstoffen ist die Anzahl der $\pi$-Elektronen insgesamt

        \[
            \sum_{i=1}^n g_i = n.
        \]

        Die HMO Approximation der gesamten $\pi$-Elektronen-Energie lautet damit

        \[
            \mathcal E_\pi
            :=
            \sum_{i=1}^n g_i \mathcal E_i
            =
            \alpha n + \beta \sum_{i=1}^n g_i \lambda_i.
        \]

    \end{block}

\end{frame}

% ---------------------------------------------------------------- %

\begin{frame}{Motivation (4 / 4)}

    \begin{block}{}

        Seien $\mathcal E_i$ nichtfallend geordnet.
        In Molekülen wird ein Zustand minimaler Energie angestrebt.
        In den meisten interessanten Anwendungen ist dabei

        \[
            g_i
            =
            \begin{cases}
                2, & \lambda_i > 0, \\
                0, & \lambda_i < 0.
            \end{cases}
        \]

        Selbst, wenn $G$ beliebig gewesen wäre, würde dennoch gelten

        \[
            0
            =
            \tr \mathbf A
            =
            \sum_{i=1}^n \lambda
            =
            \sum_{\lambda_i > 0} \lambda_i
            -
            \sum_{\lambda_i < 0} (-\lambda_i).
        \]

        Wir erhalten unsere ultimative Definition der Energie des Graphen $G$

        \[
            \mathcal E
            :=
            \mathcal E(G)
            :=
            \sum_{\lambda \in \sigma} |\lambda|
            =
            2 \sum_{\lambda_i > 0} \lambda_i
            =
            \sum_{i=1}^n g_i \lambda_i.
        \]

    \end{block}

\end{frame}

% ---------------------------------------------------------------- %

\begin{frame}{Die Coulson-Integral-Formel (1 / 2)}
    
    \begin{theorem}

        Sei $G = (V, E)$ ein Graph mit $n = |V|$ Knoten und charakteristischem Polynom $\chi(\lambda) := \chi(G, \lambda)$, dann ist dessen Energie

        \begin{align}
            \mathcal E(G)
            & =
            \frac{1}{\pi}
            \int_0^\infty
                \frac{1}{x^2}
                \ln \pbraces{x^{2 n} \chi(i/x) \chi(-i/x)}
                ~ \mathrm d x
            \label{eq:Coulson_1} \\
            & =
            \frac{1}{\pi}
            \int_0^\infty
                \frac{1}{x^2}
                \pbraces
                {
                    n \ln x
                    +
                    \ln |\chi(i/x)|
                }
                ~ \mathrm d x
            \label{eq:Coulson_2} \\
            & =
            \frac{1}{\pi}
            \int_{-\infty}^{+\infty}
                n - i x \frac{\chi^\prime(i x)}{\chi(i x)}
                ~ \mathrm d x
            \label{eq:Coulson_3} \\
            & =
            \frac{1}{\pi}
            \int_{-\infty}^{+\infty}
                n - x \frac{\mathrm d}{\mathrm d \lambda} \ln \chi(i x)
                ~ \mathrm d x.
            \label{eq:Coulson_4}
        \end{align}

    \end{theorem}

\end{frame}

% ---------------------------------------------------------------- %

\begin{frame}{Die Coulson-Integral-Formel (2 / 2)}
    
    \begin{proof}[Beweis-Idee]

        Wir verwenden zunächst die Substitution $u = |a| x$, um zu zeigen, dass für alle $a \in \R$ gilt

        \begin{align} \label{eq:lemma}
            \int_0^\infty
                \frac{1}{x^2}
                \ln(1 + a^2 x^2)
                ~ \mathrm d x
            =
            \pi |a|.
        \end{align}

        Wir stellen $\chi(\lambda)$ als Linearfaktor-Produkt dar.
        Daraus und \eqref{eq:lemma} folgt \eqref{eq:Coulson_1}.
        Daraus folgt wiederum \eqref{eq:Coulson_2}, wenn man beachtet, dass

        \begin{align*}
            \chi(-i/x) = \overline{\chi(i/x)}.
        \end{align*}

        Durch partielles Integrieren von \eqref{eq:Coulson_1} und der Substitution $u = -v$, erhalten wir \eqref{eq:Coulson_3}.
        Daraus folgt unmittelbar \eqref{eq:Coulson_4}

    \end{proof}

\end{frame}

% ---------------------------------------------------------------- %

\begin{frame}{Das Sachs-Theorem}
    
    \begin{theorem}
        Sei $G$ ein gerichteter (Multi-)Graph und sein charakteristisches Polynom

        \begin{align*}
            \chi(\lambda)
            =
            \lambda^n + a_1 \lambda^{n-1} + \cdots + a_n,
        \end{align*}

        sowie $\mathcal L_i$, für $i \in \N$, die Menge der linearen (für alle Knoten sind Eingangs- und Ausgangs-Grad gleich) Teil-Graphen $L$ von $G$, mit $i = V(L)$ und $p(L)$ Komponenten.
        Dann gilt für alle $i = 1, \dots, n$, dass

        \begin{align*}
            a_i = \sum_{L \in \mathcal L_i} (-1)^{p(L)}.
        \end{align*}

    \end{theorem}

\end{frame}

% ---------------------------------------------------------------- %

\begin{frame}{Ausgewählte aktuelle Forschung}

    \begin{block}

        \begin{itemize}
            \item 17. July 2017 - ELSEVIER - What is the meaning of the graph energy after all?
            \item 30. January 2021 - ELSEVIER - A sharp lower bound of the spectral radius with application to the energy of a graph
            \item 31. March 2021 - Open journal of Discrete Applied Mathematics - Graph energy and nullity
        \end{itemize}
        
    \end{block}

\end{frame}

% ---------------------------------------------------------------- %

\end{document}
