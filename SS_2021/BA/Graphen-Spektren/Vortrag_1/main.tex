\documentclass[aspectratio=169]{beamer}



\mode<presentation>
{
 \usetheme[reversetitle,notitle,noauthor]{Wien} 
%    \usetheme[noauthor]{Wien} 
}             
    
\usepackage{url}
\usepackage{graphicx}
\graphicspath{{./}{./Figures/}}    

\usepackage{appendixnumberbeamer} 

% To avoid a warning from the hyperref package:
\pdfstringdefDisableCommands{%
    \def\translate{}%
}

% To make sure, that the footnote is placed above and outside the
% footline (but it only works for one footnote per frame):
% 
% \addtobeamertemplate{footnote}{}{\vspace{4ex}}

%%%%%%%%%%%%%%%%%%%%%%%%%%%%%%%%%%%%%%%%%%%%%%%%%%%%%%%%%%%%%%%%%%%%%%%%%%%%% 
%%%%%%%%%%%%%%%%%%%%%%%%%%%%%%%%%%%%%%%%%%%%%%%%%%%%%%%%%%%%%%%%%%%%%%%%%%%%%
\title[Graphen-Spektren]{Graphen-Spektren}


\subtitle{Bachelorarbeit aus Diskreter Mathematik}

\author[R. Weiss]{Richard Weiss}

\institute[TU Wien]{TU Wien, Vienna, Austria}
     
\date{12. April 2021}

% Hier befinden sich Pakete, die wir beinahe immer benutzen ...

\usepackage[utf8]{inputenc}

% Sprach-Paket:
\usepackage[ngerman]{babel}

% damit's nicht so, wie beim Grill aussieht:
\usepackage{fullpage}

% Mathematik:
\usepackage{amsmath, amssymb, amsfonts, amsthm}
\usepackage{bbm, mathrsfs, stmaryrd}
\usepackage{mathtools, mathdots}

% Makros mit mehereren Default-Argumenten:
\usepackage{twoopt}

% Anführungszeichen (Makro \Quote{}):
\usepackage{babel}

% if's für Makros:
\usepackage{xifthen}
\usepackage{etoolbox}

% tikz ist kein Zeichenprogramm (doch!):
\usepackage{tikz}

% bessere Aufzählungen:
\usepackage{enumitem}

% (bessere) Umgebung für Bilder:
\usepackage{graphicx, subfig, float}

% Umgebung für Code:
\usepackage{listings}

% Farben:
\usepackage{xcolor}

% Umgebung für "plain text":
\usepackage{verbatim}

% Umgebung für mehrerer Spalten:
\usepackage{multicol}

% "nette" Brüche
\usepackage{nicefrac}

% Spaltentypen verschiedener Dicke
\usepackage{tabularx}
\usepackage{makecell}

% Für Vektoren
\usepackage{esvect}

% (Web-)Links
\usepackage{hyperref}

% Zitieren & Literatur-Verzeichnis
\usepackage[style = authoryear]{biblatex}
\usepackage{csquotes}

% so ähnlich wie mathbb
%\usepackage{mathds}

% Keine Ahnung, was das macht ...
\usepackage{booktabs}
\usepackage{ngerman}
\usepackage{placeins}

% ---------------------------------------------------------------- %
% Praetorius' macros

\def \revision #1 {{\color{red} #1}}

% ---------------------------------------------------------------- %
% my macros

\DeclareMathOperator{\dom}{dom}
\DeclareMathOperator{\ran}{ran}
\DeclareMathOperator{\supp}{supp}

\newcommand{\pbraces}[1]{{\left  ( #1 \right  )}}
\newcommand{\bbraces}[1]{{\left  [ #1 \right  ]}}
\newcommand{\Bbraces}[1]{{\left \{ #1 \right \}}}
\newcommand{\vbraces}[1]{{\left  | #1 \right  |}}
\newcommand{\Vbraces}[1]{{\left \| #1 \right \|}}

\newcommand{\Forall}{\forall \,}
\newcommand{\Exists}{\exists \,}

\DeclareMathOperator{\diag}{diag}

% ---------------------------------------------------------------- %

\theoremstyle{definition}

% unnumbered theorems
\newtheorem*{theorem*}    {Satz}
\newtheorem*{lemma*}      {Lemma}
\newtheorem*{corollary*}  {Korollar}
\newtheorem*{proposition*}{Proposition}
\newtheorem*{remark*}     {Bemerkung}
\newtheorem*{definition*} {Definition}
\newtheorem*{example*}    {Beispiel}

\renewcommand{\figurename}{Abbildung}
\renewcommand{\tablename} {Tabelle}


\begin{document}

\begin{frame}
    \titlepage
\end{frame}            

%%%%%%%%%%%%%%%%%%%%%%%%%%%%%%%%%%%%%%%%%%%%%%%%%%%%%%%%%%%%%%%%%%%%%%%%%%%%% 
%%%%%%%%%%%%%%%%%%%%%%%%%%%%%%%%%%%%%%%%%%%%%%%%%%%%%%%%%%%%%%%%%%%%%%%%%%%%%
%%%%%%%%%%%%%%%%%%%%%%%%%%%%%%%%%%%%%%%%%%%%%%%%%%%%%%%%%%%%%%%%%%%%%%%%%%%%%

\begin{frame}{Definitionen (1 / 5)}

    Man erinnere sich an die Definitionen der LVA \enquote{Diskrete und geometrische Algorithmen} WS 2020-21.

    \begin{definition*}[Mehr Graphen]
        Ungerichtete und gerichtete Graphen mit \textbf{Schleifen} haben auch Multi-Mengen bzw. Paare mit zwei-mal demselben Element der Knoten-Menge als Kanten. \\
        Bei \textbf{Multi-Graphen} ist die Kanten-Menge eine Multi-Menge.
    \end{definition*}

    Wir werden nur endliche Graphen brauchen.

\end{frame}


%%%%%%%%%%%%%%%%%%%%%%%%%%%%%%%%%%%%%%%%%%%%%%%%%%%%%%%
\begin{frame}{Mehr Graphen: Beispiele}

    \begin{figure}[H]
        \centering
        \subfloat[Ungerichteter Graph ohne Mehrfachkanten]
        {
                \includegraphics[height = 0.6 \textheight]{120px-Graph_ungerichtet.svg.png}
        }
        \hspace{1cm}
        \subfloat[Gerichteter Graph ohne Mehrfachkanten]
        {
            \includegraphics[height = 0.6 \textheight]{120px-Graph_gerichtet.svg.png}
        }
        \hspace{1cm}
        \subfloat[Ungerichteter Graph mit Mehrfachkanten]
        {
            \includegraphics[height = 0.6 \textheight]{200px-Graph_ungerichtet_Mehrfachkanten.svg.png}
        }
        \hspace{1cm}
        \subfloat[Gerichteter Graph mit Mehrfachkanten]
        {
            \includegraphics[height = 0.6 \textheight]{Graph_gerichtet_Mehrfachkanten.svg.png}
        }
        \hspace{0mm}
        % \caption
        % {
        %     Quelle:
        %     \url{https://de.wikipedia.org/wiki/Graph_(Graphentheorie)}
        % }
        \label{fig:2.1}
    \end{figure}

\end{frame}

%%%%%%%%%%%%%%%%%%%%%%%%%%%%%%%%%%%%%%%%%%%%%%%%%%%%%%%
\begin{frame}{Definitionen (2 / 5)}
    
    \begin{definition*}[Graphen-Matrizen]
        Sei $G = (V, E)$ ein (un)gerichteter (Mulit-)Graph mit oder ohne Schleifen und $V = \Bbraces{v_1, \dots, v_n}$ sowie $E = \Bbraces{e_1, \dots, e_m}$.
        Für $i, j = 1, \dots, n$ und $k = 1, \dots, m$ sei $a_{i, j}$ die Anzahl der Kanten von $v_i$ nach $v_j$, $d_i$ der (Eingangs-)Grad (ungerichtete Schleifen zählen doppelt) von $v_i$ und die Inzidenz zwischen $v_i$ und $e_k$

        \begin{align*}
            b_{i, k}
            =
            \begin{cases}
                1, & v_i \in e_k, \\
                0, & \text{sonst},
            \end{cases}
            \quad
            \text{bzw.}
            \quad
            b_{i, k}
            =
            \begin{cases}
                1, & \exists v \in V: e_k = (v_i, v), \\
             -1, & \exists v \in V: e_k = (v, v_i), \\
                0, & \text{sonst}.
            \end{cases}
        \end{align*}

        Wir nennen $\mathbf A := (a_{i, j})_{i,j=1}^n$ eine \textbf{Adjazenz-Matrix}, $\mathbf D := \diag(d_1, \dots, d_n)$ eine \textbf{Valenz-Matrix}, $\mathbf C := \mathbf D - \mathbf A$ eine \textbf{Laplace-Matrix} und $\mathbf B := (b_{i, k})_{i,k=1}^{n, m}$ eine \textbf{Inzidenz-Matrix} von $G$.

    \end{definition*}
    
\end{frame}

%%%%%%%%%%%%%%%%%%%%%%%%%%%%%%%%%%%%%%%%%%%%%%%%%%%%%%%
\begin{frame}{Graphen-Matrizen: Beispiele (0 / 4)}

    Betrachte die Graphen aus der vorherigen Abbildung.

    \begin{gather*}
        v_1 := A, v_2 := B, v_3 := C, v_4 := D, \\
        e_1 := \overline{A B},
        e_2 := \overline{B C},
        e_3 := \overline{C D},
        e_4 := \overline{D A}, \\
        e_{4, -1} := \overline{A D},
        e_{4, 1} := \overline{D A}^{(1)},
        e_{4, 2} := \overline{D A}^{(2)}
    \end{gather*}

    Die Mehrfachkanten seien dabei die letzteren $2$.
    Die Kanten seien lexikographisch bzgl. ihrer Indizes geordnet.

\end{frame}

%%%%%%%%%%%%%%%%%%%%%%%%%%%%%%%%%%%%%%%%%%%%%%%%%%%%%%%
\begin{frame}{Graphen-Matrizen: Beispiele (1 / 4)}

    \begin{multicols*}{2}
        
        \begin{figure}[H]
            \centering
            \includegraphics[height = 0.6 \textheight]{120px-Graph_ungerichtet.svg.png}
            \caption{Ungerichteter Graph ohne Mehrfachkanten}
        \end{figure}

        \begin{align*}
            \mathbf A
            & =
            \begin{pmatrix}
                0 & 1 & 0 & 1 \\
                1 & 0 & 1 & 0 \\
                0 & 1 & 0 & 1 \\
                1 & 0 & 1 & 0 \\
            \end{pmatrix}, \\
            \mathbf D
            & =
            \begin{pmatrix}
                2 & 0 & 0 & 0 \\
                0 & 2 & 0 & 0 \\
                0 & 0 & 2 & 0 \\
                0 & 0 & 0 & 2 \\
            \end{pmatrix}, \\
            \mathbf B
            & =
            \begin{pmatrix}
                1 & 0 & 0 & 1 \\
                1 & 1 & 0 & 0 \\
                0 & 1 & 1 & 0 \\
                0 & 0 & 1 & 1 \\
            \end{pmatrix}
        \end{align*}

    \end{multicols*}

\end{frame}

%%%%%%%%%%%%%%%%%%%%%%%%%%%%%%%%%%%%%%%%%%%%%%%%%%%%%%%
\begin{frame}{Graphen-Matrizen: Beispiele (4 / 4)}

    \begin{multicols*}{2}
        
        \begin{figure}[H]
            \centering
            \includegraphics[height = 0.6 \textheight]{Graph_gerichtet_Mehrfachkanten.svg.png}
            \caption{Gerichteter Graph mit Mehrfachkanten}
        \end{figure}

        \begin{align*}
            \mathbf A
            & =
            \begin{pmatrix}
                0 & 1 & 0 & 1 \\
                0 & 0 & 1 & 0 \\
                0 & 0 & 0 & 1 \\
                2 & 0 & 0 & 0 \\
            \end{pmatrix}, \\
            \mathbf D
            & =
            \begin{pmatrix}
                2 & 0 & 0 & 0 \\
                0 & 1 & 0 & 0 \\
                0 & 0 & 1 & 0 \\
                0 & 0 & 0 & 2 \\
            \end{pmatrix}, \\
            \mathbf B
            & =
            \begin{pmatrix}
                0 & 0 & 0 & 1 & 1 & 0 \\
                1 & 0 & 0 & 0 & 0 & 0 \\
                0 & 1 & 0 & 0 & 0 & 0 \\
                0 & 0 & 1 & 0 & 0 & 1 \\
            \end{pmatrix}
        \end{align*}

    \end{multicols*}

\end{frame}

%%%%%%%%%%%%%%%%%%%%%%%%%%%%%%%%%%%%%%%%%%%%%%%%%%%%%%%
    
\begin{frame}{}
    
    \begin{remark*}
        Ein Graph $G$ ist durch eine seiner Adjazenz-Matrizen eindeutig bestimmt.
        Die Umkehrung gilt im Allgemeinen nur modulo Spalten- und Zeilen-Reihenfolge, d.h.

        \begin{align*}
            \forall \mathbf A_1, \mathbf A_2 \in \mathcal A(G):
                \exists \mathbf P ~\text{Permutations-Matrix}:
                    \mathbf A_1 = \mathbf P^{-1} \mathbf A_2 \mathbf P,
        \end{align*}

        d.h. $G$ ist durch seine Klasse $\mathcal A(G)$ von Adjazenz-Matrizen eindeutig bestimmt.
        Adjazenz-Matrizen desselben Graphen haben dasselbe Spektrum.
        Dies rechtfertigt folgende Definition.
        
    \end{remark*}

\end{frame}

%%%%%%%%%%%%%%%%%%%%%%%%%%%%%%%%%%%%%%%%%%%%%%%%%%%%%%%
\begin{frame}{Definitionen (3 / 5)}
    
    \begin{definition*}[Gewöhnliches Spektrum \& Charakteristisches Polynom]
        
        Finde numerische Werte $\overline v \in \mathbb R$ für die Knoten $v \in V$ (als Variablen) eines Graphen $G = (V, E)$ mit $|V| = n$, sodass für alle Knoten $v$ das Verhältnis zwischen der Summe $\overline s_v$ der Werte der nach $v$ Knoten $w \to v$ (die über eine Kante nach $v$ führen) und $\overline v$ konstant ist, d.h.
    
        \begin{align*}
            \exists \lambda \in \mathbb C:
                \forall v \in V:
                    \overline s_v = \sum_{w \to v} \overline w = \lambda \overline v.
        \end{align*}
    
        Sei $\mathbf A \in \mathcal A(G)$, dann können wir das umformulieren zu $\mathbf A \mathbf v = \lambda \mathbf v$. \\
        Eine notwendige und hinreichende Bedingung für die Existenz eines solchen $\mathbf v$ ist, dass $\lambda$ Nullstelle des \textbf{gewöhnlichen Charakteristischen Polynoms} $P_G(\mu) := \det(\mu \mathbf I_n - \mathbf A)$. \\
        $\lambda$ ist dann Teil des \textbf{gewöhnlichen Spektrums} $\sigma_P(G) := \sigma(\mathbf A)$ von $G$.

    \end{definition*}

\end{frame}

%%%%%%%%%%%%%%%%%%%%%%%%%%%%%%%%%%%%%%%%%%%%%%%%%%%%%%%
\begin{frame}{Zusammenfassung: \enquote{spektrale} Eigenschaften von Graphen}

    \begin{block}{}

        \begin{itemize}
            \item Graphen-Matrizen z.B. $\mathbf A$, $\mathbf D$, \dots
            \item (Koeffizienten von) verschiedene(n) Charakteristische(n) Polynome(n) z.B. $P_G(\lambda)$, $Q_G(\lambda)$, \dots
            \item Graphen-Spektren z.B. $\sigma_P(G)$, $\sigma_Q(G)$, \dots
        \end{itemize}

    \end{block}

\end{frame}

%%%%%%%%%%%%%%%%%%%%%%%%%%%%%%%%%%%%%%%%%%%%%%%%%%%%%%%
\begin{frame}{Definitionen (4 / 5)}

    \begin{definition*}[Kanten-Graph]

        Der \textbf{Kanten-Graph} $L(G) = (V^\prime, E^\prime)$ eines ungerichteten Graphen $G = (V, E)$ hat als Knoten die Kanten von $G$ und diese sind adjazent, wenn sie als $G$-Kante einen gemeinsamen $G$-Knoten habend, d.h.

        \begin{align*}
            V^\prime = E,
            \quad
            E^\prime = \Bbraces{\Bbraces{e_1, e_2}: e_1, e_2 \in E, e_1 \cap e_2 \neq \emptyset}.
        \end{align*}

    \end{definition*}

    \begin{definition*}
        Die \textbf{direkte Summe} $G = G_1 \dotplus G_2$ der Graphen $G_1 = (V_1, E_1)$ und $G_2 = (V_2, E_2)$ mit $E_1 \cap E_2 = \emptyset$ ist $G = (V, E)$ mit $V = V_1 \cup V_2$ und $E = E_1 \cup E_2$.
    \end{definition*}

\end{frame}

%%%%%%%%%%%%%%%%%%%%%%%%%%%%%%%%%%%%%%%%%%%%%%%%%%%%%%%

\begin{frame}{Definitionen (5 / 5)}

    \begin{definition*}[NEPS]

        Seien $G_1 = (V_1, E_1), \dots, G_n = (V_n, E_n)$ Graphen. Deren \textbf{NEPS} (\enquote{Non-complete Extended $P$-Sum}) bzgl. $\mathcal B \subseteq \Bbraces{0, 1}^n \setminus \Bbraces{(0, \dots, 0)}$ sei der Graph $G$ mit folgender Knoten- bzw. Kanten-Menge.
    
        \begin{align*}
            V & := \prod_{i=1}^n V_i, \\
            E
            :=
            \{ &
                ((v_1, \dots, v_n), (w_1, \dots, w_n)) \in V^2:
                \exists (\beta_1, \dots, \beta_n) \in \mathcal B: \\ &
                    \forall i = 1, \dots, n:
                        (\beta_i = 1 \implies (v_i, w_i) \in E_i),
                        (\beta_i = 0 \implies v_i = w_i)
            \}.
        \end{align*}

    \end{definition*}
    
    % \begin{definition*}[prominente NEPS]
    %     \begin{itemize}
    %         \item $\mathcal B = \Bbraces{(1, 1)}$ \dots $G = G_1 \times G_2$ \textbf{direktes Produkt}
    %         \item $\mathcal B = \Bbraces{(0, 1), (1, 0)}$ \dots $G = G_1 + G_2$ \textbf{Summe}
    %     \end{itemize}
    % \end{definition*}

\end{frame}

%%%%%%%%%%%%%%%%%%%%%%%%%%%%%%%%%%%%%%%%%%%%%%%%%%%%%%%
\begin{frame}{Wichtige Fragestellungen}

    \begin{block}{}

        \begin{itemize}

            \item Zusammenhänge zwischen (unter) verschiedenen \enquote{spektralen} Eigenschaften

            \item \enquote{spektrale} Eigenschaften \dots

            \begin{itemize}
                \item von prominenter Graphen (und deren Zusammenhänge),
                \item von Graphen durch Graphen-Operationen (vs. Graphen-Matrix-Operationen z.B. $+$, $\cdot$, $\det$, $\otimes$, \dots),
                \item (effiziente) Berechnung
                \item Zusammenhänge zum Graphen selbst (geometrische Eigenschaften) in beide Richtungen!
            \end{itemize}

        \end{itemize}

    \end{block}

\end{frame}

%%%%%%%%%%%%%%%%%%%%%%%%%%%%%%%%%%%%%%%%%%%%%%%%%%%%%%%
\begin{frame}{}

    \begin{block}{}

        {
            \centering
            \huge
            Ende des Crashkurses!
        }

    \end{block}

\end{frame}

\appendix

%%%%%%%%%%%%%%%%%%%%%%%%%%%%%%%%%%%%%%%%%%%%%%%%%%%%%%%
\begin{frame}{Aussicht auf Spezialisierung}

    Jedes chemische Molekül kann als \textbf{molekularer Graph} dargestellt werden:
    Die Ecken entsprechen Atomen und die Kanten deren Bindungen.
    Die $\pi$-Elektronen-Energie (Energie der Elektronen in $\pi$-Orbitalen) von Kohlen-Wasserstoffen entspricht der \dots
    
    \begin{definition*}[Energie eines Graphen]

        Sei $G$ ein Graph, dann lautet seine Energie

        \begin{align*}
            \mathcal E(G)
            :=
            \sum_{\lambda \in \sigma_P(G)} |\lambda|.
        \end{align*}

    \end{definition*}

\end{frame}

%%%%%%%%%%%%%%%%%%%%%%%%%%%%%%%%%%%%%%%%%%%%%%%%%%%%%%%
\begin{frame}{}
 
    % \begin{figure}[H]
    %     \centering
    %     \includegraphics[width = 0.3 \textwidth]{1200px-Ethanol_Lewis.svg.png}
    %     \caption
    %     {
    %         Quelle:
    %         \href
    %         {https://de.wikipedia.org/wiki/Ethanol}
    %         {https://de.wikipedia.org/wiki/Ethanol}
    %     }
    % \end{figure}
    
    \begin{block}{}

        {
            \centering
            \huge
            To be continued ... am 7. Juni 2021.
        }

    \end{block}

\end{frame}

%%%%%%%%%%%%%%%%%%%%%%%%%%%%%%%%%%%%%%%%%%%%%%%%%%%%%%%%%%%%%%%%%%%%%%%%%%%%% 
%%%%%%%%%%%%%%%%%%%%%%%%%%%%%%%%%%%%%%%%%%%%%%%%%%%%%%%%%%%%%%%%%%%%%%%%%%%%%

\end{document}


%%% Local Variables:
%%% mode: latex
%%% TeX-master: t
%%% End:
