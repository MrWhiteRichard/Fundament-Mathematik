% ---------------------------------------------------------------- %
% special letters

\newcommand{\N}{\mathbb N}
\newcommand{\Z}{\mathbb Z}
\newcommand{\Q}{\mathbb Q}
\newcommand{\R}{\mathbb R}
\newcommand{\C}{\mathbb C}
\newcommand{\K}{\mathbb K}
\newcommand{\T}{\mathbb T}
\newcommand{\E}{\mathbb E}
\newcommand{\V}{\mathbb V}
\renewcommand{\S}{\mathbb S}
\renewcommand{\P}{\mathbb P}
\newcommand{\1}{\mathbbm 1}
\newcommand{\G}{\mathbb G}

\newcommand{\iu}{\mathrm i}

% ---------------------------------------------------------------- %
% braces

\newcommand{\pbraces}[1]{{\left  ( #1 \right  )}}
\newcommand{\bbraces}[1]{{\left  [ #1 \right  ]}}
\newcommand{\Bbraces}[1]{{\left \{ #1 \right \}}}
\newcommand{\vbraces}[1]{{\left  | #1 \right  |}}
\newcommand{\Vbraces}[1]{{\left \| #1 \right \|}}

\newcommand{\abraces}[1]{{\left \langle #1 \right \rangle}}

\newcommand{\floorbraces}[1]{{\left \lfloor #1 \right \rfloor}}
\newcommand{\ceilbraces} [1]{{\left \lceil  #1 \right \rceil }}

\newcommand{\dbbraces}    [1]{{\llbracket     #1 \rrbracket}}
\newcommand{\dpbraces}    [1]{{\llparenthesis #1 \rrparenthesis}}
\newcommand{\dfloorbraces}[1]{{\llfloor       #1 \rrfloor}}
\newcommand{\dceilbraces} [1]{{\llceil        #1 \rrceil}}

\newcommand{\dabraces}[1]{{\left \langle \left \langle #1 \right \rangle \right \rangle}}

\newcommand{\abs}  [1]{\vbraces{#1}}
\newcommand{\round}[1]{\bbraces{#1}}
\newcommand{\floor}[1]{\floorbraces{#1}}
\newcommand{\ceil} [1]{\ceilbraces{#1}}

% ---------------------------------------------------------------- %

\newcommand{\Set}{\mathbf{Set}}

\newcommand{\Top}{\mathbf{Top}}
\newcommand{\Met}{\mathbf{Met}}
\newcommand{\Haus}{\mathbf{Haus}}
\newcommand{\Normal}{\mathbf{Normal}}

\newcommand{\Grp}{\mathbf{Grp}}
\newcommand{\Ab}{\mathbf{Ab}}

\newcommand{\TVS}[1]{\mathbf{TVS}_{#1}}
\newcommand{\LCTVS}[1]{\mathbf{LCTVS}_{#1}}


\DeclareMathOperator{\Ob}{Ob}
\DeclareMathOperator{\Hom}{Hom}
\DeclareMathOperator{\id}{id}

\newcommand{\Tc}{\mathcal{T}}
\newcommand{\class}{\mathcal C}
\newcommand{\cat}{\mathcal K}



\NewDocumentCommand
{\tspace}
{ O{X} O{T}}
{\pbraces{#1, \mathcal{#2}}}

\NewDocumentCommand
{\reltspace}
{ O{A} O{T}}
{\pbraces{#1, \mathcal{#2}\vert_{#1}}}

\NewDocumentCommand
{\indtspace}
{ O{X} O{T} O{i}}
{\pbraces{#1_{#3}, \mathcal{#2}_{#3}}}

\NewDocumentCommand
{\tvspace}
{O{O} O{L} O{+} O{\pbraces{\omega_\lambda}} m}
{\pbraces{#2, #3, #4_{\lambda \in #5}, \mathcal{#1}}}


\newcommand{\aex}{absolute extensor }
\newcommand{\ar}{absolute retract }
\newcommand{\ane}{absolute neighbourhood extensor }
\newcommand{\anr}{absolute neighbourhood retract }

\NewDocumentCommand
{\ball}
{ O{0} O{1} O{}}
{\mathrm{B}_{#3}\pbraces{#1, #2}}

\DeclareMathOperator{\metric}{d}
\newcommand{\norm}[2][]{\Vbraces{#2}_{#1}}
\DeclareMathOperator{\dist}  {dist}
\DeclareMathOperator{\diam}  {diam}
\newcommand{\cl}[2][T]{\mathrm{cl}_\mathcal{#1}\pbraces{#2}}
\DeclareMathOperator{\supp}{supp}

\newcommand{\eucl}{eucl}
\newcommand{\partition}[1][F]{\mathfrak{#1}}

\newcommand{\pow}[1]{\mathfrak{P}{#1}}

\newcommand{\cover}[1][U]{\mathcal{#1}}

\newcommand{\Wlog}{without loss of generality }
