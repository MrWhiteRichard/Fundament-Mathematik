\chapter{Absolute retracts and absolute neighborhood retracts}

\begin{definition}
	Let $\tspace$ be a topological space. A set $A \subseteq X$ is said to be a \textit{retract} of $\tspace$, if and only if there exists $r \in \Hom_\Top\pbraces{\tspace, \reltspace}$ such that for all $a \in A$ we have $r(a) = a$. The function $r$ is called \textit{retraction}. The set $A$ is called a \textit{neighborhood retract} of $\tspace$, if and only if there exists a neighborhood $U$ of $A$ in $\tspace$ such that $A$ is a retract of $\reltspace[U][T]$. 
\end{definition}

\begin{definition}
	A topological space $\tspace$ is called an \textit{\ar} for a full subcategory $\cat$ of $\Top$, if and only if every closed set $A$ of an object $\tspace[Y][O]$ of $\cat$ for which $\pbraces{A, \mathcal{O}\vert_A}$ is homeomorphic to $\tspace$, is a retract of $\tspace[Y][O]$.    
\end{definition}

\begin{definition}
	A topological space $\tspace$ is called an \textit{\anr} for a full subcategory $\cat$ of $\Top$, if and only if every closed set $A$ of an object $\tspace[Y][O]$ of $\cat$, for which $\pbraces{A, \mathcal{O}\vert_A}$ is homeomorphic to $\tspace$, is a neighborhood retract of $\tspace[Y][O]$.    
\end{definition}

\begin{proposition}
	If a topological space $\tspace$ is an \ar\ for a full subcategory $\cat$ of $\Top$, then $\tspace$ is also an \anr\ for the class $\cat$. 
\end{proposition}
\begin{proof}
	Let $\tspace$ be an \ar\ for $\cat$ and consider any closed set $A$ of an object $\tspace[Y][O]$ of $\cat$ which is homeomorphic to $\tspace$. The set $A$ is a retract of $\tspace[Y][O]$, because $\tspace$ is an \ar\ for $\cat$. Since $Y$ is a neighborhood of $A$ in $\tspace[Y][O]$ we obtain that $A$ is also an \anr\ for $\cat$. This completes the proof. 
\end{proof}

\begin{theorem} \label{theorem:ane_to_anr}
	Every \ane\ for a full subcategory $\cat$ of $\Top$ is an \anr\ for $\cat$. (also works for \aex\ and \ar\ respectively.)
\end{theorem}
\begin{proof}
	Let $\tspace$ be an \ane\ for $\cat$ and let $A$ be some closed subset of an object $\tspace[Y][O]$ of $\cat$, such that $\pbraces{A, \mathcal{O}\vert_A}$ is homeomorphic to $\tspace$. Let $f \in \Hom\pbraces{\tspace, \reltspace[A][O]}$ be a homeomorphism and $g \in \Hom_\Top\pbraces{\reltspace[A][O], \tspace}$ its inverse. Since $\tspace$ is an \ane\, there exists a neighborhood $U$ of $A$ in $\tspace[Y][O]$ and an extension $\overline{g} \in \Hom_\Top\pbraces{\reltspace[U][O], \tspace}$ of $g$. The function $f \circ \overline{g} \in \Hom_\Top\pbraces{\pbraces{U, \mathcal{O}\vert_U}, \pbraces{A, \mathcal{O}\vert_A}}$ is a retraction, since for $a \in A$ we have $\overline{g}(a) = g(a)$, hence $f(\overline{g}(a)) = f(g(a)) = a$. This concludes the proof. 
\end{proof}

\begin{lemma}
	If $\tspace$ is a compact and metrizable topological space, then there exists a compact topological space $\tspace[Y][O]$ and an embedding $\iota \in \Hom_\Top\pbraces{\tspace, C(Y)}$ such that $\iota\bbraces{X}$ is linearly independent. 
\end{lemma}

\begin{proof}
	Choose some $x_0 \notin X$ and let $\pbraces{\tspace[Y][O], (f_1, f_2)}$ be the disjoint union of $\tspace$ and $x_0$. By \ref{lemma:disjoint_met} there exists a metric $\metric: Y \times Y \to [0, \infty)$, we can choose it to be bounded, that induces $\mathcal{O}$. We define
	\begin{align*}
		A := \Bbraces{f \in \Hom_\Top\pbraces{\tspace[Y][O], \pbraces{\R, \eucl}} : \forall y_1, y_2 \in Y\pbraces{\vbraces{f(y_1) - f(y_2)} \leq \metric\pbraces{y_1, y_2}} \land f\pbraces{\pbraces{x_0, 2}} = 0}.
	\end{align*}
	
	We will use the Arzelà-Ascoli theorem \cite[p. 458]{Ana1&2} to show that $\pbraces{A, \mathcal{T}\pbraces{\norm[\infty]{\cdot}}\vert_A}$ is totally bounded. We have for any $\pbraces{x,i} \in Y$ and any $f \in A$ that
	\begin{align*}
		\vbraces{f\pbraces{(x,i)}} = \vbraces{f\pbraces{(x,i)} - f\pbraces{\pbraces{x_0, 2}}} \leq \metric\pbraces{(x,i), \pbraces{x_0, 2}} \leq \diam Y.
	\end{align*}
	This shows that $A$ is pointwise bounded.
	
	It remains to show that $A$ is uniformly equicontinuous. In order to prove this, let $y \in Y$ and $\varepsilon \in \R^+$ be a given number. For every $f \in A$ and every $\tilde{y} \in Y$ with $\metric\pbraces{y, \tilde{y}} < \varepsilon$ we obtain
	\begin{align*}
		\vbraces{f(y) - f\pbraces{\tilde{y}}} \leq \metric\pbraces{y, \tilde{y}} < \varepsilon.
	\end{align*}
	Thus, $A$ is uniformly equicontinuous and hence totally bounded by the Arzelà-Ascoli Theorem.
	
	We will show that $A$ is closed in $C(Y)$. Consider any $g \in \cl{A}$. There exists a net $\pbraces{g_i}_{i \in I}$ of elements of $A$ that converges to $g$. For every $y_1, y_2 \in Y$ and every $\varepsilon \in \R^+$ we obtain
	\begin{align*}
		\vbraces{g\pbraces{y_1} - g\pbraces{y_2}} \leq \vbraces{g\pbraces{y_1} - g_i\pbraces{y_1}} + \vbraces{g_i\pbraces{y_1} - g_i\pbraces{y_2}} + \vbraces{g_i\pbraces{y_2} - g\pbraces{y_2}} < \metric\pbraces{y_1, y_2} + \varepsilon
	\end{align*} 
	and 
	\begin{align*}
		\vbraces{g\pbraces{\pbraces{x_0, 2}}} \leq \vbraces{g\pbraces{\pbraces{x_0, 2}} - g_i\pbraces{\pbraces{x_0, 2}}} + \vbraces{g_i\pbraces{\pbraces{x_0, 2}}} < \varepsilon
	\end{align*}
	for sufficiently large $i \in I$. Thus, $g \in A$ and $A$ is closed, hence $A$ is compact. 
	
	We claim that the function $\iota: X \to C(A)$ defined by $\iota(x)(f) := f(x)$ for all $x \in X$ and $f \in C(A)$ is an isometry. We have to verify that $\iota$ is well defined. Let $g \in A$ and $\epsilon \in \R^+$ be given. For any $\tilde{g} \in A$ with $\norm[\infty]{g - \tilde{g}} < \epsilon$, then clearly $\vbraces{\iota(x)(g) - \iota(x)\pbraces{\tilde{g}}} = \vbraces{g(x) - \tilde{g}(x)} < \epsilon$. Thus, $\iota(x) \in C(A)$.  
	
	It remains to show that $\iota$ is an isometry. In order to prove this, consider any $x_1, x_2 \in X$. We have 
	\begin{align*}
		\norm[\infty]{\iota\pbraces{x_2} - \iota\pbraces{x_2}} = \sup \Bbraces{\vbraces{g\pbraces{x_2} - g\pbraces{x_1}} : g \in A} \leq \metric\pbraces{x_1, x_2}.
	\end{align*} 
	Let $g: Y \to \R$ be a function defined by $g(y) := \metric\pbraces{y, \pbraces{x_2, 1}} - \metric\pbraces{\pbraces{x_0, 2}, \pbraces{x_2, 1}}$. Since $g\pbraces{\pbraces{x_0, 2}}$ and  for arbitrary $y_1, y_2 \in Y$
	\begin{align*}
		\vbraces{g\pbraces{y_1} - g\pbraces{y_2}} &= \vbraces{\metric\pbraces{y_1, \pbraces{x_2, 1}} - \metric\pbraces{\pbraces{x_0, 2}, \pbraces{x_2, 1}} - \metric\pbraces{y_2, \pbraces{x_2, 1}} + \metric\pbraces{\pbraces{x_0, 2}, \pbraces{x_2, 1}}} \\
		&= \vbraces{\metric\pbraces{y_1, \pbraces{x_2, 1}} - \metric\pbraces{y_2, \pbraces{x_2, 1}}} \leq \metric\pbraces{y_1, y_2},
	\end{align*}
	we have $g \in A$. We obtain
	\begin{align*}
		\vbraces{\iota(x_1)(g) - \iota(x_2)(g)} = \vbraces{\metric\pbraces{\pbraces{x_1, 1}, \pbraces{x_2, 1}} - \metric\pbraces{\pbraces{x_2, 1}, \pbraces{x_2, 1}}} = \metric\pbraces{\pbraces{x_1, 1}, \pbraces{x_2, 1}}.
	\end{align*}
	\colorbox{red}{attention! the metric $\metric$ is not the metric on $X$ but on $Y$!!} Therefore, $\norm[\infty]{\iota\pbraces{x_2} - \iota\pbraces{x_2}} = \metric\pbraces{\pbraces{x_1, 1}, \pbraces{x_2, 1}}$.
	
	
	Finally, we have to verify that $\iota[X]$ is linearly independent. We consider $z \in X$ a linear combination 
	\begin{align*}
		\sum_{x \in X} \lambda_x \iota(x) = 0
	\end{align*}
	and define $g: Y \to \R$ by $g(y) := \min\pbraces{\Bbraces{\metric\pbraces{y, (x, 1)} \mid x \in X \setminus \{z\} \land \lambda_x \neq 0} \cup \Bbraces{\metric\pbraces{y, \pbraces{x_0, 2}}}}$. Let $y_1, y_2 \in Y$ and \Wlog $g(y_1) \leq g(y_2)$. There exist $\tilde{y}_1, \tilde{y}_2 \in Y$ such that $g(y_1) = \metric\pbraces{y_1, \tilde{y}_1}$ and $g(y_2) = \metric\pbraces{y_2, \tilde{y}_2}$. We obtain
	\begin{align*}
		\vbraces{g(y_2) - g(y_1)} &= g(y_2) - g(y_1) = \metric\pbraces{y_2, \tilde{y}_2} - \metric\pbraces{y_1, \tilde{y}_1} \\
		&\leq \metric\pbraces{y_2, \tilde{y}_1} - \metric\pbraces{y_1, \tilde{y}_1} \leq \metric\pbraces{y_2, y_1}
	\end{align*}
	and clearly $g\pbraces{\pbraces{x_0, 2}} = 0$, hence $g \in A$. We have
	\begin{align*}
		0 = \sum_{x \in X} \lambda_x \iota(x)(g) = \sum_{x \in X} \lambda_x g\pbraces{\pbraces{x, 1}} = \lambda_z g(\pbraces{z, 1})
	\end{align*}
	and since $g\pbraces{\pbraces{z, 1}} \neq 0$ we obtain $\lambda_z = 0$. This concludes the proof, because $z$ was arbitrary.  
\end{proof}


\begin{theorem} \label{theorem:met_banach}
	For every metrizable space $\tspace$ there exists convex subset $C$ of a topological vector space $\tvspace{\R}$, a norm $\norm{\cdot}$ on $\pbraces{L, +, \pbraces{\omega_\lambda}_{\lambda \in \R}}$ that induces $\mathcal{O}$ and makes $\tvspace[\norm{\cdot}]{\R}$ a Banach space and a closed subset $Y$ of $\reltspace[C][O]$ such that $\tspace$ and $\reltspace[Y][O]$ are homeomorphic. 
\end{theorem}
\begin{proof}
	Given any metrizable topological space $\tspace$, we consider the set of all continuous and bounded real valued functions
	\begin{align*}
		L := \Bbraces{f \in \Hom_\Top\pbraces{\tspace, \tspace[\R][\eucl]} \mid \norm[\infty]{f} < \infty}. 
	\end{align*}
	It is a well known fact, which one can find for example in \cite{Ana1&2} that $\tvspace[\norm[\infty]{\cdot}]{\R}$ is a Banach space. Let $\mathcal{O}$ be the topology induced by $\norm[\infty]{\cdot}$ and $\metric$ a bounded metric that induces $\mathcal{T}$. We define a function $\iota \in \Hom_\Top\pbraces{\tspace, \tspace[L][O]}$ by $\iota(x_1)(x_2) := \metric\pbraces{x_1, x_2}$. The set $C := \conv\pbraces{\iota[X]}$ is clearly a convex subset of $L$. We define $Y := \iota[X]$ and claim that it is a closed subset of $\reltspace[C][O]$. In order to show this let $f \in C \setminus Y$ be a given function. By definition of $C$ there exists a finite set $I$ and there are families $\pbraces{\lambda_i}_{i \in I} \in [0, 1]^I$ and $\pbraces{z_i}_{i \in I} \in X^I$ such that 
	\begin{align*}
		\sum_{i \in I} \lambda_i = 1 \quad \text{and} \quad f = \sum_{i \in I} \lambda_i \iota(z_i).  
	\end{align*}
	Since $f \notin Y$ we have $\norm[\infty]{f - \iota\pbraces{z_i}} > 0$ for all $i \in I$. Hence, we can choose $\delta \in \R^+$, such that for all $i \in I$ the inequality $2\delta < \norm[\infty]{f - \iota\pbraces{z_i}}$ is satisfied. 
	
	We claim that $C \cap B_{\norm[\infty]{\cdot}}\pbraces{f, \delta} \subseteq C \setminus Y$. In order to show this assume there exists $g \in C \cap B_{\norm[\infty]{\cdot}}\pbraces{f, \delta}$ such that $g \in Y$.  By the triangle inequality we have for any $i \in I$ that
	\begin{align*}
		\norm[\infty]{f - \iota(z_i)} \leq \norm[\infty]{f - g} + \norm[\infty]{g - \iota(z_i)}.
	\end{align*}
	Therefore, we obtain
	\begin{align*}
		\norm{g - \iota(z_i)} \geq \norm[\infty]{f - \iota(z_i)} - \norm[\infty]{f - g} > 2\delta - \delta = \delta.
	\end{align*}
	Since $g \in Y$, there exists $x \in X$ such that $\iota(x) = g$. \colorbox{red}{The function $\iota$ is an isometry}, hence
	\begin{align*}
		\iota(z_i)(x) = \metric\pbraces{z_i, x} = \norm[\infty]{\iota(z_i) - \iota(x)} = \norm[\infty]{\iota(z_i) - f} > \delta.
	\end{align*}
	From this we obtain
	\begin{align*}
		\norm[\infty]{f - g} \geq \vbraces{f(x) - \iota(x)(x)} = \vbraces{f(x) - \metric\pbraces{x, x}} = \vbraces{f(x)} = \sum_{i \in I} \lambda_i \iota(z_i)(x) > \sum_{i \in I} \lambda_i \delta = \delta.
	\end{align*}
	This clearly contradicts $g \in B_{\norm[\infty]{\cdot}}\pbraces{f, \delta}$. 
\end{proof}

\begin{theorem}\cite[p. 84]{ToR} \label{theorem:anr_to_ane}
	Let $\cat$ be one of the following full subcategories of $\Top$.
	\begin{enumerate}
		\item $\Met$
		\item $\Normal$
		\item seaparable and $\Met$
	\end{enumerate}
	Then every \anr\ (resp. \ar) for $\cat$ is an \ane\ (resp. \aex) for the class $\cat$. 
\end{theorem}

\begin{definition}
	Let $\tspace$ and $\tspace[Y][O]$ be topological spaces and $\cover$ an open cover of $\tspace[Y][O]$. Two functions $f,g \in \Hom_\Top\pbraces{\tspace, \tspace[Y][O]}$ are called $\cover$\textit{-close}, if and only if for every $x \in X$ there exists a $U \in \cover$ such that $\Bbraces{f(x), g(x)} \subseteq U$. A homotopy $F \in \Hom_\Top\pbraces{\prodtspace, \tspace[Y][O]}$ is said to be \textit{limited} by $\cover$, if and only if for every $x \in X$ there exists a $Uif and only if there   \in \cover$ such that $F\bbraces{\Bbraces{x} \times \bbraces{0, 1}} \subseteq U$. The functions $f$ and $g$ are called $\cover$\textit{-homotopic}, \colorbox{red}{if and only if there}
\end{definition}

\begin{theorem}
	For every open cover $\cover$ of a metrizable space $\tspace[Y][O]$ which is an \anr\ for $\Met$ there exists an open cover $\cover[V]$ of $\tspace[Y][O]$ such that $\cover[V]$ is a refinement of $\cover$ and for every topological space $\tspace$ the following holds true. Any two $\cover[V]$-close functions $f,g \in \Hom_\Top\pbraces{\tspace, \tspace[Y][O]}$ are $\cover$-homotopic. 
\end{theorem}
\begin{proof}
	By \ref{theorem:met_banach} we can assume \Wlog\ that there exists a convex subset $C$ of a Banach space $\tvspace[\norm{\cdot}]{\R}$ such that $Y$ is a closed subset of $\pbraces{C, \mathcal{T}_{\norm{\cdot}}\vert_C}$ and $\mathcal{O} = \mathcal{T}_{\norm{\cdot}}\vert_Y$. Since $\tspace[Y][O]$ is an \anr\ for $\Met$ there exists a neighborhood $W$ of $Y$ in $\pbraces{C, \mathcal{T}_{\norm{\cdot}}\vert_C}$ and a retraction $r \in \Hom\pbraces{\pbraces{W, \mathcal{T}_{\norm{\cdot}}\vert_W}, \tspace[Y][O]}$. 
	
	The set $\tilde{\cover[V]} := \Bbraces{r^{-1}[U] \mid U \in \cover}$ is an open cover of $\pbraces{W, \mathcal{T}_{\norm{\cdot}}\vert_W}$. Hence, for a given $x \in W$ we find a $U_x \in \cover$ such that $x \in r^{-1}\bbraces{U_x}$. Since $W$ is an open subset of $\pbraces{C, \mathcal{T}_{\norm{\cdot}}\vert_C}$, the set $W \cap r^{-1}[U_x] = r^{-1}[U_x]$ is also an open subset of $\pbraces{C, \mathcal{T}_{\norm{\cdot}}\vert_C}$. Due to the construction of the subspace topology, there exists $W_x \in \mathcal{T}_{\norm{\cdot}}$ such that $r^{-1}[U_x] = W_x \cap C$. Since $L$ is locally convex, there exists a convex set $\tilde{C}_x \in \mathcal{T}_{\norm{\cdot}}$ such that $x \in \tilde{C}_x \subseteq W_x$. The set $C_x := \tilde{C}_x \cap C \subseteq W_x \cap C = r^{-1}[U_x]$ is open in $\pbraces{C, \mathcal{T}_{\norm{\cdot}}\vert_C}$ and as an intersection of two convex sets it is convex. Clearly, $\hat{\cover[V]} := \Bbraces{C_x \mid x \in W}$ is an open cover of $W$ that consists only of convex sets and is a refinement of $\tilde{\cover[V]}$. We define $\cover[V] := Y \cap \hat{\cover[V]}$. 
	
	Given a topological space $\tspace$ and two $\cover[V]$-close functions $f,g \in \Hom\pbraces{\tspace, \tspace[Y][O]}$ we define the homotopy $\tilde{F} \in \Hom\pbraces{\prodtspace, \pbraces{W, \mathcal{T}_{\norm{\cdot}}\vert_W}}$ by $\tilde{F}(x, \lambda) := \lambda g(x) + (1 - \lambda) f(x)$. The function $F \in \Hom\pbraces{\prodtspace, \tspace[Y][O]}$ defined by $F := r \circ \tilde{F}$ is clearly also a homotopy. 
	
	It remains to show that  $F$ is limited by $\cover$ and that $\ran\pbraces{\tilde{F}} \subseteq W$. In order to show this, let $x \in X$ be a given point. Since $f$ and $g$ are $\cover[V]$-close, there exists $V \in \cover[V]$ such that $\Bbraces{f(x), g(x)} \subseteq V$. By definition of $\cover[V]$ there exists $\hat{V} \in \hat{\cover[V]}$ such that $V = Y \cap \hat{V}$. Since $\hat{V}$ is convex, we clearly have $\tilde{F}\bbraces{\Bbraces{x} \times \bbraces{0, 1}} \subseteq \hat{V} \subseteq W$. Since $\cover[V]$ is a refinement of $\hat{\cover[V]}$, there exists $U \in \cover$, such that $\hat{V} \subseteq r^{-1}[U] \cap W$. Therefore,
	\begin{align*}
		 F\bbraces{\Bbraces{x} \times \bbraces{0, 1}} = r\bbraces{\tilde{F}\bbraces{\Bbraces{x} \times \bbraces{0, 1}}} \subseteq r\bbraces{\hat{V}} \subseteq r\bbraces{r^{-1}[U] \cap W} = U \cap r[W] = U \cap Y = U.
	\end{align*}
\end{proof}

\begin{remark}
	The property in the theorem above does not characterize \anrs. A counterexample can be found in \cite{IDT}.
\end{remark}

\begin{lemma} \label{lemma:closed_projection}
	If $\tspace$ be a topological space and $\tspace[K][O]$ a compact topological space, then the projection $\pi_1 \in \Hom\pbraces{\prodtspace[X][T][K][O], \tspace}$ defined by $\pi_1\pbraces{(x, k)} := x$ is a closed function.  
\end{lemma}
\begin{proof}
	Let $A$ be an arbitrary closed subset of $\pbraces{\prodtspace[X][T][K][O]}$ and $\pbraces{\pi_1\pbraces{\pbraces{x_i, k_i}}}_{i \in I}$ a net in $\pi_1\bbraces{A}$ that converges in $\tspace$ to some $x \in X$.  Since $K$ is compact, there exists a subnet $\pbraces{k_{i_j}}_{j \in J}$ of $\pbraces{k_i}_{i \in I}$ that converges in $\tspace[K][O]$ to some $k \in K$. Given any neighborhoods $U$ of $x$ in $\tspace$ and $V$ of $k$ in $\tspace[K][O]$ there exists $j_0 \in J$ such that for all $j \geq j_0$ we have $\pbraces{x_{i_j}, k_{i_j}} \in U \times V$. Hence, $\pbraces{x_{i_j}, k_{i_j}}_{j \in J}$ converges in $\prodtspace[X][T][K][O]$ to $\pbraces{x, k}$. Since $A$ is closed, we have $\pbraces{x, k} \in A$. Therefore, $x = \pi_1\pbraces{\pbraces{x, k}} \in \pi_1\bbraces{A}$, which shows that $\pi_1\bbraces{A}$ is closed in $\tspace$.  
\end{proof}

\begin{lemma} \label{lemma:ext_id}
	Let $A$ be a topological space $\tspace$ that satisfies \ref{axiom:t4}. Then for every neighborhood $U$ of $B := \pbraces{X \times \{0\}} \cup \pbraces{A \times \bbraces{0, 1}}$ in $\prodtspace$ there exists a function $F \in \Hom\pbraces{\prodtspace, \pbraces{U, \pbraces{\mathcal{T} \times \eucl}\vert_U}}$ such that $F\vert_B = \id_B$. 
\end{lemma}
\begin{proof}
	Since $U$ is a neighborhood of $B$ in $\prodtspace$, there exists an open subset $U^\prime$ of $\prodtspace$ such that $B \subseteq U^\prime \subseteq U$. Since $D := \pbraces{X \times \bbraces{0,1}} \setminus U^\prime$ is closed in $\prodtspace$, we can apply \ref{lemma:closed_projection} and obtain that $\pi_1\bbraces{D}$ is closed in $\tspace$. 
	
	By \ref{lemma:urysohn}, which requires $\tspace$ to satisfy \ref{axiom:t4}, there exists $f \in \Hom\pbraces{\tspace, \eucl}$ such that $f\bbraces{A} \subseteq \Bbraces{1}$ and $f\bbraces{\pi_1\bbraces{D}} \subseteq \bbraces{0}$. We claim that the function $F \in \Hom\pbraces{\prodtspace, \pbraces{U, \pbraces{\mathcal{T} \times \mathcal{E}}\vert_U}}$ defined by $F\pbraces{x, \lambda} := \pbraces{x, f(x) \lambda}$ satisfies $F\vert_B = \id_B$. In order to show this let $(x, \lambda) \in B$ be given. If $\pbraces{x, \lambda} \in X \times 0$, then $F\pbraces{(x, \lambda)} = F\pbraces{\pbraces{x, 0}} = \pbraces{x, 0} = \pbraces{x, \lambda}$. If, on the other hand, $\pbraces{x, \lambda} \in A \times \bbraces{0, 1}$, then $F\pbraces{\pbraces{x, \lambda}} = \pbraces{x, f(x) \lambda} = \pbraces{x, \lambda}$. 
	
	It remains to show that $\ran F \subseteq U$. Given any $\pbraces{x, \lambda} \in X \times \bbraces{0, 1}$ we can distinguish two cases. If $x \in \pi_1\bbraces{D}$, then $f(x) =1$, hence $F\pbraces{\pbraces{x, \lambda}} = \pbraces{x, f(x) \lambda} = \pbraces{x, 0} \in U$. If, on the other hand, $x \in X \setminus \pi_1\bbraces{D}$, then $\pbraces{x, \lambda} \notin D$. Therefore, $\pbraces{x, \lambda} \in U^\prime \subseteq U$.     
\end{proof}

\begin{theorem}[Borsuk Homotopy Extension Theorem]
	If $A$ is a closed subspace of a metrizable space $\tspace$, the topological space $\tspace[Y][O]$ is an \anr\ for $\Met$ and $F \in \Hom\pbraces{\pbraces{A \times \bbraces{0, 1}, \mathcal{T}\vert_A \times \eucl}, \tspace[Y][O]}$ is a homotopy such that $f \in \Hom\pbraces{\reltspace[A], \tspace[Y][O]}$ defined by $f(x) := F(x, 0)$ can be extended to a function $g \in \Hom\pbraces{\tspace, \tspace[Y][O]}$, then there exists a homotopy $G \in \Hom\pbraces{\prodtspace, \tspace[Y][O]}$ such that for all $x \in X$ the equlaity $G(x, 0) = g(x)$ is satisfied and for all $\pbraces{x, \lambda} \in A \times \bbraces{0, 1}$ we have $G(x, \lambda) = F(x, \lambda)$. 
\end{theorem}
\begin{proof}
	We define $B := \pbraces{X \times \Bbraces{0}} \cup \pbraces{A \times \bbraces{0, 1}}$ and $H \in \Hom\pbraces{\pbraces{B, \mathcal{T} \times \eucl}, \tspace[Y][O]}$ by 
	\begin{align*}
		H(x,\lambda) :=
		\begin{cases}
			F(x,\lambda) &, \text{if } (x, \lambda) \in A \times \bbraces{0, 1} \\
			g(x) &, \text{if } \pbraces{x, \lambda} \in X \times \Bbraces{0}.
		\end{cases}
	\end{align*}
	\colorbox{red}{Show continuity of} $H$. 
	
	Since $\tspace[Y][O]$ is an \anr\ for $\Met$ and $B$ is a closed subspace of $X \times \bbraces{0, 1}$, we can apply \ref{theorem:anr_to_ane} and obtain a neighborhood $V$ of $B$ in $\prodtspace$ and an extension $H^\prime \in \Hom\pbraces{\pbraces{V, \pbraces{\mathcal{T} \times \eucl}\vert_V}, \tspace[Y][O]}$ of $H$. By \ref{lemma:ext_id} there exists $\tilde{F} \in \Hom\pbraces{\prodtspace, \pbraces{V, \pbraces{\mathcal{T} \times \eucl}\vert_V}}$ such that $\tilde{F}\vert_B = id_B$. We define the function $G \in \Hom\pbraces{\prodtspace, \tspace[Y][O]}$ by $G(x, \lambda) := H^\prime \pbraces{\tilde{F}\pbraces{(x, \lambda)}}$. Given $x \in X$ we have 
	\begin{align*}
		G(x, 0) = H^\prime\pbraces{\tilde{F}\pbraces{\pbraces{x, 0}}} =  H^\prime\pbraces{\pbraces{x, 0}} = g(x).	
	\end{align*}
	For given $\pbraces{x, \lambda} \in A \times \bbraces{0, 1}$ we obtain
	\begin{align*}
		G(x, \lambda) = H^\prime\pbraces{\tilde{F}\pbraces{\pbraces{x, \lambda}}} = H^\prime\pbraces{\pbraces{x, \lambda}} = F\pbraces{x, \lambda}. 
	\end{align*} ´
\end{proof}