\section{Absolute retracts and absolute neighborhood retracts}

\begin{definition}
	Let $\tspace$ be a topological space. A set $A \subseteq X$ is said to be a \textit{retract} of $\tspace$, if and only if there exists a $r \in \Hom\_\Top\pbraces{\tspace, \reltspace}$ such that for all $a \in A$ we have $r(a) = a$. The function $r$ is called \textit{retraction}. The set $A$ is called a \textit{neighborhood retract} of $\tspace$, if and only if there exists a neighborhood $U$ of $A$ in $\tspace$ such that $A$ is a retract of $\reltspace[U][T]$. 
\end{definition}

\begin{definition}
	A topological space $\tspace$ is called an \textit{\ar} for a full subcategory $\cat$ of $\Top$, if and only if every closed set $A$ of an object $\tspace[Y][O]$ of $\cat$ for which $\pbraces{A, \mathcal{O}\vert\_A}$ is homeomorphic to $\tspace$, is a retract of $\tspace[Y][O]$.    
\end{definition}

\begin{definition}
	A topological space $\tspace$ is called an \textit{\anr} for a full subcategory $\cat$ of $\Top$, if and only if every closed set $A$ of an object $\tspace[Y][O]$ of $\cat$, for which $\pbraces{A, \mathcal{O}\vert\_A}$ is homeomorphic to $\tspace$, is a neighborhood retract of $\tspace[Y][O]$.    
\end{definition}

\begin{proposition}
	If a topological space $\tspace$ is an \ar\ for a full subcategory $\cat$ of $\Top$, then $\tspace$ is also an \anr\ for the class $\cat$. 
\end{proposition}
\begin{proof}
	Let $\tspace$ be an \ar\ for $\cat$ and consider any closed set $A$ of an object $\tspace[Y][O]$ of $\cat$ which is homeomorphic to $\tspace$. The set $A$ is a retract of $\tspace[Y][O]$, because $\tspace$ is an \ar\ for $\cat$. Since $Y$ is a neighborhood of $A$ in $\tspace[Y][O]$ we obtain that $A$ is also an \anr\ for $\cat$. This completes the proof. 
\end{proof}

\begin{theorem} \label{theorem:ane\_to\_anr}
	Every \ane\ for a full subcategory $\cat$ of $\Top$ is an \anr\ for $\cat$. (also works for \aex\ and \ar\ respectively.)
\end{theorem}
\begin{proof}
	Let $\tspace$ be an \ane\ for $\cat$ and let $A$ be some closed subset of an object $\tspace[Y][O]$ of $\cat$, such that $\pbraces{A, \mathcal{O}\vert\_A}$ is homeomorphic to $\tspace$. Let $f \in \Hom\pbraces{\tspace, \reltspace[A][O]}$ be a homeomorphism and $g \in \Hom\_\Top\pbraces{\reltspace[A][O], \tspace}$ its inverse. Since $\tspace$ is an \ane\, there exists a neighborhood $U$ of $A$ in $\tspace[Y][O]$ and an extension $\overline{g} \in \Hom\_\Top\pbraces{\reltspace[U][O], \tspace}$ of $g$. The function $f \circ \overline{g} \in \Hom\_\Top\pbraces{\pbraces{U, \mathcal{O}\vert\_U}, \pbraces{A, \mathcal{O}\vert\_A}}$ is a retraction, since for $a \in A$ we have $\overline{g}(a) = g(a)$, hence $f(\overline{g}(a)) = f(g(a)) = a$. This concludes the proof. 
\end{proof}

\begin{lemma}
	If $\tspace$ is a compact and metrizable topological space, then there exists a compact topological space $\tspace[Y][O]$ and an embedding $\iota \in \Hom\_\Top\pbraces{\tspace, C(Y)}$ such that $\iota\bbraces{X}$ is linearly independent. 
\end{lemma}

\begin{proof}
	Choose some $x\_0 \notin X$ and let $\pbraces{\tspace[Y][O], (f\_1, f\_2)}$ be the disjoint union of $\tspace$ and $x\_0$. By \ref{lemma:disjoint\_met} there exists a metric $\metric: Y \times Y \to [0, \infty)$, we can choose it to be bounded, that induces $\mathcal{O}$. We define
	\begin{align*}
		A := \Bbraces{f \in \Hom\_\Top\pbraces{\tspace[Y][O], \pbraces{\R, \eucl}} : \forall y\_1, y\_2 \in Y\pbraces{\vbraces{f(y\_1) - f(y\_2)} \leq \metric\pbraces{y\_1, y\_2}} \land f\pbraces{\pbraces{x\_0, 2}} = 0}.
	\end{align*}
	
	We will use the Arzelà-Ascoli theorem \cite[p. 458]{Ana1&2} to show that $\pbraces{A, \mathcal{T}\pbraces{\norm[\infty]{\cdot}}\vert\_A}$ is totally bounded. We have for any $\pbraces{x,i} \in Y$ and any $f \in A$ that
	\begin{align*}
		\vbraces{f\pbraces{(x,i)}} = \vbraces{f\pbraces{(x,i)} - f\pbraces{\pbraces{x\_0, 2}}} \leq \metric\pbraces{(x,i), \pbraces{x\_0, 2}} \leq \diam Y.
	\end{align*}
	This shows that $A$ is pointwise bounded.
	
	It remains to show that $A$ is uniformly equicontinuous. In order to prove this, let $y \in Y$ and $\varepsilon \in \R^+$ be a given number. For every $f \in A$ and every $\tilde{y} \in Y$ with $\metric\pbraces{y, \tilde{y}} < \varepsilon$ we obtain
	\begin{align*}
		\vbraces{f(y) - f\pbraces{\tilde{y}}} \leq \metric\pbraces{y, \tilde{y}} < \varepsilon.
	\end{align*}
	Thus, $A$ is uniformly equicontinuous and hence totally bounded by the Arzelà-Ascoli Theorem.
	
	We will show that $A$ is closed in $C(Y)$. Consider any $g \in \cl{A}$. There exists a net $\pbraces{g\_i}\_{i \in I}$ of elements of $A$ that converges to $g$. For every $y\_1, y\_2 \in Y$ and every $\varepsilon \in \R^+$ we obtain
	\begin{align*}
		\vbraces{g\pbraces{y\_1} - g\pbraces{y\_2}} \leq \vbraces{g\pbraces{y\_1} - g\_i\pbraces{y\_1}} + \vbraces{g\_i\pbraces{y\_1} - g\_i\pbraces{y\_2}} + \vbraces{g\_i\pbraces{y\_2} - g\pbraces{y\_2}} < \metric\pbraces{y\_1, y\_2} + \varepsilon
	\end{align*} 
	and 
	\begin{align*}
		\vbraces{g\pbraces{\pbraces{x\_0, 2}}} \leq \vbraces{g\pbraces{\pbraces{x\_0, 2}} - g\_i\pbraces{\pbraces{x\_0, 2}}} + \vbraces{g\_i\pbraces{\pbraces{x\_0, 2}}} < \varepsilon
	\end{align*}
	for sufficiently large $i \in I$. Thus, $g \in A$ and $A$ is closed, hence $A$ is compact. 
	
	We claim that the function $\iota: X \to C(A)$ defined by $\iota(x)(f) := f(x)$ for all $x \in X$ and $f \in C(A)$ is an isometry. We have to verify that $\iota$ is well defined. Let $g \in A$ and $\epsilon \in \R^+$ be given. For any $\tilde{g} \in A$ with $\norm[\infty]{g - \tilde{g}} < \epsilon$, then clearly $\vbraces{\iota(x)(g) - \iota(x)\pbraces{\tilde{g}}} = \vbraces{g(x) - \tilde{g}(x)} < \epsilon$. Thus, $\iota(x) \in C(A)$.  
	
	It remains to show that $\iota$ is an isometry. In order to prove this, consider any $x\_1, x\_2 \in X$. We have 
	\begin{align*}
		\norm[\infty]{\iota\pbraces{x\_2} - \iota\pbraces{x\_2}} = \sup \Bbraces{\vbraces{g\pbraces{x\_2} - g\pbraces{x\_1}} : g \in A} \leq \metric\pbraces{x\_1, x\_2}.
	\end{align*} 
	Let $g: Y \to \R$ be a function defined by $g(y) := \metric\pbraces{y, \pbraces{x\_2, 1}} - \metric\pbraces{\pbraces{x\_0, 2}, \pbraces{x\_2, 1}}$. Since $g\pbraces{\pbraces{x\_0, 2}}$ and  for arbitrary $y\_1, y\_2 \in Y$
	\begin{align*}
		\vbraces{g\pbraces{y\_1} - g\pbraces{y\_2}} &= \vbraces{\metric\pbraces{y\_1, \pbraces{x\_2, 1}} - \metric\pbraces{\pbraces{x\_0, 2}, \pbraces{x\_2, 1}} - \metric\pbraces{y\_2, \pbraces{x\_2, 1}} + \metric\pbraces{\pbraces{x\_0, 2}, \pbraces{x\_2, 1}}} \\
		&= \vbraces{\metric\pbraces{y\_1, \pbraces{x\_2, 1}} - \metric\pbraces{y\_2, \pbraces{x\_2, 1}}} \leq \metric\pbraces{y\_1, y\_2},
	\end{align*}
	we have $g \in A$. We obtain
	\begin{align*}
		\vbraces{\iota(x\_1)(g) - \iota(x\_2)(g)} = \vbraces{\metric\pbraces{\pbraces{x\_1, 1}, \pbraces{x\_2, 1}} - \metric\pbraces{\pbraces{x\_2, 1}, \pbraces{x\_2, 1}}} = \metric\pbraces{\pbraces{x\_1, 1}, \pbraces{x\_2, 1}}.
	\end{align*}
	\colorbox{red}{attention! the metric $\metric$ is not the metric on $X$ but on $Y$!!} Therefore, $\norm[\infty]{\iota\pbraces{x\_2} - \iota\pbraces{x\_2}} = \metric\pbraces{\pbraces{x\_1, 1}, \pbraces{x\_2, 1}}$.
	
	
	Finally, we have to verify that $\iota[X]$ is linearly independent. We consider $z \in X$ a linear combination 
	\begin{align*}
		\sum\_{x \in X} \lambda\_x \iota(x) = 0
	\end{align*}
	and define $g: Y \to \R$ by $g(y) := \min\pbraces{\Bbraces{\metric\pbraces{y, (x, 1)} \mid x \in X \setminus \{z\} \land \lambda\_x \neq 0} \cup \Bbraces{\metric\pbraces{y, \pbraces{x\_0, 2}}}}$. Let $y\_1, y\_2 \in Y$ and \Wlog $g(y\_1) \leq g(y\_2)$. There exist $\tilde{y}\_1, \tilde{y}\_2 \in Y$ such that $g(y\_1) = \metric\pbraces{y\_1, \tilde{y}\_1}$ and $g(y\_2) = \metric\pbraces{y\_2, \tilde{y}\_2}$. We obtain
	\begin{align*}
		\vbraces{g(y\_2) - g(y\_1)} &= g(y\_2) - g(y\_1) = \metric\pbraces{y\_2, \tilde{y}\_2} - \metric\pbraces{y\_1, \tilde{y}\_1} \\
		&\leq \metric\pbraces{y\_2, \tilde{y}\_1} - \metric\pbraces{y\_1, \tilde{y}\_1} \leq \metric\pbraces{y\_2, y\_1}
	\end{align*}
	and clearly $g\pbraces{\pbraces{x\_0, 2}} = 0$, hence $g \in A$. We have
	\begin{align*}
		0 = \sum\_{x \in X} \lambda\_x \iota(x)(g) = \sum\_{x \in X} \lambda\_x g\pbraces{\pbraces{x, 1}} = \lambda\_z g(\pbraces{\pbraces{z, 1}})
	\end{align*}
	and since $g(\pbraces{\pbraces{z, 1}}) \neq 0$ we obtain $\lambda\_z = 0$. This concludes the proof, because $z$ was arbitrary.  
\end{proof}

\begin{theorem}\cite[p. 84]{ToR} \label{theorem:anr\_to\_ane}
	Let $\cat$ be one of the following full subcategories of $\Top$.
	\begin{enumerate}
		\item $\Met$
		\item $\Normal$
		\item seaparable and $\Met$
	\end{enumerate}
	Then every \anr\ (resp. \ar) for $\cat$ is an \ane\ (resp. \aex) for the class $\cat$. 
\end{theorem}