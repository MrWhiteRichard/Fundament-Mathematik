\documentclass[aspectratio=169]{beamer}



\mode<presentation>
{
 \usetheme[reversetitle,notitle,noauthor]{Wien}
%    \usetheme[noauthor]{Wien}
}

\usepackage{url}
\usepackage{graphicx}
\graphicspath{{./}{./Figures/}}

\usepackage{appendixnumberbeamer}
\usepackage{algorithm2e}
\usepackage{float}
\usepackage{tikz}
\usetikzlibrary{arrows.meta,positioning}
\usetikzlibrary{positioning}
\usetikzlibrary{overlay-beamer-styles}

\tikzset{onslide/.code args={<#1>#2}{%
  \only<#1>{\pgfkeysalso{#2}} % \pgfkeysalso doesn't change the path
}}
\tikzset{temporal/.code args={<#1>#2#3#4}{%
  \temporal<#1>{\pgfkeysalso{#2}}{\pgfkeysalso{#3}}{\pgfkeysalso{#4}} % \pgfkeysalso doesn't change the path
}}

\tikzstyle{highlight}=[red,ultra thick]

% To avoid a warning from the hyperref package:
\pdfstringdefDisableCommands{%
    \def\translate{}%
}


% To make sure, that the footnote is placed above and outside the
% footline (but it only works for one footnote per frame):
%
% \addtobeamertemplate{footnote}{}{\vspace{4ex}}

%%%%%%%%%%%%%%%%%%%%%%%%%%%%%%%%%%%%%%%%%%%%%%%%%%%%%%%%%%%%%%%%%%%%%%%%%%%%%
%%%%%%%%%%%%%%%%%%%%%%%%%%%%%%%%%%%%%%%%%%%%%%%%%%%%%%%%%%%%%%%%%%%%%%%%%%%%%
\title[Maximum Flow Problem]{Maximal flows in networks}


\subtitle{Bachelorarbeit aus Diskreter Mathematik}

\author[F. Schager]{Florian Schager}

\institute[TU Wien]{TU Wien, Vienna, Austria}

\date{14. Juni 2021}

% Hier befinden sich Pakete, die wir beinahe immer benutzen ...

\usepackage[utf8]{inputenc}

% Sprach-Paket:
\usepackage[ngerman]{babel}

% damit's nicht so, wie beim Grill aussieht:
\usepackage{fullpage}

% Mathematik:
\usepackage{amsmath, amssymb, amsfonts, amsthm}
\usepackage{bbm}
\usepackage{mathtools, mathdots}

% Makros mit mehereren Default-Argumenten:
\usepackage{twoopt}

% Anführungszeichen (Makro \Quote{}):
\usepackage{babel}

% if's für Makros:
\usepackage{xifthen}
\usepackage{etoolbox}

% tikz ist kein Zeichenprogramm (doch!):
\usepackage{tikz}

% bessere Aufzählungen:
\usepackage{enumitem}

% (bessere) Umgebung für Bilder:
\usepackage{graphicx, subfig, float}

% Umgebung für Code:
\usepackage{listings}

% Farben:
\usepackage{xcolor}

% Umgebung für "plain text":
\usepackage{verbatim}

% Umgebung für mehrerer Spalten:
\usepackage{multicol}

% "nette" Brüche
\usepackage{nicefrac}

% Spaltentypen verschiedener Dicke
\usepackage{tabularx}
\usepackage{makecell}

% Für Vektoren
\usepackage{esvect}

% (Web-)Links
\usepackage{hyperref}

% Zitieren & Literatur-Verzeichnis
\usepackage[style = authoryear]{biblatex}
\usepackage{csquotes}

% so ähnlich wie mathbb
%\usepackage{mathds}

% Keine Ahnung, was das macht ...
\usepackage{booktabs}
\usepackage{ngerman}
\usepackage{placeins}

% special letters:

\newcommand{\N}{\mathbb{N}}
\newcommand{\Z}{\mathbb{Z}}
\newcommand{\Q}{\mathbb{Q}}
\newcommand{\R}{\mathbb{R}}
\newcommand{\C}{\mathbb{C}}
\newcommand{\K}{\mathbb{K}}
\newcommand{\T}{\mathbb{T}}
\newcommand{\E}{\mathbb{E}}
\newcommand{\V}{\mathbb{V}}
\renewcommand{\S}{\mathbb{S}}
\renewcommand{\P}{\mathbb{P}}
\newcommand{\1}{\mathbbm{1}}

% quantors:

\newcommand{\Forall}{\forall \,}
\newcommand{\Exists}{\exists \,}
\newcommand{\ExistsOnlyOne}{\exists! \,}
\newcommand{\nExists}{\nexists \,}
\newcommand{\ForAlmostAll}{\forall^\infty \,}

% MISC symbols:

\newcommand{\landau}{{\scriptstyle \mathcal{O}}}
\newcommand{\Landau}{\mathcal{O}}


\newcommand{\eps}{\mathrm{eps}}

% graphics in a box:

\newcommandtwoopt
{\includegraphicsboxed}[3][][]
{
  \begin{figure}[!h]
    \begin{boxedin}
      \ifthenelse{\isempty{#1}}
      {
        \begin{center}
          \includegraphics[width = 0.75 \textwidth]{#3}
          \label{fig:#2}
        \end{center}
      }{
        \begin{center}
          \includegraphics[width = 0.75 \textwidth]{#3}
          \caption{#1}
          \label{fig:#2}
        \end{center}
      }
    \end{boxedin}
  \end{figure}
}

% braces:

\newcommand{\pbraces}[1]{{\left  ( #1 \right  )}}
\newcommand{\bbraces}[1]{{\left  [ #1 \right  ]}}
\newcommand{\Bbraces}[1]{{\left \{ #1 \right \}}}
\newcommand{\vbraces}[1]{{\left  | #1 \right  |}}
\newcommand{\Vbraces}[1]{{\left \| #1 \right \|}}
\newcommand{\abraces}[1]{{\left \langle #1 \right \rangle}}
\newcommand{\round}[1]{\bbraces{#1}}

\newcommand
{\floorbraces}[1]
{{\left \lfloor #1 \right \rfloor}}

\newcommand
{\ceilbraces} [1]
{{\left \lceil  #1 \right \rceil }}

% special functions:

\newcommand{\norm}  [2][]{\Vbraces{#2}_{#1}}
\newcommand{\diam}  [2][]{\mathrm{diam}_{#1} \: #2}
\newcommand{\diag}  [1]{\mathrm{diag} \: #1}
\newcommand{\dist}  [1]{\mathrm{dist} \: #1}
\newcommand{\mean}  [1]{\mathrm{mean} \: #1}
\newcommand{\erf}   [1]{\mathrm{erf} \: #1}
\newcommand{\id}    [1]{\mathrm{id} \: #1}
\newcommand{\sgn}   [1]{\mathrm{sgn} \: #1}
\newcommand{\supp}  [1]{\mathrm{supp} \: #1}
\newcommand{\arsinh}[1]{\mathrm{arsinh} \: #1}
\newcommand{\arcosh}[1]{\mathrm{arcosh} \: #1}
\newcommand{\artanh}[1]{\mathrm{artanh} \: #1}
\newcommand{\card}  [1]{\mathrm{card} \: #1}
\newcommand{\Span}  [1]{\mathrm{span} \: #1}
\newcommand{\Aut}   [1]{\mathrm{Aut} \: #1}
\newcommand{\End}   [1]{\mathrm{End} \: #1}
\newcommand{\ggT}   [1]{\mathrm{ggT} \: #1}
\newcommand{\kgV}   [1]{\mathrm{kgV} \: #1}
\newcommand{\ord}   [1]{\mathrm{ord} \: #1}
\newcommand{\grad}  [1]{\mathrm{grad} \: #1}
\newcommand{\ran}   [1]{\mathrm{ran} \: #1}
\newcommand{\graph} [1]{\mathrm{graph} \: #1}
\newcommand{\Inv}   [1]{\mathrm{Inv} \: #1}
\newcommand{\pv}    [1]{\mathrm{pv} \: #1}
\newcommand{\GL}    [1]{\mathrm{GL} \: #1}
\newcommand{\Mod}{\mathrm{Mod} \:}
\newcommand{\Th}{\mathrm{Th} \:}
\newcommand{\Char}{\mathrm{char}}
\newcommand{\At}{\mathrm{At}}
\newcommand{\Ob}{\mathrm{Ob}}
\newcommand{\Hom}{\mathrm{Hom}}
\newcommand{\orthogonal}[3][]{#2 ~\bot_{#1}~ #3}
\newcommand{\Rang}{\mathrm{Rang}}
\newcommand{\NIL}{\mathrm{NIL}}
\newcommand{\Res}{\mathrm{Res}}
\newcommand{\lxor}{\dot \lor}
\newcommand{\Div}{\mathrm{div} \:}
\newcommand{\meas}{\mathrm{meas} \:}

% fractions:

\newcommand{\Frac}[2]{\frac{1}{#1} \pbraces{#2}}
\newcommand{\nfrac}[2]{\nicefrac{#1}{#2}}

% derivatives & integrals:

\newcommandtwoopt
{\Int}[4][][]
{\int_{#1}^{#2} #3 ~\mathrm{d} #4}

\newcommandtwoopt
{\derivative}[3][][]
{
  \frac
  {\mathrm{d}^{#1} #2}
  {\mathrm{d} #3^{#1}}
}

\newcommandtwoopt
{\pderivative}[3][][]
{
  \frac
  {\partial^{#1} #2}
  {\partial #3^{#1}}
}

\newcommand
{\primeprime}
{{\prime \prime}}

\newcommand
{\primeprimeprime}
{{\prime \prime \prime}}

% Text:

\newcommand{\Quote}[1]{\glqq #1\grqq{}}
\newcommand{\Text}[1]{{\text{#1}}}
\newcommand{\fastueberall}{\text{f.ü.}}
\newcommand{\fastsicher}{\text{f.s.}}

\theoremstyle{definition}

% unnumbered theorems
\newtheorem*{theorem*}    {Satz}
\newtheorem*{lemma*}      {Lemma}
\newtheorem*{corollary*}  {Korollar}
\newtheorem*{proposition*}{Proposition}
\newtheorem*{remark*}     {Bemerkung}
\newtheorem*{definition*} {Definition}
\newtheorem*{example*}    {Beispiel}
\newtheorem*{problem*}    {Problem}
\newtheorem*{algorithmus*}    {Algorithmus}
\newtheorem*{algorithmen*}    {Weiterführende Algorithmen}
\newtheorem*{anwendungen*}    {Anwendungen}

\renewcommand{\figurename}{Abbildung}
\renewcommand{\tablename} {Tabelle}


\begin{document}

\begin{frame}
    \titlepage
\end{frame}

\begin{frame}{Übersicht}
  \tableofcontents
\end{frame}

%%%%%%%%%%%%%%%%%%%%%%%%%%%%%%%%%%%%%%%%%%%%%%%%%%%%%%%%%%%%%%%%%%%%%%%%%%%%%
%%%%%%%%%%%%%%%%%%%%%%%%%%%%%%%%%%%%%%%%%%%%%%%%%%%%%%%%%%%%%%%%%%%%%%%%%%%%%
%%%%%%%%%%%%%%%%%%%%%%%%%%%%%%%%%%%%%%%%%%%%%%%%%%%%%%%%%%%%%%%%%%%%%%%%%%%%%

\section{Wiederholung}

\begin{frame}{Wiederholung - Problemstellung}

  \begin{problem*}[Maximum flow]
    Finde maximalen Fluss von $s$ nach $t$ in Netzwerk $N = (G,c,s,t)$, 
    der folgenden Bedingungen genügt:

    \begin{enumerate}
      \item $0 \leq f(e) \leq c(e)$ für jede Kante $e$; \quad (Kapazitätsbeschränkung)
      \item $\sum_{e^+ = v} f(e) = \sum_{e^- = v} f(e)$ für jeden Knoten $v \neq s,t$.
      \quad (Flusserhaltung)
    \end{enumerate}

    \vspace{0.2cm}

    Wert eines Flusses:

    \begin{align*}
      w(f) := \sum_{e^- = s}f(e) - \sum_{e^+ = s} f(e) = \sum_{e^+ = t}f(e) - \sum_{e^- = t} f(e).
    \end{align*}
    
  \end{problem*}
  

\end{frame}


\begin{frame}{Wiederholung - Algorithmus}

  \begin{algorithmus*}
    \begin{itemize}
      \item Starte mit dem trivialen Fluss: $f(e) = 0 , e \in E$.
      \item Finde einen erweiternden Pfad $P$ und berechne
      \begin{align*}
        d := \min[&\{c(e) - f(e): e \text{ Vorwärts-Kante } \in P\} \ \cup \\
        &\{f(e): e \text{ Rückwärts-Kante } \in P\} ].
      \end{align*}
      \item Konstruiere erweiterten Fluss $f'$ mit $w(f') = w(f) + d$:
      \begin{align*}
        f'(e) = \begin{cases}
          f(e) + d, & e \text{ ist Vorwärts-Kante } \in P \\
          f(e) - d, & e \text{ ist Rückwärts-Kante } \in P\\
          f(e), & \text{ sonst}
        \end{cases}
      \end{align*}
      \item Wiederhole solange, bis kein erweiternder Pfad mehr gefunden werden kann.
    \end{itemize}
  \end{algorithmus*}

\end{frame}

\section{Blocking Flows}

\begin{frame}{Hilfsnetzwerke}

  Sei $N = ((V,E), c, s, t)$ ein Flussnetzwerk.

  \begin{definition}[Hilfsnetzwerk $N' = ((V,E'), c', s, t)$]  
  
    \begin{itemize}
      \item{\makebox[5cm][l]{$e = uv \in E$ mit $f(e) < c(e)$} $\rightarrow \quad e' = uv \in E'$ mit $c(e') = c(e) - f(e)$.}
      \item{\makebox[5cm][l]{$e = uv, f(e) > 0$} $\rightarrow \quad e'' = vu \in E'$ mit $c'(e'') = f(e)$.}
    \end{itemize}
  \end{definition}

  \begin{itemize}
    \item Effizientere Implementierung: Weglassen überflüssiger Information
    \item Hilfsnetzwerk $N' \rightarrow$ Schichtnetzwerk $N''$
    \item Schichtlevel eines Knoten $v$ = Distanz zur Quelle $s: d(s,v)$
    \item Entferne alle Knoten $v \neq t: d(s,v) \geq d(s,t)$.
    \item Entferne alle Kanten, welche von einer höheren in eine niedrigere Schicht führen.
  \end{itemize}

\end{frame}


\begin{frame}{Algorithmus - Skizze}

  \begin{definition}
    Ein \textit{Flow} $f$ heißt \textit{Blocking Flow}, wenn
    es bezüglich $f$ keine erweiternden Pfade alleine aus
    Vorwärtskanten gibt.
  \end{definition}
  
  \begin{algorithmus*}[Blocking Flows]
    \begin{itemize}
      \item Starte mit dem trivialen Fluss: $f(e) = 0 , e \in E$.
      \item Wiederhole solange $f$ nicht maximal:
      \begin{itemize}
        \item Erstelle Schichtnetzwerk $N''$ bezüglich $f$.
        \item Finde \textit{Blocking Flow} $g$ in $N''$
        \item Erweitere $f$ um $g$
      \end{itemize}
    \end{itemize}
  \end{algorithmus*}

\end{frame}

\section{Algorithmus: MKM}

\begin{frame}{MKM-Algorithmus (1/3)}

  \begin{definition}[Flusspotential]
    Für einen Knoten $v$ definieren wir das Flusspotential als

    \begin{align*}
      p(v) = \min \left\{ \sum_{e^- = v} c(e), \sum_{e^+ = v} c(e) \right\}. 
    \end{align*}
  \end{definition}

  
  
\end{frame}

\begin{frame}{MKM-Algorithmus (2/3)}

  \begin{algorithmus*}[BlockMKM (Malhotra, Kumar und Mahashwari)]
    \begin{itemize}
      \item Starte mit dem trivialen Fluss: $g(e) = 0 , e \in E$.
      \item Berechne Flusspotentiale für alle $v \in V$.
      \item Wiederhole solange $s \in V$ und $t \in V$:
      \begin{itemize}
        \item Finde Knoten $w$ mit minimalem Flusspotential $p(w)$.
        \item \texttt{Push}($w,p(w)$).
        \item \texttt{Pull}($w,p(w)$).
        \item Entferne redundante Ecken und Kanten.
      \end{itemize}
    \end{itemize}
  \end{algorithmus*}
  
\end{frame}

\begin{frame}{MKM-Algorithmus (3/3)}

  {\small
  \begin{algorithmus*}
    \begin{algorithm}[H]
      \begin{algorithmic}
          \Procedure{Push}{$w, p(w)$}
          \State{Sei $Q$ Warteschlange mit einzigem Element $w$}
          \State{$\forall u \in V: b(u) \leftarrow 0; \quad b(y) \leftarrow k$}
          \State{
          \Repeat{$Q = \emptyset$}
            {\State{Entferne $v$ von $Q$}
              \State{
              \While{$v \neq t \land b(v) \neq 0$}
                {
                  \State{Wähle Kante $e = vu; \quad m \leftarrow \min\{c(e),b(v)\}$}
                  \State{$c(e) \leftarrow c(e) - m; \quad g(e) \leftarrow g(e) + m$}
                  \State{$p^+(u) \leftarrow p^+(u) - m; \quad b(u) \leftarrow b(u) +  m$}
                  \State{$p^-(v) \leftarrow p^-(v) - m; \quad b(v) \leftarrow b(v) - m$}
                  \State{Füge $u$ zu $Q$ hinzu}
                  \State{\textbf{if} $c(e) = 0$ \textbf{then} entferne $e$ von $E$ \textbf{end}}
                }
              }
            }
          }
          \EndProcedure
      \end{algorithmic}
    \end{algorithm}
  \end{algorithmus*}
  }

\end{frame}

\section{Reduktionen auf ein Maximum Flow Problem}

\begin{frame}{Reduktionen auf ein Maximum Flow Problem  (1/3)}

  \begin{problem}[Knotenkapazitäten]
    Zusätzlich zu der Kapazitätsfunktion $c: E \to \R^+$ sei
    noch eine Kapazitätsfunktion $d: V \to \R^+$ gegeben, welche
    den maximalen Fluss durch einen Knoten limitiert.
    

  \pause

    \begin{figure}
    \centering
  
      \subfloat{
  
      \begin{tikzpicture}[
          mycircle/.style={
             circle,
             draw=black,
             fill=gray,
             fill opacity = 0.3,
             text opacity=1,
             inner sep=0pt,
             minimum size=30pt},
          myarrow/.style={-Stealth},
          node distance=0.5cm and 1.0cm
          ]
          \node[mycircle] (s) {$s$};
          \node[mycircle,above right=of s] (v1) {$v_1 (2)$};
          \node[mycircle,below right=of s] (v2) {$v_2 (2)$};
          \node[mycircle,above right=of v2] (t) {$t$};
  
  
        \draw [myarrow] (s) -- node[sloped,font=\small,above] {(2)} (v1);
        \draw [myarrow] (s) -- node[sloped,font=\small,below] {(3)} (v2);
        \draw [myarrow] (v1) -- node[sloped,font=\small,above] {(3)} (t);
        \draw [myarrow] (v2) -- node[sloped,font=\small,below] {(2)} (t);
        \draw [myarrow] (v2) -- node[sloped,font=\small,below] {(1)} (v1);
  
      \end{tikzpicture}
    }
    \hspace{1cm}
    \pause
    \subfloat{
      \begin{tikzpicture}[
        mycircle/.style={
          circle,
          draw=black,
          fill=gray,
          fill opacity = 0.3,
          text opacity=1,
          inner sep=0pt,
          minimum size=30pt},
        newcircle/.style={
          circle,
          draw=black,
          fill=green,
          fill opacity = 0.3,
          text opacity=1,
          inner sep=0pt,
          minimum size=30pt},
        myarrow/.style={-Stealth},
        node distance=0.5cm and 1.0cm
        ]
        \node[mycircle] (s) {$s$};
        \node[newcircle,above right=of s] (v1) {$v_{1,1}$};
        \node[newcircle,below right=of s] (v2) {$v_{2,1}$};
        \node[newcircle,right=of v1] (v3) {$v_{1,2}$};
        \node[newcircle,right=of v2] (v4) {$v_{2,2}$};
        \node[mycircle,above right=of v4] (t) {$t$};


      \draw [myarrow] (s) -- node[sloped,font=\small,above] {(2)} (v1);
      \draw [myarrow] (s) -- node[sloped,font=\small,below] {(3)} (v2);
      \draw [myarrow] (v1) -- node[sloped,font=\small,below] {(2)} (v3);
      \draw [myarrow] (v2) -- node[sloped,font=\small,below] {(2)} (v4);
      \draw [myarrow] (v3) -- node[sloped,font=\small,above] {(3)} (t);
      \draw [myarrow] (v4) -- node[sloped,font=\small,below] {(2)} (t);
      \draw [myarrow] (v4) -- node[sloped,font=\small,below] {(1)} (v1);;

    \end{tikzpicture}
  }
  
  \end{figure}
\end{problem}
  
\end{frame}


\begin{frame}{Reduktionen auf ein Maximum Flow Problem  (2/3)}

  \begin{definition}[Bipartiter Graph]
    Ein ungerichteter Graph $G = (V, E)$ heißt bipartit, 
    falls sich seine Knoten in zwei disjunkte Teilmengen $A$ und $B$ aufteilen lassen, 
    sodass zwischen den Knoten innerhalb beider Teilmengen keine Kanten verlaufen. 
    Das heißt, für jede Kante $uv \in E$ gilt entweder $u \in A$ und $v \in B$ oder $u \in B$ und $v \in A$.
  \end{definition}

  \begin{problem}[Bipartiter Graph - Matching of maximal cardinality]
    Sei $G = (V,E)$ ein ungerichteter, bipartiter Graph.
    Gesucht ist eine Teilmenge $M \subseteq E$ mit maximaler Kardinalität,
    sodass keine zwei Kanten einen gemeinsamen Endknoten teilen.

  \end{problem} 
\end{frame}

\begin{frame}{Reduktionen auf ein Maximum Flow Problem  (3/3)}

  \begin{problem}[Bipartiter Graph - Matching of maximal cardinality]
    \begin{figure}
      \centering
    
        \subfloat{
    
        \begin{tikzpicture}[
            mycircle1/.style={
               circle,
               draw=black,
               fill=red,
               fill opacity = 0.3,
               text opacity=1,
               inner sep=0pt,
               minimum size=10pt},
            mycircle2/.style={
              circle,
              draw=black,
              fill=green,
              fill opacity = 0.3,
              text opacity=1,
              inner sep=0pt,
              minimum size=10pt},
            myarrow/.style={-Stealth},
            node distance=0.7cm and 1.4cm
            ]

            \node[mycircle1] (a1) {};
            \node[mycircle1,below=of a1] (a2) {};
            \node[mycircle1,below=of a2] (a3) {};
            \node[mycircle1,below=of a3] (a4) {};
            \node[mycircle1,below=of a4] (a5) {};
            \node[mycircle2, right= of a1] (b1) {};
            \node[mycircle2,below=of b1] (b2) {};
            \node[mycircle2,below=of b2] (b3) {};
            \node[mycircle2,below=of b3] (b4) {};
            \node[mycircle2,below=of b4] (b5) {};
    
    
          \draw (a1) -- node[sloped,font=\footnotesize,above] {} (b1);
          \draw (a2) -- node[sloped,font=\footnotesize,above] {} (b1);
          \draw (a3) -- node[sloped,font=\footnotesize,above] {} (b2);
          \draw (a3) -- node[sloped,font=\footnotesize,above] {} (b3);
          \draw (a4) -- node[sloped,font=\footnotesize,above] {} (b4);
          \draw (a5) -- node[sloped,font=\footnotesize,above] {} (b5);
    
        \end{tikzpicture}
      }
      \hspace{1cm}
      \pause
      \subfloat{
        \begin{tikzpicture}[
          mycircle/.style={
             circle,
             draw=black,
             fill=gray,
             fill opacity = 0.3,
             text opacity=1,
             inner sep=0pt,
             minimum size=15pt},
          mycircle1/.style={
             circle,
             draw=black,
             fill=red,
             fill opacity = 0.3,
             text opacity=1,
             inner sep=0pt,
             minimum size=10pt},
          mycircle2/.style={
            circle,
            draw=black,
            fill=green,
            fill opacity = 0.3,
            text opacity=1,
            inner sep=0pt,
            minimum size=10pt},
          myarrow/.style={-Stealth},
          node distance=0.7cm and 1.4cm
          ]
          \node[mycircle1] (a1) {};
          \node[mycircle1,below=of a1] (a2) {};
          \node[mycircle1,below=of a2] (a3) {};
          \node[mycircle1,below=of a3] (a4) {};
          \node[mycircle1,below=of a4] (a5) {};
          \node[mycircle2, right= of a1] (b1) {};
          \node[mycircle2,below=of b1] (b2) {};
          \node[mycircle2,below=of b2] (b3) {};
          \node[mycircle2,below=of b3] (b4) {};
          \node[mycircle2,below=of b4] (b5) {};

          \node[mycircle, left= of a3] (s) {$s$};
          \node[mycircle, right= of b3] (t) {$t$};
  
  
        \draw [myarrow] (a1) -- node[sloped,font=\footnotesize,above] {1} (b1);
        \draw [myarrow] (a2) -- node[sloped,font=\footnotesize,above] {1} (b1);
        \draw [myarrow] (a3) -- node[sloped,font=\footnotesize,above] {1} (b2);
        \draw [myarrow] (a3) -- node[sloped,font=\footnotesize,above] {1} (b3);
        \draw [myarrow] (a4) -- node[sloped,font=\footnotesize,above] {1} (b4);
        \draw [myarrow] (a5) -- node[sloped,font=\footnotesize,above] {1} (b5);

        \draw [myarrow] (s) -- node[sloped,font=\footnotesize,above, near end] {1} (a1);
        \draw [myarrow] (s) -- node[sloped,font=\footnotesize,above, near end] {1} (a2);
        \draw [myarrow] (s) -- node[sloped,font=\footnotesize,above, near end] {1} (a3);
        \draw [myarrow] (s) -- node[sloped,font=\footnotesize,below, near end] {1} (a4);
        \draw [myarrow] (s) -- node[sloped,font=\footnotesize,below, near end] {1} (a5);

        \draw [myarrow] (b1) -- node[sloped,font=\footnotesize,above, near start] {1} (t);
        \draw [myarrow] (b2) -- node[sloped,font=\footnotesize,above, near start] {1} (t);
        \draw [myarrow] (b3) -- node[sloped,font=\footnotesize,above, near start] {1} (t);
        \draw [myarrow] (b4) -- node[sloped,font=\footnotesize,below, near start] {1} (t);
        \draw [myarrow] (b5) -- node[sloped,font=\footnotesize,below, near start] {1} (t);
          
      \end{tikzpicture}
    }
    
    \end{figure}
  \end{problem} 
\end{frame}

\section{Preflows}

\begin{frame}{Preflows}

  \begin{definition}[Fluss]

    Ein Fluss ist eine Funktion $f: V \times V \to \R$ die folgenden Bedingungen genügt:

    \begin{enumerate}
      \item[(1)] $\forall (v,w) \in V \times V: f(v,w) \leq c(v,w)$
      \item[(2)] $\forall (v,w) \in V \times V: f(v,w) = -f(w,v)$
      \item[(3)] $\forall v \in V \setminus \{s, t\}: \sum_{u \in V} f(u, v) = 0$.
    \end{enumerate}
    
  \end{definition}
  
  
  \begin{definition}[Preflow]

    Ein \textit{Preflow} ist schließlich eine Funktion $f: V \times V \to \R$,
    welche (1) und (2) erfüllt, sowie eine abgeschwächte dritte Bedingung:

    \begin{enumerate}
      \item[(3')] $\forall v \in V \setminus \{s, t\}: \sum_{u \in V} f(u, v) \geq 0$.
    \end{enumerate}

    Der Wert $e(v) = \sum_{u \in V} f(u,v)$ nennen wir den \textit{flow excess}
    des Preflows $f$ in $v$.
    
  \end{definition}
  

\end{frame}

\begin{frame}{Residualgraph}

  \begin{definition}[Residualgraph]

    Zu einem gegebenen Preflow $f$ definieren wir vorerst die Residualkapazität
    $r_f: V \times V \to \R$ durch

    \begin{align*}
      r_f(v,w) := c(v,w) - f(v,w).
    \end{align*}

    Wir definieren zusätzlich einen Residualgraphen $G_f = (V,E_f)$ mit

    \begin{align*}
      E_f := \{ vw \in E: r_f(v,w) > 0\}
    \end{align*}
      
  \end{definition}
  

\end{frame}

\begin{frame}{Labels}

  \begin{definition}

    Ein \textit{valid labelling} ist eine Funktion $d: V \to \N_0 \cup \{\infty\}$ mit

    \begin{enumerate}
      \item[(4)] $d(s) = |V|, \quad d(t) = 0$
      \item[(5)] $\forall vw \in E_f: d(v) \leq d(w) + 1$.  
    \end{enumerate}

    Weiters nennen wir einen Knoten $v \neq s$ aktiv, solange $e(v) > 0$ und $d(v) < \infty$.
    
  \end{definition}

\end{frame}

\section{Algorithmus: Goldberg-Tarjan}

\begin{frame}{Algorithmus von Goldberg und Tarjan (1/4)}

  Initialisierung:

  \begin{algorithmus*}[Goldberg und Tarjan]
    \begin{algorithm}[H]
      \begin{algorithmic}[1]
          \Procedure{GoldbergTarjan}{$N,f,v,w$}
          \State{$\forall v \neq s: f(s,v) \leftarrow c(s,v);  \quad f(v,s) \leftarrow -c(s,v)$}
          \State{$\forall v,w \neq s: f(v,w) \leftarrow 0$}
          \State{$d(s) = |V|; \quad \forall v \neq s: d(v) \leftarrow 0$}
          \State \While{$\exists v $ aktiv}{Führe eine zulässige Operation aus.}
          \EndProcedure
      \end{algorithmic}
    \end{algorithm}
  \end{algorithmus*}
  
\end{frame}

\begin{frame}{Algorithmus von Goldberg und Tarjan (2/4)}

  \begin{algorithmus*}[Goldberg und Tarjan]
    \begin{algorithm}[H]
      \begin{algorithmic}[1]
          \Procedure{Push}{$N,f,v,w$}
          \State{$\delta \leftarrow \min (e(v), r_f(v,w))$}
          \State{$f(v,w) \leftarrow f(v,w) + \delta; \quad f(w,v) \leftarrow f(w,v) - \delta$}
          \State{$r_f(v,w) \leftarrow r_f(v,w) - \delta; \quad r_f(w,v) \leftarrow f_f(w,v) - \delta$}
          \State{$e(v) \leftarrow e(v) - \delta; e(w) \leftarrow e(w) + \delta$}
          \EndProcedure
      \end{algorithmic}
      \textbf{Voraussetzungen:} 
        \begin{enumerate}
          \item $v$ ist aktiv
          \item $r_f(v,w) > 0$
          \item $d(v) = d(w) + 1$
        \end{enumerate}
    \end{algorithm}
  \end{algorithmus*}
  
\end{frame}

\begin{frame}{Algorithmus von Goldberg und Tarjan (3/4)}

  \begin{algorithmus*}[Goldberg und Tarjan]
    \begin{algorithm}[H]
      \begin{algorithmic}[1]
          \Procedure{Relabel}{$N,f,v,d$}
          \State{$d(v) \leftarrow \min \{ d(w) + 1: r_f(v,w) > 0\}$}
          \EndProcedure
      \end{algorithmic}
      \textbf{Voraussetzungen:} 
        \begin{enumerate}
          \item $v$ ist aktiv
          \item $\forall w: r_f(v,w) > 0 \implies d(v) \leq d(w)$
        \end{enumerate}
    \end{algorithm}
  \end{algorithmus*}
  
\end{frame}

\begin{frame}{Algorithmus von Goldberg und Tarjan (4/4)}

  \begin{itemize}
    \item Maximal $\mathcal{O}(|V|^2|E|)$ Operationen unabhängig von der Reihenfolge.
    \item Speedup möglich durch bewusste Wahl der Reihenfolge:
  \end{itemize}
  
  Der momentan aktive Knoten wird solange weiterbearbeitet, bis entweder

  \begin{enumerate}
    \item $e(v) = 0$ (kein Exzess mehr vorhanden) oder
    \item alle Kanten inzident mit $v$ schon für einen \texttt{Push} verwendet wurden
    und ein \texttt{Relabel} stattgefunden hat. 
  \end{enumerate}

  \begin{itemize}
    \item \texttt{Fifoflow} (first in, first out): Laufzeit $\mathcal{O}(|V|^3)$.
    \item \texttt{Hlflow} (highest label): Laufzeit $\mathcal{O}(|V|^2|E|^{1/2})$.   
  \end{itemize} 
  
\end{frame}


\begin{frame}{Algorithmus von Goldberg und Tarjan (\texttt{Hlflow}) - Beispiel}

  \begin{figure}
    \begin{tikzpicture}[
        mycircle/.style={
           circle,
           draw=black,
           fill=gray,
           fill opacity = 0.3,
           text opacity=1,
           inner sep=0pt,
           minimum size=25pt},
        myarrow/.style={-Stealth},
        node distance=0.8cm and 1.6cm
        ]
        \node[mycircle, onslide=<1>{blue}, label = {\textcolor<1>{blue}{$(8)$}}] (s) {$s$};

        \node[mycircle,above right=of s, onslide=<{2-3,8}>{blue}, label = 
          \only<1>{$(0,30)$}
          \only<2>{\textcolor{blue}{$(1,15)$}}
          \only<3-6>{\textcolor<3>{blue}{$(9,0)$}}
          \only<7>{$(9,5)$}
          \only<8->{\textcolor<8>{blue}{$(9,0)$}}] 
          (a) {$a$};

        \node[mycircle,below right=of s, onslide = <4>{blue}, label = below:
          \only<1-3>{$(0,1)$}
          \only<4->{\textcolor<4>{blue}{$(1,0)$}}] 
          (b) {$b$};

        \node[mycircle,right=of b, onslide = <{5,11}>{blue}, label = below:
          \only<1>{$(0,0)$}
          \only<2-3>{$(0,9)$}
          \only<4>{$(0,10)$}
          \only<5-9>{\textcolor<5>{blue}{$(1,0)$}}
          \only<10>{$(1,2)$}
          \only<11->{\textcolor<11>{blue}{$(1,0)$}}] 
          (c) {$c$};

        \node[mycircle,right=of a, onslide = <6-7>{blue}, label = 
          \only<1>{$(0,0)$}
          \only<2-5>{$(0,6)$}
          \only<6>{\textcolor<6>{blue}{$(1,5)$}}
          \only<7->{\textcolor<7>{blue}{$(10,0)$}}] 
          (d) {$d$};

        \node[mycircle,right=of d, onslide = <9-10>{blue}, label = 
          \only<1-4>{$(0,0)$}
          \only<5>{$(0,8)$}
          \only<6-8>{$(0,9)$}
          \only<9>{$(1,2)$}
          \only<10->{$(2,0)$}]
          (e) {$e$};

        \node[mycircle, onslide =<11>{blue}, right=of c, label = below:
          \only<1-4>{$(0,0)$}
          \only<5-10>{$(0,2)$}
          \only<11->{\textcolor<11>{blue}{$(1,0)$}}] 
          (f) {$f$};

        \node[mycircle,below right=of e, label = {$(0)$}] (t) {$t$};


        \draw [myarrow, onslide=<{1,3,8}>{blue}] (s) -- node[sloped,font=\small, above] {
          \only<1-2>{(30/30)}
          \only<3-7>{(15/30)}
          \only<8->{(10/30)}
          } (a);

        \draw [myarrow, onslide=<1>{blue}] (s) -- node[sloped,font=\small, above] {(1/1)} (b);

        \draw [myarrow, onslide = <{2,7}>{blue}] (a) -- node[sloped,font=\small, above] 
          {\only<1>{(0/6)}\only<2-6>{(6/6)}\only<7->{(1/6)}} (d);

        \draw [myarrow, onslide = <2>{blue}] (a) -- node[sloped,font=\small, above] 
          {\only<1>{(0/9)}\only<2->{(9/9)}} (c);

        \draw [myarrow, onslide = <4>{blue}] (b) -- node[sloped,font=\small, above] 
          {\only<1-3>{(0/26)}\only<4->{(1/26)}} (c);

        \draw [myarrow, onslide = <6>{blue}] (d) -- node[sloped,font=\small, above] 
          {\only<1-5>{(0/1)}\only<6->{(1/1)}} (e);

        \draw [myarrow, onslide=<{5,11}>{blue}] (c) -- node[sloped,font=\small, above] 
          {\only<1-4>{(0/24)}\only<5-10>{(2/24)}\only<11>{(4/24)}} (f);

        \draw [myarrow, onslide=<{5,10}>{blue}] (c) -- node[sloped,font=\small, above] 
          {\only<1-4>{(0/8)}\only<5-9>{(8/8)}\only<10->{(6/8)}} (e);

        \draw [myarrow, onslide = <11>{blue}] (f) -- node[sloped,font=\small, above] 
          {\only<1-10>{(0/25)}\only<11->{(4/25)}} (t);

        \draw [myarrow, onslide=<9>{blue}] (e) -- node[sloped,font=\small, above] 
          {\only<1-8>{(0/7)}\only<9->{(7/7)}} (t);


        \node[below=of c, visible on=<1>]
        {\small {\color{blue} Abbildung:} Initialisierung, $Q = (a,b)$};

        \node[below=of c, visible on=<2>]
        {\small {\color{blue} Abbildung:} \texttt{Relabel}($a$), 
          \texttt{Push}($a,c$), \texttt{Push}($a,d$), $Q = (a,b,c,d)$};

        \node[below=of c, visible on=<3>]
        {\small {\color{blue} Abbildung:} \texttt{Relabel}($a$), 
        \texttt{Push}($a,s$), $Q = (b,c,d)$};

        \node[below=of c, visible on=<4>]
        {\small {\color{blue} Abbildung:} \texttt{Relabel}($b$), 
        \texttt{Push}($b,c$), $Q = (c,d)$};

        \node[below=of c, visible on=<5>]
        {\small {\color{blue} Abbildung:} \texttt{Relabel}($c$), 
        \texttt{Push}($c,e$), \texttt{Push}($c,f$), $Q = (d,e,f)$};

        \node[below=of c, visible on=<6>]
        {\small {\color{blue} Abbildung:} \texttt{Relabel}($d$), 
        \texttt{Push}($d,e$), $Q = (d,e,f)$};

        \node[below=of c, visible on=<7>]
        {\small {\color{blue} Abbildung:} \texttt{Relabel}($d$), 
        \texttt{Push}($d,a$), $Q = (a,e,f)$};

        \node[below=of c, visible on=<8>]
        {\small {\color{blue} Abbildung:} 
        \texttt{Push}($a,s$), $Q = (e,f)$};

        \node[below=of c, visible on=<9>]
        {\small {\color{blue} Abbildung:} 
        \texttt{Relabel}($e$), \texttt{Push}($e,t$), $Q = (e,f)$};

        \node[below=of c, visible on=<10>]
        {\small {\color{blue} Abbildung:} 
        \texttt{Relabel}($e$), \texttt{Push}($e,c$), $Q = (c,f)$};

        \node[below=of c, visible on=<11>]
        {\small {\color{blue} Abbildung:} 
        \texttt{Push}($c,f$), \texttt{Relabel}($f$), \texttt{Push}($f,t$), $Q = \emptyset$};
  
      \end{tikzpicture}
  
    \end{figure}
  
\end{frame}

\end{document}


%%% Local Variables:
%%% mode: latex
%%% TeX-master: t
%%% End:
