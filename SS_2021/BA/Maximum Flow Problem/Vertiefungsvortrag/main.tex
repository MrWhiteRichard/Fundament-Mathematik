\documentclass[aspectratio=169]{beamer}



\mode<presentation>
{
 \usetheme[reversetitle,notitle,noauthor]{Wien}
%    \usetheme[noauthor]{Wien}
}

\usepackage{url}
\usepackage{graphicx}
\graphicspath{{./}{./Figures/}}

\usepackage{appendixnumberbeamer}
\usepackage{algorithm2e}
\usepackage{float}
\usepackage{tikz}
\usetikzlibrary{arrows.meta,positioning}
\usetikzlibrary{positioning}
\usetikzlibrary{overlay-beamer-styles}

\tikzset{onslide/.code args={<#1>#2}{%
  \only<#1>{\pgfkeysalso{#2}} % \pgfkeysalso doesn't change the path
}}
\tikzset{temporal/.code args={<#1>#2#3#4}{%
  \temporal<#1>{\pgfkeysalso{#2}}{\pgfkeysalso{#3}}{\pgfkeysalso{#4}} % \pgfkeysalso doesn't change the path
}}

\tikzstyle{highlight}=[red,ultra thick]

% To avoid a warning from the hyperref package:
\pdfstringdefDisableCommands{%
    \def\translate{}%
}

% To make sure, that the footnote is placed above and outside the
% footline (but it only works for one footnote per frame):
%
% \addtobeamertemplate{footnote}{}{\vspace{4ex}}

%%%%%%%%%%%%%%%%%%%%%%%%%%%%%%%%%%%%%%%%%%%%%%%%%%%%%%%%%%%%%%%%%%%%%%%%%%%%%
%%%%%%%%%%%%%%%%%%%%%%%%%%%%%%%%%%%%%%%%%%%%%%%%%%%%%%%%%%%%%%%%%%%%%%%%%%%%%
\title[Maximum Flow Problem]{Maximal flows in networks}


\subtitle{Bachelorarbeit aus Diskreter Mathematik}

\author[F. Schager]{Florian Schager}

\institute[TU Wien]{TU Wien, Vienna, Austria}

\date{14. Juni 2021}

% Hier befinden sich Pakete, die wir beinahe immer benutzen ...

\usepackage[utf8]{inputenc}

% Sprach-Paket:
\usepackage[ngerman]{babel}

% damit's nicht so, wie beim Grill aussieht:
\usepackage{fullpage}

% Mathematik:
\usepackage{amsmath, amssymb, amsfonts, amsthm}
\usepackage{bbm}
\usepackage{mathtools, mathdots}

% Makros mit mehereren Default-Argumenten:
\usepackage{twoopt}

% Anführungszeichen (Makro \Quote{}):
\usepackage{babel}

% if's für Makros:
\usepackage{xifthen}
\usepackage{etoolbox}

% tikz ist kein Zeichenprogramm (doch!):
\usepackage{tikz}

% bessere Aufzählungen:
\usepackage{enumitem}

% (bessere) Umgebung für Bilder:
\usepackage{graphicx, subfig, float}

% Umgebung für Code:
\usepackage{listings}

% Farben:
\usepackage{xcolor}

% Umgebung für "plain text":
\usepackage{verbatim}

% Umgebung für mehrerer Spalten:
\usepackage{multicol}

% "nette" Brüche
\usepackage{nicefrac}

% Spaltentypen verschiedener Dicke
\usepackage{tabularx}
\usepackage{makecell}

% Für Vektoren
\usepackage{esvect}

% (Web-)Links
\usepackage{hyperref}

% Zitieren & Literatur-Verzeichnis
\usepackage[style = authoryear]{biblatex}
\usepackage{csquotes}

% so ähnlich wie mathbb
%\usepackage{mathds}

% Keine Ahnung, was das macht ...
\usepackage{booktabs}
\usepackage{ngerman}
\usepackage{placeins}

% special letters:

\newcommand{\N}{\mathbb{N}}
\newcommand{\Z}{\mathbb{Z}}
\newcommand{\Q}{\mathbb{Q}}
\newcommand{\R}{\mathbb{R}}
\newcommand{\C}{\mathbb{C}}
\newcommand{\K}{\mathbb{K}}
\newcommand{\T}{\mathbb{T}}
\newcommand{\E}{\mathbb{E}}
\newcommand{\V}{\mathbb{V}}
\renewcommand{\S}{\mathbb{S}}
\renewcommand{\P}{\mathbb{P}}
\newcommand{\1}{\mathbbm{1}}

% quantors:

\newcommand{\Forall}{\forall \,}
\newcommand{\Exists}{\exists \,}
\newcommand{\ExistsOnlyOne}{\exists! \,}
\newcommand{\nExists}{\nexists \,}
\newcommand{\ForAlmostAll}{\forall^\infty \,}

% MISC symbols:

\newcommand{\landau}{{\scriptstyle \mathcal{O}}}
\newcommand{\Landau}{\mathcal{O}}


\newcommand{\eps}{\mathrm{eps}}

% graphics in a box:

\newcommandtwoopt
{\includegraphicsboxed}[3][][]
{
  \begin{figure}[!h]
    \begin{boxedin}
      \ifthenelse{\isempty{#1}}
      {
        \begin{center}
          \includegraphics[width = 0.75 \textwidth]{#3}
          \label{fig:#2}
        \end{center}
      }{
        \begin{center}
          \includegraphics[width = 0.75 \textwidth]{#3}
          \caption{#1}
          \label{fig:#2}
        \end{center}
      }
    \end{boxedin}
  \end{figure}
}

% braces:

\newcommand{\pbraces}[1]{{\left  ( #1 \right  )}}
\newcommand{\bbraces}[1]{{\left  [ #1 \right  ]}}
\newcommand{\Bbraces}[1]{{\left \{ #1 \right \}}}
\newcommand{\vbraces}[1]{{\left  | #1 \right  |}}
\newcommand{\Vbraces}[1]{{\left \| #1 \right \|}}
\newcommand{\abraces}[1]{{\left \langle #1 \right \rangle}}
\newcommand{\round}[1]{\bbraces{#1}}

\newcommand
{\floorbraces}[1]
{{\left \lfloor #1 \right \rfloor}}

\newcommand
{\ceilbraces} [1]
{{\left \lceil  #1 \right \rceil }}

% special functions:

\newcommand{\norm}  [2][]{\Vbraces{#2}_{#1}}
\newcommand{\diam}  [2][]{\mathrm{diam}_{#1} \: #2}
\newcommand{\diag}  [1]{\mathrm{diag} \: #1}
\newcommand{\dist}  [1]{\mathrm{dist} \: #1}
\newcommand{\mean}  [1]{\mathrm{mean} \: #1}
\newcommand{\erf}   [1]{\mathrm{erf} \: #1}
\newcommand{\id}    [1]{\mathrm{id} \: #1}
\newcommand{\sgn}   [1]{\mathrm{sgn} \: #1}
\newcommand{\supp}  [1]{\mathrm{supp} \: #1}
\newcommand{\arsinh}[1]{\mathrm{arsinh} \: #1}
\newcommand{\arcosh}[1]{\mathrm{arcosh} \: #1}
\newcommand{\artanh}[1]{\mathrm{artanh} \: #1}
\newcommand{\card}  [1]{\mathrm{card} \: #1}
\newcommand{\Span}  [1]{\mathrm{span} \: #1}
\newcommand{\Aut}   [1]{\mathrm{Aut} \: #1}
\newcommand{\End}   [1]{\mathrm{End} \: #1}
\newcommand{\ggT}   [1]{\mathrm{ggT} \: #1}
\newcommand{\kgV}   [1]{\mathrm{kgV} \: #1}
\newcommand{\ord}   [1]{\mathrm{ord} \: #1}
\newcommand{\grad}  [1]{\mathrm{grad} \: #1}
\newcommand{\ran}   [1]{\mathrm{ran} \: #1}
\newcommand{\graph} [1]{\mathrm{graph} \: #1}
\newcommand{\Inv}   [1]{\mathrm{Inv} \: #1}
\newcommand{\pv}    [1]{\mathrm{pv} \: #1}
\newcommand{\GL}    [1]{\mathrm{GL} \: #1}
\newcommand{\Mod}{\mathrm{Mod} \:}
\newcommand{\Th}{\mathrm{Th} \:}
\newcommand{\Char}{\mathrm{char}}
\newcommand{\At}{\mathrm{At}}
\newcommand{\Ob}{\mathrm{Ob}}
\newcommand{\Hom}{\mathrm{Hom}}
\newcommand{\orthogonal}[3][]{#2 ~\bot_{#1}~ #3}
\newcommand{\Rang}{\mathrm{Rang}}
\newcommand{\NIL}{\mathrm{NIL}}
\newcommand{\Res}{\mathrm{Res}}
\newcommand{\lxor}{\dot \lor}
\newcommand{\Div}{\mathrm{div} \:}
\newcommand{\meas}{\mathrm{meas} \:}

% fractions:

\newcommand{\Frac}[2]{\frac{1}{#1} \pbraces{#2}}
\newcommand{\nfrac}[2]{\nicefrac{#1}{#2}}

% derivatives & integrals:

\newcommandtwoopt
{\Int}[4][][]
{\int_{#1}^{#2} #3 ~\mathrm{d} #4}

\newcommandtwoopt
{\derivative}[3][][]
{
  \frac
  {\mathrm{d}^{#1} #2}
  {\mathrm{d} #3^{#1}}
}

\newcommandtwoopt
{\pderivative}[3][][]
{
  \frac
  {\partial^{#1} #2}
  {\partial #3^{#1}}
}

\newcommand
{\primeprime}
{{\prime \prime}}

\newcommand
{\primeprimeprime}
{{\prime \prime \prime}}

% Text:

\newcommand{\Quote}[1]{\glqq #1\grqq{}}
\newcommand{\Text}[1]{{\text{#1}}}
\newcommand{\fastueberall}{\text{f.ü.}}
\newcommand{\fastsicher}{\text{f.s.}}

\theoremstyle{definition}

% unnumbered theorems
\newtheorem*{theorem*}    {Satz}
\newtheorem*{lemma*}      {Lemma}
\newtheorem*{corollary*}  {Korollar}
\newtheorem*{proposition*}{Proposition}
\newtheorem*{remark*}     {Bemerkung}
\newtheorem*{definition*} {Definition}
\newtheorem*{example*}    {Beispiel}
\newtheorem*{problem*}    {Problem}
\newtheorem*{algorithmus*}    {Algorithmus}
\newtheorem*{algorithmen*}    {Weiterführende Algorithmen}
\newtheorem*{anwendungen*}    {Anwendungen}

\renewcommand{\figurename}{Abbildung}
\renewcommand{\tablename} {Tabelle}


\begin{document}

\begin{frame}
    \titlepage
\end{frame}

%%%%%%%%%%%%%%%%%%%%%%%%%%%%%%%%%%%%%%%%%%%%%%%%%%%%%%%%%%%%%%%%%%%%%%%%%%%%%
%%%%%%%%%%%%%%%%%%%%%%%%%%%%%%%%%%%%%%%%%%%%%%%%%%%%%%%%%%%%%%%%%%%%%%%%%%%%%
%%%%%%%%%%%%%%%%%%%%%%%%%%%%%%%%%%%%%%%%%%%%%%%%%%%%%%%%%%%%%%%%%%%%%%%%%%%%%

\begin{frame}{Wiederholung - Problemstellung}
  Wir wollen in einem Flussnetzwerk $N = (G,c,s,t)$ einen maximalen Fluss 
  von der Quelle $s$ zur Senke $t$ finden, der folgenden Bedingungen genügt:

  \begin{enumerate}
    \item $0 \leq f(e) \leq c(e)$ für jede Kante $e$; \quad (Kapazitätsbeschränkung)
    \item $\sum_{e^+ = v} f(e) = \sum_{e^- = v} f(e)$ für jeden Knoten $v \neq s,t$.
    \quad (Flusserhaltung)
  \end{enumerate}



  Der Wert eines Flusses ist gegeben durch

  \begin{align*}
    w(f) := \sum_{e^- = s}f(e) - \sum_{e^+ = s} f(e) = \sum_{e^+ = t}f(e) - \sum_{e^- = t} f(e).
  \end{align*}

\end{frame}


\begin{frame}{Wiederholung - Algorithmus}

  \begin{algorithmus*}
    \begin{itemize}
      \item Starte mit dem trivialen Fluss: $f(e) = 0 , e \in E$.
      \item Finde einen erweiternden Pfad $P$.
      \item Berechne
      \begin{align*}
        d := \min[&\{c(e) - f(e): e \text{ Vorwärts-Kante } \in P\} \ \cup \\
        &\{f(e): e \text{ Rückwärts-Kante } \in P\} ].
      \end{align*}
      \item Konstruiere erweiterten Fluss $f'$ mit $w(f') = w(f) + d$:
      \begin{align*}
        f'(e) = \begin{cases}
          f(e) + d, & e \text{ ist Vorwärts-Kante } \in P \\
          f(e) - d, & e \text{ ist Rückwärts-Kante } \in P\\
          f(e), & \text{ sonst}
        \end{cases}
      \end{align*}
      \item Wiederhole solange, bis kein erweiternder Pfad mehr gefunden werden kann.
    \end{itemize}
  \end{algorithmus*}

\end{frame}

\begin{frame}{Blocking Flows}

\end{frame}


\begin{frame}{MKM-Algorithmus}
  
\end{frame}

\begin{frame}{Preflows}
  
  Wir definieren einen Fluss neu als 
  Funktion $f: V \times V \to \R$ die folgenden Bedingungen genügt:

  \begin{enumerate}
    \item[(1)] $\forall (v,w) \in V \times V: f(v,w) \leq c(v,w)$
    \item[(2)] $\forall (v,w) \in V \times V: f(v,w) = -f(w,v)$
    \item[(3)] $\forall v \in V \setminus \{s, t\}: \sum_{u \in V} f(u, v) = 0$.
  \end{enumerate}

  Ein \textit{Preflow} ist schließlich eine Funktion $f: V \times V \to \R$,
  welche (1) und (2) erfüllt, sowie eine abgeschwächte dritte Bedingung:

  \begin{enumerate}
    \item[(3')] $\forall v \in V \setminus \{s, t\}: \sum_{u \in V} f(u, v) \geq 0$.
  \end{enumerate}

  Der Wert $e(v) = \sum_{u \in V} f(u,v)$ nennen wir den \textit{flow excess}
  des Preflows $f$ in $v$.

\end{frame}

\begin{frame}{Residualgraph}
  Zu einem gegebenen Preflow $f$ definieren wir vorerst die Residualkapazität
  $r_f: V \times V \to \R$ durch

  \begin{align*}
    r_f(v,w) := c(v,w) - f(v,w).
  \end{align*}

  Wir definieren zusätzlich einen Residualgraphen $G_f = (V,E_f)$ mit

  \begin{align*}
    E_f := \{ vw \in E: r_f(v,w) > 0\}
  \end{align*}

\end{frame}

\begin{frame}{Labels}
  Ein \textit{valid labelling} ist eine Funktion $d: V \to \N_0 \cup \{\infty\}$ mit

  \begin{enumerate}
    \item[(4)] $d(s) = |V|, \quad d(t) = 0$
    \item[(5)] $\forall vw \in E_f: d(v) \leq d(w) + 1$.  
  \end{enumerate}

  Weiters nennen wir einen Knoten $v \neq s$ aktiv, solange $e(v) > 0$ und $d(v) < \infty$.

  Wir initialisieren den Algorithmus von Goldberg und Tarjan mit einem
  geeigneten Preflow wie folgt:

  \begin{align*}
    \forall v \neq s: f(s,v) &= -f(v,s) = c(s,v) \\
    \forall v,w \neq s: f(v,w) &= 0 \\
    d(s) &= |V|, \quad \forall v \neq s: d(v) = 0.
  \end{align*}
\end{frame}

\begin{frame}{Algorithmus von Goldberg und Tarjan}

  Nach der Initialisierung kann in jedem Schritt einer der folgenden
  Operationen verwendet werden, solange die jeweiligen Voraussetzungen erfüllt sind:

  \begin{algorithmus*}
    \begin{enumerate}
      \item[(1)] Test
    \end{enumerate}
  \end{algorithmus*}
  
\end{frame}

\begin{frame}{Reduktionen auf ein Maximum Flow Problem}
  
\end{frame}

\begin{frame}{Bipartite Graph - Matching of maximal cardinality}
  
\end{frame}

\end{document}


%%% Local Variables:
%%% mode: latex
%%% TeX-master: t
%%% End:
