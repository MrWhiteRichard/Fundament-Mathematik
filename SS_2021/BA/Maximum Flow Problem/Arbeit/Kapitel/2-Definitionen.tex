%%%%%%%%%%%%%%%%%%%%%%%%%%%%%%%%%%%%%%%%%%%%%%%%%%%%%%%%%%%%%%%%%%%%%%%%%%%%%%%%%%%%%%%%%%%%%%%%%%%%%%%%%%%%%%
%%%%%%%%%%%%%%%%%%%%%%%%%%%%%%%%%%%%%%%%%%%%%%%%%%%%%%%%%%%%%%%%%%%%%%%%%%%%%%%%%%%%%%%%%%%%%%%%%%%%%%%%%%%%%%
\chapter{Definitionen}
\label{chapter:definitions}

Bevor wir die Problemstellung, um die sich diese Arbeit drehen wird präzise
einführen können, benötigen wir eine handvoll graphentheoretische Definitionen.
Dafür orientieren wir uns an \cite[Chapter 1]{GraphsNetworks}.

\begin{definition}[Digraph]
    Ein Digraph oder gerichteter Graph ist 
\end{definition}