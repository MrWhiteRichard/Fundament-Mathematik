%%%%%%%%%%%%%%%%%%%%%%%%%%%%%%%%%%%%%%%%%%%%%%%%%%%%%%%%%%%%%%%%%%%%%%%%%%%%%%%%%%%%%%%%%%%%%%%%%%%%%%%%%%%%%%
% EINLEITUNG / INTRODUCTION [OBLIGATORISCH]
%%%%%%%%%%%%%%%%%%%%%%%%%%%%%%%%%%%%%%%%%%%%%%%%%%%%%%%%%%%%%%%%%%%%%%%%%%%%%%%%%%%%%%%%%%%%%%%%%%%%%%%%%%%%%%

%%%%%%%%%%%%%%%%%%%%%%%%%%%%%%%%%%%%%%%%%%%%%%%%%%%%%%%%%%%%%%%%%%%%%%%%%%%%%%%%%%%%%%%%%%%%%%%%%%%%%%%%%%%%%%
%%%%%%%%%%%%%%%%%%%%%%%%%%%%%%%%%%%%%%%%%%%%%%%%%%%%%%%%%%%%%%%%%%%%%%%%%%%%%%%%%%%%%%%%%%%%%%%%%%%%%%%%%%%%%%
\chapter{Einleitung}
\label{chapter:introduction_tipps}
%%%%%%%%%%%%%%%%%%%%%%%%%%%%%%%%%%%%%%%%%%%%%%%%%%%%%%%%%%%%%%%%%%%%%%%%%%%%%%%%%%%%%%%%%%%%%%%%%%%%%%%%%%%%%%
%%%%%%%%%%%%%%%%%%%%%%%%%%%%%%%%%%%%%%%%%%%%%%%%%%%%%%%%%%%%%%%%%%%%%%%%%%%%%%%%%%%%%%%%%%%%%%%%%%%%%%%%%%%%%%

{\color{change}

\noindent
Die Einleitung ist unter Umständen der wichtigste Teil der Arbeit, weil sie darüber entscheidet, ob jemand die Arbeit liest. Üblicherweise wird die Einleitung erst geschrieben, wenn der eigentliche Inhalt der Arbeit als Ganzes steht.

\begin{itemize}

\item Die Einleitung soll an das Thema der Arbeit heranführen und die Hauptergebnisse und Hauptmethoden der Arbeit enthalten. Sie soll Ihre Arbeit in den Kontext der mathematischen Literatur einordnen und Ihre wichtigsten Quellen angeben.

\item Worum geht es?
\item Einordnung der Arbeit ins Forschungsfeld?
\item Was ist die Fragestellung?
\item Warum ist das wichtig?
\item Was gibt es für Resultate in der vorhandenen Literatur?

\item In der Einleitung müssen Sie den eigenen Anteil der Arbeit kenntlich machen:
\begin{itemize}
\item Wos woar mei Leistung?
\item Was sind die Beiträge der vorliegenden Arbeit?
\item Was ist besser als in bisherigen Arbeiten?
\item z.B.\ \emph{Diese Arbeit orientiert sich am Artikel von...} oder \emph{Diese Arbeit
ist eine Ausarbeitung/Verallgemeinerung des Artikels von...}
\end{itemize}

\item Was sind die eigenen Resultate?
\begin{itemize}
\item Ggf. Meta-Theoreme formulieren!
\item Für genauere Formulierung des Theorems nach hinten verweisen!
\end{itemize}

\item Groben Aufbau der Arbeit skizzieren!
\item Verweise auf Notationen / Resultate!
\end{itemize}
}

%%%%%%%%%%%%%%%%%%%%%%%%%%%%%%%%%%%%%%%%%%%%%%%%%%%%%%%%%%%%%%%%%%%%%%%%%%%%%%%%%%%%%%%%%%%%%%%%%%%%%%%%%%%%%%
%%%%%%%%%%%%%%%%%%%%%%%%%%%%%%%%%%%%%%%%%%%%%%%%%%%%%%%%%%%%%%%%%%%%%%%%%%%%%%%%%%%%%%%%%%%%%%%%%%%%%%%%%%%%%%
\chapter{Los geht's}
\label{chapter:content}
%%%%%%%%%%%%%%%%%%%%%%%%%%%%%%%%%%%%%%%%%%%%%%%%%%%%%%%%%%%%%%%%%%%%%%%%%%%%%%%%%%%%%%%%%%%%%%%%%%%%%%%%%%%%%%
%%%%%%%%%%%%%%%%%%%%%%%%%%%%%%%%%%%%%%%%%%%%%%%%%%%%%%%%%%%%%%%%%%%%%%%%%%%%%%%%%%%%%%%%%%%%%%%%%%%%%%%%%%%%%%

%%%%%%%%%%%%%%%%%%%%%%%%%%%%%%%%%%%%%%%%%%%%%%%%%%%%%%%%%%%%%%%%%%%%%%%%%%%%%%%%%%%%%%%%%%%%%%%%%%%%%%%%%%%%%%
\section{Hier beginnt die Arbeit}
%%%%%%%%%%%%%%%%%%%%%%%%%%%%%%%%%%%%%%%%%%%%%%%%%%%%%%%%%%%%%%%%%%%%%%%%%%%%%%%%%%%%%%%%%%%%%%%%%%%%%%%%%%%%%%

{\color{change}
\begin{itemize}
\item Üblicherweise mit den benötigten Notationen und aus der Literatur bekannten Resultaten etc.
\item Grundlegende Definitionen und Resultate der Mathematik-Grundvorlesungen (z.B.\ Eigenwerte, Satz von Taylor, Banach-Räume etc.) dürfen Sie voraussetzen. Resultate der höheren Vorlesungen (z.B.\ Lemma von Lax--Milgram, Definition der Sobolev-Räume) können Sie (mit Referenz und \emph{ohne Beweis}) wiederholen.
\item Wesentlich ist, dass Sie die benötigte Notation \emph{eindeutig} einführen. Verwenden Sie z.B.\ \emph{entweder} $W^{1,2}(\Omega)$ \emph{oder} $H^1(\Omega)$ für den Sobolev-Raum.
\end{itemize}
}

%%%%%%%%%%%%%%%%%%%%%%%%%%%%%%%%%%%%%%%%%%%%%%%%%%%%%%%%%%%%%%%%%%%%%%%%%%%%%%%%%%%%%%%%%%%%%%%%%%%%%%%%%%%%%%
\section{Sinn einer Bachelorarbeit}
%%%%%%%%%%%%%%%%%%%%%%%%%%%%%%%%%%%%%%%%%%%%%%%%%%%%%%%%%%%%%%%%%%%%%%%%%%%%%%%%%%%%%%%%%%%%%%%%%%%%%%%%%%%%%%

{\color{change}
\begin{itemize}

\item Laut Studienplan\footnote{\href{http://www.tuwien.ac.at/?id=vt033201}{\ttfamily http://www.tuwien.ac.at/?id=vt033201}}
gilt:
\begin{quote}\itshape
Die Bachelorarbeit ist eine im Bachelorstudium eigens angefertigte schriftliche Arbeit, welche eigenständige Leistungen beinhaltet und im Rahmen der Lehrveranstaltung "`Projekt mit Bachelorarbeit"' abgefasst wird. Die fertige Bachelorarbeit soll eine intensive Beschäftigung mit einem Problem der reinen oder angewandten Mathematik nachweisen.
\end{quote}

\item Es gibt keine Mindest- oder Maximallänge für eine Bachelorarbeit. In der Regel wird sie 30--40 Seiten umfassen.

\item Eine Bachelorarbeit ist eine eigenständige Arbeit, in der Sie Ihre während
des Studiums erworbene Fähigkeiten unter Beweis stellen. Es ist {\em nicht}
Sinn der Arbeit, englische Texte ins Deutsche zu übersetzen oder ganze
Beweise wortwörtlich oder fast wortwörtlich 
aus anderen Arbeiten zu übernehmen. {\em Dies wäre ein Plagiat!} 

\item Formulieren Sie die Themenstellung, die Lösungsansätze und die Beweise 
mit Ihren eigenen Worten. In den meisten Fällen sind die Beweise
in wissenschaftlichen Arbeiten sehr knapp formuliert. Formulieren Sie diese so
aus, dass Ihre Kolleginnen und Kollegen den Beweis auch verstehen, ohne lange selbst nachdenken
oder rechnen zu müssen. Wenn Sie selbst über ein Argument lange nachdenken mussten,
geben Sie Details an.

\item Lassen Sie Ihre Arbeit von einer Kollegin/einem Kollegen Korrektur lesen.
Diese Person kann logische Fehler oder Widersprüche finden, die Sie nicht mehr
erkennen, weil Sie Ihren Text schon zu häufig gelesen haben.

\item Planen Sie ausreichend Zeit für Ihre Arbeit ein. Laut Studienplan
werden 10 ECTS-Punkte, also 300 Stunden eingeplant. Dies entspricht fast 
8 Wochen bei einer 40-Stunden-Woche. Beachten Sie insbesondere, dass das
Korrekturlesen und numerische Simulationen in der Regel viel Zeit in
Anspruch nehmen.

\item Schreiben Sie Ihre Arbeit mit \LaTeX. Verwenden Sie das Format DIN A4 und eine Schriftgröße von 11 pt
oder 12 pt. Sie können auch dieses Template verwenden!

\end{itemize}
}

%%%%%%%%%%%%%%%%%%%%%%%%%%%%%%%%%%%%%%%%%%%%%%%%%%%%%%%%%%%%%%%%%%%%%%%%%%%%%%%%%%%%%%%%%%%%%%%%%%%%%%%%%%%%%%
%%%%%%%%%%%%%%%%%%%%%%%%%%%%%%%%%%%%%%%%%%%%%%%%%%%%%%%%%%%%%%%%%%%%%%%%%%%%%%%%%%%%%%%%%%%%%%%%%%%%%%%%%%%%%%
\chapter{Gut gemeinte Ratschläge}
\label{chapter:auxiliary}
%%%%%%%%%%%%%%%%%%%%%%%%%%%%%%%%%%%%%%%%%%%%%%%%%%%%%%%%%%%%%%%%%%%%%%%%%%%%%%%%%%%%%%%%%%%%%%%%%%%%%%%%%%%%%%
%%%%%%%%%%%%%%%%%%%%%%%%%%%%%%%%%%%%%%%%%%%%%%%%%%%%%%%%%%%%%%%%%%%%%%%%%%%%%%%%%%%%%%%%%%%%%%%%%%%%%%%%%%%%%%

%%%%%%%%%%%%%%%%%%%%%%%%%%%%%%%%%%%%%%%%%%%%%%%%%%%%%%%%%%%%%%%%%%%%%%%%%%%%%%%%%%%%%%%%%%%%%%%%%%%%%%%%%%%%%%
\section{Ein paar allgemeine Hinweise}
%%%%%%%%%%%%%%%%%%%%%%%%%%%%%%%%%%%%%%%%%%%%%%%%%%%%%%%%%%%%%%%%%%%%%%%%%%%%%%%%%%%%%%%%%%%%%%%%%%%%%%%%%%%%%%

{\color{change}
\begin{itemize}

\item Regelmäßige Überprüfung der Rechtschreibung und Interpunktion!
\item Schreiben Sie stets in vollständigen Sätzen und vermeiden Sie Abkürzungen.  Verwenden Sie eine verständliche Sprache mit klaren, nicht zu langen Sätzen. Es ist üblich, im Deutschen die Wir-Form zu verwenden (z.B.\ \emph{Im Folgenden beweisen wir...}).

\item Achten Sie bei mathematischen Aussagen auf die sprachliche Reihenfolge etwaiger Quantoren (z.B.\ \emph{Auf jeden Topf passt ein Deckel!} vs.\ \emph{Es gibt einen Deckel, der auf jeden Topf passt!}).

\item Achten Sie auf eine einheitliche Verwendung von Eigennamen: \emph{Dirichlet-Rand} oder \emph{Dirichletrand}!

\item Saubere Trennung von mathematischer Aussage und Interpretation einer Aussage. Am besten vor jedem Satz einen kurzen Text über die Bedeutung dieses Ergebnisses.
\item Jede "`Section"' beginnt mit einer kurzen Einleitung, was das Ziel und was die Hauptergebnisse dieser "`Section"' sind.

\item Text sollte nie mit einer Formel beginnen.
\begin{itemize}
\item Besser: \emph{Es bezeichnet $\mathcal{S}^1(\mathcal{T}) := \big\{ ... \big\}$ den Raum ...}
\item Anstelle von: \emph{$\mathcal{S}^1(\mathcal{T}) := \big\{ ... \big\}$ ist der Raum ...}
\end{itemize}

\item In Texten sollte nie Formel auf Formel folgen. Bitte ggf.\ den Text umformulieren!
\begin{itemize}
\item Besser: \emph{Für eine Triangulierung $\mathcal{T}$ definiere $\mathcal{S}^1(\mathcal{T}) := ...$}
\item Anstelle von: \emph{Definiere für eine Triangulierung $\mathcal{T}$ $\mathcal{S}^1(\mathcal{T}) := ...$}
\end{itemize}

\item Lemmata, Sätze, Proposition etc.\ sollten alle gemeinsam gezählt/nummeriert werden (d.h.\ es gibt in dem Dokument \emph{nicht} Lemma~1 und Satz~1). Dadurch lassen sich Resultate im Dokument für den Leser leichter finden. Ggf.\ können Sie das Kapitel in den Zähler einbinden (z.B.\ Satz 3.5).
\begin{itemize}
\item Am einfachsten ist dazu die Verwendung von \verb$\newtheorem$.
\end{itemize}

\item Nur wichtige Formeln kriegen eine Nummer. Das sind üblicherweise Formeln in Sät\-zen, Lemmata etc.\ sowie Formeln, auf die Sie im Beweis referenzieren.

\item Vermeiden Sie die Verwendung von Quantoren und Folgt-Pfeilen im Fließtext, d.h.\ mischen Sie nicht logische Aussagen und Text (d.h.\ keine Verwendung von Quantoren zur Abkürzung von Text)! Quantoren und logische Symbole werden üblicherweise nur in abgesetzten Formeln verwendet. 
\begin{itemize}
\item z.B. \emph{Damit haben wir die folgende Aussage gezeigt:
\begin{align*}
 \forall \varepsilon > 0 \, \exists h_0 > 0 \, \forall \mathcal{T}_h\text{ Triangulierung}:
 \ \Big( \| h \|_{L^\infty(\Omega)} \le h_0
 \, \Longrightarrow \,
 \| u - u_h \|_{H^1(\Omega)} \le \varepsilon \Big).
\end{align*}}
\end{itemize}

\item Binden Sie Formeln als Teile von Sätzen ein. Achten Sie insbesondere auf die Zeichensetzung nach Formeln!
\begin{itemize}
\item Beispiel: \emph{Insgesamt erhalten wir damit
\begin{align*}
 \int_\Omega fg\,dx 
 \le \bigg(\int_\Omega f^p \, dx\bigg)^{1/p} \bigg(\int_\Omega g^q \, dx\bigg)^{1/q}.
\end{align*}}
\end{itemize}

\item Falls Sie auf eine Formel in einem Beweis an einer Stelle außerhalb des Beweises referenzieren müssen, so sollten Sie diese Formel als eigenes Lemma formulieren.

\item Falls Ergebnisse/Argumente mehrfach verwendet werden, sollten Sie diese als eigenes Lemma formulieren.

\item Machen Sie Ihre wesentlichen Ergebnisse (vor allem Sätze, aber auch Lemmata) zitierbar:
\begin{itemize}
\item alle Voraussetzungen hinschreiben (oder auf Generalvoraussetzung verweisen),
\item bei etwaigen Konstanten in Abschätzungen die Abhängigkeit möglichst genau formulieren.
\end{itemize}

\end{itemize}
}

%%%%%%%%%%%%%%%%%%%%%%%%%%%%%%%%%%%%%%%%%%%%%%%%%%%%%%%%%%%%%%%%%%%%%%%%%%%%%%%%%%%%%%%%%%%%%%%%%%%%%%%%%%%%%%
\section{Beweise}
%%%%%%%%%%%%%%%%%%%%%%%%%%%%%%%%%%%%%%%%%%%%%%%%%%%%%%%%%%%%%%%%%%%%%%%%%%%%%%%%%%%%%%%%%%%%%%%%%%%%%%%%%%%%%%

{\color{change}
\begin{itemize}

\item Um die Lesbarkeit der Arbeit zu erhöhen, sollten Beweise deduktiv formuliert werden. Dadurch kann der Leser der Argumentation leichter folgen.
\begin{itemize}
\item Besser: \emph{Weil ..., gilt ...} 
\item Anstelle von: \emph{Es gilt ..., weil ...}
\end{itemize}

\item Schreiben Sie kurze Sätze mit maximal ein bis zwei Nebensätzen anstelle von langen Satzgefügen.
\begin{itemize}
\item Besser: \emph{Es gilt ... Daraus folgt ... Mithilfe von ... ergibt sich ...}
\item Anstelle von: \emph{Es gilt ..., woraus wir zunächst ... und dann schließlich mithilfe von ... auch ... erhalten.}
\end{itemize}

\item Seien Sie eindeutig, aber nicht ausschweifend in Ihren Texten! Anders als bei einem Deutschaufsatz ist sprachliche Vielfalt nicht das Ziel und Wortwiederholung im Grunde OK! Dies gilt insbesondere dann, wenn ein formales Argument wiederholt wird (d.h.\ gleiche Argumente verwenden dieselben Worte).

\item Arbeiten Sie die Beweise vollständig aus! Anders als in einem Paper gibt es in der Arbeit keine Begrenzung der Seitenzahl.

\item Erklären Sie Rechnungen! Eine Rechnung, die über mehrere Zeilen geht, ist in der Regel nicht von alleine verständlich. Für jedes Argument gilt: Sie müssen es entweder zitieren oder beweisen. Standardargumente (z.B.\ Dreiecksungleichung, H\"older-Ungleichung) müssen lediglich genannt, aber nicht zitiert werden.

\item Gliedern Sie längere Beweise \emph{sichtbar} in mehrere Schritte und sagen Sie explizit, was in jedem Schritt gezeigt wird.
\begin{itemize}
\item \emph{Schritt~1.} Zunächst zeigen wir...
\item Verwenden Sie bitte keine \verb$itemize$-Umgebung für die Beweisschritte!
\end{itemize}

\item Erklären Sie, warum Sie "`o.B.d.A"' schreiben! Warum darf man das ohne Einschrän\-kung annehmen?

\item Vermeiden Sie die Einschüchterung des Lesers ("`elementare Rechnung zeigt"', "`trivialerweise gilt"' etc.).
\begin{itemize}
\item Elementare Rechnungen können Sie ggf.\ in einem Appendix sammeln, wenn diese den Lesefluss stören würden (d.h.\ in einem eigenen Abschnitt am Ende der Arbeit, der mit \verb$\appendix$ \emph{vor} den folgenden Gliederungsbefehlen \verb$\chapter{...}$ bzw.\ \verb$\section{...}$ eingeleitet wird).
\end{itemize}

\item Bei Abschätzungen können Sie z.B.\ mittels \verb$\stackrel{\eqref{eq:formel}}{\le}$ Hinweise geben, dass diese Abschätzung aus der Formel mit \verb$\label{eq:formel}$ folgt, z.B. 
$$LHS \stackrel{\normalfont(1.2)}{\le} RHS.$$ 
Dies ist für den Leser eine wesentliche Erleichterung, um umfangreiche Abschätzungen nachzuvollziehen.

\item Ein Beweis sollte nie mit einer Formel enden, sondern mit Text, z.B.\ \emph{womit die Abschätzung {\normalfont(1.3)} gezeigt ist.}
\end{itemize}
}

%%%%%%%%%%%%%%%%%%%%%%%%%%%%%%%%%%%%%%%%%%%%%%%%%%%%%%%%%%%%%%%%%%%%%%%%%%%%%%%%%%%%%%%%%%%%%%%%%%%%%%%%%%%%%%
\section{Bilder}
%%%%%%%%%%%%%%%%%%%%%%%%%%%%%%%%%%%%%%%%%%%%%%%%%%%%%%%%%%%%%%%%%%%%%%%%%%%%%%%%%%%%%%%%%%%%%%%%%%%%%%%%%%%%%%

{\color{change}
\begin{itemize}
\item In den Bild-Unterschriften genau schreiben, was gezeigt wird, insbesondere zu welchem Abschnitt / Beispiel die Abbildung gehört.
\begin{itemize}
\item am besten auch kurze beschreibende Schlagwörter.
\item Beispiel: \emph{Abbildung 3.5. Numerische Resultate zur singulären Lösung auf dem L-Shape aus  Abschnitt~3.1. Wir visualisieren den Fehler $\|u-u_h\|_{H^1(\Omega)}$ über der Anzahl $N$ der Elemente.}
\end{itemize}

\item Unterscheiden Sie bei der Beschreibung von Bildern, was Sie visualisieren, was Sie beobachten und wie Sie dies interpretieren (bzw.\ was dies belegt).

\end{itemize}
}

%%%%%%%%%%%%%%%%%%%%%%%%%%%%%%%%%%%%%%%%%%%%%%%%%%%%%%%%%%%%%%%%%%%%%%%%%%%%%%%%%%%%%%%%%%%%%%%%%%%%%%%%%%%%%%
\section{Zitieren + Literaturverzeichnis}
%%%%%%%%%%%%%%%%%%%%%%%%%%%%%%%%%%%%%%%%%%%%%%%%%%%%%%%%%%%%%%%%%%%%%%%%%%%%%%%%%%%%%%%%%%%%%%%%%%%%%%%%%%%%%%

{\color{change}
\begin{itemize}
\item Um Ergebnisse aus Arbeiten zu zitieren, sollten Sie \verb$\cite[Theorem~X]{workY}$ nutzen und auch das konkrete Resultat angeben.
\begin{itemize}
\item z.B.\ \cite[Theorem~4]{GraphsNetworks}
\end{itemize}

\item Wenn Sie ein Buch zitieren, geben Sie bitte mittels \verb$\cite[Section~X]{buchY}$
den Abschnitt an. Häufig ist es besser, Abschnitte zu zitieren als Seitennummern.

\item Verwenden Sie \verb$BibTeX$ (siehe \verb$literature.bib$ in diesem Template)!
\begin{itemize}
\item Alternativ können Sie auch die \verb$thebibliography$-Umgebung verwenden. Achten Sie in diesem Fall darauf, dass Sie die vollständigen bibliographischen Daten angeben (Artikel: Autoren, Titel, Zeitschrift, Band, Jahr, Seiten; Buch: Autoren, Titel, Verlag, Verlagsort, Auflage, Jahr), alle Einträge \emph{einheitlich} formatieren und alphabetisch sortieren. 
\end{itemize}

\item Die \verb$BibTeX$-Einträge können Sie aus \verb$http://www.ams.org/mathscinet/$ mittels Copy and Paste überneh\-men.
\item Auch wenn Sie die \verb$BibTeX$-Einträge aus \verb$http://www.ams.org/mathscinet/$ über\-neh\-men, sollten Sie darauf achten, dass am Ende die Einträge \emph{einheitlich} sind:
\begin{itemize}
\item Bei Vornamen von Autoren entweder alle abkürzen oder alle ausschreiben. 
\item Bei Journal-Namen entweder bei allen den vollständigen Namen angeben oder
bei allen die offizielle Abkürzung gemäß \verb$http://www.ams.org/mathscinet/$ verwenden.
\end{itemize}

\item siehe auch die Vorlesungsfolien auf \verb$http://www.asc.tuwien.ac.at/compmath/.$

\end{itemize}
}

%%%%%%%%%%%%%%%%%%%%%%%%%%%%%%%%%%%%%%%%%%%%%%%%%%%%%%%%%%%%%%%%%%%%%%%%%%%%%%%%%%%%%%%%%%%%%%%%%%%%%%%%%%%%%%
\section{LaTeX}
%%%%%%%%%%%%%%%%%%%%%%%%%%%%%%%%%%%%%%%%%%%%%%%%%%%%%%%%%%%%%%%%%%%%%%%%%%%%%%%%%%%%%%%%%%%%%%%%%%%%%%%%%%%%%%

{\color{change}
\begin{itemize}

\item Alle \verb$Overfull \hbox$ und \verb$Overfull \vbox$ eliminieren!
\begin{itemize}
\item \verb$Overfull \hbox$ eliminiert man durch geeignete \verb$Sil\-ben\-tren\-nung$ oder geeignete \verb$\linebreak$.
\item \verb$Overfull \vbox$ eliminiert man durch geeignete \verb$\pagebreak$.
\end{itemize}

\item Absätze macht man in \LaTeX, indem man Leerzeilen verwendet. Üblicherweise wird dadurch auch das erste Wort des neuen Absatzes eingerückt, was die Lesbarkeit erhöht. Bitte also \emph{nicht} \verb$\\$ oder \verb$\newline$ für einen neuen Absatz verwenden! 
\item Grundsätzlich machen Leerzeilen vor/nach Umgebungen (z.B.\ \verb$\begin{theorem}$ etc.) den Source-Code lesbarer.
\item Das Tilde-Symbol verhindert einen Zeilenumbruch und wird deshalb vor \verb$\ref$, \verb$\cite$ etc.\ verwendet, z.B.\ \verb$Laut Satz~\ref{satz:xxx} gilt$...

\item In Subscripts oder Superscripts verwendet man kein \verb$\frac$, sondern schreibt den Bruch aus:
\begin{itemize}
\item Also besser $\displaystyle \bigg(\sum_{T \in \mathcal T} \eta_T^2\bigg)^{1/2}$ anstelle von
$\displaystyle \bigg(\sum_{T \in \mathcal T} \eta_T^2\bigg)^{\frac{1}{2}}$.
\item Also besser $m_h^{i+1/2}$ anstelle von $m_h^{i+\frac{1}{2}}$.
\end{itemize}

\item Verwenden Sie im Fließtext möglichst kein \verb$frac$!
\begin{itemize}
\item Besser: $(n+1)^{-1}$ oder $1/(n+1)$.
\item Anstelle von: $\frac{1}{n+1}$.
\end{itemize}

\item Falls Sie ggf.\ Abkürzungen verwenden, sollten Sie in \LaTeX\ \verb$ggf.\$ schreiben, damit der Punkt nicht als Satzende (= größerer Abstand) interpretiert wird.

\item Mittels Verwendung von \verb$\input{kapitelX.tex}$ können Sie Ihr Dokument auf mehrere Dateien aufteilen (z.B.\ kapitelweise).

\item Damit in Formeln Konstanten und Klammern nicht aneinander picken, sollten Sie mittels \verb$\,$ händisch Abstände einfügen, z.B. \verb$C \, h^\alpha$. Dasselbe gilt in Integranden, z.B. \verb$\int_\Omega f^2 \, dx$. \LaTeX\ ist an diesen Stellen sehr "`knausrig"' mit Abständen.
\end{itemize}
}

%%%%%%%%%%%%%%%%%%%%%%%%%%%%%%%%%%%%%%%%%%%%%%%%%%%%%%%%%%%%%%%%%%%%%%%%%%%%%%%%%%%%%%%%%%%%%%%%%%%%%%%%%%%%%%
\section{Englisch vs.\ Deutsch}
%%%%%%%%%%%%%%%%%%%%%%%%%%%%%%%%%%%%%%%%%%%%%%%%%%%%%%%%%%%%%%%%%%%%%%%%%%%%%%%%%%%%%%%%%%%%%%%%%%%%%%%%%%%%%%

{\color{change}
\begin{itemize}

\item Sie dürfen Ihre Arbeit sowohl auf Englisch als auch auf Deutsch verfassen!

\item Bitte schreiben Sie nur auf Englisch, wenn Sie wissen, was Sie tun! Der Korrekturaufwand kann für beide Seiten immens sein! Insbesondere gelten für die Zeichensetzung im Englischen andere Regeln als im Deutschen.

\end{itemize}
}

%%%%%%%%%%%%%%%%%%%%%%%%%%%%%%%%%%%%%%%%%%%%%%%%%%%%%%%%%%%%%%%%%%%%%%%%%%%%%%%%%%%%%%%%%%%%%%%%%%%%%%%%%%%%%%
\section{Iterationen}
%%%%%%%%%%%%%%%%%%%%%%%%%%%%%%%%%%%%%%%%%%%%%%%%%%%%%%%%%%%%%%%%%%%%%%%%%%%%%%%%%%%%%%%%%%%%%%%%%%%%%%%%%%%%%%

{\color{change}
\begin{itemize}

\item Üblicherweise erfolgt das Lesen und die Korrektur der Arbeit iterativ.
\item Bitte verwenden Sie das Makro \verb$\revision{...}$, um Änderungen \revision{farbig hervorzuheben}. Dieses Makro kann bei der nächsten Iteration mit dem C-Code \verb$cleantex.c$ aus dem \LaTeX-Dokument entfernt werden.

\end{itemize}
}