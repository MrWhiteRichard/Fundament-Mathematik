\documentclass[aspectratio=169]{beamer}



\mode<presentation>
{
 \usetheme[reversetitle,notitle,noauthor]{Wien}
%    \usetheme[noauthor]{Wien}
}

\usepackage{url}
\usepackage{graphicx}
\graphicspath{{./}{./Figures/}}

\usepackage{appendixnumberbeamer}
\usepackage{algorithm2e}
\usepackage{float}
\usepackage{tikz}
\usetikzlibrary{arrows.meta,positioning}

% To avoid a warning from the hyperref package:
\pdfstringdefDisableCommands{%
    \def\translate{}%
}

% To make sure, that the footnote is placed above and outside the
% footline (but it only works for one footnote per frame):
%
% \addtobeamertemplate{footnote}{}{\vspace{4ex}}

%%%%%%%%%%%%%%%%%%%%%%%%%%%%%%%%%%%%%%%%%%%%%%%%%%%%%%%%%%%%%%%%%%%%%%%%%%%%%
%%%%%%%%%%%%%%%%%%%%%%%%%%%%%%%%%%%%%%%%%%%%%%%%%%%%%%%%%%%%%%%%%%%%%%%%%%%%%
\title[Maximum Flow Problem]{Maximum Flow Problem}


\subtitle{Bachelorarbeit aus Diskreter Mathematik}

\author[F. Schager]{Florian Schager}

\institute[TU Wien]{TU Wien, Vienna, Austria}

\date{19. April 2021}

% Hier befinden sich Pakete, die wir beinahe immer benutzen ...

\usepackage[utf8]{inputenc}

% Sprach-Paket:
\usepackage[ngerman]{babel}

% damit's nicht so, wie beim Grill aussieht:
\usepackage{fullpage}

% Mathematik:
\usepackage{amsmath, amssymb, amsfonts, amsthm}
\usepackage{bbm, mathrsfs, stmaryrd}
\usepackage{mathtools, mathdots}

% Makros mit mehereren Default-Argumenten:
\usepackage{twoopt}

% Anführungszeichen (Makro \Quote{}):
\usepackage{babel}

% if's für Makros:
\usepackage{xifthen}
\usepackage{etoolbox}

% tikz ist kein Zeichenprogramm (doch!):
\usepackage{tikz}

% bessere Aufzählungen:
\usepackage{enumitem}

% (bessere) Umgebung für Bilder:
\usepackage{graphicx, subfig, float}

% Umgebung für Code:
\usepackage{listings}

% Farben:
\usepackage{xcolor}

% Umgebung für "plain text":
\usepackage{verbatim}

% Umgebung für mehrerer Spalten:
\usepackage{multicol}

% "nette" Brüche
\usepackage{nicefrac}

% Spaltentypen verschiedener Dicke
\usepackage{tabularx}
\usepackage{makecell}

% Für Vektoren
\usepackage{esvect}

% (Web-)Links
\usepackage{hyperref}

% Zitieren & Literatur-Verzeichnis
\usepackage[style = authoryear]{biblatex}
\usepackage{csquotes}

% so ähnlich wie mathbb
%\usepackage{mathds}

% Keine Ahnung, was das macht ...
\usepackage{booktabs}
\usepackage{ngerman}
\usepackage{placeins}

% ---------------------------------------------------------------- %
% Praetorius' macros

\def \revision #1 {{\color{red} #1}}

% ---------------------------------------------------------------- %
% my macros

\DeclareMathOperator{\dom}{dom}
\DeclareMathOperator{\ran}{ran}
\DeclareMathOperator{\supp}{supp}

\newcommand{\pbraces}[1]{{\left  ( #1 \right  )}}
\newcommand{\bbraces}[1]{{\left  [ #1 \right  ]}}
\newcommand{\Bbraces}[1]{{\left \{ #1 \right \}}}
\newcommand{\vbraces}[1]{{\left  | #1 \right  |}}
\newcommand{\Vbraces}[1]{{\left \| #1 \right \|}}

\newcommand{\Forall}{\forall \,}
\newcommand{\Exists}{\exists \,}

\DeclareMathOperator{\diag}{diag}

% ---------------------------------------------------------------- %

\theoremstyle{definition}

% unnumbered theorems
\newtheorem*{theorem*}    {Satz}
\newtheorem*{lemma*}      {Lemma}
\newtheorem*{corollary*}  {Korollar}
\newtheorem*{proposition*}{Proposition}
\newtheorem*{remark*}     {Bemerkung}
\newtheorem*{definition*} {Definition}
\newtheorem*{example*}    {Beispiel}

\renewcommand{\figurename}{Abbildung}
\renewcommand{\tablename} {Tabelle}


\begin{document}

\begin{frame}
    \titlepage
\end{frame}

%%%%%%%%%%%%%%%%%%%%%%%%%%%%%%%%%%%%%%%%%%%%%%%%%%%%%%%%%%%%%%%%%%%%%%%%%%%%%
%%%%%%%%%%%%%%%%%%%%%%%%%%%%%%%%%%%%%%%%%%%%%%%%%%%%%%%%%%%%%%%%%%%%%%%%%%%%%
%%%%%%%%%%%%%%%%%%%%%%%%%%%%%%%%%%%%%%%%%%%%%%%%%%%%%%%%%%%%%%%%%%%%%%%%%%%%%

\begin{frame}{Problemstellung}
  \begin{problem*}
    Wir wollen in einem Flussnetzwerk die maximale Transportkapazität
    von einem ausgezeichnetem Knoten, der Quelle $s$, zum Abflussknoten $t$ bestimmen.
  \end{problem*}

  \begin{remark*}
    Ursprünglich wurde das Problem in den 50ern zur Modellierung des sowjetischen Schienenverkehrs gestellt.

    Anwendungen reichen von der Fluglinienplanung
  \end{remark*}
\end{frame}


\begin{frame}{Definitionen (1/2)}
  \begin{definition*}[Flussnetzwerk]
    Ein Flussnetzwerk $N = (G,c,s,t)$ besteht aus einem gerichteter Graph $G = (V,E)$ mit einer
    Kapazitätsfunktion $c: E \to \R^+$ mit zwei ausgezeichneten Knoten $s$ und $t$,
    wobei $t$ von $s$ aus erreichbar sein soll.
  \end{definition*}

  \pause

  \begin{definition*}[Fluss]
    Ein zulässiger Fluss durch $N$ ist eine Funktion $f: E \to \R_0^+$, welche
    den folgenden Bedingungen genügt:
    \begin{enumerate}
      \item<3-> $0 \leq f(e) \leq c(e)$ for each edge $e$; \quad (Kapazitätsbeschränkung)
      \item<4> $\sum_{e^+ = v} f(e) = \sum_{e^- = v} f(e)$ für jeden Knoten $v \neq s,t$. \\
      Dabei bezeichnen $e^-$ und $e^+$ jeweils den Start- und Endpunkt der Kante $e$.
      \quad (Flusserhaltung)
    \end{enumerate}
  \end{definition*}
\end{frame}

\begin{frame}{Definitionen (2/2)}
  \begin{definition*}[Wert eines Flusses]
    Die Größe
    \begin{align*}
      w(f) := \sum_{e^- = s}f(e) - \sum_{e^+ = s} f(e) = \sum_{e^+ = t}f(e) - \sum_{e^- = t} f(e)
    \end{align*}
    nennen wir den Wert eines Flusses.
  \end{definition*}

  \pause

  \begin{definition*}[Cut]
    Ein Cut ist eine Partition $V = S \dot \cup T$ eines Flussnetzwerks $N = (G,c,s,t)$
    mit $s \in S$ und $t \in T$. Die Kapazität eines Cuts ist gegeben durch
    \begin{align*}
      c(S,T) = \sum_{e^- \in S, s^+ \in T} c(e).
    \end{align*}
    Ein Cut heißt minimal, wenn $c(S,T) \leq c(S',T')$ für alle Cuts $(S',T')$.
  \end{definition*}

  \end{frame}

  \begin{frame}{Beispiel}

  \begin{center}

    -- IN ARBEIT --

  \begin{tikzpicture}[
      mycircle/.style={
         circle,
         draw=black,
         fill=gray,
         fill opacity = 0.3,
         text opacity=1,
         inner sep=0pt,
         minimum size=20pt,
         font=\small},
      myarrow/.style={-Stealth},
      node distance=0.6cm and 1.2cm
      ]
      \node[mycircle] (c1) {$s$};
      \node[mycircle,below right=of c1] (c2) {$v_2$};
      \node[mycircle,right=of c2] (c3) {$v_4$};
      \node[mycircle,above right=of c1] (c4) {$v_1$};
      \node[mycircle,right=of c4] (c5) {$v_3$};
      \node[mycircle,below right=of c5] (c6) {$t$};

    \foreach \i/\j/\txt/\p in {% start node/end node/text/position
      c1/c2/8/below,
      c1/c4/11/above,
      c2/c3/11/below,
      c3/c6/4/below,
      c4/c5/12/above,
      c5/c6/15/above,
      c5/c2/4/below,
      c3/c5/7/below,
      c2.70/c4.290/1/below}
       \draw [myarrow] (\i) -- node[sloped,font=\small,\p] {\txt} (\j);


     % draw this outside loop to get proper orientation of 10
     \draw [myarrow] (c4.250) -- node[sloped,font=\small,above,rotate=180] {10} (c2.110);
    \end{tikzpicture}

  \end{center}

  \end{frame}

  \begin{frame}{Erweiternde Pfade}

  \begin{definition*}[Pfad]
    Wir definieren einen Pfad in einem gerichteten Graphen als eine Folge
    $(e_1,\dots,e_n)$ von Kanten, sodass die zugehörige Folge an ungerichteten
    Kanten ein Pfad in dem zugehörigen ungerichteten Graph ist.
    \pause


    Sei $(v_0,\dots,v_n)$ die zum Pfad zugehörige Kantenfolge.
    Dann heißt jede Kante der Form $v_{i-1}v_{i}$ eine Vorwärts-Kante
    und jede Kante der Form $v_{i}v_{i-1}$ eine Rückwärts-Kante.

  \end{definition*}

  \pause

  \begin{definition*}[Erweiternder Pfad]
    Ein erweiternder Pfad bezüglich einem Fluss $f$ in einem Flussnetzwerk
    $N = (G,c,s,t)$ ist ein Pfad $P$,
    sodass $f(e) < c(e)$ für alle Vorwärts-Kanten $e \in P$ und $f(e) > 0$
    für alle Rückwärts-Kanten $e \in P$ gilt.
  \end{definition*}

  \end{frame}

  \begin{frame}{Wichtige Resultate}

  \begin{theorem*}[Augmenting Path Theorem]
    Ein Fluss $f$ in einem Flussnetzwerk $N = (G,c,s,t)$ ist genau dann maximal,
    wenn es keine erweiternden Pfade bezüglich $f$ gibt.
  \end{theorem*}

  \pause

  \begin{theorem*}[Integral Flow Theorem]
    Sei $N = (G,c,s,t)$ ein Flussnetzwerk mit ausschließlich ganzzahligen Kapazitäten.
    Dann existiert ein maximaler Fluss, der ebenso nur aus ganzzahligen Werten $f(e)$
    besteht.
  \end{theorem*}

  \pause

  \begin{theorem*}[Max-Flow Min-Cut Theorem]
    Der maximale Wert eines Flusses in einem Flussnetzwerk $N$ entspricht
    der minimalen Kapazität eines Cuts in $N$.
  \end{theorem*}

  \end{frame}

  \begin{frame}{Algorithmus: Grundidee}

  \begin{algorithmus*}
    \begin{itemize}
      \item<1-> Starte mit dem trivialen Fluss: $f(e) = 0 , e \in E$.
      \item<2-> Finde einen erweiternden Pfad $P$.
      \item<3-> Berechne
      \begin{align*}
        d := \min[&\{c(e) - f(e): e \text{ Vorwärts-Kante } \in P\} \\ \cup
        &\{f(e): e \text{ Rückwärts-Kante } \in P\} ].
      \end{align*}
      \item<4-> Konstruiere erweiterten Fluss $f'$ mit $v(f') = v(f) + d$:
      \begin{align*}
        f'(e) = \begin{cases}
          f(e) + d, & e \text{ ist Vorwärts-Kante } \in P \\
          f(e) - d, & e \text{ ist Rückwärts-Kante } \in P\\
          f(e), & \text{ sonst}
        \end{cases}
      \end{align*}
      \item<5-> Wiederhole solange, bis kein erweiternder Pfad mehr gefunden werden kann.
    \end{itemize}
  \end{algorithmus*}

  \end{frame}

  \begin{frame}{Algorithmus: Ford-Fulkerson - Beispiel}

  \begin{center}

    -- IN ARBEIT --



  \begin{tikzpicture}[
      mycircle/.style={
         circle,
         draw=black,
         fill=gray,
         fill opacity = 0.3,
         text opacity=1,
         inner sep=0pt,
         minimum size=20pt,
         font=\small},
      myarrow/.style={-Stealth},
      node distance=0.6cm and 1.2cm
      ]
      \node[mycircle] (c1) {$s$};
      \node[mycircle,below right=of c1] (c2) {$v_2$};
      \node[mycircle,right=of c2] (c3) {$v_4$};
      \node[mycircle,right=of c3] (c7) {$v_5$};
      \node[mycircle,above right=of c1] (c4) {$v_1$};
      \node[mycircle,right=of c4] (c5) {$v_3$};
      \node[mycircle,right=of c5] (c8) {$v_6$};
      \node[mycircle,below right=of c8] (c6) {$t$};

    \foreach \i/\j/\txt/\p in {% start node/end node/text/position
      c1/c2/10/below,
      c1/c4/10/above,
      c2/c3/10/below,
      c3/c7/10/below,
      c4/c5/10/above,
      c5/c8/10/above,
      c8/c6/10/above,
      c7/c6/10/below,
      c5/c3/1/below}
       \draw [myarrow] (\i) -- node[sloped,font=\small,\p] {\txt} (\j);


     % draw this outside loop to get proper orientation of 10
     % \draw [myarrow] (c4.250) -- node[sloped,font=\small,above,rotate=180] {10} (c2.110);
    \end{tikzpicture}

  \end{center}

  \end{frame}

  \begin{frame}{Algorithmus: Ford-Fulkerson - Analyse}

  \begin{itemize}
    \item Nicht polynomielle Laufzeit, garantiert nicht effiziente Wahl der
    erweiternden Pfade.
    \pause
    \item Kann für irrationale Kapazitäten scheitern, ist aber in der Praxis irrelevant.
  \end{itemize}

  \pause

  \begin{remark*}[Modifikation: Edmonds und Karp]
    Wird die Reihenfolge, in der die Knoten mit Labeln versehen werden, gemäß
    einer Breitensuche gewählt, so lässt sich der Aufwand auf $O(|V||E|^2)$ reduzieren.
  \end{remark*}

  \end{frame}

  \begin{frame}{Vertiefung}

  \begin{anwendungen*}
    Welche Probleme können auf ein Maximum Flow Problem reduziert werden?
    \begin{itemize}
      \item Maximale Anzahl disjunkter Pfade von der Quelle zum Abfluss.
      \item Maximale Kardinalität eines Matching in einem bipartitem Graph und perfekte Matchings.
      \item Knotenüberdeckungsproblem.
      \item Circulation Problem.
      \item Matrix-Rundungsproblem.
    \end{itemize}

  \end{anwendungen*}

  \pause

  \begin{algorithmen*}
  \begin{itemize}
    \item Blocking Flows
    \item Push-Relabel-Algorithmus von Goldberg and Tarjan
  \end{itemize}

  \end{algorithmen*}



  \end{frame}

\end{document}


%%% Local Variables:
%%% mode: latex
%%% TeX-master: t
%%% End:
