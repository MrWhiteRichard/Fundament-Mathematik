\documentclass[aspectratio=169]{beamer}



\mode<presentation>
{
 \usetheme[reversetitle,notitle,noauthor]{Wien}
%    \usetheme[noauthor]{Wien}
}

\usepackage{url}
\usepackage{graphicx}
\graphicspath{{./}{./Figures/}}

\usepackage{appendixnumberbeamer}
\usepackage{algorithm2e}
\usepackage{float}
\usepackage{tikz}
\usetikzlibrary{arrows.meta,positioning}
\usetikzlibrary{positioning}
\usetikzlibrary{overlay-beamer-styles}

\tikzset{onslide/.code args={<#1>#2}{%
  \only<#1>{\pgfkeysalso{#2}} % \pgfkeysalso doesn't change the path
}}
\tikzset{temporal/.code args={<#1>#2#3#4}{%
  \temporal<#1>{\pgfkeysalso{#2}}{\pgfkeysalso{#3}}{\pgfkeysalso{#4}} % \pgfkeysalso doesn't change the path
}}

\tikzstyle{highlight}=[red,ultra thick]

% To avoid a warning from the hyperref package:
\pdfstringdefDisableCommands{%
    \def\translate{}%
}

% To make sure, that the footnote is placed above and outside the
% footline (but it only works for one footnote per frame):
%
% \addtobeamertemplate{footnote}{}{\vspace{4ex}}

%%%%%%%%%%%%%%%%%%%%%%%%%%%%%%%%%%%%%%%%%%%%%%%%%%%%%%%%%%%%%%%%%%%%%%%%%%%%%
%%%%%%%%%%%%%%%%%%%%%%%%%%%%%%%%%%%%%%%%%%%%%%%%%%%%%%%%%%%%%%%%%%%%%%%%%%%%%
\title[Zelluläre Automaten]{Design und Analyse eines Spiels \newline mithilfe von Zellulären Automaten}


\subtitle{Bachelorarbeit aus Diskreter Mathematik}

\author[C. Göth]{Christian Göth}

\institute[TU Wien]{TU Wien, Vienna, Austria}

\date{21. Juni 2021}

% Hier befinden sich Pakete, die wir beinahe immer benutzen ...

\usepackage[utf8]{inputenc}

% Sprach-Paket:
\usepackage[ngerman]{babel}

% damit's nicht so, wie beim Grill aussieht:
\usepackage{fullpage}

% Mathematik:
\usepackage{amsmath, amssymb, amsfonts, amsthm}
\usepackage{bbm, mathrsfs, stmaryrd}
\usepackage{mathtools, mathdots}

% Makros mit mehereren Default-Argumenten:
\usepackage{twoopt}

% Anführungszeichen (Makro \Quote{}):
\usepackage{babel}

% if's für Makros:
\usepackage{xifthen}
\usepackage{etoolbox}

% tikz ist kein Zeichenprogramm (doch!):
\usepackage{tikz}

% bessere Aufzählungen:
\usepackage{enumitem}

% (bessere) Umgebung für Bilder:
\usepackage{graphicx, subfig, float}

% Umgebung für Code:
\usepackage{listings}

% Farben:
\usepackage{xcolor}

% Umgebung für "plain text":
\usepackage{verbatim}

% Umgebung für mehrerer Spalten:
\usepackage{multicol}

% "nette" Brüche
\usepackage{nicefrac}

% Spaltentypen verschiedener Dicke
\usepackage{tabularx}
\usepackage{makecell}

% Für Vektoren
\usepackage{esvect}

% (Web-)Links
\usepackage{hyperref}

% Zitieren & Literatur-Verzeichnis
\usepackage[style = authoryear]{biblatex}
\usepackage{csquotes}

% so ähnlich wie mathbb
%\usepackage{mathds}

% Keine Ahnung, was das macht ...
\usepackage{booktabs}
\usepackage{ngerman}
\usepackage{placeins}

% ---------------------------------------------------------------- %
% Praetorius' macros

\def \revision #1 {{\color{red} #1}}

% ---------------------------------------------------------------- %
% my macros

\DeclareMathOperator{\dom}{dom}
\DeclareMathOperator{\ran}{ran}
\DeclareMathOperator{\supp}{supp}

\newcommand{\pbraces}[1]{{\left  ( #1 \right  )}}
\newcommand{\bbraces}[1]{{\left  [ #1 \right  ]}}
\newcommand{\Bbraces}[1]{{\left \{ #1 \right \}}}
\newcommand{\vbraces}[1]{{\left  | #1 \right  |}}
\newcommand{\Vbraces}[1]{{\left \| #1 \right \|}}

\newcommand{\Forall}{\forall \,}
\newcommand{\Exists}{\exists \,}

\DeclareMathOperator{\diag}{diag}

% ---------------------------------------------------------------- %

\theoremstyle{definition}

% unnumbered theorems
\newtheorem*{theorem*}    {Satz}
\newtheorem*{lemma*}      {Lemma}
\newtheorem*{corollary*}  {Korollar}
\newtheorem*{proposition*}{Proposition}
\newtheorem*{remark*}     {Bemerkung}
\newtheorem*{definition*} {Definition}
\newtheorem*{example*}    {Beispiel}

\renewcommand{\figurename}{Abbildung}
\renewcommand{\tablename} {Tabelle}


\begin{document}

\begin{frame}
    \titlepage
\end{frame}

%%%%%%%%%%%%%%%%%%%%%%%%%%%%%%%%%%%%%%%%%%%%%%%%%%%%%%%%%%%%%%%%%%%%%%%%%%%%%
%%%%%%%%%%%%%%%%%%%%%%%%%%%%%%%%%%%%%%%%%%%%%%%%%%%%%%%%%%%%%%%%%%%%%%%%%%%%%
%%%%%%%%%%%%%%%%%%%%%%%%%%%%%%%%%%%%%%%%%%%%%%%%%%%%%%%%%%%%%%%%%%%%%%%%%%%%%


  \begin{frame}{Wiederholung 1/2}
    \begin{block}{Definitionen (Bestandteile eines ZA)}
      \begin{itemize}
        \item \textit{Zellularraum} $\mathbb{Z}^{d}$: $d$-dimensionales unendliches Gitter
        \item \textit{Zustandsmenge} $S$: endliche Menge
        \item \textit{Nachbarschaft}: Vektor $N = (y_1,y_2,\dots,y_n)$ aus $n$ paarweise verschiedenen Elementen
        \item \textit{Lokale Update-Regel} $f: S^n \to S$
      \end{itemize}
    \end{block}

    \pause

    \begin{definition*}
      \begin{itemize}
        \item \textit{Konfiguration}: Abbildung $c: \mathbb{Z}^{d} \to S$
        \item Die Menge $S^{\mathbb{Z}^{d}}$ aller Konfigurationen heißt $\mathcal{C}(d,S) = \mathcal{C}$
      \end{itemize}
    \end{definition*}
  \end{frame}


  \begin{frame}{Eigenschaften des Spiels 1/3}
    \begin{block}{Zustandsmenge}
      Jeder Zustand besteht aus $4$ Komponenten $(\uparrow, \rightarrow, \downarrow, \leftarrow)$, die einer (theoretischen) Mehrfach-Belegung entsprechen:
      \begin{align*}
        & S := \{0,1\}^4 \cup \{0,2,3\}^{4} \\
        \Rightarrow & |S| = 2^4 + 3^4 -1 = 96
      \end{align*}

    \end{block}

    \pause

    \begin{block}{Nachbarschaft ($n=9$)}
      \begin{figure}[H]
          \centering
          \includegraphics[width = 0.2 \textheight]{neighborhood.png}
      \end{figure}

    \end{block}
  \end{frame}


  \begin{frame}{Eigenschaften des Spiels 2/3}
    \begin{block}{Lokale Update-Regel}
      Sehr komplex ($f: S^n \to S$ mit $|S^n| = 96^9$) und diverse Fallunterscheidungen. \\
      Daher nur zwei Beispiele.
    \end{block}

    \pause

    \begin{multicols*}{3}

      \begin{figure}[H]
        \centering
        \includegraphics[width = 0.39 \textheight]{example1n_1.png}
      \end{figure}

      \vfill\null

      \pause

      \begin{figure}[H]
        \centering
        \includegraphics[width = 0.39 \textheight]{example1n_2.png}
      \end{figure}

      \vfill\null

      \pause


      \begin{figure}[H]
        \centering
        \includegraphics[width = 0.39 \textheight]{example1_3.png}
      \end{figure}

    \end{multicols*}

  \end{frame}

  \begin{frame}{Eigenschaften des Spiels 3/3}
    \begin{block}{Lokale Update-Regel}
      Sehr komplex ($f: S^n \to S$ mit $|S^n| = 96^9$) und diverse Fallunterscheidungen. \\
      Daher nur zwei Beispiele.
    \end{block}

    \begin{multicols*}{3}

      \begin{figure}[H]
        \centering
        \includegraphics[width = 0.39 \textheight]{example2n_1.png}
      \end{figure}

      \vfill\null

      \pause

      \begin{figure}[H]
        \centering
        \includegraphics[width = 0.39 \textheight]{example2n_2.png}
      \end{figure}

      \vfill\null

      \pause


      \begin{figure}[H]
        \centering
        \includegraphics[width = 0.39 \textheight]{example2_3.png}
      \end{figure}

    \end{multicols*}

  \end{frame}



  \begin{frame}{Eigenschaften des Spiels (Ausblick)}
    Erweiterung von $S, N$ und $f$ bei Eingreifen des\slash der Spieler\textunderscore in.

    \vfill\null

    \begin{multicols*}{2}

      \begin{figure}[H]
        \centering
        \includegraphics[width = 0.5 \textheight]{example3n_1.png}
      \end{figure}


      \pause

      \begin{figure}[H]
        \centering
        \includegraphics[width = 0.5 \textheight]{example3n_2.png}
      \end{figure}

    \end{multicols*}

  \end{frame}

  \begin{frame}{Wiederholung 2/2}
    \begin{definition*}
      Wir identifizieren einen zellulären Automaten $(S,N,f)$ mit seiner \textit{Globalen Überführungsfunktion} $G: \mathcal{C} \to \mathcal{C}$ und sprechen vom ZA $G$.
    \end{definition*}

    \pause

    \begin{definition*}
      Ein ZA $G$ heißt bijektiv, falls $G$ als Funktion bijektiv ist.
    \end{definition*}

    \begin{definition*}
      Ein ZA $G$ heißt reversibel, falls ein ZA $H$ existiert mit $G \circ H = H \circ G = id$.
    \end{definition*}
  \end{frame}


  \begin{frame}{Definitionen}
    \begin{definition*}
      Auf der Menge $\mathcal{C} = S^{\mathbb{Z}^d}$ der Konfigurationen definieren wir nun eine Metrik. Für $c, e \in \mathcal{C}$ sei
      \begin{align*}
        d(c, e) = \begin{cases}
          0, & c = e \\
          2^{- \min \{ \|x\|_\infty: c(x) \neq e(x)\} }& \, c \neq e
        \end{cases}
      \end{align*}
    \end{definition*}

    \pause

    \begin{remark*}
      Es ist $d$ tatsächlich eine Metrik. Die induzierte Topologie entspricht der Produkt- topologie der diskreten Topologie von $S$ und macht $\mathcal{C}$ zu einem kompakten Raum.
    \end{remark*}

    \pause

    \begin{definition*}
      Für $y \in \mathbb{Z}^d$ definieren wir den ZA $\tau_{y} = (S, (-y), id)$ und nennen ihn \textit{Translation} um den Vektor $y$.
    \end{definition*}

  \end{frame}

  \begin{frame}{Curtis-Hedlund-Lyndon Theorem}
    \begin{theorem*}[Curtis-Hedlund-Lyndon]
      Eine Funktion $G: S^{\mathcal{Z}^d} \to S^{\mathcal{Z}^d}$ ist die Globale Überführungsfunktion eines ZA genau dann, wenn $G$ stetig ist und mit Translationen kommutiert.
    \end{theorem*}

    \pause

    \begin{corollary*}
      Ein ZA $G$ ist reversibel genau dann, wenn er bijektiv ist.
    \end{corollary*}

    \begin{proof}
      Definitionsgemäß folgt aus der Reversibilität die Bijektivität.
      Sie $G$ ein bijektiver ZA. Die Inverse $G^{-1}$ einer stetigen Bijektion zwischen zwei kompakten metrischen Räumen ist wieder stetig. Offensichtlich kommutiert $G^{-1}$ mit Translationen. Somit ist $G^{-1}$ die Funktion eines ZA.
    \end{proof}
  \end{frame}


  \begin{frame}{Injektivität in Zellulären Automaten}
    \begin{corollary*}
      Ein ZA $G$ ist reversibel genau dann, wenn er bijektiv ist.
    \end{corollary*}

    \pause

    \begin{theorem*}[Garden-of-Eden/ Moore and Myhill]
      Ein ZA $G$ ist surjektiv genau dann, wenn die Einschränkung auf endliche Konfigurationen $G_F$ injektiv ist.
    \end{theorem*}

    \pause

    Es gilt also für Zelluläre Automaten:
    \begin{align*}
      \text{Injektivität} \Leftrightarrow \text{Reversibilität}.
    \end{align*}
  \end{frame}


  \begin{frame}{\sout{Injektivität}}

    \begin{multicols*}{5}

      \onslide<1->\begin{figure}[H]
        \centering
        \includegraphics[width = 0.35 \textheight]{example2_2.png}
      \end{figure}

      \vfill\null

      \onslide<2->\centering{\Huge $\rightarrow$ \par}

      \vfill\null

      \onslide<2->\begin{figure}[H]
        \centering
        \includegraphics[width = 0.35 \textheight]{example2_3.png}
      \end{figure}

      \vfill\null



      \onslide<3->\centering{\Huge $\leftarrow$ \par}
      \vfill\null

      \onslide<3->\begin{figure}[H]
        \centering
        \includegraphics[width = 0.35 \textheight]{example4_2.png}
      \end{figure}

    \end{multicols*}
  \end{frame}


  \begin{frame}{Garden of Eden-Konfigurationen}
    \begin{figure}[H]
      \centering
      \includegraphics[width = 0.55 \textheight]{example5.png}
    \end{figure}
  \end{frame}


  \begin{frame}{}
    \begin{block}{}

      {
          \centering
          \huge
          Bei Interesse am Spiel gerne melden!
      }

    \end{block}

  \end{frame}


\end{document}


%%% Local Variables:
%%% mode: latex
%%% TeX-master: t
%%% End:
