\documentclass{article}

\usepackage[utf8]{inputenc}
\usepackage{fullpage}
\usepackage[ngerman]{babel}
\usepackage{csquotes}

\title{Essay Gendergap und die Mathematik bzw. Geodäsie in der Ausbildung}
\author{Richard Weiss}
\date{29.4.2021}

\begin{document}

\maketitle

\section*{Einleitung}

Ich schicke voraus, dass wir Studierenden bereits knapp die Hälfte unseres Bachelorstudiums im Distance Learning verbracht haben.
Dadurch hat sich der Kontakt zu einander grundlegend verändert (im Wesentlichen drastisch reduziert).

Deshalb sehe ich mich nicht in einer Position, in der ich über den Gendergap an der TU Wien üppig aus Eigenerfahrung berichten kann.
Darüber hinaus, wenn ich an den regulären Uni-Betrieb zurückdenke, fallen mir auch keine Situationen ein, in denen sich die Punkte aus \blockquote{Renate Tanzbergers Blitzlichter Mathematik erlernen} deutlich wieder spiegeln.
Damit möchte ich aber keinesfalls Behaupten, dass es sie nicht grundsätzlich gegeben hätte (bzw. in der \blockquote{Distance-Epoche} weiterhin gäbe).

Dennoch fallen mir zwei Sachen ein, die meines Erachtens nach Wert sind, zu erwähnen.

\section*{Frauenanteil}

Die Tatsache, siehe Seite 13 in \blockquote{TU Wien Frauen und Männer 2019}, dass der Frauenanteil an der Fakultät für Mathematik und Geoinformation deutlich geringer als der Männeranteil ist, fällt mir natürlich schon auf.
Konsequenterweise, hatte ich bis jetzt erst in 3 LVAs, die von Mathematikerinnen geleitet wurden:

\begin{itemize}
	\item \blockquote{Diskrete und Geometrische Algorithmen UE},
	\item \blockquote{Einführung in die Statistik VO}, und
	\item \blockquote{Einführung in die Statistik UE}.
\end{itemize}

Alle dieser LVAs waren (sind) Distance-Learning LVAs.
Alle anderen Übungen und Vorlesungen, die ich besucht habe (besuche), wurden (werden) von Mathematikern geleitet.
Nachdem ich ja schon im 6. Semester bin, sind das dann doch auch Einige.

Ich finde es interessant, dass gerade die LVAs \blockquote{Einführung in die Statistik} von Mathematikerinnen geleitet werden, und nicht mathematisch \blockquote{tiefere} LVAs, wie \blockquote{Funktionalanalysis 1 \& 2} oder \blockquote{Logik und Grundlagen der Mathematik}.
Nach dem bösartigen Motto:
\blockquote{Statistik ist so anwendungsorientiert, das ist ja nicht mehr echte Mathematik!},
könnte man sich fragen, ob da etwas dran ist.
Eine durch \blockquote{Renate Tanzbergers Blitzlichter Mathematik erlernen} motivierte Aussage eines Teufels Advokaten könnte demnach lauten:
\blockquote{Dann lassen wir die Mädels halt Mathematik machen, aber echte Mathematik, das machen nur wir Männer}.
Das \blockquote{wir} in diesem Satz (den selbstverständlich niemand Anständiger jemals explizit aussprechen würde) muss man dabei wohlgemerkt nicht zwangsläufig miteinbeziehen, weil ja Männer und Frauen gleichermaßen ge-bias-ed sind.

Ich hatte das Vergnügen, die LVA \blockquote{Logik und Grundlagen der Mathematik}, von Prof. Goldstern, zu besuchen.
Das war ein Wahlfach.
Nur Männer haben dieses Wahlfach gewählt.
Seite 13 in \blockquote{TU Wien Frauen und Männer 2019} zur Folge, hätte ein Drittel der teilnehmenden Studierenden weiblich sein sollen.
Deren Anteil war aber gleich null.

\section*{Gendern}

Wir  bleiben bei der LVA \blockquote{Logik und Grundlagen der Mathematik}.
Aus dem zugehörigen Skriptum (Seite 23), sowie in jenem der LVAs \blockquote{Algebra 1 \& 2} (Seite 113) möchte ich folgende Fußnote zitieren:

\begin{center}
	Der \blockquote{Leser} ist als generisches Maskulinum zu verstehen, d.h. es sind weibliche ebenso wie männliche Leser gemeint, sowie auch small furry creatures from Alpha Centauri.
\end{center}

Diese Fußnote ist ein klassisches Beispiel einer gewissen Einstellung zur Gleichstellung der Geschlechter:
Man ist zwar grundsätzlich daran interessiert, stellt aber die Bedeutung gewisser Maßnahmen diesbezüglich in Frage.
Es steht aber auch fest, dass Autor\_innen solcher Fußnoten eindeutig beim TUWEL GenderAwareness-Test bei der 2. Frage falsch geantwortet hätten.
Die Menge der \blockquote{Individuen aller Geschlechter} beinhaltet schließlich nicht bloß Frauen, Männer und Aliens.
So wie auch in dem Beispiel eines Empfehlungsschreibens, denke ich, dass so etwas nicht böse gemeint ist;
man nimmt die Maßnahme des Genderns einfach nicht so ernst, und drückt dies auf einem entsprechenden \blockquote{Ernsthaftigkeits-Level} aus.
Häufige Pleonasmen wie \blockquote{Frau Lehrerin} helfen dem Gendern nicht dabei, seriöser zu wirken.
Ich stimme dem auch zu, dass Gendern allein nicht das Zentrum der Gleichstellung der Geschlechter darstellt.
Trotzdem glaube ich, dass es nicht schadet, sich soweit zu informieren, dass man eine solche Fußnote nicht schreibt.

\section*{Konklusio}

Die Gendergap an der TU lässt sich aus meiner persönlichen Perspektive nicht so eindeutig ausmachen wie bei Pubertierenden (15/16-jährigen) Schülern, die an der, in \blockquote{Renate Tanzbergers Blitzlichter Mathematik erlernen} erwähnten, PISA-Studie teilgenommen haben.
Das mag womöglich an der bereits hinreichend vorhandenen Reife der Kolleg\_innen liegen.
Diese ist aber im Bereich der Gleichstellung der Geschlechter noch immer nicht soweit ausgeprägt, dass mir keine Negativbeispiele aufgefallen sind.
Ich hoffe und vertraue darauf, dass diese LVA \blockquote{Technik für Menschen} seinen Beitrag leisten wird, um das zukünftig zu ändern.

\end{document}
