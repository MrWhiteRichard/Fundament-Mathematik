\documentclass{article}

\usepackage[utf8]{inputenc}
\usepackage{fullpage}
\usepackage[ngerman]{babel}
\usepackage{csquotes}

\title{Essay Gendergap und die Mathematik bzw. Geodäsie in der Ausbildung}
\author{Richard Weiss}
\date{29.4.2021}

\begin{document}

\maketitle

\section*{Einleitung}

Ich schicke voraus, dass wir Student\_innen bereits knapp die Hälfte unseres Bachelorstudiums im Distance Learning verbracht haben.
Dadurch hat sich der Kontakt zu einander grundlegend verändert (im Wesentlichen drastisch reduziert).

Deshalb sehe ich mich nicht in einer Position, in der ich über den Gendergap an der TU Wien üppig aus Eigenerfahrung berichten kann.
Darüber hinaus, wenn ich an den regulären Uni-Betrieb zurückdenke, fallen mir auch keine Situationen ein, in denen sich die Punkte aus \enquote{Renate Tanzbergers Blitzlichter Mathematik erlernen} deutlich wieder spiegeln.
Damit möchte ich aber keinesfalls behaupten, dass es sie nicht grundsätzlich gegeben hätte (bzw. in der \enquote{Distance-Epoche} weiterhin gäbe).

Dennoch fallen mir zwei Sachen ein, die es meines Erachtens nach Wert sind, zu erwähnen.

\section*{Frauenanteil}

Die Tatsache, siehe Seite 13 in \enquote{TU Wien Frauen und Männer 2019}, dass der Frauenanteil an der Fakultät für Mathematik und Geoinformation deutlich geringer als der Männeranteil ist, fällt mir natürlich schon auf.
Zum Beispiel dadurch, dass ich bis jetzt erst in 3 LVAs hatte (habe), die von Mathematikerinnen geleitet wurden (werden):

\begin{itemize}
	\item \enquote{Diskrete und Geometrische Algorithmen UE},
	\item \enquote{Einführung in die Statistik VO}, und
	\item \enquote{Einführung in die Statistik UE}.
\end{itemize}

Alle dieser LVAs waren (sind) Distance-Learning LVAs.
Alle anderen Übungen und Vorlesungen, die ich besucht habe (besuche), wurden (werden) von Mathematikern geleitet.
Nachdem ich ja schon im 6. Semester bin, sind das dann doch auch Einige.

Ich finde es interessant, dass gerade die LVAs \enquote{Einführung in die Statistik} von Mathematikerinnen geleitet werden, und nicht mathematisch \enquote{tiefere} LVAs, wie \enquote{Funktionalanalysis 1 \& 2} oder \enquote{Logik und Grundlagen der Mathematik}.
Aufbauend auf der bösartigen Aussage:
\enquote{Statistik ist so anwendungsorientiert, das ist ja nicht mehr echte Mathematik!},
könnte man sich fragen, ob da tatsächlich etwas dran ist.
Eine weitere Aussage eines Teufels Advokaten könnte demnach lauten:
\enquote{Dann lassen wir die Mädels halt Mathematik machen, aber echte Mathematik, das machen nur wir Männer}.
Das \enquote{wir} in diesem Satz (den selbstverständlich niemand Anständiger jemals explizit aussprechen würde) muss man dabei wohlgemerkt nicht zwangsläufig miteinbeziehen, weil ja Männer und Frauen gleichermaßen ge-bias-ed sind.

Ich hatte das Vergnügen, die LVA \enquote{Logik und Grundlagen der Mathematik}, von Prof. Goldstern, zu besuchen.
Das war ein Wahlfach.
Nur Männer haben dieses Wahlfach gewählt.
Seite 13 in \enquote{TU Wien Frauen und Männer 2019} zur Folge, hätte ein Drittel der teilnehmenden Student\_innen weiblich sein sollen.
Deren Anteil war aber gleich null.

\section*{Gendern}

Wir  bleiben bei der LVA \enquote{Logik und Grundlagen der Mathematik}.
Aus dem zugehörigen Skriptum (Seite 23), sowie in jenem der LVAs \enquote{Algebra 1 \& 2} (Seite 113) möchte ich folgende Fußnote zitieren:

\begin{displayquote}
	Der \enquote{Leser} ist als generisches Maskulinum zu verstehen, d.h. es sind weibliche ebenso wie männliche Leser gemeint, sowie auch small furry creatures from Alpha Centauri.
\end{displayquote}

Es wird also bereits im Vorhinein angekündigt, dass man ohne böse Absicht, auf das Gendern verzichten möchte.
Die Fußnote ist wahrscheinlich deswegen so zünisch formuliert, weil der Autor der Meinung ist, dass Gendern nichts weiter als ein sperriger Formalismus sei.
Ich stimme dem auch zu, dass Gendern allein nicht das Zentrum der Gleichstellung der Geschlechter darstellen sollte.
Häufige Pleonasmen wie \enquote{Frau Lehrerin} helfen dem Gendern auch nicht dabei, seriöser zu wirken.
Beim ersten Mal lesen habe ich mir dabei also auch nichts weiter gedacht.

Mittlerweile hat sich meine Meinung dazu aber etwas geändert.
Die Menge der \enquote{Individuen aller Geschlechter} besteht nicht ausschließlich aus Frauen, Männern und Aliens;
sie würde ihrem Namen sonst nicht gerecht werden.
Wenn man es genau nimmt, wurde hier diskriminiert;
witzigerweise gerade dabei, wie man versprechen wollte nicht zu diskriminieren.

So wie auch in dem Beispiel eines Empfehlungsschreibens, denke ich, dass so etwas nicht böse gemeint ist.
Man nimmt Gendern nicht so ernst, und drückt das mit einer entsprechenden Ernsthaftigkeit aus.
Trotzdem, wenn man schon aktiv darauf eingeht und ankündigt, eine Regel zu ignorieren, dann sollte man sich mit ihr doch wenigstens etwas auskennen.

\section*{Konklusio}

Die Gendergap an der TU lässt sich aus meiner persönlichen Perspektive nicht so eindeutig ausmachen wie bei den pubertierenden (15/16-jährigen) Schüler\_innen, die an der, in \enquote{Renate Tanzbergers Blitzlichter Mathematik erlernen} erwähnten, PISA-Studie teilgenommen haben.
Das mag an der bereits ausreichend vorhandenen Reife der Kolleg\_innen liegen.
Diese ist aber im Bereich der Gleichstellung der Geschlechter noch immer nicht soweit ausgeprägt, dass mir keine Negativbeispiele aufgefallen sind.
Ich hoffe und vertraue darauf, dass diese LVA \enquote{Technik für Menschen} seinen Beitrag leisten wird, um das zukünftig zu bessern.

\end{document}
