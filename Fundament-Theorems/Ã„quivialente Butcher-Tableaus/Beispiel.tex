\begin{example*}

Betrachte die folgenden Butcher-Tableaus.

\begin{align*}
  \begin{array}{c|c}
    1 & 1/2 \\
    \hline
      & 1
  \end{array},
  \quad
  \begin{array}{c|c c}
    1 & 1/2 & 0 \\
    1 & 1/2 & 0 \\
    \hline
      & 1/2 & 1/2
  \end{array},
  \quad
  \begin{array}{c|c c}
    1 & 1/8 & 3/8 \\
    1 & 1/8 & 3/8 \\
    \hline
      & 1/2 & 1/2
  \end{array}
\end{align*}

Das Erste liefert

\begin{align*}
  \Phi_1(t, y, h)
  =
  k
  =
  f \pbraces{t + h, y + h \frac{k}{2}}.
\end{align*}

Das Zweite liefert

\begin{align*}
  \Phi_2(t, y, h)
  =
  \frac{k_1 + k_2}{2}
  =
  \cdots, \\
  k_1 = f \pbraces{t + h, y + h \frac{k_1}{2}}, \\
  k_2 = f \pbraces{t + h, y + h \frac{k_2}{2}}.
\end{align*}

\begin{align*}
  \implies
  k_1 = k_2 = k
  \implies
  \cdots
  =
  k
\end{align*}

Das Dritte liefert

\begin{align*}
  \Phi_3(t, y, h)
  =
  \frac{\tilde{k}_1 + \tilde{k}_2}{2}
  =
  \cdots, \\
  \tilde{k}_1
  =
  f \pbraces
  {
    t + h,
    y + h \Frac{8}{3 \tilde{k}_1 + \tilde{k}_2}
  }, \\
  \tilde{k}_2
  =
  f \pbraces
  {
    t + h,
    y + h \Frac{8}{3 \tilde{k}_1 + \tilde{k}_2}
  }.
\end{align*}

\begin{align*}
  \implies
  \tilde{k}_1 = \tilde{k}_2 =: \tilde{k}
  \implies
  \tilde{k}
  =
  f (
    t + h,
    y + h \tilde{k} \underbrace{(3/8 + 1/8)}_{1/2}
  )
  \implies
  \tilde{k} = k = \cdots
\end{align*}

Wir erhalten also $\Phi_1 = \Phi_2 = \Phi_3$.
Die Butcher-Tableaus sind somit äquivalent.

\end{example*}
