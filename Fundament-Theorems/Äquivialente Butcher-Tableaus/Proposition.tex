\begin{proposition*}

Seien folgende Butcher-Tableaus von den Größen $m_1$ bzw. $m_2$.

\begin{align*}
  \begin{array}{c|c}
    c^{(1)} & A^{(1)} \\
    \hline
            & b^{(1)}
  \end{array},
  \quad
  \begin{array}{c|c}
    c^{(2)} & A^{(2)} \\
    \hline
            & b^{(2)}
  \end{array}
\end{align*}

Diese sind äquivalen, wenn Folgendes gilt.
Es gibt Zerlegungen $(I_\ell^{(1)})_{\ell = 1}^n$ und $(I_\ell^{(2)})_{\ell = 1}^n$ von

\begin{align*}
  \Bbraces{1, \ldots, m_1}
  =
  \bigcup_{\ell = 1}^n I_\ell^{(1)},
  ~\text{bzw.}~
  \Bbraces{1, \ldots, m_2}
  =
  \bigcup_{\ell = 1}^n I_\ell^{(2)},
\end{align*}

mit element-weise

\begin{align*}
  I_1^{(1)}
  <
  \cdots
  <
  I_n^{(1)},
  ~\text{und}~
  I_1^{(2)}
  <
  \cdots
  <
  I_n^{(2)},
\end{align*}

sodass $\Forall \ell = 1, \ldots, n:$

\begin{enumerate}

  \item
  $c_{i_1}^{(1)} = c_{i_1}^{(2)} =: \gamma_\ell$
  für alle
  $i_1 \in I_\ell^{(1)}$ und $i_2 \in I_\ell^{(2)}$
  gleich ist,

  \item
  $\alpha_\ell^{(1)} := A_i^{(1)}$
  und
  $\alpha_\ell^{(2)} := A_i^{(2)}$
  für alle
  $i \in I_\ell^{(1)}$ bzw $i \in I_\ell^{(2)}$
  gleich ist \footnotemark, und

  \footnotetext{Mit $A_j$, ist die $i$-te Zeile von $A$ gemeint (so, wie in NumPy und SymPy).}

  \begin{align*}
    \sum_{i \in I_l^{(1)}}
    (\alpha_\ell^{(1)})_i
    =
    \sum_{i \in I_l^{(2)}}
    (\alpha_\ell^{(2)})_i
    =:
    a_{l, \ell},
    \quad
    l = 1, \ldots, n,
  \end{align*}

  sowie

  \item
  \begin{align*}
    \sum_{i \in I_\ell^{(1)}}
    b_i^{(1)}
    =
    \sum_{i \in I_\ell^{(2)}}
    b_i^{(2)}.
  \end{align*}

\end{enumerate}

\end{proposition*}

\begin{proof}

Die erzeugten Einschritt-Funktionen sehen wie folgt aus.

\begin{align*}
  \Phi_1(t, y, h)
  =
  \sum_{i=1}^{m_1}
  b_i^{(1)} k_i^{(1)},
  \quad
  k_i^{(1)}
  =
  f \pbraces
  {
    t + c_i^{(1)} h,
    y + h \sum_{j=1}^{m_1} A_{i j}^{(1)} k_j^{(1)}
  },
  \quad
  i = 1, \ldots, m_1, \\
  \Phi_2(t, y, h)
  =
  \sum_{i=1}^{m_2}
  b_i^{(2)} k_i^{(2)},
  \quad
  k_i^{(2)}
  =
  f \pbraces
  {
    t + c_i^{(2)} h,
    y + h \sum_{j=1}^{m_2} A_{i j}^{(2)} k_j^{(2)}
  },
  \quad
  i = 1, \ldots, m_2
\end{align*}

Wir wollen nun zeigen, dass $\Phi_1 = \Phi_2$.
Dazu, zeigen wir zuerst die Gleichheit der Inkremente in einer beliebigen Zerlegungs-Menge.
Sei also $\ell = 1, \ldots, n$.
Seinen weiters $i_\text{min} := \min I_\ell^{(1)}$ und $i_\text{max} := \max I_\ell^{(1)}$.

\begin{align*}
  \implies
  k_{i_\text{min}}^{(1)}
  =
  \cdots
  =
  k_i^{(1)}
  =
  f \bigg(
    t + c_i^{(1)} h,
    y + h \sum_{j=1}^{m_1}
    \underbrace
    {A_{i j}^{(1)}}_{(\alpha_l^{(1)})_j}
    k_j^{(1)}
  \bigg)
  =
  \cdots
  =
  k_{i_\text{max}}^{(1)}
  =:
  K_\ell^{(1)}
\end{align*}

Das sieht man analog für $I_\ell^{(2)}$ und bekommt $K_\ell^{(2)}$.
Wir zeigen jetzt noch, dass sogar $K_\ell^{(1)} = K_\ell^{(2)}$.
Seien dazu $i_1 \in I_\ell^{(1)}$ und $i_2 \in I_\ell^{(2)}$.

\begin{align*}
  \implies
  \sum_{j=1}^{m_1} A_{i_1, j}^{(1)} k_j^{(1)}
  =
  \sum_{l=1}^n
  \sum_{j \in I_l^{(1)}}
  A_{i_1, j}^{(1)}
  \underbrace{k_j^{(1)}}_{K_l^{(1)}}
  =
  \sum_{l=1}^n
  K_l^{(1)}
  \underbrace
  {
    \sum_{j \in I_l^{(1)}}
    \underbrace{A_{i_1, j}^{(1)}}_{(\alpha_\ell^{(1)})_j}
  }_{a_{l, \ell}}
\end{align*}

Analoges sieht man mit $i_2$.

\begin{align*}
  \implies
  K_\ell^{(1)}
  & =
  f \pbraces
  {
    t + \gamma_\ell h,
    y + h \sum_{l=1}^n K_l^{(1)} a_{l, \ell}
  }, \\
  K_\ell^{(2)}
  & =
  f \pbraces
  {
    t + \gamma_\ell h,
    y + h \sum_{l=1}^n K_l^{(2)} a_{l, \ell}
  }
\end{align*}

Weil Inkremente eindeutig sind, folgt $K_\ell^{(1)} = K_\ell^{(2)} =: K_\ell$.

\begin{align*}
  \implies
  \Phi_1(t, y, h)
  =
  \sum_{\ell = 1}^n
  \sum_{i \in I_\ell^{(1)}}
  b_i^{(1)} k_i^{(1)}
  =
  \sum_{\ell = 1}^n
  K_\ell
  \sum_{i \in I_\ell^{(1)}}
  b_i^{(1)}
  =
  \sum_{\ell = 1}^n
  K_\ell
  \sum_{i \in I_\ell^{(2)}}
  b_i^{(2)}
  =
  \sum_{\ell = 1}^n
  \sum_{i \in I_\ell^{(1)}}
  b_i^{(2)} k_i^{(2)}
  =
  \Phi_2(t, y, h)
\end{align*}

\end{proof}
