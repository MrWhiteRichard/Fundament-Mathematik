\begin{lemma*}[Rechenregeln des Verallgemeinerten Kreuzprodukts]

Sei $\times$ das Verallgemeinerte Kreuzprodukt in $\R^n$, mit $n \in \N$.

\begin{enumerate}[label = (\roman*)]

  \item
  $\times$ verschwindet für linear abhängige Argumente.

  \item
  $\times$ ist schiefsymmetrisch.
  D.h.
  Seien $n \in \N$ und $v^{(1)}, \ldots, v^{(n-1)} \in \R^n$.
  Dann gilt $\Forall i, j = 1, \ldots, n-1, i \neq j:$

  \begin{align*}
    \times(v^{(1)}, \ldots, v^{(i)}, \ldots, v^{(j)}, \ldots, v^{(n-1)})
    =
    -
    \times(v^{(1)}, \ldots, v^{(j)}, \ldots, v^{(i)}, \ldots, v^{(n-1)}).
  \end{align*}

  \item
  $\times$ ist multi-linear.
  Insbesondere, ist es distributiv bezüglich Vektor-Addition $+$ und Skalar-Multiplikation $\cdot$.

  \item
  Seien $n \in \N$ und $v^{(1)}, \ldots, v^{(n-1)} \in C^1(D, \R^n)$, mit $D \subseteq \R$ offen.
  Dann gilt die Produktregel.

  \begin{align*}
    \frac{d}{dt}
    \times(v^{(1)}(t), \ldots, v^{(n)}(t))
    =
    \sum_{i=1}^{n-1}
    \times \pbraces
    {
      v^{(1)}(t),
      \ldots,
      \frac{d}{dt}
      v^{(j)}(t),
      \ldots,
      v^{(n)}(t)
    }
  \end{align*}

\end{enumerate}

All diese Rechenregeln gelten insbesondere für das ursprüngliche Kreuzprodukt.

\end{lemma*}

\begin{proof}

Diese Eigenschaften (i)-(iii) folgen unmittelbar aus denen von Determinanten-Formen.
Für (iv) gehen wir wie folgt vor.
Sei $i = 1, \ldots, n$ und die Matrix-Funktion

\begin{align*}
  M := (v^{(1)}, \ldots, v^{(n-1)}, e_i)
  \in C^1(D, \R^{n \times n}).
\end{align*}

Laut der Leibnitz-Formel für Determinanten-Formen, und der Tatsache, dass in der letzten Spalte von $M$ ein Einheitsvektor steht, gilt Folgendes.

\begin{align*}
  \pbraces
  {
    \frac{d}{dt}
    \times(v^{(1)}(t), \ldots, v^{(n)}(t))
  }_i
  & =
  \frac{d}{dt}
  \pbraces
  {
    \times(v^{(1)}(t), \ldots, v^{(n)}(t))
  }_i
  =
  \frac{d}{dt}
  \det{M(t)}
  =
  \frac{d}{dt}
  \sum_{\sigma \in S_n}
  \sgn{\sigma}
  \prod_{k=1}^n
  m_{\sigma(k), k}(t) \\
  & =
  \frac{d}{dt}
  \sum_{\substack{\sigma \in S_n \\ \sigma(n) = i}}
  \sgn{\sigma}
  \prod_{k=1}^{n-1}
  m_{\sigma(k), k}(t)
  =
  \sum_{\substack{\sigma \in S_n \\ \sigma(n) = i}}
  \sgn{\sigma}
  \frac{d}{dt}
  \prod_{k=1}^{n-1}
  m_{\sigma(k), k}(t) \\
  & =
  \sum_{\substack{\sigma \in S_n \\ \sigma(n) = i}}
  \sgn{\sigma}
  \sum_{\ell = 1}^{n-1}
  \pbraces
  {
    \prod_{\substack{k=1 \\ k \neq \ell}}^{n-1}
    m_{\sigma(k), k}(t)
  }
  \frac{d}{dt}
  m_{\sigma(\ell), \ell}(t) \\
  & =
  \sum_{\ell = 1}^{n-1}
  \sum_{\substack{\sigma \in S_n \\ \sigma(n) = i}}
  \sgn{\sigma}
  \pbraces
  {
    \prod_{\substack{k=1 \\ k \neq \ell}}^{n-1}
    m_{\sigma(k), k}(t)
  }
  \frac{d}{dt}
  m_{\sigma(\ell), \ell}(t) \\
  & =
  \sum_{\ell = 1}^{n-1}
  \sum_{\sigma \in S_n}
  \sgn{\sigma}
  \pbraces
  {
    \prod_{\substack{k=1 \\ k \neq \ell}}^n
    m_{\sigma(k), k}(t)
  }
  \frac{d}{dt}
  m_{\sigma(\ell), \ell}(t) \\
  & =
  \sum_{\ell = 1}^{n-1}
  \det \pbraces
  {
    v^{(1)}(t),
    \ldots,
    \frac{d}{dt}
    v^{(\ell)}(t),
    \ldots,
    v^{(n)}(t)
  } \\
  & =
  \sum_{\ell = 1}^{n-1}
  \pbraces
  {
    \times \pbraces
    {
    v^{(1)}(t),
    \ldots,
    \frac{d}{dt}
    v^{(\ell)}(t),
    \ldots,
    v^{(n)}(t)
    }
  }_i \\
  & =
  \pbraces
  {
    \sum_{\ell = 1}^{n-1}
    \times \pbraces
    {
    v^{(1)}(t),
    \ldots,
    \frac{d}{dt}
    v^{(\ell)}(t),
    \ldots,
    v^{(n)}(t)
    }
  }_i
\end{align*}

\end{proof}
