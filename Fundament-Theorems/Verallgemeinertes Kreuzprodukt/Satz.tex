\begin{theorem*}[Orthogonalität des Verallgemeinerten Kreuzprodukts]

Seien $n \in \N$ und $v^{(1)}, \ldots, v^{(n-1)} \in \R^n$.
Dann steht $v^{(n)} := \times(v^{(1)}, \ldots, v^{(n-1)})$ orthogonal auf die anderen Vektoren.
D.h.

\begin{align*}
  \Forall i = 1, \ldots, n-1:
  (v^{(n)}, v^{(i)}) = 0.
\end{align*}

\end{theorem*}

\begin{proof}

Seien $i = 1, \ldots, n-1$ und $A_i = (v^{(1)}, \ldots, v^{(n-1)}, v^{(i)})$.
Weil bei den Vektoren, die $A_i$ bilden, $v^{(i)}$ doppelt vorkommt, sind diese linear abhängig.
Also, ist $A_i$ singulär und $\det{A_i} = 0$.

\begin{align*}
  v^{(n)}_j
  :=
  \begin{vmatrix}
    v^{(1)}_1 & v^{(2)}_1 & \cdots & v^{(n-1)}_1 & \delta_{1 j} \\
    v^{(1)}_2 & v^{(2)}_2 & \cdots & v^{(n-1)}_2 & \delta_{2 j} \\
    \vdots    & \vdots    & \ddots & \vdots      & \vdots \\
    v^{(1)}_n & v^{(2)}_n & \cdots & v^{(n-1)}_n & \delta_{n j}
  \end{vmatrix},
  \quad
  j = 1, \ldots, n,
  \quad
  \det{A_i}
  =
  \begin{vmatrix}
    v^{(1)}_1 & v^{(2)}_1 & \cdots & v^{(n-1)}_1 & v^{(i)}_1 \\
    v^{(1)}_2 & v^{(2)}_2 & \cdots & v^{(n-1)}_2 & v^{(i)}_2 \\
    \vdots    & \vdots    & \ddots & \vdots      & \vdots \\
    v^{(1)}_n & v^{(2)}_n & \cdots & v^{(n-1)}_n & v^{(i)}_n
  \end{vmatrix}
  = 0
\end{align*}

Wegen der Mulitlinearität von Determinanten-Formen, erhalten damit $\orthogonal{v^{(n)}}{v^{(i)}}$.

\begin{align*}
  (v^{(n)}, v^{(i)})
  =
  \sum_{j=1}^n
  v^{(n)}_j
  v^{(i)}_j
  =
  \det{A_i} = 0
\end{align*}

\end{proof}
