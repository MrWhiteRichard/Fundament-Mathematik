\begin{definition*}[Verallgemeinertes Kreuzprodukt]

Seien $n \in \N$ und $v^{(1)}, \ldots, v^{(n-1)} \in \R^n$.
Mögen $e_1, \ldots, e_n \in \R^n$ die kanonischen Basis-Vektoren, und $(\cdot, \cdot)$ das kanonische Skalarprodukt bezeichnen.

\begin{align*}
  v^{(n)}_i := \det{(v^{(1)}, \ldots, v^{(n-1)}, e_i)},
  \quad
  i = 1, \ldots, n
\end{align*}

Das Verallgemeinerte Kreuzprodukt sei damit wie folgt definiert.

\begin{align*}
  \times:
  (\R^n)^{n-1} \to \R^n:
  (v^{(1)}, \ldots, v^{(n-1)})
  \mapsto
  v^{(n)}
\end{align*}

Insbesondere, entspricht $\times$ für $n = 3$ dem ursprünglichen Kreuzprodukt.

\end{definition*}
