\begin{remark*}

Seien $a, b \in \R^3$ und

\begin{align*}
  c_i
  :=
  \begin{vmatrix}
    a_1 & b_1 & \delta_{1 i} \\
    a_2 & b_2 & \delta_{2 i} \\
    a_3 & b_3 & \delta_{3 i}
  \end{vmatrix},
  \quad
  i = 1, 2, 3.
\end{align*}

Dann gilt für das Kreuzprodukt

\begin{align*}
  a \times b
  =
  \begin{pmatrix}
    a_1 \\
    a_2 \\
    a_3
  \end{pmatrix}
  \times
  \begin{pmatrix}
  b_1 \\
  b_2 \\
  b_3
  \end{pmatrix}
  =
  \begin{pmatrix}
      a_2 b_3 - a_3 b_2 \\
    -(a_1 b_3 - a_3 b_1) \\
      a_1 b_2 - a_2 b_1
  \end{pmatrix}
  =
  \begin{pmatrix}
    c_1 \\
    c_2 \\
    c_3
  \end{pmatrix}
  = c.
\end{align*}

Das sieht man unmittelbar durch entwickeln nach der letzten Spalte.
Dieses scheinbar \Quote{nutzlose} Wissen, hilft einem dabei, sich die Definition des Kreuzprodukts zu merken.

\end{remark*}
