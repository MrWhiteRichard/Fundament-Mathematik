\begin{remark}

Betrachte $p^{(n)}(x)$.
Wenn man $p$ von Newton- in Monom-Basis-Darstellung bringt, d.h. ausmultipliziert, und $n$-mal ableitet, so verschwinden alle Summanden, bis auf den mit $x^n$.
Es bleibt also genau $n!$ übrig.

Das Produkt in \eqref{eq_3} ist, für $k = n$, genau $1$.
Die innerste Summe wird also zu $1$, die zweit-innerste zu $1 \cdot 2$, usw.
Wie zu erwarten war, erhält man insgesamt, $1 \cdot 2 \cdots n = n! = p^{(n)}(x).$

\end{remark}
