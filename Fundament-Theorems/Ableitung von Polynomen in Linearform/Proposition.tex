\begin{proposition}

\label{prop_1}

Seien $n \in \N$ und $p \in \Pi_n$ ein komplexes Polynom vom maximalen Grad $n$.
Laut dem Fundamentalsatz der Algebra, existieren $n$, nicht zwangsläufig verschiedene, Nullstellen $x_1, \ldots, x_n \in \C$.
Sei o.B.d.A. der Führungskoeffizient von $p$ gleich $1$.
$p$ besitzt demnach folgende Darstellung.

\begin{align} \label{eq_1}
  p(x)
  =
  \prod_{i=1}^n (x - x_i)
\end{align}

Die Ableitung ist somit wie folgt gegeben.

\begin{align} \label{eq_2}
  p^\prime(x)
  =
  \sum_{i=1}^n
  \prod_{\substack{j=1 \\ j \neq i}}^n
  (x - x_j)
\end{align}

\end{proposition}

\begin{proof}

Wir führen den Beweis mit vollständiger Induktion nach $n$. \\

Induktions-Anfang ($n = 0$):
Damit, stehen in \eqref{eq_1} und \eqref{eq_2} bloß das leere Produkt $1$, bzw. die leere Summe $0$. \\

Induktions-Schritt ($n-1 \mapsto n$):
Die zu zeigende Aussage gelte für $n-1 \in \N$.
Wir benützen die Produkt-Regel und die Induktions-Voraussetzung und berechnen Folgendes.

\begin{align*}
  p^\prime(x)
  & =
  \frac{d}{dx}
  \prod_{i=1}^{n} (x - x_i)
  =
  \frac{d}{dx}
  \pbraces
  {
    (x - x_n)
    \prod_{i=1}^{n-1} (x - x_i)
  }
  =
  \prod_{i=1}^{n-1} (x - x_i)
  +
  (x - x_n)
  \frac{d}{dx}
  \prod_{i=1}^{n-1} (x - x_i) \\
  & =
  \prod_{i=1}^{n-1} (x - x_i)
  +
  (x - x_n)
  \sum_{i=1}^{n-1}
  \prod_{\substack{j=1 \\ j \neq i}}^{n-1} (x - x_j)
  =
  \prod_{\substack{i=1 \\ i \neq n}}^n (x - x_i)
  +
  \sum_{i=1}^{n-1}
  \prod_{\substack{j=1 \\ j \neq i}}^n (x - x_j)
  =
  \sum_{i=1}^n
  \prod_{\substack{j=1 \\ j \neq i}}^n
  (x - x_j)
\end{align*}

\end{proof}
