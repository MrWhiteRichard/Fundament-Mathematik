\begin{corollary}

Es gelten die Voraussetzungen aus Satz \ref{th_1}.
Sei zusätzlich $k \in \N$.
Die $k$-te Ableitung ist somit wie folgt gegeben.
Dabei, werden Summen, unter denen nicht-erfüllbare Bedingungen stehen, aus der Kette entfernt.

\begin{align} \label{eq_3}
  p^{(k)}(x)
  =
  \sum_{i_1 = 1}^n
  \sum_{\substack{i_2 = 1 \\ i_2 \neq i_1}}^n
  \cdots
  \sum_{\substack{i_k = 1 \\ i_k \neq i_1, \ldots, i_{k-1}}}^n
  \prod_{\substack{j = 1 \\ j \neq i_1, \ldots, i_{k-1}}}^n
  (x - x_j)
\end{align}

\end{corollary}

\begin{proof}

Wir führen den Beweis mit vollständiger Induktion nach $k$. \\

Induktions-Anfang ($k = 0$):
Laut Definition, verschwindet die Summen-Kette aus \eqref{eq_3} und es bleibt nur noch das Produkt übrig. \\

Induktions-Schritt ($k-1 \mapsto k$):
Die zu zeigende Aussage gelte für $k-1 \in \N$.

Wir benützen die Summen-Regel und die Induktions-Voraussetzung und berechnen Folgendes.

\begin{align*}
  p^{(k)}(x)
  & =
  \frac{d^k}{dx^k}
  \prod_{i=1}^{n} (x - x_i)
  =
  \frac{d^{k-1}}{dx^{k-1}}
  \frac{d}{dx}
  \prod_{i=1}^{n} (x - x_i)
  =
  \frac{d^{k-1}}{dx^{k-1}}
  \sum_{i_1 = 1}^n
  \prod_{\substack{j=1 \\ j \neq i_1}}^n
  (x - x_j) \\
  & =
  \sum_{i_1 = 1}^n
  \frac{d^{k-1}}{dx^{k-1}}
  \prod_{\substack{j=1 \\ j \neq i_1}}^n
  (x - x_j)
  =
  \sum_{i_1 = 1}^n
  \sum_{\substack{i_2 = 1 \\ i_2 \neq i_1}}^n
  \cdots
  \sum_{\substack{i_k = 1 \\ i_k \neq i_1, \ldots, i_{k-1}}}^n
  \prod_{\substack{j = 1 \\ j \neq i_1, \ldots, i_{k-1}}}^n
  (x - x_j)
\end{align*}

\end{proof}
