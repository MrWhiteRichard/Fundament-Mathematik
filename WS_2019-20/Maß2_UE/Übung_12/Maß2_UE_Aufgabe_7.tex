\begin{exercise}

Ein anderer Weg, das vorige Beispiel zu lösen: die Anzahl der Versuche hat eine negative Binomialverteilung und diese kann (als Summe von unabhängig geometrisch verteilten Zufallsvariablen) auch durch eine Normalverteilung approximiert werden.

\end{exercise}

\begin{solution}

Sei $X$ die Zufallsvariable, welche die Anzahl der Veruche angibt, um 100 Sechser zu erreichen.
\begin{align*}
Y_k:(\Omega,2^\Omega) \to (\R,\B): \omega \mapsto
\begin{cases}
    0, & \omega < 6, \\
    1, & \omega = 6.
\end{cases} \\
    X:(\Omega^\N,2^{\Omega^\N}) \to (\R, \B): (\omega_n)_{n\in\N} \mapsto \min\{k\in\N: \sum_{n = 0}^kY_k = 100\}
\end{align*}
$X$ ist negativ binomialverteilt, also:
\begin{align*}
  P(X=x) = {x - 1\choose 99}(\frac{1}{6})^{100}(\frac{5}{6})^{x-100}
\end{align*}
Diese negative Binomialverteilung lässt sich schließlich als Summe 100 unabhängiger, geometrisch verteilter Zufallsvariablen darstellen:
\begin{align*}
  X \sim \sum_{k=1}^{100}X_k,
\end{align*}
wobei $X_k$ geometrisch verteilte Zufallsvariablen mit Parameter $1/6$ sind.
Darauf können wir abermals den zentralen Grenzwertsatz anwenden:
\begin{align*}
  \P(X \leq x) = \P(\sum_{k=1}^{100}\frac{X_k - 600}{\sqrt{30*100}} \leq \frac{x-600}{\sqrt{30*100}}) &\approx \Phi\pbraces{\frac{x-\frac{100}{6}}{\sqrt{30*100}}}  > \frac{9}{10} \\
  \frac{x-\frac{100}{6}}{\sqrt{30*100}} > 1.28 \\
  x > 670.108
\end{align*}
\end{solution}
