\begin{exercise}

Ein anderer Weg, das vorige Beispiel zu lösen: die Anzahl der Versuche hat eine negative Binomialverteilung und diese kann (als Summe von unabhängig geometrisch verteilten Zufallsvariablen) auch durch eine Normalverteilung approximiert werden.

\end{exercise}

\begin{solution}

Eine Zufallsvariable $X$ ist negativ Binomialverteilt $B_{neg} (r,p)$, wobei
$r$ die Anzahl der erfolgreichen Versuche und $p$ die Wahrscheinlichkeit
des Erfolges ist, wenn sie sich durch die Wahrscheinlichkeitsfunktion

\begin{align*}
  \P(X=n) = \binom{n-1}{r-1}p^r(1-p)^{n-r}
\end{align*}

angeben lässt. Dabei gelten für Erwartungswert und Varianz

\begin{align*}
  \E(X) &= \frac{r(1-p)}{p} \\
  \V(X) &= \frac{r(1-p)}{p^2}
\end{align*}

Wir wollen wissen wie oft wir Würfeln müssen um $r=100$ (wobei also $r$ die
Anzahl der $6$en ist) und erhalten also:

\begin{align*}
  \P(X=n) &= \binom{n-1}{99}\pbraces{\frac{1}{6}}^{100}\pbraces{\frac{5}{6}}^{n-100} \\
  &\text{und suchen} \\
  \P(X \leq n) &> 0,9
\end{align*}

Für $r=1$, also für einen erfolgreichen Versuch, nennt man die Verteilung auch
geometrische Verteilung. Für diese gilt

\begin{align*}
  \P(X=n)=\frac{1}{6}\pbraces{\frac{5}{6}}^{n-r}
\end{align*}

Betrachten wir die Anzahl der Würfe bis zum nächsten Erfolg, ist diese wieder
geometrisch verteilt, wir können die Anzahl der Würfe bis $100$ $6$er geworfen
werden, als

\begin{align*}
  S_{100} = \sum_{i=1}^{100}X_i
\end{align*}

schreiben, wobei $X_i$ geometrisch verteilte ZV sind. Für Erwartungswert und
Varianz erhalten wir \\
$\E(X_i)=6, \V(X_i)=30$ und der zentrale Grenzwertsatz erlaubt uns nun, folgende
Wahrscheinlichkeit durch die Standard-Normalverteilung zu approximieren:

\begin{align*}
  \P(\frac{S_{100}-600}{10\sqrt{30}} \leq \frac{n-600}{10\sqrt{30}}) \thickapprox
  \Phi(\frac{n-600}{10\sqrt{30}}) > 0,9
\end{align*}

Nachschlagen gibt uns $\Phi(1,28)=0,9$ somit erhalten wir

\begin{align*}
  \frac{n-600}{10\sqrt{30}} > 1,28 \Leftrightarrow
  n > 12,8 \cdot \sqrt{30} + 600 = 670,1
\end{align*}
Somit $n=671$ (fast) wie im vorherigen Bsp.
\end{solution}
