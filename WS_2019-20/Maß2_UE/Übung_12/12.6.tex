\begin{exercise}

Wie oft muss man würfeln, damit die Wahrscheinlichkeit dafür, mindestens $100$ Sechser zu erzielen, größer als $0.9$ ist?

\end{exercise}

\begin{solution}

Für diese Aufgabe definieren wir eine neue Zufallsvariable

\begin{align*}
    Y:(\Omega,2^\Omega) \to (\R,\B): \omega \mapsto
    \begin{cases}
        0, & \omega < 6, \\
        1, & \omega = 6.
    \end{cases}
\end{align*}

Für n-maliges Würfeln brauchen wir wieder

\begin{align*}
    Y_n:(\Omega^n,2^{\Omega^n}) \to (\R, \B): (\omega_1,\dots,\omega_n)^T \mapsto \sum_{k=1}^n Y(\omega_k)
\end{align*}

Nun wollen wir das kleinste $n \in \N$ finden, für das

\begin{align*}
    \P_n(Y_n \geq 100) > \frac{9}{10}.
\end{align*}

Zuerst brauchen wir wieder Erwartungswert und Varianz von $Y_n$. Wir berechnen also

\begin{align*}
    \E(Y_n) &= \sum_{k=1}^n\E(Y) = n\frac{1}{6} \\
    \V(Y_n) &= \sum_{k=1}^n\V(Y) = \sum_{k=1}^n \frac{1}{6}\pbraces{5\pbraces{0-\frac{1}{6}}^2 + \pbraces{1-\frac{1}{6}}^2} = n\frac{5}{36}.
\end{align*}

Wir erhalten also mit einer Stetigkeitskorrektur

\begin{align*}
    \P_n(Y_n \geq 100) &= 1 - \P\pbraces{\frac{Y_n-\E(Y_n)}{\sqrt{\V(Y_n)}} < \frac{100-\E(Y_n)}{\sqrt{\V(Y_n)}}} \approx 1 - \P\pbraces{\frac{Y_n-\E(Y_n)}{\sqrt{\V(Y_n)}} \leq \frac{99.5-\E(Y_n)}{\sqrt{\V(Y_n)}}} \\
    &\approx 1 - \Phi\pbraces{\frac{99.5-\E(Y_n)}{\sqrt{\V(Y_n)}}} = \Phi\pbraces{-\frac{99.5-\E(Y_n)}{\sqrt{\V(Y_n)}}}
\end{align*}

Schauen wir nun in der Tabelle den kleinsten Wert nach, für den

\begin{align*}
    \Phi\pbraces{-\frac{99.5-\E(Y_n)}{\sqrt{\V(Y_n)}}} = \Phi\pbraces{-\frac{99.5-n\frac{1}{6}}{\sqrt{n\frac{5}{36}}}} > \frac{9}{10}
\end{align*}

gilt, so erhalten wir wie vorhin

\begin{align*}
    -\frac{99.5-n\frac{1}{6}}{\sqrt{n\frac{5}{36}}} > 1.28.
\end{align*}

Wolfram Alpha ist so nett uns dafür das Ergebnis

\begin{align*}
    n > \frac{75137 + 32\sqrt{74881}}{125} \approx 671.15
\end{align*}

zu liefern.

\end{solution}
