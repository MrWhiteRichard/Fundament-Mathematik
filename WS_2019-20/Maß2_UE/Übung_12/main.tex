\documentclass{article}

\usepackage[utf8]{inputenc}
\usepackage{fullpage}
\usepackage{amsmath, amssymb, amsfonts, amsthm}
\usepackage{bbm}
\usepackage{mathtools}
\usepackage{twoopt}

% special letters:

\newcommand{\N}{\mathbb{N}}
\newcommand{\Z}{\mathbb{Z}}
\newcommand{\Q}{\mathbb{Q}}
\newcommand{\R}{\mathbb{R}}
\newcommand{\C}{\mathbb{C}}
\newcommand{\K}{\mathbb{K}}
\newcommand{\T}{\mathbb{T}}
\newcommand{\E}{\mathbb{E}}
\newcommand{\V}{\mathbb{V}}
\renewcommand{\P}{\mathbb{P}}
\newcommand{\1}{\mathbbm{1}}

\newcommand  {\B}{\mathfrak{B}}
\renewcommand{\S}{\mathfrak{S}}

% quantors:

\newcommand{\Forall}{\forall \,}
\newcommand{\Exists}{\exists \,}
\newcommand{\ExistsOnlyOne}{\exists! \,}
\newcommand{\nExists}{\nexists \,}

% MISC symbols:

\newcommand{\landau}[1]
{
  {\scriptstyle \mathcal{O}}
  \pbraces{#1}
}

\newcommand{\Landau}[1]
{
  \mathcal{O}
  \pbraces{#1}
}

\newcommand{\eps}{\mathrm{eps}}

% graphics in a box:

\newcommandtwoopt
{\includegraphicsboxed}[3][][]
{
  \begin{figure}[!h]
    \begin{boxedin}
      \ifthenelse{\isempty{#2}}
      {
        \begin{center}
          \includegraphics[width = 0.75 \textwidth]{#3}
          \label{fig:#1}
        \end{center}
      }{
        \begin{center}
          \includegraphics[width = 0.75 \textwidth]{#3}
          \caption{#2}
          \label{fig:#1}
        \end{center}
      }
    \end{boxedin}
  \end{figure}
}

% braces:

\newcommand{\pbraces}[1]{{\left  ( #1 \right  )}}
\newcommand{\bbraces}[1]{{\left  [ #1 \right  ]}}
\newcommand{\Bbraces}[1]{{\left \{ #1 \right \}}}
\newcommand{\vbraces}[1]{{\left  | #1 \right  |}}
\newcommand{\Vbraces}[1]{{\left \| #1 \right \|}}
\newcommand{\abraces}[1]{{\left \langle #1 \right \rangle}}
\newcommand{\round}[1]{\bbraces{#1}}

\newcommand
{\floor}[1]
{{\left \lfloor #1 \right \rfloor}}

\newcommand
{\ceil} [1]
{{\left \lceil  #1 \right \rceil }}

% special functions:

\newcommand{\norm}  [2][]{\Vbraces{#2}_{#1}}
\newcommand{\diag}  [1]{\mathrm{diag} \: #1}
\newcommand{\dist}  [1]{\mathrm{dist} \: #1}
\newcommand{\mean}  [1]{\mathrm{mean} \: #1}
\newcommand{\erf}   [1]{\mathrm{erf} \: #1}
\newcommand{\id}    [1]{\mathrm{id} \: #1}
\newcommand{\sgn}   [1]{\mathrm{sgn} \: #1}
\newcommand{\supp}  [1]{\mathrm{supp} \: #1}
\newcommand{\arsinh}[1]{\mathrm{arsinh} \: #1}
\newcommand{\arcosh}[1]{\mathrm{arcosh} \: #1}
\newcommand{\artanh}[1]{\mathrm{artanh} \: #1}
\newcommand{\card}  [1]{\mathrm{card} \: #1}
\newcommand{\Span}  [1]{\mathrm{span} \: #1}
\newcommand{\Aut}   [1]{\mathrm{Aut} \: #1}
\newcommand{\End}   [1]{\mathrm{End} \: #1}
\newcommand{\ggT}   [1]{\mathrm{ggT} \: #1}
\newcommand{\kgV}   [1]{\mathrm{kgV} \: #1}
\newcommand{\ord}   [1]{\mathrm{ord} \: #1}
\newcommand{\grad}  [1]{\mathrm{grad} \: #1}
\newcommand{\ran}   [1]{\mathrm{ran} \: #1}
\newcommand{\graph} [1]{\mathrm{graph} \: #1}
\newcommand{\Inv}   [1]{\mathrm{Inv} \: #1}
\newcommand{\pv}    [1]{\mathrm{pv} \: #1}
\newcommand{\Mod}{\: \mathrm{mod} \:}
\newcommand{\Char}{\mathrm{char}}
\newcommand{\At}{\mathrm{At}}
\newcommand{\Ob}{\mathrm{Ob}}
\newcommand{\Hom}{\mathrm{Hom}}
\newcommand{\orthogonal}[3][]{#2 ~\bot_{#1}~ #3}
\newcommand{\Rang}{\mathrm{Rang}}

\newcommand
{\GL}[2][]
{\mathrm{GL}_{#1} \pbraces{#2}}

% fractions:

\newcommand{\Frac}[2]{\frac{1}{#1} \pbraces{#2}}
\newcommand{\nfrac}[2]{\nicefrac{#1}{#2}}

% derivatives & integrals:

\newcommandtwoopt
{\Int}[4][][]
{\int_{#1}^{#2} #3 ~\mathrm{d} #4}

\newcommandtwoopt
{\derivative}[3][][]
{
  \frac
  {\mathrm{d}^{#1} #2}
  {\mathrm{d} #3^{#1}}
}

\newcommandtwoopt
{\pderivative}[3][][]
{
  \frac
  {\partial^{#1} #2}
  {\partial #3^{#1}}
}

\newcommand
{\primeprime}
{{\prime \prime}}

\newcommand
{\primeprimeprime}
{{\prime \prime \prime}}

% Text:

\newcommand{\Quote}[1]{\glqq #1\grqq{}}
\newcommand{\Text}[1]{{\text{#1}}}
\newcommand{\fastueberall}{\text{f.ü.}}
\newcommand{\fastsicher}{\text{f.s.}}

% -------------------------------- %
% amsthm-stuff:

\theoremstyle{definition}

% numbered theorems
\newtheorem{theorem}    {Satz}   [section]
\newtheorem{lemma}      [theorem]{Lemma}
\newtheorem{corollary}  [theorem]{Korollar}
\newtheorem{proposition}[theorem]{Proposition}
\newtheorem{remark}     [theorem]{Bemerkung}
\newtheorem{definition} [theorem]{Definition}
\newtheorem{example}    [theorem]{Beispiel}

% unnumbered theorems
\newtheorem*{theorem*}    {Satz}
\newtheorem*{lemma*}      {Lemma}
\newtheorem*{corollary*}  {Korollar}
\newtheorem*{proposition*}{Proposition}
\newtheorem*{remark*}     {Bemerkung}
\newtheorem*{definition*} {Definition}
\newtheorem*{example*}    {Beispiel}

% Please define this stuff in project ("main.tex"):

% \def \lastexercisenumber {...}
% This will be 0 by default

% \setcounter{section}{...}
% This will be 0 by default
% and hence, completely ignored

\ifnum \thesection = 0
{
  \newtheorem{exercise}{Aufgabe}
}
\else
{
  \newtheorem{exercise}{Aufgabe}[section]
}
\fi

\ifdef
{\lastexercisenumber}
{\setcounter{exercise}{\lastexercisenumber}}

\newenvironment{solution}
{
  \begin{proof}[Lösung]
}{
  \end{proof}
}

\renewcommand{\proofname}{Beweis}

% -------------------------------- %
% environment zum einkasteln:

% dickere vertical lines
\newcolumntype
{x}
[1]
{
  !{
    \centering
    \arraybackslash
    \vrule
    width #1}
}

% environment selbst (the big cheese)
\newenvironment
{boxedin}
{
  \begin{tabular}
  {
    x{1 pt}
    p{\textwidth}
    x{1 pt}
  }
  \Xhline
  {2 \arrayrulewidth}
}
{
  \\
  \Xhline{2 \arrayrulewidth}
  \end{tabular}
}

% -------------------------------- %
% MISC "Ein-Deutschungen"

\renewcommand{\figurename}{Abbildung}
\renewcommand{\tablename} {Tabelle}

% -------------------------------- %


\parindent 0pt

\title
{
  Maß- und Wahrscheinlichkeitstheorie 2 - Übung \\
  \vspace{4pt}
  \normalsize
  \textit{12. UE am 15.01.2020}
}
\author
{
  Richard Weiss       \and
  Florian Schager     \and
  Christian Sallinger \and
  Fabian Zehetgruber  \and
  Paul Winkler        \and
  Christian Göth
}
\date{}

\begin{document}

\maketitle


\begin{align*}
  \Phi:\overline{\mathbb{R}} \to [0,1]: x \mapsto \frac{1}{\sqrt{2\pi}} \int_{-\infty}^x \mathrm{e}^\frac{-t^2}{2} \mathrm{d}t
\end{align*}
\begin{lemma}
    Wenn $\forall n\in\mathbb{N}:f_n:\mathbb{R}\to\mathbb{R}$ stetig und beschränkt sind, $f:\mathbb{R}\to\mathbb{R}$ stetig ist, $\forall n\in\mathbb{N}:P_n$ sowie $P$ Wahrscheinlichkeitsmaße auf $(\mathbb{R},\mathfrak{B})$ sind und $f_n\to f$ gleichmäßig und $P_n\to P$ schwach, dann gilt
    \begin{align*}
        \lim_{n\to\infty}\int f_n\mathrm{d}P_n=\int f\mathrm{d}P.
    \end{align*}
\end{lemma}
\begin{proof}[Beweis.]
    Wir schreiben ganz einfach mit der Dreiecksungleichung
    \begin{align*}
        \left\vert\int f_n\mathrm{d}P_n-\int f\mathrm{d}P\right\vert\leq\left\vert\int (f_n-f)\mathrm{d}P_n\right\vert+\left\vert\int f\mathrm{d}P_n-\int f\mathrm{d}P\right\vert
    \end{align*}
    Für hinreichend gorße $n\in\mathbb{N}$ wird der erste Summand wegen $f_n\to f$ gleichmäßig und der zweite Summand wegen $P_n\to P$ schwach klein.
\end{proof}

\begin{exercise}

Zeigen Sie: Wenn für ein $t \neq 0$ $\phi_X(t) = 1$ gilt, dann nimmt $X$ mit Wahrscheinlichkeit $1$ nur Werte der Form $2n \pi/t$, $n \in \Z$ an. Gilt $\phi(t_1) = \phi(t_2) = 1$ für zwei inkommensurable Werte $t_1$ und $t_2$ (d.h. $t_1/t_2$ ist irrational), dann gilt $X = 0$ fast sicher.

\end{exercise}

\begin{solution}

Trivial

\end{solution}

\begin{exercise}

Hier könnte Ihre Werbung stehen!

\begin{itemize}
    \item[(a)] $F$ und $G$ seien (Wahrscheinlichkeits-) Verteilungsfunktionen, $d$ die Lévy-Prohorov-Metrik. Zeigen Sie für die verallgemeinerten Inversen

    \begin{equation*}
      d(F^{-1},G^{-1}) \leq d(F,G).
    \end{equation*}

    \item[(b)] Zeigen Sie: $F_n \longrightarrow F$ genau dann, wenn es auf einem geeigneten Wahrscheinlichkeitsraum Zufallsvariablen $X_n \sim F_n, X \sim F$ gibt mit $X_n \rightarrow X$ fast sicher (Darstellungssatz von Skorohod).
\end{itemize}

\end{exercise}

\begin{solution}

(a) Die verallgemeinerte Inverse wird definiert für $p \in (0, 1]$.

\begin{align*}
  F^{-1}(p) := \inf \Bbraces{x \in \R: p \leq F(x)}
\end{align*}

Wir benennen die folgenden Mengen.

\begin{align*}
  A  & := \Bbraces
  {
    \epsilon > 0:
    \Forall x \in \R:
    F(x - \epsilon) - \epsilon \leq
    G(x) \leq
    F(x + \epsilon) + \epsilon
  } \\
  B & := \Bbraces
  {
    \epsilon > 0:
    \Forall x \in \R:
    F^{-1}(x - \epsilon) - \epsilon \leq
    G^{-1}(x) \leq
    F^{-1}(x + \epsilon) + \epsilon
  }
\end{align*}

Für $\inf B \leq \inf A$, genügt es $A \subseteq B$ zu zeigen. Sei also $\epsilon \in A$, dann gilt $\epsilon > 0$ und $\Forall x \in \R:$

\begin{align*}
  F(x - \epsilon) - \epsilon \leq
  G(x) \leq
  F(x + \epsilon) + \epsilon.
\end{align*}

Wir wollen zeigen, dass

\begin{align*}
  F^{-1}(x - \epsilon) - \epsilon \leq
  G^{-1}(x) \leq
  F^{-1}(x + \epsilon) + \epsilon.
\end{align*}

Wir zeigen die erste Ungleichung; die zweite geht (wahrscheinlich) analog. Für die letzte Ungleichung beachte man $M \subseteq N$.

\begin{align*}
  \inf \Bbraces{y \in \R: x - \epsilon \leq F(y)} - \epsilon
  & = \inf \Bbraces{y - \epsilon \in \R: x - \epsilon \leq F(y)} \\
  & = \inf \Bbraces{z \in \R: x - \epsilon \leq F(z + \epsilon)} \\
  & = \inf \underbrace
      {
        \Bbraces{z \in \R: x \leq F(z + \epsilon) + \epsilon}
      }_{=: N} \\
  & \leq \inf \underbrace
      {
        \Bbraces{y \in \R: x \leq G(y)}
      }_{=: M}
\end{align*}

(b) Wir betrachten den Wahrscheinlichkeitsraum $(]0, 1[, \B(]0, 1[), \lambda)$ und bemerken, dass $F_n, F: \R \to [0, 1]$. Die verallgemeinerten Inversen $F_n^{-1}, F^{-1}$ sind monoton und somit messbar, also $X_n := F_n^{-1}, X := F^{-1}$ Zufallsvariablen. \\

\say{$\Rightarrow$}: Es ist zu zeigen, dass

\begin{align*}
  \P(X_n \xrightarrow{n \to \infty} X) = 1.
\end{align*}

Schwache Konvergenz ist bekanntlich äquivalent zur Konvergenz in der Lévy-Prohorov-Metrik.

\begin{align*}
  F_n \xrightarrow[\text{schwach}]{n \to \infty} F
  \Leftrightarrow
  d(F_n, F) \xrightarrow{n \to \infty} 0.
\end{align*}

Laut (a) wissen wir also bereits, dass

\begin{align*}
  d(X_n, X) =
  d(F_n^{-1}, F^{-1}) \leq
  d(F_n, F) \xrightarrow{n \to \infty} 0.
  \label{schwache_konvergenz}
\end{align*}

Eine weitere Äquivalenz zur schwach Konvergenz, erlaubt $\Forall \omega \in \mathcal{C}(X):$

\begin{align*}
  X_n(\omega) \xrightarrow{n \to \infty} X(\omega).
\end{align*}

Weil in jedem Intervall eine rationale Zahl ist und $|\Q| = \aleph_0$, gibt es nur abzählbar viele $F_n$- bzw. $F$-konstante Bereiche. $X_n, X$ haben genau dort ihre Sprungstellen, also ebenfalls nur abzählbar viele, also bilden sie eine $\lambda$-Nullmenge. Q.E.D. \\

\say{$\Leftarrow$}: $\lambda$ ist auf unserem Wahrscheinlichkeitsraum endlich, also folgt mit dem Satz von Egorov und Übung 11, Beispiel 6, dass

\begin{align*}
  X_n \xrightarrow[\text{f.s.}]{n \to \infty} X
  \Rightarrow
  X_n \xrightarrow[\text{in WSK}]{n \to \infty} X
  \Rightarrow
  F_n \xrightarrow[\text{schwach}]{n \to \infty} F.
\end{align*}

\end{solution}

\begin{exercise}

Ein Würfel wird $100$ mal geworfen. Bestimmen Sie die Wahrscheinlichkeit, dass die Summe der Augenzahlen mehr als $375$ beträgt.

\end{exercise}

\begin{solution}

Wir definieren
\begin{align*}
    \Omega := \{1,2,3,4,5,6\}
\end{align*}
sowie das Wahrscheinlichkeitsmaß
\begin{align*}
    \mathbb{P}:2^\Omega \to \mathbb{R}: \{\{\omega_1\},\dots,\{\omega_l\}\} \mapsto \vert \{\{\omega_1\},\dots\{\omega_l\}\} \vert \frac{1}{6}
\end{align*}
und dann auf dem Wahrcheinlichkeitsraum $(\Omega,2^\Omega,\mathbb{P})$ für alle $k \in \{1,\dots,100\}$ eine Zufallsvariable
\begin{align*}
    X_k:(\Omega,2^\Omega) \to (\mathbb{R},\mathfrak{B}): \omega \mapsto \omega.
\end{align*}
Nun soll für alle $n \in \mathbb{N}$
\begin{align*}
    X^{(n)}:(\Omega^n,2^{\Omega^n}) \to (\mathbb{R},\mathfrak{B}): (\omega_1,\dots,\omega_{100})^T \mapsto \sum_{k=1}^{n} X_k(\omega_k)
\end{align*}
unseren Versuch beschreiben. Wir berechnen zuerst
\begin{align*}
    \mathbb{E}(X^{(100)}) = \sum_{k=1}^{100} \mathbb{E}(X_k) = 100\frac{7}{2} = 350
\end{align*}
sowie unter Benützung der Unabhängigkeit der $X_k$
\begin{align*}
    \mathbb{V}(X^{(100)}) = \sum_{k=1}^{100} \mathbb{V}(X_k) = \sum_{k=1}^{100} \frac{1}{6} \pbraces{\pbraces{1-\frac{7}{2}}^2+\cdots+\pbraces{6-\frac{7}{2}}^2} = 100 \frac{35}{12} = \frac{875}{3}
\end{align*}
Nun verwenden wir (den lokalen Grenzwertsatz???) um
\begin{align*}
    \mathbb{P}(X^{(100)} > 375) &= 1 - \mathbb{P}(X^{(100)} \leq 375) = 1 - \mathbb{P}\pbraces{\frac{X^{(100)} - \mathbb{E}(X^{(100)})}{\sqrt{\mathbb{V}(X^{(100)})}} \leq \frac{375 - \mathbb{E}(X^{(100)})}{\sqrt{\mathbb{V}(X^{(100)})}}} \\
    &= 1 - \mathbb{P}\pbraces{\frac{X^{(100)} - \mathbb{E}(X^{(100)})}{\sqrt{\mathbb{V}(X^{(100)})}} \leq \frac{25\sqrt{3}}{\sqrt{875}}} \approx 1 - \Phi(1.464) \approx 1 - 0.92785 = 0.07215
\end{align*}
\end{solution}

\begin{exercise}

Wie oft muss man würfeln, damit die Wahrscheinlichkeit dafür, dass die Summe der Augenzahlen größer als $200$ ist, mindestens $0.9$ ist?

\end{exercise}

\begin{solution}

Jetzt suchen wir das kleinste $n \in \N$ für das
\begin{align*}
    \P_n(X_n > 200) \geq \frac{9}{10}
\end{align*}
gilt.
\begin{align*}
    \P_n(X_n > 200) \geq \frac{9}{10} &\Leftrightarrow \P_n\pbraces{\frac{X_n-n\frac{7}{2}}{\sqrt{n\frac{35}{12}}} > \frac{200 -n\frac{7}{2}}{\sqrt{n\frac{35}{12}}}} \geq \frac{9}{10} \\
    &\Leftrightarrow \P_n\pbraces{\frac{X_n-n\frac{7}{2}}{\sqrt{n\frac{35}{12}}} \leq \frac{200 -n\frac{7}{2}}{\sqrt{n\frac{35}{12}}}} < \frac{1}{10}
\end{align*}

Nun können wir des zentralen Grenzwertsatz wegens diese Wahrscheinlichkeit approximieren durch

\begin{align*}
    \P_n\pbraces{\frac{X_n-n\frac{7}{2}}{\sqrt{n\frac{35}{12}}} \leq \frac{200 -n\frac{7}{2}}{\sqrt{n\frac{35}{12}}}} \approx \Phi\pbraces{\frac{200 -n\frac{7}{2}}{\sqrt{n\frac{35}{12}}}}
\end{align*}

Und berechnen nun unter Benützung der Symmetrie der Standardnormalverteilung
\begin{align*}
    \Phi\pbraces{\frac{200 -n\frac{7}{2}}{\sqrt{n\frac{35}{12}}}} < \frac{1}{10} \Leftrightarrow \Phi\pbraces{-\frac{200 -n\frac{7}{2}}{\sqrt{n\frac{35}{12}}}} > \frac{9}{10}
\end{align*}

Schlägt man das in der Tabelle nach dann erhält man ungefähr

\begin{align*}
    -\frac{200 -n\frac{7}{2}}{\sqrt{n\frac{35}{12}}} > 1.28.
\end{align*}

Wolfram-Alpha liefert als Lösung

\begin{align*}
    n > \frac{16(9407+8\sqrt{9391})}{2625} \approx 62.06.
\end{align*}

\end{solution}

\begin{exercise}

Wie oft muss man würfeln, damit die Wahrscheinlichkeit dafür, mindestens $100$ Sechser zu erzielen, größer als $0.9$ ist?

\end{exercise}

\begin{solution}

Für diese Aufgabe definieren wir eine neue Zufallsvariable

\begin{align*}
    Y:(\Omega,2^\Omega) \to (\R,\B): \omega \mapsto
    \begin{cases}
        0, & \omega < 6, \\
        1, & \omega = 6.
    \end{cases}
\end{align*}

Für n-maliges Würfeln brauchen wir wieder

\begin{align*}
    Y_n:(\Omega^n,2^{\Omega^n}) \to (\R, \B): (\omega_1,\dots,\omega_n)^T \mapsto \sum_{k=1}^n Y(\omega_k)
\end{align*}

Nun wollen wir das kleinste $n \in \N$ finden, für das

\begin{align*}
    \P_n(Y_n \geq 100) > \frac{9}{10}.
\end{align*}

Zuerst brauchen wir wieder Erwartungswert und Varianz von $Y_n$. Wir berechnen also

\begin{align*}
    \E(Y_n) &= \sum_{k=1}^n\E(Y) = n\frac{1}{6} \\
    \V(Y_n) &= \sum_{k=1}^n\V(Y) = \sum_{k=1}^n \frac{1}{6}\pbraces{5\pbraces{0-\frac{1}{6}}^2 + \pbraces{1-\frac{1}{6}}^2} = n\frac{5}{36}.
\end{align*}

Wir erhalten also mit einer Stetigkeitskorrektur

\begin{align*}
    \P_n(Y_n \geq 100) &= 1 - \P\pbraces{\frac{Y_n-\E(Y_n)}{\sqrt{\V(Y_n)}} < \frac{100-\E(Y_n)}{\sqrt{\V(Y_n)}}} \approx 1 - \P\pbraces{\frac{Y_n-\E(Y_n)}{\sqrt{\V(Y_n)}} \leq \frac{99.5-\E(Y_n)}{\sqrt{\V(Y_n)}}} \\
    &\approx 1 - \Phi\pbraces{\frac{99.5-\E(Y_n)}{\sqrt{\V(Y_n)}}} = \Phi\pbraces{-\frac{99.5-\E(Y_n)}{\sqrt{\V(Y_n)}}}
\end{align*}

Schauen wir nun in der Tabelle den kleinsten Wert nach, für den

\begin{align*}
    \Phi\pbraces{-\frac{99.5-\E(Y_n)}{\sqrt{\V(Y_n)}}} = \Phi\pbraces{-\frac{99.5-n\frac{1}{6}}{\sqrt{n\frac{5}{36}}}} > \frac{9}{10}
\end{align*}

gilt, so erhalten wir wie vorhin

\begin{align*}
    -\frac{99.5-n\frac{1}{6}}{\sqrt{n\frac{5}{36}}} > 1.28.
\end{align*}

Wolfram Alpha ist so nett uns dafür das Ergebnis

\begin{align*}
    n > \frac{75137 + 32\sqrt{74881}}{125} \approx 671.15
\end{align*}

zu liefern.

\end{solution}

\begin{exercise}

Ein anderer Weg, das vorige Beispiel zu lösen: die Anzahl der Versuche hat eine negative Binomialverteilung und diese kann (als Summe von unabhängig geometrisch verteilten Zufallsvariablen) auch durch eine Normalverteilung approximiert werden.

\end{exercise}

\begin{solution}

<<<<<<< HEAD
Sei $X$ die Zufallsvariable, welche die Anzahl der Veruche angibt, um 100 Sechser zu erreichen.
\begin{align*}
Y_k:(\Omega,2^\Omega) \to (\R,\B): \omega \mapsto
\begin{cases}
    0, & \omega < 6, \\
    1, & \omega = 6.
\end{cases} \\
    X:(\Omega^\N,2^{\Omega^\N}) \to (\R, \B): (\omega_n)_{n\in\N} \mapsto \min\{k\in\N: \sum_{n = 0}^kY_k = 100\}
\end{align*}
$X$ ist negativ binomialverteilt, also:
\begin{align*}
  P(X=x) = {x - 1\choose 99}(\frac{1}{6})^{100}(\frac{5}{6})^{x-100}
\end{align*}
Diese negative Binomialverteilung lässt sich schließlich als Summe 100 unabhängiger, geometrisch verteilter Zufallsvariablen darstellen:
\begin{align*}
  X \sim \sum_{k=1}^{100}X_k,
\end{align*}
wobei $X_k$ geometrisch verteilte Zufallsvariablen mit Parameter $1/6$ sind.
Darauf können wir abermals den zentralen Grenzwertsatz anwenden:
\begin{align*}
  \P(X \leq x) = \P(\sum_{k=1}^{100}\frac{X_k - 600}{\sqrt{30*100}} \leq \frac{x-600}{\sqrt{30*100}}) &\approx \Phi\pbraces{\frac{x-\frac{100}{6}}{\sqrt{30*100}}}  > \frac{9}{10} \\
  \frac{x-\frac{100}{6}}{\sqrt{30*100}} > 1.28 \\
  x > 670.108
\end{align*}
=======
Eine Zufallsvariable $X$ ist negativ Binomialverteilt $B_{neg} (r,p)$, wobei
$r$ die Anzahl der erfolgreichen Versuche und $p$ die Wahrscheinlichkeit
des Erfolges ist, wenn sie sich durch die Wahrscheinlichkeitsfunktion

\begin{align*}
  \P(X=n) = \binom{n-1}{r-1}p^r(1-p)^{n-r}
\end{align*}

angeben lässt. Dabei gelten für Erwartungswert und Varianz

\begin{align*}
  \E(X) &= \frac{r(1-p)}{p} \\
  \V(X) &= \frac{r(1-p)}{p^2}
\end{align*}

Wir wollen wissen wie oft wir Würfeln müssen um $r=100$ (wobei also $r$ die
Anzahl der $6$en ist) und erhalten also:

\begin{align*}
  \P(X=n) &= \binom{n-1}{99}\pbraces{\frac{1}{6}}^{100}\pbraces{\frac{5}{6}}^{n-100} \\
  &\text{und suchen} \\
  \P(X \leq n) &> 0,9
\end{align*}

Für $r=1$, also für einen erfolgreichen Versuch, nennt man die Verteilung auch
geometrische Verteilung. Für diese gilt

\begin{align*}
  \P(X=n)=\frac{1}{6}\pbraces{\frac{5}{6}}^{n-r}
\end{align*}

Betrachten wir die Anzahl der Würfe bis zum nächsten Erfolg, ist diese wieder
geometrisch verteilt, wir können die Anzahl der Würfe bis $100$ $6$er geworfen
werden, als

\begin{align*}
  S_{100} = \sum_{i=1}^{100}X_i
\end{align*}

schreiben, wobei $X_i$ geometrisch verteilte ZV sind. Für Erwartungswert und
Varianz erhalten wir \\
$\E(X_i)=6, \V(X_i)=30$ und der zentrale Grenzwertsatz erlaubt uns nun, folgende
Wahrscheinlichkeit durch die Standard-Normalverteilung zu approximieren:

\begin{align*}
  \P(\frac{S_{100}-600}{10\sqrt{30}} \leq \frac{n-600}{10\sqrt{30}}) \thickapprox
  \Phi(\frac{n-600}{10\sqrt{30}}) > 0,9
\end{align*}

Nachschlagen gibt uns $\Phi(1,28)=0,9$ somit erhalten wir

\begin{align*}
  \frac{n-600}{10\sqrt{30}} > 1,28 \Leftrightarrow
  n > 12,8 \cdot \sqrt{30} + 600 = 670,1
\end{align*}
Somit $n=671$ (fast) wie im vorherigen Bsp.
>>>>>>> 9c1e9e03b5cb9e0376f57573526d83fafd8c39c3
\end{solution}


\end{document}
