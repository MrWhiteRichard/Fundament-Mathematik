\begin{exercise}

Wie oft muss man würfeln, damit die Wahrscheinlichkeit dafür, dass die Summe der Augenzahlen größer als $200$ ist, mindestens $0.9$ ist?

\end{exercise}

\begin{solution}

Jetzt suchen wir das kleinste $n \in \N$ für das
\begin{align*}
    \P_n(X_n > 200) \geq \frac{9}{10}
\end{align*}
gilt.
\begin{align*}
    \P_n(X_n > 200) \geq \frac{9}{10} &\Leftrightarrow \P_n\pbraces{\frac{X_n-n\frac{7}{2}}{\sqrt{n\frac{35}{12}}} > \frac{200 -n\frac{7}{2}}{\sqrt{n\frac{35}{12}}}} \geq \frac{9}{10} \\
    &\Leftrightarrow \P_n\pbraces{\frac{X_n-n\frac{7}{2}}{\sqrt{n\frac{35}{12}}} \leq \frac{200 -n\frac{7}{2}}{\sqrt{n\frac{35}{12}}}} < \frac{1}{10}
\end{align*}

Nun können wir des zentralen Grenzwertsatz wegens diese Wahrscheinlichkeit approximieren durch

\begin{align*}
    \P_n\pbraces{\frac{X_n-n\frac{7}{2}}{\sqrt{n\frac{35}{12}}} \leq \frac{200 -n\frac{7}{2}}{\sqrt{n\frac{35}{12}}}} \approx \Phi\pbraces{\frac{200 -n\frac{7}{2}}{\sqrt{n\frac{35}{12}}}}
\end{align*}

Und berechnen nun unter Benützung der Symmetrie der Standardnormalverteilung
\begin{align*}
    \Phi\pbraces{\frac{200 -n\frac{7}{2}}{\sqrt{n\frac{35}{12}}}} < \frac{1}{10} \Leftrightarrow \Phi\pbraces{-\frac{200 -n\frac{7}{2}}{\sqrt{n\frac{35}{12}}}} > \frac{9}{10}
\end{align*}

Schlägt man das in der Tabelle nach dann erhält man ungefähr

\begin{align*}
    -\frac{200 -n\frac{7}{2}}{\sqrt{n\frac{35}{12}}} > 1.28.
\end{align*}

Wolfram-Alpha liefert als Lösung

\begin{align*}
    n > \frac{16(9407+8\sqrt{9391})}{2625} \approx 62.06.
\end{align*}

\end{solution}
