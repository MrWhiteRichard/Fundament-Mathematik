\begin{exercise}

Ein Würfel wird $100$ mal geworfen. Bestimmen Sie die Wahrscheinlichkeit, dass die Summe der Augenzahlen mehr als $375$ beträgt.

\end{exercise}

\begin{solution}

Wir definieren, um einen Wurf des Würfels darzustellen,
\begin{align*}
    \Omega := \{1,2,3,4,5,6\}
\end{align*}
sowie das Wahrscheinlichkeitsmaß
\begin{align*}
    \P:2^\Omega \to \R: \{\{\omega_1\},\dots,\{\omega_l\}\} \mapsto \vert \{\{\omega_1\},\dots\{\omega_l\}\} \vert \frac{1}{6}
\end{align*}
und dann auf dem Wahrcheinlichkeitsraum $(\Omega,2^\Omega,\P)$ eine Zufallsvariable
\begin{align*}
    X:(\Omega,2^\Omega) \to (\R,\B): \omega \mapsto \omega.
\end{align*}
Nun soll für ein beliebiges $n \in \N$, in unserem Fall $n=100$,
\begin{align*}
    X_n:(\Omega^n,2^{\Omega^n}) \to (\R,\B): (\omega_1,\dots,\omega_n)^T \mapsto \sum_{k=1}^{n} X(\omega_k)
\end{align*}
unseren Versuch beschreiben. Wir berechnen zuerst
\begin{align*}
    \E(X_{100}) = \sum_{k=1}^{100} \E(X) = 100\frac{7}{2} = 350
\end{align*}
sowie unter Benützung der Unabhängigkeit der $X_k$
\begin{align*}
    \V(X_{100}) = \sum_{k=1}^{100} \V(X) = \sum_{k=1}^{100} \frac{1}{6} \pbraces{\pbraces{1-\frac{7}{2}}^2+\cdots+\pbraces{6-\frac{7}{2}}^2} = 100 \frac{35}{12} = \frac{875}{3}
\end{align*}
Nun verwenden wir (den lokalen Grenzwertsatz???) um
\begin{align*}
    \P_{100}(X_{100} > 375) &= 1 - \P_{100}(X_{100} \leq 375) = 1 - \P_{100}\pbraces{\frac{X_{100} - \E(X_{100})}{\sqrt{\V(X_{100})}} \leq \frac{375 - \E(X_{100})}{\sqrt{\V(X_{100})}}} \\
    &= 1 - \P_{100}\pbraces{\frac{X_{100} - \E(X_{100})}{\sqrt{\V(X_{100})}} \leq \frac{25\sqrt{3}}{\sqrt{875}}} \approx 1 - \Phi(1.464) \approx 1 - 0.92785 = 0.07215
\end{align*}
\end{solution}
