\begin{exercise}

Ein Würfel wird $100$ mal geworfen. Bestimmen Sie die Wahrscheinlichkeit, dass die Summe der Augenzahlen mehr als $375$ beträgt.

\end{exercise}

\begin{solution}

Wir definieren
\begin{align*}
    \Omega := \{1,2,3,4,5,6\}
\end{align*}
sowie das Wahrscheinlichkeitsmaß
\begin{align*}
    \mathbb{P}:2^\Omega \to \mathbb{R}: \{\{\omega_1\},\dots,\{\omega_l\}\} \mapsto \vert \{\{\omega_1\},\dots\{\omega_l\}\} \vert \frac{1}{6}
\end{align*}
und dann auf dem Wahrcheinlichkeitsraum $(\Omega,2^\Omega,\mathbb{P})$ für alle $k \in \{1,\dots,100\}$ eine Zufallsvariable
\begin{align*}
    X_k:(\Omega,2^\Omega) \to (\mathbb{R},\mathfrak{B}): \omega \mapsto \omega.
\end{align*}
Nun soll für alle $n \in \mathbb{N}$
\begin{align*}
    X^{(n)}:(\Omega^n,2^{\Omega^n}) \to (\mathbb{R},\mathfrak{B}): (\omega_1,\dots,\omega_{100})^T \mapsto \sum_{k=1}^{n} X_k(\omega_k)
\end{align*}
unseren Versuch beschreiben. Wir berechnen zuerst
\begin{align*}
    \mathbb{E}(X^{(100)}) = \sum_{k=1}^{100} \mathbb{E}(X_k) = 100\frac{7}{2} = 350
\end{align*}
sowie unter Benützung der Unabhängigkeit der $X_k$
\begin{align*}
    \mathbb{V}(X^{(100)}) = \sum_{k=1}^{100} \mathbb{V}(X_k) = \sum_{k=1}^{100} \frac{1}{6} \pbraces{\pbraces{1-\frac{7}{2}}^2+\cdots+\pbraces{6-\frac{7}{2}}^2} = 100 \frac{35}{12} = \frac{875}{3}
\end{align*}
Nun verwenden wir (den lokalen Grenzwertsatz???) um
\begin{align*}
    \mathbb{P}(X^{(100)} > 375) &= 1 - \mathbb{P}(X^{(100)} \leq 375) = 1 - \mathbb{P}\pbraces{\frac{X^{(100)} - \mathbb{E}(X^{(100)})}{\sqrt{\mathbb{V}(X^{(100)})}} \leq \frac{375 - \mathbb{E}(X^{(100)})}{\sqrt{\mathbb{V}(X^{(100)})}}} \\
    &= 1 - \mathbb{P}\pbraces{\frac{X^{(100)} - \mathbb{E}(X^{(100)})}{\sqrt{\mathbb{V}(X^{(100)})}} \leq \frac{25\sqrt{3}}{\sqrt{875}}} \approx 1 - \Phi(1.464) \approx 1 - 0.92785 = 0.07215
\end{align*}
\end{solution}
