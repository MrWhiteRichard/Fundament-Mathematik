\begin{lemma}
    Zeigen Sie: Wenn $F$ eine stetige Verteilungsfunktion ist, dann konvergiert die Folge $(F_n)_{n \in \mathbb{N}}$ genau dann
    schwach gegen $F$, wenn sie gleichmäßig konvergiert.
\end{lemma}
\begin{proof}[Beweis.]
    Laut Satz 12.5 des Vorlesungsskript ist die schwache Konvergenz äquivalent zu

    \begin{align*}
    &\forall x \in \mathcal{C}(F): \lim_{n \rightarrow \infty} F_n(x) = F(x) \tag{\textit{i}} \label{eq:first} \\
    &\lim_{n \rightarrow \infty} d(F_n,F) := \inf \{\epsilon \geq 0: \forall x: F_n(x - \epsilon) - \epsilon \leq F(x) \leq F_n(x + \epsilon) + \epsilon\} = 0, \tag{\textit{ii}} \label{eq:second} \\
\end{align*}
    wobei $\mathcal{C}(F)$ die Menge aller Stetigkeitspunkte von F bezeichnet.
    Aus \ref{eq:second} erhalten wir:
\begin{align*}
    &\forall \delta > 0~ \exists n_0 \in \mathbb{N}: \forall n > n_0: d(F_n,F) < \delta\\
    \Longleftrightarrow \hspace{10mm} &\forall \delta > 0~ \exists n_0 \in \mathbb{N}: \exists \epsilon_0 < \delta: \forall \epsilon > \epsilon_0: \forall x: F_n(x - \epsilon) - \epsilon \leq F(x) \leq F_n(x + \epsilon) \\
    \Longleftrightarrow \hspace{10mm} &\forall \delta > 0~ \exists n_0 \in \mathbb{N}: \exists \epsilon_0 < \delta: \forall \epsilon > \epsilon_0: \forall x: F(x - \epsilon) - \epsilon \leq F_n(x) \leq F(x + \epsilon) \\
\end{align*}
Also reicht es aus zu zeigen, dass aus der punktweisen Konvergenz der $(F_n)_{n \in \mathbb{N}}$ die gleichmäßige Konvergenz jener Folge folgt.
Sei $\epsilon > 0$ beliebig. \\
Zu zeigen: $\exists n_0 \in \mathbb{N}: \forall n > n_0, \forall x: | F_n(x) - F(x) | < \epsilon $\\
Wähle $n_0: \forall x: F(x - \epsilon) - \epsilon \leq F_n(x) \leq F(x + \epsilon) $ und vice versa.
Mit der Dreiecksungleichung erhalten wir:
\begin{align*}
  &\forall x: | F(x) - F_n(x)| \leq | F(x) - F(x + \delta) | + | F(x + \delta) - F_n(x) | \\
\end{align*}
Wir wissen aus der Analysis, dass stetige, monotone, beschränkte Funktionen sogar gleichmäßig stetig sind. Da $F$ eine solche Funktion ist können wir den ersten Term gleichmäßig klein machen, wenn wir $\delta$ hinreichend klein wählen.

Betrachten wir nun den zweiten Term. Mit einigen Umformungen erhalten wir aus der Konvergenz der Lèvy-Prohorov-Metrik:
\begin{align*}
 \forall x: F_n(x)- F(x+\delta)\leq \delta \tag{\textit{iii}} \label{eq:third} \\
 \forall x: F(x-\delta)-\delta\leq F_n(x)\tag{\textit{iv}} \label{eq:forth} \\
\end{align*}
Nun gilt wegen \eqref{eq:forth} auch
\begin{align*}
  \forall x: F(x+\delta)-F_n(x)\leq F(x+\delta)-(F(x-\delta)-\delta)=F(x+\delta)-F(x-\delta)+\delta
\end{align*}
Wegen der gleichmäßigen Setigkeit von $F$ können wir nun auch diesen Ausdruck für hinreichend kleines $\delta$ gleichmäßig klein machen, also erhalten wir gemeinsam mit \eqref{eq:third}
\begin{align*}
  \forall x:\vert F(x+\delta)-F_n(x)\vert<\frac{\epsilon}{2}
\end{align*}
Damit haben wir auch den zweiten Term gleichmäßig abgeschätzt und erhalten insgesamt die gleichmäßige Konvergenz.
\end{proof}
