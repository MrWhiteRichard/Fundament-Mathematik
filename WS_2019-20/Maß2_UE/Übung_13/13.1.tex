\begin{exercise}
Bestimmen Sie die Momente
\begin{align*}
  M_n = \mathbb{E}(X^n)
\end{align*}
und die Momentenerzeugende
\begin{align*}
  M(t) = \mathbb{E}(e^{Xt})
\end{align*}
für die Gammaverteilung.
\end{exercise}
\begin{solution}

  Die Dichte der Gammaverteilung ist gegeben durch
  $
  f(y) = \frac{ \lambda^{\alpha}{y^{\alpha-1}}}{\Gamma(\alpha)} \mathrm{e}^{-\lambda y} [x>0]
  $.

  Es gilt
  \begin{align*}
  M_n
         = \mathbb{E}(X^n)
         = \int X^n d\mathbb{P}
         = \int \mathrm{id}^n d\mathbb{P}_X
         = \int_{-\infty}^{\infty} y^n f(y)~d\lambda(y)
         = \int_{0}^{\infty} \frac{\lambda^{\alpha} y^{n+\alpha -1}}{\Gamma(\alpha)} \mathrm{e}^{-\lambda y}~d\lambda(y)
         \\= \frac{\lambda^{\alpha}}{\Gamma(\alpha)} \int_{0}^{\infty} y^{n+\alpha -1} \mathrm{e}^{-\lambda y}~d\lambda(y)
         = \left[ u = \lambda y \right]
           \frac{\lambda^{\alpha}}{\Gamma(\alpha)} \int_{0}^{\infty} \left(\frac{u}{\lambda}\right)^{n+\alpha -1}
                \mathrm{e}^{-u} \frac{d\lambda(u)}{\lambda}
         = \frac{\lambda^{n}}{\Gamma(\alpha)} \int_{0}^{\infty} u^{n+\alpha-1} \mathrm{e}^{-u}
         = \frac{\lambda^{n}}{\Gamma(\alpha)} \Gamma(n+\alpha).
  \end{align*}



  Die Momenterzeugende berechnet sich über

  \begin{align*}
  M(t) = \mathbb{E}(\mathrm{e}^{Xt})
       = \int \mathrm{e}^{ty} f(y) d\mathbb{P}_X
       = \int_{0}^{\infty} \mathrm{e}^{ty} \frac{\lambda^{\alpha} y^{\alpha -1}}{\Gamma(\alpha)} \mathrm{e}^{-\lambda y}~d\lambda(y)
       = \frac{\lambda^\alpha}{(\lambda-t)^\alpha}
         \underbrace{
         \int_{0}^{\infty} \frac{(\lambda - t)^\alpha y^{\alpha-1}}{\Gamma(\alpha)} \mathrm{e}^{-(\lambda-t)y} d\lambda(y)
         }_{=~1}
       = \left(\frac{\lambda}{\lambda-t}\right)^\alpha.
  \end{align*}


\end{solution}
