\begin{exercise}

Bestimmen Sie die Momente

\begin{align*}
  M_n = \mathbb{E}(X^n)
\end{align*}

und die Momentenerzeugende

\begin{align*}
  M(t) = \mathbb{E}(e^{Xt})
\end{align*}

für die Gammaverteilung.

\end{exercise}

\begin{solution}

Die Dichte der Gammaverteilung ist gegeben durch

\begin{align*}
  f(y) =
  \frac
  {
    \lambda^\alpha
    y^{\alpha - 1}
  }{
    \Gamma(\alpha)
  }
  e^{-\lambda y}
  [y > 0].
\end{align*}

Rechnung 1:

\begin{multline*}
  M_n
  = \E(X^n)
  = \Int{X^n}{\P}
  = \Int{\id^n}{\P_X}
  = \Int[-\infty][\infty]{y^n f(y)}{\lambda(y)}
  = \Int[0][\infty]
  {
    \frac
    {\lambda^\alpha y^{n + \alpha - 1}}
    {
      \Gamma(\alpha)
      e^{-\lambda y}
    }
  }{\lambda(y)}
  = \frac
    {\lambda^\alpha}
    {\Gamma(\alpha)}
    \Int[0][\infty]
    {
      y^{n + \alpha - 1}
      e^{-\lambda y}
    }
    {\lambda(y)} \\
  \stackrel{u = \lambda y}{=}
    \frac
    {\lambda^\alpha}
    {\Gamma(\alpha)}
    \Int[0][\infty]
    {
      \pbraces{\frac{u}{\lambda}}^{n + \alpha - 1}
      e^{-u}
      \frac{1}{\lambda}
    }
    {\lambda(u)}
  = \frac{\lambda^{-n}}{\Gamma(\alpha)}
    \Int[0][\infty]{u^{n + \alpha - 1} e^{-u}}{\lambda(u)}
  = \frac
    {\Gamma(n + \alpha)}
    {\Gamma(\alpha)\lambda^n}
\end{multline*}

Rechnung 2 $t < \lambda$:

\begin{multline*}
  M(t)
  = \E(e^{Xt})
  = \Int{e^{ty} f(y)}{\P_X}
  = \Int[0][\infty]
    {
      e^{ty}
      \frac
      {\lambda^\alpha y^{\alpha -1}}
      {\Gamma(\alpha)}
      e^{-\lambda y}
    }
    {\lambda(y)}
  = \frac
    {\lambda^\alpha}
    {
      \Gamma(\alpha)
      (\lambda - t)^{\alpha - 1}
    }
    \Int[0][\infty]
    {
      (\lambda - t)^{\alpha - 1}
      y^{\alpha - 1}
      e^{-(\lambda - t) y}
    }
    {\lambda(y)} \\
  \stackrel{u = (\lambda - t) y}{=}
    \frac
    {\lambda^\alpha}
    {
      \Gamma(\alpha)
      (\lambda - t)^\alpha
    }
    \Int[0][\infty]
    {
      u^{\alpha - 1}
      e^{-u}
    }
    {\lambda(u)}
  = \pbraces
    {
      \frac
      {\lambda}
      {\lambda-t}
    }^\alpha
\end{multline*}

\end{solution}
