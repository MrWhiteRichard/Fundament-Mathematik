\begin{exercise}
$X$ und $Y$ seien unabhängige Zufallsvariablen. Zeigen Sie, dass $X + Y$ genau
dann integrierbar ist, wenn $X$ und $Y$ integrierbar sind. (Schätzen Sie in
$\E(Y[|X| < M])$, $|Y|$ mit der Dreiecksungleichung ab; damit kann man
zeigen, dass in der Bedingung für die Existenz von $\E(X)$ der Realteil
von $\phi_X$ durch den Betrag ersetzt werden kann).
\end{exercise}

\begin{solution}

Für "$\Rightarrow$", zeigen wir zuerst folgendes Integrierbarkeits-Kriterium, für eine Zufallsvariable $Z$.

\begin{align*}
  \E(|Z|) \leq
  \sum_{n=0}^\infty \P(|Z| > n) \leq
  \E(|Z|) + 1
\end{align*}

Erster Schritt:

\begin{align*}
  |Z|
  = \Int[0][|Z|]{}{x}
  = \Int[0][\infty]{\1_\bbraces{|Z| > x}}{x}
\end{align*}

Zweiter Schritt:

\begin{multline*}
  \E(|Z|)
  \stackrel{\text{Fubini}}{=}
    \Int[0][\infty]{\E(\1_\bbraces{|Z| > x})}{x}
  = \Int[0][\infty]{\P(|Z| > x)}{x}
  \leq
    \Int[0][\infty]{\P(|Z| > \floor{x})}{x} \\
  = \sum_{n=0}^\infty \P(|Z| > n)
  = 1 + \sum_{n=1}^\infty \P(|Z| > n)
  = 1 + \Int[0][\infty]{\P(|Z| > \ceil{x})}{x}
  \leq
    1 + \underbrace{\Int[0][\infty]{\P(|Z| > x)}{x}}_{\E(|Z|)}
\end{multline*}

Zurück zu unserem Beispiel. Mit der Dreiecksungleichung nach unten und der Unabhängigkeit von $X, Y$, folgt

\begin{align*}
  \P(|X + Y| > n)
  \geq
    \P(|X| - |Y| > n)
  \geq
    \P(|X| > n + m, -|Y| > -m)
  = \P(|X| > n + m) \cdot \P(|Y| < m).
\end{align*}

Es muss $\Exists m \in \N: \P(|Y| < m) > 0$. Damit folgt

\begin{align*}
  \sum_{n=m}^\infty \P(|X| > n)
  = \sum_{n=0}^\infty \P(|X| > n + m)
  \leq
    \Frac
    {\P(|Y| < m)}
    {\sum_{n=0}^\infty \P(|X + Y| > n)}
  < \infty.
\end{align*}

Mit unserem Integrierbarkeits-Kriterium, erhalten wir

\begin{align*}
  \E(|X|)
  \leq
    \sum_{n=0}^\infty \P(|X| > n)
  \leq
    \sum_{n=0}^{m-1} \P(|X| > n) +
    \sum_{n=m}^\infty \P(|X| > n)
  < \infty.
\end{align*}

Analog, zeigt man $Y \in L^1$. "$\Leftarrow$" ist trivial.

\begin{align*}
  \E(X + Y) = \E(X) + \E(Y) < \infty
\end{align*}

\end{solution}
