\begin{exercise}
Die kumulantenerzeugende Funktion einer Zufallsvariable $X$ ist
\begin{align*}
  K_X(t) = \log(M_X(t)).
\end{align*}
Wenn $M_X$ in einer Umgebung von 0 existiert, dann kann man $K_X$ als Potenzreihe schreiben:
\begin{align*}
  K_X(t) = \sum_n\frac{\kappa_nt^n}{n!}.
\end{align*}
Die Koeffizienten $\kappa_n$ heißen die Kumulanten von $X$. Drücken Sie $\kappa_n,
n = 2,\dots,5$ durch die zentralen Momente
\begin{align*}
  m_n = \mathbb{E}((X-\mathbb{E}(X))^n)
\end{align*}
von $X - \mathbb{E}(X)$ aus. (mit $\mu = \mathbb{E}(X)$ und $Y = X - \mu$ gilt
$K_X(t) = \mu t + K_Y(t)$; diese Darstellung der Kumulanten ist einfacher als die
durch die gewöhnlichen Momente, die allerdings im Internet leichter zu finden ist).
\end{exercise}
\begin{solution}

Trivial

\end{solution}
