\documentclass{article}

\usepackage{fullpage}
\usepackage{amsthm, amssymb, amsmath, bbm}

\newtheorem{lemma}[]{Lemma}
\newtheorem*{satz}{Satz}
\newtheorem{definition}[]{Definition}

\parindent 0pt

\title
{
  Maß- und Wahrscheinlichkeitstheorie 2 - Übung \\
  \vspace{4pt}
  \normalsize
  \textit{11. UE am 08.01.2020}
}
\author
{
  Richard Weiss       \and
  Florian Schager     \and
  Christian Sallinger \and
  Fabian Zehetgruber  \and
  Paul Winkler        \and
  Christian Göth
}
\date{}

\begin{document}

\maketitle

\section*{Aufgabe 1}
\begin{lemma}
    Wenn $\forall n\in\mathbb{N}:f_n:\mathbb{R}\to\mathbb{R}$ stetig und beschränkt sind, $f:\mathbb{R}\to\mathbb{R}$ stetig ist, $\forall n\in\mathbb{N}:P_n$ sowie $P$ Wahrscheinlichkeitsmaße auf $(\mathbb{R},\mathfrak{B})$ sind und $f_n\to f$ gleichmäßig und $P_n\to P$ schwach, dann gilt
    \begin{align*}
        \lim_{n\to\infty}\int f_n\mathrm{d}P_n=\int f\mathrm{d}P.
    \end{align*}
\end{lemma}
\begin{proof}[Beweis.]
    Wir schreiben ganz einfach mit der Dreiecksungleichung
    \begin{align*}
        \left\vert\int f_n\mathrm{d}P_n-\int f\mathrm{d}P\right\vert\leq\left\vert\int (f_n-f)\mathrm{d}P_n\right\vert+\left\vert\int f\mathrm{d}P_n-\int f\mathrm{d}P\right\vert
    \end{align*}
    Für hinreichend gorße $n\in\mathbb{N}$ wird der erste Summand wegen $f_n\to f$ gleichmäßig und der zweite Summand wegen $P_n\to P$ schwach klein.
\end{proof}


\section*{Aufgabe 2}
\begin{exercise}

Zeigen Sie: Wenn für ein $t \neq 0$ $\phi_X(t) = 1$ gilt, dann nimmt $X$ mit Wahrscheinlichkeit $1$ nur Werte der Form $2n \pi/t$, $n \in \Z$ an. Gilt $\phi(t_1) = \phi(t_2) = 1$ für zwei inkommensurable Werte $t_1$ und $t_2$ (d.h. $t_1/t_2$ ist irrational), dann gilt $X = 0$ fast sicher.

\end{exercise}

\begin{solution}

Für die erste Aussage ist zu zeigen, dass

\begin{align*}
  \P(X \in \frac{2 \pi}{t} \Z) = 1.
\end{align*}

Man betrachte folgende Gleichungskette, wobei der letzte Summand wegfällt, weil $1 \in \R$.

\begin{align}
  1
  = \phi_X(t)
  = \E(e^{iXt})
  = \Int{e^{iX(\omega)t}}{\P(\omega)}
  = \Int{\cos(X(\omega)t)}{\P(\omega)} +
    i \underbrace{\Int{\sin(X(\omega)t)}{\P(\omega)}}_0
  \label{gleichungskette}
\end{align}

Es ist bekanntlich $\cos^{-1}(\Bbraces{1}) = 2 \pi \Z$. Angenommen, $\Exists N \in \S: \P(N) > 0, \: \Forall \omega \in N:$

\begin{align*}
  X(\omega) \not \in \frac{2 \pi}{t} \Z
  \Rightarrow
  \cos(X(\omega)t) < 1.
\end{align*}

Dann gilt aber für das Integral über $N$, dass

\begin{align*}
  \Int[N]{\cos(X(\omega)t)}{\P(\omega)} <
  \Int[N]{}{\P(\omega)} =
  \P(N).
\end{align*}

Man setzt die obere Gleichungskette \eqref{gleichungskette} fort und erhält einen Widerspruch.

\begin{align*}
  \Int[N]{\cos(X(\omega)t)}{\P(\omega)} +
  \Int[N^c]{\cos(X(\omega)t)}{\P(\omega)} <
  P(N) + P(N^c) = P(\Omega) = 1
\end{align*}

Somit lässt sich \eqref{gleichungskette} richtig weiterführen.

\begin{align*}
  \Int[\bbraces{X \in \frac{2 \pi}{t} \Z}]{\cos(X(\omega)t)}{\P(\omega)}
  = \Int[\bbraces{X \in \frac{2 \pi}{t} \Z}]{}{\P}
  = \P(X \in \frac{2 \pi}{t} \Z)
\end{align*}

Für die zweite Aussage ist zu zeigen, dass

\begin{align*}
  \P(X = 0) = 1.
\end{align*}

Laut Vorher, gilt für $t = t_1, t_2$, dass

\begin{align*}
  \sum_{n \in \Z} \P(X = \frac{2 \pi}{t} n)
  = \P(X \in \frac{2 \pi}{t} \Z)
  = 1.
\end{align*}

Also

\begin{align*}
  \Exists N_1 \in \S: \P(N_1^c) = 0, \:
  \Forall \omega \in N_1:
  \Exists n_1 \in \Z:
  X(\omega) = \frac{2 \pi n_1}{t_1}, \\
  \Exists N_2 \in \S: \P(N_2^c) = 0, \:
  \Forall \omega \in N_2:
  \Exists n_2 \in \Z:
  X(\omega) = \frac{2 \pi n_2}{t_2}.
\end{align*}

Wir wollen eine Aussage über die Elemente des Schnittes $N := N_1 \cap N_2 \in \S$ dieser Mengen treffen.

\begin{align*}
  \frac{2 \pi n_1}{t_1} = \frac{2 \pi n_2}{t_2}
  \Rightarrow
  n_1 = \underbrace{t_1/t_2}_{\in \R \setminus \Q} n_2
  \Rightarrow
  n_1, n_2 = 0
\end{align*}

Dazu brauchen wir aber auch noch, dass

\begin{align*}
  \P(N^c) =
  \P((N_1 \cap N_2)^c) =
  \P(N_1^c \cup N_2^c) \leq
  \P(N_1^c) + \P(N_2^c) = 0.
\end{align*}

Somit erhält man,

\begin{align*}
  \Exists N \in \S: \P(N^c) = 0, \:
  \Forall \omega \in N:
  X(\omega) = 0.
\end{align*}

\end{solution}


\section*{Aufgabe 3}
\begin{exercise}

Hier könnte Ihre Werbung stehen!

\begin{itemize}
    \item[(a)] $F$ und $G$ seien (Wahrscheinlichkeits-) Verteilungsfunktionen, $d$ die Lévy-Prohorov-Metrik. Zeigen Sie für die verallgemeinerten Inversen

    \begin{equation*}
      d(F^{-1},G^{-1}) \leq d(F,G).
    \end{equation*}

    \item[(b)] Zeigen Sie: $F_n \longrightarrow F$ genau dann, wenn es auf einem geeigneten Wahrscheinlichkeitsraum Zufallsvariablen $X_n \sim F_n, X \sim F$ gibt mit $X_n \rightarrow X$ fast sicher (Darstellungssatz von Skorohod).
\end{itemize}

\end{exercise}

\begin{solution}

(a) Die verallgemeinerte Inverse wird definiert für $p \in (0, 1]$.

\begin{align*}
  F^{-1}(p) := \inf \Bbraces{x \in \R: p \leq F(x)}
\end{align*}

Wir benennen die folgenden Mengen.

\begin{align*}
  A  & := \Bbraces
  {
    \epsilon > 0:
    \Forall x \in \R:
    F(x - \epsilon) - \epsilon \leq
    G(x) \leq
    F(x + \epsilon) + \epsilon
  } \\
  B & := \Bbraces
  {
    \epsilon > 0:
    \Forall x \in \R:
    F^{-1}(x - \epsilon) - \epsilon \leq
    G^{-1}(x) \leq
    F^{-1}(x + \epsilon) + \epsilon
  }
\end{align*}

Für $\inf B \leq \inf A$, genügt es $A \subseteq B$ zu zeigen. Sei also $\epsilon \in A$, dann gilt $\epsilon > 0$ und $\Forall x \in \R:$

\begin{align*}
  F(x - \epsilon) - \epsilon \leq
  G(x) \leq
  F(x + \epsilon) + \epsilon.
\end{align*}

Wir wollen zeigen, dass

\begin{align*}
  F^{-1}(x - \epsilon) - \epsilon \leq
  G^{-1}(x) \leq
  F^{-1}(x + \epsilon) + \epsilon.
\end{align*}

Wir zeigen die erste Ungleichung; die zweite geht (wahrscheinlich) analog. Für die letzte Ungleichung beachte man $M \subseteq N$.

\begin{align*}
  \inf \Bbraces{y \in \R: x - \epsilon \leq F(y)} - \epsilon
  & = \inf \Bbraces{y - \epsilon \in \R: x - \epsilon \leq F(y)} \\
  & = \inf \Bbraces{z \in \R: x - \epsilon \leq F(z + \epsilon)} \\
  & = \inf \underbrace
      {
        \Bbraces{z \in \R: x \leq F(z + \epsilon) + \epsilon}
      }_{=: N} \\
  & \leq \inf \underbrace
      {
        \Bbraces{y \in \R: x \leq G(y)}
      }_{=: M}
\end{align*}

(b) Wir betrachten den Wahrscheinlichkeitsraum $(]0, 1[, \B(]0, 1[), \lambda)$ und bemerken, dass $F_n, F: \R \to [0, 1]$. Die verallgemeinerten Inversen $F_n^{-1}, F^{-1}$ sind monoton und somit messbar, also $X_n := F_n^{-1}, X := F^{-1}$ Zufallsvariablen. \\

\say{$\Rightarrow$}: Es ist zu zeigen, dass

\begin{align*}
  \P(X_n \xrightarrow{n \to \infty} X) = 1.
\end{align*}

Schwache Konvergenz ist bekanntlich äquivalent zur Konvergenz in der Lévy-Prohorov-Metrik.

\begin{align*}
  F_n \xrightarrow[\text{schwach}]{n \to \infty} F
  \Leftrightarrow
  d(F_n, F) \xrightarrow{n \to \infty} 0.
\end{align*}

Laut (a) wissen wir also bereits, dass

\begin{align*}
  d(X_n, X) =
  d(F_n^{-1}, F^{-1}) \leq
  d(F_n, F) \xrightarrow{n \to \infty} 0.
  \label{schwache_konvergenz}
\end{align*}

Eine weitere Äquivalenz zur schwach Konvergenz, erlaubt $\Forall \omega \in \mathcal{C}(X):$

\begin{align*}
  X_n(\omega) \xrightarrow{n \to \infty} X(\omega).
\end{align*}

Weil in jedem Intervall eine rationale Zahl ist und $|\Q| = \aleph_0$, gibt es nur abzählbar viele $F_n$- bzw. $F$-konstante Bereiche. $X_n, X$ haben genau dort ihre Sprungstellen, also ebenfalls nur abzählbar viele, also bilden sie eine $\lambda$-Nullmenge. Q.E.D. \\

\say{$\Leftarrow$}: $\lambda$ ist auf unserem Wahrscheinlichkeitsraum endlich, also folgt mit dem Satz von Egorov und Übung 11, Beispiel 6, dass

\begin{align*}
  X_n \xrightarrow[\fastsicher]{n \to \infty} X
  \Rightarrow
  X_n \xrightarrow[\text{in WSK}]{n \to \infty} X
  \Rightarrow
  F_n \xrightarrow[\text{schwach}]{n \to \infty} F.
\end{align*}

\end{solution}


\section*{Aufgabe 4}
\begin{exercise}

Ein Würfel wird $100$ mal geworfen. Bestimmen Sie die Wahrscheinlichkeit, dass die Summe der Augenzahlen mehr als $375$ beträgt.

\end{exercise}

\begin{solution}

Wir definieren, um einen Wurf des Würfels darzustellen,
\begin{align*}
    \Omega := \{1,2,3,4,5,6\}
\end{align*}
sowie das Wahrscheinlichkeitsmaß
\begin{align*}
    \mathbb{P}:2^\Omega \to \mathbb{R}: \{\{\omega_1\},\dots,\{\omega_l\}\} \mapsto \vert \{\{\omega_1\},\dots\{\omega_l\}\} \vert \frac{1}{6}
\end{align*}
und dann auf dem Wahrcheinlichkeitsraum $(\Omega,2^\Omega,\mathbb{P})$ eine Zufallsvariable
\begin{align*}
    X:(\Omega,2^\Omega) \to (\mathbb{R},\mathfrak{B}): \omega \mapsto \omega.
\end{align*}
Nun soll für ein beliebiges $n \in \mathbb{N}$, in unserem Fall $n=100$,
\begin{align*}
    X_n:(\Omega^n,2^{\Omega^n}) \to (\mathbb{R},\mathfrak{B}): (\omega_1,\dots,\omega_n)^T \mapsto \sum_{k=1}^{n} X(\omega_k)
\end{align*}
unseren Versuch beschreiben. Wir berechnen zuerst
\begin{align*}
    \mathbb{E}(X_{100}) = \sum_{k=1}^{100} \mathbb{E}(X) = 100\frac{7}{2} = 350
\end{align*}
sowie unter Benützung der Unabhängigkeit der $X_k$
\begin{align*}
    \mathbb{V}(X_{100}) = \sum_{k=1}^{100} \mathbb{V}(X) = \sum_{k=1}^{100} \frac{1}{6} \pbraces{\pbraces{1-\frac{7}{2}}^2+\cdots+\pbraces{6-\frac{7}{2}}^2} = 100 \frac{35}{12} = \frac{875}{3}
\end{align*}
Nun verwenden wir (den lokalen Grenzwertsatz???) um
\begin{align*}
    \mathbb{P}_{100}(X_{100} > 375) &= 1 - \mathbb{P}_{100}(X_{100} \leq 375) = 1 - \mathbb{P}_{100}\pbraces{\frac{X_{100} - \mathbb{E}(X_{100})}{\sqrt{\mathbb{V}(X_{100})}} \leq \frac{375 - \mathbb{E}(X_{100})}{\sqrt{\mathbb{V}(X_{100})}}} \\
    &= 1 - \mathbb{P}_{100}\pbraces{\frac{X_{100} - \mathbb{E}(X_{100})}{\sqrt{\mathbb{V}(X_{100})}} \leq \frac{25\sqrt{3}}{\sqrt{875}}} \approx 1 - \Phi(1.464) \approx 1 - 0.92785 = 0.07215
\end{align*}
\end{solution}


\section*{Aufgabe 5}
\begin{exercise}

Wie oft muss man würfeln, damit die Wahrscheinlichkeit dafür, dass die Summe der Augenzahlen größer als $200$ ist, mindestens $0.9$ ist?

\end{exercise}

\begin{solution}

Jetzt suchen wir das kleinste $n \in \N$ für das
\begin{align*}
    \P_n(X_n > 200) \geq \frac{9}{10}
\end{align*}
gilt.
\begin{align*}
    \P_n(X_n > 200) \geq \frac{9}{10} &\Leftrightarrow \P_n\pbraces{\frac{X_n-n\frac{7}{2}}{\sqrt{n\frac{35}{12}}} > \frac{200 -n\frac{7}{2}}{\sqrt{n\frac{35}{12}}}} \geq \frac{9}{10} \\
    &\Leftrightarrow \P_n\pbraces{\frac{X_n-n\frac{7}{2}}{\sqrt{n\frac{35}{12}}} \leq \frac{200 -n\frac{7}{2}}{\sqrt{n\frac{35}{12}}}} < \frac{1}{10}
\end{align*}

Nun können wir wegen ??? diese Wahrscheinlichkeit approximieren durch

\begin{align*}
    \P_n\pbraces{\frac{X_n-n\frac{7}{2}}{\sqrt{n\frac{35}{12}}} \leq \frac{200 -n\frac{7}{2}}{\sqrt{n\frac{35}{12}}}} \approx \Phi\pbraces{\frac{200 -n\frac{7}{2}}{\sqrt{n\frac{35}{12}}}}
\end{align*}

Und berechnen nun unter Benützung der Symmetrie der Standardnormalverteilung
\begin{align*}
    \Phi\pbraces{\frac{200 -n\frac{7}{2}}{\sqrt{n\frac{35}{12}}}} < \frac{1}{10} \Leftrightarrow \Phi\pbraces{-\frac{200 -n\frac{7}{2}}{\sqrt{n\frac{35}{12}}}} > \frac{9}{10}
\end{align*}

Schlägt man das in der Tabelle nach dann erhält man ungefähr

\begin{align*}
    -\frac{200 -n\frac{7}{2}}{\sqrt{n\frac{35}{12}}} > 1.28.
\end{align*}

Wolfram-Alpha liefert als Lösung

\begin{align*}
    n > \frac{16(9407+8\sqrt{9391})}{2625} \approx 62.06.
\end{align*}

\end{solution}


\section*{Aufgabe 6}
\begin{lemma}
    Es gelten folgende Aussagen:
    \begin{itemize}
        \item[(a)] Seien $(X_n)$ und $(Y_n)$ Folgen von Zufallsvariablen sowie $X$ eine Zufallsvariable auf dem Maßraum $(\Omega,\mathfrak{S},P)$. Es gelte $X_n\to X$ in Verteilung und $Y_n\to 0$ in Wahrscheinlichkeit. Dann gilt $X_n+Y_n\to X$ in Verteilung.
        \item[(b)] Konvergiert eine Folge $X_n$ auf dem Maßraum $(\Omega,\mathfrak{S},P)$ in Wahrscheinlichkeit gegen $X$, so gilt auch $X_n\to X$ in Verteilung.
        \item[(c)] Eine Folge $X_n$ auf dem Maßraum $(\Omega,\mathfrak{S},P)$ konvergiert in Verteilung gegen $0$ genau dann, wenn $X_n$ in Verteilung gegen 0 konvergiert.
    \end{itemize}
\end{lemma}
\begin{proof}[Beweis.]
(a): \\
Durch Einsetzen in die Definition, sowie Satz 12.5 erhalten wir:
\begin{align*}
&\forall \epsilon > 0:  \lim_{n \rightarrow \infty} \mathbb{P}(| Y_n | > \epsilon) = 0  \tag{\textit{i}} \label{eq:first} \\
&\lim_{n \rightarrow \infty} d(X_n,X) := \inf \{\epsilon \geq 0: \forall x: X_n(x - \epsilon) - \epsilon \leq X(x) \leq X_n(x + \epsilon) + \epsilon\} = 0 \tag{\textit{ii}} \label{eq:second} \\
\end{align*}
Wir zeigen:
\begin{align*}
  \forall x \in \mathcal{C}(F_X): \lim_{n \rightarrow \infty} F_{X_n}(x) + F_{Y_n}(x) = F_X(x)& \\
  | F_{X_n}(x) + F_{Y_n}(x) - F_X(x) | = | \mathbb{P}(X_n + Y_n \leq x) - \mathbb{P}(X \leq x) | &\leq \\
  | \mathbb{P}(X_n + Y_n \leq x) - \mathbb{P}(X_n \leq x) | + | \mathbb{P}(X_n \leq x) - \mathbb{P}(X \leq x) |
\end{align*}
Der zweite Term lässt sich dabei aufgrund der schwachen Konvergenz von $X_n$ gegen $X$ zu Null diskutieren.
\begin{align*}
  | &\mathbb{P}(X_n + Y_n \leq x) - \mathbb{P}(X_n \leq x) | &\leq \\
  | &\mathbb{P}([X_n + Y_n \leq x] \cap [|Y_n| < \epsilon]) - \mathbb{P}([X_n \leq x] \cap [|Y_n| \geq \epsilon]) | &\leq \\
  | &\mathbb{P}([|Y_n| < \epsilon]) | + | \mathbb{P}([X_n \leq x + \epsilon]\backslash ([X_n + Y_n \leq x] \cap [|Y_n| < \epsilon])) |
  + | \mathbb{P}([X_n \leq x]) - \mathbb{P}([X_n \leq x + \epsilon]) |
\end{align*}
Jetzt wird der Ausdruck schon wieder ziemlich lang, Zeit Ballast abzuwerfen: \\
Der erste Ausdruck konvergiert aufgrund \ref{eq:first} gegen 0 und der letzte wegen der Rechtsstetigkeit von Verteilungsfunktionen. \\
Und munter weiter:
\begin{align*}
  | \mathbb{P}([X_n \leq x + \epsilon] \backslash ([X_n + Y_n \leq x] \cap [|Y_n| \leq \epsilon])) | &\leq \\
  | \mathbb{P}([X_n \leq x + \epsilon] \cap [X_n + Y_n > x]) + \mathbb{P}([X_n \leq x + \epsilon] \cap [|Y_n| > \epsilon]) | &
\end{align*}
Mal wieder lassen wir den zweiten Ausdruck verschwinden.
\begin{align*}
  | \mathbb{P}([X_n \leq x + \epsilon] \cap [X_n + Y_n > x])| &\leq \\
  | \mathbb{P}([x - Y_n < X_n \leq x + \epsilon] \cap [|Y_n| \leq \epsilon]) +  \mathbb{P}([x - Y_n < X_n \leq x + \epsilon] \cap [|Y_n| > \epsilon])|
\end{align*}
Selber Trick wie immer.
\begin{align*}
  | \mathbb{P}([x - Y_n < X_n \leq x + \epsilon] \cap [|Y_n| \leq \epsilon]) | &\leq \\
  | \mathbb{P}([x - \epsilon \leq X_n \leq x + \epsilon]) | &= \\
  | F_{X_n}(x + \epsilon) - F_{X_n}(x - \epsilon) |
\end{align*}
Um diese letzte Hürde noch zu bezwingen müssen wir wieder die Lèvy-Prohorov-Metrik zurate ziehen:
\begin{align*}
  | F_{X_n}(x + \epsilon) - F_{X_n}(x - \epsilon) | &\leq \\
  | 2\epsilon + F_X(x + 2\epsilon) - F_X(x -2\epsilon) |
\end{align*}
Und dieser Ausdruck verschwindet schließlich, da wir $x$ als Stetigkeitspunkt von $F_X$ vorausgesetzt haben. \\
(b): \\
Folgt direkt aus (a), wenn man für die Folge $X_n$ die konstante Nullfolge wählt. \\
(c): \\
Aus (b) erhalten wir die Rückrichtung der Aussage, Satz 17.5. Kusolitsch liefert uns die Hinrichtung:
\begin{satz}\textbf{17.5.}\\
Sind $X_n$ Zufallsvariablen auf beliebigen Wahrscheinlichkeitsräumen $(\Omega_n, \Sigma_n, \mathbb{P}_n)$, dann folgt aus
$X_n \implies a, a \in \mathbb{R} $ auch
\[\forall \epsilon > 0: \lim_{n \rightarrow \infty} \mathbb{P}_n(|X_n - a| > \epsilon) = 0 \]
\end{satz}
\end{proof}


\section*{Aufgabe 7}
\begin{lemma}
    Die Levy-Prokhorov-Metrik ist eine Metrik auf der Menge $M:=\{F:\mathbb{R}\to\mathbb{R}\mid F \text{ ist eine Verteilungsfunktion}\}$.
    \begin{center}
        $d(F,G) := \inf\{\epsilon > 0~|~ \forall x \in \mathbb{R}: F(x-\epsilon) - \epsilon \leq G(x) \leq F(x+\epsilon) + \epsilon \}.$ 
    \end{center}
\end{lemma}
\begin{proof}[Beweis.] Es sind drei Eigenschaften nachzuweisen.
    \begin{description}
    \item[(M1)] $d(F,G) = 0 \Leftrightarrow F = G$.

    Aus $d(F,G) = 0$ folgt definitionsgemäß $F(x-\epsilon) - \epsilon \leq G(x) \leq F(x+\epsilon) + \epsilon$ für beliebig kleine $\epsilon > 0$. Da $F$ monoton nichtfallend ist, existieren der links- und rechtsseitige Grenzwert bei $x$ und mit $\epsilon \rightarrow 0$ erhält man $F(x-) \leq G(x) \leq F(x+)$. $F$ und $G$ stimmen also an allen Stetigkeitspunkten von $F$ überein. $F$ und $G$ haben als Verteilungsfunktionen nur abzählbar viele Unstetigkeitsstellen. Für jedes $x \in \mathbb{R}$ gibt es eine Folge $x_{k} \searrow x$, die nur aus Stetigkeitsstellen von F und G besteht. Daher gilt $F(x) = \lim\limits_{k}{F(x_k)} = \lim\limits_{k}{G(x_k)} = G(x)$.\newline Die andere Richtung ist klar.

    \item[(M2)] $d(F,G) = d(G,F)$.

    $E_{FG} := \{\epsilon > 0~|~\forall x \in \mathbb{R}: F(x-\epsilon) - \epsilon \leq G(x) \leq F(x+\epsilon) + \epsilon\},\newline E_{GF} := \{\epsilon > 0~|~\forall x \in \mathbb{R}: G(x-\epsilon) - \epsilon \leq F(x) \leq G(x+\epsilon) + \epsilon\}$.

    Für alle $x \in \mathbb{R}$ gilt $G(x-\epsilon) - \epsilon \leq F(x) \Leftrightarrow G(x) \leq F(x+\epsilon) + \epsilon$; das erhält man sofort durch Einsetzen von $x+\epsilon$ und Addition von $\epsilon.$

    Analog zeigt man $F(x-\epsilon) - \epsilon \leq G(x) \Leftrightarrow F(x) \leq G(x+\epsilon) + \epsilon$. Daher gilt $E_{FG}$ = $E_{GF}$ und folglich 
    \begin{center}
        $d(F,G) = \inf(E_{FG}) = \inf(E_{GF}) = d(G,F).$
    \end{center}

    \item[(M3)] $d(F,H) + d(H,G) \geq d(F,G)$.

    Sei $d(F,H) \leq \epsilon_1$, $d(H,G) \leq \epsilon_2$. Dann gilt 

    $F(x - \epsilon_1 - \epsilon_2) - \epsilon_1 - \epsilon_2 \leq H(x - \epsilon_2) + \epsilon_2 \leq G(x) \leq H(x + \epsilon_2) +\epsilon_2 \leq F(x + \epsilon_1 + \epsilon_2) + \epsilon_1 + \epsilon_2$, also $\epsilon_1+\epsilon_2 \in E_{FG}$ und somit $\epsilon_1 +\epsilon_2 \geq d(F,G)$.

    Nun gilt $d(F,H) + d(H,G) = \inf\limits_{\epsilon_1 \in E_{FH}} \epsilon_1 + \inf\limits_{\epsilon_2 \in E_{HF}} \epsilon_2 = \inf\limits_{\epsilon_1 \in E_{FH},~\epsilon_2 \in E_{HF}} \epsilon_1 + \epsilon_2$. Infima erhalten Ungleichungen und wir die gewünschte Aussage.
    \end{description}
\end{proof}


\end{document}
