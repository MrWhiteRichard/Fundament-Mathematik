\begin{lemma}
    Wenn $\forall n\in\mathbb{N}:f_n:\mathbb{R}\to\mathbb{R}$ stetig und beschränkt sind, $f:\mathbb{R}\to\mathbb{R}$ stetig ist, $\forall n\in\mathbb{N}:P_n$ sowie $P$ Wahrscheinlichkeitsmaße auf $(\mathbb{R},\mathfrak{B})$ sind und $f_n\to f$ gleichmäßig und $P_n\to P$ schwach, dann gilt
    \begin{align*}
        \lim_{n\to\infty}\int f_n\mathrm{d}P_n=\int f\mathrm{d}P.
    \end{align*}
\end{lemma}
\begin{proof}[Beweis.]
    Wir schreiben ganz einfach mit der Dreiecksungleichung
    \begin{align*}
        \left\vert\int f_n\mathrm{d}P_n-\int f\mathrm{d}P\right\vert\leq\left\vert\int (f_n-f)\mathrm{d}P_n\right\vert+\left\vert\int f\mathrm{d}P_n-\int f\mathrm{d}P\right\vert
    \end{align*}
    Für hinreichend gorße $n\in\mathbb{N}$ wird der erste Summand wegen $f_n\to f$ gleichmäßig und der zweite Summand wegen $P_n\to P$ schwach klein.
\end{proof}
