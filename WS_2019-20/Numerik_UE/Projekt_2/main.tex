\documentclass{article}

\usepackage[utf8]{inputenc}
\usepackage{fullpage}
\usepackage{amsmath, amssymb, amsfonts, amsthm}
\usepackage{mathtools}
\usepackage{twoopt}
\usepackage{graphicx, subfig, float}
\usepackage{listings, xcolor}
\usepackage{placeins}
\usepackage{babel, ngerman}
\usepackage{verbatim}
\usepackage{booktabs}

\usepackage{multicol}
\setlength{\columnsep}{0cm}

\parindent 0pt

% special letters:

\newcommand{\N}{\mathbb{N}}
\newcommand{\Z}{\mathbb{Z}}
\newcommand{\Q}{\mathbb{Q}}
\newcommand{\R}{\mathbb{R}}
\newcommand{\C}{\mathbb{C}}
\newcommand{\K}{\mathbb{K}}
\newcommand{\T}{\mathbb{T}}
\newcommand{\E}{\mathbb{E}}
\newcommand{\V}{\mathbb{V}}
\renewcommand{\P}{\mathbb{P}}
\newcommand{\1}{\mathbbm{1}}

\newcommand  {\B}{\mathfrak{B}}
\renewcommand{\S}{\mathfrak{S}}

% quantors:

\newcommand{\Forall}{\forall \,}
\newcommand{\Exists}{\exists \,}
\newcommand{\ExistsOnlyOne}{\exists! \,}
\newcommand{\nExists}{\nexists \,}

% MISC symbols:

\newcommand{\landau}[1]
{
  {\scriptstyle \mathcal{O}}
  \pbraces{#1}
}

\newcommand{\Landau}[1]
{
  \mathcal{O}
  \pbraces{#1}
}

\newcommand{\eps}{\mathrm{eps}}

% graphics in a box:

\newcommandtwoopt
{\includegraphicsboxed}[3][][]
{
  \begin{figure}[!h]
    \begin{boxedin}
      \ifthenelse{\isempty{#2}}
      {
        \begin{center}
          \includegraphics[width = 0.75 \textwidth]{#3}
          \label{fig:#1}
        \end{center}
      }{
        \begin{center}
          \includegraphics[width = 0.75 \textwidth]{#3}
          \caption{#2}
          \label{fig:#1}
        \end{center}
      }
    \end{boxedin}
  \end{figure}
}

% braces:

\newcommand{\pbraces}[1]{{\left  ( #1 \right  )}}
\newcommand{\bbraces}[1]{{\left  [ #1 \right  ]}}
\newcommand{\Bbraces}[1]{{\left \{ #1 \right \}}}
\newcommand{\vbraces}[1]{{\left  | #1 \right  |}}
\newcommand{\Vbraces}[1]{{\left \| #1 \right \|}}
\newcommand{\abraces}[1]{{\left \langle #1 \right \rangle}}
\newcommand{\round}[1]{\bbraces{#1}}

\newcommand
{\floor}[1]
{{\left \lfloor #1 \right \rfloor}}

\newcommand
{\ceil} [1]
{{\left \lceil  #1 \right \rceil }}

% special functions:

\newcommand{\norm}  [2][]{\Vbraces{#2}_{#1}}
\newcommand{\diag}  [1]{\mathrm{diag} \: #1}
\newcommand{\dist}  [1]{\mathrm{dist} \: #1}
\newcommand{\mean}  [1]{\mathrm{mean} \: #1}
\newcommand{\erf}   [1]{\mathrm{erf} \: #1}
\newcommand{\id}    [1]{\mathrm{id} \: #1}
\newcommand{\sgn}   [1]{\mathrm{sgn} \: #1}
\newcommand{\supp}  [1]{\mathrm{supp} \: #1}
\newcommand{\arsinh}[1]{\mathrm{arsinh} \: #1}
\newcommand{\arcosh}[1]{\mathrm{arcosh} \: #1}
\newcommand{\artanh}[1]{\mathrm{artanh} \: #1}
\newcommand{\card}  [1]{\mathrm{card} \: #1}
\newcommand{\Span}  [1]{\mathrm{span} \: #1}
\newcommand{\Aut}   [1]{\mathrm{Aut} \: #1}
\newcommand{\End}   [1]{\mathrm{End} \: #1}
\newcommand{\ggT}   [1]{\mathrm{ggT} \: #1}
\newcommand{\kgV}   [1]{\mathrm{kgV} \: #1}
\newcommand{\ord}   [1]{\mathrm{ord} \: #1}
\newcommand{\grad}  [1]{\mathrm{grad} \: #1}
\newcommand{\ran}   [1]{\mathrm{ran} \: #1}
\newcommand{\graph} [1]{\mathrm{graph} \: #1}
\newcommand{\Inv}   [1]{\mathrm{Inv} \: #1}
\newcommand{\pv}    [1]{\mathrm{pv} \: #1}
\newcommand{\Mod}{\: \mathrm{mod} \:}
\newcommand{\Char}{\mathrm{char}}
\newcommand{\At}{\mathrm{At}}
\newcommand{\Ob}{\mathrm{Ob}}
\newcommand{\Hom}{\mathrm{Hom}}
\newcommand{\orthogonal}[3][]{#2 ~\bot_{#1}~ #3}
\newcommand{\Rang}{\mathrm{Rang}}

\newcommand
{\GL}[2][]
{\mathrm{GL}_{#1} \pbraces{#2}}

% fractions:

\newcommand{\Frac}[2]{\frac{1}{#1} \pbraces{#2}}
\newcommand{\nfrac}[2]{\nicefrac{#1}{#2}}

% derivatives & integrals:

\newcommandtwoopt
{\Int}[4][][]
{\int_{#1}^{#2} #3 ~\mathrm{d} #4}

\newcommandtwoopt
{\derivative}[3][][]
{
  \frac
  {\mathrm{d}^{#1} #2}
  {\mathrm{d} #3^{#1}}
}

\newcommandtwoopt
{\pderivative}[3][][]
{
  \frac
  {\partial^{#1} #2}
  {\partial #3^{#1}}
}

\newcommand
{\primeprime}
{{\prime \prime}}

\newcommand
{\primeprimeprime}
{{\prime \prime \prime}}

% Text:

\newcommand{\Quote}[1]{\glqq #1\grqq{}}
\newcommand{\Text}[1]{{\text{#1}}}
\newcommand{\fastueberall}{\text{f.ü.}}
\newcommand{\fastsicher}{\text{f.s.}}

\input{../../../Fundament-LaTeX/listings.tex}

\renewcommand{\figurename}{Abbildung}
\renewcommand{\tablename}{Tabelle}
\newtheorem*{lemma}{Lemma}

\title{Numerische Mathematik - Projektteil 2}
\author
{
  Richard Weiss       \and
  Florian Schager     \and
  Christian Sallinger \and
  Jakob Guttmann      \and
  Paul Winkler        \and
  Christian Göth
}
\date{}

\begin{document}

\maketitle

\section{Dünn besetzte Matrizen}
Eine häufige Problemstellung in der Numerischen Mathematik lautet lineare Gleichungssysteme mit großen, dünn besetzten Matrizen zu lösen.
Dabei kommen meist iterative Verfahren zum Einsatz, die in diesem Projekt effizient implementiert werden.
\subsection*{a)}
Wir generieren eine symmetrisch positiv definite Zufallsmatrix $A \in \mathbb{R}^{n\times n}$, wobei pro Zeile eine fixe Anzahl an Einträgen
ungleich Null sind und testen für eine zufällige rechte Seite $b \in \mathbb{R}^n$ bis zu welcher Größe $n$ das lineare
Gleichungssystem
\begin{align*}
  Ax = b
\end{align*}
mit einem direkten Löser (numpy.linalg.solve) in akzeptabler Zeit gelöst werden kann.
\begin{lstlisting}[language=Python]

def zufallsmatrix(n, nonzeros):
	A = np.concatenate((np.zeros((n,nonzeros)), np.random.rand(n, n-nonzeros)), axis = 1)
	for i in range(n):
		np.random.shuffle(A[i])
	return A + A.T + np.diag(np.random.rand(n)*10)

A_base = zufallsmatrix(5000,100)

x = [i for i in range(400,n+1,200)]
y = []
for n in x:
	A = A_base[:n,:n]
	b = np.random.rand(n)
	start = time.process_time()
	z = np.linalg.solve(A,b)
	end = time.process_time()
	y.append(end-start)
\end{lstlisting}
\begin{figure}
    \centering
    \includegraphics[width=\linewidth]{Aufgabe_1/plot_a.png}
    \caption{Rechenzeit abhängig von der Problemgröße}
    \label{fig:my_label}
\end{figure}
Wie in Abbildung 1 ersichtlich verhält sich die Rechenzeit kubisch in Relation zur Problemgröße.
Zum Testen wurde eine Zufallsmatrix mit 100 Nicht-Null-Einträgen pro Zeile (nicht exakt, da die Symmetriesierung den Wert pro
Zeile verzerrt) und einer Gesamtgröße von 5000 erstellt. Der direkte Löser wurde schließlich auf die $(200k \times 200k)$-dimensionalen Ausschnitte der oberen rechten Ecke angewandt ($2 \leq k \leq 25$). Wie man sieht erreichen wir damit schon langsam die Grenze
des akzeptabel Berechenbaren, für die volle $5000\times5000$-Matrix braucht der Algorithmus schon über 10 Sekunden.
\FloatBarrier
\subsection*{b)}
Um die Effizienz der Problemlösung zu steigern greifen wir also nicht mehr auf den direkten Löser zurück,
sondern implementieren eine optimierte Version des CG-Verfahrens.
\begin{lstlisting}[language=Python]
def cg(A,b,x0,tol):
    xt = x0
    r0 = b - np.dot(A,xt)
    d = r0
    while(np.linalg.norm(r0) > tol):
        prod = np.dot(np.transpose(r0),r0)
        prod2 = np.dot(A,d)
        alpha = prod/np.dot(np.transpose(d),prod2)
        xt = xt + alpha*d
        r0 = r0 - alpha*prod2
        beta = np.dot(np.transpose(r0),r0)/prod
        d = r0 + beta*d
    return xt
\end{lstlisting} \label{cg}
Der Algorithmus ähnelt in vielen Schritten Algorithmus 8.10 aus dem Numerik-Skript, weist aber durchaus einige Unterschiede auf.
Daher müssen wir noch beweisen, dass die beiden Algorithmen wirklich äquivalent sind.
Wir führen den Beweis mittels Induktion über die Anzahl der Iterationen. \\
Dabei bezeichnen wir mit * die Variablen aus unserem Algorithmus und ohne * die Variablen des Algorithmus
aus dem Skript. \\
Starten wir mit dem Induktionsanfang. \\
\begin{align*}
   &A = A^*, b = b^*, x_0 = x_0^* \\
   &r_0 = b - Ax_o = b^* - A^*x_0^* = r_0^* \\
   &d_0 = r_0 = r_0^* = d_0^* \\
   &\alpha_0 = \frac{r_0^Td_0}{d_0^TAd_0} = \frac{r_0^{*T}d_0^*}{d_0^{*T}Ad_0^*} = \frac{r_0^{*T}r_0^*}{d_0^{*T}Ad_0^*} = \alpha_0^* \\
   &x_1 = x_0 + \alpha_0d_0 = x_0^* + \alpha_0^*d_0^* = x_1^* \\
   &r_1 = b - Ax_1 = b^* - A^*x_1^* = b^* - A^*x_0^* - \alpha_0^*Ad_0^* = r_0^* - \alpha_0^*A^*d_0^* = r_1^* \\
   &\beta_0 = - \frac{r_1^TAd_0}{d_0^TAd_0} = \frac{-r_1^{*T}Ad_0^*}{d_0^{*T}Ad_0^*} = \frac{-r_1^{*T}Ar_0^*}{r_0^{*T}Ar_0^*}
\end{align*}
Unter Ausnutzung der Orthogonalität der Residuen erhalten wir: $r_1^{*T}r_0^* = 0$ und somit können wir den Bruch
folgendermaßen erweitern:
\begin{align*}
  \frac{-r_1^{*T}Ar_0^*}{r_0^{*T}Ar_0^*} = \frac{r_1^{*T}r_0^* -\alpha_0^*r_1^{*T}Ar_0^*}{\alpha_0^*r_0^{*T}Ar_0^*}
\end{align*}
Nun berechen wir
\begin{align*}
  r_1^{*T}r_1^* = r_1^{*T}(r_0^*-\alpha_0^*Ad_0^*) = r_1^{*T}r_0^* - \alpha_0^*r_1^{*T}Ar_0^*
\end{align*}
und setzen $\alpha_0^* = \frac{r_0^{*T}r_0^*}{d_0^{*T}Ad_0^*}$ ein:
\begin{align*}
  \frac{r_1^{*T}r_0^* -\alpha_0^*r_1^{*T}Ar_0^*}{\alpha_0^*r_0^{*T}Ar_0^*} = \frac{r_1^{*T}r_1^*}{\frac{r_0^{*T}r_0^*}{r_0^{*T}Ar_0^*}r_0^{*T}Ar_0^*}
  = \frac{r_1{*T}r_1^*}{r_0^{*T}r_0^*} = \beta_0^* \\
  d_1 = r_1 + \beta_0d_0 = r_1^* + \beta_0^*d_0^* = d_1^*
\end{align*}
Damit haben wir die Gleichheit der Variablen nach dem ersten Schleifendurchlauf gezeigt. \\
Sei nun nach $n$ Schleifendurchläufen die Gleichheit aller vorhergehenden Variablen vorausgesetzt: \\
\begin{align*}
  \alpha_{n-1} = \alpha_{n-1}*, x_n = x_n^*, r_n = r_n^*, \beta_{n-1} = \beta_{n-1}^*, d_n = d_n^* \\
  \alpha_n = \frac{r_n^Td_n}{d_n^TAd_n} = \frac{r_n^{*T}d_n^*}{d_n^{*T}A^*d_n^*} = \frac{r_n^{*T}(r_n^* + \beta_{n-1}^*d_{n-1}^*)}{d_n^{*T}A^*d_n^*} \\
\end{align*}
Jetzt nutzen wir die Eigenschaft: $\forall 0 \leq j < m: r_m^Td_j = 0$ und erhalten:
\begin{align*}
  \frac{r_n^{*T}(r_n^* + \beta_{n-1}^*d_{n-1}^*)}{d_n^{*T}A^*d_n^*} = \frac{r_n^{*T}r_n^*}{d_n^{*T}A^*d_n^*} = \alpha_n^* \\
  x_{n+1} = x_n + \alpha_nd_n = x_n^* + \alpha_n^*d_n^* = x_{n+1}^* \\
  r_{n+1} = b - Ax_{n+1} = b - A(x_n + \alpha_nd_n) = b - Ax_n - \alpha_nd_n = \\
  = r_n - \alpha_nAd_n = r_n^* - \alpha_n^*A^*d_n^* = r_{n+1}^* \\
  \beta_n = - \frac{r_{n+1}^TAd_n}{d_n^TAd_n} = \frac{r_{n+1}^{*T}r_n^* - \alpha_n^*r_{n+1}^{*T}A^*d_n^*}{d_n^{*T}A^*d_n^*}
  = \frac{r_{n+1}{*T}r_{n+1}^*}{r_n^{*T}r_n^*} = \beta_n^* \\
  d_{n+1} = r_{n+1} + \beta_nd_n = r_{n+1}^* + \beta_n^*d_n^* = d_{n+1}^*
\end{align*}
Und der Beweis ist vollständig. \\

Nun stellt sich natürlich die Frage, welcher der beiden Algorithmen zu bevorzugen wäre.
Nachdem sie mathematisch äquivalent sind, bleibt nur noch die Frage nach dem Aufwand zu überprüfen.
Am aufwändigsten ist natürlich die Matrix-Vektor-Multiplikation, wovon wir in unserem Algorithmus
im Gegensatz zu jenem aus dem Skript nur eine pro Schleifendurchlauf benötigen.
Zusätzlich dazu brauchen wir 3 Vektor-Vektor-Multiplikationen und 4 Vektor-Skalar-Operationen. \\
Damit erhalten wir insgesamt $n^2+7n$ Flops pro Durchlauf. \\
Im Vergleich dazu verwendet der Algorithmus aus dem Numerik-Skript pro Iteration zwei Matrix-Vektor-Multiplikationen
und ist daher aus Effizienzgründen unterlegen, da wir damit alleine schon $2n^2$ Flops pro Durchlauf brauchen. \\

Nachdem geklärt ist, in welcher Version der Algorithmus implementiert werden soll, ist es noch essentiell sich mit der
Konvergenztheorie dahinter zu beschäftigen.
Die Theorie besagt, dass das CG-Verfahren spätentens nach $n$ Durchläufen die exakte Lösung liefert und für die Iterierten
folgende Fehlerabschätzung gilt:
\begin{align*}
  \Vbraces{x^{(t)}-A^{-1}b}_A \leq 2\left(\frac{1-1/\sqrt{\kappa}}{1+1/\sqrt{\kappa}}\right)^t\Vbraces{x^{(0)}-A^{-1}b}_A, \quad t \in \N,
\end{align*}
mit der spektralen Konditionszahl $\kappa = \text{cond}_2(A)$. \\
Also sollte das Verfahren exponentiell kovergieren ($\Landau{AB^t}$) mit Konstanten\\
$A = 2\Vbraces{x^{(0)}-A^{-1}b}_A, \quad B = \frac{1-1/\sqrt{\kappa}}{1+1/\sqrt{\kappa}}$\\
Wir haben das Verfahren mit diagonaldominanten , symmetrisch, positiv definiten Zufallsmatrizen getestet. Eine weitere
Möglichkeit wäre gewesen, eine Zufallsmatrix $A$ zu erstellen und das CG-Verfahren auf $A\cdot A^T$ anzuwenden. Allerdings ist
im letzteren Fall die Konditionszahl deutlich schlechter und teilweise erreichen wir die gewünschte Toleranz erst nach über $n$
Schritten, wo wir in der Theorie ohne Rechenfehler bereits exakt sein sollten. Also haben wir uns für erstere Methode entschieden.
Wie in untenstehender Grafik ersichtlich scheint bei unseren Zufallsmatrizen der Wert von $B$ sich in etwa bei $0.92$ einzupendeln
und unsere vorgegebene Toleranz von $10^{-8}$ wird bereits nach etwa $350$ Iterationen erreicht, noch weit vor dem theoretisch
(bis auf Rechenfehler) garantierten exakten Resultat nach $n = 5000$ Durchläufen.
Ansonsten verhält sich der Fehler wie wir ihn erwarten würden. \\
Unsere Testwerte: \\

\begin{lstlisting}[language=Python]
n = 5000
A = np.random.rand(n,n)
A = np.dot(A,np.transpose(A))
A += 10*np.diag(abs(np.random.random(n)))
b = np.random.rand(n)
tol = 10**(-8)
x0 = np.random.rand(n)*10
\end{lstlisting}

\begin{figure}
    \centering
    \includegraphics[width=\linewidth]{Aufgabe_1/plot_b.png}
    \caption{Residuum abhängig von der Anzahl an Iterationen}
    \label{fig:my_label}
\end{figure}
\FloatBarrier
\subsection*{c)}
Um die Effizienz weiter zu steigern, müssen wir natürlich noch die dünne Besetztheit ausnutzen.
Dies bewerkstelligen wir mit einer eigens geschrieben Klasse in Python, welche die Matrix-Vektor-Multiplikation
deutlich effizienter ausführen sollte als die Standard-Multiplikation.
Dünn besetzte Matrizen erlauben effizientere Implementierungen als voll besetzte, indem beim Speichern und Rechnen
nur Einträge die ungleich Null sind berücksichtigt werden. Eine Möglichkeit einer solchen Implementierung ist das sogenannte
\textit{compressed sparse row} Format. Anstelle aller Einträge $A_{ij}, i,j = 1,\dots,n$ einer Matrix $A \in \mathbb{R}^{n\times n}$
werden ein Vektor $v \in \mathbb{R}^{n\times n}$ aller Einträge ungleich Null, ein Vektor $J \in \mathbb{N}_0^m$ von Spaltenindizes
und ein Vektor $I \in \mathbb{N}_0^{n+1}$ von Zeigern gespeichert. Die $i$-te Zeile von $A$ ist dann gegeben durch
\begin{align*}
  A_{ij} = \begin{cases}
    v_{k(j)}, & \text{falls}~ j \in \{J_{I_i}, J_{I_i} + 1, \dots, J_{I_i + 1} - 1\} \\
    0, & \text{sonst}
  \end{cases}
\end{align*}
\begin{lstlisting}[language=Python]
class Sparse:

    def __init__(self,b,v, J = np.zeros(0), I = np.zeros(0)):
        if b:
            self.v = np.array(v)
            self.J = np.array(J)
            self.I = np.array(I)
            self.n = len(self.I)-1
        else:
            self.v, self.J, self.I = self.fromdense(v)
            self.n = len(self.I)-1

    def __matmult__(self,b):
        d = np.zeros(self.n)
        for i in range(self.n):
            x = np.array(self.J[self.I[i]:self.I[i+1]]).astype(int)
            d[i] = self.v[self.I[i]:self.I[i+1]]@b[x]
        return d


    def todense(self):
        A = np.zeros([self.n,self.n])
        for i in range(self.n):
            for j in range(self.I[i],self.I[i+1]):
                A[i][self.J[j]] = self.v[j]
        return A

    def fromdense(self,A):
        v,J = np.zeros(0), np.zeros(0)
        I = np.array([0])
        c = 0
        for i in range(np.shape(A)[0]):
            for j in range((np.shape(A))[0]):
                if A[i][j] != 0:
                    v = np.append(v,A[i][j])
                    J = np.append(J,j)
                    c += 1
            I = np.append(I,c)
        return v, J, I
\end{lstlisting}


\begin{figure}
    \centering
    \includegraphics[width=0.8\linewidth]{Aufgabe_1/matmult_plot.png}
    \caption{Vergleich Numpy Matrixmultiplikation vs. Sparse Matrixmultiplikation}
    \label{mul}
\end{figure}
\FloatBarrier

In Abbildung \ref{mul} sehen wir, dass die Sparse-Matrix-Vektor-Multiplikation schneller ist, als die Implementierung in numpy.
Wir starten erst ab einer Matrixgröße $n=3000$, da der Messfehler bei einer kleineren Größe überwiegt und der Plot nicht aussagekräftig wäre.


\subsection*{d)}
Jetzt können wir unsere neue Klasse noch mit der bereits vorhandenen CG-Implementierung kombinieren. \\
\begin{lstlisting}[language=Python]
def Scg(A,b,x0,tol):
    xt = x0
    r0 = b - A.__matmult__(xt)
    d = r0
    while(np.linalg.norm(r0) > tol):
        prod = np.dot(np.transpose(r0),r0)
        prod2 = A.__matmult__(d)
        alpha = prod/np.dot(np.transpose(d),prod2)
        xt = xt + alpha*d
        r0 = r0 - alpha*prod2
        beta = np.dot(np.transpose(r0),r0)/prod
        d = r0 + beta*d
    return xt
\end{lstlisting}

Bei dieser CG-Implementierung verwenden wir anstelle der Numpy-Matrix-Vektor-Multiplikation die Implementierung von der Sparse-Klasse.
Zu beachten ist, dass die Matrix A ein Objekt der Klasse Sparse sein muss, damit die Funktion durchgeführt werden kann.
Ansonsten ist die Implementierung ident zum vorherigen cg-Verfahren.
\newpage
\begin{figure}
    \centering
    \includegraphics[width=0.9\linewidth]{Aufgabe_1/Cg_Scg.png}
    \caption{cg-Verfahren vs. cg-Verfahren mit Klasse Sparse}
    \label{scg}
\end{figure}
\FloatBarrier
Wie erhofft liefert die Kombinierung mit der Sparse-Klasse ab einer gewissen Größe noch einen zusätzlichen Effizienzschub.
Ab einer Matrixgröße von ca. 1000 ist das Scg-Verfahren effizienter als die übliche Implementierung.
Man sieht außerdem, dass beide Algorithmen eine Laufzeit von etwa $\Landau{n^2}$ (siehe Abb. \ref{scg}).

\begin{figure}
    \centering
    \includegraphics[width=0.8\linewidth]{Aufgabe_1/Scg_npsolve.png}
    \caption{Vergleich numpy.linalg.solve mit cg-Verfahren aus Klasse Sparse}
    \label{linalgvs}
\end{figure}
\FloatBarrier

In Abbildung \ref{linalgvs}  sieht man, dass bei kleiner Matrixgröße das Verfahren in numpy deutlich schneller als das cg-Verfahren ist, aber bei zunehmender Größe benötigt das cg-Verfahren weniger Zeit.
Das liegt daran, dass die numpy-Implementierung eine Laufzeit von $\Landau{n^3}$ hat, wohingegen das cg-Verfahren $\Landau{n^2}$ hat.

\subsection*{e)}
Der letzte Schritt zur Effizienz-Optimierung ist nun noch die Matrix selbst noch für unsere Zwecke zu verbessern.
Die Konvergenzgeschwindigkeit des CG-Verfahrens ist durch die spektrale Konditionszahl cond($A$) der Matrix $A$ bestimmt.
Um die Konvergenzgeschwindigkeit zu erhöhen löst man das vorkonditionierte System
\begin{align*}
  D^{-1}AD^{-1T} = D^{-1}b
\end{align*}
und gewinnt die Lösung $x$ dann durch $x = D^{-1T}y$. Die Matrix $D$ wird dabei so gewählt, dass
\begin{itemize}
\item für beliebige Vektoren $z \in \mathbb{R}^n$ der Vektor $D^{-1T}D^{-1}z$ einfach zu berechen ist und
\item $\text{cond}(D^{-1}AD^{-1T}) < \text{cond}(A)$.
\end{itemize}
Implementierung des vorkonditionierten CG-Verfahrens:
\begin{lstlisting}[language=Python]
    def vcg(A,b,x0,P,tol):
        r0 = b - A.__matmult__(x0)
        P_inv = np.linalg.inv(P)
        z0 = P_inv@r0
        d = z0
        while(np.linalg.norm(r0) > tol):
            prod = np.dot(np.transpose(z0),r0)
            prod2 = A.__matmult__(d)
            alpha = np.dot(np.transpose(r0),z0)/np.dot(np.transpose(d),prod2)
            x0 = x0 + alpha*d
            r0 = r0 - alpha*prod2
            z0 = P_inv@r0
            beta = np.dot(np.transpose(z0),r0)/prod
            d = z0 + beta*d
    return x0
\end{lstlisting}

\subsection*{f)}
Wenn wir schließlich das vorkonditionierte CG-Verfahren mit $P = \text{diag}(A_{11},\dots,A_{nn})$ an strikt diagonaldominanten
Zufallsmatrizen mit positiven Diagonaleinträgen und an beliebigen symmetrisch, positiv definiten Zufallsmatrizen testen, sehen
wir unseren finalen Effizienzschub in diesem Projekt. \\
Wie in untenstehenden Grafik ersichtlich konnte mit der Vorkonditionierung die Anzahl der benötigten Iterationen
zur Erreichung der erwünschten Toleranz massiv gesenkt werden. Der Zeitgewinn schlägt sich leider nicht ganz so deutlich wieder,
aber ist dennoch vorhanden und merkbar.
\begin{figure}
    \centering
    \includegraphics[width=\linewidth]{Aufgabe_1/f.png}
    \caption{Vergleich Anzahl Interationen}
    \label{fig:my_label}
\end{figure}
\begin{figure}
    \centering
    \includegraphics[width=\linewidth]{Aufgabe_1/vcg.png}
    \caption{Vergleich Zeit}
    \label{fig:my_label}
\end{figure}

\FloatBarrier
\section{Eigenschwingungen}
\subsection{Aufgabestellung}

Das Projekt beschäftigt sich mit den Eigenschwingeungen einer fest eingespannten Saite. Sei dazu $u(t, x)$ die vertikale Auslenkung der Saite an der Position $x \in [0, 1]$ zur Zeit $t$. $u$ wird näherungsweise durch die sogenannte Wellengleichung

\begin{align} \label{Wellengleichung}
  \frac{\partial^2 u}{\partial x^2} (t, x) =
  \frac{1}{c^2}
  \frac{\partial^2 u}{\partial t^2} (t, x)
\end{align}

für alle $x \in (0, 1)$ und $t \in \R$ beschrieben, wobei $c$ die Ausbreitungsgeschwindigkeit der Welle ist. Wenn die Saite an beiden Enden fest eingespannt ist, so gelten die Randbedingungen

\begin{align} \label{Randbedingungen}
  u(t, 0) = u(t, 1) = 0
\end{align}

für alle $t \in \R$. \\

Zur Berechnung der Eigenschwingungen suchen wir nach Lösungen $u$, die in der Zeit harmonisch schwingen. Solche erfüllen folgenden Ansatz

\begin{align*}
  u(x, t) = \mathfrak{R} (v(x) e^{-i \omega t})
\end{align*}

mit einer festen, aber unbekannten Kreisfrequenz $\omega > 0$ und einer Funktion $v$, welche nur noch vom Ort $x$ abhängt. Durch Einsetzen erhalten wir für $v$ die sogenannte Helmholz-Gleichung

\begin{align} \label{Helmholz-Gleichung}
  -v^\primeprime(x) = \kappa^2 v(x), \qquad
  x \in (0, 1),
\end{align}

mit der unbekannten Wellenzahl $\kappa := \frac{\omega}{c}$ und den Randbedingungen

\begin{align} \label{Helmholz-Randbedingungen}
  v(0) = v(1) = 0.
\end{align}

\subsection{Analytische Lösung}

\begin{align*}
  v_\kappa(x) = C_1 \cos{(\kappa x)} + C_2 \sin{(\kappa x)}, \qquad
  x \in [0, 1],
  \label{Analytische_Lösung}
\end{align*}

mit beliebigen Konstanten $C_1, C_2$ löst die Helmholz-Gleichung \eqref{Helmholz-Gleichung}. Das erkennt man durch stumpfes Einsetzen.

\begin{multline*}
  - v_\kappa^\primeprime(x)
  = - \frac{\partial^2}{\partial x^2}
    (C_1 \cos{(\kappa x)} + C_2 \sin{(\kappa x)})
  = - \frac{\partial}{\partial x}
    (- C_1 \kappa \sin{(\kappa x)} + C_2 \kappa \cos{(\kappa x)}) \\
  = - (- C_1 \kappa^2 \cos{(\kappa x)} - C_2 \kappa^2 \sin{(\kappa x)})
  = \kappa (C_1 \cos{(\kappa x)} + C_2 \sin{(\kappa x)})
  = \kappa^2 v_\kappa(x)
\end{multline*}

Wir fragen uns, für welche $\kappa > 0$, Konstanten $C_1$ und $C_2$ existieren, sodass $v_\kappa$ auch die Randbedingungen \eqref{Helmholz-Randbedingungen} erfüllt.

\begin{align*}
  0 \stackrel{!}{=}
  \begin{cases}
    v_\kappa(0)
    = C_1 \cos{0} + C_2 \sin{0}
    = C_1 \\
    v_\kappa(1)
    = C_1 \cos{\kappa} + C_2 \sin{\kappa}
    = C_2 \sin{\kappa}
  \end{cases}
\end{align*}

Nachdem $\cos{0} = 1$ und $\sin{0} = 0$, erhält man, aus der oberen Gleichung, $C_1 = 0$. Mit der unteren Gleichung folgt aber auch $C_2 \sin{\kappa} = 0$. Wenn nun auch $C_2 = 0$, dann erhielte man die triviale Lösung $v_\kappa = 0$. Für eine realistischere Modellierung, d.h. $v_\kappa \neq 0$, müsste $\sin{\kappa} = 0$, also $\kappa \in \pi \Z$. \\

Das sind die gesuchten $\kappa > 0$. Sei nun eines dieser $\kappa$ fest. Offensichtlich ist $C_1 = 0$ eindeutig, $C_2 \in \R$ jedoch beliebig.

\subsection{Numerische Approximation}

Häufig lassen sich solche Probleme nicht analytisch lösen, sodass auf numerische Verfahren zur+ckgegriffen wird, welche möglichst gute Näherungen an die exakten Lösungen berechnen sollen. Als einfachstes Mittel dienen sogenannte Differenzenverfahren. Sei dazu $x_j := jh$, $j = 0, \ldots, n$ eine Zerlegung des Intervalls $[0, 1]$ mit äquidistanter Schrittweite $h = 1/n$. Die zweite Ableitung in \eqref{Helmholz-Gleichung} wird approximiert durch den Differenzenquotienten

\begin{align} %\label{Differenzenquotient}
  v^\primeprime(x_j) \approx
  D_h v(x_j) :=
  \frac{1}{h^2} (v(x_{j-1}) - 2 v(x_j) + v(x_{j+1})), \qquad
  j = 1, \ldots, n-1.
\end{align}

Für hinreichend glatte Funktionen $v$ mit einer geeigneten Konstanten $C > 0$ wird der Approximationsfehler quadratisch in $h$ klein, d.h. dass

\begin{align} \label{quadratische_Konvergenz}
  \vbraces{v^\primeprime(x_j) - D_h v(x_j)} \leq C h^2.
\end{align}

Es sei zunächst bemerkt, dass \eqref{Differenzenquotient} tatsächlich einen Differenzenquotienten beschreibt. Um das einzusehen, verwenden wir den links- und rechts-seitigen Differenzenquotient erster Ordnung, sowie $x_{j-1} = x_j - h$, $x_{j+1} = x_j + h$. Wir erhalten $\Forall j = 1, \ldots, n-1:$

\begin{align*}
  v^\primeprime(x_j)
  & = \lim_{h \to 0}
      \Frac{h}
      {
        v^\prime(x_j + h) - v^\prime(x_j)
      } \\
  & = \lim_{h \to 0}
      \Frac{h}
      {
        \Frac{h}
        {
          v(x_j + h) - v(x_j)
        } -
        \Frac{h}
        {
          v(x_j) - v(x_j - h)
        }
      } \\
  & = \lim_{h \to 0}
      \Frac{h^2}
      {
        v(x_j + h) - 2 v(x_j) + v(x_j - h)
      } \\
  & = \lim_{h \to 0}
      D_h v(x_j)
\end{align*}

Nachdem $v$ hinreichend glatt ist, gilt nach dem Satz von Taylor, dass $\Forall j = 1, \ldots, n-1:$

\begin{align*}
  v(x_j + h) & =
  \sum_{\ell = 0}^{n+2}
  \frac{h^\ell}{\ell !}
  v^{(\ell)}(x_j) +
  \Landau{h^{n+3}}, \\
  v(x_j - h) & =
  \sum_{\ell = 0}^{n+2}
  \frac{(-h)^\ell}{\ell !}
  v^{(\ell)}(x_j) +
  \Landau{h^{n+3}}.
\end{align*}

Man beachte, dass sich die Summanden der oberen Taylor-Polynome für ungerade $\ell \in 2 \N - 1$ aufheben. Damit erhalten wir für den Differenzenquotient $D_h v(x_j)$, $j = 1, \ldots, n-1$ eine asymptotische Entwicklung.

\begin{align*}
  D_h v(x_j)
  & = \Frac{h^2}
  {
    v(x_j - h) + v(x_j + h) - 2 v(x_j)
  } \\
  & = \Frac{h^2}
      {
        2 v(x_j) +
        h^2 v^\primeprime(x_j) +
        \sum_{ell = 4}^{n+2}
        \frac{h^\ell}{\ell !}
        v^{(\ell)}(x_j)
        (1 + (-1)^\ell)
      } +
      \Landau{h^{n+3}} -
      2 v(x_j) \\
  & = v^\primeprime(x_j) +
      2 \sum_{\ell = 1}^\floor{\frac{n}{2}}
      \frac{h^{2 \ell}}{(2 \ell + 2)!}
      v^{(2 \ell)}(x_j) +
      \Landau{h^{n+1}}
\end{align*}

Daraus folgt unmittelbar die quadratische Konvergenz \eqref{quadratische_Konvergenz}, $\Forall j = 1, \ldots, n-1:$

\begin{align*}
  D_h v(x_j) - v^\primeprime(x_j) = \Landau{h^2}, \qquad
  h \to 0.
\end{align*}

Wir wollen nun den Differenzenquotienten $D_h v(x_j)$ verwenden, um ein Eigenwertproblem der Form $A \vec v = \lambda \vec v$ mit einer Matrix $A \in \R^{(n-1) \times (n-1)}$ zu dem Eigenvektor $\vec v := (v(x_1, \ldots, v(x_{n-1}))^T)$ und dem Eigenwert $\lambda := -\kappa^2$ herzuleiten. \\

Es wird eine Matrix $A_n$ gesucht, die den Differenzenquotienten $D_h v(x_j)$ auf den Vektor $\vec v$ komponentenweise anwendet. Wir rufen in Erinnerung, dass $h = 1/n$ und definieren die naheliegende Matrix

\begin{align*}
  A_n :=
  \frac{1}{h^2}
  \begin{pmatrix}
    -2 &  1      &        &    \\
     1 &  \ddots & \ddots &    \\
       &  \ddots & \ddots &  1 \\
       &         & 1      & -2
  \end{pmatrix}
  \in \R^{(n-1) \times (n-1)}.
\end{align*}

Weil nun die Randbedingungen \eqref{Helmholz-Randbedingungen} gelten, d.h. $v(x_0), v(x_n) = 0$, leistet diese Matrix $A_n$ tatsächlich das Gewünschte.

\begin{align*}
  A_n \vec v =
  \frac{1}{h^2}
  \begin{pmatrix}
    v(x_0) - 2 v(x_1) + v(x_2)             \\
    v(x_1) - 2 v(x_2) + v(x_3)             \\
    \vdots                                 \\
    v(x_{n-3}) - 2 v(x_{n-2}) + v(x_{n-1}) \\
    v(x_{n-2}) - 2 v(x_{n-1}) + v(x_{n-0})
  \end{pmatrix} =
  \begin{pmatrix}
    D_h v(x_1) \\
    \vdots     \\
    D_h v(x_{n-1})
  \end{pmatrix}
\end{align*}

Das Eigenwertproblem wurde mit \verb|np.linalg.eig|, für beliebige $n \geq 2$, gelöst. Wir vergleichen die Eigenwerte und Eigenvektoren mit den analytischen Ergebnissen. \\

Betrachtet man die, unten aufgelisteten, Eigenwerte, der ersten paar Matrizen $A_2, \ldots, A_{10}$, so legen diese ein gewisses (quadratisches) Konvergenzverhalten nahe. Die Matrix $A_n$ besitzt also scheinbar $n-1$ paarweise verschiedene Eigenwerte $\lambda_{1, n} < \cdots < \lambda_{n-1, n}$, welche jeweils gegen $\lambda_j := -(\pi j)^2$, $j \in \N$ konvergieren.

% \begin{align*}
%   \lambda_{j, n}
%   \xrightarrow{n \to \infty}
%   \lambda_j
% \end{align*}

\begin{multicols}{3}
\begin{verbatim}
n = 2
-----
-8.0

n = 3
-----
-9.0
-27.0

n = 4
-----
-9.372583002030478
-31.999999999999996
-54.62741699796946

n = 5
-----
-9.549150281252611
-34.54915028125264
-65.45084971874735
-90.45084971874735

n = 6
-----
-9.646170927520402
-35.99999999999999
-71.99999999999997
-108.00000000000001
-134.35382907247953
n = 7
-----
-9.705050945562961
-36.89799941784412
-76.19294847228122
-119.80705152771888
-159.10200058215582
-186.29494905443664

n = 8
-----
-9.743419838555344
-37.49033200812192
-79.01652065726852
-127.9999999999999
-176.98347934273144
-218.50966799187793
-246.25658016144442

n = 9
-----
-9.769795432682793
-37.90080021472559
-80.99999999999997
-133.8689952179573
-190.13100478204277
-243.00000000000014
-286.09919978527444
-314.2302045673173
n = 10
-----
-9.788696740969272
-38.19660112501045
-82.44294954150533
-138.1966011250105
-200.00000000000006
-261.80339887498934
-317.5570504584944
-361.8033988749895
-390.2113032590302

.
.
.

n -> inf
--------
-(1 * pi)^2 = -9.869604401089358
-(2 * pi)^2 = -39.47841760435743
-(3 * pi)^2 = -88.82643960980423
-(4 * pi)^2 = -157.91367041742973
-(5 * pi)^2 = -246.74011002723395
-(6 * pi)^2 = -355.3057584392169
-(7 * pi)^2 = -483.61061565337855
-(8 * pi)^2 = -631.6546816697189
-(9 * pi)^2 = -799.437956488238

...
\end{verbatim}
\end{multicols}

Wir bezeichnen mit $\epsilon_j(n) := |\lambda_j - \lambda_{j, n}|$, $j = 1, \ldots, n-1$ den absoluten Konvergenz-Fehler des $j$-ten Eigenwertes. In der folgenden Abbildung wurde dieser für $j = 1, 2, 3$ gegen $\id^2$, doppelt logarithmisch, geplottet. Allem Anschein nach, verschwindet $\epsilon_j$ quadratisch. Das korreliert mit dem Ergebnis \eqref{quadratische_Konvergenz}. \\

\begin{figure}[h!]
  \centering
  \includegraphics[width = 0.5 \textwidth]{Aufgabe_2/Konvergenz-Fehler_der_Eigenwerte_von_A_n}
  \caption{Konvergenz-Fehler der Eigenwerte von $A_n$}
  \label{Konvergenz-Fehler_EW}
\end{figure}

\FloatBarrier
\section{Cholesky-Zerlegung schwach besetzter Matrizen}
\subsection{Projektbeschreibung}

Ziel dieses Projekts ist es, das Integral von Funktionen $f : \Omega \to \R$ für ein gegebenes Gebiet $\Omega \subset \R^2$ zu approximieren. Eine mögliche Strategie besteht darin, das Gebiet in disjunkte einfache Teilgebiete $T_i, i \in I$ für eine Indexmenge $I$ zu zerlegen, sodass
\begin{align*}
\Omega = \sum_{i \in I}{T_i}.
\end{align*}

Nun konstruiert man Quadraturformeln für die einfacheren Teilgebiete $T_i$ und summiert über die gebietweisen Integrale. Eine häufige Wahl für $T_i$ sind Dreiecke, deshalb konstruieren wir zunächst Integrationsregeln für das Einheitsdreieck. Sei $\hat{Q} := (0, 1) \times (0, 1)$ und $\hat{T}$ das offene Dreieck mit den Eckpunkten $(0, 0), (1, 0), (0, 1)$. Sei weiters

\begin{align*}
    \Psi:
    \begin{cases}
        \R^2    \to     \R^2        \\
        (x, y)  \mapsto (x, (1-x)y).
    \end{cases}
\end{align*}

\subsection{Vorüberlegungen}

Zuerst stellen wir fest, dass die Abbildung $\Psi$ ein Diffeomorphismus von $\R^2$ nach $\R^2$ ist: Wir betrachten die Funktionalmatrix
$\mathrm{d}\Psi =
                \left(\begin{array}{cccc}                                
                1 & 0  \\                                               
                -y & 1-x  \\
                \end{array}
                \right) $
 und erhalten aus der Stetigkeit der partiellen Ableitungen $\Psi \in \mathrm{C^{1}}$.
 
 
 Andererseits ist $\Psi^{-1}: (x, y) \mapsto \left( x, \frac{y}{1-x} \right)$ die Inverse von $\Psi$. Aus $\mathrm{d}\Psi^{-1} =
    \left(\begin{array}{ccccc}                                
                1 & 0  \\                                               
                \frac{y}{1-x^2} & \frac{1}{1-x}  \\
                \end{array}
                \right)$ schließen wir analog $\Psi^{-1} \in \mathrm{C^{1}}$.
 
 
 Für ein beliebiges $(x, y) \in \hat{Q}$ gibt es positive Zahlen $\epsilon_{x}$ und $\epsilon_{y}$ mit $x = 1 - \epsilon_{x}$ und $y = 1 - \epsilon_{y}$, damit gilt $\Psi(x,y) = (1 - \epsilon_{x}, \epsilon_{x} - \epsilon_{x} \epsilon_{y})$. Ein Punkt im ersten Quadranten liegt genau dann in $\hat{T}$, wenn die Summe seiner Komponenten kleiner als 1 ist. Es gilt $(1 - \epsilon_{x})+(\epsilon_{x} - \epsilon_{x} \epsilon_{y}) = 1 - \epsilon_{x}\epsilon_{y} < 1$ und damit $\Psi(\hat{Q}) \subseteq \hat{T}$.
 
Sei umgekehrt $(x, y) \in \hat{T}$, dann gilt $x + y < 1$. Dieser Punkt hat unter $\Psi$ das Urbild $\left(x, \frac{y}{1-x}\right)$. Aus $\frac{y}{1-x} < 1 \Leftrightarrow x + y < 1$ folgt $\Psi(\hat{Q}) \supseteq \hat{T}$ und somit insgesamt $\Psi(\hat{Q}) = \hat{T}$.

Also ist $\Psi$ auch ein Diffeomorphismus von $\hat{Q}$ nach $\hat{T}$.
\newline
\newline
\newpage
\subsection{Quadraturformeln auf $\hat{Q}$ und $\hat{T}$}
Für $N, M \in \N$ seien $Q_{N} = \sum_{j=0}^{n(N)}\alpha_{j}f(x_{j})$ und $Q_{M} = \sum_{j=0}^{n(M)}\beta_{j}f(y_{j})$ zwei Quadraturformeln der Ordnung $N+1$ bzw. $M+1$ auf dem Einheitsintervall. Wir definieren daraus auf $\hat{Q}$ eine Quadraturformel durch \begin{align*}{Q_{\hat{Q}} := \sum_{i=0}^{n(N)}\sum_{j=0}^{n(M)}\alpha_{i}\beta_{j}f(x_{i},y_{j}).}
\end{align*}



$\Pi_{N,M}$ sei der Raum aller Linearkombinationen von Polynomen der Form $p_{ij}: (x,y) \mapsto x^i y^j$ mit $i \leq N, j \leq M$.
Durch die Quadraturformel $Q_{\hat{Q}}$ werden alle Polynome aus $\Pi_{N,M}$ exakt integriert, denn es gilt für beliebiges $r \in \Pi_{N,M}$

\begin{align*}
    Q_{\hat{Q}}(r) &= \sum_{i=0}^{n(N)}\alpha_{i}\sum_{j=0}^{n(M)}\beta_{j}r(x_{i},y_{j}) \stackrel{(1)}{=} \sum_{i=0}^{n(N)}\alpha_{i}\int_{0}^{1}r(x_{i},y)~dy \\
    &= \int_{0}^{1}\sum_{i=0}^{n(N)}\alpha_{i}r(x_{i},y)~dy \stackrel{(2)}{=} \int_{0}^{1}\int_{0}^{1}r(x,y)~dxdy,
\end{align*}
wobei (1) [(2)] gilt, da $r(x_{i}, y)$ [$r(x,y)$] als Funktion in Abhängigkeit von $y$ [$x$] ein Polynom vom Grad $\leq M$ [$\leq N$] ist und daher durch $Q_{M}$ [$Q_{N}$] exakt integriert wird.
\newline
\newline
Mithilfe von $Q_{\hat{Q}}$ können wir nun auch eine Quadraturformel auf dem Einheitsdreieck $\hat{T}$ konstruieren: Wir definieren
\begin{align*}Q_{\hat{T}}(f) := Q_{\hat{Q}}((f\circ\Psi)|{\mathrm{det}(\mathrm{d}\Psi)}|)
\end{align*} und erhalten für $f$ mit $(f\circ\Psi)|\mathrm{det}(\mathrm{d}\Psi)| \in \Pi_{N,M}$
unter Verwendung der bereits gezeigten Eigenschaften von $\Psi$ und der Transformationsformel

\begin{align*}
    Q_{\hat{T}}(f) = \int_{0}^{1}\int_{0}^{1}(f\circ\Psi)(x,y)|\det(\mathrm{d}\Psi
    \left(\begin{array}{rr}
    x \\
    y \\
    \end{array}\right)
    )|~dx~dy =\int_{\hat{Q}}(f\circ\Psi)|\mathrm{det}(\mathrm{d}\Psi)|~d\lambda^{2} = \int_{\Psi(\hat{Q}) = \hat{T}} f~d\lambda^{2}.
\end{align*}
Die Polynome f, die durch $Q_{\hat{T}}$ exakt integriert werden, sind genau jene, die sich als Linearkombination von $p_{ij}: (x,y) \mapsto x^i y^j$ mit $i+j+1\leq N, j \leq M$ darstellen lassen. Für diese Basispolynome gilt nämlich
\begin{align*}
    (p_{ij}\circ\Psi)
    \left(\begin{array}{rr}
    x \\
    y \\
    \end{array}\right)
    |\det(\mathrm{d}\Psi
    \left(\begin{array}{rr}
    x \\
    y \\
    \end{array}\right)
    )| &= x^{i}(y-xy)^{j}|1-x| \\
    &= x^{i}(y^{j}+\dots\pm x^{j}y^{j})(1-x) \\
    &= x^{i}y^{j}+\dots\pm x^{i+j}y^{j} - x^{i+1}y^{j}+\dots\pm x^{i+j+1}y^{j}.
\end{align*}
\newline

\subsubsection{Gaußquadraturen}
Um Quadraturformeln auf dem Einheitsquadrat bzw. auf dem Einheitsdreieck anzugeben, die aus dem Produkt von Gauß-Quadraturen entstehen, müssen zunächst die Gauß-Quadraturen zur Gewichtsfunktion $\omega \equiv 1$ auf dem Einheitsintervall bestimmt werden.

Die zugehörigen Quadraturknoten auf dem Intervall $[-1,1]$ lassen sich mithilfe der Orthogonalpolynome bestimmen, die durch die Rekursion
\begin{align*}
    L_{0}(x) = 1,~~~~L_{1}(x)=x,~~~~L_{n+1}(x) = xL_{n}(x)-\frac{n^{2}}{4n^{2}-1}L_{n-1}(x),~~~~n \in \N
\end{align*}
gegeben sind.

Gemäß Satz 4.23 des Skripts sind die Nullstellen des $n$-ten Orthogonalpolynoms $L_{n}$ (und somit die $n$ Quadraturknoten der Gauß-Quadratur $Q_{n-1}$) genau die Eigenwerte der Matrix\newline
$
\left(\begin{array}{cccccc}                                
                \beta_{0} & \gamma_{1} &&& \\               \gamma_{1} & \beta_{1} & \gamma_{2} && \\
                & \gamma_{2} & \ddots & \ddots && \\
                && \ddots & \ddots & \gamma_{n-1} \\
                &&& \gamma_{n-1} & \beta_{n-1}
                \end{array}
                \right)$, wobei in unserem Fall für alle $m \in \N_{0}$ gilt $\beta_{m}=0, \gamma_{m}=\sqrt{\frac{m^{2}}{4m^{2}-1}}.$
                
                
Die affin-lineare Abbildung 
$
    \Phi : [-1,1] \to [0,1] : \zeta \mapsto \frac{1}{2}(1+\zeta)
$
liefert die entsprechenden Quadraturknoten auf dem Einheitsintervall.

Die zugehörigen Gewichte $\alpha_{j}$ werden ebenso wie in Satz 4.23 berechnet, wobei für normierte Eigenvektoren aus $\int_{0}^{1}\omega(x) = 1$ folgt, dass
\begin{align*}
    \alpha_{j} = ((v_{j})_{1})^{2},~~j = 0,\dots,n.
\end{align*}

\lstset{language=Python}
\lstset{frame=lines}
\lstset{caption={Berechnung von Knoten und Gewichten für die Gauß-Quadratur}}
\lstset{label={lst:code_direct}}
\lstset{basicstyle=\footnotesize}
\begin{lstlisting}
def nodesnweights(n): 

    gamma = np.array([np.sqrt((i+1)**2/(4*(i+1)**2-1)) for i in range(n-1)])
    T = np.zeros((n,n)) + np.diag(gamma,1) + np.diag(gamma,-1)
    [vals,vecs] = np.linalg.eig(T)  #vecs sind bereits normiert
    
    vals = phi(vals)
    alpha = np.array([(vecs[0][i])**2 for i in range(n)]) 
    return [vals,alpha] 
\end{lstlisting}
\hspace{10pt}

Die aus dem Produkt von solchen Gauß-Quadraturen entstehenden Quadraturformeln $Q_{\hat{Q}}$ und $Q_{\hat{T}}$ kann man nun analog zur Beschreibung von oben implementieren:


\lstset{language=Python}
\lstset{frame=lines}
\lstset{caption={Implementierung von $\mathrm{Q_\hat{Q}}$}}
\lstset{label={lst:code_direct}}
\lstset{basicstyle=\footnotesize}
\begin{lstlisting}
def gaussQ(f,n):
    [x,a] = nodesnweights(n+1)
    sum = 0
    for i in range(n+1):
        for j in range(n+1):
          sum += a[i]*a[j]*f(x[i],x[j])
    return sum
\end{lstlisting}


\lstset{language=Python}
\lstset{frame=lines}
\lstset{caption={Implementierung von $\mathrm{Q_\hat{T}}$}}
\lstset{label={lst:code_direct}}
\lstset{basicstyle=\footnotesize}
\begin{lstlisting}
def gaussUT(f,n):
    def z(x,y):
        return f(x,(1-x)*y)*(1-x)
    return gaussQ(z,n)
\end{lstlisting}

\newline
\subsubsection{Quadraturen auf $\hat{T}$ mithilfe von Lagrange-Polynomen}
Alternativ dazu konstruieren wir uns Quadraturformeln $Q_{n}(f) := \int_{\hat{T}}p_{f}~d\lambda^{2}$ auf dem Einheitsdreieck, wobei wir das interpolierende Polynom mit den in Übungsaufgabe 22 berechneten Lagrange-Polynomen darstellen können. Wie im eindimensionalen Fall erhalten wir also eine Quadraturformel der Form $Q_{n}(f) =~\sum_{j=0}^{n}\alpha_{j}f(x_{j}, y_{j})$ mit Quadraturgewichten $\alpha_{j}=\int_{\hat{T}}L_{j}(x,y)~d\lambda^{2}$.

\lstset{language=Python}
\lstset{frame=lines}
\lstset{caption={Quadraturformeln erster und zweiter Ordnung auf $\mathrm{\hat{T}}$} mit Lagrange-Polynomen}
\lstset{label={lst:code_direct}}
\lstset{basicstyle=\footnotesize}
\begin{lstlisting}
def interp1UT(f):
    z = [f(0,0),f(1,0),f(0,1)]
    return z[0]*(1/6) + z[1]*(1/6) + z[2]*(1/6)

def interp2UT(f):
    z = [f(0,0),f(1,0),f(0,1),f(1/2,1/2),f(1/2,0),f(0,1/2)]
    return z[0]*0 + z[1]*0 + z[2]*0 + z[3]*(1/6) + z[4]*(1/6) + z[5]*(1/6)
\end{lstlisting}
\newline
\newline

\subsection{Quadraturformeln auf beliebigen Dreiecken}
Nun können wir die bisher konstruierten Quadraturformeln auf dem Einheitsdreieck auch für ein beliebiges Dreieck $T$ verallgemeinern. Für ein solches Dreieck (gegeben durch die 3 affin unabhängigen Eckpunkte $x,y$ und $z$) definieren wir uns die affine Abbildung
\begin{align*}
    \pi_{T}: \hat{T} \to T : a \mapsto A_{T}(a) + x, \text{~~~~wobei~} A_{T} = \left(\begin{array}{cc}                                
                z_{1}-x_{1} & y_{1}-x_{1}  \\
                z_{2}-x_{2} & y_{2}-x_{2}  \\
                \end{array}
                \right). 
\end{align*}
Man kann leicht nachprüfen, dass diese Abbildung die drei Eckpunkte des Einheitsdreiecks auf $x$, $y$ und $z$ abbildet und somit $\pi_{T}(\hat{T}) = T$ leistet. 

Da A regulär ist, ist $\pi$ ein Diffeomorphismus und es gilt mit der Transformationsformel
\begin{align*}
    \int_{T}f~d\lambda^{2} = \int_{\pi_{T}(\hat{T})}f~d\lambda^{2} = \int_{\hat{T}}(f\circ\pi_{T})|\det(\mathrm{d}\pi_{T})|~d\lambda^{2}.
\end{align*}
Wir definieren also für eine auf dem Einheitsdreieck gegebene Quadraturformel $Q_{\hat{T}}$ die neue Quadraturformel auf $T$ als
\begin{align*}
    Q_{T}(f) :=  Q_{\hat{T}}((f\circ\pi_{T})|\det(\mathrm{d}\pi_{T})|) = Q_{\hat{T}}((f\circ\pi_{T})|\mathrm{det}(A_{T})|).
\end{align*}

\lstset{language=Python}
\lstset{frame=lines}
\lstset{caption={Gauß-Quadratur auf beliebigem Dreieck T}}
\lstset{label={lst:code_direct}}
\lstset{basicstyle=\footnotesize}
\begin{lstlisting}
def gaussT(f,n,T):
    def w(x,y):
        A = np.array([ [T[2][0]-T[0][0], T[1][0]-T[0][0]],
                       [T[2][1]-T[0][1], T[1][1]-T[0][1]] ])
        l = A@np.array([x,y]) + np.array([T[0][0],T[0][1]])
        det = abs(A[0][0]*A[1][1] - A[0][1]*A[1][0])
        return f(l[0],l[1])*det
    return gaussUT(w,n)
\end{lstlisting}

\lstset{language=Python}
\lstset{frame=lines}
\lstset{caption={Quadratur mit Lagrange-Polynomen auf beliebigem Dreieck T}}
\lstset{label={lst:code_direct}}
\lstset{basicstyle=\footnotesize}
\newpage
\begin{lstlisting}
def interp1T(f,T):
    def w(x,y):
        A = np.array([ [T[2][0]-T[0][0], T[1][0]-T[0][0]],
                       [T[2][1]-T[0][1], T[1][1]-T[0][1]] ])
        l = A@np.array([x,y]) + np.array([T[0][0],T[0][1]])
        det = abs(A[0][0]*A[1][1] - A[0][1]*A[1][0])
        return f(l[0],l[1])*det
    return interp1UT(w)


def interp2T(f,T):
    def w(x,y):
        A = np.array([ [T[2][0]-T[0][0], T[1][0]-T[0][0]],
                       [T[2][1]-T[0][1], T[1][1]-T[0][1]] ])
        l = A@np.array([x,y]) + np.array([T[0][0],T[0][1]])
        det = abs(A[0][0]*A[1][1] - A[0][1]*A[1][0])
        return f(l[0],l[1])*det
    return interp2UT(w)
\end{lstlisting}
\subsection{Quadraturformeln auf einem beliebigen Gebiet mithilfe von Triangulierung}
Jetzt sind wir fast am Ziel angelangt; wir betrachten nun ein Gebiet $\Omega \subset \R^2$ und eine Funktion $f : \Omega \to \R$. Unsere bisherige Arbeit liefert uns die Werkzeuge, um das Integral $\int_{\Omega}f~d\lambda^{2}$ näherungsweise zu berechnen. Hiezu betrachten wir disjunkte Dreiecke $T_i, i \in I$ mit $\sum_{i \in I}{T_i} \subseteq \Omega$. Es stellt sich die Frage, wie gut die Approximation
\begin{align*}
    \int_{\Omega}f~d\lambda^{2} \approx \sum_{i \in I}{Q_{T_{i}}(f)}
\end{align*}
in Abhängigkeit von der Feinheit der Triangulierung ist.\newline

Zuerst betrachten wir das Integral der Funktion $f: \R^2 \rightarrow \R, (x, y) \mapsto \mathrm{exp}(\frac{x+y}{x-y})$ über dem Viereck mit den Eckpunkten $(0, -1), (0, -2), (2, 0), (1, 0)$, das wir mithilfe der Transformationsformel exakt berechnen können. Wir verwenden das Python-Package $\mathrm{dmsh}$, um Triangulierungen zu generieren, wobei wir jeweils Dreiecke mit den Kantenlängen $h = (\frac{2}{3})^i, i = 1, ..., 10$ wählen. Dabei integrieren wir auf zwei Arten: Mit der aus dem Produkt von Gauß-Quadraturen konstruierten Formel $Q_{T}$ mit Ordnung 4 und mit der mithilfe von Lagrange-Interpolation konstruierten Quadraturformel 2. Ordnung. In diesem Test erweist sich letztere als mindestens ebenbürtig und liefert für $i \ge 7$ sogar einen besseren Schätzwert für das Integral. Mithilfe einer Vergleichsgerade können wir auf einem doppelt logarithmischen Plot ablesen, dass beide Formeln ungefähr Konvergenzordnung 4 haben (Abb. 1).
\newline \newline
Als nächstes testen wir unsere Formeln, indem wir die Funktion $g: \R^2 \rightarrow \R, (x, y) \mapsto \mathrm{sin}(x^2+y^2)$ über den Einheitskreis integrieren. Auch hier lässt sich der exakte Wert sofort über die Transformationsformel bestimmen. Aus dem Konvergenzplot (Abb. 2) schließen wir auf Konvergenzordnung 2. Der Fehler im ersten Beispiel war vergleichsweise klein und kann auf die Fehler der Quadraturformeln zurückgeführt werden; hier hingegen wird die Abhängigkeit von der Triangulierung deutlich, die ein rundes Gebiet nicht vollständig überdecken kann. Der dadurch entstandene Fehler ist so groß, dass er die Ungenauigkeit der Quadraturformeln verdeckt und kein Unterschied zwischen den beiden mehr ausgemacht werden kann.

\begin{center}
\includegraphics[width=120mm]{Aufgabe_3/Abb1.png}
\end{center}

\begin{center}
\includegraphics[width=120mm]{Aufgabe_3/Abb2.png}
\end{center}

\end{document}
