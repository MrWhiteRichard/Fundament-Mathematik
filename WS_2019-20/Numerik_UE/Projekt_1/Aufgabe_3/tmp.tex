Zuerst stellen wir fest, dass die Abbildung $\Psi$ ein Diffeomorphismus von $\hat{Q}$ nach $\hat{T}$ ist: Wegen \newline
$\mathrm{det}(\mathrm{d}\Psi) =
    \mathrm{det} \left(\begin{array}{rrrr}                                
                1 & 0  \\                                               
                -y & 1-x  \\
                \end{array}
                \right) 
= 1-x \ne 0$ ist $\Psi$ nach dem Umkehrsatz ein lokaler Diffeomorphismus. Für ein beliebiges $(x, y) \in \hat{Q}$ gibt es positive Zahlen $\epsilon_{x}$ und $\epsilon_{y}$ mit $x = 1 - \epsilon_{x}$ und $y = 1 - \epsilon_{y}$, damit gilt $\Psi(x,y) = (1 - \epsilon_{x}, \epsilon_{x} - \epsilon_{x} \epsilon_{y})$. Ein Punkt liegt genau dann in $\hat{T}$, wenn die Summe seiner Komponenten kleiner als 1 ist. Es gilt $(1 - \epsilon_{x})+(1-(1 - \epsilon_{x}))(\epsilon_{x} - \epsilon_{x} \epsilon_{y}) = 1 - \epsilon_{x}\epsilon_{y} < 1$ und damit $\Psi(x,y) \in \hat{T}$.

Andererseits ist $\mathrm{det}(\mathrm{d}\Psi) =
    \mathrm{det} \left(\begin{array}{rrrr}                                
                1 & 0  \\                                               
                -y & 1-x  \\
                \end{array}
                \right) 