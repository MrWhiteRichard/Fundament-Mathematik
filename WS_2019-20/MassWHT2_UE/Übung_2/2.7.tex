% --------------------------------------------------------------------------------

\begin{exercise}

$\nu_n$, $n \in \N$ seien sigmaendliche Maße auf dem sigmaendlichen Maßraum $(\Omega, \mathfrak S, \mu)$, $\nu = \sum \nu_n$, $\nu_{n \mathrm c} \ll \mu$, $\nu_{n \mathrm{s}} \Bot \mu$ sei die Lebesgue-Zerlegung vun $\nu$ bezüglich $\mu$.
Zeigen Sie, dass $\nu_\mathrm{s} = \sum_n \nu_{n \mathrm{s}}$ und $\nu_\mathrm{c} = \sum_n \nu_{n \mathrm c}$ eine Lebesgue-Zerlegung von $\nu$ ist.
Ist $\nu$ auch sigmaendlich, dann ist

\begin{align*}
    \derivative[][\nu_\mathrm{c}]{\mu}
    =
    \sum_n
        \derivative[][\nu_{n \mathrm c}]{\mu}
\end{align*}

(ist $\nu$ nicht sigmaendlich, dann ist die rechte Seite immer noch \textit{eine} Radon-Nikodym Dichte, aber diese muss nicht eindeutig bestimmt sein.)

\end{exercise}

% --------------------------------------------------------------------------------

\begin{solution}

\phantom{}

\begin{enumerate}[label = \arabic*.]

    \item Schritt:
    
    \begin{align*}
        \nu
        =
        \sum_n \nu_n
        =
        \sum_n (\nu_{n \mathrm{s}} + \nu_{n \mathrm c})
        =
        \sum_n \nu_{n \mathrm{s}} + \sum_n \nu_{n \mathrm c}
        =
        \nu_\mathrm{s} + \nu_\mathrm{c}
    \end{align*}
    
    \item Schritt ($\nu_\mathrm{c} \ll \mu$):
    
    Sei $\Forall A \in \mathfrak S$, mit $\mu(A) = 0$.
    Weil $(\nu_{n \mathrm{s}}) \ll \mu$, gilt

    \begin{align*}
        \Forall n \in \N:
            \nu_{n \mathrm c} = 0.
    \end{align*}

    Nun ist $\nu_\mathrm{c} \ll \mu$, weil

    \begin{align*}
        \nu_\mathrm{c}(A) = \sum_n \underbrace{\nu_{n \mathrm c}}_0 = 0.
    \end{align*}

    \item Schritt ($\nu_\mathrm{s} \Bot \mu$):
    
    Weil $(\nu_{n \mathrm{s}}) \Bot \mu$, gilt
    
    \begin{align*}
        \Forall n \in \N:
            \Exists A_n \in \mathfrak S:
                \nu_{n \mathrm{s}}(A_n) = \mu(A_n^C) = 0.
    \end{align*}

    Sei $A := \bigcap_n A_n$.
    Nun ist $\nu_\mathrm{s} \Bot \mu$, weil

    \begin{align*}
        \nu_\mathrm{s}(A) & = \sum_n \nu_{n \mathrm{s}} \pbraces{\bigcap_k A_k} \leq \sum_n \nu_{n \mathrm{s}}(A_n) = 0, \\
        \mu(A^C) & = \sum_n \mu \pbraces{\bigcup_n A_k^C} \leq \sum_k \mu(A_k^C) = 0
    \end{align*}

    \item Schritt:
    
    Laut Satz 7.1: Radon-Nikodym, existieren alle Dichten.
    Sei $A \in \mathfrak S$.
    Laut Satz 5.3: monotone Konvergenz, Beppo Levi-Theorem, gilt

    \begin{multline*}
        \Int[A]
        {
            \lim_{N \to \infty}
                \sum_{n=1}^N
                    \derivative[][\nu_{n \mathrm c}]{\mu}
        }{\mu}
        \stackrel
        {
            \text{MK}
        }{=}
        \lim_{N \to \infty}
        \Int[A]
        {
            \sum_{n=1}^N
                \derivative[][\nu_{n \mathrm c}]{\mu}
        }{\mu}
        =
        \lim_{N \to \infty}
            \sum_{n=1}^N
                \Int[A]
                {
                    \derivative[][\nu_{n \mathrm c}]{\mu}
                }{\mu} \\
        =
        \lim_{N \to \infty}
            \sum_{n=1}^N
                \nu_{n \mathrm c}(A)
        =
        \nu_\mathrm{c}(A)
        =
        \Int[A]
        {
            \derivative[][\nu_\mathrm{c}]{\mu}
        }{\mu}.
    \end{multline*}

\end{enumerate}

\end{solution}

% --------------------------------------------------------------------------------
