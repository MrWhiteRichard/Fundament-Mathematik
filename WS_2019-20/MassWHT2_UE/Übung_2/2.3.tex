% --------------------------------------------------------------------------------

\begin{exercise}

Geben Sie ein Beispiel für zwei Maßfunktionen $\mu$ und $\nu$ mit $\mu$ endlich, $\nu$ sigmaendlich aber nicht endlich, und $\nu \ll \mu$, und die $\epsilon$-$\delta$-Bedingung gilt, oder beweisen Sie, dass es kein solches Beispiel gibt (das vorige Beispiel wird gelegentlich so formuliert, dass \Quote{nicht einmal dann, wenn $\mu$ endlich und $\nu$ sigmaendlich ist}, die $\epsilon$-$\delta$-Bedingung gelten muss; diese Formulierung ist nicht ganz gerechtfertigt, weil es ja umso schwieriger ist, die Bedingung zu erfüllen, je kleiner $\mu$ ist).

\end{exercise}

% --------------------------------------------------------------------------------

\begin{solution}

Wir zeigen, dass es kein solches Beispiel gibt, d.h.

\begin{align*}
    &
    \mu, \nu ~\text{Maßfunktionen}, \quad
    \mu ~\text{endlich}, \quad
    \nu ~\text{sigmaendlich, nicht endlich}, \quad
    \nu \ll \mu \\
    \implies &
    \Exists \epsilon > 0:
        \Forall \delta > 0:
            \Exists A \in \mathfrak S:
                \mu(A) < \delta, \nu(A) \geq \epsilon.
\end{align*}

Dass $\mu$ endlich ist, heißt

\begin{align*}
    \Forall A \in \mathfrak S:
        \mu(A) < \infty.
\end{align*}

Dass $\nu$ sigmaendlich, aber nicht endlich ist, heißt

\begin{align*}
    \Exists \vec A = (A_n) \subset \mathfrak S:
        \sum \vec A = \Omega,
        \Forall n \in \N:
            \nu(A_n) < \infty,
    \quad
    \Exists A \in \mathfrak S:
        \nu(A) = \infty,
    \quad
    \text{bzw.}
    \quad
    \nu(\Omega) = \infty.
\end{align*}

\begin{align*}
    N := \Bbraces{n \in \N: \nu(A_n) = 0},
    \quad
    B_0 := \sum_{n \in N} A_n,
    \quad
    B_n := A_n,
    \quad
    n \in M := \N \setminus N
\end{align*}

Nun ist $M$ abzählbar, weil

\begin{align*}
    \infty
    =
    \nu(\Omega)
    =
    \underbrace{\nu(B_0)}_0 + \nu \pbraces{\sum_{n \in M} B_n}
    =
    \sum_{n \in M} \underbrace{\nu(B_n)}_{< \infty}.
\end{align*}

Wir finden also eine Bijektion $f: \N \to M$.

\begin{align*}
    C_0 := B_0,
    \quad
    C_n := B_{f(n)},
    \quad
    n \in \N
\end{align*}

Wir haben also eine weiteren Zeugen der Sigmaendlichkeit von $\nu$ gefunden.

\begin{align*}
    \implies &
    \sum_{n \in \N_0} C_n
    =
    B_0 + \sum_{n \in M} B_n
    =
    \sum_{n \in N} A_n + \sum_{n \in M} A_n
    =
    \sum \vec A
    =
    \Omega, \\
    &
    \Forall n \in \N:
        (
            f(n) \in M
            \implies
            0 < \nu(A_{f(n)}) = \nu(B_{f(n)}) = \nu(C_n) < \infty
        )
\end{align*}

\begin{align*}
    \implies &
    \sum_{n \in \N_0}
        \mu(C_n)
    =
    \mu \pbraces{\sum_{n \in \N_0} C_n}
    =
    \mu(\Omega)
    <
    \infty \\
    \implies &
    \Forall \delta > 0:
        \Exists m \in \N:
            \sum_{n=m}^\infty
                \mu(C_n)
            <
            \delta
\end{align*}

\begin{align*}
    \implies
    \Forall \epsilon > 0:
        \mu \pbraces{\sum_{n=m}^\infty C_n} & = \sum_{n=m}^\infty \mu(C_n) < \delta, \\
        \nu \pbraces{\sum_{n=m}^\infty C_n} & = \sum_{n=m}^\infty \nu(C_n) = \nu(\Omega) - \underbrace{\sum_{n=0}^{m-1} \nu(C_n)}_{< \infty} = \infty \geq \epsilon
\end{align*}

\end{solution}

% --------------------------------------------------------------------------------
    