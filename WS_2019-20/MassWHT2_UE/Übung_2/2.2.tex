% --------------------------------------------------------------------------------

\begin{exercise}

Die $\epsilon$-$\delta$-Bedingung (zu jedem $\epsilon > 0$ gibt es ein $\delta > 0$, sodass aus $\mu(A) < \delta$ die Ungleichung $\nu(A) < \epsilon$ folgt) für die absolute Stetigkeit muss nicht gelten, wenn $\nu$ nicht endlich ist:
geben Sie ein Beispiel für zwei maßfunktionen $\mu$ und $\nu$ mit $\mu$ endlich, $\nu$ sigmaendlich, und $\nu \ll \mu$, aber die $\epsilon$-$\delta$-Bedingung gilt nicht.

\end{exercise}

% --------------------------------------------------------------------------------

\begin{solution}

\phantom{}

\includegraphicsboxed{MassWHT1&2/MassWHT1&2 - Satz 7.3 - epsilon-delta-Kriterium.png}

Betrachte dem Messraum $(\N, 2^\N)$.

\begin{align*}
    \mu:
        \begin{Bmatrix}
            \emptyset \mapsto 0 \\
            \Bbraces{n} \mapsto 1/2^n
        \end{Bmatrix},
    \quad
    \text{bzw.}
    \quad
    \mu(A) := \sum_{n \in A} 1 / 2^n,
    \quad
    A \in 2^\N
\end{align*}

Sei $\nu$ das Zählmaß.
Offenbar ist $\nu$ nicht endlich, weil $\nu(\N) = |\N| = \aleph_0 = \infty$.
$\mu$ ist endlich, weil

\begin{align*}
    \Forall A \in 2^\N:
        \mu(A) \leq \mu(\N) = \sum_{n \in \N} 1 / 2^n = 2 < \infty.
\end{align*}

$\nu \ll \mu$, weil

\begin{align*}
    \Forall A \in 2^\N:
        \pbraces
        {
            \sum_{n \in A} 1/2^n = \mu(A) = 0
            \implies
            A = \emptyset
            \implies
            \nu(A) = |A| = 0
        }.
\end{align*}

Die $\epsilon$-$\delta$-Bedingung ist nicht erfüllt, d.h.

\begin{align*}
    \Exists \epsilon > 0:
        \Forall \delta > 0:
            \Exists A \in 2^\N:
                \mu(A) < \delta,
                \quad
                \nu(A) > \epsilon.
\end{align*}

Zeugen dafür sind $\epsilon \in (0, 1)$, ($\delta > 0$), und $A := \Bbraces{\log_2 \alpha}$ mit $\alpha > 1 / \delta$, weil dann

\begin{align*}
    \mu(A) = 1 / 2^{\log_2 \alpha} = 1 / \alpha < \delta,
    \quad
    \nu(A) = |\Bbraces{1 / 2^{\log_2 \alpha}}| = 1 \not < \epsilon.
\end{align*}

\end{solution}

% --------------------------------------------------------------------------------
