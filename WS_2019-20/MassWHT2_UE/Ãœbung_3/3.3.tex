% -------------------------------------------------------------------------------- %

\begin{exercise}

$(q_n, n \in \N)$ sei eine Abzählung der rationalen Zahlen in $]0, 1[$, $\epsilon > 0$.
Die Menge

\begin{align*}
    A = (\bigcup_n ]q_n - \frac{\epsilon}{2^n}, q_n + \frac{\epsilon}{2^n}[) \cap ]0, 1[
\end{align*}

ist offen und daher als disjunkte Vereinigung von offenen Intervallen darstellbar:

\begin{align*}
    A = \sum_n ]a_n, b_n[.
\end{align*}

jedes offene intervall hat mit $A$ nichtleeren Schnitt, insbesondere gibt es in jeder Umgebung eines Punktes $x \in ]0, 1[ \setminus A$ einen $a_n$ oder $b_n$.
Die Funktion

\begin{align*}
    F = \sum_n h_{a_n, b_n}
\end{align*}

ist dann überall differenzierbar, und ihre Ableitung $F^\prime$ ist beschränkt, auf $A$ stetig und auf $A^C$ unstetig.
Insbesondere hat die Menge der Unstetigkeitsstellen von $F^\prime$ positives Lebesguemaß, und daher ist $F^\prime$ nicht Riemann-integrierbar.

\end{exercise}

% -------------------------------------------------------------------------------- %

\begin{comment}

    \begin{solution}

    \phantom{}

    \begin{enumerate}

        \item $A$ ist, als Vereinigung offener Intervalle (und endlichem Schnitt), offen.
        
        \item Der Schnitt der $q_n$-Intervalle mit $]0, 1[$ ist auch offen.
        
        \item Sei $\emptyset \neq ]a, b[ \subset ]0, 1[$.
        Laut Konstruktion gilt

        \begin{align*}
            \Forall n \in \N:
                q_n \in A.
        \end{align*}

        $\Q$ ist dicht in $\R$.

        \begin{align*}
            \implies &
            \Exists n \in \N:
                q_n \in ]a, b[ \\
            \implies &
            ]a, b[ \cap A \supset \Bbraces{q_n} \neq \emptyset
        \end{align*}

        \item Sei $x \in ]0, 1[ \setminus A$ und $\epsilon > 0$.
        Nun gilt ja $]x - \epsilon, x + \epsilon[ \cap A \neq \emptyset$.
        Also muss $a_n \in B_\epsilon(x)$ oder $b_n \in B_\epsilon(x)$.

        \item Sei $x \in ]0, 1[$.
        
        \begin{multline*}
            h_{a, b}(x)
            =
            h_{0, b-a}(x - a)
            =
            \begin{cases}
                g(x - a),                    & 0 \leq x - a \leq \frac{1}{n \pi},                 \\
                g \pbraces{\frac{1}{n \pi}}, & \frac{1}{n \pi} < x - a < b - a - \frac{1}{n \pi}, \\
                g(b - a - x + a),            & b - a - \frac{1}{n \pi} \leq x - a \leq b - a,     \\
                0,                           & \text{sonst}
            \end{cases} \\
            =
            \begin{cases}
                g(x - a),                    & a \leq x \leq \frac{1}{n \pi} + a,             \\
                g \pbraces{\frac{1}{n \pi}}, & \frac{1}{n \pi} + a < x < b - \frac{1}{n \pi}, \\
                g(b - x),                    & b - \frac{1}{n \pi} \leq x \leq b,             \\
                0,                           & \text{sonst}
            \end{cases}
        \end{multline*}

        \begin{enumerate}[label = \arabic*.]

            \item Fall ($x \in A$):
            
            \begin{align*}
                \implies &
                \Exists n \in \N:
                    x \in ]a_n, b_n[ \\
                \implies &
                F^\prime(x) = h_{a_n, b_n}^\prime(x)
            \end{align*}

            \item Fall ($x \in A^C$):

            \begin{enumerate}[label = \arabic*.]

                \item Fall ($\Exists n \in \N: x \in \Bbraces{a_n, b_n}$):
                
                Laut der vorigen Aufgabe, verschwinden die linksseitigen und rechtsseitigen Differentialquotienten von $h$ am rechten bzw. linken Randpunkt.

                \begin{align*}
                    \implies
                    &
                    F^\prime(a_n)_+
                    =
                    \lim_{\epsilon \to 0+}
                        \frac{F(a_n + \epsilon) - F(a_n)}{\epsilon}
                    =
                    \lim_{\epsilon \to 0+}
                        \frac{h_{a_n, b_n}(a_n + \epsilon) - h_{a_n, b_n}(a_n)}{\epsilon}
                    =
                    0, \\
                    &
                    F^\prime(b_n)_-
                    =
                    \lim_{\epsilon \to 0-}
                        \frac{F(b_n + \epsilon) - F(b_n)}{\epsilon}
                    =
                    \lim_{\epsilon \to 0-}
                        \frac{h_{b_n, b_n}(b_n + \epsilon) - h_{b_n, b_n}(b_n)}{\epsilon}
                    =
                    0
                \end{align*}

                \item Fall ($\nExists n \in \N: x \in \Bbraces{a_n, b_n}$):

            \end{enumerate}

        \end{enumerate}

    \end{enumerate}

    \end{solution}

\end{comment}

% -------------------------------------------------------------------------------- %
