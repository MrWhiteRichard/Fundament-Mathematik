% --------------------------------------------------------------------------------

\begin{exercise}

Die Funktion

\begin{align*}
    F(x) = x^2 \sin(1 / x^2)
\end{align*}

mit $F(0) = 0$ ist im Intervall $[-1, 1]$ überall differenzierbar, aber $F^\prime$ ist nicht Lebesgue-integrierbar (und auch nicht eigentlich Riemann-integrierbar, weil unbeschränkt, aber immerhin noch uneigentlich Riemann-integrierbar).
Anmerkung:
Das zeigt uns Grenzen unserer Hauptsatz-Variationen.
Wir haben schöne Ergebnisse, wenn wir mit integrierbaren Funktionen (im Legesgueschen oder Riemannschen Sinn) beginnen:
dann ist das Integral (als Funktion der Obergrenze) fast überall differenzierbar, und die Ableitung stimmt fast überall mit der Ausgangsfunktion überein.
In der anderen Richtung sind wir weniger glücklich:
wenn eine Funktion überall differenzierbar ist, muss ihre Ableitung weder im (uneigentlich) Riemannschen noch im Lebesguschen Sinn integrierbar sein.
Wenn wir nur Differenzierbarkeit fast überall fordern, dann gibt es sogar ein negatives Resultat:
da jede singuläre monotone Funktion fast überall Ableitung $0$ hat, ist es hoffnungslos, aus dieser die ursprüngliche Funktion zurückbekommen zu wollen.
Es bleibt die Frage, ob es für überall differenzierbare Funktionen ein Integral gibt, das die ursprungliche Funktion rekonstruiert, und erstaunlicherweise gibt es das mit dem Henstock-Kurzweil-Integral (auch als verallgemeinertes Riemann-Integral bekannt).

\end{exercise}

% --------------------------------------------------------------------------------

\begin{solution}

\phantom{}

\begin{enumerate}[label = \arabic*.]

    \item Teil ($F$ überall differenzierbar):
    
    \begin{align*}
        \Forall x \in [-1, 1] \setminus \Bbraces{0}:
            F^\prime(x)
            =
            2 x \sin(1 / x^2) + x^2 (-2) \frac{1}{x^3} \cos(1 / x^2)
            =
            2 (x \sin(1 / x^2) - \frac{1}{x} \cos(1/x^2))
    \end{align*}
    
    Die einzige interessante Stelle ist $x = 0$.

    \begin{align*}
        \vbraces
        {
            \frac{F(0 + \epsilon) - F(0)}{\epsilon}
        }
        =
        \vbraces
        {
            \frac{F(\epsilon)}{\epsilon}
        }
        =
        \vbraces
        {
            \frac{\epsilon^2 \sin(1 / \epsilon^2)}{\epsilon}
        }
        \leq
        |\epsilon|
        \xrightarrow{\epsilon \to 0}
        0
    \end{align*}

    \item Teil ($F^\prime$ nicht Lebesgue-integrierbar):
    
    Wir bestimmen

    \begin{align*}
        [F^\prime \geq 0]
        =
        \Bbraces{x \in [-1, 1]: 2 (x \sin(1 / x^2) - \frac{1}{x} \cos(1 / x^2)) \geq 0}
        \supset
        \Bbraces{x \in ]0, 1]: \sin(1 / x^2) \geq 0 \geq \cos(1 / x^2)}
    \end{align*}

    Sei $x \in ]0, 1]$.
    Durch malen der $\sin$- und $\cos$-Plots und markieren der richtigen Stellen, kann man sich leicht überlegen, dass

    \begin{align*}
        \sin(1 / x^2) \geq 0
        \iff &
        1 / x^2 \in \sum_{n \in \Z} [2 n \pi, (2 n + 1) \pi] \\
        \iff &
        x \in \sum_{n \in \Z} [a_n, b_n] \subset ]0, 1], \\
        \cos(1 / x^2) \leq 0
        \iff &
        1 / x^2 \in \sum_{n \in \Z} \bbraces{\frac{(4 n + 1) \pi}{2}, \frac{(4 n + 3) \pi}{2}} \\
        \iff &
        x \in \sum_{n \in \N} [c_n, d_n],
    \end{align*}

    wobei

    \begin{align*}
        a_n := \sqrt{\frac{1}{(2 n + 1) \pi}}
        \leq
        c_n := \sqrt{\frac{2}{(4 n + 3) \pi}}
        \leq
        b_n := \sqrt{\frac{1}{2 n \pi}}
        \leq
        d_n := \sqrt{\frac{2}{(4 n + 1) \pi}},
        \quad
        n \in \N.
    \end{align*}

    \begin{align*}
        \implies
        [F^\prime \geq 0]
        \supset
        \sum_{n \in \N}
            [a_n, b_n]
        \cap
        \sum_{n \in \N}
            [c_n, d_n]
        =
        \sum_{n \in \N}
            [a_n, b_n]
            \cap
            [c_n, d_n]
        \supset
        \sum_{n \in \N}
            [c_n, b_n]
    \end{align*}

    \begin{align*}
        \implies
        \Int{(F^\prime)^+}{\lambda}
        & =
        \Int[\bbraces{F^\prime \geq 0}]{F^\prime}{\lambda} \\
        & \geq
        \Int
        {
            F^\prime
            \sum_{n \in \N}
                [c_n, b_n]
        }{\lambda} \\
        & =
        \sum_{n \in \N}
        \Int[c_n][b_n]
        {
            F^\prime
        }{\lambda} \\
        & =
        \sum_{n \in \N}
            F(b_n) - F(c_n) \\
        & =
        \sum_{n \in \N}
            \frac{2}{(4 n + 1) \pi}
            \underbrace
            {
                \sin \pbraces
                {
                    \frac{(4 n + 1) \pi}{2}
                }
            }_1
            -
            \frac{1}{(2 n + 1) \pi}
            \underbrace
            {
                \sin((2 n + 1) \pi)
            }_0 \\
        & =
        \infty
    \end{align*}

\end{enumerate}

\end{solution}

% --------------------------------------------------------------------------------
