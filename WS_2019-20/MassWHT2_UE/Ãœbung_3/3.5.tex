% -------------------------------------------------------------------------------- %

\begin{exercise}

Zeigen Sie:
wenn $F$ eine stetige Verteilungsfunktion ist, dann gilt $\mu_F$-fast überall

\begin{align*}
    \lim_{h \downarrow 0}
        \frac{\Delta F}{\Delta F}(x, h)
    =
    \lim_{h \uparrow 0}
        \frac{\Delta F}{\Delta F}(x, h)
    =
    1,
\end{align*}

wobei wir für zwei Verteilungsfunktionen $F$ und $G$

\begin{align*}
    \frac{\Delta G}{\Delta F}(x, h)
    =
    \begin{cases}
        \frac{G(x + h) - G(x)}{F(x + h) - F(x)} & \text{wenn}~ F(x + h) \neq F(x), \\
        0                                       & \text{sonst}
    \end{cases}
\end{align*}

setzen.

\end{exercise}

% -------------------------------------------------------------------------------- %

\begin{solution}

Weil $\Q$ dicht in $\R$ liegt, gilt

\begin{align*}
    \Forall a, b \in \R, a < b:
        \Exists q \in \Q:
            q \in ]a, b].
\end{align*}

\begin{align*}
    \implies
    |K| \leq |\Q| = \aleph_0,
    \quad
    K := \Bbraces{]a, b] \in \mathfrak J \setminus \Bbraces{\emptyset}: F |_{]a, b]} ~\text{konst.}}
\end{align*}

\begin{align*}
    \implies
    \mu_F \pbraces{\lim_{h \downarrow 0} \frac{\Delta F}{\Delta F}(\cdot, h) \neq 1}
    & =
    \mu_F \pbraces{\lim_{h \downarrow 0} \frac{\Delta F}{\Delta F}(\cdot, h) = 0} \\
    & =
    \mu_F \pbraces{\Bbraces{x \in \R: \Exists \epsilon > 0: \Forall h \in (0, \epsilon): F(x + h) = F(x)}} \\
    & =
    \mu_F \pbraces{\bigcup K} \\
    & \leq
    \sum_{{]a, b]} \in K}
        \mu_F(]a, b]) \\
    & =
    \sum_{{]a, b]} \in K}
        \underbrace{F(b) - F(a)}_0 \\
    & =
    0
\end{align*}

\end{solution}

% -------------------------------------------------------------------------------- %
