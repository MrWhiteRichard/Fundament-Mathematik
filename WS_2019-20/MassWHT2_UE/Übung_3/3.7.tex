% -------------------------------------------------------------------------------- %

\begin{exercise}

Es ist naheliegend, dass für eine diskrete Verteilungsfunktion $F$ die Menge der Punkte, in denen $F^\prime$ nicht $0$ ist, mit der Menge der Sprungstellen von $F$ übereinstimmen sollte (und dass $F^\prime$, wenn definiert, nur die Werte $0$ und $\infty$ annimmt).
Das gilt so, wenn die Sprungstellen schön getrennt liegen (was eigentlich im \enquote{diskret} drinsteckt), muss aber im allgemeinen nicht so sein:

Zeigen sie, dass für

\begin{align*}
    F(x)
    =
    \begin{cases}
        -1               & \text{für}~ x < -1,                                           \\
        \frac{-1}{n + 1} & \text{für}~ n \in \N, -\frac{1}{n} \leq x < -\frac{1}{n + 1}, \\
        0                & \text{für}~ x = 0,                                            \\
        \frac{1}{n + 1}  & \text{für}~ n \in \N, \frac{1}{n + 1} \leq x < \frac{1}{n},   \\
        1                & \text{für}~ x \geq 1
    \end{cases}
\end{align*}

$|F(x) - x| \leq x^2$ und daher $F^\prime(0) = 1$ gilt.

\end{exercise}

% -------------------------------------------------------------------------------- %

\begin{solution}

Sei $x \in \R$.

\begin{enumerate}[label = \arabic*.]

    \item Fall ($x < -1$):
    
    \begin{align*}
        \implies &
        0 < -1 - x, \quad |x| \leq x^2 \\
        \implies &
        |F(x) - x|
        =
        |-1 - x|
        =
        -1 - x
        =
        -1 + |x|
        <
        |x|
        \leq
        x^2
    \end{align*}

    \item Fall ($\Exists n \in \N: -\frac{1}{n} \leq x < -\frac{1}{n+1}$):
    
    \begin{align*}
        \implies &
        0 < \frac{-1}{n+1} - x, \quad -x - x^2 = -x (1 + x) \leq \frac{1}{n} \pbraces{1 - \frac{1}{n+1}} = \frac{1}{n} \frac{n}{n+1} = \frac{1}{n+1} \\
        \implies &
        |F(x) - x|
        =
        \vbraces{\frac{-1}{n+1} - x}
        =
        -\frac{1}{n+1} - x
        \leq
        x^2
    \end{align*}

    \item Fall ($\Exists n \in \N: \frac{1}{n+1} \leq x < \frac{1}{n}$):
    
    Wir wenden den $2$-ten Fall auf $-x$ an, weil

    \begin{align*}
        |F(x) - x| = \vbraces{\frac{1}{n+1} - x} = \vbraces{\frac{-1}{n+1} - (-x)},
        \quad
        -\frac{1}{n+1} \geq -x > \frac{1}{n}.
    \end{align*}

\end{enumerate}

Weil $F$ symmetrisch ist, betrachten wir bloß den rechtsseitigen Differentialquotienten.
Für $\epsilon > 0$ hinreichend klein

\begin{align*}
    \Exists n_\epsilon \in \N:
        \frac{1}{n_\epsilon + 1} \leq x < \frac{1}{n_\epsilon}.
\end{align*}

\begin{align*}
    \implies
    F^\prime(0)
    =
    \lim_{\epsilon \to 0+}
        \frac{F(0 + \epsilon) - F(0)}{\epsilon}
    \begin{cases}
        \leq \lim_{\epsilon \to 0+} \frac{\frac{1}{n_\epsilon + 1}}{\frac{1}{n_\epsilon + 1}} = 1 \\
        \geq \lim_{\epsilon \to 0+} \frac{\frac{1}{n_\epsilon + 1}}{\frac{1}{n_\epsilon    }} = 1
    \end{cases}
\end{align*}

\end{solution}

% -------------------------------------------------------------------------------- %
