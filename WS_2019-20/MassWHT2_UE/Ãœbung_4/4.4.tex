% --------------------------------------------------------------------------------

\begin{exercise}

Zeigen Sie, dass jede messbare additive FUnktion $f: \R \to \R$ linear ist (zeigen Sie, dass es ein $M > 0$ gibt, sodass $A = [|f| < M]$ positives Lebesguemaß hat -- bei genauerer Betrachtung gilt das sogar für jedes $M > 0$. Was gilt dann für $A \ominus A$?).

\end{exercise}

% --------------------------------------------------------------------------------

\begin{solution}

\begin{align*}
    \bigcup_{n \in \N}
        f^{-1}(]-n, n[)
    =
    f^{-1} \pbraces{\bigcup_{n \in \N} ]-n, n[}
    =
    f^{-1}(\R)
    =
    \Bbraces{x \in \R: f(x) \in \R}
    =
    \R
\end{align*}

\begin{align*}
    \implies
    \infty
    =
    \lambda(\R)
    =
    \lambda \pbraces{\bigcup_{n \in \N} f^{-1}(]-n, n[)}
    \leq
    \sum_{n \in \N}
        \lambda(f^{-1}(]-n, n[))
\end{align*}

\begin{align*}
    \implies
    \Exists n \in \N:
        \lambda(A) > 0,
        \quad
        A := [|f| < M],
        \quad
        M := n    
\end{align*}

Laut der $2$-ten Aufgabe, enthält $A \ominus A$ ein nichtleeres Intervall $[a, b]$.

\begin{align*}
    \implies
    \Forall x \in [a, b] \subset A \ominus A:
        \Exists a_1, a_2 \in A:
            & ~
            x = a_1 - a_2 \\
        \implies & ~
        f(x) = f(a_1) - f(a_2) \leq |f(a_1)| + |f(a_2)| \leq 2 M
\end{align*}

$f$ ist also auf $[a, b]$ beschränkt.
Laut der $3$-ten Aufgabe, ist $f$ linear.

\end{solution}

% --------------------------------------------------------------------------------
