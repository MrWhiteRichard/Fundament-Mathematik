% --------------------------------------------------------------------------------

\begin{exercise}

Additive Funktionen auf $\R$ (etwas hochtrabender: Homomorphismen von $(\R, +)$ nach $(\R, +)$) sind Funktionen $f: \R \to \R$ mit

\begin{align*}
    f(x + y) = f(x) + f(y)
    \Forall x, y \in \R.
\end{align*}

Jede lineare Funktion $f(x) = c x$ zählt dazu;
aus dem Auswahlaxiom folgt die Existenz von additiven Funktionen, die nicht linear sind.
Zeigen Sie, dass diese \enquote{nichttrivialen Homomorphismen} auf jedem nichtleeren Intervall unbeschränkt sind, also dass jede additive Funktion linear ist, wenn es ein nichtleeres Intervall gibt, auf dem sie beschränkt ist.

\end{exercise}

% --------------------------------------------------------------------------------

\begin{solution}

\begin{align*}
    \implies &
    f(0) = f(0 + 0) = f(0) + f(0) \\
    \implies &
    0 = f(0) = f(x - x) = f(x) + f(-x) \\
    \implies &
    f(x) = -f(-x)
\end{align*}

\begin{itemize}
    \item Sei $n \in \N$.
    
    \begin{align*}
        \implies
        f(n x)
        =
        f(\underbrace{x + \cdots + x}_{\displaystyle \text{$n$-mal}})
        =
        \underbrace{f(x) + \cdots + f(x)}_{\displaystyle \text{$n$-mal}}
        =
        n f(x)
    \end{align*}

    \item Sei $m \in -\N$.
    
    \begin{align*}
        \implies
        f(m x)
        =
        f(|m| (-x))
        =
        |m| f(-x)
        =
        m (-f(-x))
        =
        m f(x)
    \end{align*}

    \item Sei $q \in \Q$.
    
    \begin{align*}
        \implies &
        \Exists m \in \Z, \Exists n \in \N:
            q = \frac{m}{n} \\
        \implies &
        n f \pbraces{\frac{1}{n} x} = \frac{n}{n} f(x) = f(x) \\
        \implies &
        f(q x)
        =
        m f \pbraces{\frac{1}{n} x}
        =
        \frac{m}{n} f(x)
        =
        q f(x)
    \end{align*}

\end{itemize}

Angenommen, $f$ wäre nicht linear.
o.B.d.A. $f(1) = 1$

\begin{align*}
    \implies
    \Forall q \in \Q:
        f(q) = q f(1) = q
\end{align*}

Nachdem $f$ nicht linear ist, gilt $f \neq \id$.

\begin{align*}
    \implies
    \Exists \alpha \in \R \setminus \Q:
        f(\alpha) \neq \alpha
\end{align*}

Setze $\delta := f(\alpha) - \alpha \neq 0$.
Sei $r > 0$, und $x, y \in \Q$ mit $x \neq y$, und setze $\beta := \frac{y - x}{\delta}$.
$\Q$ liegt dicht in $\R$.

\begin{align*}
    \implies &
    \Exists b \in \Q:
        |\beta - b| < \frac{r}{2 |\delta|}, \\
    &
    \Exists a \in \Q:
        |\alpha - a| < \frac{r}{2 |b|}
\end{align*}

Setze $X := x + b (\alpha - a)$ und $Y := f(X)$.

\begin{multline*}
    \implies
    Y
    =
    f(x + b \alpha - b a))
    =
    f(x) + f(b \alpha) - f(b a)
    =
    x + b f(\alpha) - b a
    =
    (y - \delta \beta) + b (\alpha + \delta) - b a \\
    =
    y + b (\alpha - a) - \delta (\beta - b)
\end{multline*}

\begin{align*}
    \implies
    \vbraces
    {
        \begin{pmatrix}
            x \\ y
        \end{pmatrix}
        -
        \begin{pmatrix}
            X \\ Y
        \end{pmatrix}
    }^2
    & =
    \vbraces
    {
        \begin{pmatrix}
            x - x - b (\alpha - a) \\
            y - y - b (\alpha - a) + \delta (\beta - b)
        \end{pmatrix}
    }^2 \\
    & =
    b^2 (\alpha - a)^2 + \delta^2 (\beta - b)^2 - 2 \delta b (\beta - b) (\alpha - a) + b^2 (\alpha - a)^2 \\
    & \leq
    b^2 \frac{r^2}{4 b^2} + \delta^2 \frac{r^2}{4 \delta^2} + 2 |\delta b| \frac{r}{2 |b|} \frac{r}{2 |\delta|} + b^2 \frac{r^2}{4 b^2} \\
    & =
    \frac{5 r^2}{4}
\end{align*}

\begin{align*}
    \implies
    \begin{pmatrix}
        X \\ Y
    \end{pmatrix}
    \in
    B
    \pbraces
    {
        \begin{pmatrix}
            x \\ y
        \end{pmatrix},
        \frac{5 r^2}{4}
    }
\end{align*}

Sei nun $I$ ein nichtleeres Intervall.
Wähle $x \in I^\circ$ und $y > 0$ beliebig.
Nach dem bisher Gezeigtem, gibt es ein $X \in I^\circ$, sodass $Y = f(X)$ beliebig nahe an $y$ ist.
$f$ ist also auf jedem nichtleerem Intervall unbeschränkt.

\end{solution}

% --------------------------------------------------------------------------------
