\documentclass{article}

\def \lastexercisenumber {0}

\usepackage[utf8]{inputenc}
\usepackage{fullpage}
\usepackage{amsmath, amssymb, amsfonts, amsthm}
\usepackage{bbm}
\usepackage{mathtools}
\usepackage{twoopt}

% special letters:

\newcommand{\N}{\mathbb{N}}
\newcommand{\Z}{\mathbb{Z}}
\newcommand{\Q}{\mathbb{Q}}
\newcommand{\R}{\mathbb{R}}
\newcommand{\C}{\mathbb{C}}
\newcommand{\K}{\mathbb{K}}
\newcommand{\T}{\mathbb{T}}
\newcommand{\E}{\mathbb{E}}
\newcommand{\V}{\mathbb{V}}
\renewcommand{\P}{\mathbb{P}}
\newcommand{\1}{\mathbbm{1}}

\newcommand  {\B}{\mathfrak{B}}
\renewcommand{\S}{\mathfrak{S}}

% quantors:

\newcommand{\Forall}{\forall \,}
\newcommand{\Exists}{\exists \,}
\newcommand{\ExistsOnlyOne}{\exists! \,}
\newcommand{\nExists}{\nexists \,}

% MISC symbols:

\newcommand{\landau}[1]
{
  {\scriptstyle \mathcal{O}}
  \pbraces{#1}
}

\newcommand{\Landau}[1]
{
  \mathcal{O}
  \pbraces{#1}
}

\newcommand{\eps}{\mathrm{eps}}

% graphics in a box:

\newcommandtwoopt
{\includegraphicsboxed}[3][][]
{
  \begin{figure}[!h]
    \begin{boxedin}
      \ifthenelse{\isempty{#2}}
      {
        \begin{center}
          \includegraphics[width = 0.75 \textwidth]{#3}
          \label{fig:#1}
        \end{center}
      }{
        \begin{center}
          \includegraphics[width = 0.75 \textwidth]{#3}
          \caption{#2}
          \label{fig:#1}
        \end{center}
      }
    \end{boxedin}
  \end{figure}
}

% braces:

\newcommand{\pbraces}[1]{{\left  ( #1 \right  )}}
\newcommand{\bbraces}[1]{{\left  [ #1 \right  ]}}
\newcommand{\Bbraces}[1]{{\left \{ #1 \right \}}}
\newcommand{\vbraces}[1]{{\left  | #1 \right  |}}
\newcommand{\Vbraces}[1]{{\left \| #1 \right \|}}
\newcommand{\abraces}[1]{{\left \langle #1 \right \rangle}}
\newcommand{\round}[1]{\bbraces{#1}}

\newcommand
{\floor}[1]
{{\left \lfloor #1 \right \rfloor}}

\newcommand
{\ceil} [1]
{{\left \lceil  #1 \right \rceil }}

% special functions:

\newcommand{\norm}  [2][]{\Vbraces{#2}_{#1}}
\newcommand{\diag}  [1]{\mathrm{diag} \: #1}
\newcommand{\dist}  [1]{\mathrm{dist} \: #1}
\newcommand{\mean}  [1]{\mathrm{mean} \: #1}
\newcommand{\erf}   [1]{\mathrm{erf} \: #1}
\newcommand{\id}    [1]{\mathrm{id} \: #1}
\newcommand{\sgn}   [1]{\mathrm{sgn} \: #1}
\newcommand{\supp}  [1]{\mathrm{supp} \: #1}
\newcommand{\arsinh}[1]{\mathrm{arsinh} \: #1}
\newcommand{\arcosh}[1]{\mathrm{arcosh} \: #1}
\newcommand{\artanh}[1]{\mathrm{artanh} \: #1}
\newcommand{\card}  [1]{\mathrm{card} \: #1}
\newcommand{\Span}  [1]{\mathrm{span} \: #1}
\newcommand{\Aut}   [1]{\mathrm{Aut} \: #1}
\newcommand{\End}   [1]{\mathrm{End} \: #1}
\newcommand{\ggT}   [1]{\mathrm{ggT} \: #1}
\newcommand{\kgV}   [1]{\mathrm{kgV} \: #1}
\newcommand{\ord}   [1]{\mathrm{ord} \: #1}
\newcommand{\grad}  [1]{\mathrm{grad} \: #1}
\newcommand{\ran}   [1]{\mathrm{ran} \: #1}
\newcommand{\graph} [1]{\mathrm{graph} \: #1}
\newcommand{\Inv}   [1]{\mathrm{Inv} \: #1}
\newcommand{\pv}    [1]{\mathrm{pv} \: #1}
\newcommand{\Mod}{\: \mathrm{mod} \:}
\newcommand{\Char}{\mathrm{char}}
\newcommand{\At}{\mathrm{At}}
\newcommand{\Ob}{\mathrm{Ob}}
\newcommand{\Hom}{\mathrm{Hom}}
\newcommand{\orthogonal}[3][]{#2 ~\bot_{#1}~ #3}
\newcommand{\Rang}{\mathrm{Rang}}

\newcommand
{\GL}[2][]
{\mathrm{GL}_{#1} \pbraces{#2}}

% fractions:

\newcommand{\Frac}[2]{\frac{1}{#1} \pbraces{#2}}
\newcommand{\nfrac}[2]{\nicefrac{#1}{#2}}

% derivatives & integrals:

\newcommandtwoopt
{\Int}[4][][]
{\int_{#1}^{#2} #3 ~\mathrm{d} #4}

\newcommandtwoopt
{\derivative}[3][][]
{
  \frac
  {\mathrm{d}^{#1} #2}
  {\mathrm{d} #3^{#1}}
}

\newcommandtwoopt
{\pderivative}[3][][]
{
  \frac
  {\partial^{#1} #2}
  {\partial #3^{#1}}
}

\newcommand
{\primeprime}
{{\prime \prime}}

\newcommand
{\primeprimeprime}
{{\prime \prime \prime}}

% Text:

\newcommand{\Quote}[1]{\glqq #1\grqq{}}
\newcommand{\Text}[1]{{\text{#1}}}
\newcommand{\fastueberall}{\text{f.ü.}}
\newcommand{\fastsicher}{\text{f.s.}}

% -------------------------------- %
% amsthm-stuff:

\theoremstyle{definition}

% numbered theorems
\newtheorem{theorem}    {Satz}   [section]
\newtheorem{lemma}      [theorem]{Lemma}
\newtheorem{corollary}  [theorem]{Korollar}
\newtheorem{proposition}[theorem]{Proposition}
\newtheorem{remark}     [theorem]{Bemerkung}
\newtheorem{definition} [theorem]{Definition}
\newtheorem{example}    [theorem]{Beispiel}

% unnumbered theorems
\newtheorem*{theorem*}    {Satz}
\newtheorem*{lemma*}      {Lemma}
\newtheorem*{corollary*}  {Korollar}
\newtheorem*{proposition*}{Proposition}
\newtheorem*{remark*}     {Bemerkung}
\newtheorem*{definition*} {Definition}
\newtheorem*{example*}    {Beispiel}

% Please define this stuff in project ("main.tex"):

% \def \lastexercisenumber {...}
% This will be 0 by default

% \setcounter{section}{...}
% This will be 0 by default
% and hence, completely ignored

\ifnum \thesection = 0
{
  \newtheorem{exercise}{Aufgabe}
}
\else
{
  \newtheorem{exercise}{Aufgabe}[section]
}
\fi

\ifdef
{\lastexercisenumber}
{\setcounter{exercise}{\lastexercisenumber}}

\newenvironment{solution}
{
  \begin{proof}[Lösung]
}{
  \end{proof}
}

\renewcommand{\proofname}{Beweis}

% -------------------------------- %
% environment zum einkasteln:

% dickere vertical lines
\newcolumntype
{x}
[1]
{
  !{
    \centering
    \arraybackslash
    \vrule
    width #1}
}

% environment selbst (the big cheese)
\newenvironment
{boxedin}
{
  \begin{tabular}
  {
    x{1 pt}
    p{\textwidth}
    x{1 pt}
  }
  \Xhline
  {2 \arrayrulewidth}
}
{
  \\
  \Xhline{2 \arrayrulewidth}
  \end{tabular}
}

% -------------------------------- %
% MISC "Ein-Deutschungen"

\renewcommand{\figurename}{Abbildung}
\renewcommand{\tablename} {Tabelle}

% -------------------------------- %


\parindent 0pt

\title
{
  Maß- und Wahrscheinlichkeitstheorie 2 - Übung 12\\
  \vspace{4pt}
  \normalsize
  \textit{12. UE am 15.01.2020}
}
\author
{
  Richard Weiss       \and
  Florian Schager     \and
  Christian Sallinger \and
  Fabian Zehetgruber  \and
  Paul Winkler        \and
  Christian Göth
}
\date{}

\begin{document}

\maketitle

\begin{align*}
  \Phi:
  \overline{\R} \to [0, 1]:
  x \mapsto \frac{1}{\sqrt{2 \pi}} \Int[-\infty][x]{e^\frac{-t^2}{2}}{t}
\end{align*}

% -------------------------------------------------------------------------------- %

\begin{exercise}[\textbf{Comparing two populations 1}]

A Study of the differences in cognitive function between normal
individuals and patients diagnosed with schizophrenia was published
in the American Journal of Psychiatry (Apr. 2010). The total time
(in minutes) a subject spent on the Trail Making Test (a standard
psychological test) was used as a measure of cognitive function.
The researchers theorize that the mean time on the Trail Making Test
for schizophrenics will be larger than the corresponding mean for normal
subjects. The data for independent random samples of 41 schizophrenics
and 49 normal individuals yielded the following results:

\begin{center}
\begin{tabular}{c|cc}
    & Schizophrenia & Normal \\
    \hline
    Sample size & 41 & 49 \\
    Mean time & 104.23 & 62.24 \\
    Standard deviation & 62.24 & 16.34
\end{tabular}
\end{center}

\begin{enumerate}[label = (\alph*)]
    \item Define the parameter of interest to the researchers.
    \item Set up the null and alternative hypothesis for testing
    the researchers' theory.
    \item The researchers conducted the test, part (b), and reported a
    $p$-value of .001. What conclusions can you draw from this result?
    (Use $\alpha = 0.01$)
    \item Find a 99\% confidence interval for the target parameter.
    Interpret the result. Does your conclusion agree with that
    of the previous part?
\end{enumerate}

\end{exercise}

% -------------------------------------------------------------------------------- %

\begin{solution}

Since we work with large sample sizes ($n_1 = 41 \geq 30$ and $n_2 = 49 \geq 30$),
we can use a normal-based test for independent samples.

The parameter of interest are the sample means $\bar{X}_1 = 104.23, \bar{X}_2 = 62.24$,
as well as the sample standard deviations $s_{1} = 62.24, s_{2} = 16.34$.

Let $\mu_1, \mu_2$ denote the true means of the respective populations.
Our null hypothesis thus states that there is no difference in the means,
i.e. $\mu_1 - \mu_2 = 0$.
For the alternative, we use the one-sided hypothesis $\mu_1 - \mu_2 > 0$.

Our test statistic then reads

\begin{align*}
    Z = \frac{\bar{X}_1 - \bar{X}_2}{\sqrt{\frac{s_1^2}{n_1} + \frac{s_2^2}{n_2}}}
    \approx_{H_0} \mathcal{N}(0,1)
\end{align*}

Using $\alpha = 0.01$ as our significance level, we can safely reject the null, since $0.001 < \alpha$.


Plugging in the given values we obtain 

\begin{align*}
    \text{p-value} = \P\left(Z \geq \frac{41.99}{\sqrt{\frac{62.24^2}{41} + \frac{16.34^2}{49}}}\right)
    \approx 0.0000133 \neq .001 ?
\end{align*}

Furthermore, the 99\% confidence interval for our target parameter $\mu_1 - \mu_2$ reads

\begin{align*}
    41.99 \pm z_{\alpha/2} \sqrt{\frac{62.24^2}{41} + \frac{16.34^2}{49}}
    \approx 41.99 \pm 25.74957.
\end{align*}

Since $0$ is significantly lower than the lower bound of our confidence interval,
our conclusions agree with the previous part.
\end{solution}

% -------------------------------------------------------------------------------- %

% -------------------------------------------------------------------------------- %

\begin{exercise}[\textbf{Comparing two populations 2}]

Suppose you wish to compare a new method of teaching reading to slow
learners with the current standard method. You decide to base your
comparison on the results of a reading test given at the end of a learning
period of six months. Of a random sample of 22 slow learners, 10 are taught
by the new method and 12 are taught by the standard method.
All 22 children are taught by qualified instructors under similar
conditions for the designated six-month period. The results of the reading
test at the end of this period are given below.

\begin{enumerate}[label = (\alph*)]
    \item[] New Method: 80, 76, 70, 80, 66, 85, 79, 71, 81, 76.
    \item[] Standard Method: 79, 73, 72, 62, 76, 68, 70, 86, 75, 68, 73, 66.
    \item Use the data in the table to estimate the true mean difference
    between the test scores for the new method and the standard method.
    Use a 95\% confidence interval.
    \item Interpret the interval you found in the previous part.
    \item What assumptions must be made in order that the estimate be valid?
    Are they reasonably satisfied?
\end{enumerate}

\end{exercise}

% -------------------------------------------------------------------------------- %

\begin{solution}

We are given small sample sizes (10 and 12) of two independent populations
with sample means $\bar{X}_1 = 76.4$ and $\bar{X}_2 = 72.333$ and sample
variances $s_1^2 = 34.044$ and $s_2^2 = 40.242$.

The 95\%-confidence interval thus reads

\begin{align*}
    (76.4 - 72.333) \pm t_{\alpha/2}(\nu)\sqrt{\frac{s_1^2}{n_1} + \frac{s_2^2}{n}}
    \approx 4.067 \pm 5.427.
\end{align*}

with 

\begin{align*}
    \nu = \frac{\left(\frac{s_1^2}{n_1} +  \frac{s_2^2}{n_2}\right)^2}
            {\frac{(s_1^2/n_1)^2}{n_1 - 1} + \frac{(s_2^2/n_2)^2}{n_2 - 1}}
\end{align*}

Since $0$ is part of our confidence interval, we cannot reject the null
at significance level $\alpha = 0.05$.

Necessary assumptions are that the true mean and variance of the compared
populations need be finite, which seems to be satisfied.
\end{solution}

% -------------------------------------------------------------------------------- %

% -------------------------------------------------------------------------------- %

\begin{exercise}[\textbf{Missing information}]

An investigation of ethnic differences in reports of pain perception
was presented at the annual meeting of the American Psychosomatic Society
(Mar. 2001). A sample of 55 blacks and 159 whites participated in the study.
Subjects rated (on a 13-pointt scale) the intensity and unpleasantness of
pain felt when a bag of ice was placed on their foreheads for two minutes.
(Higher ratings correspond to higher pain intensity.)
A summary of the results is provided in the following table.

\begin{center}
    \begin{tabular}{c|cc}
        & Blacks & Whites \\
        \hline
        Sample Size & 55 & 159 \\
        Mean pain intensity & 8.2 & 6.9
    \end{tabular}
\end{center}

\begin{enumerate}[label = (\alph*)]
    \item Why is it dangerous to draw a statistical inference from the summarized data? Explain.
    \item What values of the missing sample standard deviations would lead
    you to conclude (at $\alpha = 0.05$) that blacks, on average, have
    a higher pain intensity rating than whites?
    \item What values of the missing sample standard deviation would lead
    you to an inconclusive decision (at $\alpha = 0.05$) regarding whether
    blacks or whites have a higher mean intensity rating?
\end{enumerate}

\end{exercise}

% -------------------------------------------------------------------------------- %

\begin{solution}

\phantom{}

\end{solution}

% -------------------------------------------------------------------------------- %

% -------------------------------------------------------------------------------- %

\begin{exercise}[265]

Zeigen Sie (nicht formal) $A \cup B \subseteq \bigcup \bigcup (A \times B)$.
(Achtung, das ist im Allgemeinen falsch. Fügen Sie zuerst eine sinnvolle Annahme
über $A$ und $B$ hinzu, damit der Satz auch stimmt.)

\end{exercise}

% -------------------------------------------------------------------------------- %

\begin{solution}
$A,B$ müssen beide nichtleer sein.
\begin{align*}
  A \times B &= \{(a,b): a \in A, b \in B\} = \{\{\{a\},\{a,b\}\}: a \in A, b \in B\} \\
  \bigcup (A \times B) &= \{ \{a\}: a \in A\} \cup \{\{a,b\}: a \in A, b \in B\}\\
  \bigcup \bigcup (A \times B) &= \bigcup \pbraces{\{ \{a\}: a \in A\} \cup \{\{a,b\}: a \in A, b \in B\}} = A \cup \Bbraces{x : x \in A \lor x \in B}
  \supseteq A \cup B.
\end{align*}

\end{solution}

% ---------------------------------------------------------------- %

\begin{exercise}[Hypnosis]

Some researchers claim that susceptibility to hypnosis can be acquired or improved through training.
To investigate this claim six subjects were rated on a scale of $1$-$20$ according to their initial susceptibility to hypnosis and then given $4$ weeks of training.
Each subject was rated again after the training period. In the ratings below, higher numbers represent greater susceptibility to hypnosis.

\begin{align*}
    \begin{array}{c|c|c}
        \hline
        \text{Subject} & \text{Before} & \text{After} \\
        \hline
        1 & 10 & 18 \\
        2 & 16 & 19 \\
        3 & 7  & 11 \\
        4 & 4  & 3  \\
        5 & 7  & 5  \\
        6 & 2  & 3  \\ \hline
    \end{array}
\end{align*}

Specify and perform the appropriate hypothesis test.
What potential issues exist with this analysis?

\end{exercise}

% ---------------------------------------------------------------- %

\begin{solution}

\phantom{}

\end{solution}

% ---------------------------------------------------------------- %
\begin{algebraUE}{397}
Sei $p \in \P$ eine Primzahl. Zeigen Sie, dass $GF(p^{\infty})$ überabzählbar viele
nichtisomorphe Unterkörper hat. \\
\textit{Anleitung:} Für jede (unendliche) Menge $A \subseteq \P$ sei $K_A$ Vereinigung
aller $GF(p^n)$, für die gilt, dass alle Primfaktoren von $n$ in $A$ liegen.
Schreiben Sie $K_A$ als aufsteigende Vereinigung $\bigcup_{j=1}^{\infty}U_j$
von Unterkörpern, um zu beweisen, dass $K_A$ ein Körper ist. Für $A \neq A^{\prime}$
zeigen Sie $K_A \ncong K_{A^\prime}$, indem Sie ein Polynom (wo liegen die
Koeffizienten dieses Polynoms?) finden, das zwar in $K_A$, aber nicht in $K_{A^{\prime}}$
eine Nullstelle hat, oder umgekehrt.
\end{algebraUE}

\begin{solution}

ToDo!

\end{solution}

% --------------------------------------------------------------------------------

\begin{exercise}[270]

\phantom{}

\end{exercise}

% --------------------------------------------------------------------------------

\begin{solution}

\phantom{}

\end{solution}


\end{document}
