% -------------------------------------------------------------------------------- %

\begin{exercise}

$F$ sei eine nichtnegative Verteilungsfunktion und $G(x) = F(x)^2$, $x \in \R$.

\begin{enumerate}[label = (\alph*)]

    \item Zeigen Sie:
    $G$ ist eine Verteilungsfunktion

    \item Zeigen Sie:
    $\mu_G \ll \mu_F$

    \item Bestimmen Sie $\derivative[][\mu_G]{\mu_F}$

\end{enumerate}

\end{exercise}

% -------------------------------------------------------------------------------- %

\begin{solution}

\phantom{}

\begin{enumerate}[label = (\alph*)]

    \item $G$ ist monoton nichtfallend und rechtsstetig, weil $F$ es ist.
    
    \item Sei $]a, b] \in \mathfrak J$, mit
    
    \begin{align*}
        0 = \mu_F(]a, b]) = F(b) - F(a).
    \end{align*}

    \begin{align*}
        \implies
        \mu_G(]a, b])
        =
        G(b) - G(a)
        =
        F(b)^2 - F(a)^2
        =
        \underbrace{(F(b) - F(a))}_0 (F(b) + F(a))
        =
        0
    \end{align*}

    Der Fortsetzungssatz erledigt den Rest.
    
    \begin{comment}

        \item Sei $A \in \mathfrak B$ mit $\mu_F(A) = 0$.
        $\mu_F$ ist ein Lebesgue-Stieltjes-Maß, also sigmaendlich.

        \begin{align*}
            \implies &
            \mu_F(A)
            =
            \inf \Bbraces{\sum_{n \in \N} \mu_F(E_n): E_n \in \mathfrak J, A \subseteq \bigcup_{n \in \N} E_n} \\
            \implies &
            \Forall \epsilon > 0:
                \Exists (]a_n, b_n[) \subset \mathfrak J:
                    \sum_{n \in \N} ]a_n, b_n[ \supseteq A,
                    \quad
                    \mu_F \pbraces{\sum_{n \in \N} ]a_n, b_n[} \leq \underbrace{\mu_F(A)}_0 + \epsilon
        \end{align*}

        Sei $K \subseteq A$ kompakt.
        Laut dem Satz von Heine-Borel, gilt also

        \begin{align*}
            \Exists N \in \N:
                \sum_{n=1}^N:
                    ]a_n, b_n[ \supseteq K.
        \end{align*}

        \begin{align*}
            \implies
            \mu_G(K)
            \leq
            \mu_G \pbraces{\sum_{n=1}^N ]a_n, b_n[}
            \leq
            \sum_{n=1}^N
                \mu_G(]a_n, b_n[)
            \leq
            \sum_{n=1}^N
                \mu_G(]a_n, b_n])
            =
            \sum_{n=1}^N
                F(b_n)^2 - F(a_n)^2
            =
            \sum_{n=1}^N
                (F(b_n) - F(a_n)) (F(b_n) + F(a_n))
            \leq
            M \sum_{n=1}^N \mu_F(]a_n, b_n[)
            \leq
            M \cdot \epsilon
        \end{align*}

        Weil das für alle $\epsilon$ gilt, muss $\mu_G(K) = 0$.
        $\mu_G$ ist sigmaendlich, also insbesondere regulär von unten, d.h.

        \begin{align*}
            \mu_G(A) = \sup \Bbraces{\mu_G(K): K \subseteq A, ~\text{kompakt}}.
        \end{align*}

        Damit gilt aber $\mu_G(A) = 0$.

    \end{comment}

    \item

    \begin{multline*}
        \derivative[][\mu_G]{\mu_F}(x)
        =
        \lim_{h \to 0}
            \frac{G(x + h) - G(x - h)}{F(x + h) - F(x - h)}
        =
        \lim_{h \to 0}
            \frac{F(x + h)^2 - F(x - h)^2}{F(x + h) - F(x - h)} \\
        =
        \lim_{h \to 0}
            \frac{F(x + h) - F(x - h)}{F(x + h) - F(x - h)}
            (F(x + h) + F(x - h))
        =
        F(x) F(x - 0)
    \end{multline*}

\end{enumerate}


\end{solution}

% -------------------------------------------------------------------------------- %
    