% -------------------------------------------------------------------------------- %

\begin{exercise}

$f: [a, b] \to \R$ heißt Lipschitz-stetig, wenn es eine konstante $M$ gibt, sodass für $a \leq s \leq t \leq b$ $|f(t) - f(s)| \leq M (t - s)$ gilt.
Zeigen Sie:

\begin{enumerate}[label = (\alph*)]
    \item $f$ ist absolutstetig,
    \item $g = \derivative[][\mu_f]{\lambda}$ erfüllt $|g| \leq M$ fast überall.
\end{enumerate}

Insgesamt ist $f$ genau dann Lipschitz-stetig, wenn

\begin{align*}
    f(x)
    =
    \Int[\bbraces{a, x}]{g}{\lambda}
\end{align*}

mit einer beschränkten maessbaren Funktion $g$.

\end{exercise}

% -------------------------------------------------------------------------------- %

\begin{solution}

\phantom{}

\begin{enumerate}[label = (\alph*)]

    \item

    \begin{align*}
        \Forall \epsilon > 0:
            \Exists \delta > 0:
                \Forall a \leq s_1 < t_1 \leq \cdots \leq s_n < t_n \leq b:
                    \pbraces
                    {
                        \sum_{i=1}^n
                            |t_i - s_i|
                        \leq
                        \delta
                        \implies
                        \sum_{i=1}^n
                            |f(t_i) - f(s_i)|
                        \leq
                        \epsilon
                    }
    \end{align*}

    nämlich $\delta := \epsilon / M$, weil

    \begin{align*}
        \sum_{i=1}^n
            |f(t_i) - f(s_i)|
        \leq
        \sum_{i=1}^n
            M |t_i - s_i|
        \leq
        M \delta
        =
        \epsilon.
    \end{align*}

    \item Laut Satz 7.1: Radon-Nikodym, ist $g$ nicht-negativ.
    
    \includegraphicsboxed{MassWHT1&2/MassWHT1&2 - Satz 7.1 - Radon-Nikodym.png}

    \begin{align*}
        \Forall ]s, t] \subset [a, b]:
            \Int[s][t]{|g|}{\lambda}
            =
            \Int[s][t]{g}{\lambda}
            =
            \mu_f(]s, t])
            =
            |f(t) - f(s)|
            \leq
            M |t - s|
            =
            \Int[s][t]{M}{\lambda}
    \end{align*}

    Laut dem Fortsetzungssatz, gilt diese Gleichung nicht nur für Intervalle, sondern alle Borelmengen.
    Daraus folgt die Behauptung.

\end{enumerate}

\begin{itemize}

    \item [\enquote{$\Rightarrow$}]:

    \begin{align*}
        \implies
        f(x)
        \stackrel{?}{=}
        |f(x) - f(a)|
        =
        \mu_f(]a, x])
        =
        \Int[a][x]{g}{\lambda}
    \end{align*}

    \item [\enquote{$\Leftarrow$}]:
    Sei $g \leq M$ messbar und

    \begin{align*}
        f(x) = \Int[\bbraces{a, x}]{g}{\lambda}.
    \end{align*}

    \begin{align*}
        \Forall ]s, t] \subset [a, b]:
            |f(t) - f(s)|
            =
            \vbraces
            {
                \Int[a][t]{g}{\lambda}
                -
                \Int[a][s]{g}{\lambda}
            }
            =
            \vbraces
            {
                \Int[s][t]{g}{\lambda}
            }
            \leq
            \Int[s][t]{|g|}{\lambda}
            \leq
            M \lambda(]s, t])
            =
            M |t - s|
    \end{align*}    

\end{itemize}
    
\end{solution}

% -------------------------------------------------------------------------------- %
    