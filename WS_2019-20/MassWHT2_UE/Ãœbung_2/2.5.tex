% --------------------------------------------------------------------------------

\begin{exercise}

Es sei

\begin{align*}
    F(x)
    =
    \begin{cases}
        0         & \text{für}~ x < 0 \\
        x         & \text{für}~ 0 \leq x < 1 \\
        (x + 1)^2 & \text{für}~ 1 \leq x < 2 \\
        9         & \text{für}~ x \geq 2
    \end{cases}
\end{align*}

und

\begin{align*}
    G(x)
    =
    \begin{cases}
        0     & \text{für}~ x < 0 \\
        x + 1 & \text{für}~ 0 \leq x < 2 \\
        x^2   & \text{für}~ 2 \leq x < 3 \\
        9     & \text{für}~ x \geq 3
    \end{cases}
\end{align*}

Bestimmen Sie die Lebesguezerlegung von $\mu_G$ bezüglich $\mu_F$ und die Radon-Nikodym Dichte des absolutstetigen Anteils.

\end{exercise}

% --------------------------------------------------------------------------------

\begin{solution}

$F$ und $G$ sind monoton nichtfallend und rechtsstetig, also Verteilungsfunktionen.
Wir bestimmen die Sprungstellen von $F$.

\begin{enumerate}
    \item $0$ ist keine, weil $F(0 - 0) = 0 = F(0) = 0$;
    \item $1$ ist eine,  weil $F(1 - 0) = 1 \neq F(1) = (1 + 1)^2 = 4$;
    \item $2$ ist keine, weil $F(2 - 0) = (2 + 1)^2 = 9 = F(2) = 9$;
\end{enumerate}

Wir bestimmen die Sprungstellen von $G$.

\begin{enumerate}
    \item $0$ ist eine,  weil $G(0 - 0) = 0 \neq G(0) = 0 + 1 = 1$;
    \item $2$ ist eine,  weil $G(2 - 0) = 2 + 1 = 3 \neq G(2) = 2^2 = 4$;
    \item $3$ ist keine, weil $G(3 - 0) = 3^2 = 9 = G(3) = 9$;
\end{enumerate}

$F$ hat also die Sprungstelle $1$ und $G$ die Sprungstellen $0$ und $2$.

Mit dem \enquote{Kochrezept} erhalten wir folgende Tabelle.

\begin{align*}
    \begin{array}{c|c|c|c|c}
        \text{Kommentar}  & x         & f(x)              & G_\mathrm{c}(x) & G_\mathrm{s}(x) \\ \hline
        1.                & x < 0     & 0                 & 0               & 0               \\
        2.                & x = 0     & 0                 & 0               & 1               \\
        3.                & 0 < x < 1 & 1                 & x               & 1               \\
        4.                & x = 1     & 0                 & 1               & 1               \\
        5.                & 1 < x < 2 & \frac{1}{2 x + 2} & x               & 1               \\
        6.                & x = 2     & 0                 & 2               & 2               \\
        7.                & 2 < x < 3 & 0                 & 2               & x^2 - 2         \\
        8.                & 3 \leq x  & 0                 & 2               & 7
    \end{array}
\end{align*}

\begin{enumerate}

    \item Intervall ($x < 0$); $F^\prime(x) = 0$: \\
    $\derivative[][\mu_{G_\mathrm{c}}]{\mu_F}(x)$ kann beliebig gesetzt werden, etwa $\derivative[][\mu_{G_\mathrm{c}}]{\mu_F}(x) = 0$;

    \begin{align*}
        &
        G_\mathrm{s}^\prime(x) = G^\prime(x) = 0 \\
        \implies &
        G_\mathrm{s}(x) = \int G^\prime(x) := 0 \\
        \implies &
        G_\mathrm{c}(x) := G(x) - G_\mathrm{s}(x) = 0 - 0 = 0
    \end{align*}

    \item Sprungstelle ($x = 0$) von $G$; $F(x) = F(x - 0)$: \\
    $\derivative[][\mu_{G_\mathrm{c}}]{\mu_F}(x)$ kann beliebig gesetzt werden, etwa $\derivative[][\mu_{G_\mathrm{c}}]{\mu_F}(x) = 0$;
    der Sprung von $G$ kommt in den singulären Anteil, also

    \begin{align*}
        &
        G_\mathrm{s}(x) - G_\mathrm{s}(x - 0) = G(x) - G(x - 0) \\
        \implies &
        G_\mathrm{s}(x) := G(x) - G(x - 0) + G_\mathrm{s}(x - 0) = (0 + 1) - 0 + 0 = 1 \\
        \implies &
        G_\mathrm{c}(x) := G(x) - G_\mathrm{s}(x) = x + 1 - 1 = x = 0
    \end{align*}

    \item Intervall ($0 < x < 1$); $F^\prime(x) = 1 > 0$:
    
    \begin{align*}
        \derivative[][\mu_{G_\mathrm{c}}]{\mu_F}(x) & = \frac{G^\prime(x)}{F^\prime(x)} = \frac{1}{1} = 1, \\
        G_\mathrm{c}^\prime(x) & = G^\prime(x) = 1. \\
        \implies &
        G_\mathrm{c}(x) = \int G^\prime(x) := x \\
        \implies &
        G_\mathrm{s}(x) := G(x) - G_\mathrm{s}(x) = x + 1 - x = 1
    \end{align*}

    \item Sprungstelle ($x = 1$) von $F$; $F(x) = (1 + 1)^2 = 4 \neq F(x - 0) = 1$:
    
    \begin{align*}
        \derivative[][\mu_{G_\mathrm{c}}]{\mu_F}(x)
        =
        \frac{G(x) - G(x - 0)}{F(x) - F(x - 0)}
        =
        \frac{(1 + 1) - (1 + 1)}{(1 + 1)^2 - 1}
        =
        0,
    \end{align*}

    der Sprung von $G$ kommt in den absolutstetigen Anteil, also

    \begin{align*}
        &
        G_\mathrm{c}(x) - G_\mathrm{c}(x - 0) = G(x) - G(x - 0). \\
        \implies &
        G_\mathrm{c}(x) = G(x) - G(x - 0) + G_\mathrm{c}(x - 0) = (1 + 1) - (1 + 1) + x = 1 \\
        \implies &
        G_\mathrm{s}(x) := G(x) - G_\mathrm{c}(x) = (1 + 1) - x = 2 - 1 = 1
    \end{align*}

    \item Intervall ($1 < x < 2$); $F^\prime(x) = 2 (x + 1) > 0$:

    \begin{align*}
        \derivative[][\mu_{G_\mathrm{c}}]{\mu_F}(x) & = \frac{G^\prime(x)}{F^\prime(x)} = \frac{1}{2 (x + 1)} = \frac{1}{2 x + 2}, \\
        G_\mathrm{c}^\prime(x) & = G^\prime(x) = 1. \\
        \implies &
        G_\mathrm{c}(x) = \int G^\prime(x) := x \\
        \implies &
        G_\mathrm{s}(x) := G(x) - G_\mathrm{s}(x) = x + 1 - x = 1
    \end{align*}

    \item Sprungstelle ($x = 2$) von $G$; $F(x) = 9 = F(x - 0) = (2 + 1)^2 = 9$: \\
    $\derivative[][\mu_{G_\mathrm{c}}]{\mu_F}(x)$ kann beliebig gesetzt werden, etwa $\derivative[][\mu_{G_\mathrm{c}}]{\mu_F}(x) = 0$;
    der Sprung von $G$ kommt in den singulären Anteil, also

    \begin{align*}
        &
        G_\mathrm{s}(x) - G_\mathrm{s}(x - 0) = G(x) - G(x - 0) \\
        \implies &
        G_\mathrm{s}(x) := G(x) - G(x - 0) + G_\mathrm{s}(x - 0) = 2^2 - (2 + 1) + 1 = 2 \\
        \implies &
        G_\mathrm{c}(x) := G(x) - G_\mathrm{s}(x) = 2^2 - 2 = 2
    \end{align*}

    \item Intervall ($2 < x < 3$); $F^\prime(x) = 0$:
    $\derivative[][\mu_{G_\mathrm{c}}]{\mu_F}(x)$ kann beliebig gesetzt werden, etwa $\derivative[][\mu_{G_\mathrm{c}}]{\mu_F}(x) = 0$;

    \begin{align*}
        &
        G_\mathrm{s}^\prime(x) = G^\prime(x) = 2 x. \\
        \implies &
        G_\mathrm{s}(x) = \int G^\prime(x) = x^2 + ~\text{konst.}, quad G_\mathrm{s}(2) = 2 \implies G_\mathrm{s}(x) = x^2 - 2 \\
        \implies &
        G_\mathrm{c}(x) := G(x) - G_\mathrm{s}(x) = x^2 - (x^2 - 2) = 2
    \end{align*}

    \item Intervall ($3 \leq x$); $F^\prime(x) = 0$:

    $\derivative[][\mu_{G_\mathrm{c}}]{\mu_F}(x)$ kann beliebig gesetzt werden, etwa $\derivative[][\mu_{G_c}]{\mu_F}(x) = 0$;

    \begin{align*}
        &
        G_\mathrm{s}^\prime(x) = G^\prime(x) = 0. \\
        \implies &
        G_\mathrm{s}(x) = \int G^\prime(x) = ~\text{konst.}, G_\mathrm{s}(3) = G(3) - G_c(3) = 9 - 2 = 7 \implies G_\mathrm{s}(x) = 7 \\
        \implies &
        G_\mathrm{c}(x) = G(x) - G_\mathrm{s}(x) = 9 - 7 = 2
    \end{align*}

\end{enumerate}

\end{solution}

% --------------------------------------------------------------------------------
    