% --------------------------------------------------------------------------------

\begin{exercise}

$X$ sei standardnormalverteilt.

\begin{enumerate}[label = (\alph*)]

    \item Bestimmen Sie $\E(e^{t X})$ für $t \in \R$.

    \item Wenden Sie die Markov-Ungleichung auf $e^{t X}$ an, um eine obere Schranke für $\P(X \geq x)$ ($x > 0$) zu erhalten.
    Was ist die kleinste Schranke, die man so erhält?

\end{enumerate}

\end{exercise}

% --------------------------------------------------------------------------------

\begin{solution}

\begin{enumerate}[label = (\alph*)]

    \item

    \begin{multline*}
        \E(e^{t X})
        =
        \Int{e^{t X}}{\P}
        =
        \Int[-\infty][\infty]
        {
            e^{t x}
            \derivative{x}
            \pbraces
            {
                \frac{1}{\sqrt{2 \pi}}
                \Int[-\infty]{x}
                {
                    e^{-u^2 / 2}
                }{u}
            }
        }{x}
        =
        \frac{1}{\sqrt{2 \pi}}
        \Int[-\infty][\infty]
        {
            e^{t x - x^2 / 2}
        }{x} \\
        =
        e^{t^2 / 2}
        \frac{1}{\sqrt{2 \pi}}
        \Int[-\infty][\infty]
        {
            e^{-(x - t)^2 / 2}
        }{x}
        =
        e^{t^2 / 2}
        \underbrace
        {
            \frac{1}{\sqrt{2 \pi}}
            \Int[-\infty][\infty]
            {
                e^{-u^2 / 2}
            }{u}
        }_1
    \end{multline*}

    Dabei haben wir die folgende Substitution verwendet.

    \begin{align*}
        u = x - t
        \implies
        \derivative[][u]{x} = 1
        \implies
        \mathrm d u = \mathrm d x
    \end{align*}

    \item \phantom{}

    \includegraphicsboxed{MassWHT1&2/MassWHT1&2 - Satz 5.16 - Ungleichung von Markov.png}

    \begin{align*}
        \P(X \geq x)
        =
        \P(e^{t X} \geq e^{t x})
        \stackrel
        {
            \text{Markov}
        }{\leq}
        e^{-t x} \Int{e^{t x}}{\P}
        =
        e^{t^2 / 2 - t x}
    \end{align*}

    \begin{align*}
        0
        \stackrel{!}{=}
        \derivative{t}
        \pbraces
        {
            t^2 / 2 - t x
        }
        =
        t - x
    \end{align*}

    \begin{align*}
        \implies
        \inf \Bbraces{e^{t^2 / 2 - t x}: t \in \R}
        =
        e^{\inf \Bbraces{t^2 / 2 - t x: t \in \R}}
        =
        e^{-x^2 / 2}
    \end{align*}

\end{enumerate}

\end{solution}

% --------------------------------------------------------------------------------
