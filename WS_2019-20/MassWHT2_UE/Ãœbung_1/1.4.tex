% --------------------------------------------------------------------------------

\begin{exercise}

In einem Spiel kann auf die Ausgänge $1, \dots, m$ gesetzt werden, die mit Wahrscheinlichkeit $p_1, \dots, p_m$ gezogen werden.
Wird $i$ gezogen, weren die Einsätze auf $i$ $m$-fach zurückgezahlt, alle anderen Einsätze verfallen.
Eine Spielerin teilt ihr Kapital $K$ (mit Anfangswert $K_0$) in jeder Runde so auf, dass sie $K$ $q_i$ auf $i$ setzt, wobei $q_1, \dots, q_n$ Zahlen mit $q_i > 0$ und $\sum_i q_i = 1$ sind.

\begin{enumerate}[label = (\alph*)]

    \item Zeigen Sie:
    für das Kapital $K_n$ der Spielerin nach $n$ Runden gilt

    \begin{align*}
        K_n
        =
        K_0 m^n \prod_{j=1}^n q_{X_j},
    \end{align*}

    wobei $(X_j)$ eine Folge von unabhängigen Zufallsvariablen mit

    \begin{align*}
        \P(X_j = i) = p_i
    \end{align*}

    ist.

    \item Bestimmen Sie
    
    \begin{align*}
        \lim_{n \to \infty}
            \frac{\log K_n}{n}.
    \end{align*}

    \item Für welche Wahl von $q_1, \dots, q_n$ ist dieser Grenzwert maximal?

\end{enumerate}

\end{exercise}

% --------------------------------------------------------------------------------

\begin{solution}

\phantom{}

\begin{enumerate}[label = (\alph*)]

    \item Wir verwenden vollständige Induktion.
    Der Induktionsanfang ($k = 0$) ist trivial.
    Der Induktionsschritt ($k \mapsto k + 1$) ist auch nicht schwierig.

    \begin{align*}
        K_{k+1}
        =
        K_k \cdot m q_{X_{k+1}}
        \stackrel
        {
            \text{IV}
        }{=}
        K_0 m^{k+1} \prod_{j=1}^{k+1} q_{X_j}
    \end{align*}

    Die Aussage folgt für $k = n$.

    \item

    \begin{multline*}
        \lim_{n \to \infty}
            \frac{\log K_n}{n}
        =
        \lim_{n \to \infty}
            \frac{1}{n}
            \pbraces
            {
                \log K_0 + n \log m + \sum_{i \leq m} \log q_{X_i}
            }
        \stackrel
        {
            \text{GGZ}
        }{=}
        \log m + \E(\log q_{X_1}) \\
        =
        \log m + \Int{\log q_{X_1}}{\P}
        =
        \log m + \sum_{i \leq m} \log q_{X_1} \cdot \P(X_1 = i)
        =
        \log m + \sum_{i \leq m} \log q_{X_1} \cdot p_i
    \end{multline*}

    \item

    \begin{align*}
        q := (q_1, \dots, q_m),
        \quad
        \sum q := \sum_{i \leq m} q_i,
        \quad
        p := (p_1, \dots, p_m),
        \quad
        \sum p := \sum_{i \leq m} p_i
    \end{align*}

    \begin{gather*}
        f(q) := \sum_{i \leq m} \log q_i \cdot p_i,
        \quad
        G(q) := 1 - \sum q \stackrel{!}{=} 0,
        \quad
        q > 0,
        \quad
        F(q, \lambda) := f(q) - \lambda G(q)
    \end{gather*}

    \begin{align*}
        \implies
            &   \pderivative[][F]{q_i}    (q, \lambda) = \frac{p_i}{q_i} + \lambda \stackrel{!}{=} 0,
                \quad
                \pderivative[][F]{\lambda}(q, \lambda) = \sum q - 1   \stackrel{!}{=} 0  \\
        \implies
            &   1 = \sum q,
                \quad
                q_i = -\frac{p_i}{\lambda} \implies \sum q = -\frac{1}{\lambda} \sum p \\
        \implies
            & \lambda = -\sum p \\
        \implies
            & q_i = \frac{p_i}{\sum p}
    \end{align*}

\end{enumerate}

\end{solution}

% --------------------------------------------------------------------------------
