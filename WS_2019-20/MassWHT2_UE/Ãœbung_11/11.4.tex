\begin{lemma}
    Wenn $(X_n)_{n\in\mathbb{N}}$ eine Folge von Zufallsvariablen auf dem Maßraum $(\Omega,\mathfrak{S},\mathbb{P})$ mit $\forall n\in\mathbb{N}:X_n:\Omega\to\mathbb{Z}$ ist dann konvergiert $X_n$ in Verteilung genau dann, wenn für alle $k\in\mathbb{Z}$ der Grenzwert $p_k:=\lim_{n\to\infty}\mathbb{P}(X_n=k)$ existiert und $\sum_{k\in\mathbb{Z}}p_k=1$ gilt.
\end{lemma}
\begin{proof}[Beweis.]
    Wir zeigen zuerst die Hinrichtung, also "$\Rightarrow$". In einem ersten Schritt wollen wir zeigen, dass $F$ auf ganz $\mathbb{R}\setminus\mathbb{Z}$ stetig ist. Dafür wählen wir ein beliebiges $x\in\mathbb{R}\setminus\mathbb{Z}$ und ein $\epsilon>0$ so, dass $U_\epsilon(x)\cap\mathbb{Z}=\emptyset$ gilt. Da die $X_n$ nur Werte in $\mathbb{Z}$ annehmen muss für alle $n\in\mathbb{N}$ gelten, dass $F_n\vert_{U_\epsilon(x)}$ konstant ist. Aus der Charakterisierung der Stetigkeit durch die Levy-Metrik wissen wir, dass es für alle $\delta>0$ ein $n_0\in\mathbb{N}$ so gibt, das für alle $n\geq n_0:F_n(x-\delta)-\delta\leq F(x)\leq F_n(x+\delta)+\delta$ gilt. Daraus erhält man unmittelbar $F_n(x)\to F(x)$ und da die $F_n\vert_{U_\epsilon(x)}$ konstant ist muss auch der Grenzwert für alle Punkte der gleiche sein, also $F\vert_{U_\epsilon(x)}$ konstant und damit $F$ auf $\mathbb{R}\setminus\mathbb{Z}$ stetig. Die Werte von $F$ in $\mathbb{Z}$ ergeben sich nun schon automatisch durch die Rechtsstetigkeit von $F$. Betrachte dazu
    \begin{align*}
        \lim_{n\to\infty}P(X_n = k)&=\lim_{n\to\infty}\left(\sum_{j=-\infty}^kP(X_n=j)-\sum_{j=-\infty}^{k-1}P(X_n=j)\right)\\
        &=\lim_{n\to\infty}\left(F_n(k)-F_n(k-1)\right)\\
        &=\lim_{n\to\infty}\left(F_n\left(k+\frac{1}{2}\right)-F_n\left(k-\frac{1}{2}\right)\right)\\
        &=F\left(k+\frac{1}{2}\right)-F\left(k-\frac{1}{2}\right)\\
        &=F(k)-F(k-1)=P(X=k)=:p_k
    \end{align*}
    Da $P$ ein Wahrscheinlichkeitsmaß ist und $X$ nach unseren obigen Überlegungen nur Werte in $\mathbb{Z}$ annehmen kann gilt $\sum_{k\in\mathbb{Z}}p_k=1$

    Um die andere Richtung zu beweisen definieren wir eine Verteilungsfunktion
    \begin{align*}
        F:\mathbb{R}\to\mathbb{R}:x\mapsto\sum_{k=-\infty}^{\lfloor x\rfloor}p_k
    \end{align*}
    Wir wählen zusätzlich $k_n \in \mathbb{Z}$, sodass: $\lim_{n\to\infty}F(k_n) = 0$. \\
    Es gilt für alle $x\in\mathbb{R}$
    \begin{align*}
        \lim_{n\to\infty}| F_n(x) - F(x) | &=|\lim_{n\to\infty}\sum_{k=-\infty}^{k_n}P(X_n=k) - p_k + \sum_{k= k_n + 1}^{\lfloor x\rfloor}P(X_n=k) + p_k |\\
        &=\sum_{k= k_n + 1}^{\lfloor x\rfloor}\lim_{n\to\infty}P(X_n=k) - p_k = \sum_{k= k_n + 1}^{\lfloor x\rfloor}0
    \end{align*}
Da weiters $F(-\infty)=0$ und $F(\infty)=1$ ist $F$ eine Verteilungsfunktion im engeren Sinn und $F_n\to F$ in allen Stetigkeitspunkten von $F$. Also gilt für $X\sim F$, dass $X_n\to X$ in Verteilung.
\end{proof}
