% --------------------------------------------------------------------------------

\begin{exercise}

Aus dem Dichtesatz von Lebesgue folgt, dass es zu jeder Menge mit positivem Lebesguemaß ein Intervall $[a, b]$ gibt, sodass $\lambda(A \cap [a, b]) > \frac{3 (b - a)}{4}$.
Für $h < (b - a) / 2$ folgt daraus $A \cap (A \oplus h) \neq \emptyset$.
Mit anderen Worten:
$A \ominus A$ enthält ein Intervall.

\end{exercise}

% --------------------------------------------------------------------------------

\begin{solution}

\phantom{}

\begin{enumerate}[label = (\arabic*)]

    \item Aus dem Dichtesatz von Lebesgue folgt, dass
    
    \begin{align*}
        \Exists x \in A:
            \Forall \epsilon > 0:
                \Exists h > 0:
                    1 - \frac{\lambda(A \cap [x - h, x + h])}{2 h} < \epsilon.
    \end{align*}

    \begin{align*}
        a := x - h, \quad b := x + h \\
        & \implies
        1 - \frac{\lambda(A \cap [a, b])}{b - a} < \epsilon \\
        & \implies
        (1 - \epsilon) (b - a) < \lambda(A \cap [a, b])
    \end{align*}

    Für $\epsilon := 1/4$ folgt die Behauptung.

    \item $\lambda$ ist translationsinvariant und

    \begin{align*}
        \Forall A, B \in \mathfrak S:
            \lambda(A \cap B) = \lambda(A) + \lambda(B) - \lambda(A \cup B).
    \end{align*}

    \begin{align*}
        \implies
        \lambda((A \cap [a, b]) \oplus h) & = \lambda(A \cap [a, b]) > \frac{3}{4} (b - a), \\
        \lambda(A \cap [a, b + h]) & < \lambda([a, b + h]) = b + h - a < b - a + \frac{b - a}{2} = \frac{3}{2} (b - a)
    \end{align*}

    \begin{multline*}
        \implies
        \lambda((A \cap [a, b]) \cap ((A \cap [a, b]) \oplus h))
        =
        \lambda(A \cap [a, b]) + \lambda(((A \cap [a, b]) \oplus h)) - \lambda(A \cap [a, b + h]) \\
        >
        \pbraces{2 \cdot \frac{3}{4} + 1} (b - a)
        >
        0
    \end{multline*}

\end{enumerate}

\end{solution}

% --------------------------------------------------------------------------------
