\begin{exercise}

Zeigen Sie: Wenn für ein $t \neq 0$ $\phi_X(t) = 1$ gilt, dann nimmt $X$ mit Wahrscheinlichkeit $1$ nur Werte der Form $2n \pi/t$, $n \in \Z$ an. Gilt $\phi(t_1) = \phi(t_2) = 1$ für zwei inkommensurable Werte $t_1$ und $t_2$ (d.h. $t_1/t_2$ ist irrational), dann gilt $X = 0$ fast sicher.

\end{exercise}

\begin{solution}

Für die erste Aussage ist zu zeigen, dass

\begin{align*}
  \P(X \in \frac{2 \pi}{t} \Z) = 1.
\end{align*}

Man betrachte folgende Gleichungskette, wobei der letzte Summand wegfällt, weil $1 \in \R$.

\begin{align}
  1
  = \phi_X(t)
  = \E(e^{iXt})
  = \Int{e^{iX(\omega)t}}{\P(\omega)}
  = \Int{\cos(X(\omega)t)}{\P(\omega)} +
    i \underbrace{\Int{\sin(X(\omega)t)}{\P(\omega)}}_0
  \label{gleichungskette}
\end{align}

Es ist bekanntlich $\cos^{-1}(\Bbraces{1}) = 2 \pi \Z$. Angenommen, $\Exists N \in \S: \P(N) > 0, \: \Forall \omega \in N:$

\begin{align*}
  X(\omega) \not \in \frac{2 \pi}{t} \Z
  \Rightarrow
  \cos(X(\omega)t) < 1.
\end{align*}

Dann gilt aber für das Integral über $N$, dass

\begin{align*}
  \Int[N]{\cos(X(\omega)t)}{\P(\omega)} <
  \Int[N]{}{\P(\omega)} =
  \P(N).
\end{align*}

Man setzt die obere Gleichungskette \eqref{gleichungskette} fort und erhält einen Widerspruch.

\begin{align*}
  \Int[N]{\cos(X(\omega)t)}{\P(\omega)} +
  \Int[N^c]{\cos(X(\omega)t)}{\P(\omega)} <
  P(N) + P(N^c) = P(\Omega) = 1
\end{align*}

Somit lässt sich \eqref{gleichungskette} richtig weiterführen.

\begin{align*}
  \Int[\bbraces{X \in \frac{2 \pi}{t} \Z}]{\cos(X(\omega)t)}{\P(\omega)}
  = \Int[\bbraces{X \in \frac{2 \pi}{t} \Z}]{}{\P}
  = \P(X \in \frac{2 \pi}{t} \Z)
\end{align*}

Für die zweite Aussage ist zu zeigen, dass

\begin{align*}
  \P(X = 0) = 1.
\end{align*}

Laut Vorher, gilt für $t = t_1, t_2$, dass

\begin{align*}
  \sum_{n \in \Z} \P(X = \frac{2 \pi}{t} n)
  = \P(X \in \frac{2 \pi}{t} \Z)
  = 1.
\end{align*}

Also

\begin{align*}
  \Exists N_1 \in \S: \P(N_1^c) = 0, \:
  \Forall \omega \in N_1:
  \Exists n_1 \in \Z:
  X(\omega) = \frac{2 \pi n_1}{t_1}, \\
  \Exists N_2 \in \S: \P(N_2^c) = 0, \:
  \Forall \omega \in N_2:
  \Exists n_2 \in \Z:
  X(\omega) = \frac{2 \pi n_2}{t_2}.
\end{align*}

Wir wollen eine Aussage über die Elemente des Schnittes $N := N_1 \cap N_2 \in \S$ dieser Mengen treffen.

\begin{align*}
  \frac{2 \pi n_1}{t_1} = \frac{2 \pi n_2}{t_2}
  \Rightarrow
  n_1 = \underbrace{t_1/t_2}_{\in \R \setminus \Q} n_2
  \Rightarrow
  n_1, n_2 = 0
\end{align*}

Dazu brauchen wir aber auch noch, dass

\begin{align*}
  \P(N^c) =
  \P((N_1 \cap N_2)^c) =
  \P(N_1^c \cup N_2^c) \leq
  \P(N_1^c) + \P(N_2^c) = 0.
\end{align*}

Somit erhält man,

\begin{align*}
  \Exists N \in \S: \P(N^c) = 0, \:
  \Forall \omega \in N:
  X(\omega) = 0.
\end{align*}

\end{solution}
