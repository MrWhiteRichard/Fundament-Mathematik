\begin{exercise}

Hier könnte Ihre Werbung stehen!

\begin{itemize}
    \item[(a)] $F$ und $G$ seien (Wahrscheinlichkeits-) Verteilungsfunktionen, $d$ die Lévy-Prohorov-Metrik. Zeigen Sie für die verallgemeinerten Inversen

    \begin{equation*}
      d(F^{-1},G^{-1}) \leq d(F,G).
    \end{equation*}

    \item[(b)] Zeigen Sie: $F_n \longrightarrow F$ genau dann, wenn es auf einem geeigneten Wahrscheinlichkeitsraum Zufallsvariablen $X_n \sim F_n, X \sim F$ gibt mit $X_n \rightarrow X$ fast sicher (Darstellungssatz von Skorohod).
\end{itemize}

\end{exercise}

\begin{solution}

(a) Die verallgemeinerte Inverse wird definiert für $p \in (0, 1]$.

\begin{align*}
  F^{-1}(p) := \inf \Bbraces{x \in \R: p \leq F(x)}
\end{align*}

Wir benennen die folgenden Mengen.

\begin{align*}
  A  & := \Bbraces
  {
    \epsilon > 0:
    \Forall x \in \R:
    F(x - \epsilon) - \epsilon \leq
    G(x) \leq
    F(x + \epsilon) + \epsilon
  } \\
  B & := \Bbraces
  {
    \epsilon > 0:
    \Forall x \in \R:
    F^{-1}(x - \epsilon) - \epsilon \leq
    G^{-1}(x) \leq
    F^{-1}(x + \epsilon) + \epsilon
  }
\end{align*}

Für $\inf B \leq \inf A$, genügt es $A \subseteq B$ zu zeigen. Sei also $\epsilon \in A$, dann gilt $\epsilon > 0$ und $\Forall x \in \R:$

\begin{align*}
  F(x - \epsilon) - \epsilon \leq
  G(x) \leq
  F(x + \epsilon) + \epsilon.
\end{align*}

Wir wollen zeigen, dass

\begin{align*}
  F^{-1}(x - \epsilon) - \epsilon \leq
  G^{-1}(x) \leq
  F^{-1}(x + \epsilon) + \epsilon.
\end{align*}

Wir zeigen die erste Ungleichung; die zweite geht (wahrscheinlich) analog. Für die letzte Ungleichung beachte man $M \subseteq N$.

\begin{align*}
  \inf \Bbraces{y \in \R: x - \epsilon \leq F(y)} - \epsilon
  & = \inf \Bbraces{y - \epsilon \in \R: x - \epsilon \leq F(y)} \\
  & = \inf \Bbraces{z \in \R: x - \epsilon \leq F(z + \epsilon)} \\
  & = \inf \underbrace
      {
        \Bbraces{z \in \R: x \leq F(z + \epsilon) + \epsilon}
      }_{=: N} \\
  & \leq \inf \underbrace
      {
        \Bbraces{y \in \R: x \leq G(y)}
      }_{=: M}
\end{align*}

(b) Wir betrachten den Wahrscheinlichkeitsraum $(]0, 1[, \B(]0, 1[), \lambda)$ und bemerken, dass $F_n, F: \R \to [0, 1]$. Die verallgemeinerten Inversen $F_n^{-1}, F^{-1}$ sind monoton und somit messbar, also $X_n := F_n^{-1}, X := F^{-1}$ Zufallsvariablen. \\

$\Rightarrow$: Es ist zu zeigen, dass

\begin{align*}
  \P(X_n \xrightarrow{n \to \infty} X) = 1.
\end{align*}

Schwache Konvergenz ist bekanntlich äquivalent zur Konvergenz in der Lévy-Prohorov-Metrik.

\begin{align*}
  F_n \xrightarrow[\text{schwach}]{n \to \infty} F
  \Leftrightarrow
  d(F_n, F) \xrightarrow{n \to \infty} 0.
\end{align*}

Laut (a) wissen wir also bereits, dass

\begin{align*}
  d(X_n, X) =
  d(F_n^{-1}, F^{-1}) \leq
  d(F_n, F) \xrightarrow{n \to \infty} 0.
  \label{schwache_konvergenz}
\end{align*}

Eine weitere Äquivalenz zur schwach Konvergenz, erlaubt $\Forall \omega \in \mathcal{C}(X):$

\begin{align*}
  X_n(\omega) \xrightarrow{n \to \infty} X(\omega).
\end{align*}

Weil in jedem Intervall eine rationale Zahl ist und $|\Q| = \aleph_0$, gibt es nur abzählbar viele $F_n$- bzw. $F$-konstante Bereiche. $X_n, X$ haben genau dort ihre Sprungstellen, also ebenfalls nur abzählbar viele, also bilden sie eine $\lambda$-Nullmenge. Q.E.D. \\

$\Leftarrow$: $\lambda$ ist auf unserem Wahrscheinlichkeitsraum endlich, also folgt mit dem Satz von Egorov und Übung 11, Beispiel 6, dass

\begin{align*}
  X_n \xrightarrow[\fastsicher]{n \to \infty} X
  \Rightarrow
  X_n \xrightarrow[\text{in WSK}]{n \to \infty} X
  \Rightarrow
  F_n \xrightarrow[\text{schwach}]{n \to \infty} F.
\end{align*}

\end{solution}
