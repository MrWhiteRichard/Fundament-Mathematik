\begin{exercise}
Zeigen Sie: Wenn für ein $t \neq 0: \phi_X(t) = 1 $ gilt, dann nimmt $X$ mit Wahrscheinlichkeit 1
nur Werte der Form $2n\pi/t, n \in \mathbb{Z}$ an. Gilt $\phi(t_1) = \phi(t_2) = 1$
für zwei inkommensurable Werte $t_1$ und $t_2$ (d. h. $t_1/t_2$ ist irrational),
dann gilt $X = 0$ fast sicher.
\end{exercise}
\begin{solution}
  
\end{solution}
