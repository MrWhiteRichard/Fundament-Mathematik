\begin{lemma}
    Wenn $(X_n)_{n\in\mathbb{N}}$ eine Folge von Zufallsvariablen auf dem Maßraum $(\Omega,\mathfrak{S},\mathbb{P})$ mit $\forall n\in\mathbb{N}:X_n:\Omega\to\mathbb{Z}$ ist dann konvergiert $X_n$ in Verteilung genau dann, wenn für alle $k\in\mathbb{Z}$ der Grenzwert $p_k:=\lim_{n\to\infty}\mathbb{P}(X_n=k)$ existiert und $\sum_{k\in\mathbb{Z}}p_k=1$ gilt.
\end{lemma}
\begin{proof}[Beweis.]
    Wir zeigen zuerst die Hinrichtung, also $\Rightarrow$. Betrachte dazu
    \begin{align*}
        \lim_{n\to\infty} P(X_n = k) =
    \end{align*}
\end{proof}