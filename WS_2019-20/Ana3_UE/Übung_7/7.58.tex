% --------------------------------------------------------------------------------

\begin{exercise}

Bestimmen Sie das Flächenmaß der durch

\begin{align*}
    z(x, y) = (1 - x^2 - y^2),
    \quad
    x^2 + y^2 \leq 1
\end{align*}

definierten Fläche.

\end{exercise}

% --------------------------------------------------------------------------------

\begin{solution}

\begin{align*}
    z(x, y) = 1 - \norm[2]{(x, y)^\top}^2
\end{align*}

Die Fläche $A$ ist also $x$-$y$-rotationssymmetrisch.
Wir wollen Satz 4.2.15 anwenden.
Dazu, müssen wir eine Formel für den Radius $f(z)$ der Höhe $z$ aufstellen.
Setzen wir $y = 0$ und stellen den oberen Zusammenhang zwischen $x$ und $z$ so her, dass $(x, y, z)$ auf der Fläche liegt, so ist $x$ genau jener Radius.

\begin{align*}
    \implies
    z(x, 0) = 1 - x^2
    \implies
    x = \sqrt{1 - z} =: f(z)
    \implies
    f^\prime(z) = \frac{1}{2} \frac{1}{\sqrt{1 - z} (-1)} = -\frac{1}{2 \sqrt{1 - z}}
\end{align*}

Für $x^2 + y^2 \leq 1$, muss $z \in [0, 1]$ sein.

\begin{align*}
    \implies
    \mathcal{H}^2(A)
    & =
    2 \pi
    \Int[0][1]
    {
        |f(z)|
        \sqrt{1 + f^\prime(z)^2}
    } \\
    & =
    2 \pi
    \Int[0][1]
    {
        \sqrt{1 - z}
        \sqrt
        {
            1
            +
            \pbraces
            {
                -\frac{1}{2 \sqrt{1 - z}}
            }^2
        }
    }{z} \\
    & =
    2 \pi
    \Int[0][1]
    {
        \sqrt{1 - z}
        \sqrt
        {
            1
            +
            \frac{1}{4}
            (1 - z)
        }
    }{z} \\
    & =
    2 \pi
    \Int[0][1]
    {
        \sqrt{1 - z}
        \sqrt
        {
            1 - z + \frac{1}{4}
        }
    }{z} \\
    & =
    2 \pi
    \Int[\frac{1}{4}][\frac{5}{4}]{\sqrt u}{u} \\
    & =
    2 \pi
    \frac{2}{3}
    u^\frac{3}{2} \Big |_{u = \frac{1}{4}}^\frac{5}{4} \\
    & =
    \frac{4 \pi}{3}
    \pbraces
    {
        \sqrt{\frac{5}{4}}^3
        -
        \sqrt{\frac{1}{4}}^3
    } \\
    & =
    \frac{4 \pi}{3}
    \pbraces
    {
        \frac{\sqrt{5}}{2^3}
        -
        \frac{1}{2^3}
    } \\
    & =
    \frac{4 \pi}{3 \cdot 8}
    (5 \sqrt 5 - 1) \\
    & =
    \frac{(5 \sqrt 5 - 1) \pi}{6}
\end{align*}

\end{solution}

% --------------------------------------------------------------------------------
