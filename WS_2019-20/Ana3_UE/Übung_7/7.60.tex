% --------------------------------------------------------------------------------

\begin{exercise}

Berechnen Sie

\begin{align*}
    \Int[A]{\exp \pbraces{\frac{x_1 + x_2}{x_1 - x_2}}}{\lambda^2(x_1, x_2)},
    \quad
    A = \Bbraces{(x_1, x_2): x_1 \geq 0, x_2 \leq 0, x_2 + 1 \leq x_1 \leq x_2 + 2}
\end{align*}

und

\begin{align*}
    \Int[B]{\sin x_1}{\lambda^2(x_1, x_2)},
    \quad
    B = \Bbraces{(x_1, x_2): x_1 x_2 \geq 0, x_1^2 + x_2^2 \leq \pi}.
\end{align*}

\end{exercise}

% --------------------------------------------------------------------------------

\begin{solution}

\phantom{}

\begin{enumerate}[label = \arabic*.]

    \item Teil:
    
    Die ersten beiden Ungleichungen von $A$ schränken auf den $4$-ten Quadranten ein.
    Wenn man bei den letzten Beiden $\leq$ durch $=$ ersetzt, stehen die (linearen) Funktionen da, die untere bzw. obere Schranken für $x_2$ an der Stelle $x_1$ vorgeben.

    \begin{center}
        
        \begin{tikzpicture}
    
            \begin{scope}[xshift = -5 cm]
                
                \draw [->] (-1, 0) -- (3, 0) node [right] {$x_1$};
                \draw [->] (0, -3) -- (0, 1) node [right] {$x_2$};
    
                \foreach \x in {1, 2}
                    \filldraw (\x, 0) circle (1 pt) node [above] {$\x$};
    
                \foreach \y in {-1, -2}
                    \filldraw (0, \y) circle (1 pt) node [left]  {$\y$};
                
                \filldraw [pattern = dots]
                    (0, -2) --
                    (2,  0) --
                    (1,  0) --
                    (0, -1) --
                    cycle;
    
            \end{scope}
    
            \draw (0, 0) node {$\stackrel{!}{\mapsto}$};
    
            \begin{scope}[xshift = 3 cm]
                
                \draw [->] (-2, -2) -- (2,  2) node [above right] {$x_1$};
                \draw [->] (-1,  1) -- (2, -2) node [below right] {$x_2$};

                \filldraw (-1, -1) circle (1 pt) node [above left] {$-1$};
                \filldraw ( 1,  1) circle (1 pt) node [above left] {$ 1$};
                \filldraw ( 1, -1) circle (1 pt) node [below] {$1$};
        
                \filldraw [pattern = dots]
                    (0, -2) --
                    (2,  0) --
                    (1,  0) --
                    (0, -1) --
                    cycle;
    
            \end{scope}    
    
        \end{tikzpicture}

    \end{center}

    Wir wollen mit einer linearen Transformation das Koordinaten-System um $45^\circ$ drehen, um die $x_2$-Achse flippen und mit $\sqrt 2$ strecken.
    Dazu stellen wir folgende Bedingungen an die kanonischen Basisvektoren.

    \begin{align*}
        \begin{pmatrix}
            1 \\ 0
        \end{pmatrix}
        \stackrel{!}{=}
        \begin{pmatrix}
            1 \\ 1
        \end{pmatrix},
        \quad
        \begin{pmatrix}
            0 \\ 1
        \end{pmatrix}
        \stackrel{!}{=}
        \begin{pmatrix}
            1 \\ -1
        \end{pmatrix}
    \end{align*}

    Um die zugehörige Transformation zu erhalten, müssen wir nur die Zielvektoren in die Spalten der Transformations-Matrix schreiben.

    \begin{align*}
        T
        :=
        \begin{pmatrix}
            1 &  1 \\
            1 & -1
        \end{pmatrix}
        \in
        \GL_2(\R)
    \end{align*}

    Unsere Transformation $\varphi$ definieren wir also wie folgt.

    \begin{align*}
        \varphi:
        (x_1, x_2)
        \mapsto
        T
        \begin{pmatrix}
            x_1 \\ x_2
        \end{pmatrix}
        =
        \begin{pmatrix}
            x_1 + x_2 \\ x_1 - x_2
        \end{pmatrix}
        ~\text{Diffeomorphismus},
        \quad
        |\det \mathrm{d} \varphi| = |\det T| = |-1 - 1| = 2
    \end{align*}

    Wir wollen nun das Bild von $A$ unter der Transformation $\varphi$ schöner darstellen, damit wir leichter integrieren können.

    \begin{align*}
        \varphi(A)
        & =
        \Bbraces
        {
            \varphi(x_1, x_2):
            x_1 \geq 0,
            x_2 \leq 0,
            x_2 + 1 \leq x_1 \leq x_2 + 2} \\
        & =
        \Bbraces
        {
            \begin{pmatrix}
                x_1 + x_2 \\ x_1 - x_2
            \end{pmatrix}:
            x_1 \geq 0,
            x_2 \leq 0,
            1 \leq x_1 - x_2 \leq 2
        } \\
        & =
        \Bbraces
        {
            (y_1, y_2):
            \frac{y_1 + y_2}{2} \geq 0,
            \frac{y_1 - y_2}{2} \leq 0,
            1 \leq y_2 \leq 2
        } \\
        & \stackrel{!}{=}
        \Bbraces
        {
            (y_1, y_2):
            y_2 \in [1, 2],
            y_1 \in [-y_2, y_2]
        } \\
    \end{align*}

    Wir haben dabei die Substitution $y_1 := x_1 + x_2$ und $y_2 := x_1 - x_2$ verwendet.

    \begin{align*}
        \implies
        x_1 = \frac{y_1 + y_2}{2},
        \quad
        x_2 = \frac{y_1 - y_2}{2},
        \quad
        x_1 - x_2 = y_2
    \end{align*}

    Die letzte $=$ lässt sich damit wie folgt erklären.

    \begin{align*}
        \frac{y_1 + y_2}{2} \geq 0,
        \quad
        \frac{y_1 - y_2}{2} \leq 0
        \iff
        y_1 \geq -y_2,
        \quad
        y_1 \leq y_2
        \iff
        -y_2 \leq y_1 \leq y_2
    \end{align*}

    \begin{align*}
        \implies
        \Int[A]
        {
            \exp \frac{x_1 + x_2}{x_1 - x_2}
        }{(x_1, x_2)}
        & =
        \frac{1}{2}
        \Int[A]
        {
            \exp \varphi(x_1, x_2)
            |\det \mathrm{d} \varphi(x_1, x_2)|
            }{(x_1, x_2)} \\
        & \stackrel
        {
            \text{TRAFO}
        }{=}
        \frac{1}{2}
        \Int[\varphi(A)]
        {
            \exp \frac{y_1}{y_2}
        }{(y_1, y_2)} \\
        & =
        \frac{1}{2}
        \Int[1][2]
        {
            \Int[-y_2][y_2]
            {
                \exp \frac{y_1}{y_2}
            }{y_1}
        }{y_2} \\
        & =
        \frac{1}{2}
        \Int[1][2]
        {
            y_2 \exp \frac{y_1}{y_2} \Big |_{y_1 = -y_2}^{y_2}
        }{y_2} \\
        & =
        \frac{1}{2}
        \Int[1][2]
        {
            y_2 \pbraces{e - \frac{1}{e}}
        }{y_2} \\
        & =
        \frac{1}{2}
        \frac{1}{2}
        y_2^2 \Big |_{y_1 = 1}^2
        \pbraces{e - \frac{1}{e}} \\
        & =
        \frac{1}{4}
        (4 - 1)
        \pbraces{e - \frac{1}{e}} \\
        & =
        \frac{3}{4}
        \pbraces{e - \frac{1}{e}}
    \end{align*}

    \item Teil:
    
    Die erste Ungleichung von $B$ sagt, dass $x_1$ und $x_2$ dasselbe Vorzeichen haben sollen, schränkt also auf den $1$-ten und $3$-ten Quadranten ein.
    Die zweite Ungleichung schränkt auf die $\sqrt \pi$-Kugel um $0$ bzgl. $\norm[2]{\cdot}$ ein.

    \begin{center}
        \begin{tikzpicture}[scale = 2]
            
            \draw [->] (-2,  0) -- (2, 0) node [right] {$x_1$};
            \draw [->] ( 0, -2) -- (0, 2) node [above] {$x_2$};

            \draw (0, 0) circle (1);

            \filldraw [pattern = dots] (0, 0) -- ( 1, 0) arc (0  :90 :1) -- cycle;
            \filldraw [pattern = dots] (0, 0) -- (-1, 0) arc (180:270:1) -- cycle;

            \draw (1, 0) circle (1 pt) node [above right] {$(\sqrt \pi, 0)$};
            \draw (0, 1) circle (1 pt) node [above right] {$(0, \sqrt \pi)$};

        \end{tikzpicture}
    \end{center}

    Nachdem $\sin$ ungerade ist, und die Fläche ebenfalls, werden sich der obere rechte und untere linke Teil gegenseitig auslöschen.
    Das wollen wir noch im Detail durch-exerzieren.

    \includegraphicsboxed{Ana3/Ana3 - Beispiel 4.3.2.png}

    Wir wollen $B$ mit Polarkoordinaten parametrisieren.
    Das wurde netterweise in Beispiel 4.3.2 bereits gemacht.

    \begin{align*}
        \varphi:
        D
        :=
        (0, \sqrt \pi)
        \times
        \pbraces
        {
            \pbraces
            {
                0,
                \frac{\pi}{2}
            }
            \cup
            \pbraces
            {
                \pi,
                \frac{3 \pi}{2}
            }
        }
        \to
        B
        :
        (r, \theta)
        \mapsto
        r
        \begin{pmatrix}
            \cos \theta \\ \sin \theta
        \end{pmatrix}
        ~\text{Diffeomorphismus}
    \end{align*}

    \begin{align*}
        \Int[B]
        {
            \sin x_1
        }{(x_1, x_2)}
        & =
        \Int[D]
        {
            \sin \varphi_1(r, \theta)
            |\det \mathrm{d} \varphi(r, \theta)|
        }{(r, \theta)} \\
        & =
        \Int[0][\pi]
        {
            r
            \Int[0][\frac{\pi}{2}]
            {
                \sin(r \cos \theta)
            }{\theta}
        }{r}
        +
        \Int[0][\pi]
        {
            r
            \Int[\pi][\frac{3 \pi}{2}]
            {
                \sin(r \cos \theta)
            }{\theta}
        }{r} \\
        & =
        \Int[0][\pi]
        {
            r
            \Int[0][\frac{\pi}{2}]
            {
                \sin(r \cos \theta)
            }{\theta}
        }{r}
        +
        \Int[0][\pi]
        {
            r
            \Int[0][\frac{\pi}{2}]
            {
                \sin
                (
                    r
                    \underbrace
                    {
                        \cos(u + \pi))
                    }_{
                        -\cos u
                    }
            }{u}
        }{r} \\
        & =
        0
    \end{align*}

    Weil der $\sin$ (wie bereits erwähnt) ungerade ist, rutschte das $-$ durch und die Integrale löschen sich gegenseitig aus.

\end{enumerate}

\end{solution}

% --------------------------------------------------------------------------------
