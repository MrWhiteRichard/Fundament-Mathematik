% --------------------------------------------------------------------------------

\begin{exercise}

Berechnen Sie aus obiger Volumsformel und der Koflächenformel das Flächenmaß des Spindeltorus.

Hinw.:
Für festes $R$ sei $\T_r$ der Spindeltorus wie zuvor für $R, r$ definiert und $V_t$ das Volumen der von $\T_r$ berandeten Teilmenge des $\R^3$.

Für $R = r$ gilt $V = 2 \pi^2 R^3$.
Für $f(x, y, z) = (\sqrt{x^2 + y^2} - R)^2 + z^2$ gilt $|\nabla f| = 2 \sqrt f$.
Wenden Sie die Koflächenformel auf die Funkton $\1_\bbraces{R^2, r^2}(f(x))$ an.

\end{exercise}

% --------------------------------------------------------------------------------

\begin{solution}

\begin{align*}
    \Omega_{R, r}
    :=
    \Bbraces
    {
        (x, y, z) \in R^3:
        (R - \sqrt{x^2 - y^2})^2 + z^2 \leq r^2
    }
\end{align*}

Formal gilt $\lambda^3(\Omega_{R, t}) = V_t$.
Die obige Volumsformel gibt ja das Volumen des Spindeltorus an; also genau $V_R$.

\begin{align*}
    g := \frac{1}{|\nabla f|} (\1_\bbraces{R^2, r^2} \circ f)
    \implies
    \supp g = \Omega_{R, r} \setminus \Omega_{R, R} =: \Omega
\end{align*}

\begin{align*}
    V_R - V
    & =
    \lambda^3(\Omega) \\
    & =
    \Int[\Omega]
    {
        g |\nabla f|
    }{\lambda^3} \\
    & \stackrel
    {
        \text{KOFLFO}
    }{=}
    \Int[\R]
    {
        \Int[f^{-1}(t)]
        {
            g(y)
        }{\mathcal{H}^2(y)}
    }{t} \\
    & =
    \Int[R^2][r^2]
    {
        \Int[f^{-1}(t)]
        {
            \frac{1}{|\nabla f(y)|}
        }{\mathcal{H}^2(y)}
    }{t} \\
    & =
    \Int[R^2][r^2]
    {
        \Int[f^{-1}(t)]
        {
            \frac{1}{2 \sqrt{f(f^{-1}(t))}}
        }{\mathcal{H}^2(y)}
    }{t} \\
    & =
    \Int[R^2][r^2]
    {
        \frac{1}{2 \sqrt t}
        \Int[f^{-1}(t)]{}{\mathcal{H}^2}
    }{t} \\
    & \stackrel{!}{=}
    \Int[R^2][r^2]
    {
        \frac{1}{2 \sqrt t}
        \mathcal{H}^2(\T_{\sqrt{t}})
    }{t} \\
    & =
    \Int[R][r]
    {
        \mathcal{H}^2(\T_u)
    }{u}
\end{align*}

Dabei haben wir folgende Substitution verwendet.

\begin{align*}
    u = \sqrt t
    \implies
    \derivative[][u]{t} = \frac{1}{2 \sqrt t} \implies \mathrm{d} t = 2 \sqrt t \mathrm{d} u
\end{align*}

Für das \Quote{!}, also die räumliche Interpretation von $f$ konsultiere man die Lösung der vorigen Aufgabe 63.
Nun müssen wir nur noch links und rechts nach $r$ ableiten.

\begin{align*}
    \stackrel
    {
        \derivative{r}
    }{\mapsto}
    \mathcal{H}^2(\T_r)
    =
    4 \pi r
    \pbraces
    {
        R
        \arccos \pbraces{-\frac{R}{r}}
        +
        \sqrt{r^2 - R^2}
    }
\end{align*}

Das stimmt, weil wir das Maß der Oberfläche, in der Lösung der vorigen Aufgabe 63, bereits ausgerechnet haben.

\end{solution}

% --------------------------------------------------------------------------------
