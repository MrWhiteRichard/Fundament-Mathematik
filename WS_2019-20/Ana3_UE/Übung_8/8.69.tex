% -------------------------------------------------------------------------------- %

\begin{exercise}

Berechnen Sie das Flächenmaß eines Kegelmantels

\begin{align*}
    \Bbraces
    {
        x \in \R^3:
        0 < x_3 < \pbraces{R - \sqrt{x_1^2 + x_2^2}} h
    }
\end{align*}

mit der Flächenformel.

\end{exercise}

% -------------------------------------------------------------------------------- %

\begin{solution}

Sei $A_R$ der obere Kegelmantel.
Wir wollen Satz 4.2.15 anwenden, um $\mathcal{H}(A_R)$ auszurechnen.

\includegraphicsboxed{Ana3/Ana3 - Satz 4.2.15.png}

Um das $f$ für die Parametrisierung herauszufinden, setzen wir für $x_1$, $x_2$, und $x_3$ ein.

\begin{align*}
    &
    z
    \stackrel{!}{<}
    (R - \sqrt{(f(z) \cos \theta)^2 + (f(z) \sin \theta)^2}) h
    =
    (R - |f(z)|) h \\
    \impliedby &
    f(z) = R - \frac{z}{h} \geq 0
\end{align*}

Weil $x_1 = x_2 = 0$ sein kann, ist $z \in (a, b)$, mit $a := 0$ und $b := R h$.

\begin{align*}
    \implies
    \mathcal{H}(A_R)
    & =
    2 \pi
    \Int[a][b]
    {
        |f(z)| \sqrt{1 + f^\prime(z)^2}
    }{z} \\
    & =
    2 \pi
    \Int[0][R h]
    {
        \pbraces{R - \frac{z}{h}}
        \sqrt{1 + \pbraces{-\frac{1}{h}}^2}
    }{z} \\
    & =
    2 \pi \sqrt{1 + \frac{1}{h^2}}
    \pbraces
    {
        R
        \Int[0][R h]{}{z}
        -
        \frac{1}{h}
        \Int[0][R h]{z}{z}
    } \\
    & =
    2 \pi \sqrt{1 + \frac{1}{h^2}}
    \pbraces{R^2 h - \frac{1}{h} \frac{1}{2} R^2 h^2} \\
    & =
    2 \pi \sqrt{1 + \frac{1}{h^2}}
    \frac{R^2 h}{2} \\
    & =
    \pi R^2 \sqrt{h^2 + 1}
\end{align*}

\end{solution}

% -------------------------------------------------------------------------------- %
