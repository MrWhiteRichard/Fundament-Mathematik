% --------------------------------------------------------------------------------

\begin{exercise}

Berechnen Sie das Flächenmaß eines Kegelmantels

\begin{align*}
    \Bbraces
    {
        x \in \R^3:
        0 < x_3 = \pbraces{R - \sqrt{x_1^2 + x_2^2}} h
    }
\end{align*}

mit der Koflächenformel aus der Formel für das Volumen

\begin{align*}
    V_{R, h} = \pi R^2 h / 3
\end{align*}

des Kegels.

\end{exercise}

% --------------------------------------------------------------------------------

\begin{solution}

Sei $A_R$ der obere Kegelmantel.
Sei $\Omega$ die dadurch eingeschlossene Fläche.

\begin{align*}
    \Omega
    =
    \bigcup_{t \in (0, R)}
        A_t
\end{align*}

Sei nun $t \in (0, R)$.

\begin{align*}
    &
    A_t \stackrel{!}{=} f^{-1}(t) = \Bbraces{x \in \R: f(x) = t} \\
    \iff &
    f
    \begin{pmatrix}
        x_1 \\ x_2 \\ x_3 = \pbraces{t - \sqrt{x_1^2 + x_2^2}} h
    \end{pmatrix}
    =
    f(x)
    \stackrel{!}{=}
    t \\
    \iff &
    f(x) := \frac{x_3}{h} \sqrt{x_1^2 + x_2^2},
    \quad
    x \in \Omega \\
    \implies &
    |\nabla f(x)|^2
    =
    \pbraces
    {
        \frac{1}{2}
        \frac{1}{\sqrt{x_1^2 + x_2^2}}
        2 x_1
    }^2
    +
    \pbraces
    {
        \frac{1}{2}
        \frac{1}{\sqrt{x_1^2 + x_2^2}}
        2 x_2
    }^2
    +
    \frac{1}{h^2}
    =
    1 + \frac{1}{h^2} > 0
\end{align*}

Wir setzen nun $g := \frac{1}{|\nabla f|}$ und wenden die Koflächenformel an.

\begin{align*}
    \implies &
    \pi R^2 h / 3
    =
    V_{R, h}
    =
    \lambda^3(\Omega)
    =
    \Int[\Omega]{g |\nabla f|}{\lambda}
    \stackrel
    {
        \text{KOFLFO}
    }{=}
    \Int[\R]
    {
        \Int[f^{-1}(t)]
        {
            g
        }{\mathcal{H}^2}
    }{t}
    =
    \frac{1}{\sqrt{1 + \frac{1}{h^2}}}
    \Int[0][R]
    {
        \mathcal{H}^2(A_t)
    }{t} \\
    \implies &
    \frac{1}{3}
    \pi R^3 h
    \sqrt{1 + \frac{1}{h^2}}
    =
    \Int[0][R]
    {
        \mathcal{H}^2(A_t)
    }{t} \\
    \implies &
    \mathcal{H}^2(A_R)
    =
    \pi R^2 \sqrt{h^2 + 1}
\end{align*}

\end{solution}

% --------------------------------------------------------------------------------
