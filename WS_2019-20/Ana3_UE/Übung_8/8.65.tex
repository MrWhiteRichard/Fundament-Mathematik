% -------------------------------------------------------------------------------- %

\begin{exercise}

Es sei $G \subseteq \R^3$ der Graph der Funktion $f: [0, 1] \times [-1, 1] \to \R$ mit $f(x, y) = x^2 + y$.
Berechnen Sie

\begin{align*}
    \Int[G]{x}{\mathcal{H}^2}.
\end{align*}

\end{exercise}

% -------------------------------------------------------------------------------- %

\begin{solution}

\begin{align*}
    g := \id_{\R^3},
    \quad
    \Omega := (0, 1) \times (-1, 1)
\end{align*}

\includegraphicsboxed{Ana3/Ana3 - Satz 4.2.13.png}

Wir übernehmen sämtliche Bezeichnungen aus Satz 4.2.13.

\begin{align*}
    \implies
    \Int[G]{x}{\mathcal{H}^2(x)}
    & =
    \Int[\varphi(\Omega)]{g}{\mathcal{H}^2} \\
    & \stackrel
    {
        \mathrm{4.2.13}
    }{=}
    \Int[\Omega]
    {
        g(x, f(x))
        \sqrt{1 + |\nabla f(x)|^2}
    }{\lambda^2(x)} \\
    & =
    \Int[-1][1]
    {
        \Int[0][1]
        {
            \begin{pmatrix}
                x \\ y \\ f(x, y)
            \end{pmatrix}
            \sqrt{1 + (2 x)^2 + 1^2}
        }{x}
    }{y} \\
    & =
    \sqrt 2
    \Int[-1][1]
    {
        \Int[0][1]
        {
            \begin{pmatrix}
                x \\ y \\ x^2 + y
            \end{pmatrix}
            \sqrt{1 + 2 x^2}
        }{x}
    }{y}
\end{align*}

\begin{enumerate}[label = \arabic*.]

    \item Komponente:
    
    \begin{multline*}
        \sqrt 2
        \Int[-1][1]
        {
            \Int[0][1]
            {
                x \sqrt{1 + 2 x^2}
            }{x}
        }{y}
        =
        \sqrt 2
        \Int[-1][1]{}{y}
        \Int[0][1]
        {
            x \sqrt{1 + 2 x^2}
        }{x} \\
        =
        2 \sqrt 2
        \Int[1][3]
        {
            x \sqrt u \frac{1}{4 x}
        }{u}
        =
        2 \sqrt 2
        \frac{1}{4}
        \frac{2}{3}
        u^\frac{3}{2} \Big |_{u=1}^3
        =
        \frac{\sqrt 2}{3}
        \pbraces{\sqrt 3^3 - 1}
    \end{multline*}

    Dabei haben wir folgende Substitution verwendet.

    \begin{align*}
        u = 1 + 2 x^2
        \implies
        \derivative[][u]{x} = 4 x
        \implies
        \mathrm{d} x = \frac{1}{4 x} \mathrm{d} u
    \end{align*}

    \item Komponente:

    \item Komponente:

\end{enumerate}

\end{solution}

% -------------------------------------------------------------------------------- %
