% --------------------------------------------------------------------------------

\begin{exercise}

Definieren Sie auf einem metrischen Raum $(X, \metric)$ Mengenfunktionen $\mathcal{B}_\delta^s$ analog zum Hausdorffmaß aber mit der zusätzlichen Forderung dass die Mengen der Überdeckung Kugeln $B(x_i, r_i)$ um Punkte $x_i \in X$ und $2 r_i \leq \delta$ sind.
Zeigen Sie, dass diese Mengenfunktionen $\mathcal{B}_\delta^s$ äußere Maße sind, dass der Grenzwert
$\mathcal{B}^s(A) = \lim_{\delta \to 0} \mathcal{B}_\delta^s(A)$ für jede Teilmenge $A$ von $X$ existiert und $\mathcal{B}$ ein Borelmaß ist.

Geben sie pos. Schranken $a, b$ für die $a \mathcal{B}^s(A) \leq \mathcal{H}^s(A) \leq b \mathcal{B}^s(A)$ gilt.

\end{exercise}

% --------------------------------------------------------------------------------

\begin{solution}

\phantom{}

\begin{enumerate}[label = \arabic*.]

    \item Teil:
    
    \begin{align*}
        \mathcal{B}_\delta^s(A)
        :=
        \omega_s
        \inf
        \Bbraces
        {
            \sum_{i \in \N}
                r_i^s:
            \Forall i \in \N:
                2 r_i < \delta,
            \bigcup_{i \in \N}
                B(x_i, r_i)
            \supseteq
            A
        }
    \end{align*}

    \item Teil:
    
    Wir zeigen also die folgenden $4$ Eigenschaften.
    
    \begin{enumerate}[label = \arabic*.]

        \item Eigenschaft:
        
        \begin{align*}
            \mathcal{B}_\delta^s(\emptyset) = 0
        \end{align*}

        \item Eigenschaft (Nicht-Negativität):
        
        \begin{align*}
            \text{d.h.}~
            \mathcal{B}_\delta^s \geq 0
        \end{align*}

        \item Eigenschaft (Monotonie):
        
        \begin{align*}
            \text{d.h.}~
            \Forall A, B \subseteq X:
                A \subseteq B
                \implies
                \mathcal{B}_\delta^s(A) \leq \mathcal{B}_\delta^s(B)
        \end{align*}

        \item Eigenschaft ($\sigma$-Subadditivität):
        
        \begin{align*}
            \text{d.h.}~
            \Forall (A_i)_{i \in \N} \in X^\N, ~\text{disjunkt}:
                \mathcal{B}_\delta^s
                \pbraces
                {
                    \bigcup_{i \in \N}
                        A_i
                }
                \leq
                \sum_{i \in \N}
                    \mathcal{B}_\delta^s(A_i)
        \end{align*}

        Sei $(B(x_{i, j}, r_{i, j}))_{j \in \N}$ eine Überdeckung von $A_i$, für $i \in \N$, d.h.

        \begin{align*}
            \Forall i \in \N:
                \bigcup_{j \in \N}
                    B(x_{i, j}, r_{i, j})
                \supseteq
                A_i.
        \end{align*}

        \begin{align*}
            \implies
            \mathrm{lhs}
            \leq
            \mathcal{B}_\delta^s
            \pbraces
            {
                \bigcup_{i, j \in \N}
                    B(x_{i, j}, r_{i, j})
            }
            \leq
            \omega_s
            \sum_{i, j \in \N}
                r_{i, j}^2
        \end{align*}

        Weil die Überdeckung beliebig war, gilt auch die Behauptung, wobei sie für das $\inf$ verwendet werden.

    \end{enumerate}

    \item Teil:
    
    Der Grenzwert existsiert, weil das Netz monoton und beschränkt ist.

    \item Teil:
    
    \includegraphicsboxed{Ana3/Ana3 - Satz 4.1.7.png}

    Seien also $A, B \subseteq X$ mit $\dist(A, B) > 0$.

    \begin{enumerate}

        \item Ungleichung (\enquote{$\leq$}):
        
        Trivial!

        \item Ungleichung (\enquote{$\geq$}):

        Sei $(B(x_i, r_i))_{i \in \N}$ eine $\dist(A, B)$-Überdeckung von $A \cup B$, d.h.

        \begin{align*}
            \bigcup_{i \in \N}
                B(x_i, r_i)
            \supseteq
            A \cup B,
            \quad
            \Forall i \in \N:
                2 r_i < \dist(A, B).
        \end{align*}

        Betrachte jene Kugeln, die mit $A$ bzw. $B$ disjunkt sind.

        \begin{align*}
            I_M
            :=
            \Bbraces
            {
                i \in \N:
                B(x_i, r_i) \cap M = \emptyset
            },
            \quad
            M = A, B
        \end{align*}

        Der Durchmesser aller Kuglen ist kleiner als dem Abstand von $A$ zu $B$.
        Keine Kugel kann also mit $A$ und $B$ gleichzeitig nicht-leeren Schnitt haben.

        \begin{align*}
            \iff
            &
            I_A \cap I_B = \emptyset \\
            \implies
            &
            \bigcup_{i \in I_A}
                B(x_i, r_i)
            \supseteq
            B,
            \quad
            \bigcup_{i \in I_B}
                B(x_i, r_i)
            \supseteq
            A, \\
            &
            I_A \setminus I_B = I_A,
            \quad
            I_B \setminus I_A = I_B, \\
            &
            \sum_{i \in \N}
                r_i^s
            =
            \underbrace
            {
                \sum_{i \in \N \setminus (I_A \cup I_B)}
                r_i^s
            }_{\geq 0}
            +
            \sum_{i \in I_A \setminus I_B}
                r_i^s
            +
            \sum_{i \in I_B \setminus I_A}
                r_i^s
            +
            \underbrace
            {
                \sum_{i \in I_A \cap I_B}
                r_i^s                
            }_0
            \geq
            \sum_{i \in I_A}
                r_i^s
            +
            \sum_{i \in I_B}
                r_i^s            
        \end{align*}



    \end{enumerate}

    \item Teil:
    
    Seien $A \subseteq X$ und $\delta, \varepsilon > 0$.

    \begin{multline*}
        \mathcal{U}_\text{Kugeln}
        :=
        \Bbraces
        {
            (B(x_i, r_i))_{i \in \N}:
            \Forall i \in \N:
                x_i \in X,
                2 r_i < \delta,
            \bigcup_{i \in \N}
                B(x_i, r_i)
            \supseteq
            A
        } \\
        \subseteq
        \Bbraces
        {
            (C_i)_{i \in \N}:
            \Forall i \in \N:
                \diam C_i < \delta,
            \bigcup_{i \in \N}:
                C_i
            \supseteq
            A
        }
        =:
        \mathcal{U}_\text{allgemein}
    \end{multline*}

    \begin{enumerate}[label = \arabic*.]

        \item Ungleichung ($a \mathcal{B}^s(A) \leq \mathcal{H}^s(A)$):
        
        Sei $(C_i)_{i \in \N} \in \mathcal{U}_\text{allgemein}$, sodass die letzte Ungleichung (die mit dem \enquote{!}) gilt.

        \begin{align*}
            x_i \in C_i,
            \quad
            r_i := \diam C_i,
            \quad
            i \in \N
        \end{align*}

        \begin{align*}
            \implies
            \frac{1}{\omega_s}
            \mathcal{B}_{2 \delta}^s(A)
            \leq
            \sum_{i \in \N}
                (\diam C_i)^s
            =
            2^s
            \sum_{i \in \N}
                \pbraces
                {
                    \frac{\diam C_i}{2}
                }^s
            \stackrel{!}{\leq}
            \frac{2^s}{\omega_s}
            \mathcal{H}_\delta^s(A) + \varepsilon
        \end{align*}

        \begin{align*}
            \rightsquigarrow
            a := \frac{1}{2^s}
        \end{align*}

        \item Ungleichung ($\mathcal{H}^s(A) \leq b \mathcal{B}^s(A)$):
        
        Auf der linken Seite wird das Infimum von einer größeren Menge gebildet als bei der rechten.
    
        \begin{align*}
            \rightsquigarrow
            b := 1
        \end{align*}

    \end{enumerate}

\end{enumerate}

\end{solution}

% --------------------------------------------------------------------------------
