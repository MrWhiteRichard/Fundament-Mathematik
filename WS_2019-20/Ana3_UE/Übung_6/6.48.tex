% --------------------------------------------------------------------------------

\begin{exercise}

Zeigen Sie:
Ist $f \in L^1(\R)$ ungerade, so existiert $\Int[0][t]{\frac{\hat f(\xi)}{\xi}}{\xi}$ für $t > 0$ und es gilt

\begin{align*}
    \Forall t > 0:
    \abs
    {
        \Int[0][t]
        {
            \frac{\hat f(\xi)}{\xi}
        }{\xi}
    }
    \leq
    \sqrt \frac{2}{\pi}
    C
    \norm[1]{f},
    \quad
    C
    =
    \sup
    \Bbraces
    {
        \abs
        {
            \Int[0][t]
            {
                \frac{\sin \xi}{\xi}
            }{\xi}:
            t > 0
        }
    }
    <
    \infty.
\end{align*}

Zeigen Sie damit, dass die Fouriertransformation nicht surjektiv von $L^1(\R)$ nach $C_0(\R)$ abbildet.

Hinw.:
Zeigen Sie, dass $g$ nicht auf $\R_+$ integrierbar ist für $g(x) = \frac{1}{x \ln x}$, $x > 2$.

\end{exercise}

% --------------------------------------------------------------------------------

\begin{solution}

\phantom{}

\begin{enumerate}[label = \arabic*.]

    \item Teil:

    \begin{align*}
        \mathrm{lhs}
        & =
        \abs
        {
            \Int[0][t]
            {
                \frac{1}{\xi}
                \frac{1}{\sqrt{2 \pi}}
                \Int[\R]
                {
                    e^{-i \xi x}
                    f(x)
                }{x}
            }{\xi}
        } \\
        & =
        \frac{1}{\sqrt{2 \pi}}
        \Bigg |
            \Int[0][t]
            {
                \frac{1}{\xi}
                \Bigg (
                    \underbrace
                    {
                        \Int[\R]
                        {
                            \cos(\xi x)
                            f(x)
                        }{x}    
                    }_0
                    -
                    i
                    \Int[\R]
                    {
                        \sin(\xi x)
                        f(x)
                    }{x}
                \Bigg )
            }{\xi}
        \Bigg | \\
        & =
        \frac{1}{\sqrt{2 \pi}}
        \abs
        {
            \Int[0][t]
            {
                2
                \Int[0][\infty]
                {
                    \frac{\sin(\xi x)}{\xi}
                    f(x)
                }{x}
            }{\xi}
        } \\
        & \stackrel
        {
            \text{Fubini}
        }{=}
        \frac{2}{\sqrt{2 \pi}}
        \abs
        {
            \Int[0][\infty]
            {
                \Int[0][t]
                {
                    \frac{\sin(\xi x)}{\xi}
                }{x}
                f(x)
            }{\xi}
        } \\
        & =
        \sqrt \frac{2}{\pi}
        \abs
        {
            \Int[0][\infty]
            {
                \Int[0][t]
                {
                    \frac{\sin(\xi x)}{\xi}
                }{\xi}
                f(x)
            }{x}    
        } \\
        & \leq
        \sqrt \frac{2}{\pi}
        \Int[0][\infty]
        {
            \abs
            {
                \Int[0][t x]
                {
                    \frac{\sin \eta}{\eta}
                }{\eta}
            }
            |f(x)|
        }{x} \\
        & \leq
        \mathrm{rhs}
    \end{align*}

    Dabei haben wir folgende Substitution verwendet.

    \begin{align*}
        \eta = \xi x
        & \implies
        \begin{cases}
            \derivative[][\eta]{\xi} = x \implies \mathrm{d} \xi = \frac{1}{x} \mathrm{d} \eta \\
            \xi = \frac{\eta}{x}
        \end{cases} \\
        & \implies
        \Int[0][t]
        {
            \frac{\sin(\xi x)}{\xi}
        }{\xi}
        =
        \Int[0][t x]
        {
            \frac{\sin \eta}{\frac{\eta}{x}}
            \frac{1}{x}
        }{\eta}
        =
        \Int[0][t x]
        {
            \frac{\sin \eta}{\eta}
        }{\eta}
    \end{align*}

    Den Satz von Fubini dürfen wir anwenden, weil der Integrand messbar und integrierbar ist.

    \begin{multline*}
        \abs
        {
            \Int[0][t]
            {
                \Int[0][\infty]
                {
                    \frac{\sin(\xi x)}{\xi}
                    f(x)
                }{x}
            }{\xi}
        }
        =
        \abs
        {
            \Int[0][t]
            {
                \Int[0][\infty]
                {
                    \sin(\xi^2 u)
                    f(\xi u)
                }{u}
            }{\xi}
        } \\
        \leq
        \Int[0][t]
        {
            \Int[0][\infty]
            {
                \underbrace
                {
                    |\sin(\xi^2 u)|
                }_{\leq 1}
                |f(\xi u)|
            }{u}
        }{\xi}
        \leq
        \norm[1]{f} t
        <
        \infty
    \end{multline*}

    Dabei haben wir folgende Substitution verwendet.

    \begin{align*}
        u = \frac{x}{\xi}
        \implies
        \begin{cases}
            \derivative[][u]{x} = \frac{1}{\xi} \implies \mathrm{d} x = \xi \mathrm{d} u \\
            x = \xi u
        \end{cases}
    \end{align*}

    \item Teil:

    \includegraphicsboxed{Ana3/Ana3 - Satz 3.2.1 (Riemann-Lebesgue Lemma).png}

    \begin{align*}
        C_0(\R)
        =
        \Bbraces
        {
            f \in C(\R):
            \Forall x \in \R:
            f(x) \xrightarrow{x \to \infty} 0
        }
    \end{align*}

    \includegraphicsboxed{Ana3/Ana3 - Korollar 3.2.4.png}

    \begin{align*}
        h(x)
        :=
        -
        \frac{1}{\ln{|x|}} \1_{(-\infty, -2)}(x)
        +
        \frac{1}{\ln 2}    \1_{[-2, 2]}(x)
        +
        \frac{1}{\ln x}    \1_{(2, \infty)}(x),
        \quad
        x \in \R
    \end{align*}

    Angenommen, die Fouriertransformation $^\wedge$ wäre surjektiv auf $L^1$ (also sogra bijektiv).
    Dann hätte auch $h$ ein Urbild $f \in L^1$, d.h. $f^\wedge = h$.
    Nun ist $h$ ungerade, laut Konstruktion.

    \begin{align*}
        h(\xi)
        =
        f^\wedge(\xi)
        =
        \frac{1}{\sqrt{2 \pi}}
        \Int[\R]
        {
            e^{-i \xi x}
            f(\xi)
        }{x}
    \end{align*}

    \begin{multline*}
        -h(-\xi)
        =
        f^\wedge(-\xi)
        =
        f^\vee(\xi)
        =
        \frac{1}{\sqrt{2 \pi}}
        \Int[\R]
        {
            e^{i \xi x}
            f(\xi)
        }{x} \\
        =
        -\frac{1}{\sqrt{2 \pi}}
        \Int[\R]
        {
            e^{i \xi (-u)}
            f(\xi)
        }{u}
        =
        \frac{1}{\sqrt{2 \pi}}
        \Int[\R]
        {
            e^{-i \xi u}
            f(-\xi)
        }{u}
    \end{multline*}

    Wir haben dabei folgende Substitution verwendet.

    \begin{align*}
        u = -x
        \implies
        \begin{cases}
            \derivative[][u]{x} = -1 \implies \mathrm{d} x = -\mathrm{d} u \\
            x = -u
        \end{cases}
    \end{align*}

    \begin{align*}
        \implies
        0
        =
        h(\xi) + h(-\xi)
        =
        \Int[\R]
        {
            e^{-i \xi x}
            (
                f(\xi)
                +
                f(-\xi)
            )
        }
        =
        \widehat{f(\xi) + f(-\xi)}
        \stackrel
        {
            \mathrm{3.2.4}
        }{\implies}
        f(\xi) + f(-\xi) = 0
        \implies
        f(\xi) = -f(-\xi)
    \end{align*}

    $f$ ist also auch ungerade.
    Wir können somit den $1$-ten Teil darauf anwenden.

    \begin{align*}
        \sqrt \frac{2}{\pi}
        C
        \norm[1]{f}
        \geq
        \Int[0][\infty]
        {
            \frac{f^\wedge(\xi)}{\xi}
        }{\xi}
        =
        \Int[0][\infty]
        {
            \frac{h(\xi)}{\xi}
        }{\xi}
        =
        \Int[0][2]
        {
            \frac{1}{\xi}
            \frac{1}{\ln 2}
        }{\xi}
        +
        \Int[2][\infty]
        {
            \frac{1}{\xi}
            \frac{1}{\ln \xi}
        }{\xi}
        =
        \infty
    \end{align*}

    Also ist $f \not \in L^1$; Widerspruch!

\end{enumerate}

\end{solution}

% --------------------------------------------------------------------------------
