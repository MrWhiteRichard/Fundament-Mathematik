% --------------------------------------------------------------------------------

\begin{exercise}

Berechnen Sie den Fluss

\begin{align*}
    \Int[M]
    {
        \mathbf F^t \mathbf n
    }{
        \mathcal H^2
    }
\end{align*}

des Vektorfeldes

\begin{align*}
    \mathbf F(x, y, z)
    =
    (x y, y z, x^2 + y^2)
\end{align*}

durch den Zylindermantel $x^2 + y^2 = 1$, $0 < z < 1$ direkt und mithilfe des Gauß'schen Integralsatzes.

\end{exercise}

% --------------------------------------------------------------------------------

\begin{solution}

\phantom{}

\begin{enumerate}

    \item Teil (\enquote{direkt}):
    
    Wir parametrisieren den Mantel mit Zylinderkoordinaten, d.h. vermöge

    \begin{align*}
        \psi:
            D := [0, 2 \pi) \times (0, 1) \to M,
            (\varphi, z) \mapsto (\cos \varphi, \sin \varphi, z).
    \end{align*}

    Das Normalvektorfeld lautet

    \begin{align*}
        \mathbf n:
            M \to \R^3,
            (x, y, z) \mapsto (x, y, 0).
    \end{align*}

    Wir wollen wieder Satz 4.2.15 anwenden.
    Also berechnen wir zunächst

    \begin{align*}
        \vbraces
        {
            \derivative[][\psi]{\varphi}
            \times
            \derivative[][\psi]{z}
        }
        =
        \vbraces
        {
            \begin{pmatrix}
                -\sin \varphi \\
                 \cos \varphi \\
                 0
            \end{pmatrix}
            \times
            \begin{pmatrix}
                0 \\ 0 \\ 1
            \end{pmatrix}
        }
        =
        \vbraces
        {
            \begin{pmatrix}
                \cos \varphi \\
                \sin \varphi \\
                0
            \end{pmatrix}
        }
        =
        \sin^2 \varphi + \cos^2 \varphi
        =
        1,
    \end{align*}

    und zu guter Letzt

    \begin{align*}
        \Int[M]
        {
            \mathbf F^t \mathbf n
        }{
            \mathcal H^2
        }
        & =
        \Int[\psi(D)]
        {
            x^2 y + y^2 z
        }{
            \mathcal H^2(x, y, z)
        } \\
        & =
        \Int[D]
        {
            (
                \cos^2 \varphi \sin \varphi
                +
                \sin^2 \varphi z
            )
            \underbrace
            {
                \vbraces
                {
                    \derivative[][\psi]{\varphi}
                    \times
                    \derivative[][\psi]{z}
                }
            }_1
        }{
            \lambda^2(\varphi, z)
        } \\
        & =
        \Int[0][2 \pi]
        {
            \Int[0][1]
            {
                \cos^2 \varphi \sin \varphi
                +
                \sin^2 \varphi z
            }{z}
        }{\varphi} \\
        & =
        \underbrace
        {
            \Int[0][1]{}{z}
        }_1
        \underbrace
        {
            \Int[0][2 \pi]
            {
                \cos^2 \varphi \sin \varphi
            }{\varphi}
        }_0
        +
        \underbrace
        {
            \Int[0][1]{z}{\varphi}
        }_\frac{1}{2}
        \underbrace
        {
            \Int[0][2 \pi]
            {
                \sin^2 \varphi
            }{z}
        }_\pi \\
        & =
        \frac{\pi}{2}.
    \end{align*}

    Dabei haben wir verwendet, dass $\cos^2 \sin$ $2 \pi$-periodisch und ungerade ist und damit

    \begin{multline*}
        \Int[0][2 \pi]
        {
            \cos^2 \varphi \sin \varphi
        }{\varphi}
        =
        \Int[0][\pi]
        {
            \cos^2 \varphi \sin \varphi
        }{\varphi}
        +
        \Int[\pi][2 \pi]
        {
            \cos^2 \varphi \sin \varphi
        }{\varphi} \\
        =
        \Int[0][\pi]
        {
            \cos^2 \varphi \sin \varphi
        }{\varphi}
        +
        \Int[-\pi][0]
        {
            \cos^2 \varphi \sin \varphi
        }{\varphi}
        =
        0.
    \end{multline*}

    Letzteres Integral sollte allgemein bekannt sein.

    \item Teil (\enquote{mithilfe des Gauß'schen Integralsatzes}):
    
    Sei $Z$ der gesamte Zylinder.
    Wir parametrisieren den Zylinder mit Zylinderkoordinaten, d.h. vermöge

    \begin{align*}
        \psi:
            D := [0, 2 \pi) \times (0, 1)^2 \to Z,
            (\varphi, r, z) \mapsto (r \cos \varphi, r \sin \varphi, z).
    \end{align*}

    Seien $A$ und $B$ der obere bzw. untere Deckel von $Z$.
    Das Normalvektorfeld können wir darauf fortsetzen durch

    \begin{align*}
        \mathbf n(v) =  e_3,
        \quad
        v \in A,
        \quad
        \mathbf n(v) = -e_3,
        \quad
        v \in B.
    \end{align*}

    \includegraphicsboxed{Ana3/Ana3 - Satz 5.3.5 (Integralsatz v. Gauß).png}

    Wir verwenden Satz 5.3.5 (Integralsatz v. Gauß) und berechnen zu guter Letzt

    \begin{align*}
        \Int[M]
        {
            \mathbf F^t \mathbf n
        }{
            \mathcal H^2
        }
        & =
        \Int[\partial Z]
        {
            \mathbf F^t \mathbf n
        }{
            \mathcal H^2
        }
        -
        \underbrace
        {
            \pbraces
            {
                \Int[A]
                {
                    \mathbf F^t \mathbf n
                }{
                    \mathcal H^2
                }
                +
                \Int[B]
                {
                    \mathbf F^t \mathbf n
                }{
                    \mathcal H^2
                }
            }
        }_0 \\
        & \stackrel
        {
            \text{Gauß}
        }{=}
        \Int[Z]
        {
            \Div \mathbf F
        }{
            \lambda^3
        } \\
        & =
        \Int[\psi(D)]
        {
            y + z
        }{(x, y, z)} \\
        & \stackrel
        {
            \text{TRAFO}
        }{=}
        \Int[D]
        {
            (r \sin \varphi + z) r
        }{(z, r, \varphi)} \\
        & =
        \Int[0][2 \pi]
        {
            \Int[0][1]
            {
                \Int[0][1]
                {
                    (r \sin \varphi + z) r
                }{z}
            }{r}
        }{\varphi} \\
        & =
        \underbrace
        {
            \Int[0][1]{}{z}
        }_1
        \underbrace
        {
            \Int[0][1]{r^2}{r}
        }_\frac{1}{3}
        \underbrace
        {
            \Int[0][2 \pi]
            {
                \sin \varphi
            }{\varphi}
        }_0
        +
        \underbrace
        {
            \Int[0][1]{z}{z}
        }_\frac{1}{2}
        \underbrace
        {
            \Int[0][1]{r}{r}
        }_\frac{1}{2}
        \underbrace
        {
            \Int[0][2 \pi]{}{\varphi}
        }_{2 \pi} \\
        & =
        \frac{\pi}{2}.
    \end{align*}

\end{enumerate}

\end{solution}

% --------------------------------------------------------------------------------
