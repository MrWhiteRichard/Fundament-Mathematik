% --------------------------------------------------------------------------------

\begin{exercise}

Verifizieren Sie den Integralsatz von Gauß dür das Gebiet $\Omega = \Bbraces{(x, y, z): 0 < z < 1 - x^2 - y^2}$ und die Funktion $f(x, y, z) = (x, x + y, x^2 + z)^t$.

\end{exercise}

% --------------------------------------------------------------------------------

\begin{solution}

Wir bestimmen zunächst das Normalvektorfeld.
Dazu zerlegen wir $\partial \Omega$ in

\begin{align*}
    A := \Bbraces{(x, y, z): 0 = z < 1 - x^2 - y^2},
    \quad
    \text{und}
    \quad
    B := \Bbraces{(x, y, z): 0 < z = 1 - x^2 - y^2}.
\end{align*}

Nun sind $A$ und $B$ die Nullstellenmengen der Funktionen $F_A$ und $F_B$.

\begin{gather*}
    F_A(x, y, z) := z,
    \quad
    \nabla F_A(x, y, z) = (0, 0, 1)^t, \\
    F_B(x, y, z) := 1 - x^2 - y^2 - z,
    \quad
    \nabla F_B(x, y, z) = (-2 x, -2 y, -1)^t, \\
    (x, y, z) \in \partial \Omega
\end{gather*}

\includegraphicsboxed{Ana3/Ana3 - Satz 5.1.9.png}

Laut Satz 5.1.9, lautet das Normalvektorfeld daher

\begin{align*}
    \nu:
        \partial \Omega \to \R^3,
        (x, y, z)
        \mapsto
        \begin{cases}
            (0, 0, -1)^t,                                  & z = 0, \\
            \sqrt{4 (x^2 + y^2) + 1}^{-1} (2 x, 2 y, 1)^t, & \text{sonst}.
        \end{cases}
\end{align*}

Das $-$ rührt daher, dass das Normalvektorfeld nach Außen zeigen muss.
Wir parametrisieren das Gebiet mit Zylinderkoordinaten, d.h. vermöge

\begin{align*}
    \psi:
        D := \Bbraces{(r, \varphi, z): 0 \leq \varphi < 2 \pi, 0 < z < 1, 0 \leq r < \sqrt{1 - z}} \to \Omega,
        (r, \varphi, z) \mapsto (r \cos \varphi, r \sin \varphi, z).
\end{align*}

\begin{align*}
    \implies
    \Int[\partial \Omega]
    {
        f^t \nu
    }{
        \mathcal H^2
    }
    & \stackrel
    {
        \text{Gauß}
    }{=}
    \Int[\Omega]
    {
        \Div f
    }{\lambda^3} \\
    & =
    \Int[\psi(D)]
    {
        \derivative[][f_1]{x}
        +
        \derivative[][f_2]{y}
        +
        \derivative[][f_3]{z}
    }{
        \lambda^3
    } \\
    & =
    \Int[\psi(D)]
    {
        3
    }{
        \lambda^3
    } \\
    & \stackrel
    {
        \text{TRAFO}
    }{=}
    3
    \Int[D]{r}{(r, z, \varphi)} \\
    & =
    3
    \Int[0][2 \pi]
    {
        \Int[0][1]
        {
            \Int[0][\sqrt{1 - z}]
            {
                r
            }{r}
        }{z}
    }{\varphi} \\
    & =
    3
    \Int[0][2 \pi] {}{\varphi}
    \Int[0][1]
    {
        \frac{1}{2}
        \sqrt{1 - z}^2
    }{z} \\
    & =
    3 \pi
    \pbraces
    {
        \Int[0][1]{}{z}
        -
        \Int[0][1]{z}{z}
    } \\
    & =
    3 \pi
    \pbraces
    {
        1 - \frac{1}{2}
    } \\
    & =
    \frac{3 \pi}{2}
\end{align*}

\end{solution}

% --------------------------------------------------------------------------------
