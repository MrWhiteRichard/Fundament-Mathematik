% --------------------------------------------------------------------------------

\begin{exercise}

Berechnen Sie das Wegintegral

\begin{align*}
    \int_{\partial B}
            (x - y^3)
            ~\mathrm d x
        +
        y^3
        ~\mathrm d y
\end{align*}

direkt und über den Stoke'schen Integralsatz wobei $B$ der offene Einheitskreis in $\R^2$ ist.

(Das Wegintegral
$\Int[\gamma]
{
    \begin{pmatrix}
        f_1(\mathbf z) \\
        f_2(\mathbf z)
    \end{pmatrix}
}{\mathbf z}$
im $\R^2$ wird hier als
$\int_\gamma
        f_1(\mathbf z)
        ~\mathrm d x
        +
    f_2(\mathbf z)
    ~\mathrm d y$
geschrieben).

\end{exercise}

% --------------------------------------------------------------------------------

\begin{solution}

Sei $\mathbf f$ unser Vektorfeld, d.h.

\begin{align*}
    \mathbf f:
        \partial B \to \R^2,
        (x, y)
        \mapsto
        \begin{pmatrix}
            x - y^3 \\ y^3
        \end{pmatrix}.
\end{align*}

Wir parametrisieren den Rand des offenen Einheitskreises vermöge

\begin{align*}
    \gamma:
        [0, 2 \pi) \to \partial B,
        t
        \mapsto
        \begin{pmatrix}
            \cos t \\ \sin t
        \end{pmatrix}.
\end{align*}

\begin{enumerate}[label = \arabic*.]

    \item Teil (\enquote{direkt}):
    
    \includegraphicsboxed{Ana1&2/Ana1&2 - 11.2.5 Satz.png}
    
    Wir verwenden Satz 11.2.5 und berechnen

    \begin{align*}
        \int_{\partial B}
        (x - y^3)
        ~\mathrm d x
    +
    y^3
    ~\mathrm d y
    & =
    \Int[\gamma]
    {
        \mathbf f(x, y)
    }{(x, y)} \\
    & =
    \Int[0][2 \pi]
    {
        \mathbf f(\gamma(t)) \gamma^\prime(t)
    }{t} \\
    & =
    \Int[0][2 \pi]
    {
        (\cos t - \sin^3 t, \sin^3 t)
        \begin{pmatrix}
            -\sin t \\
             \cos t
        \end{pmatrix}
        }{t} \\
        & =
        -
        \underbrace
        {
            \Int[0][2 \pi]
            {
                \sin t \cos t
            }{t}
        }_0
        +
        \Int[0][2 \pi]
        {
            \sin^4 t
        }{t}
        +
        \underbrace
        {
            \Int[0][2 \pi]
            {
                \cos t \sin^3 t
            }{t}
        }_0 \\
        & =
        \frac{3 \pi}{4}.
    \end{align*}

    \item Teil (\enquote{über den den Stoke'schen Integralsatz}):
    
    \includegraphicsgraphicsboxed
    {Ana3/Ana3 - Satz 5.4.1.1 (Integralsatz von Stokes für die Ebene).png}
    {Ana3/Ana3 - Satz 5.4.1.2 (Integralsatz von Stokes für die Ebene).png}

    Wir verwenden Satz 5.4.1 und Polarkoordinaten und berechnen

    \begin{align*}
        \int_{\partial B}
                (x - y^3)
                ~\mathrm d x
            +
            y^3
            ~\mathrm d y
        & \stackrel
        {
            \text{Stokes ?}
        }{=}
        \Int[B]
        {
            \rot \mathbf f
        }{\lambda^2} \\
        & =
        \Int[B]
        {
            \pderivative[][\mathbf f_2]{x}
            -
            \pderivative[][\mathbf f_1]{y}
        }{\lambda^2} \\
        & =
        \Int[B]
        {
            3 y^2
        }{\lambda^2(x, y)} \\
        & \stackrel
        {
            \text{TRAFO}
        }{=}
        3
        \Int[0][2 \pi]
        {
            \Int[0][1]
            {
                (r \sin \varphi)^2 r
            }{r}
        }{\varphi} \\
        & =
        3
        \underbrace
        {
            \Int[0][1]
            {
                r^3
            }{r}
        }_\frac{1}{4}
        \underbrace
        {
            \Int[0][2 \pi]
            {
                \sin^2 \varphi
            }{\varphi}
        }_\pi \\
        & =
        \frac{3 \pi}{4}.
    \end{align*}

\end{enumerate}

\end{solution}

% --------------------------------------------------------------------------------
