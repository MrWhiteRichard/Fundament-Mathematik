% --------------------------------------------------------------------------------

\begin{exercise}

Berechnen Sie den Fluss des Vektorfeldes

\begin{align*}
    \mathbf F = (x z, x y, z^2 - x)
\end{align*}

durch den Rand des Gebietes

\begin{align*}
    0 < x < 1,
    0 < y < x + 1,
    0 < z < x + y.
\end{align*}

\end{exercise}

% --------------------------------------------------------------------------------

\begin{solution}

Sei $\Omega$ das obere Gebiet.

\begin{center}
    \begin{tikzpicture}[scale = 0.5]

        \draw [->] (0, 0) -- ( 4,  0) node [right]      {$x$};
        \draw [->] (0, 0) -- (-6, -3) node [below left] {$y$};
        \draw [->] (0, 0) -- ( 0,  7) node [above]      {$z$};

        \filldraw (0, 0) circle (1 pt);

        \filldraw (2, 0) circle (1pt) node [below] {$1$};

        \filldraw (-2, -1) circle (1pt) node [above left] {$1$};
        \filldraw (-4, -2) circle (1pt) node [above left] {$2$};

        \filldraw (0, 2) circle (1pt) node [left] {$1$};
        \filldraw (0, 4) circle (1pt) node [left] {$2$};
        \filldraw (0, 6) circle (1pt) node [left] {$3$};
        
        \draw [dashed] (0, 0) -- ( 2,  0);
        \draw [dashed] (0, 0) -- (-2, -2);
        \draw [dashed] (0, 0) -- (-2,  5);

        \draw [dashed] (2, 0) -- (-2, -2) -- (-2, 5) -- cycle;

        \draw [dotted] (-4, -2) -- (-2, -2);
        \draw [dotted] ( 0,  6) -- (-4,  5) -- (-2,  5);
        \draw [dotted] ( 0,  6) -- ( 2,  6) -- (-2,  5);

    \end{tikzpicture}
\end{center}

\begin{align*}
    \Int[\partial \Omega]
     {
         \mathbf F^t \mathbf n
     }{
         \mathcal H^2
     }
     & \stackrel
     {
         \text{Gauß}
     }{=}
     \Int[\Omega]
     {
         \Div \mathbf F
     }{
         \lambda^3
     } \\
     & =
     \Int[0][1]
     {
        \Int[0][x + 1]
        {
            \Int[0][x+y]
            {
                x + 3 z
            }{z}
        }{y}
     }{x} \\
     & =
     \Int[0][1]
     {
        \Int[0][x + 1]
        {
            x (x + y)
            +
            3 \frac{1}{2} (x + y)^2
        }{y}
    }{x} \\
    & =
     \Int[0][1]
     {
        \Int[0][x + 1]
        {
            x^2
            +
            x y
            +
            \frac{3}{2} x^2
            +
            \frac{3}{2} 2 x y
            +
            \frac{2}{3} y^2
        }{y}
    }{x} \\
    & =
     \Int[0][1]
     {
        \Int[0][x + 1]
        {
            \frac{5}{2} x^2
            +
            4 x y
            +
            \frac{3}{2} y^2
        }{y}
    }{x} \\
    & =
     \Int[0][1]
     {
        (x + 1) \frac{5}{2} x^2
        +
        4 x \frac{1}{2} (x + 1)^2
        +
        \frac{3}{2} \frac{1}{3} (x + 1)^3
    }{x} \\
    & =
     \Int[0][1]
     {
        \frac{5}{2} x^3
        +
        \frac{5}{2} x^2
        +
        2 x^3
        +
        4 x^2
        +
        2 x
        +
        \frac{1}{2} x^3
        +
        \frac{3}{2} x^2
        +
        \frac{3}{2} x
        +
        \frac{1}{2}
    }{x} \\
    & =
    5
    \underbrace
    {
        \Int[0][1]
        {
            x^3
        }{x}
    }_\frac{1}{4}
    +
    8
    \underbrace
    {
        \Int[0][1]
        {
            x^2
        }{x}
    }_\frac{1}{3}
    +
    \frac{7}{2}
    \underbrace
    {
        \Int[0][1]
        {
            x
        }{x}
    }_\frac{1}{2}
    +
    \frac{1}{2}
    \underbrace
    {
        \Int[0][1]{}{x}
    }_1 \\
    & =
    \frac{37}{6}
\end{align*}

\end{solution}

% --------------------------------------------------------------------------------
