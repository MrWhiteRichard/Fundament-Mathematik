% -------------------------------------------------------------------------------- %

\begin{exercise}

Beweisen Sie den Cauchyschen Integralsatz:
Eine Funktion $f$ sei in einer offenen Teilmenge $\Omega$ von $\C$ holomorph, $\gamma$ sei eine glatte Kurve in $\Omega$ die der Rand einer Teilmenge $D$ von $\Omega$ ist.
Dann gilt $\Int[\gamma]{f(z)}{z} = 0$ indem Sie das komplexe Wegintegral über $\gamma$ als

\begin{align*}
    \Int[\gamma]
    {
        f(z)
    }{z}
    =
    \int_\gamma
            \Re f
            ~\mathrm d x
            -
        \Im f
        ~\mathrm d y
    +
    \iu
    \int_\gamma
            \Im f
            ~\mathrm d x
            +
        \Re f
        ~\mathrm d y
\end{align*}

($z = x + \iu y$) auffassen, das Vektorfeld

\begin{align*}
    F
    =
    \begin{pmatrix}
         \Re f + \iu \Im f \\
        -\Im f + \iu \Re f
    \end{pmatrix}
    =
    (F_1) + \iu (F_2)
\end{align*}

betrachen und die Cauchy-Riemann'schen Differentialgleichungen sowie einen Integralsatz verwenden.

\end{exercise}

% -------------------------------------------------------------------------------- %

\begin{solution}

Laut den Cauchy-Riemann'schen Differentialgleichungen, gilt

\begin{multline*}
    \rot F
    =
    \derivative[][F_2]{y}
    -
    \derivative[][F_1]{x}
    =
    -
    \derivative[][\Im f]{y}
    +
    \iu
    \derivative[][\Re f]{y}
    -
    \derivative[][\Re f]{x}
    +
    \iu
    \derivative[][\Im f]{x} \\
    \stackrel
    {
        \text{CR}
    }{=}
    \derivative[][\Re f]{x}
    -
    \iu
    \derivative[][\Im f]{x}
    -
    \derivative[][\Re f]{x}
    +
    \iu
    \derivative[][\Im f]{x}
    =
    0.
\end{multline*}

\begin{multline*}
    \Forall (\xi, \eta)^t \in \R^2:
        F
        \begin{pmatrix}
            \xi \\ \eta
        \end{pmatrix}
        =
        \begin{pmatrix}
            \Re f + \iu \Im f \\
           -\Im f + \iu \Re f
       \end{pmatrix}
       \begin{pmatrix}
        \xi \\ \eta
    \end{pmatrix}
    =
    \xi \Re f + \iu \xi \Im f - \eta \Im f + \iu \eta \Re f \\
    =
    (\xi + \iu \eta) \Re f + \iu (\xi + \iu \eta) \Im f
    =
    (\xi + \iu \eta) f
\end{multline*}

\includegraphicsboxed{Ana1&2/Ana1&2 - (11.22).png}

Sei $\dom f = [a, b]$.

\begin{align*}
    \Int[\gamma]
    {
        f(z)
    }{z}
    & \stackrel
    {
        \text{(11.22)}
    }{=}
    \Int[\gamma]
    {
        F^t(z)
    }{z} \\
    & =
    \Int[a][b]
    {
        F^t(\gamma)
        \frac{\gamma^\prime}{|\gamma^\prime|}
        |\gamma^\prime|
    }{\lambda} \\
    & =
    \Int[a][b]
    {
        F^t(\gamma) t(\gamma)
    }{\mathcal H^1 \gamma} \\
    & =
    \Int[\partial D]
    {
        F^t t
    }{\mathcal H^1} \\
    & =
    \Int[\partial D]
    {
        F^t t
    }{s} \\
    & =
    \Int[\partial D]
    {
        F_1^t t
    }{s}
    +
    \iu
    \Int[\partial D]
    {
        F_2^t t
    }{s} \\
    & \stackrel
    {
        \text{Stokes}
    }{=}
    \Int[D]
    {
        \rot F_1
    }{\lambda^2}
    +
    \iu
    \Int[D]
    {
        \rot F_2
    }{\lambda^2} \\
    & =
    \Int[D]
    {
        \rot F
    }{\lambda^2} \\
    & =
    0
\end{align*}

Dabei haben wir verwendet, dass $|\gamma^\prime|$ die Dichte von $\mathcal H^1 \gamma$ bzgl. $\lambda$ ist, weil

\begin{align*}
    (\mathcal H^1 \gamma)([\alpha, \beta])
    =
    \ell(\gamma |_\bbraces{\alpha, \beta})
    =
    \Int[\alpha][\beta]
    {
        |\gamma^\prime|
    }{\lambda}.
\end{align*}

\end{solution}

% -------------------------------------------------------------------------------- %
