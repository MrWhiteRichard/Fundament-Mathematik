\begin{exercise}

Verschwindet für eine Funktion $f: \mathbb{R} \mapsto \mathbb{R}$ die schwache Ableitung der Ordnung $n$, so ist f ein Polynom der Ordnung $n-1$ fast überall.\\

Hinweis: Zeigen Sie, dass jede Testfunktion als Summe der n-ten Ableitung einer Testfunktion und einer Linearkombination der Funktionen $\Psi_0^{(l)}, l < n, \Psi_0$ wie
im Beweis von 6.1.4. dargestellt werden kann und berechnen Sie $\int x^k\xi^{(l)}(x)dx$ für Testfunktionen $\xi$ und $k \leq l$.

\end{exercise}

\begin{solution}

Es seinen $\Phi \in C_c^\infty$ eine beliebige Testfunktion und $m \in \N$. Sei $\Psi_0 \in C_c^\infty$ eine weitere Testfunktion mit $\int \Psi_0 d \lambda = 1$. Wir zeigen, dass $\Exists \zeta \in C_c^\infty: \Exists \alpha_0, \ldots, \alpha_{m-1} \in \R:$

\begin{align*}
  \Phi =
  \zeta^{(m)} +
  \sum_{l=0}^{m-1}
  \alpha_l \Psi_0^{(l)}.
\end{align*}

Wir benützen vollständige Induktion. Für $m = 0$, ist die Aussage richtig. Es gebe eine Darstellung

\begin{align*}
  \Phi =
  \theta^{(m)} +
  \sum_{l=0}^{m-1}
  \alpha_l \Psi_0^{(l)}, \:
  \theta \in C_c^\infty.
\end{align*}

Laut dem Beweis von Blümlinger Satz 6.1.4, ist folgendes $\eta \in C_c^\infty$ eine Testfunktion mit

\begin{align*}
    \eta :=
    \theta -
    \int \theta d \lambda \Psi_0, \:
    \int \eta d \lambda = 0.
\end{align*}

Deren Stammfunktion $\zeta \in C_c^\infty$ ist auch eine Testfunktion.

\begin{align*}
  \zeta: x \mapsto \int_{-\infty}^x \eta d \lambda
\end{align*}

Wegen $\zeta^{\prime}=\eta$ und der Definition von $\eta$ erhalten wir

\begin{align*}
    \theta =
    \zeta^{\prime} + \int \theta d \lambda\Psi_0.
\end{align*}

Mit $\alpha_m:=\int\theta d\lambda$ erhalten wir durch die Induktionsvoraussetzung

\begin{align*}
  \Phi
  = \theta^{(m)} +
    \sum_{l=0}^{m-1}
    \alpha_l \Psi_0^{(l)}
  = \zeta^{(m+1)} +
    \sum_{l=0}^{m}
    \alpha_l \Psi_0^{(l)}.
\end{align*}

Nun können wir uns der eigentlichen Aufgabe widmen. Dafür definieren wir ein Polynom

\begin{align*}
    p: \R \to \R: x \mapsto \sum_{j=0}^{n-1} \beta_j x^j
\end{align*}

Um $f = p$ zu zeigen, genügt es, laut Blümlinger Lemma 6.1.1 bzw. dem Fundamentallemma der Variationsrechnung, wenn $\Exists \beta_0, \dots, \beta_{n-1}: \Forall \Phi \in C_c^\infty:$

\begin{align*}
    \int (f-p) \Phi d \lambda = 0.
\end{align*}

Aufgrund $n$-facher partieller Integration bzw. $D^n f = 0$, folgt für $i>j$, dass $ \Forall \xi \in C_c^\infty:$

\begin{align*}
    \int x^j \xi^{(i)}(x) d \lambda(x) = 0, \quad
    \int f \xi^{(n)}(x) d \lambda(x) = 0.
\end{align*}

Mit dem Induktionsresultat, erhalten wir

\begin{align*}
  \int (f-p) \Phi d \lambda
  = \underbrace{\int (f-p) \zeta^{(n)} d \lambda}_0 +
    \sum_{i=0}^{n-1} \alpha_i
    \int(f-p) \Psi_0^{(i)} d \lambda
  \stackrel{!}{=} 0.
\end{align*}

Der erste Summand verschwindet aufgrund der obigen Überlegungen, weil $f-p \in \text{span} \Bbraces{x^0, \ldots, x^{n-1}, f}$. Damit das auch für die anderen gilt, muss

\begin{align*}
    \sum_{i=0}^{n-1} \alpha_i
    \int f \Psi_0^{(i)} d \lambda =
    \sum_{i=0}^{n-1}
    \alpha_i
    \int p \Psi_0^{(i)} d \lambda.
\end{align*}

Summandenweise heißt das $\Forall i = 0, \dots, n-1:$

\begin{align*}
    \underbrace{\int f \Psi_0^{(i)} d \lambda}_{=: b_i}
    = \int p \Psi_0^{(i)} d \lambda
    = \sum_{j=0}^{n-1}
      \beta_j
      \underbrace{\int x^j \Psi_0^{(i)}(x) d \lambda(x)}_{=: a_{ij}}.
\end{align*}

Dies führt uns auf das lösbare Gleichungssystem $A \beta = b$, wobei

\begin{align*}
  A =
  \begin{pmatrix}
    a_{00} & \cdots & a_{0, n-1}   \\
           & \ddots & \vdots       \\
    0      &        & a_{n-1, n-1}
  \end{pmatrix}, \:
  \beta =
  \begin{pmatrix}
  \beta_0 \\
  \vdots  \\
  \beta_{n-1}
  \end{pmatrix}, \:
  b =
  \begin{pmatrix}
    b_0    \\
    \vdots \\
    b_{n-1}
  \end{pmatrix}.
\end{align*}

 Dabei ist die Matrix $A$ regulär, da $\Forall i = 0, \ldots, n-1:$

 \begin{equation*}
   a_{ii} = (-1)^ii!\int\Psi_0 d\lambda(x) = (-1)^ii!
 \end{equation*}

\end{solution}
