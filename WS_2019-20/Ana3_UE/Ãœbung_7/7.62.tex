% -------------------------------------------------------------------------------- %

\begin{exercise}

Berechnen Sie den Schwerpunkt der inhomogenen Halbkugel \\
($\rho(x) = |x|^2$, $A = \Bbraces{x: x_1^2 + x_2^2 + x_3^2 \leq R^2, x_2 \geq 0}$) mit Radius $R$ mithilfe von Kugelkoordinaten.

\end{exercise}

% -------------------------------------------------------------------------------- %

\begin{solution}

\phantom{}

\includegraphicsgraphicsboxed
{Ana3/Ana3 - Beispiel 4.3.4.1.png}
{Ana3/Ana3 - Beispiel 4.3.4.2.png}

Wir wollen $A$ mit Kugelkoordinaten parametrisieren.
Das wurde netterweise in Beispiel 4.3.4 bereits gemacht.

\begin{align*}
    \psi:
    D := (0, R) \times (0, 2 \pi) \times \pbraces{0, \frac{\pi}{2}}
    \to
    A:
    (r, \varphi, \theta)
    \mapsto
    r
    \begin{pmatrix}
        \cos \theta \cos \varphi \\
        \cos \theta \sin \varphi \\
        \sin \theta
    \end{pmatrix}
\end{align*}

\begin{multline*}
    \rho(\psi(r, \varphi, \theta))
    =
    \abs
    {
        r
        \begin{pmatrix}
            \cos \theta \cos \varphi \\
            \cos \theta \sin \varphi \\
            \sin \theta
        \end{pmatrix}
    }^2
    =
    r^2
    (
        \cos^2 \theta \cos^2 \varphi
        +
        \cos^2 \theta \sin^2 \varphi
        +
        \sin^2 \theta
    ) \\
    =
    r^2
    (
        \cos^2 \theta
        \underbrace
        {
            (
                \cos^2 \varphi
                +
                \sin^2 \varphi
            )
        }_1
        +
        \sin^2 \theta
    )
    =
    r^2
    \underbrace
    {
        (
            \cos^2 \theta
            +
            \sin^2 \theta
        )
    }_1
    =
    r^2
\end{multline*}

\begin{align*}
    \Int[A]{\rho(x)}{\lambda^3(x)}
    & \stackrel
    {
        \mathrm{TRAFO}
    }{=}
    \Int[D]
    {
        \rho(\psi(r, \varphi, \theta))
        r^2 \cos \theta
    }{(r, \varphi, \theta)} \\
    & =
    \Int[0][\frac{\pi}{2}]
    {
        \Int[0][2 \pi]
        {
            \Int[0][R]
            {
                r^4
                \cos \theta
            }{r}
        }{\varphi}
    }{\theta} \\
    & =
    \Int[0][\frac{\pi}{2}]
    {
        \cos \theta
    }{\theta}
    \Int[0][2 \pi]{}{\varphi}
    \Int[0][R]
    {
        r^4
    }{r} \\
    & =
    \sin \theta \Big |_{\theta = 0}^\frac{\pi}{2}
    2 \pi
    \frac{1}{5}
    r^5 \Big |_{r=0}^R \\
    & =
    \frac{2 \pi R^5}{5}
\end{align*}

\begin{multline*}
    \Int[A]
    {
        \begin{pmatrix}
            x_1 \\ x_2 \\ x_3
        \end{pmatrix}
        \rho(x)
    }{\lambda^3(x)}
    \stackrel
    {
        \mathrm{TRAFO}
    }{=}
    \Int[D]
    {
        (\rho \cdot \id)(\psi(r, \varphi, \theta))
        r^2 \cos \theta
    }{(r, \varphi, \theta)} \\
    =
    \Int[0][\frac{\pi}{2}]
    {
        \Int[0][2 \pi]
        {
            \Int[0][R]
            {
                r^5 \cos \theta
                \begin{pmatrix}
                    \cos \theta \cos \varphi \\
                    \cos \theta \sin \varphi \\
                    \sin \theta
                \end{pmatrix}    
            }{r}
        }{\varphi}
    }{\theta}
\end{multline*}

\begin{enumerate}

    \item [1.] und 2. Komponente:

    \begin{align*}
        \Int[0][\frac{\pi}{2}]
        {
            \Int[0][2 \pi]
            {
                \Int[0][R]
                {
                    r^5
                    \cos \theta
                    \begin{pmatrix}
                        \cos \theta \cos \varphi \\
                        \cos \theta \sin \varphi
                    \end{pmatrix}
                }{r}
            }{\varphi}
        }{\theta}
        =
        \Int[0][\frac{\pi}{2}]
        {
            \cos^2 \theta
        }{\theta}
        \underbrace
        {
            \Int[0][2 \pi]
            {
                \begin{pmatrix}
                    \cos \varphi \\
                    \sin \varphi
                \end{pmatrix}
            }{\varphi}
        }_0
        \Int[0][R]
        {
            r^5
        }{r}
        =
        0
    \end{align*}

    \item [3.] Komponente:
    
    \begin{align*}
        \Int[0][\frac{\pi}{2}]
        {
            \Int[0][2 \pi]
            {
                \Int[0][R]
                {
                    r^5
                    \cos \theta
                    \sin \theta
                }{r}
            }{\varphi}
        }{\theta}
        =
        \Int[0][\frac{\pi}{2}]
        {
            \cos \theta
            \sin \theta
        }{\theta}
        \Int[0][2 \pi]{}{\varphi}
        \Int[0][R]
        {
            r^5
        }{r}
        =
        \frac{1}{2}
        2 \pi
        \frac{1}{6}
        r^6 \Big |_{r=0}^R
        =
        \frac{\pi R^6}{6}
    \end{align*}

\end{enumerate}    

\begin{align*}
    \implies
    \pbraces
    {
        \Int[A]{\rho(x)}{\lambda^3(x)}
    }^{-1}
    \Int[A]
    {
        \begin{pmatrix}
            x_1 \\ x_2 \\ x_3
        \end{pmatrix}
        \rho(x)
    }{\lambda^3(x)}
    =
    \frac{5}{2 \pi R^5}
    \frac{\pi R^6}{6}
    \begin{pmatrix}
        0 \\ 0 \\ 1
    \end{pmatrix}
    =
    \frac{5 R}{12}
    \begin{pmatrix}
        0 \\ 0 \\ 1
    \end{pmatrix}
\end{align*}

\end{solution}

% -------------------------------------------------------------------------------- %
