% --------------------------------------------------------------------------------

\begin{exercise}

Bestimmen Sie das Flächenmaß des Ellipsoides

\begin{align*}
    a^2 x^2 + a^2 y^2 + c^2 z^2 = 1,
    \quad
    c > a > 0.
\end{align*}

(Bemerkung: Die Flächenformel für das allgemeine Ellipsoid $a^2 x^2 + b^2 y^2 + c^2 z^2 = 1$, führt auf \blockquote{elliptische Integrale} die nicht elementar dargestellt werden können.)

\end{exercise}

% --------------------------------------------------------------------------------

\begin{solution}

\phantom{}

\begin{center}
    \begin{tikzpicture}[scale = 2]

        \draw[->] (-3,  0) -- ( 3,  0) node [right]      {$x$};
        \draw[->] ( 2,  3) -- (-2, -3) node [below left] {$y$};
        \draw[->] ( 0, -4) -- ( 0,  4) node [above]      {$z$};

        \draw (0, 0) ellipse (2 and 3);

        \draw [dotted] (0, -1.5) ellipse (1.75 and 0.25);
        \draw [dotted] (0,  0)   ellipse (2    and 0.5 );
        \draw [dotted] (0,  1.5) ellipse (1.75 and 0.25);

        \filldraw ( 0,     3)   circle (1 pt) node [above left]  {$(0, 0, 1/c)$};
        \filldraw ( 2,     0)   circle (1 pt) node [below right] {$(1/a, 0, 0)$};
        \filldraw (-0.33, -0.5) circle (1 pt) node [below left]  {$(0, 1/a, 0)$};

        \draw (1.75, 1.5) node {$\times$} node [above right] {$(x, 0, z)$};

    \end{tikzpicture}
\end{center}

\includegraphicsboxed{Ana3/Ana3 - Satz 4.2.15.png}

Wir wollen Satz 4.2.15 auf die ($x$-$y$-rotationssymmetrische) Oberfläche unseres Ellipsoides anwenden.
Dazu, müssen wir eine Formel für den Radius $f(z)$ der punktierten Kreise der Höhe $z$ aufstellen.

Wir betrachten dazu zunächst den mit \blockquote{$\times$} markierten Punkt $(x, 0, z)$.
Dieser liegt auf der Oberfläche des Ellipsoides und hat $y$-Komponente gleich $0$.

\begin{align*}
    & \implies
    \sqrt{a^2 x^2 + a^2 y^2 + c^2 z^2} = 1 \\
    & \implies
    a^2 x^2 + c^2 z^2 = 1 \\
    & \implies
    x = \frac{1}{a} \sqrt{1 - c^2 z^2} =: f(z) \\
    & \implies
    f^\prime(z) = \frac{1}{a} \frac{-c^2 z}{\sqrt{1 - c^2 z^2}}
\end{align*}

Weil das Ellipsoid ja $x$-$y$-rotationssymmetrisch ist, hätten wir dafür auch einen beliebigen anderen Punkt am punktierten Kreis wählen können.

\begin{align*}
    \implies
    \mathcal{H}^2(E)
    & \stackrel
    {
        \mathrm{4.2.15}
    }{=}
    2 \pi
    \Int[-1/c][1/c]
    {
        |f(z)|
        \sqrt{1 + f^\prime(z)^2}
    }{z} \\
    & =
    2 \pi
    \Int[-1/c][1/c]
    {
        \frac{\sqrt{1 - c^2 z^2}}{a}
        \sqrt
        {
            1
            +
            \pbraces
            {
                \frac{1}{a}
                \frac{-c^2 z}{\sqrt{1 - c^2 z^2}}
            }^2
        }
    }{z} \\
    & =
    \frac{2 \pi}{a}
    \Int[-1/c][1/c]
    {
        \sqrt{1 - c^2 z^2}
        \sqrt
        {
            1
            +
            \frac{c^4 z^2}{a^2 (1 - c^2 z^2)}
        }
    }{z} \\
    & =
    \frac{4 \pi}{a}
    \Int[0][1/c]
    {
        \sqrt
        {
            (1 - c^2 z^2)
            +
            \frac{c^4 z^2}{a^2}
        }
    }{z} \\
    & =
    \frac{4 \pi}{a}
    \Int[0][1/c]
    {
        \sqrt
        {
            1
            +
            c^2 z^2
            \pbraces
            {
                \pbraces{\frac{c}{a}}^2
                -
                1
            }
        }
    }{z} \\
    & =
    \frac{4 \pi}
    {
        a c
        \sqrt
        {
            \pbraces{\frac{c}{a}}^2 - 1
        }
    }
    \Int[0]
    [
        \sqrt
        {
            \pbraces{\frac{c}{a}}^2 - 1
        }
    ]
    {
        \sqrt{1 + u^2}
    }{u} \\
    & =
    \frac{4 \pi}
    {
        a c
        \sqrt
        {
            \pbraces{\frac{c}{a}}^2 - 1
        }
    }
    \Int[0]
    [
        \areasinh
        \sqrt
        {
            \pbraces{\frac{c}{a}}^2 - 1
        }
    ]
    {
        \cosh^2 t
    }{t} \\
    & \stackrel{!}{=}
    \frac{4 \pi}
    {
        a c
        \sqrt
        {
            \pbraces{\frac{c}{a}}^2 - 1
        }
    }
    \frac{1}{2}
    \pbraces
    {
        \frac{1}{2}
        \sinh(2 t)
        +
        t
    } \Big|_{t = 0}^{
        \areasinh
        \sqrt
        {
            \pbraces{\frac{c}{a}}^2 - 1
        }
    }
\end{align*}

Laut
\href{https://de.wikipedia.org/wiki/Sinus_hyperbolicus_und_Kosinus_hyperbolicus}{Wikipedia}
gelten folgende Formeln.

\begin{align*}
    \cosh \areasinh u & = \sqrt{1 + u^2}
    \implies
    \cosh t = \sqrt{1 + \sinh^2 t} \\
    \cosh^2 t & = \frac{1}{2} (\cosh(2 t) + 1)
\end{align*}

Dabei haben wir folgende Substitution verwendet.

\begin{align*}
    u = \sinh t
    \implies
    \begin{cases}
        \derivative[][u]{t} = \cosh t \implies \mathrm{d} u = \cosh t \mathrm{d} t \\
        t = \areasinh u
    \end{cases}
\end{align*}

Und dann haben wir folgende Stammfunktion verwendet.

\begin{multline*}
    \Int{\cosh^2 t}{t}
    =
    \frac{1}{2}
    \Int{\cosh(2 t) + \frac{1}{2}}{t}
    =
    \frac{1}{4}
    \Int{\cosh s}{s}
    +
    \frac{1}{2}
    \Int{}{t} \\
    =
    \frac{1}{2}
    \pbraces
    {
        \frac{1}{2}
        \sinh s
        +
        t
    }
    =
    \frac{1}{2}
    \pbraces
    {
        \frac{1}{2}
        \sinh(2 t)
        +
        t
    }
\end{multline*}

Dabei haben wir folgende Substitution verwendet.

\begin{align*}
    s = 2 t
    \implies
    \derivative[][s]{t} = 2
    \implies
    \mathrm{d} t = \frac{1}{2} \mathrm{d} s
\end{align*}

\end{solution}

% --------------------------------------------------------------------------------
