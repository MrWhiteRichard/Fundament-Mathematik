% -------------------------------------------------------------------------------- %

\begin{exercise}

Zeigen Sie:
Die Hausdorffdimension der Vereinigung von abzählbar vielen Mengen ist gleich dem Supremum der Hausdorffdimensionen dieser Mengen.

\end{exercise}

% -------------------------------------------------------------------------------- %

\begin{solution}

Sei $(A_n)_{n \in \N}$ eine Folge von (abzählbar vielen) Mengen und $A = \bigcup_{n \in \N} A_n$ deren Vereinigung.

\begin{align*}
    \dim_\mathcal{H} A
    \stackrel{!}{=}
    \sup_{n \in \N} \dim_\mathcal{H} A_n
\end{align*}

\includegraphicsboxed{Ana3/Ana3 - Satz 4.1.4.png}

Sei $t \in (0, \infty)$.

\begin{enumerate}[label = \arabic*.]

    \item Fall ($t < \sup_{n \in \N} \dim_\mathcal{H} A_n$):

    \begin{align*}
        & \implies
        \Exists n \in \N:
            t < \dim_\mathcal{H} A_n = \sup \Bbraces{s \geq 0: \mathcal{H}^s(A_n) = \infty} \\
        & \implies
        \infty = \mathcal{H}^t(A_n) \leq \mathcal{H}^t(A) \\
        & \implies
        t \leq \sup \Bbraces{s \geq 0: \mathcal{H}^s(A) = \infty} = \dim_\mathcal{H} A
    \end{align*}

    \item Fall ($t > \sup_{n \in \N} \dim_\mathcal{H} A_n$):
    
    \begin{align*}
        & \implies
        \Forall n \in \N:
            t > \dim_\mathcal{H} A_n = \inf \Bbraces{s \geq 0: \mathcal{H}^s(A_n) = 0} \\
        & \implies
        \mathcal{H}^t(A) \leq \sum_{n \in \N} \underbrace{\mathcal{H}^t(A_n)}_0 = 0 \\
        & \implies
        t \geq  \inf \Bbraces{s \geq 0: \mathcal{H}^s(A) = 0} = \dim_\mathcal{H} A
    \end{align*}

\end{enumerate}

Sei nun $(t_n)_{n \in \N} \in (-\infty, \sup_{n \in \N} \dim_\mathcal{H} A_n)^\N$ eine Folge, die gegen $\sup_{n \in \N} \dim_\mathcal{H} A_n$ konvergiert.
Die Folgenglieder erfüllen den $1$-ten Fall, d.h. $(t_n)_{n \in \N} \leq \dim_\mathcal{H} A$.
Folglich, gilt dies auch für den Grenzwert.
Analog geht man vor für die andere Ungleichung.

\end{solution}

% -------------------------------------------------------------------------------- %
