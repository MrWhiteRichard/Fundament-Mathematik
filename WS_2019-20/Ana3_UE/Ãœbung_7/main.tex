\documentclass{article}

\def \lastexercisenumber{54}

% Hier befinden sich Pakete, die wir beinahe immer benutzen ...

\usepackage[utf8]{inputenc}

% Sprach-Paket:
\usepackage[ngerman]{babel}

% damit's nicht so, wie beim Grill aussieht:
\usepackage{fullpage}

% Mathematik:
\usepackage{amsmath, amssymb, amsfonts, amsthm}
\usepackage{bbm}
\usepackage{mathtools, mathdots}

% Makros mit mehereren Default-Argumenten:
\usepackage{twoopt}

% Anführungszeichen (Makro \Quote{}):
\usepackage{babel}

% if's für Makros:
\usepackage{xifthen}
\usepackage{etoolbox}

% tikz ist kein Zeichenprogramm (doch!):
\usepackage{tikz}

% bessere Aufzählungen:
\usepackage{enumitem}

% (bessere) Umgebung für Bilder:
\usepackage{graphicx, subfig, float}

% Umgebung für Code:
\usepackage{listings}

% Farben:
\usepackage{xcolor}

% Umgebung für "plain text":
\usepackage{verbatim}

% Umgebung für mehrerer Spalten:
\usepackage{multicol}

% "nette" Brüche
\usepackage{nicefrac}

% Spaltentypen verschiedener Dicke
\usepackage{tabularx}
\usepackage{makecell}

% Für Vektoren
\usepackage{esvect}

% (Web-)Links
\usepackage{hyperref}

% Zitieren & Literatur-Verzeichnis
\usepackage[style = authoryear]{biblatex}
\usepackage{csquotes}

% so ähnlich wie mathbb
%\usepackage{mathds}

% Keine Ahnung, was das macht ...
\usepackage{booktabs}
\usepackage{ngerman}
\usepackage{placeins}

% special letters:

\newcommand{\N}{\mathbb{N}}
\newcommand{\Z}{\mathbb{Z}}
\newcommand{\Q}{\mathbb{Q}}
\newcommand{\R}{\mathbb{R}}
\newcommand{\C}{\mathbb{C}}
\newcommand{\K}{\mathbb{K}}
\newcommand{\T}{\mathbb{T}}
\newcommand{\E}{\mathbb{E}}
\newcommand{\V}{\mathbb{V}}
\renewcommand{\S}{\mathbb{S}}
\renewcommand{\P}{\mathbb{P}}
\newcommand{\1}{\mathbbm{1}}

% quantors:

\newcommand{\Forall}{\forall \,}
\newcommand{\Exists}{\exists \,}
\newcommand{\ExistsOnlyOne}{\exists! \,}
\newcommand{\nExists}{\nexists \,}
\newcommand{\ForAlmostAll}{\forall^\infty \,}

% MISC symbols:

\newcommand{\landau}{{\scriptstyle \mathcal{O}}}
\newcommand{\Landau}{\mathcal{O}}


\newcommand{\eps}{\mathrm{eps}}

% graphics in a box:

\newcommandtwoopt
{\includegraphicsboxed}[3][][]
{
  \begin{figure}[!h]
    \begin{boxedin}
      \ifthenelse{\isempty{#1}}
      {
        \begin{center}
          \includegraphics[width = 0.75 \textwidth]{#3}
          \label{fig:#2}
        \end{center}
      }{
        \begin{center}
          \includegraphics[width = 0.75 \textwidth]{#3}
          \caption{#1}
          \label{fig:#2}
        \end{center}
      }
    \end{boxedin}
  \end{figure}
}

% braces:

\newcommand{\pbraces}[1]{{\left  ( #1 \right  )}}
\newcommand{\bbraces}[1]{{\left  [ #1 \right  ]}}
\newcommand{\Bbraces}[1]{{\left \{ #1 \right \}}}
\newcommand{\vbraces}[1]{{\left  | #1 \right  |}}
\newcommand{\Vbraces}[1]{{\left \| #1 \right \|}}
\newcommand{\abraces}[1]{{\left \langle #1 \right \rangle}}
\newcommand{\round}[1]{\bbraces{#1}}

\newcommand
{\floorbraces}[1]
{{\left \lfloor #1 \right \rfloor}}

\newcommand
{\ceilbraces} [1]
{{\left \lceil  #1 \right \rceil }}

% special functions:

\newcommand{\norm}  [2][]{\Vbraces{#2}_{#1}}
\newcommand{\diam}  [2][]{\mathrm{diam}_{#1} \: #2}
\newcommand{\diag}  [1]{\mathrm{diag} \: #1}
\newcommand{\dist}  [1]{\mathrm{dist} \: #1}
\newcommand{\mean}  [1]{\mathrm{mean} \: #1}
\newcommand{\erf}   [1]{\mathrm{erf} \: #1}
\newcommand{\id}    [1]{\mathrm{id} \: #1}
\newcommand{\sgn}   [1]{\mathrm{sgn} \: #1}
\newcommand{\supp}  [1]{\mathrm{supp} \: #1}
\newcommand{\arsinh}[1]{\mathrm{arsinh} \: #1}
\newcommand{\arcosh}[1]{\mathrm{arcosh} \: #1}
\newcommand{\artanh}[1]{\mathrm{artanh} \: #1}
\newcommand{\card}  [1]{\mathrm{card} \: #1}
\newcommand{\Span}  [1]{\mathrm{span} \: #1}
\newcommand{\Aut}   [1]{\mathrm{Aut} \: #1}
\newcommand{\End}   [1]{\mathrm{End} \: #1}
\newcommand{\ggT}   [1]{\mathrm{ggT} \: #1}
\newcommand{\kgV}   [1]{\mathrm{kgV} \: #1}
\newcommand{\ord}   [1]{\mathrm{ord} \: #1}
\newcommand{\grad}  [1]{\mathrm{grad} \: #1}
\newcommand{\ran}   [1]{\mathrm{ran} \: #1}
\newcommand{\graph} [1]{\mathrm{graph} \: #1}
\newcommand{\Inv}   [1]{\mathrm{Inv} \: #1}
\newcommand{\pv}    [1]{\mathrm{pv} \: #1}
\newcommand{\GL}    [1]{\mathrm{GL} \: #1}
\newcommand{\Mod}{\mathrm{Mod} \:}
\newcommand{\Th}{\mathrm{Th} \:}
\newcommand{\Char}{\mathrm{char}}
\newcommand{\At}{\mathrm{At}}
\newcommand{\Ob}{\mathrm{Ob}}
\newcommand{\Hom}{\mathrm{Hom}}
\newcommand{\orthogonal}[3][]{#2 ~\bot_{#1}~ #3}
\newcommand{\Rang}{\mathrm{Rang}}
\newcommand{\NIL}{\mathrm{NIL}}
\newcommand{\Res}{\mathrm{Res}}
\newcommand{\lxor}{\dot \lor}
\newcommand{\Div}{\mathrm{div} \:}
\newcommand{\meas}{\mathrm{meas} \:}

% fractions:

\newcommand{\Frac}[2]{\frac{1}{#1} \pbraces{#2}}
\newcommand{\nfrac}[2]{\nicefrac{#1}{#2}}

% derivatives & integrals:

\newcommandtwoopt
{\Int}[4][][]
{\int_{#1}^{#2} #3 ~\mathrm{d} #4}

\newcommandtwoopt
{\derivative}[3][][]
{
  \frac
  {\mathrm{d}^{#1} #2}
  {\mathrm{d} #3^{#1}}
}

\newcommandtwoopt
{\pderivative}[3][][]
{
  \frac
  {\partial^{#1} #2}
  {\partial #3^{#1}}
}

\newcommand
{\primeprime}
{{\prime \prime}}

\newcommand
{\primeprimeprime}
{{\prime \prime \prime}}

% Text:

\newcommand{\Quote}[1]{\glqq #1\grqq{}}
\newcommand{\Text}[1]{{\text{#1}}}
\newcommand{\fastueberall}{\text{f.ü.}}
\newcommand{\fastsicher}{\text{f.s.}}

% -------------------------------- %
% amsthm-stuff:

\theoremstyle{definition}

% numbered theorems
\newtheorem{theorem}{Satz}
\newtheorem{lemma}{Lemma}
\newtheorem{corollary}{Korollar}
\newtheorem{proposition}{Proposition}
\newtheorem{remark}{Bemerkung}
\newtheorem{definition}{Definition}
\newtheorem{example}{Beispiel}

% unnumbered theorems
\newtheorem*{theorem*}{Satz}
\newtheorem*{lemma*}{Lemma}
\newtheorem*{corollary*}{Korollar}
\newtheorem*{proposition*}{Proposition}
\newtheorem*{remark*}{Bemerkung}
\newtheorem*{definition*}{Definition}
\newtheorem*{example*}{Beispiel}

% Please define this stuff in project ("main.tex"):

% \def \lastexercisenumber {...}
% This will be 0 by default

% \setcounter{section}{...}
% This will be 0 by default
% and hence, completely ignored

\ifnum \thesection = 0
{\newtheorem{exercise}{Aufgabe}}
\else
{\newtheorem{exercise}{Aufgabe}[section]}
\fi

\ifdef
{\lastexercisenumber}
{\setcounter{exercise}{\lastexercisenumber}}

\newcommand{\solution}
{
    \renewcommand{\proofname}{Lösung}
    \renewcommand{\qedsymbol}{}
    \proof
}

\renewcommand{\proofname}{Beweis}

% -------------------------------- %
% environment zum einkasteln:

% dickere vertical lines
\newcolumntype
{x}
[1]
{!{\centering\arraybackslash\vrule width #1}}

% environment selbst (the big cheese)
\newenvironment
{boxedin}
{
  \begin{tabular}
  {
    x{1 pt}
    p{\textwidth}
    x{1 pt}
  }
  \Xhline
  {2 \arrayrulewidth}
}
{
  \\
  \Xhline{2 \arrayrulewidth}
  \end{tabular}
}

% -------------------------------- %
% MISC "Ein-Deutschungen"

\renewcommand
{\figurename}
{Abbildung}

\renewcommand
{\tablename}
{Tabelle}

% -------------------------------- %

% ---------------------------------------------------------------- %
% https://www.overleaf.com/learn/latex/Code_listing

\definecolor{codegreen} {rgb}{0, 0.6, 0}
\definecolor{codegray}    {rgb}{0.5, 0.5, 0.5}
\definecolor{codepurple}{rgb}{0.58, 0, 0.82}
\definecolor{backcolour}{rgb}{0.95, 0.95, 0.92}

\lstdefinestyle{overleaf}
{
    backgroundcolor = \color{backcolour},
    commentstyle = \color{codegreen},
    keywordstyle = \color{magenta},
    numberstyle = \tiny\color{codegray},
    stringstyle = \color{codepurple},
    basicstyle = \ttfamily \footnotesize,
    breakatwhitespace = false,
    breaklines = true,
    captionpos = b,
    keepspaces = true,
    numbers = left,
    numbersep = 5pt,
    showspaces = false,
    showstringspaces = false,
    showtabs = false,
    tabsize = 2
}

% ---------------------------------------------------------------- %
% https://en.wikibooks.org/wiki/LaTeX/Source_Code_Listings

\lstdefinestyle{customc}
{
    belowcaptionskip = 1 \baselineskip,
    breaklines = true,
    frame = L,
    xleftmargin = \parindent,
    language = C,
    showstringspaces = false,
    basicstyle = \footnotesize \ttfamily,
    keywordstyle = \bfseries \color{green!40!black},
    commentstyle = \itshape \color{purple!40!black},
    identifierstyle = \color{blue},
    stringstyle = \color{orange},
}

\lstdefinestyle{customasm}
{
    belowcaptionskip = 1 \baselineskip,
    frame = L,
    xleftmargin = \parindent,
    language = [x86masm] Assembler,
    basicstyle = \footnotesize\ttfamily,
    commentstyle = \itshape\color{purple!40!black},
}

% ---------------------------------------------------------------- %
% https://tex.stackexchange.com/questions/235731/listings-syntax-for-literate

\definecolor{maroon}        {cmyk}{0, 0.87, 0.68, 0.32}
\definecolor{halfgray}      {gray}{0.55}
\definecolor{ipython_frame} {RGB}{207, 207, 207}
\definecolor{ipython_bg}    {RGB}{247, 247, 247}
\definecolor{ipython_red}   {RGB}{186, 33, 33}
\definecolor{ipython_green} {RGB}{0, 128, 0}
\definecolor{ipython_cyan}  {RGB}{64, 128, 128}
\definecolor{ipython_purple}{RGB}{170, 34, 255}

\lstdefinestyle{stackexchangePython}
{
    breaklines = true,
    %
    extendedchars = true,
    literate =
    {á}{{\' a}} 1 {é}{{\' e}} 1 {í}{{\' i}} 1 {ó}{{\' o}} 1 {ú}{{\' u}} 1
    {Á}{{\' A}} 1 {É}{{\' E}} 1 {Í}{{\' I}} 1 {Ó}{{\' O}} 1 {Ú}{{\' U}} 1
    {à}{{\` a}} 1 {è}{{\` e}} 1 {ì}{{\` i}} 1 {ò}{{\` o}} 1 {ù}{{\` u}} 1
    {À}{{\` A}} 1 {È}{{\' E}} 1 {Ì}{{\` I}} 1 {Ò}{{\` O}} 1 {Ù}{{\` U}} 1
    {ä}{{\" a}} 1 {ë}{{\" e}} 1 {ï}{{\" i}} 1 {ö}{{\" o}} 1 {ü}{{\" u}} 1
    {Ä}{{\" A}} 1 {Ë}{{\" E}} 1 {Ï}{{\" I}} 1 {Ö}{{\" O}} 1 {Ü}{{\" U}} 1
    {â}{{\^ a}} 1 {ê}{{\^ e}} 1 {î}{{\^ i}} 1 {ô}{{\^ o}} 1 {û}{{\^ u}} 1
    {Â}{{\^ A}} 1 {Ê}{{\^ E}} 1 {Î}{{\^ I}} 1 {Ô}{{\^ O}} 1 {Û}{{\^ U}} 1
    {œ}{{\oe}}  1 {Œ}{{\OE}}  1 {æ}{{\ae}}  1 {Æ}{{\AE}}  1 {ß}{{\ss}}  1
    {ç}{{\c c}} 1 {Ç}{{\c C}} 1 {ø}{{\o}} 1 {å}{{\r a}} 1 {Å}{{\r A}} 1
    {€}{{\EUR}} 1 {£}{{\pounds}} 1
}


% Python definition (c) 1998 Michael Weber
% Additional definitions (2013) Alexis Dimitriadis
% modified by me (should not have empty lines)

\lstdefinelanguage{iPython}{
    morekeywords = {access, and, break, class, continue, def, del, elif, else, except, exec, finally, for, from, global, if, import, in, is, lambda, not, or, pass, print, raise, return, try, while}, %
    %
    % Built-ins
    morekeywords = [2]{abs, all, any, basestring, bin, bool, bytearray, callable, chr, classmethod, cmp, compile, complex, delattr, dict, dir, divmod, enumerate, eval, execfile, file, filter, float, format, frozenset, getattr, globals, hasattr, hash, help, hex, id, input, int, isinstance, issubclass, iter, len, list, locals, long, map, max, memoryview, min, next, object, oct, open, ord, pow, property, range, raw_input, reduce, reload, repr, reversed, round, set, setattr, slice, sorted, staticmethod, str, sum, super, tuple, type, unichr, unicode, vars, xrange, zip, apply, buffer, coerce, intern}, %
    %
    sensitive = true, %
    morecomment = [l] \#, %
    morestring = [b]', %
    morestring = [b]", %
    %
    morestring = [s]{'''}{'''}, % used for documentation text (mulitiline strings)
    morestring = [s]{"""}{"""}, % added by Philipp Matthias Hahn
    %
    morestring = [s]{r'}{'},     % `raw' strings
    morestring = [s]{r"}{"},     %
    morestring = [s]{r'''}{'''}, %
    morestring = [s]{r"""}{"""}, %
    morestring = [s]{u'}{'},     % unicode strings
    morestring = [s]{u"}{"},     %
    morestring = [s]{u'''}{'''}, %
    morestring = [s]{u"""}{"""}, %
    %
    % {replace}{replacement}{lenght of replace}
    % *{-}{-}{1} will not replace in comments and so on
    literate = 
    {á}{{\' a}} 1 {é}{{\' e}} 1 {í}{{\' i}} 1 {ó}{{\' o}} 1 {ú}{{\' u}} 1
    {Á}{{\' A}} 1 {É}{{\' E}} 1 {Í}{{\' I}} 1 {Ó}{{\' O}} 1 {Ú}{{\' U}} 1
    {à}{{\` a}} 1 {è}{{\` e}} 1 {ì}{{\` i}} 1 {ò}{{\` o}} 1 {ù}{{\` u}} 1
    {À}{{\` A}} 1 {È}{{\' E}} 1 {Ì}{{\` I}} 1 {Ò}{{\` O}} 1 {Ù}{{\` U}} 1
    {ä}{{\" a}} 1 {ë}{{\" e}} 1 {ï}{{\" i}} 1 {ö}{{\" o}} 1 {ü}{{\" u}} 1
    {Ä}{{\" A}} 1 {Ë}{{\" E}} 1 {Ï}{{\" I}} 1 {Ö}{{\" O}} 1 {Ü}{{\" U}} 1
    {â}{{\^ a}} 1 {ê}{{\^ e}} 1 {î}{{\^ i}} 1 {ô}{{\^ o}} 1 {û}{{\^ u}} 1
    {Â}{{\^ A}} 1 {Ê}{{\^ E}} 1 {Î}{{\^ I}} 1 {Ô}{{\^ O}} 1 {Û}{{\^ U}} 1
    {œ}{{\oe}}  1 {Œ}{{\OE}}  1 {æ}{{\ae}}  1 {Æ}{{\AE}}  1 {ß}{{\ss}}  1
    {ç}{{\c c}} 1 {Ç}{{\c C}} 1 {ø}{{\o}} 1 {å}{{\r a}} 1 {Å}{{\r A}} 1
    {€}{{\EUR}} 1 {£}{{\pounds}} 1
    %
    {^}{{{\color{ipython_purple}\^ {}}}} 1
    { = }{{{\color{ipython_purple} = }}} 1
    %
    {+}{{{\color{ipython_purple}+}}} 1
    {*}{{{\color{ipython_purple}$^\ast$}}} 1
    {/}{{{\color{ipython_purple}/}}} 1
    %
    {+=}{{{+=}}} 1
    {-=}{{{-=}}} 1
    {*=}{{{$^\ast$ = }}} 1
    {/=}{{{/=}}} 1,
    literate = 
    *{-}{{{\color{ipython_purple} -}}} 1
     {?}{{{\color{ipython_purple} ?}}} 1,
    %
    identifierstyle = \color{black}\ttfamily,
    commentstyle = \color{ipython_cyan}\ttfamily,
    stringstyle = \color{ipython_red}\ttfamily,
    keepspaces = true,
    showspaces = false,
    showstringspaces = false,
    %
    rulecolor = \color{ipython_frame},
    frame = single,
    frameround = {t}{t}{t}{t},
    framexleftmargin = 6mm,
    numbers = left,
    numberstyle = \tiny\color{halfgray},
    %
    %
    backgroundcolor = \color{ipython_bg},
    % extendedchars = true,
    basicstyle = \scriptsize,
    keywordstyle = \color{ipython_green}\ttfamily,
}

% ---------------------------------------------------------------- %
% https://tex.stackexchange.com/questions/417884/colour-r-code-to-match-knitr-theme-using-listings-minted-or-other

\geometry{verbose, tmargin = 2.5cm, bmargin = 2.5cm, lmargin = 2.5cm, rmargin = 2.5cm}

\definecolor{backgroundCol}  {rgb}{.97, .97, .97}
\definecolor{commentstyleCol}{rgb}{0.678, 0.584, 0.686}
\definecolor{keywordstyleCol}{rgb}{0.737, 0.353, 0.396}
\definecolor{stringstyleCol} {rgb}{0.192, 0.494, 0.8}
\definecolor{NumCol}         {rgb}{0.686, 0.059, 0.569}
\definecolor{basicstyleCol}  {rgb}{0.345, 0.345, 0.345}

\lstdefinestyle{stackexchangeR}
{
    language = R,                                        % the language of the code
    basicstyle = \small \ttfamily \color{basicstyleCol}, % the size of the fonts that are used for the code
    % numbers = left,                                      % where to put the line-numbers
    numberstyle = \color{green},                         % the style that is used for the line-numbers
    stepnumber = 1,                                      % the step between two line-numbers. If it is 1, each line will be numbered
    numbersep = 5pt,                                     % how far the line-numbers are from the code
    backgroundcolor = \color{backgroundCol},             % choose the background color. You must add \usepackage{color}
    showspaces = false,                                  % show spaces adding particular underscores
    showstringspaces = false,                            % underline spaces within strings
    showtabs = false,                                    % show tabs within strings adding particular underscores
    % frame = single,                                      % adds a frame around the code
    % rulecolor = \color{white},                           % if not set, the frame-color may be changed on line-breaks within not-black text (e.g. commens (green here))
    tabsize = 2,                                         % sets default tabsize to 2 spaces
    captionpos = b,                                      % sets the caption-position to bottom
    breaklines = true,                                   % sets automatic line breaking
    breakatwhitespace = false,                           % sets if automatic breaks should only happen at whitespace
    keywordstyle = \color{keywordstyleCol},              % keyword style
    commentstyle = \color{commentstyleCol},              % comment style
    stringstyle = \color{stringstyleCol},                % string literal style
    literate = %
    *{0}{{{\color{NumCol} 0}}} 1
     {1}{{{\color{NumCol} 1}}} 1
     {2}{{{\color{NumCol} 2}}} 1
     {3}{{{\color{NumCol} 3}}} 1
     {4}{{{\color{NumCol} 4}}} 1
     {5}{{{\color{NumCol} 5}}} 1
     {6}{{{\color{NumCol} 6}}} 1
     {7}{{{\color{NumCol} 7}}} 1
     {8}{{{\color{NumCol} 8}}} 1
     {9}{{{\color{NumCol} 9}}} 1
}

% ---------------------------------------------------------------- %
% Fundament Mathematik

\lstdefinestyle{fundament}{basicstyle = \ttfamily}

% ---------------------------------------------------------------- %


\addbibresource{../../../Fundament-LaTeX/references.bib}

\graphicspath{{../../../Fundament-LaTeX/images/}}

\parskip 0pt
\parindent 0pt

\title
{
  Analysis 3 \\
  \vspace{4pt}
  \normalsize
  \textit{7. Übung}
}
\author
{
  Richard Weiss
  \and
  Florian Schager
  \and
  Christian Sallinger
  \and
  Fabian Zehetgruber
  \and
  Paul Winkler
  \and
  Christian Göth
}
\date{25.11.2019}

\begin{document}

\maketitle

% -------------------------------------------------------------------------------- %

\begin{exercise}

Zeigen Sie:
Die Hausdorffdimension der Vereinigung von abzählbar vielen Mengen ist gleich dem Supremum der Hausdorffdimensionen dieser Mengen.

\end{exercise}

% -------------------------------------------------------------------------------- %

\begin{solution}

Sei $(A_n)_{n \in \N}$ eine Folge von (abzählbar vielen) Mengen und $A = \bigcup_{n \in \N} A_n$ deren Vereinigung.

\begin{align*}
    \dim_\mathcal{H} A
    \stackrel{!}{=}
    \sup_{n \in \N} \dim_\mathcal{H} A_n
\end{align*}

\includegraphicsboxed{Ana3/Ana3 - Satz 4.1.4.png}

Sei $t \in (0, \infty)$.

\begin{enumerate}[label = \arabic*.]

    \item Fall ($t < \sup_{n \in \N} \dim_\mathcal{H} A_n$):

    \begin{align*}
        & \implies
        \Exists n \in \N:
            t < \dim_\mathcal{H} A_n = \sup \Bbraces{s \geq 0: \mathcal{H}^s(A_n) = \infty} \\
        & \implies
        \infty = \mathcal{H}^t(A_n) \leq \mathcal{H}^t(A) \\
        & \implies
        t \leq \sup \Bbraces{s \geq 0: \mathcal{H}^s(A) = \infty} = \dim_\mathcal{H} A
    \end{align*}

    \item Fall ($t > \sup_{n \in \N} \dim_\mathcal{H} A_n$):
    
    \begin{align*}
        & \implies
        \Forall n \in \N:
            t > \dim_\mathcal{H} A_n = \inf \Bbraces{s \geq 0: \mathcal{H}^s(A_n) = 0} \\
        & \implies
        \mathcal{H}^t(A) \leq \sum_{n \in \N} \underbrace{\mathcal{H}^t(A_n)}_0 = 0 \\
        & \implies
        t \geq  \inf \Bbraces{s \geq 0: \mathcal{H}^s(A) = 0} = \dim_\mathcal{H} A
    \end{align*}

\end{enumerate}

Sei nun $(t_n)_{n \in \N} \in (-\infty, \sup_{n \in \N} \dim_\mathcal{H} A_n)^\N$ eine Folge, die gegen $\sup_{n \in \N} \dim_\mathcal{H} A_n$ konvergiert.
Die Folgenglieder erfüllen den $1$-ten Fall, d.h. $(t_n)_{n \in \N} \leq \dim_\mathcal{H} A$.
Folglich, gilt dies auch für den Grenzwert.
Analog geht man vor für die andere Ungleichung.

\end{solution}

% -------------------------------------------------------------------------------- %

% -------------------------------------------------------------------------------- %

\begin{exercise}

Bestimmen Sie das Flächenmaß des Ellipsoides

\begin{align*}
    a^2 x^2 + a^2 y^2 + c^2 z^2 = 1,
    \quad
    c > a > 0.
\end{align*}

(Bemerkung: Die Flächenformel für das allgemeine Ellipsoid $a^2 x^2 + b^2 y^2 + c^2 z^2 = 1$, führt auf \enquote{elliptische Integrale} die nicht elementar dargestellt werden können.)

\end{exercise}

% -------------------------------------------------------------------------------- %

\begin{solution}

\phantom{}

\begin{center}
    \begin{tikzpicture}[scale = 2]

        \draw[->] (-3,  0) -- ( 3,  0) node [right]      {$x$};
        \draw[->] ( 2,  3) -- (-2, -3) node [below left] {$y$};
        \draw[->] ( 0, -4) -- ( 0,  4) node [above]      {$z$};

        \draw (0, 0) ellipse (2 and 3);

        \draw [dotted] (0, -1.5) ellipse (1.75 and 0.25);
        \draw [dotted] (0,  0)   ellipse (2    and 0.5 );
        \draw [dotted] (0,  1.5) ellipse (1.75 and 0.25);

        \filldraw ( 0,     3)   circle (1 pt) node [above left]  {$(0, 0, 1/c)$};
        \filldraw ( 2,     0)   circle (1 pt) node [below right] {$(1/a, 0, 0)$};
        \filldraw (-0.33, -0.5) circle (1 pt) node [below left]  {$(0, 1/a, 0)$};

        \draw (1.75, 1.5) node {$\times$} node [above right] {$(x, 0, z)$};

    \end{tikzpicture}
\end{center}

\includegraphicsboxed{Ana3/Ana3 - Satz 4.2.15.png}

Wir wollen Satz 4.2.15 auf die ($x$-$y$-rotationssymmetrische) Oberfläche unseres Ellipsoides anwenden.
Dazu, müssen wir eine Formel für den Radius $f(z)$ der punktierten Kreise der Höhe $z$ aufstellen.

Wir betrachten dazu zunächst den mit \enquote{$\times$} markierten Punkt $(x, 0, z)$.
Dieser liegt auf der Oberfläche des Ellipsoides und hat $y$-Komponente gleich $0$.

\begin{align*}
    & \implies
    \sqrt{a^2 x^2 + a^2 y^2 + c^2 z^2} = 1 \\
    & \implies
    a^2 x^2 + c^2 z^2 = 1 \\
    & \implies
    x = \frac{1}{a} \sqrt{1 - c^2 z^2} =: f(z) \\
    & \implies
    f^\prime(z) = \frac{1}{a} \frac{-c^2 z}{\sqrt{1 - c^2 z^2}}
\end{align*}

Weil das Ellipsoid ja $x$-$y$-rotationssymmetrisch ist, hätten wir dafür auch einen beliebigen anderen Punkt am punktierten Kreis wählen können.

\begin{align*}
    \implies
    \mathcal{H}^2(E)
    & \stackrel
    {
        \mathrm{4.2.15}
    }{=}
    2 \pi
    \Int[-1/c][1/c]
    {
        |f(z)|
        \sqrt{1 + f^\prime(z)^2}
    }{z} \\
    & =
    2 \pi
    \Int[-1/c][1/c]
    {
        \frac{\sqrt{1 - c^2 z^2}}{a}
        \sqrt
        {
            1
            +
            \pbraces
            {
                \frac{1}{a}
                \frac{-c^2 z}{\sqrt{1 - c^2 z^2}}
            }^2
        }
    }{z} \\
    & =
    \frac{2 \pi}{a}
    \Int[-1/c][1/c]
    {
        \sqrt{1 - c^2 z^2}
        \sqrt
        {
            1
            +
            \frac{c^4 z^2}{a^2 (1 - c^2 z^2)}
        }
    }{z} \\
    & =
    \frac{4 \pi}{a}
    \Int[0][1/c]
    {
        \sqrt
        {
            (1 - c^2 z^2)
            +
            \frac{c^4 z^2}{a^2}
        }
    }{z} \\
    & =
    \frac{4 \pi}{a}
    \Int[0][1/c]
    {
        \sqrt
        {
            1
            +
            c^2 z^2
            \pbraces
            {
                \pbraces{\frac{c}{a}}^2
                -
                1
            }
        }
    }{z} \\
    & =
    \frac{4 \pi}
    {
        a c
        \sqrt
        {
            \pbraces{\frac{c}{a}}^2 - 1
        }
    }
    \Int[0]
    [
        \sqrt
        {
            \pbraces{\frac{c}{a}}^2 - 1
        }
    ]
    {
        \sqrt{1 + u^2}
    }{u} \\
    & =
    \frac{4 \pi}
    {
        a c
        \sqrt
        {
            \pbraces{\frac{c}{a}}^2 - 1
        }
    }
    \Int[0]
    [
        \areasinh
        \sqrt
        {
            \pbraces{\frac{c}{a}}^2 - 1
        }
    ]
    {
        \cosh^2 t
    }{t} \\
    & \stackrel{!}{=}
    \frac{4 \pi}
    {
        a c
        \sqrt
        {
            \pbraces{\frac{c}{a}}^2 - 1
        }
    }
    \frac{1}{2}
    \pbraces
    {
        \frac{1}{2}
        \sinh(2 t)
        +
        t
    } \Big|_{t = 0}^{
        \areasinh
        \sqrt
        {
            \pbraces{\frac{c}{a}}^2 - 1
        }
    }
\end{align*}

Laut
\href{https://de.wikipedia.org/wiki/Sinus_hyperbolicus_und_Kosinus_hyperbolicus}{Wikipedia}
gelten folgende Formeln.

\begin{align*}
    \cosh \areasinh u & = \sqrt{1 + u^2}
    \implies
    \cosh t = \sqrt{1 + \sinh^2 t} \\
    \cosh^2 t & = \frac{1}{2} (\cosh(2 t) + 1)
\end{align*}

Dabei haben wir folgende Substitution verwendet.

\begin{align*}
    u = \sinh t
    \implies
    \begin{cases}
        \derivative[][u]{t} = \cosh t \implies \mathrm{d} u = \cosh t \mathrm{d} t \\
        t = \areasinh u
    \end{cases}
\end{align*}

Und dann haben wir folgende Stammfunktion verwendet.

\begin{multline*}
    \Int{\cosh^2 t}{t}
    =
    \frac{1}{2}
    \Int{\cosh(2 t) + \frac{1}{2}}{t}
    =
    \frac{1}{4}
    \Int{\cosh s}{s}
    +
    \frac{1}{2}
    \Int{}{t} \\
    =
    \frac{1}{2}
    \pbraces
    {
        \frac{1}{2}
        \sinh s
        +
        t
    }
    =
    \frac{1}{2}
    \pbraces
    {
        \frac{1}{2}
        \sinh(2 t)
        +
        t
    }
\end{multline*}

Dabei haben wir folgende Substitution verwendet.

\begin{align*}
    s = 2 t
    \implies
    \derivative[][s]{t} = 2
    \implies
    \mathrm{d} t = \frac{1}{2} \mathrm{d} s
\end{align*}

\end{solution}

% -------------------------------------------------------------------------------- %

% --------------------------------------------------------------------------------

\begin{exercise}

Zeigen Sie, dass auf $Q := \Bbraces{(x, y) \in \R^2: x > 0, y > 0}$ durch $\phi(x, y) = (x^2 - y^2, x y)$ ein Diffeomorphismus von $Q$ auf $\phi(Q) = \Bbraces{(a, b) \in \R^2: b > 0}$ gegeben ist.
Berechnen Sie mithilfe dieser Koordinatentransformation das Integral

\begin{align*}
    \int_\Omega
    x^2 + y^2
    ~\mathrm{d} x ~\mathrm{d} y
\end{align*}

für $\Omega = \Bbraces{(x, y) \in Q: 1 < x y < 3, 5 < x^2 - y^2 < 9}$.

\end{exercise}

% --------------------------------------------------------------------------------

\begin{solution}

\phantom{}

\begin{enumerate}[label = \arabic*.]

    \item Teil:
    
    \begin{align*}
        \begin{pmatrix}
            x^2 - y^2 \\ x y
        \end{pmatrix}
        \stackrel{!}{=}
        \begin{pmatrix}
            a \\ b
        \end{pmatrix}
        & \implies
        x = \frac{b}{y} \\
        & \implies
        a = \pbraces{\frac{b}{y}}^2 - y^2 \\
        & \implies
        a y^2 = b^2 - y^4 \\
        & \implies
        y^4 + y <^2 - b^2 = 0 \\
        & \implies
        y^2 = -\frac{a}{2} \pm \sqrt{\frac{a^2}{4} + b^2} \\
        & \stackrel{!}{\implies}
        y = \sqrt{\sqrt{\frac{a^2}{4} + b^2} - \frac{a}{2}} \\
        & \implies
        x = \frac{b}{\sqrt{\sqrt{\frac{a^2}{4} + b^2} - \frac{a}{2}}} \\
        & \implies
        \begin{pmatrix}
            x^2 - y^2 \\ x y
        \end{pmatrix}
        =
        \begin{pmatrix}
            a \\ b
        \end{pmatrix}
    \end{align*}

    Man beachte, dass dabei
    
    \begin{enumerate}[label = \arabic*.]

        \item wegen $y^2 > 0$, das $\pm$ zu einem $+$ wurde und

        \item weil 
        
        \begin{align*}
            b > 0
            \implies
            \sqrt{\frac{a^2}{4} + b^2} - \frac{a}{2} > 0,
        \end{align*}

        alle $\sqrt{}$ reell sein müssen.

    \end{enumerate}
    
    \item Teil:
    
    \includegraphicsboxed{Ana3/Ana3 - Satz 4.3.1 (Transformationsformel).png}

    \begin{align*}
        \phi(\Omega) = (5, 9) \times (1, 3)
    \end{align*}

    \begin{align*}
        \det \mathrm{d} \phi(x, y)
        =
        \det
        \begin{pmatrix}
            2 x & -2 y \\
            y   &  x
        \end{pmatrix}
        =
        2 x^2 - (-2 y^2)
        =
        2 (x^2 + y^2)
    \end{align*}

    \begin{multline*}        
        \implies
        \Int[\Omega]{x^2 + y^2}{(x, y)}
        =
        \frac{1}{2}
        \Int[\Omega]{|\det \mathrm{d} \phi(x, y)|}{(x, y)}
        =
        \frac{1}{2}
        \Int[\phi(\Omega)]{}{(x, y)} \\
        \stackrel
        {
            \text{TRAFO}
        }{=}
        \frac{1}{2}
        \Int[1][3]
        {
            \Int[9][5]{}{x}
        }{y}
        =
        \frac{1}{2}
        (3 - 1)
        (9 - 5)
        = 4
    \end{multline*}

\end{enumerate}

\end{solution}

% --------------------------------------------------------------------------------

% --------------------------------------------------------------------------------

\begin{exercise}

Bestimmen Sie das Flächenmaß der durch

\begin{align*}
    z(x, y) = (1 - x^2 - y^2),
    \quad
    x^2 + y^2 \leq 1
\end{align*}

definierten Fläche.

\end{exercise}

% --------------------------------------------------------------------------------

\begin{solution}

\begin{align*}
    z(x, y) = 1 - \norm[2]{(x, y)^\top}^2
\end{align*}

Die Fläche $A$ ist also $x$-$y$-rotationssymmetrisch.
Wir wollen Satz 4.2.15 anwenden.
Dazu, müssen wir eine Formel für den Radius $f(z)$ der Höhe $z$ aufstellen.
Setzen wir $y = 0$ und stellen den oberen Zusammenhang zwischen $x$ und $z$ so her, dass $(x, y, z)$ auf der Fläche liegt, so ist $x$ genau jener Radius.

\begin{align*}
    \implies
    z(x, 0) = 1 - x^2
    \implies
    x = \sqrt{1 - z} =: f(z)
    \implies
    f^\prime(z) = \frac{1}{2} \frac{1}{\sqrt{1 - z} (-1)} = -\frac{1}{2 \sqrt{1 - z}}
\end{align*}

Für $x^2 + y^2 \leq 1$, muss $z \in [0, 1]$ sein.

\begin{align*}
    \implies
    \mathcal{H}^2(A)
    & =
    2 \pi
    \Int[0][1]
    {
        |f(z)|
        \sqrt{1 + f^\prime(z)^2}
    } \\
    & =
    2 \pi
    \Int[0][1]
    {
        \sqrt{1 - z}
        \sqrt
        {
            1
            +
            \pbraces
            {
                -\frac{1}{2 \sqrt{1 - z}}
            }^2
        }
    }{z} \\
    & =
    2 \pi
    \Int[0][1]
    {
        \sqrt{1 - z}
        \sqrt
        {
            1
            +
            \frac{1}{4}
            (1 - z)
        }
    }{z} \\
    & =
    2 \pi
    \Int[0][1]
    {
        \sqrt{1 - z}
        \sqrt
        {
            1 - z + \frac{1}{4}
        }
    }{z} \\
    & =
    2 \pi
    \Int[\frac{1}{4}][\frac{5}{4}]{\sqrt u}{u} \\
    & =
    2 \pi
    \frac{2}{3}
    u^\frac{3}{2} \Big |_{u = \frac{1}{4}}^\frac{5}{4} \\
    & =
    \frac{4 \pi}{3}
    \pbraces
    {
        \sqrt{\frac{5}{4}}^3
        -
        \sqrt{\frac{1}{4}}^3
    } \\
    & =
    \frac{4 \pi}{3}
    \pbraces
    {
        \frac{\sqrt{5}}{2^3}
        -
        \frac{1}{2^3}
    } \\
    & =
    \frac{4 \pi}{3 \cdot 8}
    (5 \sqrt 5 - 1) \\
    & =
    \frac{(5 \sqrt 5 - 1) \pi}{6}
\end{align*}

\end{solution}

% --------------------------------------------------------------------------------

% --------------------------------------------------------------------------------

\begin{exercise}

Seien $r, z$ $C^1$-Funktionen, $r > 0$ dann wird durch

\begin{align*}
    \phi:
    (a, b) \times [0, 2 \pi) \to \R,
    \phi(t, \varphi) = (r(t) \cos \varphi, r(t) \sin \varphi, z(t))
\end{align*}

falls diese Funktion injektiv ist eine um die $z$-Achse rotationssymmetrische $2$-dimensionale Fläche $F$ im $\R^3$ definiert.
Zeigen Sie

\begin{align*}
    \mathcal{H}^2(F)
    =
    2 \pi
    \Int[(a, b)]{r(t) \sqrt{\dot r(t)^2 + \dot z(t)^2}}{t}.
\end{align*}

\end{exercise}

% --------------------------------------------------------------------------------

\begin{solution}

\begin{align*}
    &
    \mathrm{d} \phi(t, \varphi)^\top
    \mathrm{d} \phi(t, \varphi) \\
    & =
    \begin{pmatrix}
        \dot r(t) \cos \varphi & \dot r(t) \sin \varphi & \dot z(t) \\
            -r(t) \sin \varphi &      r(t) \cos \varphi & 0
    \end{pmatrix}
    \begin{pmatrix}
        \dot r(t) \cos \varphi & -r(t) \sin \varphi \\
        \dot r(t) \sin \varphi &  r(t) \cos \varphi \\
        \dot z(t)              &  0
    \end{pmatrix} \\
    & =
    \begin{pmatrix}
        \dot r(t)^2 \cos^2 \varphi + \dot r(t)^2 \sin^2 \varphi + \dot z(t)^2
        &
        (\dot r(t) \cos \varphi) (-r(t) \sin \varphi) + (\dot r(t) \sin \varphi) (r(t) \cos \varphi)
        \\
        (-r(t) \sin \varphi) (\dot r(t) \cos \varphi) + (r(t) \cos \varphi) (\dot r(t) \sin \varphi)
        &
        r(t)^2 \sin^2 \varphi + r(t)^2 \cos^2 \varphi
    \end{pmatrix} \\
    & =
    \begin{pmatrix}
        \dot r(t)^2 + \dot z(t)^2 & 0 \\
        0                         & r(t)^2
    \end{pmatrix}
\end{align*}

\begin{align*}
    \implies
    \sqrt{\det(\mathrm{d} \phi(t, \varphi)^\top \mathrm{d} \phi(r, \varphi))}
    =
    r(t) \sqrt{\dot r(t)^2 + \dot z(t)^2}
\end{align*}

\includegraphicsboxed{Ana3/Ana3 - Satz 4.2.10 (Flächenformel für C^1-Abbildungen).png}

\end{solution}

% --------------------------------------------------------------------------------

% -------------------------------------------------------------------------------- %

\begin{exercise}

Berechnen Sie

\begin{align*}
    \Int[A]{\exp \pbraces{\frac{x_1 + x_2}{x_1 - x_2}}}{\lambda^2(x_1, x_2)},
    \quad
    A = \Bbraces{(x_1, x_2): x_1 \geq 0, x_2 \leq 0, x_2 + 1 \leq x_1 \leq x_2 + 2}
\end{align*}

und

\begin{align*}
    \Int[B]{\sin x_1}{\lambda^2(x_1, x_2)},
    \quad
    B = \Bbraces{(x_1, x_2): x_1 x_2 \geq 0, x_1^2 + x_2^2 \leq \pi}.
\end{align*}

\end{exercise}

% -------------------------------------------------------------------------------- %

\begin{solution}

\phantom{}

\begin{enumerate}[label = \arabic*.]

    \item Teil:
    
    Die ersten beiden Ungleichungen von $A$ schränken auf den $4$-ten Quadranten ein.
    Wenn man bei den letzten Beiden $\leq$ durch $=$ ersetzt, stehen die (linearen) Funktionen da, die untere bzw. obere Schranken für $x_2$ an der Stelle $x_1$ vorgeben.

    \begin{center}
        
        \begin{tikzpicture}
    
            \begin{scope}[xshift = -5 cm]
                
                \draw [->] (-1, 0) -- (3, 0) node [right] {$x_1$};
                \draw [->] (0, -3) -- (0, 1) node [right] {$x_2$};
    
                \foreach \x in {1, 2}
                    \filldraw (\x, 0) circle (1 pt) node [above] {$\x$};
    
                \foreach \y in {-1, -2}
                    \filldraw (0, \y) circle (1 pt) node [left]  {$\y$};
                
                \filldraw [pattern = dots]
                    (0, -2) --
                    (2,  0) --
                    (1,  0) --
                    (0, -1) --
                    cycle;
    
            \end{scope}
    
            \draw (0, 0) node {$\stackrel{!}{\mapsto}$};
    
            \begin{scope}[xshift = 3 cm]
                
                \draw [->] (-2, -2) -- (2,  2) node [above right] {$x_1$};
                \draw [->] (-1,  1) -- (2, -2) node [below right] {$x_2$};

                \filldraw (-1, -1) circle (1 pt) node [above left] {$-1$};
                \filldraw ( 1,  1) circle (1 pt) node [above left] {$ 1$};
                \filldraw ( 1, -1) circle (1 pt) node [below] {$1$};
        
                \filldraw [pattern = dots]
                    (0, -2) --
                    (2,  0) --
                    (1,  0) --
                    (0, -1) --
                    cycle;
    
            \end{scope}    
    
        \end{tikzpicture}

    \end{center}

    Wir wollen mit einer linearen Transformation das Koordinaten-System um $45^\circ$ drehen, um die $x_2$-Achse flippen und mit $\sqrt 2$ strecken.
    Dazu stellen wir folgende Bedingungen an die kanonischen Basisvektoren.

    \begin{align*}
        \begin{pmatrix}
            1 \\ 0
        \end{pmatrix}
        \stackrel{!}{=}
        \begin{pmatrix}
            1 \\ 1
        \end{pmatrix},
        \quad
        \begin{pmatrix}
            0 \\ 1
        \end{pmatrix}
        \stackrel{!}{=}
        \begin{pmatrix}
            1 \\ -1
        \end{pmatrix}
    \end{align*}

    Um die zugehörige Transformation zu erhalten, müssen wir nur die Zielvektoren in die Spalten der Transformations-Matrix schreiben.

    \begin{align*}
        T
        :=
        \begin{pmatrix}
            1 &  1 \\
            1 & -1
        \end{pmatrix}
        \in
        \GL_2(\R)
    \end{align*}

    Unsere Transformation $\varphi$ definieren wir also wie folgt.

    \begin{align*}
        \varphi:
        (x_1, x_2)
        \mapsto
        T
        \begin{pmatrix}
            x_1 \\ x_2
        \end{pmatrix}
        =
        \begin{pmatrix}
            x_1 + x_2 \\ x_1 - x_2
        \end{pmatrix}
        ~\text{Diffeomorphismus},
        \quad
        |\det \mathrm{d} \varphi| = |\det T| = |-1 - 1| = 2
    \end{align*}

    Wir wollen nun das Bild von $A$ unter der Transformation $\varphi$ schöner darstellen, damit wir leichter integrieren können.

    \begin{align*}
        \varphi(A)
        & =
        \Bbraces
        {
            \varphi(x_1, x_2):
            x_1 \geq 0,
            x_2 \leq 0,
            x_2 + 1 \leq x_1 \leq x_2 + 2} \\
        & =
        \Bbraces
        {
            \begin{pmatrix}
                x_1 + x_2 \\ x_1 - x_2
            \end{pmatrix}:
            x_1 \geq 0,
            x_2 \leq 0,
            1 \leq x_1 - x_2 \leq 2
        } \\
        & =
        \Bbraces
        {
            (y_1, y_2):
            \frac{y_1 + y_2}{2} \geq 0,
            \frac{y_1 - y_2}{2} \leq 0,
            1 \leq y_2 \leq 2
        } \\
        & \stackrel{!}{=}
        \Bbraces
        {
            (y_1, y_2):
            y_2 \in [1, 2],
            y_1 \in [-y_2, y_2]
        } \\
    \end{align*}

    Wir haben dabei die Substitution $y_1 := x_1 + x_2$ und $y_2 := x_1 - x_2$ verwendet.

    \begin{align*}
        \implies
        x_1 = \frac{y_1 + y_2}{2},
        \quad
        x_2 = \frac{y_1 - y_2}{2},
        \quad
        x_1 - x_2 = y_2
    \end{align*}

    Die letzte $=$ lässt sich damit wie folgt erklären.

    \begin{align*}
        \frac{y_1 + y_2}{2} \geq 0,
        \quad
        \frac{y_1 - y_2}{2} \leq 0
        \iff
        y_1 \geq -y_2,
        \quad
        y_1 \leq y_2
        \iff
        -y_2 \leq y_1 \leq y_2
    \end{align*}

    \begin{align*}
        \implies
        \Int[A]
        {
            \exp \frac{x_1 + x_2}{x_1 - x_2}
        }{(x_1, x_2)}
        & =
        \frac{1}{2}
        \Int[A]
        {
            \exp \varphi(x_1, x_2)
            |\det \mathrm{d} \varphi(x_1, x_2)|
            }{(x_1, x_2)} \\
        & \stackrel
        {
            \text{TRAFO}
        }{=}
        \frac{1}{2}
        \Int[\varphi(A)]
        {
            \exp \frac{y_1}{y_2}
        }{(y_1, y_2)} \\
        & =
        \frac{1}{2}
        \Int[1][2]
        {
            \Int[-y_2][y_2]
            {
                \exp \frac{y_1}{y_2}
            }{y_1}
        }{y_2} \\
        & =
        \frac{1}{2}
        \Int[1][2]
        {
            y_2 \exp \frac{y_1}{y_2} \Big |_{y_1 = -y_2}^{y_2}
        }{y_2} \\
        & =
        \frac{1}{2}
        \Int[1][2]
        {
            y_2 \pbraces{e - \frac{1}{e}}
        }{y_2} \\
        & =
        \frac{1}{2}
        \frac{1}{2}
        y_2^2 \Big |_{y_1 = 1}^2
        \pbraces{e - \frac{1}{e}} \\
        & =
        \frac{1}{4}
        (4 - 1)
        \pbraces{e - \frac{1}{e}} \\
        & =
        \frac{3}{4}
        \pbraces{e - \frac{1}{e}}
    \end{align*}

    \item Teil:
    
    Die erste Ungleichung von $B$ sagt, dass $x_1$ und $x_2$ dasselbe Vorzeichen haben sollen, schränkt also auf den $1$-ten und $3$-ten Quadranten ein.
    Die zweite Ungleichung schränkt auf die $\sqrt \pi$-Kugel um $0$ bzgl. $\norm[2]{\cdot}$ ein.

    \begin{center}
        \begin{tikzpicture}[scale = 2]
            
            \draw [->] (-2,  0) -- (2, 0) node [right] {$x_1$};
            \draw [->] ( 0, -2) -- (0, 2) node [above] {$x_2$};

            \draw (0, 0) circle (1);

            \filldraw [pattern = dots] (0, 0) -- ( 1, 0) arc (0  :90 :1) -- cycle;
            \filldraw [pattern = dots] (0, 0) -- (-1, 0) arc (180:270:1) -- cycle;

            \draw (1, 0) circle (1 pt) node [above right] {$(\sqrt \pi, 0)$};
            \draw (0, 1) circle (1 pt) node [above right] {$(0, \sqrt \pi)$};

        \end{tikzpicture}
    \end{center}

    Nachdem $\sin$ ungerade ist, und die Fläche ebenfalls, werden sich der obere rechte und untere linke Teil gegenseitig auslöschen.
    Das wollen wir noch im Detail durch-exerzieren.

    \includegraphicsboxed{Ana3/Ana3 - Beispiel 4.3.2.png}

    Wir wollen $B$ mit Polarkoordinaten parametrisieren.
    Das wurde netterweise in Beispiel 4.3.2 bereits gemacht.

    \begin{align*}
        \varphi:
        D
        :=
        (0, \sqrt \pi)
        \times
        \pbraces
        {
            \pbraces
            {
                0,
                \frac{\pi}{2}
            }
            \cup
            \pbraces
            {
                \pi,
                \frac{3 \pi}{2}
            }
        }
        \to
        B
        :
        (r, \theta)
        \mapsto
        r
        \begin{pmatrix}
            \cos \theta \\ \sin \theta
        \end{pmatrix}
        ~\text{Diffeomorphismus}
    \end{align*}

    \begin{align*}
        \Int[B]
        {
            \sin x_1
        }{(x_1, x_2)}
        & =
        \Int[D]
        {
            \sin \varphi_1(r, \theta)
            |\det \mathrm{d} \varphi(r, \theta)|
        }{(r, \theta)} \\
        & =
        \Int[0][\pi]
        {
            r
            \Int[0][\frac{\pi}{2}]
            {
                \sin(r \cos \theta)
            }{\theta}
        }{r}
        +
        \Int[0][\pi]
        {
            r
            \Int[\pi][\frac{3 \pi}{2}]
            {
                \sin(r \cos \theta)
            }{\theta}
        }{r} \\
        & =
        \Int[0][\pi]
        {
            r
            \Int[0][\frac{\pi}{2}]
            {
                \sin(r \cos \theta)
            }{\theta}
        }{r}
        +
        \Int[0][\pi]
        {
            r
            \Int[0][\frac{\pi}{2}]
            {
                \sin
                (
                    r
                    \underbrace
                    {
                        \cos(u + \pi))
                    }_{
                        -\cos u
                    }
            }{u}
        }{r} \\
        & =
        0
    \end{align*}

    Weil der $\sin$ (wie bereits erwähnt) ungerade ist, rutschte das $-$ durch und die Integrale löschen sich gegenseitig aus.

\end{enumerate}

\end{solution}

% -------------------------------------------------------------------------------- %


Der Schwerpunkt eines Objektes $A \subset \R^3$ mit Dichtefunktion $\rho(x)$ ist durch

\begin{align*}
    \pbraces
    {
        \Int[A]{\rho(x)}{\lambda^3(x)}
    }^{-1}
    \Int[A]
    {
        \begin{pmatrix}
            x_1 \\ x_2 \\ x_3
        \end{pmatrix}
        \rho(x)
    }{\lambda^3(x)}
\end{align*}

gegeben.

% -------------------------------------------------------------------------------- %

\begin{exercise}

Berechnen Sie den Schwerpunkt einer homogenen Halbkugel \\
($\rho(x) = 1$, $A = \Bbraces{x: x_1^2 + x_2^2 + x_3^2 \leq R^2, x_3 \geq 0}$) mit Radius $R$ mithilfe von Zylinderkoordinaten.

\end{exercise}

% -------------------------------------------------------------------------------- %

\begin{solution}

\phantom{}

\includegraphicsgraphicsboxed
{Ana3/Ana3 - Beispiel 4.3.3.1.png}
{Ana3/Ana3 - Beispiel 4.3.3.2.png}

Wir wollen $A$ mit Zylinderkoordinaten parametrisieren.
Das wurde netterweise in Beispiel 4.3.3 bereits gemacht.

\begin{align*}
    \psi:
    D \to A:
    (r, \varphi, z)
    \mapsto
    \begin{pmatrix}
        r \cos \varphi \\ r \sin \varphi \\ z
    \end{pmatrix}
\end{align*}

\begin{align*}
    D
    & :=
    \Bbraces
    {
        (r, \varphi, z):
        \varphi \in (0, 2 \pi),
        z \geq 0,
        r^2 + z^2 \leq R^2
    } \\
    & =
    \Bbraces
    {
        (r, \varphi, z):
        r \in \bbraces{0, \sqrt{R^2 - z^2}},
        \varphi \in (0, 2 \pi),
        z \in [0, R]
    } \\
    & =
    \Bbraces
    {
        (r, \varphi, z):
        r \in [0, R],
        \varphi \in (0, 2 \pi),
        z \in \bbraces{0, \sqrt{R^2 - r^2}}
    }
\end{align*}

\begin{align*}
    \Int[A]{\rho(x)}{\lambda^3(x)}
    & \stackrel
    {
        \mathrm{TRAFO}
    }{=}
    \Int[D]{r}{(r, \varphi, z)} \\
    & =
    \Int[0][R]
    {
        \Int[0][2 \pi]
        {
            \Int[0][\sqrt{R^2 - z^2}]
            {
                r
            }{r}
        }{\varphi}
    }{z} \\
    & =
    \Int[0][2 \pi]{}{\varphi}
    \Int[0][R]
    {
        \frac{1}{2}
        r^2 \Big |_{r=0}^{\sqrt{R^2 - z^2}}
    }{z} \\
    & =
    2 \pi
    \frac{1}{2}
    \Int[0][R]
    {
        R^2 - z^2
    }{z} \\
    & =
    \pi
    \pbraces
    {
        R^2
        \Int[0][R]{}{z}
        -
        \Int[0][R]{z^2}{z}
    } \\
    & =
    \pi
    \pbraces
    {
        R^3
        -
        \frac{1}{3}
        z^3 \Big |_{z=0}^R
    } \\
    & =
    \pi
    \pbraces
    {
        R^3
        -
        \frac{1}{3}
        R^3
    } \\
    & =
    \frac{2 \pi R^3}{3}
\end{align*}

\begin{align*}
    \Int[A]
    {
        \begin{pmatrix}
            x_1 \\ x_2 \\ x_3
        \end{pmatrix}
        \rho(x)
    }{\lambda^3(x)}
    \stackrel
    {
        \text{TRAFO}
    }{=}
    \Int[D]
    {
        \begin{pmatrix}
            r \cos \varphi \\ r \sin \varphi \\ z
        \end{pmatrix}
        r
    }{(r, \varphi, z)}
    =
    \Int[0][R]
    {
        \Int[0][2 \pi]
        {
            \Int[0][\sqrt{R^2 - z^2}]
            {
                \begin{pmatrix}
                    r^2 \cos \varphi \\ r^2 \sin \varphi \\ r z
                \end{pmatrix}
            }{r}
        }{\varphi}
    }{z}
\end{align*}

\begin{enumerate}

    \item [1.] und 2. Komponente:
    
    \begin{align*}
        \Int[0][R]
        {
            \Int[0][2 \pi]
            {
                \Int[0][\sqrt{R^2 - z^2}]
                {
                    \begin{pmatrix}
                        r^2 \cos \varphi \\ r^2 \sin \varphi
                    \end{pmatrix}
                }{r}
            }{\varphi}
        }{z}
        =
        \Int[0][R]
        {
            \Int[0][\sqrt{R^2 - z^2}]
            {
                r^2
            }{r}
        }{z}
        \underbrace
        {
            \Int[0][2 \pi]
            {
                \begin{pmatrix}
                    \cos \varphi \\ \sin \varphi
                \end{pmatrix}
            }{\varphi}
        }_0
        =
        0
    \end{align*}

    \item [3.] Komponente:
    
    \begin{align*}
        \Int[0][R]
        {
            \Int[0][2 \pi]
            {
                \Int[0][\sqrt{R^2 - z^2}]
                {
                    r z
                }{r}
            }{\varphi}
        }{z}
        & =
        \Int[0][2 \pi]{}{\varphi}
        \Int[0][R]
        {
            z
            \Int[0][\sqrt{R^2 - z^2}]
            {
                r
            }{r}
        }{z} \\
        & =
        2 \pi
        \Int[0][R]
        {
            z
            \frac{1}{2}
            r^2 \Big |_{r=0}^{\sqrt{R^2 - z^2}}
        }{z} \\
        & =
        \pi
        \Int[0][R]{z (R^2 - z^2)}{z} \\
        & =
        \pi
        \pbraces
        {
            R^2
            \Int[0][R]{z}{z}
            -
            \Int[0][R]{z^3}{z}
        } \\
        & =
        \pi
        \pbraces
        {
            R^2
            \frac{1}{2}
            z^2 \Big |_{z=0}^R
            -
            \frac{1}{4}
            z^4 \Big |_{z=0}^R
        } \\
        & =
        \pi
        \pbraces
        {
            \frac{R^4}{2}
            -
            \frac{R^4}{4}
        } \\
        & =
        \frac{R^4 \pi}{4}
    \end{align*}

\end{enumerate}

\begin{align*}
    \implies
    \pbraces
    {
        \Int[A]{\rho(x)}{\lambda^3(x)}
    }^{-1}
    \Int[A]
    {
        \begin{pmatrix}
            x_1 \\ x_2 \\ x_3
        \end{pmatrix}
        \rho(x)
    }{\lambda^3(x)}
    =
    \frac{3}{2 \pi R^3}
    \frac{R^4 \pi}{4}
    \begin{pmatrix}
        0 \\ 0 \\ 1
    \end{pmatrix}
    =
    \frac{3 R}{8}
    \begin{pmatrix}
        0 \\ 0 \\ 1
    \end{pmatrix}
\end{align*}

\end{solution}

% -------------------------------------------------------------------------------- %

% --------------------------------------------------------------------------------

\begin{exercise}

Berechnen Sie den Schwerpunkt der inhomogenen Halbkugel \\
($\rho(x) = |x|^2$, $A = \Bbraces{x: x_1^2 + x_2^2 + x_3^2 \leq R^2, x_2 \geq 0}$) mit Radius $R$ mithilfe von Kugelkoordinaten.

\end{exercise}

% --------------------------------------------------------------------------------

\begin{solution}

\phantom{}

\includegraphicsgraphicsboxed
{Ana3/Ana3 - Beispiel 4.3.4.1.png}
{Ana3/Ana3 - Beispiel 4.3.4.2.png}

Wir wollen $A$ mit Kugelkoordinaten parametrisieren.
Das wurde netterweise in Beispiel 4.3.4 bereits gemacht.

\begin{align*}
    \psi:
    D := (0, R) \times (0, 2 \pi) \times \pbraces{0, \frac{\pi}{2}}
    \to
    A:
    (r, \varphi, \theta)
    \mapsto
    r
    \begin{pmatrix}
        \cos \theta \cos \varphi \\
        \cos \theta \sin \varphi \\
        \sin \theta
    \end{pmatrix}
\end{align*}

\begin{multline*}
    \rho(\psi(r, \varphi, \theta))
    =
    \abs
    {
        r
        \begin{pmatrix}
            \cos \theta \cos \varphi \\
            \cos \theta \sin \varphi \\
            \sin \theta
        \end{pmatrix}
    }^2
    =
    r^2
    (
        \cos^2 \theta \cos^2 \varphi
        +
        \cos^2 \theta \sin^2 \varphi
        +
        \sin^2 \theta
    ) \\
    =
    r^2
    (
        \cos^2 \theta
        \underbrace
        {
            (
                \cos^2 \varphi
                +
                \sin^2 \varphi
            )
        }_1
        +
        \sin^2 \theta
    )
    =
    r^2
    \underbrace
    {
        (
            \cos^2 \theta
            +
            \sin^2 \theta
        )
    }_1
    =
    r^2
\end{multline*}

\begin{align*}
    \Int[A]{\rho(x)}{\lambda^3(x)}
    & \stackrel
    {
        \mathrm{TRAFO}
    }{=}
    \Int[D]
    {
        \rho(\psi(r, \varphi, \theta))
        r^2 \cos \theta
    }{(r, \varphi, \theta)} \\
    & =
    \Int[0][\frac{\pi}{2}]
    {
        \Int[0][2 \pi]
        {
            \Int[0][R]
            {
                r^4
                \cos \theta
            }{r}
        }{\varphi}
    }{\theta} \\
    & =
    \Int[0][\frac{\pi}{2}]
    {
        \cos \theta
    }{\theta}
    \Int[0][2 \pi]{}{\varphi}
    \Int[0][R]
    {
        r^4
    }{r} \\
    & =
    \sin \theta \Big |_{\theta = 0}^\frac{\pi}{2}
    2 \pi
    \frac{1}{5}
    r^5 \Big |_{r=0}^R \\
    & =
    \frac{2 \pi R^5}{5}
\end{align*}

\begin{multline*}
    \Int[A]
    {
        \begin{pmatrix}
            x_1 \\ x_2 \\ x_3
        \end{pmatrix}
        \rho(x)
    }{\lambda^3(x)}
    \stackrel
    {
        \mathrm{TRAFO}
    }{=}
    \Int[D]
    {
        (\rho \cdot \id)(\psi(r, \varphi, \theta))
        r^2 \cos \theta
    }{(r, \varphi, \theta)} \\
    =
    \Int[0][\frac{\pi}{2}]
    {
        \Int[0][2 \pi]
        {
            \Int[0][R]
            {
                r^5 \cos \theta
                \begin{pmatrix}
                    \cos \theta \cos \varphi \\
                    \cos \theta \sin \varphi \\
                    \sin \theta
                \end{pmatrix}    
            }{r}
        }{\varphi}
    }{\theta}
\end{multline*}

\begin{enumerate}

    \item [1.] und 2. Komponente:

    \begin{align*}
        \Int[0][\frac{\pi}{2}]
        {
            \Int[0][2 \pi]
            {
                \Int[0][R]
                {
                    r^5
                    \cos \theta
                    \begin{pmatrix}
                        \cos \theta \cos \varphi \\
                        \cos \theta \sin \varphi
                    \end{pmatrix}
                }{r}
            }{\varphi}
        }{\theta}
        =
        \Int[0][\frac{\pi}{2}]
        {
            \cos^2 \theta
        }{\theta}
        \underbrace
        {
            \Int[0][2 \pi]
            {
                \begin{pmatrix}
                    \cos \varphi \\
                    \sin \varphi
                \end{pmatrix}
            }{\varphi}
        }_0
        \Int[0][R]
        {
            r^5
        }{r}
        =
        0
    \end{align*}

    \item [3.] Komponente:
    
    \begin{align*}
        \Int[0][\frac{\pi}{2}]
        {
            \Int[0][2 \pi]
            {
                \Int[0][R]
                {
                    r^5
                    \cos \theta
                    \sin \theta
                }{r}
            }{\varphi}
        }{\theta}
        =
        \Int[0][\frac{\pi}{2}]
        {
            \cos \theta
            \sin \theta
        }{\theta}
        \Int[0][2 \pi]{}{\varphi}
        \Int[0][R]
        {
            r^5
        }{r}
        =
        \frac{1}{2}
        2 \pi
        \frac{1}{6}
        r^6 \Big |_{r=0}^R
        =
        \frac{\pi R^6}{6}
    \end{align*}

\end{enumerate}    

\begin{align*}
    \implies
    \pbraces
    {
        \Int[A]{\rho(x)}{\lambda^3(x)}
    }^{-1}
    \Int[A]
    {
        \begin{pmatrix}
            x_1 \\ x_2 \\ x_3
        \end{pmatrix}
        \rho(x)
    }{\lambda^3(x)}
    =
    \frac{5}{2 \pi R^5}
    \frac{\pi R^6}{6}
    \begin{pmatrix}
        0 \\ 0 \\ 1
    \end{pmatrix}
    =
    \frac{5 R}{12}
    \begin{pmatrix}
        0 \\ 0 \\ 1
    \end{pmatrix}
\end{align*}

\end{solution}

% --------------------------------------------------------------------------------


\printbibliography

\end{document}
