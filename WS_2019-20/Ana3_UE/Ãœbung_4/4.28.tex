% -------------------------------------------------------------------------------- %

\begin{exercise}

Bestimmen Sie alle Lösungen der Gleichung $f \ast f = c f$ für $f \in L^1(\T)$, $c \in \C$.
Für $c \neq 0$ können dann wegen dem Riemann Lebesguelemma nur endlich viele Koeffizienten ungl. 0 sein.
$f$ ist also ein trigonometrisches Polynom mit allen nichtverschwindenden Koeffizienten gleich $\frac{c}{\sqrt{2 \pi}}$.

\end{exercise}

% -------------------------------------------------------------------------------- %

\begin{solution}

\phantom{}

\begin{enumerate}[label = \arabic*.]

    \item Fall ($c = 0$):
    
    \includegraphicsboxed{Ana3/Ana3 - (3.1).png}
    \includegraphicsboxed{Ana3/Ana3 - Fourier-Series Cheat-Sheet - 1.png}
    \includegraphicsboxed{Ana3/Ana3 - Satz 3.1.2 (Riemann-Lebesgue Lemma).png}
    \includegraphicsboxed{Ana3/Ana3 - Satz 3.1.5.png}

    \begin{align*}
        f \ast f = c f = 0
        \iff
        \pbraces
        {
            \Forall n \in \Z:
            \sqrt{2 \pi} \hat f(n) \hat f(n) \stackrel{\mathrm{3.1.5}}{=} \widehat{f \ast f}(n) = 0
            \iff
            \hat f(n) = 0    
        }
        \iff
        f = 0
    \end{align*}

    \item Fall ($c \neq 0$):
    
    \begin{align*}
        \Forall n \in \Z:
        \sqrt{2 \pi} \hat f(n) \hat f(n)
        \stackrel{
            \mathrm{3.1.5}
        }{=}
        \widehat{f \ast f}(n)
        =
        \widehat{c f}(n)
        =
        c \hat f(n)
    \end{align*}

    \begin{align*}
        Z
        :=
        \supp \hat f
        =
        \Bbraces{n \in \Z: \hat f(n) \neq 0}
    \end{align*}

    \begin{align*}
        \Forall n \in Z:
        \hat f(n) = \frac{c}{\sqrt{2 \pi}}
    \end{align*}

    \begin{align*}
        \begin{rcases}
            \stackrel
            {
                \mathrm{3.1.2}
            }{\implies}
            & \lim_{|n| \to \infty} \hat f(n) = 0 \\
            & \hat f = \frac{c}{\sqrt{2 \pi}} \1_Z
        \end{rcases}
        \implies
        Z \in \mathcal{E}(\Z)
    \end{align*}

    \begin{align*}
        f(x)
        =
        \sum_{n \in \Z}
        \hat f(n)
        e^{-i n x}
        =
        \sum_{n \in Z}
        \hat f(n)
        e^{-i n x}
        =
        \sum_{n \in Z}
        \frac{c}{\sqrt{2 \pi}}
        e^{-i n x}
    \end{align*}

\end{enumerate}

\end{solution}

% -------------------------------------------------------------------------------- %
