% -------------------------------------------------------------------------------- %

\begin{exercise}

Sei $\mu$ ein endliches Maß auf $\R$, $a, b \in \R$, $a < b$, und sei für $y \in (0, 1)$

\begin{align*}
    g(y)
    =
    \Int[\R]
    {
        (1 + t^2)
        \bbraces
        {
            \frac{1}{\pi}
            \arctan \frac{b - t}{y}
            -
            \frac{1}{\pi}
            \arctan \frac{a - t}{y}
        }
    }{
        \mu(t)
    }.
\end{align*}

Zeigen Sie

\begin{align*}
    \lim_{y \searrow 0} g(y)
    =
    \Int[(a, b)]{(1 + t^2)}{\mu(t)}
    +
    \frac{1}{2}
    \Int[\Bbraces{a, b}]{(1 + t^2)}{\mu(t)}
\end{align*}

Hinweis:
Berechnen Sie $\lim_{y \searrow 0}$ des Integranden.

Zeigen Sie mit Hilfe des Mittelwertsatzes def Differentialrechnung, dass der Integrand durch eine von $y$ unabhängige Konstante beschränkt ist.
Unterscheiden Sie dazu die Fälle, $t \in [a - 1, b + 1]$, $t \in (-\infty, a - 1)$ und $t \in (b + 1, +\infty)$.

\end{exercise}

% -------------------------------------------------------------------------------- %

\begin{solution}

\phantom{}

\begin{align*}
    f_y(t)
    :=
    (1 + t^2)
    \pbraces
    {
        \frac{1}{\pi}
        \arctan \frac{b - t}{y}
        -
        \frac{1}{\pi}
        \arctan \frac{a - t}{y}
    },
    \quad
    y \in (0, 1),
    \quad
    t \in \R
\end{align*}

\begin{enumerate}[label = \arabic*.]

    \item Fall ($t \in (-\infty, a - 1)$):
    
    Laut dem Mittelwertsatze der Differentialrechnung, $\Exists \zeta_t \in (a, b):$

    \begin{align*}
        \frac
        {
            \arctan \frac{b - t}{y}
            -
            \arctan \frac{a - t}{y}
        }{
            b - a
        }
        =
        \pderivative{x}
        \arctan \frac{x - t}{y}
        \Big |_{x = \zeta_t}
        =
        \frac{1}
        {
            1
            +
            \pbraces
            {
                \frac{\zeta_t - t}{y}
            }^2
        }
        \frac{1}{y}
        =
        \frac{y}{y^2 + (\zeta_t - t)^2}.
    \end{align*}

    Nun ist $y \in (0, 1)$.

    \begin{align*}
        \implies
        |f_y(t)|
        =
        \abs
        {
            \frac{1}{\pi}
            (1 + t^2)
            (b - a)
            \frac{y}{y^2 + (\zeta_t - t)^2}
        }
        \leq
        \frac{1}{\pi}
        (1 + t^2)
        (b - a)
        \frac{1}{(\zeta_t - t)^2}
        =:
        h(t)
    \end{align*}

    Die Majorante $h$ ist auch integrierbar.

    \begin{align*}
        \Int[(-\infty, a-1)]{h}{\mu}
        \leq
        \frac{1}{\pi}
        (b - a)
        \underbrace
        {
            \sup_{t \in (-\infty, a-1)}
            \frac{1 + t^2}{(\zeta_t - t)^2}
        }_{
            < \infty
        }
        \underbrace
        {
            \mu((-\infty, a-1))
        }_{
            < \infty
        }
        <
        \infty
    \end{align*}

    Dabei beachte man, dass das Maß $\mu$ endlich ist und das $\sup$ ebenfalls, weil

    \begin{align*}
        \lim_{t \to -\infty}
        \frac{1 + t^2}{(\zeta_t - t)^2}
        =
        \lim_{t \to \infty}
        \frac{1 + t^2}{\zeta_t^2 + 2 \zeta_t t + t^2}
        \stackrel
        {
            \mathrm{L'Hospital}
        }{=}
        \lim_{t \to \infty}
        \frac{2 t}{2 t + 2 \zeta_t}
        =
        \lim_{t \to \infty}
        \frac{2}{2}
        =
        1.
    \end{align*}

    \item Fall ($t \in [a-1, b+1]$):
    
    \begin{align*}
        |f_y(t)|
        \leq
        (1 + t)^2
        \frac{1}{\pi}
        \Bigg (
            \underbrace
            {
                \abs{\arctan \frac{b - t}{y}}
            }_{
                \leq \pi / 2
            }
            +
            \underbrace
            {
                \abs{\arctan \frac{a - t}{y}}
            }_{
                \leq \pi / 2
            }
        \Big )
        \leq
        (1 + t)^2
        =:
        h(t)
    \end{align*}

    Die Majorante $h$ ist auch integrierbar.

    \begin{align*}
        \Int[{[a-1, b+1]}]{h}{\mu}
        \leq
        \sup_{t \in [a-1, b+1]} h(t)
        \mu([a-1, b+1])
        <
        \infty
    \end{align*}

    Dabei beachte man, dass das Maß $\mu$ endlich ist und das $\sup$ ebenfalls, weil $h$ stetig auf dem kompakten $[a-1, b+1]$ ist.

    \item Fall ($t \in (b+1, \infty)$):
    
    Dieser Fall ist analog zu Fall 1.

\end{enumerate}

\includegraphicsboxed{Ana3/Ana3 - Satz 2.1.5.png}

Wir können nun Satz 2.1.5 auf die Familie stetiger, also messbarer, Funktionen $(f_y)_{y \in (0, 1)}$ anwenden.

\begin{align*}
    \mathrm{lhs}
    =
    \Int[(-\infty, a)]
    {
        \lim_{y \searrow 0}
        f_y(t)
    }{\mu(t)}
    +
    \Int[(a, b)]
    {
        \lim_{y \searrow 0}
        f_y(t)
    }{\mu(t)}
    +
    \Int[(b, +\infty)]
    {
        \lim_{y \searrow 0}
        f_y(t)
    }{\mu(t)}
    +
    \Int[\Bbraces{a, b}]
    {
        \lim_{y \searrow 0}
        f_y(t)
    }{\mu(t)}
    \stackrel{!}{=}
    \mathrm{rhs}
\end{align*}

\begin{enumerate}[label = \arabic*.]

    \item Integral:
    
    \begin{align*}
        \Forall t \in (-\infty, a):
        b - t, a - t > 0
        \implies
        \Int[(-\infty, a)]
        {
            \lim_{y \searrow 0}
            f_y(t)
        }{\mu(t)}
        =
        0            
    \end{align*}

    \item Integral:
    
    \begin{align*}
        \Forall t \in (a, b):
        b - t > 0,
        a - t < 0
        \implies
        \Int[(a, b)]
        {
            \lim_{y \searrow 0}
            f_y(t)
        }{\mu(t)}
        =
        \Int[(a, b)]
        {
            (1 + t)^2
        }{\mu(t)}        
    \end{align*}

    \item Integral:
    
    \begin{align*}
        \Forall t \in (b, +\infty):
        b - t, a - t < 0
        \implies
        \Int[(b, +\infty)]
        {
            \lim_{y \searrow 0}
            f_y(t)
        }{\mu(t)}
        =
        0            
    \end{align*}

    \item Integral:

    \begin{multline*}
        \Forall t \in \Bbraces{a, b}:
        b - t
        \begin{cases}
            = 0, & t = b \\
            > 0, & t = a
        \end{cases},
        \quad
        a - t
        \begin{cases}
            = 0, t = a \\
            < 0, t = b
        \end{cases} \\
        \implies
        \Int[\Bbraces{a, b}]
        {
            \lim_{y \searrow 0}
            f_y(t)
        }{\mu(t)}
        =
        \frac{1}{2}
        \Int[\Bbraces{a, b}]{(1 + t)^2}{\mu(t)}
    \end{multline*}

\end{enumerate}

\end{solution}

% -------------------------------------------------------------------------------- %
