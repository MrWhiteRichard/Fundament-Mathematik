% -------------------------------------------------------------------------------- %

\begin{exercise}

Zeigen Sie für die Betafunktion $B(x, y)$

\begin{align*}
    \pderivative[n]{x}
    B(x, y)
    =
    \Int[0][1]
    {
        t^{x-1}
        (1 - t)^{y-1}
        (\ln t)^n
    }{t}.
\end{align*}

\end{exercise}

% -------------------------------------------------------------------------------- %

\begin{solution}

Wir wollen wieder Satz 2.1.7 (siehe oben) anwenden; diesmal auf die folgende Funktionen-Familie.

\begin{align*}
    f_x(t) := t^{x-1} (1 - t)^{y-1},
    \quad
    x \in (0, 1],
    \quad
    t \in (0, \infty)
\end{align*}

Diese besteht aus stetigen, also insbesondere messbaren, Funktionen.
Seien nun $x_0 \in (0, \infty)$ und $\delta > 0$, sodass $B(x_0, \delta) \subset (0, \infty)$.
Wir nehmen o.B.d.A. an, dass $0 < x, y \leq 1$, weil sonst schätzen wir ab mit $t^{x-1} \leq 1$ bzw. $(1 - t)^{y-1} \leq 1$.

\begin{align*}
    \implies
    \abs
    {
        \pderivative[n][f_x(t)]{x}
    }
    & =
    t^{x-1} (1 - t)^{y-1} |\ln t|^n \\
    & \leq
    (
        \1_{(0, 1/2]}(t)
        t^{x-1} (1/2)^{y-1}
        +
        \1_{[1/2, 1)}(t)
        (1/2)^{x-1} (1 - t)^{y-1}
    )
    |\ln t|^n \\
    & \leq
    \1_{(0, 1/2]}(t)
    t^{x-1} (1/2)^{y-1}
    |\ln t|^n
    +
    \1_{(1/2, 1)}(t)
    (1/2)^{x-1} (1 - t)^{y-1}
    |\ln(1 - t)|^n \\
    & =:
    g_n(t)
\end{align*}

Dabei haben wir verwendet, dass

\begin{align*}
    \Forall t \in (1/2, 1):
    1 \leq 2 t
    \implies
    1 - t \leq t
    \implies
    \ln(1 - t) \leq \ln t
\end{align*}

Die Majorante $g_n$ ist auch integrierbar.

\begin{align*}
    \Int[0][1]{g_n(t)}{t}
    =
    \underbrace
    {
        \Int[0][1/2]
        {
            t^{x-1} (1/2)^{y-1} |\ln t|^n
        }{t}    
    }_{
        =: I_1
    }
    +
    \underbrace
    {
        \Int[1/2][1]
        {
            (1/2)^{x-1} (1 - t)^{y-1} |\ln(1 - t)|^n
        }{t}
    }_{
        =: I_2
    }
    \stackrel{!}{<}
    \infty
\end{align*}

\begin{enumerate}[label = \arabic*.]

    \item Integral:

    \begin{align*}
        2^{y-1} I_1
        =
        (-1)^n
        \Int[0][1/2]{t^{x-1} (\ln t)^n}{t}
        =
        (-1)^n
        \Bigg (
            \frac{1}{x}
            t^x (\ln t)^n
            -
            \frac{n}{x}
            \Int[0][1/2]{t^{x-1} (\ln t)^{n-1}}
        \Bigg )
        \stackrel{!}{<}
        \infty
    \end{align*}
    
    Dabei haben wir verwendet, dass
    
    \begin{align*}
        \lim_{t \to 0}
        t^x (\ln t)^n
        =
        \lim_{t \to 0}
        \frac
        {
            (\ln t)^n
        }{
            t^{-x}
        }
        \stackrel
        {
            \mathrm{L'Hospital}
        }{=}
        \lim_{t \to 0}
        \frac
        {
            \frac{n}{t (\ln t)^{n-1}}
        }{-x t^{-x - 1}}
        =
        -\frac{n}{x}
        \lim_{t \to 0}
        \frac
        {
            (\ln t)^{n-1}
        }{
            t^{-x}
        }
        =
        \cdots
        =
        0
    \end{align*}
    
    Damit ist der Integrand auf $0$ stetig fortsetzbar und auf dem kompakten $[0, 1/2]$ integrierbar.

    \item Integral:

    \begin{align*}
        2^{x-1} I_2
        =
        (-1)^n
        \Int[1/2][1]
        {
            (1 - t)^{y-1} (\ln(1 - t))^n
        }{t}
        =
        (-1)^n
        \Int[0][1/2]
        {
            u^{y-1} (\ln u)^n
        }{u}
        <
        \infty
    \end{align*}
    
    Dabei haben wir die folgende Substitution verwendet.
    
    \begin{align*}
        u = 1 - t
        \implies
        \derivative[][u]{t} = -1
        \implies
        \mathrm{d}t = -\mathrm{d}u
    \end{align*}

\end{enumerate}

Laut Satz 2.1.7, dürfen wir die Ableitungen ins Integral ziehen.

\begin{align*}
    \implies
    \mathrm{lhs}
    =
    \pderivative[n]{x}
    \Int[0][1]
    {
        t^{x-1} (1 - t)^{y-1}
    }{t}
    =
    \mathrm{rhs}
\end{align*}

\end{solution}

% -------------------------------------------------------------------------------- %
