% --------------------------------------------------------------------------------

\begin{exercise}

Für $a, t > 0$ sei

\begin{align*}
    u(t, a)
    =
    \Int[0][+\infty]
    {
        \frac{t}{t^2 + x^2}
        \cos{a x}
    }{x}.
\end{align*}

Zeigen Sie, dass $u: (0, +\infty) \times (0, +\infty) \to \R$ stetig ist und dass für jedes feste $a > 0$ die Funktion $t \mapsto u(t, a)$ differenzierbar ist.

Weiters berechne man $\lim_{a \to 0} u(t, a)$.

\end{exercise}

% --------------------------------------------------------------------------------

\begin{solution}

\phantom{}

\begin{enumerate}[label = \arabic*.]

    \item Teil ($u$ stetig):
    
    \begin{multline*}
        \text{d.h.}~
        \Forall a, t \in (0, +\infty):
        \Forall (t_n)_{n \in \N}, (a_n)_{n \in \N} \in (0, +\infty)^\N: \\
        \pbraces
        {
            t_n \xrightarrow{n \to \infty} t,
            a_n \xrightarrow{n \to \infty} a
        }
        \implies
        \lim_{n \to \infty} u(t_n, a_n) = u(t, a)
    \end{multline*}

    Weil $(a_n)_{n \in \N}$ konvergiert, finden wir ein Intervall $(\alpha, \beta) \subset (0, +\infty)$, sodass für fast alle (o.B.d.A. alle) $n \in \N$, $t_n \in (\alpha, \beta)$.

    \includegraphicsboxed{MassWHT1&2/MassWHT1&2 - Satz 5.7.png}

    Wir wollen Satz 5.7 (Satz von der dominierten Konvergenz) auf die folgende Folge von stetigen, also messbaren, Funktionen anwenden.

    \begin{align*}
        f_n(x)
        :=
        \frac{t_n}{t_n^2 + x^2}
        \cos(a_n x),
        \quad
        x \in (0, +\infty),
        \quad
        n \in \N
    \end{align*}

    \begin{align*}
        \implies
        |f_n(x)|
        =
        \abs
        {
            \frac{t_n}{t_n^2 + x^2}
        }
        \underbrace
        {
            |\cos(a_n x)|
        }_{
            \leq 1
        }
        \leq
        \frac{\beta}{\alpha^2 + x^2}
        =:
        g(x)
    \end{align*}

    Die Majorante $g$ ist auch integrierbar.

    \begin{multline*}
        \Int[0][\infty]{g(x)}{x}
        =
        \Int[0][\infty]
        {
            \frac{\beta}{\alpha^2 + x^2}
        }{x}
        =
        \frac{\beta}{\alpha^2}
        \Int[0][\infty]
        {
            \frac{1}{1 + (x / \alpha)^2}
        }{x} \\
        \stackrel{!}{=}
        \frac{\beta}{\alpha}
        \Int[0][\infty]{\frac{1}{1 + u^2}}{u}
        =
        \frac{\beta}{\alpha}
        \arctan u \Big |_{u=0}^\infty
        =
        \frac{\beta}{\alpha}
        \frac{\pi}{2}
        <
        \infty
    \end{multline*}

    Dabei haben wir folgende Substitution verwendet.

    \begin{align*}
        u = \frac{x}{\alpha}
        \implies
        \derivative[][u]{x} = \frac{1}{\alpha}
        \implies
        \mathrm{d}x = \alpha \mathrm{d}u
    \end{align*}

    \item Teil ($\Forall a > 0: u(\cdot, a)$ differenzierbar):
    
    Seien $t_0 \in (0, +\infty)$ und $\delta > 0$, sodass $B(t_0, \delta) \subset (0, +\infty)$.
    Wir wollen wieder Satz 2.1.7 (siehe oben) anwenden; diesmal auf die folgende Funktionen-Familie.

    \begin{align*}
        f_t(x)
        :=
        \frac{t}{t^2 + x^2}
        \cos(a x)
        =
        \frac{1}{t + x^2 / t}
        \cos(a x),
        \quad
        x, t \in (0, +\infty)
    \end{align*}

    Wir finden eine, von $t$ unabhängige, Majorante.

    \begin{align*}
        \abs
        {
            \pderivative[][f_t]{t}(x)
        }
        \leq
        \abs
        {
            \frac{-1}{(t + x^2 / t)^2}
            (1 - x^2 / t^2)
            \cos(a x)
        }
        =
        \abs
        {
            \frac{x^2 - t^2}{(x^2 + t^2)^2}
        }
        \underbrace{|\cos(a x)|}_{\leq 1}
        \leq
        \frac
        {
            (t_0 + \delta)^2 + x^2
        }{
            ((t_0 - \delta)^2 + x^2)^2
        }
        =:
        g(x)
    \end{align*}

    Die Majorante $g$ ist auch integrierbar.

    \begin{multline*}
        \Int[0][\infty]{g(x)}{x}
        \stackrel
        {
            \mathrm{WolframAlpha}
        }{=}
        \frac{1}{4 (t_0 - \delta)^3}
        \pi
        (
            (t_0 + \delta)^2
            +
            (t_0 - \delta)^2
        ) \\
        =
        \frac{1}{4 (t_0 - \delta)^3}
        \pi
        (
            t_0^2 + 2 t_0 \delta + \delta^2
            +
            t_0^2 - 2 t_0 \delta + \delta^2
        )
        =
        \frac{\pi}{2}
        \frac{t_0^2 + \delta^2}{(t_0 - \delta)^3}
        <
        \infty
    \end{multline*}

    \item Teil ($\lim_{a \to 0} u(t, a)$):
    
    Wir wissen ja bereits, dass $u$ stetig ist.
    Daher dürfen dir den $\lim$ in $u$ hineinziehen.

    \begin{align*}
        \implies
        \lim_{a \to 0} u(t, a)
        =
        u(t, 0)
        =
        \Int[0][\infty]
        {
            \frac{t}{t^2 + x^2}
        }{x}
        =
        \frac{1}{t}
        \Int[0][\infty]
        {
            \frac{1}{1 + (x / t)^2}
        }{x}
        \stackrel{!}{=}
        \arctan u \Big |_{u=0}^\infty
        =
        \frac{\pi}{2}
    \end{align*}

    Dabei haben wir folgende Substitution verwendet.

    \begin{align*}
        u = \frac{x}{t}
        \implies
        \derivative[][u]{x} = \frac{1}{t}
        \implies
        \mathrm{d}x = t \mathrm{d}x
    \end{align*}

\end{enumerate}

\end{solution}

% --------------------------------------------------------------------------------
