\begin{exercise}

Zeigen Sie, dass $f \in L^p(\Omega), 1 < p < \infty$ genau dann in $W^{m,p}(\Omega)$ liegt, wenn die Abbildungen $\varphi \mapsto \int_{\Omega}fD^{\alpha}\varphi d\lambda^n$ fpr $|\alpha| \leq m$ stetig vom Raum der Testfunktionen versehen mit der $L^q$-Norm nach $\mathbb{R}$ ist. \\

Hinweis: Verwenden Sie, dass der Dualraum von $L^p$ der $L^q$ ist, das heißt jede beschränkte lineare Abbildung von einem dichten Teilraum des $L^p$ nach $\mathbb{C}$ ist von der Form $\varphi \mapsto \int \varphi g$ mit $g \in L^q$.

\end{exercise}

\begin{solution}

Beginnen wir mit $\Rightarrow$ : Dazu verwenden wir die Hölder-Ungleichung
und erhalten:
\begin{equation*}
  \vbraces{\int_{\Omega}fD^{\alpha}\varphi d\lambda^n} =
  \vbraces{\int_{\Omega}D^{\alpha}f\varphi d\lambda^n} \leq
  \int_{\Omega}\vbraces{D^{\alpha}f}\vbraces{\varphi}d\lambda^n \leq
  \norm[p]{D^{\alpha}f}\norm[q]{\varphi}
\end{equation*}
Somit ist die Abbildung in der $q$-Norm stetig. \\ \\
Um nun $\Leftarrow$ zu zeigen, verwenden wir
\end{solution}
