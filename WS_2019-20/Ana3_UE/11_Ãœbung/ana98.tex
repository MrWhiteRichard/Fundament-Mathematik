\begin{exercise}

Zeigen Sie, dass es eine eindeutige Lösung $u$ der Gleichung

\begin{equation*}
  u(x) = x + \frac{1}{2}\sin(u(x)+x)
\end{equation*}

in $C[-1,1]$ gibt.

\end{exercise}

\begin{solution}

Setzen wir
\begin{equation*}
  T(u) :=  x + \frac{1}{2}\sin(u(x)+x)
\end{equation*}
so sehen wir dass $T: C[-1,1] \rightarrow C[-1,1]$.
Aus dem MWS der Differentialrechnung erhalten wir für $a,b \in \R$
(Da $\sin^\prime = \cos$)
\begin{align*}
  \exists \xi \in [-1,1]:
  \vbraces{\frac{\sin(a)-\sin(b)}{a-b}} = \vbraces{\cos(\xi)} \Rightarrow
  \vbraces{\frac{\sin(a)-\sin(b)}{a-b}} \leq 1 \Leftrightarrow
  \vbraces{\sin(a)-\sin(b)} \leq \vbraces{a-b}
\end{align*}
Seien nun $u,v \in C[-1,1]$ so gilt
\begin{align*}
  d(T(u),T(v)) = \norm{\frac{1}{2}(\sin(u(x)+x)-\sin(v(x)+x))}{\infty}
  \leq \frac{1}{2}\norm{u(x)+x-(v(x)+x)}{\infty}
  = \frac{1}{2}\norm{u(x)-v(x)}{\infty}
\end{align*}
Somit ist die Abbildung eine Kontraktion mit Kontraktionsfaktor
$\kappa = \frac{1}{2}$ und es existiert laut Banach'schem Fixpunktsatz eine
eindeutige Lösung.
\end{solution}
