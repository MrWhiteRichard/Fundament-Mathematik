\begin{exercise}

Sind $\Omega\subseteq\mathbb{R}^n$ und $u,v \in W^{1,2}(\Omega)$, so ist $uv \in W^{1,1}(\Omega)$ mit $D_i(uv) = D_iuv + uD_iv$

\end{exercise}

\begin{solution}
Um zu zeigen, dass $uv\in L^1(\Omega)$ ist benötigt man nur einmal die Ungleichung von Hölder (siehe Kusolitsch Satz 13.4) um 
\begin{align*}
    \norm[1]{uv}\leq\norm[2]{u}\norm[2]{v}<\infty
\end{align*}
einzusehen.

Für den zweiten Teil wollen wir zuerst den Fall betrachten, dass $u\in W^{1,2}(\Omega)$ und $v\in W^{1,2}\Omega\cap C^\infty(\Omega)$. Wählen wir eine beliebige Testfunktion $\Phi\in C_c^\infty(\Omega)$ so gilt auch $v\Phi\in C_c^\infty(\Omega)$ und daher auch $D_i(c\Phi)\in C_c^\infty(\Omega)$. Mit diesem Wissen und Proposition 6.2.6 aus dem Analysis 3 Skriptum vom Professor Blümlinger, das es erlaubt $D_i(v\Phi)=D_iv\Phi+vD_i\Phi$ zu schreiben, erhalten wir
\begin{align*}
    \int(D_iuv+uD_iv)\Phi\mathrm{d}\lambda&=\int(D_iuv\Phi+uD_iv\Phi)\mathrm{d}\lambda\\
    &=\int(D_iuv\Phi+uD_i(v\Phi)-uvD_i\Phi)\mathrm{d}\lambda\\
    &=\int D_iuv\Phi\mathrm{d}\lambda+\int uD_i(v\Phi)\mathrm{d}\lambda-\int uvD_i\Phi\mathrm{d}\lambda\\
    &=\int D_iuv\Phi\mathrm{d}\lambda-\int D_iuv\Phi\mathrm{d}\lambda-\int uvD_i\Phi \mathrm{d}\lambda\\
    &=-\int uvD_i\Phi \mathrm{d}\lambda
\end{align*}
Wieder ganz einfach mit der Linearität des Integrals und der Ungleichung von Hölder erhalten wir $D_iuv+uD_iv\in L^1(\Omega)$ und damit folgt bereits unmittelbar die gewünschte Aussage $D_i(uv)=D_iuv+uD_iv$.

Nun betrachten wir den Allgemeinen Fall, nämlich $u,v\in W^{1,2}(\Omega)$. Nach dem Satz von Meyers-Serrin (siehe Analysis 3 Skriptum Blümlinger Satz 6.2.8) wissen wir, dass $W^{1,2}(\Omega)\cap C^\infty(\Omega)$ dicht in $W^{1,2}(\Omega)$ liegt. Deshalb können wir eine Folge $(v_l)$ aus $W^{1,2}(\Omega)\cap C^\infty(\Omega)$ finden mit $lim_{l\to\infty}v_l=v$ in $W^{1,2}(\Omega)$. Natürlich ist $(v_l)$ in $W^{1,2}(\Omega)$ auch eine Cauchy-Folge und daher gilt für ein beliebiges $\epsilon>0$ und hinreichend große $m,l\in\mathbb{N}$ unter Benützung der Ungleichung von Hölder
\begin{align*}
    \norm[1]{uv_l-uv_m}&\leq\norm[2]{u}\norm[2]{v_l-v_m}\leq\norm[1,2]{u}\norm[1,2]{v_l-v_m}<\frac{\epsilon}{n+1},\\
    \norm[1]{D_i(uv_l-uv_m)}&\leq\norm[1]{D_iuv_l-D_iuv_m}+\norm[1]{uD_iv_l-uD_iv_m}\\
    &\leq\norm[2]{D_iu}\norm[1,2]{v_l-v_m}+\norm[1,2]{u}\norm[1,2]{v_l-v_m}<\frac{\epsilon}{n+1}
\end{align*}
Damit folgt sofort
\begin{align*}
    \norm[1,1]{uv_l-uv_m}^\sim=\norm[1]{uv_l-uv_m}+\sum_{i=1}^n\norm[1]{D_i(uv_l-uv_m)}<\epsilon,
\end{align*}
also dass $uv_l$ eine Cauchy-Folge in $W^{1,1}(\Omega)$ ist. Da nach Satz 6.2.3 gilt, dass $W^{1,1}(\Omega)$ ein Banachraum ist konvergiert also $uv_l\to u\tilde{v}\in W^{1,1}(\Omega)$. Wieder mit der Ungleichung von Hölder erkennt man, dass $u\tilde{v}=uv$ gelten muss (hab ich nicht nachgeprüft). Also wissen wir jetzt, dass $uv\in W^{1,1}(\Omega)$ gilt. Genauso kann man zeigen, dass $D_iuv_l+uD_iv_l$ im Banachraum $L^1(\Omega)$ gegen $D_iuv+uD_iv$ konvergiert (hab ich nicht nachgeprüft). Nun können wir nützen, dass für eine beliebige Testfunktion $\Phi \in C_c^\infty$ die Funktion $\zeta:W^{1,1}(\Omega)\to\mathbb{R}:u\mapsto\int u\Phi\mathrm{d}\lambda$ eine stetiges lineares Funktional auf $W^{1,1}(\Omega)$ ist (vgl. Analysis 3 Skriptum Blümlinger Seite 131). Mit der Ungleichung von Hölder erkennt man auch leicht, dass $\xi:L^1(\Omega)\to\mathbb{R}:u\mapsto\int u\Phi\mathrm{d}\lambda$ ein stetiges lineares Funktional auf $L^1(\Omega)$ ist. Damit sind die folgenden Grenzwertvertauschungen erlaubt und wir dürfen
\begin{align*}
    \int (D_iuv+uD_iv)\Phi\mathrm{d}\lambda&=\int \lim_{n\to\infty}(D_iuv_n+uD_iv_n)\Phi\mathrm{d}\lambda\\
    &=\lim_{n\to\infty}\int D_i(uv_n)\Phi\mathrm{d}\lambda\\
    &=\lim_{n\to\infty}\int uv_nD_i\Phi\mathrm{d}\lambda\\
    &=\int\lim_{n\to\infty}uv_nD_i\Phi\mathrm{d}\lambda\\
    &=\int uvD_i\Phi\mathrm{d}\lambda
\end{align*} 
schreiben und sind damit fertig.
\end{solution}
