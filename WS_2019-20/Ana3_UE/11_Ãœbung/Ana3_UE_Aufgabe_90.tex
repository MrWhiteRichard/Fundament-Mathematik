\begin{exercise}

Sind $\Omega\subseteq\mathbb{R}^n$ und $u,v \in W^{1,2}(\Omega)$, so ist $uv \in W^{1,1}(\Omega)$ mit $D_i(uv) = D_iuv + uD_iv$

\end{exercise}

\begin{solution}
Durch die Hölder-Ungleichung und $u, v \in L^2(\Omega)$, erhält man unmittelbar, dass $\norm[1]{uv} \leq \norm[2]{u} \norm[2]{v} < \infty$, also $uv \in L^1(\Omega)$. \\

Laut Meyers, Serrin, liegt der Unterraum $W^{m, p}(\Omega) \cap C^\infty(\Omega)$, für $1 \leq p < \infty$ und $\Omega$ offen in $\R^n$, dicht in $W^{m, p}(\Omega)$. $u, v$ lassen sich also durch glatte Funktionen, in der Sobolevnorm $\norm[m, p]{\cdot}$, approximieren.

\begin{align*}
  \norm[m, p, \Omega]{f} =
  \norm[m, p]{f} & =
  \pbraces
  {
    \sum_{|\alpha| \leq m}
    \norm[p]{D^\alpha f}^p
  }^\frac{1}{p} \:
  \text{für} \:
  1 \leq p < \infty, \\
  \norm[m, \infty]{f} & =
  \max_{|\alpha| \leq m}
  \norm[\infty]{D^\alpha f}
\end{align*}

\begin{align*}
  \Exists (u_k), (v_k) \in C^\infty(\Omega):
  \lim_{k \to \infty} u_k = u, \:
  \lim_{k \to \infty} v_k = v
\end{align*}

Die oberen Limitten rutschen aber ins Integral rein, weil aufgrund der Hölder-Ungleichung und $q := \frac{p}{1 - p}$ gilt $\Forall \phi \in C^\infty(\Omega): \Forall (f_k) \in C^\infty(\Omega), \: \lim_{k \to \infty} f_k = f:$

\begin{align*}
  \vbraces
  {
    \int f         \phi d \lambda^n -
    \int f_k       \phi d \lambda^n
  } \leq
  \norm[1]{(f - f_k) \phi} \leq
  \norm[p]{f - f_k}
  \norm[q]{\phi} \leq
  \norm[m, p]{f - f_k}
  \norm[q]{\phi}
  \xrightarrow[k \to \infty]{} 0.
\end{align*}

Hat $f$ eine schwache Ableitung $Df$, so können wir damit, und partieller Integration, weil $\supp \phi$ beschränkt ist, weiter schließen, dass

\begin{align*}
  \lim_{k \to \infty}   \Int{D_i f_k \phi}{\lambda^n} =
  \lim_{k \to \infty} - \Int{f_k D_i \phi}{\lambda^n} =
  - \Int{f D_i \phi}{\lambda^n} =
    \Int{D_i f \phi}{\lambda^n}.
\end{align*}

Nachdem, wegen der Produktregel, $D_i(u_k v_k) = D_i u_k v_k + u_k D_i v_k$ durchaus gilt, erhält man, nochmal mit partieller Integration

\begin{align*}
  - \Int{uv D_i \phi}{\lambda^n}
  & = \lim_{k \to \infty} - \Int{u_k v_k D_i \phi}{\lambda^n} \\
  & = \lim_{k \to \infty} \Int{D_i(u_k v_k)}{\lambda^n} \\
  & = \lim_{k \to \infty}
      \pbraces
      {
        \Int{D_i u_k v_k \phi}{\lambda^n} +
        \Int{u_k D_i v_k \phi}{\lambda^n}
      } \\
  & = \Int{D_i u v \phi}{\lambda^n} +
      \Int{u D_i v \phi}{\lambda^n} \\
  & = \Int{(D_i u v + u D_i v) \phi}{\lambda^n}.
\end{align*}


\end{solution}
\begin{solution}
    Um zu zeigen, dass $uv\in L^1(\Omega)$ ist benötigt man nur einmal die Ungleichung von Hölder (siehe Kusolitsch Satz 13.4) um 
\begin{align*}
    \norm[1]{uv}\leq\norm[2]{u}\norm[2]{v}<\infty
\end{align*}
einzusehen.

 Nun betrachten wir den Allgemeinen Fall, nämlich $u,v\in W^{1,2}(\Omega)$. Nach dem Satz von Meyers-Serrin (siehe Analysis 3 Skriptum Blümlinger Satz 6.2.8) wissen wir, dass $W^{1,2}(\Omega)\cap C^\infty(\Omega)$ dicht in $W^{1,2}(\Omega)$ liegt. Deshalb können wir eine Folge $(v_l)$ aus $W^{1,2}(\Omega)\cap C^\infty(\Omega)$ finden mit $lim_{l\to\infty}v_l=v$ in $W^{1,2}(\Omega)$. Natürlich ist $(v_l)$ in $W^{1,2}(\Omega)$ auch eine Cauchy-Folge und daher gilt für ein beliebiges $\epsilon>0$ und hinreichend große $m,l\in\mathbb{N}$ unter Benützung der Ungleichung von Hölder
\begin{align*}
    \norm[1]{uv_l-uv_m}&\leq\norm[2]{u}\norm[2]{v_l-v_m}\leq\norm[1,2]{u}\norm[1,2]{v_l-v_m}<\frac{\epsilon}{n+1},\\
    \norm[1]{D_i(uv_l-uv_m)}&\leq\norm[1]{D_iuv_l-D_iuv_m}+\norm[1]{uD_iv_l-uD_iv_m}\\
    &\leq\norm[2]{D_iu}\norm[1,2]{v_l-v_m}+\norm[1,2]{u}\norm[1,2]{v_l-v_m}<\frac{\epsilon}{n+1}
\end{align*}
Damit folgt sofort
\begin{align*}
    \norm[1,1]{uv_l-uv_m}^\sim=\norm[1]{uv_l-uv_m}+\sum_{i=1}^n\norm[1]{D_i(uv_l-uv_m)}<\epsilon,
\end{align*}
also dass $uv_l$ eine Cauchy-Folge in $W^{1,1}(\Omega)$ ist. Da nach Satz 6.2.3 gilt, dass $W^{1,1}(\Omega)$ ein Banachraum ist konvergiert also $uv_l\to u\tilde{v}\in W^{1,1}(\Omega)$. Da mit der Ungleichung von Hölder offensichtlich $uv_l\to uv$ in $L^1(\Omega)$ muss $u\tilde{v}=uv$ gelten (stimmt das?). Also wissen wir jetzt, dass $uv\in W^{1,1}(\Omega)$ gilt. Genauso kann man zeigen, dass $D_iuv_l+uD_iv_l$ im Banachraum $L^1(\Omega)$ gegen $D_iuv+uD_iv$ konvergiert (hab ich nicht nachgeprüft). Nun können wir nützen, dass für eine beliebige Testfunktion $\Phi \in C_c^\infty$ die Funktion $\zeta:W^{1,1}(\Omega)\to\mathbb{R}:u\mapsto\int u\Phi\mathrm{d}\lambda$ eine stetiges lineares Funktional auf $W^{1,1}(\Omega)$ ist (vgl. Analysis 3 Skriptum Blümlinger Seite 131). Mit der Ungleichung von Hölder erkennt man auch leicht, dass $\xi:L^1(\Omega)\to\mathbb{R}:u\mapsto\int u\Phi\mathrm{d}\lambda$ ein stetiges lineares Funktional auf $L^1(\Omega)$ ist. Damit sind die folgenden Grenzwertvertauschungen erlaubt und wir dürfen
\begin{align*}
    \int (D_iuv+uD_iv)\Phi\mathrm{d}\lambda&=\int \lim_{n\to\infty}(D_iuv_n+uD_iv_n)\Phi\mathrm{d}\lambda\\
    &=\lim_{n\to\infty}\int D_i(uv_n)\Phi\mathrm{d}\lambda\\
    &=\lim_{n\to\infty}\int uv_nD_i\Phi\mathrm{d}\lambda\\
    &=\int\lim_{n\to\infty}uv_nD_i\Phi\mathrm{d}\lambda\\
    &=\int uvD_i\Phi\mathrm{d}\lambda
\end{align*} 
schreiben und sind fertig.
\end{solution}
