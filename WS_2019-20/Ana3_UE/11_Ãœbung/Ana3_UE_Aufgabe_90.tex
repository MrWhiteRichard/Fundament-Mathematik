\begin{exercise}

Sind $\Omega\subseteq\mathbb{R}^n$ und $u,v \in W^{1,2}(\Omega)$, so ist $uv \in W^{1,1}(\Omega)$ mit $D_i(uv) = D_iuv + uD_iv$

\end{exercise}

\begin{solution}

Durch die Hölder-Ungleichung und $u, v \in L^2(\Omega)$, erhält man unmittelbar, dass $\norm[1]{uv} \leq \norm[2]{u} \norm[2]{v} < \infty$, also $uv \in L^1(\Omega)$. \\

Laut Blümlinger Satz 6.2.8 bzw. Meyers, Serrin, liegt der Unterraum $W^{m, p}(\Omega) \cap C^\infty(\Omega)$, für $1 \leq p < \infty$ und $\Omega$ offen in $\R^n$, dicht in $W^{m, p}(\Omega)$. $u, v$ lassen sich also durch glatte Funktionen, in der Sobolevnorm $\norm[m, p]{\cdot}$, approximieren.

\begin{align*}
  \norm[m, p, \Omega]{f} =
  \norm[m, p]{f} & =
  \pbraces
  {
    \sum_{|\alpha| \leq m}
    \norm[p]{D^\alpha f}^p
  }^\frac{1}{p} \:
  \text{für} \:
  1 \leq p < \infty, \\
  \norm[m, \infty]{f} & =
  \max_{|\alpha| \leq m}
  \norm[\infty]{D^\alpha f}
\end{align*}

\begin{align*}
  \Exists (u_k), (v_k) \in C^\infty(\Omega):
  \lim_{k \to \infty} u_k = u, \:
  \lim_{k \to \infty} v_k = v
\end{align*}

Eine, zur Sobolevnorm $\norm[m, p]{\cdot}$, äquivalente Norm $\norm[m, p]{\cdot}^\sim$, ist, laut Blümlinger Korollar 6.2.2, gegeben durch

\begin{align*}
  \norm[m, p]{f}^\sim :=
  \sum_{|\alpha| \leq m}
  \norm[p]{D^\alpha f}.
\end{align*}

Weil $(u_k), (v_k)$ bezüglich der Sobolevnorm $\norm[1, 2]{\cdot}$ konvergieren, folgt mit der Hölder-Ungleichung auch, dass

\begin{align*}
  \norm[1]{D_i u_k v_k - D_i uv}
  & \leq \norm[1]{D_i u_k v_k - D_i u_k v} +
        \norm[1]{D_i u_k v - D_i uv} \\
  & \leq \norm[1]{D_i u_k (v_k - v)} +
         \norm[1]{(D_i u_k - D_i u) v} \\
  & \leq \norm[2]{D_i u_k}
         \norm[2]{v_k - v} +
         \norm[2]{v}
         \norm[2]{D_i u_k - D_i u}
         \xrightarrow{k \to \infty} 0.
\end{align*}

Analog erhält man auch

\begin{align*}
  \norm[1]{u_k D_i v_k - u D_i v}
  \xrightarrow{k \to \infty} 0, \:
  \norm[1]{u_k v_k - uv}
  \xrightarrow{k \to \infty} 0.
\end{align*}

Laut Kusolitsch Satz 13.25 gilt: Ist $(\Omega, \mathfrak{S}, \mu)$ ein Maßraum und $1 \leq p < \infty$, so konvergiert eine Folge $(f_n)$ aus $\mathcal{L}_p$ genau dann im $p$-ten Mittel, wenn $(f_n)$ im Maß gegen ein $f \in \mathcal{L}_p$ konvergiert und gilt $\lim_n \norm[p]{f_n} = \norm[p]{f}$.

\begin{align*}
  \norm[p]{f - f_n}
  \xrightarrow{n \to \infty} 0
  \Leftrightarrow
  \Forall \epsilon > 0:
  \lim_{n \to \infty} \mu(|f - f_n| > \epsilon) = 0, \:
  \lim_{n \to \infty} \norm[p]{f_n} = \norm[p]{f}
\end{align*}

Daher sind die folgenden Grenzwertvertauschungen gerechtfertigt. Nachdem, wegen der Produktregel, $D_i(u_k v_k) = D_i u_k v_k + u_k D_i v_k$ durchaus gilt, erhält man also

\begin{align*}
  - \Int{uv D_i \phi}{\lambda^n}
  & = \lim_{k \to \infty} - \Int{u_k v_k D_i \phi}{\lambda^n} \\
  & = \lim_{k \to \infty} \Int{D_i(u_k v_k)}{\lambda^n} \\
  & = \lim_{k \to \infty}
      \pbraces
      {
        \Int{D_i u_k v_k \phi}{\lambda^n} +
        \Int{u_k D_i v_k \phi}{\lambda^n}
      } \\
  & = \Int{D_i u v \phi}{\lambda^n} +
      \Int{u D_i v \phi}{\lambda^n} \\
  & = \Int{(D_i u v + u D_i v) \phi}{\lambda^n}.
\end{align*}

\end{solution}

\begin{solution}
    Um zu zeigen, dass $uv\in L^1(\Omega)$ ist benötigt man nur einmal die Ungleichung von Hölder (siehe Kusolitsch Satz 13.4) um
\begin{align*}
    \norm[1]{uv}\leq\norm[2]{u}\norm[2]{v}<\infty
\end{align*}
einzusehen.

Für den zweiten Teil wollen wir zuerst den Fall betrachten, dass $u\in W^{1,2}(\Omega)$ und $v\in W^{1,2}(\Omega)\cap C^\infty(\Omega)$. Wählen wir eine beliebige Testfunktion $\Phi\in C_c^\infty(\Omega)$ so gilt auch $v\Phi\in C_c^\infty(\Omega)$ und daher auch $D_i(c\Phi)\in C_c^\infty(\Omega)$. Mit diesem Wissen und Proposition 6.2.6 aus dem Analysis 3 Skriptum vom Professor Blümlinger, das es erlaubt $D_i(v\Phi)=D_iv\Phi+vD_i\Phi$ zu schreiben, erhalten wir
\begin{align*}
    \int(D_iuv+uD_iv)\Phi\mathrm{d}\lambda&=\int(D_iuv\Phi+uD_iv\Phi)\mathrm{d}\lambda\\
    &=\int(D_iuv\Phi+uD_i(v\Phi)-uvD_i\Phi)\mathrm{d}\lambda\\
    &=\int D_iuv\Phi\mathrm{d}\lambda+\int uD_i(v\Phi)\mathrm{d}\lambda-\int uvD_i\Phi\mathrm{d}\lambda\\
    &=\int D_iuv\Phi\mathrm{d}\lambda-\int D_iuv\Phi\mathrm{d}\lambda-\int uvD_i\Phi \mathrm{d}\lambda\\
    &=-\int uvD_i\Phi \mathrm{d}\lambda
\end{align*}
Damit folgt bereits unmittelbar die gewünschte Aussage $D_i(uv)=D_iuv+uD_iv$.

Nun betrachten wir den Allgemeinen Fall, nämlich $u,v\in W^{1,2}(\Omega)$. Nach dem Satz von Meyers-Serrin (siehe Analysis 3 Skriptum Blümlinger Satz 6.2.8) wissen wir, dass $W^{1,2}(\Omega)\cap C^\infty(\Omega)$ dicht in $W^{1,2}(\Omega)$ liegt. Deshalb können wir eine Folge $(v_l)$ aus $W^{1,2}(\Omega)\cap C^\infty(\Omega)$ finden mit $lim_{l\to\infty}v_l=v$ in $W^{1,2}(\Omega)$. Natürlich ist $(v_l)$ in $W^{1,2}(\Omega)$ auch eine Cauchy-Folge und daher gilt für ein beliebiges $\epsilon>0$ und hinreichend große $m,l\in\mathbb{N}$ unter Benützung der Ungleichung von Hölder
\begin{align*}
    \norm[1]{uv_l-uv_m}&\leq\norm[2]{u}\norm[2]{v_l-v_m}\leq\norm[1,2]{u}\norm[1,2]{v_l-v_m}<\frac{\epsilon}{n+1},\\
    \norm[1]{D_i(uv_l-uv_m)}&\leq\norm[1]{D_iuv_l-D_iuv_m}+\norm[1]{uD_iv_l-uD_iv_m}\\
    &\leq\norm[2]{D_iu}\norm[1,2]{v_l-v_m}+\norm[1,2]{u}\norm[1,2]{v_l-v_m}<\frac{\epsilon}{n+1}
\end{align*}
Damit folgt sofort
\begin{align*}
    \norm[1,1]{uv_l-uv_m}^\sim=\norm[1]{uv_l-uv_m}+\sum_{i=1}^n\norm[1]{D_i(uv_l-uv_m)}<\epsilon,
\end{align*}
also dass $uv_l$ eine Cauchy-Folge in $W^{1,1}(\Omega)$ ist. Da nach Satz 6.2.3 gilt, dass $W^{1,1}(\Omega)$ ein Banachraum ist konvergiert also $uv_l\to u\tilde{v}\in W^{1,1}(\Omega)$. Da mit der Ungleichung von Hölder offensichtlich $uv_l\to uv$ in $L^1(\Omega)$ muss $u\tilde{v}=uv$ gelten. Also wissen wir jetzt, dass $uv\in W^{1,1}(\Omega)$ gilt. Genauso kann man zeigen, dass $D_iuv_l+uD_iv_l$ im Banachraum $L^1(\Omega)$ gegen $D_iuv+uD_iv$ konvergiert. Nun können wir nützen, dass für eine beliebige Testfunktion $\Phi \in C_c^\infty$ die Funktion $\zeta:W^{1,1}(\Omega)\to\mathbb{R}:u\mapsto\int u\Phi\mathrm{d}\lambda$ eine stetiges lineares Funktional auf $W^{1,1}(\Omega)$ ist (vgl. Analysis 3 Skriptum Blümlinger Seite 131). Indem man einfach $\vbraces{\Phi}\leq\norm[\infty]{\Phi}$ nützt erkennt man auch leicht, dass $\xi:L^1(\Omega)\to\mathbb{R}:u\mapsto\int u\Phi\mathrm{d}\lambda$ ein stetiges lineares Funktional auf $L^1(\Omega)$ ist. Damit sind die folgenden Grenzwertvertauschungen erlaubt und wir dürfen
\begin{align*}
    \int (D_iuv+uD_iv)\Phi\mathrm{d}\lambda&=\int \lim_{n\to\infty}(D_iuv_n+uD_iv_n)\Phi\mathrm{d}\lambda\\
    &=\lim_{n\to\infty}\int D_i(uv_n)\Phi\mathrm{d}\lambda\\
    &=\lim_{n\to\infty}-\int uv_nD_i\Phi\mathrm{d}\lambda\\
    &=-\int\lim_{n\to\infty}uv_nD_i\Phi\mathrm{d}\lambda\\
    &=-\int uvD_i\Phi\mathrm{d}\lambda
\end{align*}
schreiben und sind fertig.
\end{solution}
