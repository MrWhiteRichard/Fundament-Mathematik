\begin{exercise}

Zeigen Sie, dass das Gleichungssystem

\begin{align*}
  \frac{1}{2} x_1^3 - \frac{1}{8} x_2^3 = x_1 \\
  \frac{1}{2} x_1^5 + \frac{1}{4} x_1^2 x_2^4 + \frac{1}{4} = x_2
\end{align*}

eine Lösung besitzt.

\end{exercise}

\begin{solution}

$[-1, 1]^2 \in \R^2$!!! Es ist kompakt, es ist konvex, es ist ein ... (laut Brouwer) ... ein Fixpunktraum! $T$ besitzt darin einen Fixpunkt.

\begin{align*}
  T:
  \begin{cases}
    [-1, 1]^2 & \to [-1, 1]^2 \\
    x         & \mapsto
    \begin{pmatrix}
      \frac{1}{2} x_1^3 - \frac{1}{8} x_2^3 \\
      \frac{1}{2} x_1^5 + \frac{1}{4} x_1^2 x_2^4 + \frac{1}{4}
    \end{pmatrix}
  \end{cases}
\end{align*}

Lieblingsfrage: "Wieso existiert das $T$?" \\
Antwort: Dreiecksungleichung, d.h. $\Forall x \in [-1, 1]^2:$

\begin{align*}
  \vbraces{\abraces{e_1, T(x)}} & \leq
  \frac{1}{2} + \frac{1}{8} \leq
  1, \\
  \vbraces{\abraces{e_2, T(x)}} & \leq
  \frac{1}{2} + \frac{1}{4} + \frac{1}{4} \leq
  1.
\end{align*}

\end{solution}
