Zeigen Sie, dass $f \in L^p(\Omega), 1 < p < \infty$ genau dann in $W^{m,p}(\Omega)$ liegt, wenn die Abbildungen
$\phi \mapsto \int_{\Omega}fD^{\alpha}\phi d\lambda^n$ fpr $|\alpha| \leq m$ setig vom Raum der Testfunktionen versehen mit der $L^q$-Norm nach $\mathbb{R}$ ist.\\
Hinweis: Verwenden Sie, dass der Dualraum von $L^p$ der $L^q$ ist, das heißt jede beschränkte lineare Abbilund von einem dichten Teilraum
des $L^p$ nach $\mathbb{C}$ ist von der Form $\phi \mapsto \int \phi g$ mit $g \in L^q$.