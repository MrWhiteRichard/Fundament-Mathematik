Ein Punkt $x \in \mathbb{R}^n$ heißt \textit{Dichtepunkt} einer messbaren Teilmenge $E$ von $\mathbb{R}^n$, wenn $x$ Lebesguepunkt der Funktion $\mathbbm{1}_{E\cup\{x\}}$ ist.\\
Zeigen Sie: Ist jeder Punkt $x \in [0,1]^n$ ein Dichtepunkt einer messbaren Teilmenge $E$ von $\mathbb{R}^n$, so gilt $\lambda^n(E) \geq 1$.\\
Gibt es eine messbare Teilmenge $E$ von $\mathbb{R}$, für die $\mathbb{R}\backslash\{0\}$ die Menge der Dichtepunkte von $E$ ist?