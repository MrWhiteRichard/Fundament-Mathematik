\begin{exercise}

Verschwindet für eine Funktion $f: \mathbb{R} \mapsto \mathbb{R}$ die schwache Ableitung der Ordnung $n$, so ist f ein Polynom der Ordnung $n-1$ fast überall.\\

Hinweis: Zeigen Sie, dass jede Testfunktion als Summe der n-ten Ableitung einer Testfunktion und einer Linearkombination der Funktionen $\Psi_0^{(l)}, l < n, \Psi_0$ wie
im Beweis von 6.1.4. dargestellt werden kann und berechnen Sie $\int x^k\xi^{(l)}(x)dx$ für Testfunktionen $\xi$ und $k \leq l$.

\end{exercise}

\begin{solution}
    Zuerst wollen wir für eine beliebige Testfunktion $\Phi\in C_c^\infty$  und ein beliebiges $m\in\mathbb{N}$ und eine weitere Testfunktion $\Psi_0\in C_c^\infty$ mit $\int \Psi_0d\lambda=1$ zeigen, dass es ein $\zeta\in C_c^\infty$ und $\alpha_0,\dots,\alpha_{m-1}$ aus $\mathbb{R}$ gibt welche $\Phi=\zeta^{(m)}+\sum_{l=0}^{m-1}\alpha_l\Psi_0^{(l)}$ erfüllen. Wir gehen induktiv vor und erkennen unmittelbar, dass für $m=0$ die Aussage richtig ist. Für den Induktionsschritt gehen wir davon aus, dass wir eine Darstellung $\Phi=\theta^{(m)}+\sum_{l=0}^{m-1}\alpha_l\Psi_0^{(l)}$ mit einer Testfunktion $\theta\in C_c^\infty$ haben. Aus dem Beweis von Satz 6.1.4 im Analysis 3 Skriptum Blümlinger wissen wir dass
    \begin{align*}
        \eta:=\theta-\int\theta d\lambda\Psi_0\in C_c^\infty
    \end{align*} 
    eine Testfunktion mit
    \begin{align*}
        \int\eta d\lambda=0
    \end{align*} 
    ist. Nun ist klar, dass dann auch
    \begin{align*}
        C_c^\infty\ni\zeta:\mathbb{R}\to\mathbb{R}:x\mapsto\int_{-\infty}^x\eta d\lambda
    \end{align*}
    eine Testfunktion sein muss. Wegen $\zeta^{\prime}=\eta$ und der Definition von $\zeta$ erhalten wir die Darstellung 
    \begin{align*}
        \theta=\zeta^{\prime}+\int\theta d\lambda\Psi_0.
    \end{align*}
    Mit der Definition $\alpha_m:=\int\theta d\lambda$ erhalten wir so durch einsetzten unserer gewonnen Darstellung von $\theta$ die gewünschte Darstellung $\Phi=\zeta^{(m+1)}+\sum_{l=0}^{m}\alpha_l\Psi_0^{(l)}$.

    Nun können wir uns der eigentlichen Aufgabe widmen. Dafür definieren wir ein Polynom 
    \begin{align*}
        p:\mathbb{R}\to\mathbb{R}:x\mapsto\sum_{j=0}^{n-1}\beta_jx^j
    \end{align*}
    Da wir das Fundamentallemma der Variationsrechnung bereits kennen wäre es ausreichend Koeffizienten $\beta_0,\dots,\beta_{n-1}$ zu finden, für die für alle Testfunktionen $\Phi\in C_c^\infty$
    \begin{align}
        \int (f-p)\Phi d\lambda=0\label{eq:wunsch}
    \end{align}
    gilt. Unter der Annahme, dass für $i>j$ und eine beliebige Testfunktion $\xi$
    \begin{align*}
        \int x^j\xi^{(i)}(x)d\lambda(x) = 0
    \end{align*}
    gilt rechnen wir
    \begin{align*}
        \int (f-p)\Phi d\lambda&=\int\pbraces{f-p}\zeta^{(n)}d\lambda+\sum_{i=0}^{n-1}\alpha_i\int(f-p)\Psi_0^{(i)}d\lambda
    \end{align*}
    Der erste Summand wird nach Voraussetzung Null und wenn wir uns \eqref{eq:wunsch} wünschen, dann muss $p$
    \begin{align*}
        \sum_{i=0}^{n-1}\alpha_i\int f\Psi_0^{(i)}d\lambda=\sum_{i=0}^{n-1}\alpha_i\int p\Psi_0^{(i)}d\lambda
    \end{align*}
    erfüllen, also beispielsweise für alle $i\in\{0,\dots,n-1\}$ die Gleichheit
    \begin{align*}
        \int f\Psi_0^{(i)}d\lambda=\int p\Psi_0^{(i)}d\lambda=\sum_{j=0}^{n-1}\beta_j\int x^j\Psi_0^{(i)}(x)d\lambda(x)
    \end{align*} 
    Dies führt uns auf das Gleichungssystem
    \begin{align*}
        \begin{pmatrix}
            \int f\Psi_0^{(0)}d\lambda\\
            \vdots\\
            \int f\Psi_0^{(n-1)}d\lambda
        \end{pmatrix} 
        =
        \begin{pmatrix}
            \int x^0\Psi_0^{(0)}(x)d\lambda(x) & * & \cdots & \cdots & * \\
            0 & \int x^1\Psi_1^{(1)}(x)d\lambda(x) & * & \cdots & * \\
            \vdots & \ddots & \ddots & \ddots &\vdots \\
            0 & \cdots &  0 & \int x^{n-1}\Psi_0^{(n-1)}(x)d\lambda(x) & * \\
            0 & \cdots & \cdots & 0 & \int x^{n-1}\Psi_0^{(n-1)}(x)d\lambda(x)
        \end{pmatrix}
    \end{align*}
    Nun stellt sich natürlich die Frage ob diese ganzen Nuller tatsächlich entstehen (ich bin schon müde und kann nicht mehr gut denken) und überhaupt ob das Gleichungssystem lösbar ist.
\end{solution}
