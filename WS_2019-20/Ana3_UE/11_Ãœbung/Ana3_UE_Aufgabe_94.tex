\begin{exercise}

Ein Punkt $x \in \mathbb{R}^n$ heißt \textit{Dichtepunkt} einer messbaren Teilmenge $E$ von $\mathbb{R}^n$, wenn $x$ Lebesguepunkt der Funktion $\mathbbm{1}_{E\cup\{x\}}$ ist. \\

Zeigen Sie: Ist jeder Punkt $x \in [0,1]^n$ ein Dichtepunkt einer messbaren Teilmenge $E$ von $\mathbb{R}^n$, so gilt $\lambda^n(E) \geq 1$. \\

Gibt es eine messbare Teilmenge $E$ von $\mathbb{R}$, für die $\mathbb{R}\backslash\{0\}$ die Menge der Dichtepunkte von $E$ ist?

\end{exercise}

\begin{solution}

  Trivial!

% Laut Blümlinger Satz 6.3.6 bzw. dem Differenzierbarkeitssatz von Lebesgue gilt, dass $\lambda^n$-f.a. Punkte aus $E$ Dichtepunkte von $E$ sind und $\lambda^n$-f.a. Punkte aus $\R^n \setminus E$ erfüllen

% \begin{align*}
%     \lim_{r\searrow 0}\frac{\lambda^n(E\cap B(x,r))}{r^n\omega_n}=0.
% \end{align*}

% $\omega_n$ sei das Volumen der $n$-dimensionalen Einheitskugel. Angenommen,

% \begin{align*}
%     \lambda^n([0,1]^n \setminus E) > 0.
% \end{align*}

% Laut Blümlinger Lemma 6.3.5, sieht man (hoffentlich) ein, dass

% \begin{align*}
%     \lambda^n(E\cap[0,1]^n)>0.
% \end{align*}

% Setze für beliebiges $r > 0$

% \begin{align*}
%   d_r:
%   \R^n \to \R:
%   x \mapsto \frac{\lambda^n (E \cap B(x,r))}{r^n\omega_n}.
% \end{align*}

% Damit $\Exists y_1 \in [0,1]^n \setminus E:$

% \begin{align*}
%     \lim_{r\searrow 0} d_r(y_1) = 0
% \end{align*}

% Ebenso $\Exists z_1 \in [0,1]^n \cap E:$

% \begin{align*}
%     \lim_{r \searrow 0} d_r(z_1) = 1
% \end{align*}

% Daher $\Exists r_1 > 0:$

% \begin{align*}
%   B(y_1, r_1) \subseteq [0,1]^n, \:
%   B(z_1, r_1) \subseteq [0,1]^n. \:
%   d_{r_1}(y_1) < \frac{1}{4}, \:
%   d_{r_1}(z_1) > \frac{3}{4}.
% \end{align*}

% Jetzt definieren wir eine Funktion.

% \begin{align*}
%     f_{r_1}:
%     [0,1] \to \R:
%     t \mapsto d_{r_1}(y_1(1-t)+z_1t)
% \end{align*}

% Diese Funktion $f_{r_1}$ ist stetig, weshalb es nach dem Zwischenwertsatz und der Konvexität von $[0,1]^n$, gilt $\Exists x_1 \in [0,1]^n:$

% \begin{align*}
%   d_{r_1}(x_1) = \frac{1}{2}, \:
%   B(x_1, r_1) \subseteq [0,1]^n.
% \end{align*}

% Nehmen wir nun an wir haben für $l \in \N$

% \begin{align*}
%   x_l \in [0,1]^n,
%   r_l \in \mathbb{R}^+:
%   B(x_l,r_l) \subseteq [0,1]^n, \:
%   d_{r_l}(x_l)=\frac{1}{2}.
% \end{align*}

% Laut Definition, erhalten wir

% \begin{align*}
%     \lambda^n (E \cap B(x_l, r_l)) =
%     \frac{r_l^n \omega_n}{2} > 0,
% \end{align*}

% sowie

% \begin{align*}
%     \lambda^n(B(x_l, r_l) \setminus E)
%     = \lambda^n(B(x_l, r_l)) -
%       \lambda^n(E \cap B(x_l, r_l))
%     = r_l^n \omega_n -
%       \frac{r_l^n \omega_n}{2}
%     = \frac{r_l^n \omega_n}{2} > 0.
% \end{align*}

% Wie davor, wählen wir $y_{l+1} \in B(x_l, r_l) \setminus E$ und $z_{l+1} \in B(x_l,r_l) \in B(x_l,r_l) \cap B$ und ein hinreichend kleines $r_{l+1} \in \R^+:$

% \begin{align*}
%     d_{r_{l+1}}(y_1) < \frac{1}{4}, \:
%     d_{r_{l+1}}(z_1) > \frac{3}{4}, \:
%     r_{l+1}<\frac{r_l}{2}.
% \end{align*}

% Nun definieren wir wieder eine (hoffentlich) stetige Funktion.

% \begin{align*}
%   f_{r_{l+1}}:
%   [0,1] \to \R:
%   t \mapsto d_{r_{l+1}}(y_l(1-t)+z_lt)
% \end{align*}

% Wegen dem Zwischenwertsatz und der Konvexität von $B(x_l, r_l)$, $\Exists x_{l+1} \in B(x_l,r_l):$

% \begin{align*}
%   d_{r_{l+1}}(x_{l+1}) = \frac{1}{2}.
% \end{align*}

% Sei $r_{l+1}$ so klein, dass $B(x_{l+1},r_{l+1}) \subseteq B(x_l,r_l)$. Nun haben wir zwei Folgen $(r_k)$ und $(x_k)$ definiert mit $r_k \to 0$ und

% \begin{align*}
%   \{ \tilde{x} \}
%   = \bigcap_{l \in \mathbb{N}}
%     \overline{B(x_l, r_l)}
%   \Rightarrow
%   x_k \to \tilde{x} \in [0,1]^n.
% \end{align*}

% Da $(\tilde{x})$ ein Dichtepunkt von $E$ ist wissen wir

% \begin{align*}
%     1
%     & =  \lim_{k \to \infty}
%          \frac
%          {\lambda^n(B(\tilde{x}, 2 r_k) \cap (E \cup \tilde{x}))}
%          {2^n r_k^n \omega_n}
%     \leq \lim_{k\to\infty}
%          \frac
%          { \lambda^n (B(\tilde{x}, 2r_k)) + \lambda^n(B(x_k,2r_k) \cap E) }
%          { 2^n r_k^n \omega_n } \\
%     & =  \lim_{k\to\infty}
%          \frac
%          {
%            2^n r_k^n \omega_n +
%            \frac
%            { 2^n r_k^n \omega_n }
%            { 2 }
%          }{
%            2^nr_k^n\omega_n
%          }
%       =  \lim_{k\to\infty}
%          \frac{3}{2}
%       =  \frac{3}{2} < 1.
% \end{align*}

% Nun haben wir den Widerspruch und daher $\lambda^n([0,1]^n \setminus E) = 0$, also

% \begin{align*}
%   \lambda^n(E) \geq
%   \lambda^n(E \cap [0,1]^n) =
%   \lambda^n([0,1]^n) -
%   \lambda^n([0,1]^n \setminus E) = 1.
% \end{align*}

\end{solution}
