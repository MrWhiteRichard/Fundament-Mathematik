% -------------------------------------------------------------------------------- %

\begin{exercise}

Zeigen Sie:

\begin{enumerate}[label = (\roman*)]

    \item Für $f \in L^1(\T)$ mit Stammfunktion $F$ mit $\Int[\T]{F(t)}{t} = 0$ gilt $\hat F(n) = \frac{-i}{n} \hat f(n)$ $n \neq 0$, $F(0) = 0$.
    
    \item Sei $\norm[A(\T)]{f} = \sum_{n \in \Z} \abs{\hat f(n)}$ auf dem Raum $A(\T)$ aller Funktionen mit absolut konvergenter Fourierreihe.
    Dann gilt $\norm[A(\T)]{f g} \leq \norm[A(\T)]{f} \norm[A(\T)]{g}$

\end{enumerate}

\end{exercise}

% -------------------------------------------------------------------------------- %

\begin{solution}

\phantom{}

\begin{enumerate}[label = (\roman*)]

    \item \phantom{}
    
    \begin{align*}
        \hat F(n)
        & =
        \frac{1}{\sqrt{2 \pi}}
        \Int[-\pi][\pi]
        {
            F(x)
            e^{-i n x}
        }{x} \\
        & =
        \frac{1}{\sqrt{2 \pi}}
        \Bigg (
            F(x) \frac{-1}{i n} e^{-i n x} \Big |_{x = -\pi}^\pi
            -
            \frac{-1}{i n}
            \Int[-\pi][\pi]
            {
                f(x)
                e^{-i n x}
            }{x}    
        \Bigg ) \\
        & =
        \frac{1}{\sqrt{2 \pi}}
        F(\pi) \frac{-1}{i n} \underbrace{e^{-i n \pi}}_1
        -
        \frac{1}{\sqrt{2 \pi}}
        F(-\pi) \frac{-1}{i n} \underbrace{e^{i n \pi}}_1
        +
        \frac{-i}{n}
        \frac{1}{\sqrt{2 \pi}}
        \Int[-\pi][\pi]
        {
            f(x)
            e^{-i n x}
        }{x} \\
        & =
        \frac{1}{\sqrt{2 \pi}}
        \frac{-1}{i n}
        \underbrace
        {
            (
                F(\pi)
                -
                F(-\pi)
            )
        }_0
        +
        \frac{-i}{n}
        \frac{1}{\sqrt{2 \pi}}
        \Int[-\pi][\pi]
        {
            f(x)
            e^{-i n x}
        }{x} \\
        & =
        \frac{-i}{n}
        \hat f(n)
    \end{align*}

    \begin{align*}
        \hat F(0)
        =
        \frac{1}{\sqrt{2 \pi}}
        \Int[\T]
        {
            F(x)
            \underbrace{e^{-i 0 x}}_1
        }{x}
        =
        0
    \end{align*}

    \item \phantom{}
    
    \includegraphicsboxed{Ana3/Ana3 - Satz 3.1.5.png}

    \begin{multline*}
        \implies
        \norm[A(\T)]{f g}
        =
        \sum_{n \in \Z}
        \abs{\widehat{f g}(n)}
        \stackrel
        {
            \mathrm{3.1.5}
        }{=}
        \sum_{n \in \Z}
        \abs
        {
            \frac{1}{\sqrt{2 \pi}}
            \sum_{l \in \Z}
            \hat f(n - l)
            \hat g(l)
        }
        \leq
        \frac{1}{\sqrt{2 \pi}}
        \sum_{n \in \Z}
        \sum_{l \in \Z}
        \abs
        {
            \hat f(n - l)
            \hat g(l)
        } \\
        =
        \frac{1}{\sqrt{2 \pi}}
        \sum_{l \in \Z}
        \abs{\hat g(l)}
        \sum_{n \in \Z}
        \abs{\hat f(n - l)}
        =
        \frac{1}{\sqrt{2 \pi}}
        \norm[A(\T)]{g}
        \norm[A(\T)]{f}
        \leq
        \norm[A(\T)]{g}
        \norm[A(\T)]{f}
    \end{multline*}

\end{enumerate}

\end{solution}

% -------------------------------------------------------------------------------- %
