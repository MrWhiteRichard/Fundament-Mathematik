% -------------------------------------------------------------------------------- %

\begin{exercise}

Zeigen Sie, dass für eine $2 \pi / n$-periodische Funktion $f \in L^1(\T)$ gilt $\hat f(l) = 0$ für $l \not \in n \Z$.

\end{exercise}

% -------------------------------------------------------------------------------- %

\begin{solution}

Dass $f$ eine $2 \pi / n$-periodische Funktion ist heißt, $\Forall x \in \R:$

\begin{align*}
    f(x)
    =
    f(x \pm 2 \pi / n)
    =
    f(x \pm 2 (2 \pi / n))
    =
    \cdots
    =
    f(x \pm 2 \pi)
    =
    \cdots.
\end{align*}

\begin{align*}
    \implies
    \Int[-\pi + 2 \pi / n][\pi + 2 \pi / n]
    {
        f(u) e^{-i l u}
    }{u}
    & =
    \Int[-\pi + 2 \pi / n][\pi]
    {
        f(u) e^{-i l u}
    }{u}
    +
    \Int[\pi][\pi + 2 \pi / n]
    {
        f(u) e^{-i l u}
    }{u} \\
    & =
    \Int[-\pi + 2 \pi / n][\pi]
    {
        f(u) e^{-i l u}
    }{u}
    +
    \Int[-\pi][-\pi + 2 \pi / n]
    {
        f(s + 2 \pi) e^{-i l (s + 2 \pi)}
    }{s} \\
    & =
    \Int[-\pi + 2 \pi / n][\pi]
    {
        f(u) e^{-i l u}
    }{u}
    +
    \Int[-\pi][-\pi + 2 \pi / n]
    {
        f(s) e^{-i l s}
    }{s}
    \underbrace{(e^{2 \pi i})^{-l}}_1 \\
    & =
    \Int[-\pi][\pi]
    {
        f(t) e^{-i l t}
    }{t}
\end{align*}

Dabei haben wir folgende Substitution verwendet.

\begin{align*}
    s = u - 2 \pi
    \implies
    \begin{cases}
        \derivative[][s]{u} = 1 \implies \mathrm{d} s = \mathrm{d} u \\
        u = s + 2 \pi
    \end{cases}
\end{align*}

\begin{multline*}
    \implies
    \hat f(l)
    =
    \frac{1}{\sqrt{2 \pi}}
    \Int[-\pi][\pi]
    {
        f(x)
        e^{-i l x}
    }{x}
    =
    \frac{1}{\sqrt{2 \pi}}
    \Int[-\pi][\pi]
    {
        f(x + 2 \pi / n)
        e^{-i l x}
    }{x}
    =
    \frac{1}{\sqrt{2 \pi}}
    \Int[-\pi + 2 \pi / n][\pi + 2 \pi / n]
    {
        f(u)
        e^{-i l (u - 2 \pi / n)}
    }{u} \\
    =
    \frac{1}{\sqrt{2 \pi}}
    e^{i l 2 \pi / n}
    \Int[-\pi + 2 \pi / n][\pi + 2 \pi / n]
    {
        f(u)
        e^{-i l u}
    }{u}
    =
    e^{i l 2 \pi / n}
    \frac{1}{\sqrt{2 \pi}}
    \Int[-\pi][\pi]
    {
        f(t)
        e^{-i l t}
    }{t}
    =
    e^{i l 2 \pi / n}
    \hat f(l)
\end{multline*}

Dabei haben wir folgende Substitution verwendet.

\begin{align*}
    u = x + 2 \pi / n
    \implies
    \begin{cases}
        \derivative[][u]{x} = 1 \implies \mathrm{d} x = \mathrm{d} u \\
        x = u - 2 \pi / n
    \end{cases}
\end{align*}

Insgesamt, erhalten wir nun folgende Beziehung.

\begin{align*}
    \implies
    \pbraces{1 - e^{2 \pi i \frac{l}{n}}} \hat f(l) = 0
\end{align*}

Wir argumentieren mit Kontraposition weiter.
Angenommen, $\hat f(l) \neq 0$.

\begin{align*}
    \implies
    1 - e^{2 \pi i \frac{l}{n}} = 0
    \implies
    1 = e^{2 \pi i \frac{l}{n}}
    \implies
    \frac{l}{n} \in \Z
    \implies
    l \in n \Z
\end{align*}

Damit ist die Kontraposition gezeigt.

\end{solution}

% -------------------------------------------------------------------------------- %
