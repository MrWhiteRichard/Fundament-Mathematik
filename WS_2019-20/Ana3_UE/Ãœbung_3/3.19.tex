% -------------------------------------------------------------------------------- %

\begin{exercise}

In einem unendlichdimensionalen Hilbertraum ist die abgeschlossene Einheitskugel nicht kompakt.

Hinw.:
Betrachten Sie *berdeckungen mit $\rho$-Kugeln und $e_i \in B(\rho, x_i)$ für geeignetes $\rho$.

\end{exercise}

% -------------------------------------------------------------------------------- %

\begin{solution}

Sei $(e_i)_{i \in I}$ eine ONB des unendlichdimensionalen Hilbertraums $H$.
Dieses Vektoren sind normiert, also liegen sie im Rand der Einheitskugel $\partial B(1, 0)$.

\begin{align*}
    \Forall i \in I:
    \norm[H]{e_i} = 1
    \implies
    \Bbraces{e_i}_{i \in I} \subset \partial B(1, 0)
\end{align*}

Wir berechnen deren Abstand für $i, j \in I$.

\begin{multline*}
    \implies
    \norm[H]{e_i - e_j}^2
    =
    (e_i - e_j, e_i - e_j)_H
    =
    (e_i - e_j, e_i)_H - (e_i - e_j, e_j)_H \\
    =
    (e_i, e_i)_H - (e_j, e_i)_H - (e_i, e_j)_H + (e_j, e_j)_H
    =
    \delta_{ij} - \delta_{ji} - \delta_{ij} + \delta_{jj}
    =
    \begin{cases}
        0, & i = j \\
        2, & \text{sonst}
    \end{cases}
\end{multline*}

Seien $\rho := \frac{1}{\sqrt 2} = \frac{\sqrt 2}{2}$ und $(x_j)_{j \in J} \in H^J$ eine Überdeckung der Einheitskugel, d.h.

\begin{align*}
    \bigcup_{j \in J} B(\rho, x_j) \supseteq B(1, 0).
\end{align*}

Jeder ONB-Vektor liegt in einer dieser $\rho$-Kugeln, d.h.

\begin{align*}
    \Forall i \in I:
    \Exists j \in J:
    e_i \in B(\rho, x_j).
\end{align*}

Die Elemente der $\rho$-Kugel haben maximalen Abstand $\sqrt 2$.

\begin{align*}
    \Forall j \in J:
    \Forall a, b \in B(\rho, x_j):
    \norm[H]{a - b}
    \leq
    \norm[H]{a - x_j} + \norm[H]{x_j - b}
    <
    \rho + \rho
    =
    \sqrt 2
\end{align*}

Daher liegen keine zwei verschiedene ONB-Vektoren in einer $\rho$-Kugel.

\begin{align*}
    \Forall i_1, i_2 \in I:
    i_1 \neq i_2:
    \nExists j \in J:
    e_{i_1}, e_{i_2} \in B(\rho, x_j)
\end{align*}

In jeder $\rho$-Kugel liegt also genau ein ONB-Vektor.
Wir wenden das Schubfach-Prinzip an.
Weil $H$ unendlichdimensional ist, kann es keine endliche Teilüberdeckung geben, weil darin einer der ONB-Vektoren nicht enthalten wäre.

\begin{align*}
    |I| = \dim H = \infty
    \implies
    & \Forall J^\prime \in \mathcal{E}(J):
    \Exists i \in I:
    e_i \not \in \bigcup_{j \in J^\prime} B(\rho, x_j) \\
    \implies
    & e_i \in B(1, 0) \not \subseteq \bigcup_{j \in J^\prime} B(\rho, x_j)
\end{align*}

\end{solution}

% -------------------------------------------------------------------------------- %
