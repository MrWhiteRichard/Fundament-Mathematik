% --------------------------------------------------------------------------------

\begin{exercise}

Zeigen Sie, dass es keine nichttriviale Funktion in $L^1$ mit kompaktem Träger gibt, deren Fouriertransformiert ebenfalls kompakten Träger hat.

Hinw.:
Verwenden Sie das voriger Beispiel und $\sin(\pi (x - n)) = (-1)^n \sin (\pi x)$.

\end{exercise}

% --------------------------------------------------------------------------------

\begin{solution}

Sei $f \in L^1$ mit $\supp f$ und $\supp f^\wedge$ kompakt.
Weil diese laut Definition abgeschlossen sind, heißt das im $\R^n$, dass sie beschränkt sind.

Wir können also einerseits o.B.d.A. $\supp f^\wedge \in [-\pi, \pi]$ annehmen.
Wir können also tatsächlich das vorige Beispiel verwenden.
Wir wissen aber andererseits auch, dass $Z$ endlich ist.

\begin{align*}
    Z := \Z \cap \supp f \in \mathcal{E}(\Z)
\end{align*}

\begin{align*}
    \implies
    \Forall x \in \R:
    0
    \stackrel{!}{=}
    f(x)
    =
    \sum_{n \in \Z}
    f(n)
    \frac
    {
        \sin(\pi (x - n))
    }{
        \pi (x - n)
    }
    =
    \sum_{n \in Z}
    f(n)
    \frac
    {
        (-1)^n
        \sin(\pi x)
    }{
        \pi (x - n)
    }
\end{align*}

Wir müssen zeigen, dass auf der $\text{rhs}$ die leere $\sum$ steht, d.h. $Z \neq \emptyset$.
Angenommen, $\Exists k \in Z$.

\begin{align*}
    \implies
    f(k)
    & =
    \lim_{x \to k}
    f(x) \\
    & =
    \lim_{x \to k}
    \frac{\sin(\pi x)}{\pi}
    \pbraces
    {
        \sum_{n \in Z \setminus \Bbraces{k}}
        \frac
        {
            (-1)^n
            f(n)
        }{
            \pi (x - n)
        }    
        +
        \frac
        {
            (-1)^k
            f(k)
        }{
            \pi (x - k)
        }
    } \\
    & =
    \underbrace
    {
        \frac{\sin(\pi k)}{\pi}
    }_0
    \sum_{n \in Z \setminus \Bbraces{k}}
    \frac
    {
        (-1)^n
        f(n)
    }{
        \pi (x - n)
    }
    +
    \frac{1}{\pi}
    (-1)^k
    f(k)
    \underbrace
    {
        \lim_{x \to k}
        \frac{\sin(\pi x) - \sin(\pi k)}{x - k}
    }_1 \\
    & =
    \frac{1}{\pi}
    (-1)^k
    f(k)
\end{align*}

Dabei haben wir verwendet, dass

\begin{align*}
    \lim_{x \to k}
    \frac{\sin(\pi x) - \sin(\pi k)}{x - k}
    =
    \derivative{x}
    (
        \sin(\pi x)
    )
    \Big |_{x=k}
    =
    \pi \cos(\pi k)
    =
    1.
\end{align*}

\begin{align*}
    & \implies
    \underbrace
    {
        \pbraces
        {
            1
            -
            \frac{1}{\pi}
            (-1)^k
        }
    }_{\neq 0}
    f(k)
    =
    0 \\
    & \implies
    f(k) = 0 \\
    & \implies
    k \not \in \supp f \supseteq Z \\
    & \implies
    k \not \in Z
\end{align*}

Widerspruch!

\end{solution}

% --------------------------------------------------------------------------------
