% -------------------------------------------------------------------------------- %

\begin{exercise}

Für die Ooperatoren $L^1(\R^n) \to L^1(\R^n)$ und die Translationen $\tau_s$, Modulationen $\Mod_s$ resp. Dilatationen $\Dil_\lambda$

\begin{align*}
    \tau_s f(x) = f(x - s),
    \quad
    \Mod_s f(x) = \exp(i s x) f(x),
    \quad
    \Dil_\lambda f(x) = \frac{1}{\sqrt \lambda} f(x / \lambda)
\end{align*}

gilt

\begin{align*}
    \widehat{\tau_s f}(x) = (\Mod_{-s} \hat f)(x),
    \quad
    \widehat{\Mod_s f}(x) = (\tau_s \hat f)(x)
    \quad
    \text{und}
    \quad
    \widehat{\Dil_\lambda f}(x) = \Dil_{1 / \lambda} \hat f(x)
\end{align*}

\end{exercise}

% -------------------------------------------------------------------------------- %

\begin{solution}

\phantom{}

\begin{enumerate}[label = \arabic*.]

    \item Teil:
    
    \begin{align*}
        \mathrm{lhs}
        & =
        \frac{1}{(2 \pi)^\frac{n}{2}}
        \Int[\R^n]
        {
            e^{-i x \cdot y}
            \tau_{-s} f(y)
        }{y} \\
        & =
        \frac{1}{(2 \pi)^\frac{n}{2}}
        \Int[\R^n]
        {
            e^{-i x \cdot y}
            f(y - s)
        }{y} \\
        & =
        \frac{1}{(2 \pi)^\frac{n}{2}}
        \Int[\R]
        {
            \cdots
            \Int[\R]
            {
                \exp
                \pbraces
                {
                    -i
                    \sum_{k=1}^n
                    x_k y_k
                }
                f(y - s)
            }{y_1}
            \cdots
        }{y_n} \\
        & =
        \frac{1}{(2 \pi)^\frac{n}{2}}
        \Int[\R]
        {
            \cdots
            \Int[\R]
            {
                \exp
                \pbraces
                {
                    -i
                    \sum_{k=1}^n
                    x_k (u_k + s_k)
                }
                f(u)
            }{u_1}
            \cdots
        }{u_n} \\
        & =
        \frac{1}{(2 \pi)^\frac{n}{2}}
        \exp
        \pbraces
        {
            -i
            \sum_{k=1}^n
            x_k s_k
        }
        \Int[\R]
        {
            \cdots
            \Int[\R]
            {
                \exp
                \pbraces
                {
                    -i
                    \sum_{k=1}^n
                    x_k u_k
                }
                f(u)
            }{u_1}
            \cdots
        }{u_n} \\
        & =
        e^{-i x \cdot s}
        \frac{1}{(2 \pi)^\frac{n}{2}}
        \Int[\R^n]
        {
            e^{-i x \cdot u}
            f(u)
        }{u} \\
        & =
        e^{i (-s) \cdot x}
        f^\wedge(x) \\
        & =
        \mathrm{rhs}
    \end{align*}

    Dabei haben wir folgende Substitution verwenden.

    \begin{align*}
        u = y - s
        \implies
        \Forall k = 1, \dots, n:
        u_k = y_k - s_k
        \implies
        \begin{cases}
            \derivative[][u_k]{y_k} = 1 \implies \mathrm{d} u_k = \mathrm{d} y_k \\
            y_k = u_k + s_k
        \end{cases}
    \end{align*}

    \item Teil:
    
    \begin{align*}
        \mathrm{lhs}
        & =
        \frac{1}{(2 \pi)^\frac{n}{2}}
        \Int[\R^n]
        {
            e^{-i x \cdot y}
            \Mod_s f(y)
        }{y} \\
        & =
        \frac{1}{(2 \pi)^\frac{n}{2}}
        \Int[\R^n]
        {
            e^{-i x \cdot y}
            \exp(i s \cdot x)
            f(y)
        }{y} \\
        & =
        \frac{1}{(2 \pi)^\frac{n}{2}}
        \Int[\R^n]
        {
            e^{i y \cdot (x - s)}
            f(y)
        }{y} \\
        & =
        \hat f(x - s) \\
        & =
        \mathrm{rhs}
    \end{align*}

    \item Teil:
    
    \begin{align*}
        \mathrm{lhs}
        & =
        \frac{1}{(2 \pi)^\frac{n}{2}}
        \Int[\R^n]
        {
            e^{-i x \cdot y}
            \Dil_\lambda f(y)
        }{y} \\
        & =
        \frac{1}{(2 \pi)^\frac{n}{2}}
        \Int[\R^n]
        {
            e^{-i x \cdot y}
            \frac{1}{\sqrt \lambda}
            f(y / \lambda)
        }{y} \\
        & =
        \frac{1}{(2 \pi)^\frac{n}{2}}
        \frac{1}{\sqrt \lambda}
        \Int[\R]
        {
            \cdots
            \Int[\R]
            {
                \exp
                \pbraces
                {
                    -i
                    \sum_{k=1}^n
                    x_k y_k
                }
                f(y / \lambda)
            }{y_1}
            \cdots
        }{y_n} \\
        & =
        \frac{1}{(2 \pi)^\frac{n}{2}}
        \frac{\lambda}{\sqrt \lambda}
        \Int[\R]
        {
            \cdots
            \Int[\R]
            {
                \exp
                \pbraces
                {
                    -i
                    \sum_{k=1}^n
                    x_k \lambda u_k
                }
                f(u)
            }{u_1}
            \cdots
        }{u_n} \\
        & =
        \frac{1}{(2 \pi)^\frac{n}{2}}
        \sqrt \lambda
        \Int[\R^n]
        {
            e^{-i x \lambda \cdot u}
            f(u)
        }{u} \\
        & =
        \frac{1}{\sqrt \frac{1}{\lambda}}
        \frac{1}{(2 \pi)^\frac{n}{2}}
        \Int[\R^n]
        {
            e^{-i \pbraces{x / \frac{1}{\lambda}} \cdot u }
            f(u)
        }{u} \\
        & =
        \mathrm{rhs}
    \end{align*}

    Dabei haben wir folgende Substitution verwenden.

    \begin{align*}
        u = y / \lambda
        \implies
        \Forall k = 1, \dots, n:
        u_k = y_k / \lambda
        \implies
        \begin{cases}
            \derivative[][u_k]{y_k} = 1 / \lambda \implies \lambda \mathrm{d} u_k = \mathrm{d} y_k \\
            y_k = \lambda u_k
        \end{cases}
    \end{align*}

\end{enumerate}

\end{solution}

% -------------------------------------------------------------------------------- %
