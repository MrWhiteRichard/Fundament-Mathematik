% -------------------------------------------------------------------------------- %

\begin{exercise}

Den Sierpinskiteppich $T$ erhält man indem man das abg. Einheitsquadrat in $9$ gleichgroße abg. Teilquadrate der Kantenlänge $1/3$ zerlegt und das Innere des mittleren entfernt.
Im nächsten Schritt zerlegt man die verbleibenden $8$ Teilquadrate in $9$ gleichgroße abgeschlossene Teilquadrate der Kantenlänge $1/9$ und entfernt das Innere des mittleren.
Mit den verbliebenen $82$ fährt man so fort.
Geben Sie eine Abbildung $A$ des Einheitsquadrates $W$ auf sich mit $T = A T = \lim_{n \to \infty} A W$.

\includegraphicsunboxed[0.25]{6.52.1.png}
\includegraphicsunboxed{6.52.2.png}

Bestimmen Sie eine möglichst gute obere Schranke für die Hausdorffdimension des Sierpinskiteppichs.
Definieren Sie den Mengen-Würfel (Bild 2) und berechnen Sie für ihn eine möglichst gute obere Schranke.

\end{exercise}

% -------------------------------------------------------------------------------- %

\begin{solution}

\phantom{}

\begin{enumerate}[label = \arabic*.]

    \item Teil:
    
    \begin{gather*}
        I_{k, i}
        :=
        \bbraces
        {
            \frac{3 i + 1}{3^k},
            \frac{3 i + 2}{3^k}
        }, \\
        T_n
        :=
        W_2
        \setminus
        \bigcup_{k=1}^n
            \bigcup_{i, j = 0}^{3^{k-1} - 1}
                I_{k, i} \times I_{k, j}
        =
        T_{n-1}
        \setminus
        \bigcup_{i, j = 0}^{3^{n-1} - 1}
            I_{n, i} \times I_{n, j},
        \quad
        T
        :=
        T_\infty
        =
        \lim_{n \to \infty}
            T_n \\
        A_n
        :=
        \id_W \1_{T_n},
        \quad
        A
        :=
        \id_W \1_T
        =
        \lim_{n \to \infty}
            A_n
    \end{gather*}

    \item Teil:

    \begin{gather*}
        \dim_\mathcal{H} T
        =
        \inf
        \Bbraces
        {
            s \geq 0:
            \mathcal{H}^s(T) = 0
        }, \\
        \lim_{\delta \to 0}
            \mathcal{H}_\delta^s(T)
    \end{gather*}

    \begin{align*}
        \Forall \delta > 0:
            \Exists N \in \N:
                \Forall n \geq N:
                    \diam(I_{n, i} \times I_{n, j})
                    =
                    \frac{\sqrt 2}{3^n}
                    <
                    \delta
    \end{align*}

    Wir können also folgende Grenzwert-Bildung parallel laufen lassen.

    \begin{multline*}
        \mathcal{H}_\delta^s(T)
        =
        \omega_s
        \inf
        \Bbraces
        {
            \sum_{i \in \N}
                \pbraces
                {
                    \frac{\diam C_i}{2}
                }^s:
            \Forall i \in \N:
                \diam C_i < \delta,
            \bigcup_{i \in \N}
                C_i
            \supseteq
            T
        } \\
        \leq
        8^n
        \pbraces
        {
            \frac
            {
                \frac{\sqrt 2}{3^n}
            }{2}
        }^s
        =
        \pbraces
        {
            \frac{\sqrt 2}{2}
        }^s
        \pbraces
        {
            \frac{8}{3^s}
        }^n
        =
        \xrightarrow[n \to \infty]{\delta \to 0}
        0
    \end{multline*}

    \begin{align*}
        \iff
        \frac{8}{3^s} < 1
        \iff
        s > \frac{\log 8}{\log 3}
    \end{align*}

    \begin{align*}
        \implies
        \dim_\mathcal{H} T \leq \frac{\log 8}{\log 3}
    \end{align*}

    \item Teil:
    
    \begin{align*}
        A_1 \otimes \cdots \otimes A_n
        :=
        \bigcup_{\sigma \in S_n}
            A_{\sigma(1)} \times \cdots \times A_{\sigma(n)}
    \end{align*}

    \begin{gather*}
        S_n
        :=
        W_3
        \setminus
        \bigcup_{k=1}^n
            \bigcup_{i, j = 0}^{3^{n-1} - 1}
                I_{k, i} \otimes I_{k, j} \otimes [0, 1],
        \quad
        S
        :=
        S_\infty
        =
        \lim_{n \to \infty}
            S_n
    \end{gather*}

    \begin{gather*}
        \dim_\mathcal{H} S
        =
        \inf
        \Bbraces
        {
            s \geq 0:
            \mathcal{H}^s(S) = 0
        }, \\
        \lim_{\delta \to 0}
            \mathcal{H}_\delta^s(S)
    \end{gather*}

    \begin{align*}
        \Forall \delta > 0:
            \Exists N \in \N:
                \Forall n \geq N:
                    \diam(I_{n, i} \otimes I_{n, j} \otimes [0, 1])
                    =
                    \frac{\sqrt 3}{3^n}
                    <
                    \delta
    \end{align*}

    Wir können also folgende Grenzwert-Bildung parallel laufen lassen.

    \begin{multline*}
        \mathcal{H}_\delta^s(S)
        =
        \omega_s
        \inf
        \Bbraces
        {
            \sum_{i \in \N}
                \pbraces
                {
                    \frac{\diam C_i}{2}
                }^s:
            \Forall i \in \N:
                \diam C_i < \delta,
            \bigcup_{i \in \N}
                C_i
            \supseteq
            S
        } \\
        \leq
        20^n
        \pbraces
        {
            \frac
            {
                \frac{\sqrt 3}{3^n}
            }{2}
        }^s
        =
        \pbraces
        {
            \frac{\sqrt 3}{2}
        }^s
        \pbraces
        {
            \frac{20}{3^s}
        }^n
        =
        \xrightarrow[n \to \infty]{\delta \to 0}
        0
    \end{multline*}

    \begin{align*}
        \iff
        \frac{20}{3^s} < 1
        \iff
        s > \frac{\log{20}}{\log 3}
    \end{align*}

    \begin{align*}
        \implies
        \dim_\mathcal{H} S \leq \frac{\log{20}}{\log 3}
    \end{align*}

\end{enumerate}

\end{solution}

% -------------------------------------------------------------------------------- %
