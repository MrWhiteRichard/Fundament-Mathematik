% --------------------------------------------------------------------------------

\begin{exercise}

Seien $(M_1, \metric_1),(M_2, \metric_2)$ metrische Räume und $f: M_1 \to M_2$ eine Hölder-stetige Abbildung, d.h. es gibt $C, \gamma$ mit $\metric_2(f(x), f(y)) \leq C \metric_1(x, y)^\gamma \Forall x, y \in M_1$.
Zeigen Sie, dass die Hausdorffdimension von $M_2$ durch $\frac{1}{\gamma}$ mal der Hausdorffdimension von $M_1$ beschränkt ist.

\end{exercise}

% --------------------------------------------------------------------------------

\begin{solution}

\begin{gather*}
    \dim_\mathcal{H} M_i
    =
    \inf A_i,
    \quad
    \mathcal{H}^s(M_i)
    =
    \lim_{\delta \to 0}
        \mathcal{H}_\delta^s(M_i),
    \quad
    \mathcal{H}_\delta^s(M_i)
    =
    \omega_s
    \inf B_{\delta, i}, \\
    \quad
    A_i
    =
    \Bbraces
    {
        s \geq 0:
        \mathcal{H}^s(M_i) = 0
    },
    \quad
    B_{\delta, i}
    =
    \Bbraces
    {
        \sum_{n \in \N}
            \pbraces
            {
                \frac{\diam_i C_n^\delta}{2}}^s:
                \Forall n \in \N:
                    \diam_i C_n^\delta < \delta,
                \bigcup_{n \in \N} C_n^\delta \supseteq M_i
    }, \\
    \quad
    \diam_i C_n^\delta
    =
    \sup
    \Bbraces
    {
        \metric_i(x, y):
        x, y \in C_n^\delta
    }, \\
    \quad i = 1, 2,
    \quad
    \omega_s
    =
    \frac
    {
        \Gamma(1/2)^s
    }{
        \Gamma(1 + s/2)
    }
    ~\text
    {
        \blockquote{Volumen der $s$-dimensionalen Einheitskugel}
    }
\end{gather*}

Wir gehen zusätzlich an, dass $f$ surjektiv ist.
Wir machen uns das Leben noch leichter, indem wir $\omega_s$ weglassen.
($2$ hätten wir auch weglassen können.)
Das dürfen wir, weil ...

\begin{center}
    \blockquote{Die in dieser Definition auftretenden Faktoren $2$ und $\omega_s$ wirken willkürlich und werden in der Literatur nicht einheitlich bei der Definition des Hausdorffmaßes herangezogen.}
\end{center}

\begin{align*}
    \dim_\mathcal{H} M_2
    \stackrel{!}{\leq}
    \frac{1}{\gamma}
    \dim_\mathcal{H} M_1
    & \iff
    \Forall s \in A_1:
        \frac{s}{\gamma} \in A_2 \\
    & \iff
    \Forall s \geq 0:
        \pbraces
        {
            \mathcal{H}^s(M_1) = 0
            \implies
            \mathcal{H}^\frac{s}{\gamma}(M_2) = 0
        }
\end{align*}

Sei also $s \geq 0$.
Wir nehmen nun an, dass $\mathcal{H}^s(M_1) = 0$.
Wir wollen nun zeigen, dass $\mathcal{H}^\frac{s}{\gamma}(M_2) = 0$.

\begin{align*}
    \iff
    \Forall \varepsilon_2 > 0:
    \Exists \delta_2 > 0:
    \Forall \delta^\prime \in (0, \delta_2):
        \mathcal{H}_{\delta^\prime}^\frac{s}{\gamma}(M_2) < \varepsilon_2
\end{align*}

Sei also $\varepsilon_2 > 0$.
Wir können immer beliebig nahe an $\inf$-s herran und jedes $<$ (bzw. $>$) \blockquote{durchstoßen}; $\inf$ ist schließlich der kleinste Häufungspunkt.
Letzteres bedeutet, dass, für fixes $\delta^\prime \in (0, \delta_2)$, $\Exists (D_n^{\delta^\prime})_{n \in \N} \in M_2^\N:$

\begin{align*}
    \Forall n \in \N:
        \diam_2 D_n^{\delta^\prime} < \delta^\prime,
    \quad
    \bigcup_{n \in \N}
        D_n^{\delta^\prime} \supseteq M_2,
    \quad
    \sum_{n \in \N}
        \pbraces
        {
            \frac{\diam_2 D_n^{\delta^\prime}}{2}
        }^\frac{s}{\gamma}
        <
        \varepsilon_2.
\end{align*}

Wir haben ja angenommen, dass $\mathcal{H}^s(M_1) = 0$.

\begin{align*}    
    \iff
    \Forall \varepsilon_1 > 0:
    \Exists \delta_1 > 0:
    \Forall \delta \in (0, \delta_1):
    \mathcal{H}_\delta^s(M_1) < \varepsilon_1
\end{align*}

Letzteres bedeutet, dass, für fixes $\delta \in (0, \delta_1)$, $\Exists (C_n^\delta)_{n \in \N} \in M_1^\N:$

\begin{align*}
    \Forall n \in \N:
        \diam_1 C_n^\delta < \delta,
    \quad
    \bigcup_{n \in \N} C_n^\delta \supseteq M_1,
    \quad
    \sum_{n \in \N} \pbraces{\frac{\diam_i C_n^\delta}{2}}^s < \varepsilon_1.
\end{align*}

Daraus konstruieren wir unseren Kandidaten für die $M_2$-Überdeckung.

\begin{align*}
    D_n^{\delta^\prime} := f(C_n^\delta),
    \quad
    n \in \N,
    \quad
    \delta^\prime
    \stackrel{!}{\mapsto}
    \delta
\end{align*}

$(D_n^{\delta^\prime})_{n \in \N} \in f(M_1)^\N = M_2^\N$ muss nun folgende $3$ Eigenschaften erfüllen.

\begin{enumerate}[label = \arabic*.]

    \item Eigenschaft (Feinheit):
    
    Weil $f$ Hölder-stetig ist, gilt $\Forall n \in \N: \Forall \delta \in (0, \delta_1):$

    \begin{align*}
        \diam_2 D_n^{\delta^\prime}
        & =
        \diam_2 f(C_n^\delta) \\
        & =
        \sup \Bbraces{\metric_2(y_1, y_2): y_1, y_2 \in f(C_n^\delta)} \\
        & =
        \sup \Bbraces{\metric_2(f(x_1), f(x_2)): x_1, x_2 \in C_n^\delta} \\
        & \leq
        \sup \Bbraces{C \metric_1(x_1, x_2)^\gamma: x_1, x_2 \in C_n^\delta} \\
        & =
        C (\diam_1 C_n^\delta)^\gamma \\
        & \leq
        C \delta^\gamma \\
        & \stackrel{!}{\leq}
        \delta^\prime
        \in (0, \delta_2)
    \end{align*}

    \item Eigenschaft (Überdeckung):
    
    \begin{align*}
        \bigcup_{n \in \N}
            D_n^{\delta^\prime}
        =
        \bigcup_{n \in \N}
            f(C_n^\delta)
        \supseteq
        f
        \pbraces
        {
            \bigcup_{n \in \N}
                C_n^\delta
        }
        \supseteq
        f(M_1)
        =
        M_2
    \end{align*}

    \item Eigenschaft ($\to 0$):
    
    \begin{multline*}
        \sum_{n \in \N}
            \pbraces
            {
                \frac{\diam_2 D_n^{\delta^\prime}}{2}
            }^\frac{s}{\gamma}
        \leq
        \sum_{n \in \N}
            \pbraces
            {
                \frac{C (\diam_1 C_n^\delta)^\gamma}{2}
            }^\frac{s}{\gamma} \\
        =
        \sum_{n \in \N}
            \pbraces
            {
                \frac
                {
                    C^\frac{1}{\gamma}
                    \diam_1 C_n^\delta
                }{
                    2^{\frac{1}{\gamma} - 1}
                    2
                }
            }^s
        =
        \pbraces
        {
            \frac{C}
            {
                2^{1 - \gamma}
            }    
        }^\frac{s}{\gamma}
        \underbrace
        {
            \sum_{n \in \N}
            \pbraces
            {
                \frac{\diam_1 C_n^\delta}{2}
            }^s
        }_{< \varepsilon_1}
        \stackrel{!}{<}
        \varepsilon_2
    \end{multline*}

\end{enumerate}

Nun zum großen Finale!

Wir wählen zuerst $\varepsilon_1$, sodass die $3$-te Eigenschaft erfüllt ist.
Dadurch bekommen wir auch ein $\delta_1$.

\begin{align*}
    \text{d.h.}~
    \varepsilon_1
    <
    \pbraces
    {
        \frac{C}
        {
            2^{1 - \gamma}
        }
    }^\frac{\gamma}{s}
    \varepsilon_2
\end{align*}

Sei nun $\delta^\prime \in (0, \delta_1)$.
Wir wählen $\delta$ hinreichend klein, sodass die $1$-te (und $2$-te) Eigenschaft erfüllt ist.

\begin{align*}
    \text{d.h.}~
    C \delta^\gamma
    \leq
    \delta^\prime
\end{align*}

\end{solution}

% --------------------------------------------------------------------------------
