% --------------------------------------------------------------------------------

\begin{exercise}

Zeigen Sie den folgenden Überdeckungssatz:
Sei $K$ ein kompakter metrischer Raum und $\bigcup_{i \in I} (B(x_i, r_i)) = K$ eine Überdeckung.
Dann gibt es eine endliche Teilmenge $I_0 \subset I$ mit $\bigcup_{n \in I_0} B(x_j, 3 r_j) = K$ und die Kugeln $\Bbraces{B(x_i, r_i): i \in I_0}$ sind paarweise disjunkt.

\end{exercise}

% --------------------------------------------------------------------------------

\begin{solution}

Weil $K$ kompakt ist, gibt es eine endliche Teil-Überdeckung, d.h.

\begin{align*}
    \Exists J \in \mathcal{E}(I):
    \bigcup_{i \in J}
        B(x_i, r_i)
    \supseteq
    K
    ~\text{bzw. sogar}~
    =
    K.
\end{align*}

\begin{align*}
    \text{o.B.d.A.}
    \quad
    J = \Bbraces{1, \dots, n},
    \quad
    r_1 \geq \cdots \geq r_n
\end{align*}

Wir wählen die kleinsten Kugeln ($i_1$-te, $i_2$-te, \dots), sodass diese paarweise disjunkt sind.

\begin{gather*}
    i_1 := 1,
    \quad
    C_{i_1} := B(x_{i_1}, r_{i_1}), \\
    i_k
    :=
    \min
    \Bbraces
    {
        j \in J:
        C_{k-1} \cap B(x_j, r_j) = \emptyset
    },
    \quad
    C_k
    :=
    C_{k-1} \dot \cup B(x_{i_k}, r_{i_k})
    =
    \dot \bigcup_{\ell = 1}^k
        B(x_{i_\ell}, r_{i_\ell})
\end{gather*}

Wir wählen $I_0$ als die größte obige Auswahl, sodass die Vereinigung der, auf das $3$-fache aufgeblasenen, Kugeln nicht ganz $K$ ist.

\begin{align*}
    m
    :=
    \max
    \Bbraces
    {
        k = 1, \dots, n:
        \bigcup_{\ell = 1}^k
            B(x_{i_\ell}, 3 r_{i_\ell})
        \subsetneq
        K
    }
    +
    1,
    \quad
    I_0 := \Bbraces{i_1, \dots, i_m}
\end{align*}

\end{solution}

% --------------------------------------------------------------------------------
