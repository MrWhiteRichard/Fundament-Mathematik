% -------------------------------------------------------------------------------- %

\begin{exercise}

Zeigen Sie

\begin{align*}
    \frac{\sin x}{x}
    =
    \frac{a_0}{2}
    +
    \sum_{n=1}^\infty
    a_n
    \cos{n x}
\end{align*}

mit $a_n = \frac{1}{\pi} \Int[(n - 1) \pi][(n + 1) \pi]{\frac{\sin x}{x}}{x}$ und berechnen Sie damit

\begin{align*}
    \Int[0][\infty]{\frac{\sin x}{x}}{x}.
\end{align*}

\end{exercise}

% -------------------------------------------------------------------------------- %

\begin{solution}

\phantom{}

\begin{enumerate}[label = \arabic*.]

    \item Teil:

    Wir werden folgendes Additionstheorem verwenden.

    \begin{align*}
        \sin x \cos(n x)
        =
        \frac{1}{2}
        (
            \sin(x + n x)
            +
            \sin(x - n x)
        )
        =
        \frac{1}{2}
        (
            \sin(x (1 + n))
            +
            \sin(x (1 - n))
        )
    \end{align*}

    Die $a_n$, $n \in \N_0$ werden genau die $\cos$-Fourierkoeffizienten sein.
    Die $\sin$-Fourierkoeffizienten fallen weg, weil $x \mapsto \frac{\sin x}{x}$, also Quotient $2$-er ungerader Funktionen, gerade ist.

    \begin{align*}
        a_n
        & =
        \frac{1}{\pi}
        \Int[-\pi][\pi]
        {
            \frac{\sin x}{x}
            \cos(n x)
        }{x} \\
        & =
        \frac{1}{2 \pi}
        \pbraces
        {
            \Int[-\pi][\pi]
            {
                \frac{\sin((n + 1) x)}{x}
            }{x}
            -
            \Int[-\pi][\pi]
            {
                \frac{\sin((n - 1) x)}{x}
            }{x}
        } \\
        & =
        \frac{1}{2 \pi}
        \pbraces
        {
            (n + 1)
            \Int[-(n + 1) \pi][(n + 1) \pi]
            {
                \frac{\sin u}
                {
                    \frac{u}{n + 1}
                }
            }{u}
            -
            (n - 1)
            \Int[-(n - 1) \pi][(n - 1) \pi]
            {
                \frac{\sin u}
                {
                    \frac{u}{n - 1}
                }
            }{u}
        } \\
        & =
        \frac{1}{2 \pi}
        \pbraces
        {
            \Int[-(n + 1) \pi][-(n - 1) \pi]
            {
                \frac{\sin u}{u}
            }{u}
            +
            \Int[(n - 1) \pi][(n + 1) \pi]
            {
                \frac{\sin u}{u}
            }{u}
        } \\
        & =
        \frac{1}{2 \pi}
        \pbraces
        {
            -\Int[(n + 1) \pi][(n - 1) \pi]
            {
                \frac{\sin(-t)}{-t}
            }{t}
            +
            \Int[(n - 1) \pi][(n + 1) \pi]
            {
                \frac{\sin u}{u}
            }{u}
        } \\
        & =
        \frac{1}{\pi}
        \Int[(n - 1) \pi][(n + 1) \pi]
        {
            \frac{\sin x}{x}
        }{x}
    \end{align*}

    Dabei haben wir folgende Substitutionen verwendet.

    \begin{align*}
        u = (n \pm 1) x
        & \implies
        \begin{cases}
            \derivative[][u]{x} = n \pm 1 \implies \mathrm{d} x = \frac{1}{n \pm 1} \mathrm{d} u \\
            x = \frac{u}{n \pm 1}
        \end{cases} \\
        t = -u
        & \implies
        \begin{cases}
            \derivative[][t]{u} = -1 \implies \mathrm{d} t = -\mathrm{d} u \\
            u = -t
        \end{cases}
    \end{align*}

    \item Teil:
    
    Es bezeichne $f(x) := \frac{\sin x}{x}$, $x \in \R$ die linke Seite.
    Wir setzen diese stetig auf $0$ durch $f(0) := 1$ fort.
    Die im $1$-ten Teil gezeigte Darstellung von $f$ gilt nach wie vor.

    \begin{align*}
        1
        & =
        f(0) \\
        & =
        \frac{a_0}{2}
        +
        \sum_{n=1}^\infty
        a_n
        \underbrace{\cos(n 0)}_1 \\
        & =
        \frac{1}{2 \pi}
        \Int[-\pi][\pi]
        {
            \frac{\sin x}{x}
        }{x}
        +
        \sum_{n=1}^\infty
        \frac{1}{\pi}
        \Int[(n - 1) \pi][(n + 1) \pi]
        {
            \frac{\sin x}{x}
        }{x} \\
        & =
        \frac{1}{\pi}
        \pbraces
        {
            \frac{2}{2}
            \Int[0][\pi]
            {
                \frac{\sin x}{x}
            }{x}
            +
            \sum_{n \in 2 \N - 1}
            \Int[(n - 1) \pi][(n + 1) \pi]
            {
                \frac{\sin x}{x}
            }{x}
            +
            \sum_{n \in 2 \N}
            \Int[(n - 1) \pi][(n + 1) \pi]
            {
                \frac{\sin x}{x}
            }{x}
        } \\
        & =
        \frac{1}{\pi}
        \pbraces
        {
            \Int[0][\pi]
            {
                \frac{\sin x}{x}
            }{x}
            +
            \Int[0][\infty]
            {
                \frac{\sin x }{x}
            }{x}
            +
            \Int[\pi][\infty]
            {
                \frac{\sin x}{x}
            }{x}
        } \\
        & =
        \frac{2}{\pi}
        \Int[0][\infty]
        {
            \frac{\sin x}{x}
        }{x}
    \end{align*}

    \begin{align*}
        \implies
        \Int[0][\infty]
        {
            \frac{\sin x}{x}
        }{x}
        =
        \frac{\pi}{2}
    \end{align*}

\end{enumerate}

\end{solution}

% -------------------------------------------------------------------------------- %
