% --------------------------------------------------------------------------------

\begin{exercise}

Zeigen Sie:
Hat $f \in L^1(\R)$ eine Fouriertransformierte mit Träger in $[-\pi, \pi]$, deren $2 \pi$-periodische Fortsetzung auf $\R$ eine absolut konvergente Fouriereihe, so gilt:

\begin{align*}
    f(x)
    =
    \sum_{n \in \N}
    f(n)
    \frac
    {
        \sin(\pi (x - n))
    }{
        \pi (x - n)
    }
\end{align*}

\end{exercise}

% --------------------------------------------------------------------------------

\begin{solution}

Wir gehen davon aus, dass $\Z$ statt $\N$ gemeint ist.

\begin{enumerate}[label = \arabic*.]

    \item Fall ($x \in \Z$):
    
    \begin{align*}
        \Forall n \in \Z \setminus \Bbraces{x}:
        x - n \in \Z \setminus \Bbraces{0}
        \implies
        \sin(\pi (x - n)) = 0
    \end{align*}
    
    \begin{multline*}
        \mathrm{rhs}
        =
        \lim_{t \to x}
        f(t)
        \frac
        {
            \sin(\pi (t - x))
        }{
            \pi (t - x)
        }
        +
        \underbrace
        {
            \sum_{n \in \Z \setminus \Bbraces{x}}
            f(n)
            \frac
            {
                \sin(\pi (x - n))
            }{
                \pi (x - n)
            }
        }_0 \\
        =
        f(x)
        \lim_{y \to 0}
        \frac{\sin y - \sin 0}{y - 0}
        =
        f(x)
        \sin^\prime 0
        =
        f(x)
        \cos 0
        =
        \mathrm{lhs}
    \end{multline*}

    \item Fall ($x \not \in \Z$):
    
    Dass die Fouriertransformierte von $f$ Träger in $[-\pi, \pi]$ hat führt zu

    \begin{align*}
        \Forall x \in \R \setminus [-\pi, \pi]:
        f^\wedge(x) = 0.
    \end{align*}

    Sei $g$ deren $2 \pi$-periodische Fortsetzung, d.h.

    \begin{align*}
        f^\wedge(x \modulo 2 \pi)
        =:
        g(x)
        =
        \sum_{n \in \Z}
        \hat g(n)
        e^{i n x}
        ~\text{absolut konvergent},
        \quad
        x \in \R.
    \end{align*}

    \begin{align*}
        \mathrm{rhs}
        & =
        (f^\wedge)^\vee(x) \\
        & =
        \frac{1}{\sqrt{2 \pi}}
        \Int[\R]
        {
            e^{i x y}
            f^\wedge(y)
        }{y} \\
        & =
        \frac{1}{\sqrt{2 \pi}}
        \Int[-\pi][\pi]
        {
            e^{i x y}
            f^\wedge(y)
        }{y} \\
        & =
        \frac{1}{\sqrt{2 \pi}}
        \Int[-\pi][\pi]
        {
            e^{i x y}
            \frac{1}{\sqrt{2 \pi}}
            \sum_{n \in \Z}
            \hat g(n)
            e^{i n y}
        }{y} \\
        & \stackrel{!}{=}
        \frac{1}{2 \pi}
        \sum_{n \in \Z}
        \hat g(n)
        \Int[-\pi][\pi]
        {
            e^{i (x + n) y}
        }{y} \\
        & \stackrel{!}{=}
        \frac{1}{2 \pi}
        \sum_{n \in \Z}
        f(-n)
        2 \frac{\sin(\pi (x + n))}{x + n} \\
        & =
        \mathrm{lhs}
    \end{align*}

    Die Vertauschung beim $1$-ten \blockquote{!} dürfen wir durchführen, weil wir es mit einem Parameterintegal in $x$ auf dem Kompaktum $[-\pi, \pi]$ zu tun haben.
    Für das $2$-te brauchen wir $2$ Nebenrechnungen.

    \begin{enumerate}[label = \arabic*.]

        \item Nebenrechnung:
        
        \begin{align*}
            \hat g(n)
            =
            \frac{1}{\sqrt{2 \pi}}
            \Int[-\pi][\pi]
            {
                g(y)
                e^{-i n y}
            }{y}
            =
            \frac{1}{\sqrt{2 \pi}}
            \Int[-\pi][\pi]
            {
                e^{i (-n) y}
                f^\wedge(y)
            }{y}
            =
            (f^\wedge)^\vee(-n)
            =
            f(-n)
        \end{align*}

        \item Nebenrechnung:

        \begin{multline*}
            \Int[-\pi][\pi]
            {
                e^{i (x + n) y}
            }{y}
            =
            \Int[-\pi][\pi]
            {
                \cos((x + n) y)
            }{y}
            +
            i
            \underbrace
            {
                \Int[-\pi][\pi]
                {
                    \sin((x + n) y)
                }{y}
            }_0 \\
            =
            \frac{1}{x + n}
            \sin((x + n) y) \Big |_{y = -\pi}^\pi
            =
            2 \frac{\sin(\pi (x + n))}{x + n}
        \end{multline*}
    
    \end{enumerate}

\end{enumerate}

\end{solution}

% --------------------------------------------------------------------------------
