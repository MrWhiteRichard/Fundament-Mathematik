% --------------------------------------------------------------------------------

\begin{exercise}

Sei $(a_n)_{n \in \Z}$ eine Folge mit $a_n = a_{-n}$ und $a_n - 2 a_{n+1} + a_{n+2} \geq 0$ für $n \geq 1$, $a_n \to 0$ für $n \to \infty$.
Zeigen Sie $(a_n - a_{n+1})$ ist auf $\N$ monoton fallend mit $\lim_{n \to \infty} n (a_n - a_{n+1}) = 0$ und

\begin{align} \label{eq:1}
    \sum_{n=1}^N
    n (a_{n-1} - 2 a_n + a_{n+1})
    =
    a_0 - a_N - N (a_n - a_{N+1})
    \to_{N \to \infty}
    a_0.
\end{align}

Zeigen Sie, dass für den Féjerkern $(F_n)$ die Konvergenz von

\begin{align} \label{eq:2}
    \sum_{n=1}^N
    n (a_n - 2 a_{n+1} + a_{n+2}) F_n(t)
\end{align}

in der $L^1$-Norm gegen eine Funktion $f$ folgt.
Zeigen Sie $\hat f(l) = a_{|l|}$ und begründen Sie damit, dass das Riemann-Lebesgue-Lemma in dem Sinn nicht verschärft werden kann, dass die Folge der Fourierkoeffizienten \blockquote{beliebig langsam} gegen $0$ konvergieren kann (d.h. für jedes $\alpha > 0$ gibt es $f \in L^1$ mit $\limsup \hat f(n) h^\alpha > 0$).

\end{exercise}

% --------------------------------------------------------------------------------

\begin{solution}

\phantom{}

\begin{enumerate}[label = \arabic*.]

    \item Teil ($(a_n - a_{n+1})$ ist auf $\N$ monoton fallend):

    \begin{align*}
        a_{n+1} - a_{n+1} \leq a_n - a_{n+1}
        \iff
        0 \leq a_n - 2 a_{n+1} + a_{n+1}
    \end{align*}

    \item Teil ($\lim_{n \to \infty} n (a_n - a_{n+1}) = 0$):

    \begin{align*}
        \Forall n \in \N:
        a_n - \sum_{n=k}^{N-1} a_k - a_{k+1}
        =
        a_N
        \xrightarrow{N \to \infty}
        0
        \implies
        \sum_{k=n}^\infty a_k - a_{k+1} = a_n
    \end{align*}

    Angenommen, $\Exists \varepsilon > 0: \Forall N \in \N: \Exists n \geq N:$

    \begin{align*}
        n (a_n - a_{n+1}) > \varepsilon
        \implies
        a_1
        =
        \sum_{n=1}^\infty
        a_n - a_{n+1}
        >
        \sum_{n=1}^\infty
        \frac{\varepsilon}{n}
        =
        \infty
    \end{align*}

    Widerspruch!

    \item Teil (Gleichung \eqref{eq:1}):

    \begin{align*}
        &
        \sum_{n=1}^N n (a_{n-1} - 2 a_n + a_{n+1}) \\
        & =
        \sum_{n=1}^N a_{n-1} + (n-1) a_{n-1} - n a_n - a_n - (n-1) a_n + n a_n \\
        & =
        \sum_{n=1}^N a_{n-1}
        +
        \sum_{n=1}^N (n-1) a_{n-1}
        -
        \sum_{n=1}^N n a_n
        -
        \sum_{n=1}^N a_n
        -
        \sum_{n=1}^N (n-1) a_n
        +
        \sum_{n=1}^N n a_n \\
        & =
        \sum_{n=0}^{N-1} a_n
        +
        \sum_{n=0}^{N-1} n a_n
        -
        \sum_{n=1}^N n a_n
        -
        \sum_{n=1}^N a_n
        -
        \sum_{n=0}^{N-1} n a_{n+1}
        +
        \sum_{n=1}^N n a_{n+1} \\
        & =
        \pbraces
        {
            \sum_{n=0}^{N-1} a_n
            -
            \sum_{n=1}^N a_n
        }
        +
        \pbraces
        {
            \sum_{n=0}^{N-1} n a_n
            -
            \sum_{n=1}^N n a_n
        }
        +
        \pbraces
        {
            \sum_{n=1}^N n a_{n+1}
            -
            \sum_{n=0}^{N-1} n a_{n+1}
        } \\
        & =
        a_0 - a_N - N a_N + N a_{N+1} \\
        & =
        a_0
        -
        \underbrace{a_N}_{\xrightarrow{N \to \infty} 0}
        -
        \underbrace{N (a_N - a_{N+1})}_{\xrightarrow{N \to \infty} 0}
        \xrightarrow{N \to \infty}
        a_0
    \end{align*}

    \item Teil (Gleichung \eqref{eq:2}):
    
    Offenbar sind die Partialsummen von \eqref{eq:2} $\in L^1$.
    Dieser ist vollständig, also können wir das Cauchy'sche Konvergenz-Kriterium für Reihen verwenden.
    Der folgende Grenzübergang glit wegen dem $3$-ten Teil.

    \includegraphicsboxed{Ana3/Ana3 - Lemma 3.1.13.png}

    \begin{multline*}
        \implies
        \norm[1]
        {
            \sum_{k=m}^n
            k (a_k - 2 a_{k+1} + a_{k+2})
            F_k
        }
        =
        \sum_{k=m}^n
        \underbrace{k (a_k - 2 a_{k+1} + a_{k+2})}_{\geq 0}
        \underbrace
        {
            \norm[1]{F_k}
        }_{2 \pi} \\
        =
        2 \pi
        \sum_{k=m}^n
        k (a_k - 2 a_{k+1} + a_{k+2})
        \xrightarrow{n, m \to \infty}
        0
    \end{multline*}

    \item Teil ($\hat f(l) = a_{|l|}$):
    
    \includegraphicsboxed{Ana3/Ana3 - (3.1).png}

    \begin{align*}
        \implies
        f(x)
        & =
        \sum_{n=1}^\infty
        n (a_n - 2 a_{n+1} + a_{n+2})
        F_n(x) \\
        & =
        \sum_{n=1}^\infty
        n (a_n - 2 a_{n+1} + a_{n+2})
        \sum_{k = -n-1}^{n-1}
        \frac{n - |k|}{n}
        e^{i k x} \\
        & =
        \sum_{n=1}^\infty
        \sum_{k = -n-1}^{n-1}
        (a_n - 2 a_{n+1} + a_{n+2})
        (n - |k|)
        e^{i k x} \\
        & \stackrel{!}{=}
        \sum_{k \in \Z}
        \sum_{n = |k|}^\infty
        (a_n - 2 a_{n+1} + a_{n+2})
        (n - |k|)
        e^{i k x} \\
        & \stackrel{!}{=}
        \sum_{k \in \Z}
        a_{|k|}
        e^{i k x}
    \end{align*}

    \begin{enumerate}[label = \arabic*.]

        \item \blockquote{!}:

        Die Vertauschung dürfen wir durchführen, weil ...

        \begin{enumerate}[label = \arabic*.]
    
            \item $f(x)$ für alle $x$ absolut, also insbesondere unbedingt, konvergiert.
            Dazu muss man bloß feststellen, dass
    
            \begin{align*}
                \sup_{x \in \R} |e^{i k x}|
                =
                1
                =
                e^{i k \pi},
            \end{align*}
    
            und alle anderen Beiträge der Summanden nicht-negativ sind.
            Daher ist die vermeintliche punktweise absolute Konvergenz äquivalent zur (bekannten) Wohldefiniertheit von $f(\pi)$.
    
            \item \phantom{}
    
            \rotatebox{90}
            {
                $
                \begin{array}{ccccccccccccccc}
                    \sum_{n=0}^\infty \sum_{k = -n-1}^{n-1} (n, k) &        &         &   &                           &   &                           &   &                           &   &                          &   &                          &   &        \\
                    =                                              &        &         &   &                           &   &                           &   &                           &   &                          &   &                          &   &        \\
                    \sum_{k= - 1}^{-1} (0, k)                      & =      &         &   &                           &   &                           &   & (0, -1)                   &   &                          &   &                          &   &        \\
                    +                                              &        &         &   &                           &   &                           &   & +                         &   &                          &   &                          &   &        \\
                    \sum_{k = -2}^0 (1, k)                         & =      &         &   &                           &   & (1, -2)                   & + & (1, -1)                   & + & (1, 0)                   &   &                          &   &        \\
                    +                                              &        &         &   &                           &   & +                         &   & +                         &   & +                        &   &                          &   &        \\
                    \sum_{k = -3}^1 (2, k)                         & =      &         &   & (2, -3)                   & + & (2, -2)                   & + & (2, -1)                   & + & (2, 0)                   & + & (2, 1)                   &   &        \\
                    +                                              &        &         &   & +                         &   & +                         &   & +                         &   & +                        &   & +                        &   &        \\
                    \vdots                                         & \vdots & \iddots &   & \vdots                    &   & \vdots                    &   & \vdots                    &   & \vdots                   &   & \vdots                   &   & \ddots \\
                    \stackrel{!}{=}                                &        &         &   & =                         &   & =                         &   & =                         &   & =                        &   & =                        &   &        \\
                    \sum_{k \in \Z} \sum_{n = |k|}^\infty (n, k)   & =      & \cdots  & + & \sum_{n=2}^\infty (n, -3) & + & \sum_{n=1}^\infty (n, -2) & + & \sum_{n=0}^\infty (n, -1) & + & \sum_{n=1}^\infty (n, 0) & + & \sum_{n=2}^\infty (n, 1) & + & \cdots
                \end{array}
                $
            }
        
        \end{enumerate}

        \item \blockquote{!}:
        
        Die $(a_n)_{n \in \N_0}$ um $|k|$ verschobene Folge $(b_n)_{n \in \N_0}$ hat hinreichende Eigenschaften, um den $3$-ten Teil darauf anzuwenden.

        \begin{align*}
            b_n := a_{n + |k|},
            \quad
            n \in \N_0
        \end{align*}

        \begin{align*}
            \implies
            \sum_{n = |k|}^\infty
            (a_n - 2 a_{n+1} + a_{n+2})
            (n - |k|)
            =
            \sum_{n=0}^\infty
            n (b_n - 2 b_{n+1} + b_{n+2})
            =
            b_0
            =
            a_{|k|}
        \end{align*}

    \end{enumerate}

\end{enumerate}

\end{solution}

% --------------------------------------------------------------------------------
