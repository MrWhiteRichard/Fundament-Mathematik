% --------------------------------------------------------------------------------

\begin{exercise}

Für $f \in L^1(0, \infty)$ wird durch $\mathcal{L} f(x) := \Int[0][\infty]{f(t) e^{-s t}}{t}$ eine Funktion $\mathcal{L} f$ auf $[0, \infty)$ definiert, die auf $[0, \infty]$ stetig und in $(0, \infty)$ unendlich oft differenzierbar ist.

\end{exercise}

% --------------------------------------------------------------------------------

\begin{solution}

Wir wollen wieder Satz 2.1.5 bzw. 2.1.7 (siehe oben) anwenden; diesmal auf die folgende Funktionen-Familie.

\begin{align*}
    g_s(t)
    :=
    f(t) e^{-s t},
    \quad
    s, t \in [0, \infty)
\end{align*}

Diese Funktionen sind also Zusammensetzung von stetigen Funktionen ebenfalls stetig, also messbar.

\begin{enumerate}[label = \arabic*.]

    \item Teil (auf $[0, \infty]$ stetig):
    
    \begin{enumerate}[label = \arabic*.]

        \item Fall:
        
        Sei $s_0 \in (0, \infty)$ und $\delta > 0$, sodass $B(s_0, \delta) \subset (0, \infty)$.
        Sei $s \in B(x_0, \delta)$.
    
        \begin{align*}
            \implies
            |g_s(t)|
            =
            \abs
            {
                f(t) e^{-s t}
            }
            \leq
            \norm[{L^\infty[0, \infty)}]{f}
            e^{-(s_0 - \delta) t}
            =:
            h(t)
        \end{align*}

        Die Majorante $h$ ist auch integrierbar.

        \begin{multline*}
            \Int[0][\infty]{h(t)}{t}
            =
            \norm[{L^\infty[0, \infty)}]{f}
            \Bigg (
                \underbrace
                {
                    \lim_{t \to \infty}
                    \frac{1}{-(s_0 - \delta)}
                    e^{-(s_0 - \delta) t}    
                }_0
                -
                \frac{1}{-(s_0 - \delta)}
                \underbrace
                {
                    e^{-(s_0 - \delta) 0}
                }_1
            \Bigg ) \\
            =
            \frac
            {
                \norm[{L^\infty[0, \infty)}]{f}
            }{
                s_0 - \delta
            }
            <
            \infty
        \end{multline*}

        \item Fall:
        
        Sei $s_0 = 0$, so führen wir den 1. Fall mit $B(s_0, \delta) \cap [0, \infty]$ aus.

    \end{enumerate}    

    \item Teil (auf $(0, \infty)$ unendlich oft differenzierbar):
    
    \begin{align*}
        \abs
        {
            \pderivative[n][g_s(t)]{s}
        }
        =
        \abs
        {
            (-t)^n f(t) e^{-s t}
        }
        \leq
        \norm[{L^\infty[0, \infty)}]{f}
        \pbraces
        {
            t^n e^{-(s_0 - \delta) t}
        }
        =:
        h_n(t)
    \end{align*}

    Die Majorante $h_n$ ist auch integrierbar (o.B.d.A. $f \neq 0$).

    \begin{multline*}
        \frac{1}
        {
            \norm[{L^\infty[0, \infty)}]{f}
        }
        \Int[0][\infty]{h_n(t)}{t}
        =
        \Int[0][\infty]{t^n e^{-(s_0 - \delta) t}}{t} \\
        \stackrel{!}{=}
        \frac{1}{(s_0 - \delta)^{n+1}}
        \Int[0][\infty]{x^n e^{-x}}{x}
        =
        \frac
        {
            \Gamma(n+1)
        }{
            (s_0 - \delta)^{n+1}
        }
        =
        \frac{n!}{(s_0 - \delta)^{n+1}}
        <
        \infty
    \end{multline*}

    Dabei haben wir folgende Substitution verwendet.

    \begin{align*}
        x = (s_0 - \delta) t
        \implies
        & t = \frac{x}{s_0 - \delta} ,\\
        & \derivative[][x]{t} = s_0 - \delta \implies \mathrm{d}t = \frac{1}{s_0 - \delta} \mathrm{d}x
    \end{align*}

\end{enumerate}

\end{solution}

% --------------------------------------------------------------------------------
