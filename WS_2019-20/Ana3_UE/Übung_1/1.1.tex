% -------------------------------------------------------------------------------- %

\begin{exercise}

Für eine stetige Funktion $K: [0, 1] \times [0, 1] \to \R$ sei $T: C[0, 1] \to C[0, 1]$ durch $Tf(x) = \Int[0][1]{K(x, y) f(y)}{y}$ definiert.
Zeigen Sie, dass $T$ tatsächlich $C[0, 1]$ in sich abbildet und das Bild der Einheitskugel von $(C[0, 1], \norm[\infty]{\cdot})$ unter $T$ relativ kompakt ist.

\end{exercise}

% -------------------------------------------------------------------------------- %

\begin{solution}

\phantom{}

\begin{enumerate}

    \item Laut Korollar 8.7.9 ist, für alle $f \in C[0, 1]$, $Tf$ als Parameter-Integral stetig.
    
    \includegraphicsboxed{Ana1&2/Ana1&2 - 8.7.9 Korollar.png}

    \begin{align*}
        \implies
        \Forall f \in C[0, 1]:
        Tf
        =
        \underbrace
        {
            \Int[0][1]{K(\cdot, y) f(y)}{y}
        }_\text{Parameter-Integral}
        \in
        C[0, 1]
    \end{align*}

    \item Das Bild der Einheitskugel ist $T(B(0, 1))$.
    Für die Präkompaktheit zeigen wir also, dass $\overline{T(B(0, 1))}$ komkpakt ist.

    Wir weichen vermöge Satz 1.4.10 auf Abgeschlossenheit (welche bereits laut Definition erfüllt ist) und Totalbeschränktheit aus.

    \includegraphicsboxed{Ana3/Ana3 - Satz 1.4.10.png}

    \begin{enumerate}

        \item Dazu müssen wir zunächst einsehen, dass $(C[0, 1], \metric_\infty)$ ein vollständiger Metrischer Raum ist.

        \includegraphicsboxed{Ana1&2/Ana1&2 - 6.6.11 Satz.png}

        $(\R, \metric)$ ist bekannterweise ein vollständiger Metrischer Raum und $[0, 1] \neq \emptyset$.
        Laut Satz 6.6.11, ist $(\mathcal{B}([0, 1], \R), \metric_\infty)$ daher auch ein vollständiger Metrischer Raum.
        
        \includegraphicsboxed{Ana1&2/Ana1&2 - 9.1.6 Lemma.png}

        Weil $C[0, 1] \subset \mathcal{B}([0, 1], \R)$ und jeder $\metric_\infty$-Grenzwert einer Folge aus $C[0, 1]$-Funktionen wieder eine ist, ist $C[0, 1]$ abgeschlossen.
        Laut Satz 9.1.6, ist also tatsächlich $(C[0, 1], \metric_\infty)$ ein vollständiger Metrischer Raum.

        \item Die Totalbeschränktheit zeigen wir nun mit Satz 1.5.1 (Arzelà-Ascoli).

        \includegraphicsboxed{Ana3/Ana3 - Satz 1.5.1 (Arzelà-Ascoli).png}

        \begin{enumerate}[label = \arabic*.]

            \item Schritt ($T(B(0, 1))$ punktweise beschränkt):
            
            \begin{align*}
                \text{d.h.}~
                \Forall x \in [0, 1]:
                \Bbraces{|g(x)|: g \in T(B(0, 1))} ~\text{beschränkt}
            \end{align*}

            $K$ ist stetig, also auf dem kompakten Quader $[0, 1]^2$ beschränkt, d.h. $\norm[\infty]{K} < \infty$.
            Seien nun $x \in [0, 1]$ und $g \in T(B(0, 1))$, dann gibt es ein $f \in B(0, 1)$, d.h. $\norm[\infty]{f} < 1$, sodass $Tf = g$.

            \begin{multline*}
                \implies
                |g(x)|
                =
                |Tf(x)|
                =
                \abs
                {
                    \Int[0][1]{K(\cdot, y) f(y)}{y}
                }
                \leq
                \Int[0][1]{|K(x, y)| |f(y)|}{y} \\
                \leq
                \norm[\infty]{K}
                \underbrace
                {
                    \norm[\infty]{f}
                }_{
                    \leq 1
                }
                \Int[0][1]{}{y}
                \leq
                \norm[\infty]{K}
                <
                \infty
            \end{multline*}

            \item Schritt ($T(B(0, 1))$ gleichgradig stetig):
            
            \begin{align*}
                \text{d.h.}~
                \Forall x \in [0, 1],
                \Forall \varepsilon > 0:
                \Exists \delta > 0:
                \Forall z \in B(0, \delta),
                \Forall g \in T(B(0, 1)):
                |g(x) - g(z)| < \varepsilon
            \end{align*}

            $K$ ist stetig, also auf dem kompakten Quader $[0, 1]^2$ sogar gleichmäßig stetig,
            
            \begin{gather*}
                \text{d.h.}~
                \Forall x \in [0, 1]^2:
                \Forall \varepsilon > 0:
                \Exists \delta > 0:
                \Forall y \in [0, 1]^2:
                \pbraces
                {
                    \metric_{\max}(x, y) < \delta
                    \implies
                    |K(x) - K(y)| < \varepsilon   
                }, \\
                \metric_{\max}(x, y)
                :=
                \max \Bbraces{|x_1 - y_1|, |x_2 - y_2|}.
            \end{gather*}

            Seien also $x \in [0, 1]$ und $\varepsilon > 0$.
            Seien zusätzlich $z \in B(x, \delta)$ und wieder $g \in T(B(0, 1))$ mit $Tf = g$, wobei $f \in B(0, 1)$ und o.B.d.A. $f \neq 0$.

            \begin{align*}
                \Forall y \in [0, 1]:
                \metric_{\max}((x, y), (z, y))
                =
                \max \Bbraces{|x - z|, |y - y|}
                =
                |x - z|
                <
                \delta
            \end{align*}

            Weil $K$ ja gleichmäßig stetig ist, gilt daher $|K(x) - K(y)| < \varepsilon$.

            \begin{multline*}
                \implies
                |g(x) - g(z)|
                =
                |Tf(x) - Tf(z)|
                =
                \abs
                {
                    \Int[0][1]{K(x, y) f(y)}{y}
                    -
                    \Int[0][1]{K(z, y) f(y)}{y}
                } \\
                \leq
                \underbrace
                {
                    \norm[\infty]{f}
                }_{
                    \leq 1
                }
                \Int[0][1]{|K(x, y) - K(z, y)|}{y}
                <
                \varepsilon
            \end{multline*}

        \end{enumerate}

        Der Satz 1.5.1 (Arzelà-Ascoli) liefert uns nun

        \begin{align*}
            \implies
            T(B(0, 1))
            ~\text{totalbeschränkt}~
            \implies
            \overline{T(B(0, 1))}
            ~\text{totalbeschränkt}.
        \end{align*}

    \end{enumerate}

\end{enumerate}

\end{solution}

% -------------------------------------------------------------------------------- %
