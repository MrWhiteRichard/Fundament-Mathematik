% -------------------------------------------------------------------------------- %

\begin{exercise}

Berechnen Sie das Volumen von $K \cap Z$, wobei $K = \Bbraces{(x, y, z): x^2 + y^2 + z^2 \leq 16}$ die Kugel um $0$ mit Radius $4$ und $Z$ der Zylinder $x^2 + y^2 < 4$ ist.

\end{exercise}

% -------------------------------------------------------------------------------- %

\begin{solution}

Der Radius von $K$ ist $\sqrt{16} = 4$ und der $x$-$y$-Radius von $Z$ ist $\sqrt 4 = 2$.
Wir teilen $K \cap Z$ in zwei beschränkte Zyliner $B_\pm$ und Kappen $A_\pm$.
Sei $a$ die Höhe von $B_\pm$ und $b$ der Radius von $K$.
Dann ist $b - a$ die Höhe von $A_\pm$.

\begin{center}
    \begin{tikzpicture}[scale = 0.5]

        % Achsen
        \draw [->] (-10, 0) -- (10, 0)  node [right] {$x$};
        \draw [->] (0, -10) -- (0, 10)  node [above] {$z$};
        \draw [->] (5, 10) -- (-5, -10) node [below left] {$y$};

        % Kugel $K$
        \draw (0, 0) circle (6 cm);

        \draw [dashed] (-6, 5.2) node [left] {$a$} -- (6, 5.2);
        \draw [dashed] (-6, 6)   node [left] {$b$} -- (6, 6);

        % Zylinder $Z$
        \draw (-3, 8) -- (-3, -8)
                      .. controls (-3, -10) and ( 3, -10) .. ( 3, -8)
                      -- ( 3,  8)
                      .. controls ( 3,  10) and (-3,  10) .. (-3,  8)
                      .. controls (-3,  6)  and ( 3,  6)  .. ( 3,  8);

        % Kappen-Boden bzw. -Deckel
        \draw (-3,  5.2) -- (3,  5.2);
        \draw (-3, -5.2) -- (3, -5.2);

        \draw (0, 3) -- (5.2, 3);
        \draw (0, 0) -- (5.2, 3);
        \draw (0.5, 3) arc (0 : -90 : 0.5) (0, 2.5);
        \draw (0.23, 2.75) node {$\cdot$};

        \draw (-0.5,  5.6) node {\scriptsize $A_+$};
        \draw ( 0.5, -5.6) node {\scriptsize $A_-$};
        \draw (-0.5,  3)   node {\scriptsize $B_+$};
        \draw ( 0.5, -3)   node {\scriptsize $B_-$};

        \draw (-1.5, 0) node [below]       {\scriptsize $2$};
        \draw ( 4,   3) node [above]       {\scriptsize $f(z)$};
        \draw ( 4,   2) node [below right] {\scriptsize $4$};

    \end{tikzpicture}
\end{center}

Der Abbildung ist Folgendes zu entnehmen.

\begin{gather*}
    a = \sqrt{4 - 2} = \sqrt 2,
    \quad
    b = \sqrt{16} = 4, \\
    \sqrt{z^2 + f(z)^2} \stackrel{!}{=} \sqrt 4
    \iff
    f(z) := \sqrt{4 - z^2}
\end{gather*}

Wir wollen $A_+$ mit Zylinderkoordinaten parametrisieren.
Das wurde netterweise in Beispiel 4.3.3 bereits gemacht.

\includegraphicsgraphicsboxed
{Ana3/Ana3 - Beispiel 4.3.3.1.png}
{Ana3/Ana3 - Beispiel 4.3.3.2.png}

\begin{align*}
    \psi:
    D_+ := \Bbraces{(r, \varphi, z): r \in [0, f(z)], \varphi \in [0, 2 \pi), z \in [a, b]} \to A_+,
    (r, \varphi, z)
    \mapsto
    \begin{pmatrix}
        r \cos \varphi \\
        r \sin \varphi \\
        z
    \end{pmatrix}
\end{align*}

\begin{multline*}
    \implies
    \lambda^3(A_+)
    =
    \Int[A_+]{}{\lambda^3}
    \stackrel
    {
        \text{TRAFO}
    }{=}
    \Int[D_+]{r}{(r, \varphi, z)}
    =
    \Int[a][b]
    {
        \Int[0][2 \pi]
        {
            \Int[0][f(z)]
            {
                r
            }{r}
        }{\varphi}
    }{z} \\
    =
    2 \pi
    \Int[a][b]
    {
        \frac{1}{2}
        f(z)^2
    }{z}
    =
    \pi
    \Int[a][b]
    {
        4 - z^2
    }{z}
    =
    \pi
    \pbraces
    {
        4 (b - a)
        -
        \frac{1}{3}
        (b^3 - a^3)
    }
\end{multline*}

$A_-$ hat dasselbe Volumen, wie $A_+$.
Das Volument der Zyliner $B_\pm$ ist Grundfläche (Radius Quadrat mal Pi) mal Höhe.

\begin{align*}
    \implies
    \lambda^3(B_\pm)
    =
    \sqrt 4^2 \pi a
\end{align*}

Nun müssen wir nur noch die Volumina der, bis auf $\lambda^3$-Nullmengen disjunkten, Teile von $K \cap Z$ aufsummieren.

\begin{align*}
    \implies
    \lambda^3(K \cap Z)
    =
    \lambda^3(A_+)
    +
    \lambda^3(A_-)
    +
    \lambda^3(B_+)
    +
    \lambda^3(B_-)
\end{align*}

\end{solution}

% -------------------------------------------------------------------------------- %
