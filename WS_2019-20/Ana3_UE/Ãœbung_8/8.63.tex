% --------------------------------------------------------------------------------

\begin{exercise}

Zeigen Sie mit obiger Parametrisierung, dass das Volumen der von einem Spindeltorus berandeten Teilmenge des $\R^3$ gleich

\begin{align*}
    2 \pi
    \pbraces
    {
        R r^2 \arccos(-R / r)
        +
        \frac{2}{3} r^2 \sqrt{r^2 - R^2}
        +
        \frac{1}{3} R^2 \sqrt{r^2 - R^2}
    }
\end{align*}

ist.

\end{exercise}

% --------------------------------------------------------------------------------

\begin{solution}

Wir wollen zuerst das Maß der Oberfläche $F_{R, r}$ und dann, mit der Koflächenformel das des eingeschlossenen Volumen $\Omega_{R, r}$ ausrechnen.

\begin{enumerate}[label = \arabic*.]

    \item Teil (Oberfläche):
    
    \begin{align*}
        F_{R, r}
        :=
        \Bbraces
        {
            \begin{pmatrix}
                (R + r \cos \theta) \cos \varphi \\
                (R + r \cos \theta) \sin \varphi \\
                \sin \theta
            \end{pmatrix}:
            R + r \cos \theta \geq 0,
            \varphi \in [0, 2 \pi)
        }
    \end{align*}
    
    \begin{tcolorbox}[standard jigsaw, opacityback = 0]

        \textbf{Aufgabe 59.}
        Seien $r, z$ $C^1$-Funktionen, $r > 0$ dann wird durch
    
        \begin{align*}
            \phi:
            (a, b) \times [0, 2 \pi) \to \R,
            \phi(t, \varphi) = (r(t) \cos \varphi, r(t) \sin \varphi, z(t))
        \end{align*}
    
        falls diese Funktion injektiv ist eine um die $z$-Achse rotationssymmetrische $2$-dimensionale Fläche $F$ im $\R^3$ definiert.
        Zeigen Sie
    
        \begin{align*}
            \mathcal{H}^2(F)
            =
            2 \pi
            \Int[(a, b)]{r(t) \sqrt{\dot r(t)^2 + \dot z(t)^2}}{t}.
        \end{align*}
    
    \end{tcolorbox}

    Wir machen die Vorbereitungen, um Aufgabe 59 (von letzter Woche) anzuwenden.
    Dabei wird $\theta$ zu $t$.

    \begin{align*}
        r(t) := R + r \cos t & \implies \dot r(t) = -r \sin t \\
        z(t) :=     r \sin t & \implies \dot z(t) =  r \cos t
    \end{align*}

    Für die Integrationsgrenzen $a, b$ von $t$, betrachten wir die Nullstellen der Letzten Bedingung an $t$.

    \begin{align*}
        R + r \cos t = 0
        \iff
        t = \arccos \pbraces{-\frac{R}{r}} =: -a =: b
    \end{align*}

    \begin{center}
        \begin{tikzpicture}
            
            \draw [->] (-5,  0) -- (5, 0) node [right] {$t$};
            \draw [->] ( 0, -4) -- (0, 2);

            \draw (-4, -3) cos (-2, -1) sin (0, 1) cos (2, -1) sin (4, -3);

            \draw [dotted] (-2, -1) -- (2, -1);
            \draw [dotted] (-1.32, -1) -- (-1.32, 0);
            \filldraw (-1.32, 0) circle (1 pt) node [above left]  {$a$};
            \draw [dotted] ( 1.32, -1) -- ( 1.32, 0);
            \filldraw ( 1.32, 0) circle (1 pt) node [above right] {$b$};

            \draw [<->] (3, -1) -- node [above right] {$r$} (3, 1);
            \draw [<->] (4, -1) -- node [right] {$R$} (4, 0);

        \end{tikzpicture}
    \end{center}

    \begin{align*}
        \mathcal{H}^2(F_{R, r})
        & =
        2 \pi
        \Int[(a, b)]{r(t) \sqrt{\dot r(t)^2 + \dot z(t)^2}}{t} \\
        & =
        2 \pi
        \Int
        [-\arccos \pbraces{-\frac{R}{r}}]
        [ \arccos \pbraces{-\frac{R}{r}}]
        {
            (R + r \cos t)
            \sqrt
            {
                (-r \sin t)^2
                +
                ( r \cos t)^2
            }
        }{t} \\
        & =
        4 \pi
        \Int[0][\arccos \pbraces{-\frac{R}{r}}]
        {
            (R + r \cos t) r
        }{t} \\
        & =
        4 \pi r
        \pbraces
        {
            R
            \Int[0][\arccos \pbraces{-\frac{R}{r}}]{}{t}
            +
            r^2
            \Int[0][\arccos \pbraces{-\frac{R}{r}}]
            {
                \cos t
            }{t}
        } \\
        & =
        4 \pi r
        \pbraces
        {
            R
            \arccos \pbraces{-\frac{R}{r}}
            +
            r
            \sin \Big |_{t = 0}^{\arccos \pbraces{-\frac{R}{r}}}
        } \\
        & =
        4 \pi r
        \pbraces
        {
            R
            \arccos \pbraces{-\frac{R}{r}}
            +
            r
            \sin \arccos \pbraces{-\frac{R}{r}}
        } \\
        & =
        4 \pi r
        \pbraces
        {
            R
            \arccos \pbraces{-\frac{R}{r}}
            +
            r
            \sqrt{1 - \pbraces{-\frac{R}{r}}^2}
        } \\
        & =
        4 \pi r
        \pbraces
        {
            R
            \arccos \pbraces{-\frac{R}{r}}
            +
            \sqrt{r^2 - R^2}
        }
    \end{align*}

    \item Teil (Volumen):
    
    \begin{align*}
        \Omega_{R, r}
        :=
        \Bbraces
        {
            (x, y, z) \in R^3:
            (R - \sqrt{x^2 - y^2})^2 + z^2 \leq r^2
        }
    \end{align*}

    Man kann sich letztere Definition so vorstellen:
    Sei $p = (x, y, z)$.
    $\sqrt{x^2 - y^2}$ beschreibt die Länge der Projektion $\pi_{x y}(p)$ von $p$ auf die $x$-$y$-Ebene.
    $R - \sqrt{x^2 - y^2}$ beschreibt, wie weit $\pi_{x y}(p)$ über den Ring, der entsteht, wenn man die Punkte in der oberen Abbildung um die $z$-Achse rotiert (genau genommen der Normalen-Abstand zwischen $\pi_{x y}(p)$ und einem Punkt $q$ aus jenem Ring).
    $\sqrt{(R - \sqrt{x^2 - y^2})^2 + z^2}$ beschreibt den Abstand zwischen $q$ und $p$; dieser ist $\leq r$, genau dann, wenn $p \in \Omega_{R, r}$.

    \begin{align*}
        f & \in C(\Omega_{R, r})^1: (x, y, z) \mapsto \sqrt{(R - \sqrt{x^2 - y^2})^2 + z^2} \\
        g
        & :=
        \frac{1}{|\nabla f|}
        \stackrel
        {
            \text{.ipnb}
        }{=}
        1
    \end{align*}

    Man erinnere sich, was $f(x, y, z)$ bedeutet.
    $f^{-1}(t)$ sind dann genau jene Punkte $p \in \Omega_{R, r}$, für welche $f(p) = t$ gilt.
    Sei $F^\prime_{R, t}$ die Oberfläche des (normalen) Torus mit Schlauch-Dicke $R$ und Ring-Radius (d.h. Länge der Schlauch-Mittelpunkte) $t$.


    \begin{align*}
        f^{-1}(t)
        =
        \begin{cases}
            \emptyset,       & t < 0,           \\
            F^\prime_{R, t}, & 0 \leq t \leq R, \\
            F_{R, t},        & R < t \leq r,    \\
            \emptyset,       & r < t
        \end{cases}
    \end{align*}

    \includegraphicsboxed{Ana3/Ana3 - Satz 4.4.1 (Koflächenformel).png}

    \begin{align*}
        \lambda^3(\Omega_{R, r})
        & =
        \Int[\Omega_{R, r}]
        {
            g |\nabla f|
        }{\lambda^3} \\
        & \stackrel
        {
            \text{KOFLFO}
        }{=}
        \Int[\R]
        {
            \Int[f^{-1}(t)]
            {
                g(y)
            }{\mathcal{H}^2(y)}
        }{t} \\
        & =
        \Int[R][r]
        {
            4 \pi t
            \pbraces
            {
                R \arccos \pbraces{-\frac{R}{t}}
                +
                \sqrt{t^2 - R^2}
            }
        }{t}
        +
        \underbrace
        {
            \lambda^3(\Omega{R, R})
        }_{
            2 \pi^2 R^3
        } \\
        & \stackrel{!}{=}
        \mathrm{rhs}
    \end{align*}

    Die Formel für $\lambda^3(\Omega{R, R})$ kommt aus dem Hinweis vom nächsten Beispiel.
    Für die letzte Rechnung hab ich keinen Bock mehr.

\end{enumerate}

\end{solution}

% --------------------------------------------------------------------------------
