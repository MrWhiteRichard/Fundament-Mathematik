\documentclass{article}

\def \lastexercisenumber{62}

% Hier befinden sich Pakete, die wir beinahe immer benutzen ...

\usepackage[utf8]{inputenc}

% Sprach-Paket:
\usepackage[ngerman]{babel}

% damit's nicht so, wie beim Grill aussieht:
\usepackage{fullpage}

% Mathematik:
\usepackage{amsmath, amssymb, amsfonts, amsthm}
\usepackage{bbm}
\usepackage{mathtools, mathdots}

% Makros mit mehereren Default-Argumenten:
\usepackage{twoopt}

% Anführungszeichen (Makro \Quote{}):
\usepackage{babel}

% if's für Makros:
\usepackage{xifthen}
\usepackage{etoolbox}

% tikz ist kein Zeichenprogramm (doch!):
\usepackage{tikz}

% bessere Aufzählungen:
\usepackage{enumitem}

% (bessere) Umgebung für Bilder:
\usepackage{graphicx, subfig, float}

% Umgebung für Code:
\usepackage{listings}

% Farben:
\usepackage{xcolor}

% Umgebung für "plain text":
\usepackage{verbatim}

% Umgebung für mehrerer Spalten:
\usepackage{multicol}

% "nette" Brüche
\usepackage{nicefrac}

% Spaltentypen verschiedener Dicke
\usepackage{tabularx}
\usepackage{makecell}

% Für Vektoren
\usepackage{esvect}

% (Web-)Links
\usepackage{hyperref}

% Zitieren & Literatur-Verzeichnis
\usepackage[style = authoryear]{biblatex}
\usepackage{csquotes}

% so ähnlich wie mathbb
%\usepackage{mathds}

% Keine Ahnung, was das macht ...
\usepackage{booktabs}
\usepackage{ngerman}
\usepackage{placeins}

% special letters:

\newcommand{\N}{\mathbb{N}}
\newcommand{\Z}{\mathbb{Z}}
\newcommand{\Q}{\mathbb{Q}}
\newcommand{\R}{\mathbb{R}}
\newcommand{\C}{\mathbb{C}}
\newcommand{\K}{\mathbb{K}}
\newcommand{\T}{\mathbb{T}}
\newcommand{\E}{\mathbb{E}}
\newcommand{\V}{\mathbb{V}}
\renewcommand{\S}{\mathbb{S}}
\renewcommand{\P}{\mathbb{P}}
\newcommand{\1}{\mathbbm{1}}

% quantors:

\newcommand{\Forall}{\forall \,}
\newcommand{\Exists}{\exists \,}
\newcommand{\ExistsOnlyOne}{\exists! \,}
\newcommand{\nExists}{\nexists \,}
\newcommand{\ForAlmostAll}{\forall^\infty \,}

% MISC symbols:

\newcommand{\landau}{{\scriptstyle \mathcal{O}}}
\newcommand{\Landau}{\mathcal{O}}


\newcommand{\eps}{\mathrm{eps}}

% graphics in a box:

\newcommandtwoopt
{\includegraphicsboxed}[3][][]
{
  \begin{figure}[!h]
    \begin{boxedin}
      \ifthenelse{\isempty{#1}}
      {
        \begin{center}
          \includegraphics[width = 0.75 \textwidth]{#3}
          \label{fig:#2}
        \end{center}
      }{
        \begin{center}
          \includegraphics[width = 0.75 \textwidth]{#3}
          \caption{#1}
          \label{fig:#2}
        \end{center}
      }
    \end{boxedin}
  \end{figure}
}

% braces:

\newcommand{\pbraces}[1]{{\left  ( #1 \right  )}}
\newcommand{\bbraces}[1]{{\left  [ #1 \right  ]}}
\newcommand{\Bbraces}[1]{{\left \{ #1 \right \}}}
\newcommand{\vbraces}[1]{{\left  | #1 \right  |}}
\newcommand{\Vbraces}[1]{{\left \| #1 \right \|}}
\newcommand{\abraces}[1]{{\left \langle #1 \right \rangle}}
\newcommand{\round}[1]{\bbraces{#1}}

\newcommand
{\floorbraces}[1]
{{\left \lfloor #1 \right \rfloor}}

\newcommand
{\ceilbraces} [1]
{{\left \lceil  #1 \right \rceil }}

% special functions:

\newcommand{\norm}  [2][]{\Vbraces{#2}_{#1}}
\newcommand{\diam}  [2][]{\mathrm{diam}_{#1} \: #2}
\newcommand{\diag}  [1]{\mathrm{diag} \: #1}
\newcommand{\dist}  [1]{\mathrm{dist} \: #1}
\newcommand{\mean}  [1]{\mathrm{mean} \: #1}
\newcommand{\erf}   [1]{\mathrm{erf} \: #1}
\newcommand{\id}    [1]{\mathrm{id} \: #1}
\newcommand{\sgn}   [1]{\mathrm{sgn} \: #1}
\newcommand{\supp}  [1]{\mathrm{supp} \: #1}
\newcommand{\arsinh}[1]{\mathrm{arsinh} \: #1}
\newcommand{\arcosh}[1]{\mathrm{arcosh} \: #1}
\newcommand{\artanh}[1]{\mathrm{artanh} \: #1}
\newcommand{\card}  [1]{\mathrm{card} \: #1}
\newcommand{\Span}  [1]{\mathrm{span} \: #1}
\newcommand{\Aut}   [1]{\mathrm{Aut} \: #1}
\newcommand{\End}   [1]{\mathrm{End} \: #1}
\newcommand{\ggT}   [1]{\mathrm{ggT} \: #1}
\newcommand{\kgV}   [1]{\mathrm{kgV} \: #1}
\newcommand{\ord}   [1]{\mathrm{ord} \: #1}
\newcommand{\grad}  [1]{\mathrm{grad} \: #1}
\newcommand{\ran}   [1]{\mathrm{ran} \: #1}
\newcommand{\graph} [1]{\mathrm{graph} \: #1}
\newcommand{\Inv}   [1]{\mathrm{Inv} \: #1}
\newcommand{\pv}    [1]{\mathrm{pv} \: #1}
\newcommand{\GL}    [1]{\mathrm{GL} \: #1}
\newcommand{\Mod}{\mathrm{Mod} \:}
\newcommand{\Th}{\mathrm{Th} \:}
\newcommand{\Char}{\mathrm{char}}
\newcommand{\At}{\mathrm{At}}
\newcommand{\Ob}{\mathrm{Ob}}
\newcommand{\Hom}{\mathrm{Hom}}
\newcommand{\orthogonal}[3][]{#2 ~\bot_{#1}~ #3}
\newcommand{\Rang}{\mathrm{Rang}}
\newcommand{\NIL}{\mathrm{NIL}}
\newcommand{\Res}{\mathrm{Res}}
\newcommand{\lxor}{\dot \lor}
\newcommand{\Div}{\mathrm{div} \:}
\newcommand{\meas}{\mathrm{meas} \:}

% fractions:

\newcommand{\Frac}[2]{\frac{1}{#1} \pbraces{#2}}
\newcommand{\nfrac}[2]{\nicefrac{#1}{#2}}

% derivatives & integrals:

\newcommandtwoopt
{\Int}[4][][]
{\int_{#1}^{#2} #3 ~\mathrm{d} #4}

\newcommandtwoopt
{\derivative}[3][][]
{
  \frac
  {\mathrm{d}^{#1} #2}
  {\mathrm{d} #3^{#1}}
}

\newcommandtwoopt
{\pderivative}[3][][]
{
  \frac
  {\partial^{#1} #2}
  {\partial #3^{#1}}
}

\newcommand
{\primeprime}
{{\prime \prime}}

\newcommand
{\primeprimeprime}
{{\prime \prime \prime}}

% Text:

\newcommand{\Quote}[1]{\glqq #1\grqq{}}
\newcommand{\Text}[1]{{\text{#1}}}
\newcommand{\fastueberall}{\text{f.ü.}}
\newcommand{\fastsicher}{\text{f.s.}}

% -------------------------------- %
% amsthm-stuff:

\theoremstyle{definition}

% numbered theorems
\newtheorem{theorem}{Satz}
\newtheorem{lemma}{Lemma}
\newtheorem{corollary}{Korollar}
\newtheorem{proposition}{Proposition}
\newtheorem{remark}{Bemerkung}
\newtheorem{definition}{Definition}
\newtheorem{example}{Beispiel}

% unnumbered theorems
\newtheorem*{theorem*}{Satz}
\newtheorem*{lemma*}{Lemma}
\newtheorem*{corollary*}{Korollar}
\newtheorem*{proposition*}{Proposition}
\newtheorem*{remark*}{Bemerkung}
\newtheorem*{definition*}{Definition}
\newtheorem*{example*}{Beispiel}

% Please define this stuff in project ("main.tex"):

% \def \lastexercisenumber {...}
% This will be 0 by default

% \setcounter{section}{...}
% This will be 0 by default
% and hence, completely ignored

\ifnum \thesection = 0
{\newtheorem{exercise}{Aufgabe}}
\else
{\newtheorem{exercise}{Aufgabe}[section]}
\fi

\ifdef
{\lastexercisenumber}
{\setcounter{exercise}{\lastexercisenumber}}

\newcommand{\solution}
{
    \renewcommand{\proofname}{Lösung}
    \renewcommand{\qedsymbol}{}
    \proof
}

\renewcommand{\proofname}{Beweis}

% -------------------------------- %
% environment zum einkasteln:

% dickere vertical lines
\newcolumntype
{x}
[1]
{!{\centering\arraybackslash\vrule width #1}}

% environment selbst (the big cheese)
\newenvironment
{boxedin}
{
  \begin{tabular}
  {
    x{1 pt}
    p{\textwidth}
    x{1 pt}
  }
  \Xhline
  {2 \arrayrulewidth}
}
{
  \\
  \Xhline{2 \arrayrulewidth}
  \end{tabular}
}

% -------------------------------- %
% MISC "Ein-Deutschungen"

\renewcommand
{\figurename}
{Abbildung}

\renewcommand
{\tablename}
{Tabelle}

% -------------------------------- %

% ---------------------------------------------------------------- %
% https://www.overleaf.com/learn/latex/Code_listing

\definecolor{codegreen} {rgb}{0, 0.6, 0}
\definecolor{codegray}    {rgb}{0.5, 0.5, 0.5}
\definecolor{codepurple}{rgb}{0.58, 0, 0.82}
\definecolor{backcolour}{rgb}{0.95, 0.95, 0.92}

\lstdefinestyle{overleaf}
{
    backgroundcolor = \color{backcolour},
    commentstyle = \color{codegreen},
    keywordstyle = \color{magenta},
    numberstyle = \tiny\color{codegray},
    stringstyle = \color{codepurple},
    basicstyle = \ttfamily \footnotesize,
    breakatwhitespace = false,
    breaklines = true,
    captionpos = b,
    keepspaces = true,
    numbers = left,
    numbersep = 5pt,
    showspaces = false,
    showstringspaces = false,
    showtabs = false,
    tabsize = 2
}

% ---------------------------------------------------------------- %
% https://en.wikibooks.org/wiki/LaTeX/Source_Code_Listings

\lstdefinestyle{customc}
{
    belowcaptionskip = 1 \baselineskip,
    breaklines = true,
    frame = L,
    xleftmargin = \parindent,
    language = C,
    showstringspaces = false,
    basicstyle = \footnotesize \ttfamily,
    keywordstyle = \bfseries \color{green!40!black},
    commentstyle = \itshape \color{purple!40!black},
    identifierstyle = \color{blue},
    stringstyle = \color{orange},
}

\lstdefinestyle{customasm}
{
    belowcaptionskip = 1 \baselineskip,
    frame = L,
    xleftmargin = \parindent,
    language = [x86masm] Assembler,
    basicstyle = \footnotesize\ttfamily,
    commentstyle = \itshape\color{purple!40!black},
}

% ---------------------------------------------------------------- %
% https://tex.stackexchange.com/questions/235731/listings-syntax-for-literate

\definecolor{maroon}        {cmyk}{0, 0.87, 0.68, 0.32}
\definecolor{halfgray}      {gray}{0.55}
\definecolor{ipython_frame} {RGB}{207, 207, 207}
\definecolor{ipython_bg}    {RGB}{247, 247, 247}
\definecolor{ipython_red}   {RGB}{186, 33, 33}
\definecolor{ipython_green} {RGB}{0, 128, 0}
\definecolor{ipython_cyan}  {RGB}{64, 128, 128}
\definecolor{ipython_purple}{RGB}{170, 34, 255}

\lstdefinestyle{stackexchangePython}
{
    breaklines = true,
    %
    extendedchars = true,
    literate =
    {á}{{\' a}} 1 {é}{{\' e}} 1 {í}{{\' i}} 1 {ó}{{\' o}} 1 {ú}{{\' u}} 1
    {Á}{{\' A}} 1 {É}{{\' E}} 1 {Í}{{\' I}} 1 {Ó}{{\' O}} 1 {Ú}{{\' U}} 1
    {à}{{\` a}} 1 {è}{{\` e}} 1 {ì}{{\` i}} 1 {ò}{{\` o}} 1 {ù}{{\` u}} 1
    {À}{{\` A}} 1 {È}{{\' E}} 1 {Ì}{{\` I}} 1 {Ò}{{\` O}} 1 {Ù}{{\` U}} 1
    {ä}{{\" a}} 1 {ë}{{\" e}} 1 {ï}{{\" i}} 1 {ö}{{\" o}} 1 {ü}{{\" u}} 1
    {Ä}{{\" A}} 1 {Ë}{{\" E}} 1 {Ï}{{\" I}} 1 {Ö}{{\" O}} 1 {Ü}{{\" U}} 1
    {â}{{\^ a}} 1 {ê}{{\^ e}} 1 {î}{{\^ i}} 1 {ô}{{\^ o}} 1 {û}{{\^ u}} 1
    {Â}{{\^ A}} 1 {Ê}{{\^ E}} 1 {Î}{{\^ I}} 1 {Ô}{{\^ O}} 1 {Û}{{\^ U}} 1
    {œ}{{\oe}}  1 {Œ}{{\OE}}  1 {æ}{{\ae}}  1 {Æ}{{\AE}}  1 {ß}{{\ss}}  1
    {ç}{{\c c}} 1 {Ç}{{\c C}} 1 {ø}{{\o}} 1 {å}{{\r a}} 1 {Å}{{\r A}} 1
    {€}{{\EUR}} 1 {£}{{\pounds}} 1
}


% Python definition (c) 1998 Michael Weber
% Additional definitions (2013) Alexis Dimitriadis
% modified by me (should not have empty lines)

\lstdefinelanguage{iPython}{
    morekeywords = {access, and, break, class, continue, def, del, elif, else, except, exec, finally, for, from, global, if, import, in, is, lambda, not, or, pass, print, raise, return, try, while}, %
    %
    % Built-ins
    morekeywords = [2]{abs, all, any, basestring, bin, bool, bytearray, callable, chr, classmethod, cmp, compile, complex, delattr, dict, dir, divmod, enumerate, eval, execfile, file, filter, float, format, frozenset, getattr, globals, hasattr, hash, help, hex, id, input, int, isinstance, issubclass, iter, len, list, locals, long, map, max, memoryview, min, next, object, oct, open, ord, pow, property, range, raw_input, reduce, reload, repr, reversed, round, set, setattr, slice, sorted, staticmethod, str, sum, super, tuple, type, unichr, unicode, vars, xrange, zip, apply, buffer, coerce, intern}, %
    %
    sensitive = true, %
    morecomment = [l] \#, %
    morestring = [b]', %
    morestring = [b]", %
    %
    morestring = [s]{'''}{'''}, % used for documentation text (mulitiline strings)
    morestring = [s]{"""}{"""}, % added by Philipp Matthias Hahn
    %
    morestring = [s]{r'}{'},     % `raw' strings
    morestring = [s]{r"}{"},     %
    morestring = [s]{r'''}{'''}, %
    morestring = [s]{r"""}{"""}, %
    morestring = [s]{u'}{'},     % unicode strings
    morestring = [s]{u"}{"},     %
    morestring = [s]{u'''}{'''}, %
    morestring = [s]{u"""}{"""}, %
    %
    % {replace}{replacement}{lenght of replace}
    % *{-}{-}{1} will not replace in comments and so on
    literate = 
    {á}{{\' a}} 1 {é}{{\' e}} 1 {í}{{\' i}} 1 {ó}{{\' o}} 1 {ú}{{\' u}} 1
    {Á}{{\' A}} 1 {É}{{\' E}} 1 {Í}{{\' I}} 1 {Ó}{{\' O}} 1 {Ú}{{\' U}} 1
    {à}{{\` a}} 1 {è}{{\` e}} 1 {ì}{{\` i}} 1 {ò}{{\` o}} 1 {ù}{{\` u}} 1
    {À}{{\` A}} 1 {È}{{\' E}} 1 {Ì}{{\` I}} 1 {Ò}{{\` O}} 1 {Ù}{{\` U}} 1
    {ä}{{\" a}} 1 {ë}{{\" e}} 1 {ï}{{\" i}} 1 {ö}{{\" o}} 1 {ü}{{\" u}} 1
    {Ä}{{\" A}} 1 {Ë}{{\" E}} 1 {Ï}{{\" I}} 1 {Ö}{{\" O}} 1 {Ü}{{\" U}} 1
    {â}{{\^ a}} 1 {ê}{{\^ e}} 1 {î}{{\^ i}} 1 {ô}{{\^ o}} 1 {û}{{\^ u}} 1
    {Â}{{\^ A}} 1 {Ê}{{\^ E}} 1 {Î}{{\^ I}} 1 {Ô}{{\^ O}} 1 {Û}{{\^ U}} 1
    {œ}{{\oe}}  1 {Œ}{{\OE}}  1 {æ}{{\ae}}  1 {Æ}{{\AE}}  1 {ß}{{\ss}}  1
    {ç}{{\c c}} 1 {Ç}{{\c C}} 1 {ø}{{\o}} 1 {å}{{\r a}} 1 {Å}{{\r A}} 1
    {€}{{\EUR}} 1 {£}{{\pounds}} 1
    %
    {^}{{{\color{ipython_purple}\^ {}}}} 1
    { = }{{{\color{ipython_purple} = }}} 1
    %
    {+}{{{\color{ipython_purple}+}}} 1
    {*}{{{\color{ipython_purple}$^\ast$}}} 1
    {/}{{{\color{ipython_purple}/}}} 1
    %
    {+=}{{{+=}}} 1
    {-=}{{{-=}}} 1
    {*=}{{{$^\ast$ = }}} 1
    {/=}{{{/=}}} 1,
    literate = 
    *{-}{{{\color{ipython_purple} -}}} 1
     {?}{{{\color{ipython_purple} ?}}} 1,
    %
    identifierstyle = \color{black}\ttfamily,
    commentstyle = \color{ipython_cyan}\ttfamily,
    stringstyle = \color{ipython_red}\ttfamily,
    keepspaces = true,
    showspaces = false,
    showstringspaces = false,
    %
    rulecolor = \color{ipython_frame},
    frame = single,
    frameround = {t}{t}{t}{t},
    framexleftmargin = 6mm,
    numbers = left,
    numberstyle = \tiny\color{halfgray},
    %
    %
    backgroundcolor = \color{ipython_bg},
    % extendedchars = true,
    basicstyle = \scriptsize,
    keywordstyle = \color{ipython_green}\ttfamily,
}

% ---------------------------------------------------------------- %
% https://tex.stackexchange.com/questions/417884/colour-r-code-to-match-knitr-theme-using-listings-minted-or-other

\geometry{verbose, tmargin = 2.5cm, bmargin = 2.5cm, lmargin = 2.5cm, rmargin = 2.5cm}

\definecolor{backgroundCol}  {rgb}{.97, .97, .97}
\definecolor{commentstyleCol}{rgb}{0.678, 0.584, 0.686}
\definecolor{keywordstyleCol}{rgb}{0.737, 0.353, 0.396}
\definecolor{stringstyleCol} {rgb}{0.192, 0.494, 0.8}
\definecolor{NumCol}         {rgb}{0.686, 0.059, 0.569}
\definecolor{basicstyleCol}  {rgb}{0.345, 0.345, 0.345}

\lstdefinestyle{stackexchangeR}
{
    language = R,                                        % the language of the code
    basicstyle = \small \ttfamily \color{basicstyleCol}, % the size of the fonts that are used for the code
    % numbers = left,                                      % where to put the line-numbers
    numberstyle = \color{green},                         % the style that is used for the line-numbers
    stepnumber = 1,                                      % the step between two line-numbers. If it is 1, each line will be numbered
    numbersep = 5pt,                                     % how far the line-numbers are from the code
    backgroundcolor = \color{backgroundCol},             % choose the background color. You must add \usepackage{color}
    showspaces = false,                                  % show spaces adding particular underscores
    showstringspaces = false,                            % underline spaces within strings
    showtabs = false,                                    % show tabs within strings adding particular underscores
    % frame = single,                                      % adds a frame around the code
    % rulecolor = \color{white},                           % if not set, the frame-color may be changed on line-breaks within not-black text (e.g. commens (green here))
    tabsize = 2,                                         % sets default tabsize to 2 spaces
    captionpos = b,                                      % sets the caption-position to bottom
    breaklines = true,                                   % sets automatic line breaking
    breakatwhitespace = false,                           % sets if automatic breaks should only happen at whitespace
    keywordstyle = \color{keywordstyleCol},              % keyword style
    commentstyle = \color{commentstyleCol},              % comment style
    stringstyle = \color{stringstyleCol},                % string literal style
    literate = %
    *{0}{{{\color{NumCol} 0}}} 1
     {1}{{{\color{NumCol} 1}}} 1
     {2}{{{\color{NumCol} 2}}} 1
     {3}{{{\color{NumCol} 3}}} 1
     {4}{{{\color{NumCol} 4}}} 1
     {5}{{{\color{NumCol} 5}}} 1
     {6}{{{\color{NumCol} 6}}} 1
     {7}{{{\color{NumCol} 7}}} 1
     {8}{{{\color{NumCol} 8}}} 1
     {9}{{{\color{NumCol} 9}}} 1
}

% ---------------------------------------------------------------- %
% Fundament Mathematik

\lstdefinestyle{fundament}{basicstyle = \ttfamily}

% ---------------------------------------------------------------- %


\addbibresource{../../../Fundament-LaTeX/references.bib}

\graphicspath{{../../../Fundament-LaTeX/images/}}

\parskip 0pt
\parindent 0pt

% special letters:

\newcommand{\N}{\mathbb{N}}
\newcommand{\Z}{\mathbb{Z}}
\newcommand{\Q}{\mathbb{Q}}
\newcommand{\R}{\mathbb{R}}
\newcommand{\C}{\mathbb{C}}
\newcommand{\K}{\mathbb{K}}
\newcommand{\T}{\mathbb{T}}
\newcommand{\E}{\mathbb{E}}
\newcommand{\V}{\mathbb{V}}
\renewcommand{\S}{\mathbb{S}}
\renewcommand{\P}{\mathbb{P}}
\newcommand{\1}{\mathbbm{1}}

% quantors:

\newcommand{\Forall}{\forall \,}
\newcommand{\Exists}{\exists \,}
\newcommand{\ExistsOnlyOne}{\exists! \,}
\newcommand{\nExists}{\nexists \,}
\newcommand{\ForAlmostAll}{\forall^\infty \,}

% MISC symbols:

\newcommand{\landau}{{\scriptstyle \mathcal{O}}}
\newcommand{\Landau}{\mathcal{O}}


\newcommand{\eps}{\mathrm{eps}}

% graphics in a box:

\newcommandtwoopt
{\includegraphicsboxed}[3][][]
{
  \begin{figure}[!h]
    \begin{boxedin}
      \ifthenelse{\isempty{#1}}
      {
        \begin{center}
          \includegraphics[width = 0.75 \textwidth]{#3}
          \label{fig:#2}
        \end{center}
      }{
        \begin{center}
          \includegraphics[width = 0.75 \textwidth]{#3}
          \caption{#1}
          \label{fig:#2}
        \end{center}
      }
    \end{boxedin}
  \end{figure}
}

% braces:

\newcommand{\pbraces}[1]{{\left  ( #1 \right  )}}
\newcommand{\bbraces}[1]{{\left  [ #1 \right  ]}}
\newcommand{\Bbraces}[1]{{\left \{ #1 \right \}}}
\newcommand{\vbraces}[1]{{\left  | #1 \right  |}}
\newcommand{\Vbraces}[1]{{\left \| #1 \right \|}}
\newcommand{\abraces}[1]{{\left \langle #1 \right \rangle}}
\newcommand{\round}[1]{\bbraces{#1}}

\newcommand
{\floorbraces}[1]
{{\left \lfloor #1 \right \rfloor}}

\newcommand
{\ceilbraces} [1]
{{\left \lceil  #1 \right \rceil }}

% special functions:

\newcommand{\norm}  [2][]{\Vbraces{#2}_{#1}}
\newcommand{\diam}  [2][]{\mathrm{diam}_{#1} \: #2}
\newcommand{\diag}  [1]{\mathrm{diag} \: #1}
\newcommand{\dist}  [1]{\mathrm{dist} \: #1}
\newcommand{\mean}  [1]{\mathrm{mean} \: #1}
\newcommand{\erf}   [1]{\mathrm{erf} \: #1}
\newcommand{\id}    [1]{\mathrm{id} \: #1}
\newcommand{\sgn}   [1]{\mathrm{sgn} \: #1}
\newcommand{\supp}  [1]{\mathrm{supp} \: #1}
\newcommand{\arsinh}[1]{\mathrm{arsinh} \: #1}
\newcommand{\arcosh}[1]{\mathrm{arcosh} \: #1}
\newcommand{\artanh}[1]{\mathrm{artanh} \: #1}
\newcommand{\card}  [1]{\mathrm{card} \: #1}
\newcommand{\Span}  [1]{\mathrm{span} \: #1}
\newcommand{\Aut}   [1]{\mathrm{Aut} \: #1}
\newcommand{\End}   [1]{\mathrm{End} \: #1}
\newcommand{\ggT}   [1]{\mathrm{ggT} \: #1}
\newcommand{\kgV}   [1]{\mathrm{kgV} \: #1}
\newcommand{\ord}   [1]{\mathrm{ord} \: #1}
\newcommand{\grad}  [1]{\mathrm{grad} \: #1}
\newcommand{\ran}   [1]{\mathrm{ran} \: #1}
\newcommand{\graph} [1]{\mathrm{graph} \: #1}
\newcommand{\Inv}   [1]{\mathrm{Inv} \: #1}
\newcommand{\pv}    [1]{\mathrm{pv} \: #1}
\newcommand{\GL}    [1]{\mathrm{GL} \: #1}
\newcommand{\Mod}{\mathrm{Mod} \:}
\newcommand{\Th}{\mathrm{Th} \:}
\newcommand{\Char}{\mathrm{char}}
\newcommand{\At}{\mathrm{At}}
\newcommand{\Ob}{\mathrm{Ob}}
\newcommand{\Hom}{\mathrm{Hom}}
\newcommand{\orthogonal}[3][]{#2 ~\bot_{#1}~ #3}
\newcommand{\Rang}{\mathrm{Rang}}
\newcommand{\NIL}{\mathrm{NIL}}
\newcommand{\Res}{\mathrm{Res}}
\newcommand{\lxor}{\dot \lor}
\newcommand{\Div}{\mathrm{div} \:}
\newcommand{\meas}{\mathrm{meas} \:}

% fractions:

\newcommand{\Frac}[2]{\frac{1}{#1} \pbraces{#2}}
\newcommand{\nfrac}[2]{\nicefrac{#1}{#2}}

% derivatives & integrals:

\newcommandtwoopt
{\Int}[4][][]
{\int_{#1}^{#2} #3 ~\mathrm{d} #4}

\newcommandtwoopt
{\derivative}[3][][]
{
  \frac
  {\mathrm{d}^{#1} #2}
  {\mathrm{d} #3^{#1}}
}

\newcommandtwoopt
{\pderivative}[3][][]
{
  \frac
  {\partial^{#1} #2}
  {\partial #3^{#1}}
}

\newcommand
{\primeprime}
{{\prime \prime}}

\newcommand
{\primeprimeprime}
{{\prime \prime \prime}}

% Text:

\newcommand{\Quote}[1]{\glqq #1\grqq{}}
\newcommand{\Text}[1]{{\text{#1}}}
\newcommand{\fastueberall}{\text{f.ü.}}
\newcommand{\fastsicher}{\text{f.s.}}


\title
{
  Analysis 3 \\
  \vspace{4pt}
  \normalsize
  \textit{8. Übung}
}
\author
{
  Richard Weiss
  \and
  Florian Schager
  \and
  Christian Sallinger
  \and
  Fabian Zehetgruber
  \and
  Paul Winkler
  \and
  Christian Göth
}
\date{2.12.2019}

\begin{document}

\maketitle

Ein \textit{Spindeltorus} entsteht durch Rotation des Kreissegmentes

\begin{align*}
  x & = (R + r \cos \theta) \cos \varphi \\
  y & = (R + r \cos \theta) \sin \varphi \\
  z & = \sin \theta
\end{align*}

um die $z$-Achse für $0 < R < r$, $R + r \cos \theta \geq 0$.

\includegraphicsunboxed[0.5]{Spindeltorus.png}

% --------------------------------------------------------------------------------

\begin{exercise}

Zeigen Sie mit obiger Parametrisierung, dass das Volumen der von einem Spindeltorus berandeten Teilmenge des $\R^3$ gleich

\begin{align*}
    2 \pi
    \pbraces
    {
        R r^2 \arccos(-R / r)
        +
        \frac{2}{3} r^2 \sqrt{r^2 - R^2}
        +
        \frac{1}{3} R^2 \sqrt{r^2 - R^2}
    }
\end{align*}

ist.

\end{exercise}

% --------------------------------------------------------------------------------

\begin{solution}

Wir wollen zuerst das Maß der Oberfläche $F_{R, r}$ und dann, mit der Koflächenformel das des eingeschlossenen Volumen $\Omega_{R, r}$ ausrechnen.

\begin{enumerate}[label = \arabic*.]

    \item Teil (Oberfläche):
    
    \begin{align*}
        F_{R, r}
        :=
        \Bbraces
        {
            \begin{pmatrix}
                (R + r \cos \theta) \cos \varphi \\
                (R + r \cos \theta) \sin \varphi \\
                \sin \theta
            \end{pmatrix}:
            R + r \cos \theta \geq 0,
            \varphi \in [0, 2 \pi)
        }
    \end{align*}
    
    \begin{tcolorbox}[standard jigsaw, opacityback = 0]

        \textbf{Aufgabe 59.}
        Seien $r, z$ $C^1$-Funktionen, $r > 0$ dann wird durch
    
        \begin{align*}
            \phi:
            (a, b) \times [0, 2 \pi) \to \R,
            \phi(t, \varphi) = (r(t) \cos \varphi, r(t) \sin \varphi, z(t))
        \end{align*}
    
        falls diese Funktion injektiv ist eine um die $z$-Achse rotationssymmetrische $2$-dimensionale Fläche $F$ im $\R^3$ definiert.
        Zeigen Sie
    
        \begin{align*}
            \mathcal{H}^2(F)
            =
            2 \pi
            \Int[(a, b)]{r(t) \sqrt{\dot r(t)^2 + \dot z(t)^2}}{t}.
        \end{align*}
    
    \end{tcolorbox}

    Wir machen die Vorbereitungen, um Aufgabe 59 (von letzter Woche) anzuwenden.
    Dabei wird $\theta$ zu $t$.

    \begin{align*}
        r(t) := R + r \cos t & \implies \dot r(t) = -r \sin t \\
        z(t) :=     r \sin t & \implies \dot z(t) =  r \cos t
    \end{align*}

    Für die Integrationsgrenzen $a, b$ von $t$, betrachten wir die Nullstellen der Letzten Bedingung an $t$.

    \begin{align*}
        R + r \cos t = 0
        \iff
        t = \arccos \pbraces{-\frac{R}{r}} =: -a =: b
    \end{align*}

    \begin{center}
        \begin{tikzpicture}
            
            \draw [->] (-5,  0) -- (5, 0) node [right] {$t$};
            \draw [->] ( 0, -4) -- (0, 2);

            \draw (-4, -3) cos (-2, -1) sin (0, 1) cos (2, -1) sin (4, -3);

            \draw [dotted] (-2, -1) -- (2, -1);
            \draw [dotted] (-1.32, -1) -- (-1.32, 0);
            \filldraw (-1.32, 0) circle (1 pt) node [above left]  {$a$};
            \draw [dotted] ( 1.32, -1) -- ( 1.32, 0);
            \filldraw ( 1.32, 0) circle (1 pt) node [above right] {$b$};

            \draw [<->] (3, -1) -- node [above right] {$r$} (3, 1);
            \draw [<->] (4, -1) -- node [right] {$R$} (4, 0);

        \end{tikzpicture}
    \end{center}

    \begin{align*}
        \mathcal{H}^2(F_{R, r})
        & =
        2 \pi
        \Int[(a, b)]{r(t) \sqrt{\dot r(t)^2 + \dot z(t)^2}}{t} \\
        & =
        2 \pi
        \Int
        [-\arccos \pbraces{-\frac{R}{r}}]
        [ \arccos \pbraces{-\frac{R}{r}}]
        {
            (R + r \cos t)
            \sqrt
            {
                (-r \sin t)^2
                +
                ( r \cos t)^2
            }
        }{t} \\
        & =
        4 \pi
        \Int[0][\arccos \pbraces{-\frac{R}{r}}]
        {
            (R + r \cos t) r
        }{t} \\
        & =
        4 \pi r
        \pbraces
        {
            R
            \Int[0][\arccos \pbraces{-\frac{R}{r}}]{}{t}
            +
            r^2
            \Int[0][\arccos \pbraces{-\frac{R}{r}}]
            {
                \cos t
            }{t}
        } \\
        & =
        4 \pi r
        \pbraces
        {
            R
            \arccos \pbraces{-\frac{R}{r}}
            +
            r
            \sin \Big |_{t = 0}^{\arccos \pbraces{-\frac{R}{r}}}
        } \\
        & =
        4 \pi r
        \pbraces
        {
            R
            \arccos \pbraces{-\frac{R}{r}}
            +
            r
            \sin \arccos \pbraces{-\frac{R}{r}}
        } \\
        & =
        4 \pi r
        \pbraces
        {
            R
            \arccos \pbraces{-\frac{R}{r}}
            +
            r
            \sqrt{1 - \pbraces{-\frac{R}{r}}^2}
        } \\
        & =
        4 \pi r
        \pbraces
        {
            R
            \arccos \pbraces{-\frac{R}{r}}
            +
            \sqrt{r^2 - R^2}
        }
    \end{align*}

    \item Teil (Volumen):
    
    \begin{align*}
        \Omega_{R, r}
        :=
        \Bbraces
        {
            (x, y, z) \in R^3:
            (R - \sqrt{x^2 - y^2})^2 + z^2 \leq r^2
        }
    \end{align*}

    Man kann sich letztere Definition so vorstellen:
    Sei $p = (x, y, z)$.
    $\sqrt{x^2 - y^2}$ beschreibt die Länge der Projektion $\pi_{x y}(p)$ von $p$ auf die $x$-$y$-Ebene.
    $R - \sqrt{x^2 - y^2}$ beschreibt, wie weit $\pi_{x y}(p)$ über den Ring, der entsteht, wenn man die Punkte in der oberen Abbildung um die $z$-Achse rotiert (genau genommen der Normalen-Abstand zwischen $\pi_{x y}(p)$ und einem Punkt $q$ aus jenem Ring).
    $\sqrt{(R - \sqrt{x^2 - y^2})^2 + z^2}$ beschreibt den Abstand zwischen $q$ und $p$; dieser ist $\leq r$, genau dann, wenn $p \in \Omega_{R, r}$.

    \begin{align*}
        f & \in C(\Omega_{R, r})^1: (x, y, z) \mapsto \sqrt{(R - \sqrt{x^2 - y^2})^2 + z^2} \\
        g
        & :=
        \frac{1}{|\nabla f|}
        \stackrel
        {
            \text{.ipnb}
        }{=}
        1
    \end{align*}

    Man erinnere sich, was $f(x, y, z)$ bedeutet.
    $f^{-1}(t)$ sind dann genau jene Punkte $p \in \Omega_{R, r}$, für welche $f(p) = t$ gilt.
    Sei $F^\prime_{R, t}$ die Oberfläche des (normalen) Torus mit Schlauch-Dicke $R$ und Ring-Radius (d.h. Länge der Schlauch-Mittelpunkte) $t$.


    \begin{align*}
        f^{-1}(t)
        =
        \begin{cases}
            \emptyset,       & t < 0,           \\
            F^\prime_{R, t}, & 0 \leq t \leq R, \\
            F_{R, t},        & R < t \leq r,    \\
            \emptyset,       & r < t
        \end{cases}
    \end{align*}

    \includegraphicsboxed{Ana3/Ana3 - Satz 4.4.1 (Koflächenformel).png}

    \begin{align*}
        \lambda^3(\Omega_{R, r})
        & =
        \Int[\Omega_{R, r}]
        {
            g |\nabla f|
        }{\lambda^3} \\
        & \stackrel
        {
            \text{KOFLFO}
        }{=}
        \Int[\R]
        {
            \Int[f^{-1}(t)]
            {
                g(y)
            }{\mathcal{H}^2(y)}
        }{t} \\
        & =
        \Int[R][r]
        {
            4 \pi t
            \pbraces
            {
                R \arccos \pbraces{-\frac{R}{t}}
                +
                \sqrt{t^2 - R^2}
            }
        }{t}
        +
        \underbrace
        {
            \lambda^3(\Omega{R, R})
        }_{
            2 \pi^2 R^3
        } \\
        & \stackrel{!}{=}
        \mathrm{rhs}
    \end{align*}

    Die Formel für $\lambda^3(\Omega{R, R})$ kommt aus dem Hinweis vom nächsten Beispiel.
    Für die letzte Rechnung hab ich keinen Bock mehr.

\end{enumerate}

\end{solution}

% --------------------------------------------------------------------------------

% --------------------------------------------------------------------------------

\begin{exercise}

Berechnen Sie aus obiger Volumsformel und der Koflächenformel das Flächenmaß des Spindeltorus.

Hinw.:
Für festes $R$ sei $\T_r$ der Spindeltorus wie zuvor für $R, r$ definiert und $V_t$ das Volumen der von $\T_r$ berandeten Teilmenge des $\R^3$.

Für $R = r$ gilt $V = 2 \pi^2 R^3$.
Für $f(x, y, z) = (\sqrt{x^2 + y^2} - R)^2 + z^2$ gilt $|\nabla f| = 2 \sqrt f$.
Wenden Sie die Koflächenformel auf die Funkton $\1_\bbraces{R^2, r^2}(f(x))$ an.

\end{exercise}

% --------------------------------------------------------------------------------

\begin{solution}

\begin{align*}
    \Omega_{R, r}
    :=
    \Bbraces
    {
        (x, y, z) \in R^3:
        (R - \sqrt{x^2 - y^2})^2 + z^2 \leq r^2
    }
\end{align*}

Formal gilt $\lambda^3(\Omega_{R, t}) = V_t$.
Die obige Volumsformel gibt ja das Volumen des Spindeltorus an; also genau $V_R$.

\begin{align*}
    g := \frac{1}{|\nabla f|} (\1_\bbraces{R^2, r^2} \circ f)
    \implies
    \supp g = \Omega_{R, r} \setminus \Omega_{R, R} =: \Omega
\end{align*}

\begin{align*}
    V_R - V
    & =
    \lambda^3(\Omega) \\
    & =
    \Int[\Omega]
    {
        g |\nabla f|
    }{\lambda^3} \\
    & \stackrel
    {
        \text{KOFLFO}
    }{=}
    \Int[\R]
    {
        \Int[f^{-1}(t)]
        {
            g(y)
        }{\mathcal{H}^2(y)}
    }{t} \\
    & =
    \Int[R^2][r^2]
    {
        \Int[f^{-1}(t)]
        {
            \frac{1}{|\nabla f(y)|}
        }{\mathcal{H}^2(y)}
    }{t} \\
    & =
    \Int[R^2][r^2]
    {
        \Int[f^{-1}(t)]
        {
            \frac{1}{2 \sqrt{f(f^{-1}(t))}}
        }{\mathcal{H}^2(y)}
    }{t} \\
    & =
    \Int[R^2][r^2]
    {
        \frac{1}{2 \sqrt t}
        \Int[f^{-1}(t)]{}{\mathcal{H}^2}
    }{t} \\
    & \stackrel{!}{=}
    \Int[R^2][r^2]
    {
        \frac{1}{2 \sqrt t}
        \mathcal{H}^2(\T_{\sqrt{t}})
    }{t} \\
    & =
    \Int[R][r]
    {
        \mathcal{H}^2(\T_u)
    }{u}
\end{align*}

Dabei haben wir folgende Substitution verwendet.

\begin{align*}
    u = \sqrt t
    \implies
    \derivative[][u]{t} = \frac{1}{2 \sqrt t} \implies \mathrm{d} t = 2 \sqrt t \mathrm{d} u
\end{align*}

Für das \Quote{!}, also die räumliche Interpretation von $f$ konsultiere man die Lösung der vorigen Aufgabe 63.
Nun müssen wir nur noch links und rechts nach $r$ ableiten.

\begin{align*}
    \stackrel
    {
        \derivative{r}
    }{\mapsto}
    \mathcal{H}^2(\T_r)
    =
    4 \pi r
    \pbraces
    {
        R
        \arccos \pbraces{-\frac{R}{r}}
        +
        \sqrt{r^2 - R^2}
    }
\end{align*}

Das stimmt, weil wir das Maß der Oberfläche, in der Lösung der vorigen Aufgabe 63, bereits ausgerechnet haben.

\end{solution}

% --------------------------------------------------------------------------------

% -------------------------------------------------------------------------------- %

\begin{exercise}

Es sei $G \subseteq \R^3$ der Graph der Funktion $f: [0, 1] \times [-1, 1] \to \R$ mit $f(x, y) = x^2 + y$.
Berechnen Sie

\begin{align*}
    \Int[G]{x}{\mathcal{H}^2}.
\end{align*}

\end{exercise}

% -------------------------------------------------------------------------------- %

\begin{solution}

\begin{align*}
    g := \id_{\R^3},
    \quad
    \Omega := (0, 1) \times (-1, 1)
\end{align*}

\includegraphicsboxed{Ana3/Ana3 - Satz 4.2.13.png}

Wir übernehmen sämtliche Bezeichnungen aus Satz 4.2.13.

\begin{align*}
    \implies
    \Int[G]{x}{\mathcal{H}^2(x)}
    & =
    \Int[\varphi(\Omega)]{g}{\mathcal{H}^2} \\
    & \stackrel
    {
        \mathrm{4.2.13}
    }{=}
    \Int[\Omega]
    {
        g(x, f(x))
        \sqrt{1 + |\nabla f(x)|^2}
    }{\lambda^2(x)} \\
    & =
    \Int[-1][1]
    {
        \Int[0][1]
        {
            \begin{pmatrix}
                x \\ y \\ f(x, y)
            \end{pmatrix}
            \sqrt{1 + (2 x)^2 + 1^2}
        }{x}
    }{y} \\
    & =
    \sqrt 2
    \Int[-1][1]
    {
        \Int[0][1]
        {
            \begin{pmatrix}
                x \\ y \\ x^2 + y
            \end{pmatrix}
            \sqrt{1 + 2 x^2}
        }{x}
    }{y}
\end{align*}

\begin{enumerate}[label = \arabic*.]

    \item Komponente:
    
    \begin{multline*}
        \sqrt 2
        \Int[-1][1]
        {
            \Int[0][1]
            {
                x \sqrt{1 + 2 x^2}
            }{x}
        }{y}
        =
        \sqrt 2
        \Int[-1][1]{}{y}
        \Int[0][1]
        {
            x \sqrt{1 + 2 x^2}
        }{x} \\
        =
        2 \sqrt 2
        \Int[1][3]
        {
            x \sqrt u \frac{1}{4 x}
        }{u}
        =
        2 \sqrt 2
        \frac{1}{4}
        \frac{2}{3}
        u^\frac{3}{2} \Big |_{u=1}^3
        =
        \frac{\sqrt 2}{3}
        \pbraces{\sqrt 3^3 - 1}
    \end{multline*}

    Dabei haben wir folgende Substitution verwendet.

    \begin{align*}
        u = 1 + 2 x^2
        \implies
        \derivative[][u]{x} = 4 x
        \implies
        \mathrm{d} x = \frac{1}{4 x} \mathrm{d} u
    \end{align*}

    \item Komponente:

    \item Komponente:

\end{enumerate}

\end{solution}

% -------------------------------------------------------------------------------- %

% --------------------------------------------------------------------------------

\begin{exercise}

Berechnen Sie $\int_D \exp(x / (x + y)) ~\mathrm{d} x ~\mathrm{d} y$, wobei $D$ durch

\begin{align*}
    D = \Bbraces{(x, y) \in \R^2: x, y > 0, x + y < 1}
\end{align*}

definiert ist durch Transformation auf die Variablen $u = x + y$, $\nu = x / (x + y)$.

\end{exercise}

% --------------------------------------------------------------------------------

\begin{solution}

\phantom{}

\begin{center}
    \begin{tikzpicture}

        \draw [->] (-1,  0) -- (3, 0) node [right] {$x$};
        \draw [->] ( 0, -1) -- (0, 3) node [above] {$y$};

        \filldraw [pattern = dots] (0, 0) -- (0, 2) -- node [above right] {$x = 1 - y$} (2, 0) -- cycle;

        \draw (0, 0) node {$\times$} node [below left] {$0$};
        \filldraw (2 cm, 2 pt) -- (2 cm, -2 pt) node [below] {$1$};
        \filldraw (2 pt, 2 cm) -- (-2 pt, 2 cm) node [left]  {$1$};

    \end{tikzpicture}
\end{center}

\includegraphicsboxed{Ana3/Ana3 - Satz 4.3.1 (Transformationsformel).png}

\begin{align*}
    \varphi:
    D \to (0, 1)^2:
    (x, y)
    \mapsto
    \begin{pmatrix}
        x + y \\ \frac{x}{x + y}
    \end{pmatrix},
    \quad
    f: (0, 1)^2 \to \R: (u, \nu) \mapsto u e^\nu
\end{align*}

$\varphi$ ist tatsächlich ein Diffeomorphismus.

\begin{align*}
    \begin{pmatrix}
        u \\ \nu
    \end{pmatrix}
    \stackrel{!}{=}
    \begin{pmatrix}
        x + y \\ \frac{x}{x + y}
    \end{pmatrix}
    & \implies
    y = u - x \\
    & \implies
    \nu = \frac{x}{	x + (u - x)} = \frac{x}{u} \\
    & \implies
    x = \nu u \\
    & \implies
    y = u - \nu u = (1 - \nu) u
\end{align*}

\begin{align*}
    |\det \mathrm{d} \varphi(x, y)|
    =
    \abs
    {
        \det
        \begin{pmatrix}
            1                             & 1 \\
            \frac{(x + y) - x}{(x + y)^2} & \frac{-1}{x (x + y)^2}
        \end{pmatrix}
    }
    =
    \abs
    {
        -
        \frac{x}{(x + y)^2}
        -
        \frac{y}{(x + y)^2}
    }
    =
    \frac{x + y}{(x + y)^2}
    =
    \frac{1}{x + y}
\end{align*}

\begin{multline*}
    \Int[D]
    {
        \exp \frac{x}{x + y}
    }{(x, y)}
    =
    \Int[D]
    {
        (x + y)
        \exp \frac{x}{x + y}
        \frac{1}{x + y}
    }{(x, y)}
    =
    \Int[D]
    {
        f(\varphi(x, y))
        |\det \mathrm{d} \varphi(x, y)|
    }{(x, y)} \\
    \stackrel
    {
        \text{TRAFO}
    }{=}
    \Int[\varphi(D)]
    {
        f(u, \nu)
    }{(u, \nu)}
    =
    \Int[0][1]
    {
        \Int[0][1]
        {
            u e^\nu
        }{u}
    }{\nu}
    =
    \Int[0][1]{u}{u}
    \Int[0][1]{e^\nu}{\nu}
    =
    \frac{1}{2}
    u^2 \Big |_{u=0}^1
    e^\nu \Big |_{\nu = 0}^1
    =
    \frac{1}{2}
    (e - 1)
\end{multline*}

\end{solution}

% --------------------------------------------------------------------------------

% --------------------------------------------------------------------------------

\begin{exercise}

Berechnen Sie das Volumen von $K \cap Z$, wobei $K = \Bbraces{(x, y, z): x^2 + y^2 + z^2 \leq 16}$ die Kugel um $0$ mit Radius $4$ und $Z$ der Zylinder $x^2 + y^2 < 4$ ist.

\end{exercise}

% --------------------------------------------------------------------------------

\begin{solution}

Der Radius von $K$ ist $\sqrt{16} = 4$ und der $x$-$y$-Radius von $Z$ ist $\sqrt 4 = 2$.
Wir teilen $K \cap Z$ in zwei beschränkte Zyliner $B_\pm$ und Kappen $A_\pm$.
Sei $a$ die Höhe von $B_\pm$ und $b$ der Radius von $K$.
Dann ist $b - a$ die Höhe von $A_\pm$.

\begin{center}
    \begin{tikzpicture}[scale = 0.5]

        % Achsen
        \draw [->] (-10, 0) -- (10, 0)  node [right] {$x$};
        \draw [->] (0, -10) -- (0, 10)  node [above] {$z$};
        \draw [->] (5, 10) -- (-5, -10) node [below left] {$y$};

        % Kugel $K$
        \draw (0, 0) circle (6 cm);

        \draw [dashed] (-6, 5.2) node [left] {$a$} -- (6, 5.2);
        \draw [dashed] (-6, 6)   node [left] {$b$} -- (6, 6);

        % Zylinder $Z$
        \draw (-3, 8) -- (-3, -8)
                      .. controls (-3, -10) and ( 3, -10) .. ( 3, -8)
                      -- ( 3,  8)
                      .. controls ( 3,  10) and (-3,  10) .. (-3,  8)
                      .. controls (-3,  6)  and ( 3,  6)  .. ( 3,  8);

        % Kappen-Boden bzw. -Deckel
        \draw (-3,  5.2) -- (3,  5.2);
        \draw (-3, -5.2) -- (3, -5.2);

        \draw (0, 3) -- (5.2, 3);
        \draw (0, 0) -- (5.2, 3);
        \draw (0.5, 3) arc (0 : -90 : 0.5) (0, 2.5);
        \draw (0.23, 2.75) node {$\cdot$};

        \draw (-0.5,  5.6) node {\scriptsize $A_+$};
        \draw ( 0.5, -5.6) node {\scriptsize $A_-$};
        \draw (-0.5,  3)   node {\scriptsize $B_+$};
        \draw ( 0.5, -3)   node {\scriptsize $B_-$};

        \draw (-1.5, 0) node [below]       {\scriptsize $2$};
        \draw ( 4,   3) node [above]       {\scriptsize $f(z)$};
        \draw ( 4,   2) node [below right] {\scriptsize $4$};

    \end{tikzpicture}
\end{center}

Der Abbildung ist Folgendes zu entnehmen.

\begin{gather*}
    a = \sqrt{4 - 2} = \sqrt 2,
    \quad
    b = \sqrt{16} = 4, \\
    \sqrt{z^2 + f(z)^2} \stackrel{!}{=} \sqrt 4
    \iff
    f(z) := \sqrt{4 - z^2}
\end{gather*}

Wir wollen $A_+$ mit Zylinderkoordinaten parametrisieren.
Das wurde netterweise in Beispiel 4.3.3 bereits gemacht.

\includegraphicsgraphicsboxed
{Ana3/Ana3 - Beispiel 4.3.3.1.png}
{Ana3/Ana3 - Beispiel 4.3.3.2.png}

\begin{align*}
    \psi:
    D_+ := \Bbraces{(r, \varphi, z): r \in [0, f(z)], \varphi \in [0, 2 \pi), z \in [a, b]} \to A_+,
    (r, \varphi, z)
    \mapsto
    \begin{pmatrix}
        r \cos \varphi \\
        r \sin \varphi \\
        z
    \end{pmatrix}
\end{align*}

\begin{multline*}
    \implies
    \lambda^3(A_+)
    =
    \Int[A_+]{}{\lambda^3}
    \stackrel
    {
        \text{TRAFO}
    }{=}
    \Int[D_+]{r}{(r, \varphi, z)}
    =
    \Int[a][b]
    {
        \Int[0][2 \pi]
        {
            \Int[0][f(z)]
            {
                r
            }{r}
        }{\varphi}
    }{z} \\
    =
    2 \pi
    \Int[a][b]
    {
        \frac{1}{2}
        f(z)^2
    }{z}
    =
    \pi
    \Int[a][b]
    {
        4 - z^2
    }{z}
    =
    \pi
    \pbraces
    {
        4 (b - a)
        -
        \frac{1}{3}
        (b^3 - a^3)
    }
\end{multline*}

$A_-$ hat dasselbe Volumen, wie $A_+$.
Das Volument der Zyliner $B_\pm$ ist Grundfläche (Radius Quadrat mal Pi) mal Höhe.

\begin{align*}
    \implies
    \lambda^3(B_\pm)
    =
    \sqrt 4^2 \pi a
\end{align*}

Nun müssen wir nur noch die Volumina der, bis auf $\lambda^3$-Nullmengen disjunkten, Teile von $K \cap Z$ aufsummieren.

\begin{align*}
    \implies
    \lambda^3(K \cap Z)
    =
    \lambda^3(A_+)
    +
    \lambda^3(A_-)
    +
    \lambda^3(B_+)
    +
    \lambda^3(B_-)
\end{align*}

\end{solution}

% --------------------------------------------------------------------------------

% --------------------------------------------------------------------------------

\begin{exercise}

Beweisen Sie die Leibnizsche Sektorformel:

Die Fläche des Sektors

\begin{align*}
    \Bbraces
    {
        (x, y):
        x = r \cos \varphi,
        y = r \sin \varphi,
        \alpha < \varphi < \beta,
        0 < r < f(\varphi)
    }
\end{align*}

ist

\begin{align*}
    \frac{1}{2}
    \Int[\alpha][\beta]
    {
        f^2(\varphi)
    }{\varphi}.
\end{align*}

\end{exercise}

% --------------------------------------------------------------------------------

\begin{solution}

ToDo!

\end{solution}

% --------------------------------------------------------------------------------

% -------------------------------------------------------------------------------- %

\begin{exercise}

Berechnen Sie das Flächenmaß eines Kegelmantels

\begin{align*}
    \Bbraces
    {
        x \in \R^3:
        0 < x_3 < \pbraces{R - \sqrt{x_1^2 + x_2^2}} h
    }
\end{align*}

mit der Flächenformel.

\end{exercise}

% -------------------------------------------------------------------------------- %

\begin{solution}

Sei $A_R$ der obere Kegelmantel.
Wir wollen Satz 4.2.15 anwenden, um $\mathcal{H}(A_R)$ auszurechnen.

\includegraphicsboxed{Ana3/Ana3 - Satz 4.2.15.png}

Um das $f$ für die Parametrisierung herauszufinden, setzen wir für $x_1$, $x_2$, und $x_3$ ein.

\begin{align*}
    &
    z
    \stackrel{!}{<}
    (R - \sqrt{(f(z) \cos \theta)^2 + (f(z) \sin \theta)^2}) h
    =
    (R - |f(z)|) h \\
    \impliedby &
    f(z) = R - \frac{z}{h} \geq 0
\end{align*}

Weil $x_1 = x_2 = 0$ sein kann, ist $z \in (a, b)$, mit $a := 0$ und $b := R h$.

\begin{align*}
    \implies
    \mathcal{H}(A_R)
    & =
    2 \pi
    \Int[a][b]
    {
        |f(z)| \sqrt{1 + f^\prime(z)^2}
    }{z} \\
    & =
    2 \pi
    \Int[0][R h]
    {
        \pbraces{R - \frac{z}{h}}
        \sqrt{1 + \pbraces{-\frac{1}{h}}^2}
    }{z} \\
    & =
    2 \pi \sqrt{1 + \frac{1}{h^2}}
    \pbraces
    {
        R
        \Int[0][R h]{}{z}
        -
        \frac{1}{h}
        \Int[0][R h]{z}{z}
    } \\
    & =
    2 \pi \sqrt{1 + \frac{1}{h^2}}
    \pbraces{R^2 h - \frac{1}{h} \frac{1}{2} R^2 h^2} \\
    & =
    2 \pi \sqrt{1 + \frac{1}{h^2}}
    \frac{R^2 h}{2} \\
    & =
    \pi R^2 \sqrt{h^2 + 1}
\end{align*}

\end{solution}

% -------------------------------------------------------------------------------- %

% -------------------------------------------------------------------------------- %

\begin{exercise}

Berechnen Sie das Flächenmaß eines Kegelmantels

\begin{align*}
    \Bbraces
    {
        x \in \R^3:
        0 < x_3 = \pbraces{R - \sqrt{x_1^2 + x_2^2}} h
    }
\end{align*}

mit der Koflächenformel aus der Formel für das Volumen

\begin{align*}
    V_{R, h} = \pi R^2 h / 3
\end{align*}

des Kegels.

\end{exercise}

% -------------------------------------------------------------------------------- %

\begin{solution}

Sei $A_R$ der obere Kegelmantel.
Sei $\Omega$ die dadurch eingeschlossene Fläche.

\begin{align*}
    \Omega
    =
    \bigcup_{t \in (0, R)}
        A_t
\end{align*}

Sei nun $t \in (0, R)$.

\begin{align*}
    &
    A_t \stackrel{!}{=} f^{-1}(t) = \Bbraces{x \in \R: f(x) = t} \\
    \iff &
    f
    \begin{pmatrix}
        x_1 \\ x_2 \\ x_3 = \pbraces{t - \sqrt{x_1^2 + x_2^2}} h
    \end{pmatrix}
    =
    f(x)
    \stackrel{!}{=}
    t \\
    \iff &
    f(x) := \frac{x_3}{h} \sqrt{x_1^2 + x_2^2},
    \quad
    x \in \Omega \\
    \implies &
    |\nabla f(x)|^2
    =
    \pbraces
    {
        \frac{1}{2}
        \frac{1}{\sqrt{x_1^2 + x_2^2}}
        2 x_1
    }^2
    +
    \pbraces
    {
        \frac{1}{2}
        \frac{1}{\sqrt{x_1^2 + x_2^2}}
        2 x_2
    }^2
    +
    \frac{1}{h^2}
    =
    1 + \frac{1}{h^2} > 0
\end{align*}

Wir setzen nun $g := \frac{1}{|\nabla f|}$ und wenden die Koflächenformel an.

\begin{align*}
    \implies &
    \pi R^2 h / 3
    =
    V_{R, h}
    =
    \lambda^3(\Omega)
    =
    \Int[\Omega]{g |\nabla f|}{\lambda}
    \stackrel
    {
        \text{KOFLFO}
    }{=}
    \Int[\R]
    {
        \Int[f^{-1}(t)]
        {
            g
        }{\mathcal{H}^2}
    }{t}
    =
    \frac{1}{\sqrt{1 + \frac{1}{h^2}}}
    \Int[0][R]
    {
        \mathcal{H}^2(A_t)
    }{t} \\
    \implies &
    \frac{1}{3}
    \pi R^3 h
    \sqrt{1 + \frac{1}{h^2}}
    =
    \Int[0][R]
    {
        \mathcal{H}^2(A_t)
    }{t} \\
    \implies &
    \mathcal{H}^2(A_R)
    =
    \pi R^2 \sqrt{h^2 + 1}
\end{align*}

\end{solution}

% -------------------------------------------------------------------------------- %

% -------------------------------------------------------------------------------- %

\begin{exercise}

Berechnen Sie für $0 < a < b$ und $0 < c < d$ die durch $a x < y^2 < b x$ und $c y < x^2 < d y$ definierte Fläche in $\R^2$ durch eine Koordinatentransformation unter der die Fläche in ein Rechteck übergeht.

\end{exercise}

% -------------------------------------------------------------------------------- %

\begin{solution}

\begin{align*}
    \varphi:
    A \to B:
    (x, y)
    \mapsto
    \begin{pmatrix}
        \frac{y^2}{x} \\ \frac{x^2}{y}
    \end{pmatrix}
\end{align*}

\begin{align*}
    A
    & :=
    \Bbraces
    {
        (x, y) \in \R^2:
        a < \frac{y^2}{x} < b,
        c < \frac{x^2}{y} < d
    } \\
    B
    & :=
    \Bbraces
    {
        \pbraces
        {
            \frac{y^2}{x},
            \frac{x^2}{y}
        }:
        a < \frac{y^2}{x} < b,
        c < \frac{x^2}{y} < d
    }
    =
    (a, b) \times (c, d)
\end{align*}

\begin{align*}
    |\det \mathrm{d} \varphi(x, y)|
    =
    \abs
    {
        \det
        \begin{pmatrix}
            -\frac{y^2}{x^2} &  \frac{2 y}{x} \\
            \frac{2 x}{y}    & -\frac{x^2}{y^2}
        \end{pmatrix}
    }
    =
    \abs
    {
        \pbraces{-\frac{y^2}{x^2}}
        \pbraces{-\frac{x^2}{y^2}}
        -
        \frac{2 x}{y}
        \frac{2 y}{x}
    }
    =
    |1 - 4|
    =
    3
\end{align*}

\begin{align*}
    \lambda^2(A)
    =
    \frac{1}{3}
    \Int[A]
    {
        |\det \mathrm{d} \varphi|
    }{\lambda^2}
    \stackrel
    {
        \text{TRAFO}
    }{=}
    \frac{1}{3}
    \Int[\varphi(A)]{}{\lambda^2}
    =
    \frac{1}{3}
    \lambda^2(B)
    =
    \frac{1}{3}
    (b - a)
    (d - c)
\end{align*}

\end{solution}

% -------------------------------------------------------------------------------- %


\printbibliography

\end{document}
