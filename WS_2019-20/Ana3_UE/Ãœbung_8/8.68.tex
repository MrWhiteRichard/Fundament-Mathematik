% --------------------------------------------------------------------------------

\begin{exercise}

Beweisen Sie die Leibnizsche Sektorformel:

Die Fläche des Sektors

\begin{align*}
    \Bbraces
    {
        (x, y):
        x = r \cos \varphi,
        y = r \sin \varphi,
        \alpha < \varphi < \beta,
        0 < r < f(\varphi)
    }
\end{align*}

ist

\begin{align*}
    \frac{1}{2}
    \Int[\alpha][\beta]
    {
        f^2(\varphi)
    }{\varphi}.
\end{align*}

\end{exercise}

% --------------------------------------------------------------------------------

\begin{solution}

Sei $B$ der obige Sektor.

\begin{align*}
    A := \Bbraces{(r, \varphi): \alpha < \varphi < \beta, 0 < r < f(\varphi)}
\end{align*}

Wir wollen $A$ auf $B$, mitttels Polarkoordiniaten $\phi$, transformieren.

\includegraphicsboxed{Ana3/Ana3 - Beispiel 4.3.2.png}

\begin{align*}
    \implies
    \mathcal{H}(B)
    =
    \mathcal{H}(\phi(A))
    =
    \Int[\phi(A)]{}{\lambda}
    \stackrel
    {
        \text{TRAFO}
    }{=}
    \Int[\alpha][\beta]
    {
        \Int[0][f(\varphi)]
        {
            r
        }{r}
    }{\varphi}
    =
    \frac{1}{2}
    \Int[\alpha][\beta]
    {
        f^2(\varphi)
    }{
        \varphi
    }
\end{align*}

\end{solution}

% --------------------------------------------------------------------------------
