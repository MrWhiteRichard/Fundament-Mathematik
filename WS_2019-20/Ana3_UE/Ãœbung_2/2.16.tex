% --------------------------------------------------------------------------------

\begin{exercise}

Zeigen Sie, dass für $f, g \in L^1(\R^n)$ $\supp(f \ast g)$ keine Teilmenge von $\supp(f) + \supp(g)$ sein muss.

\end{exercise}

% --------------------------------------------------------------------------------

\begin{solution}

Wir erinnern uns zunächst an folgende Definitionen.

\begin{align*}
    \supp f
    :=
    \overline
    {
        \Bbraces
        {
            x \in \R^n:
            f(x) \neq 0
        }
    },
    \quad
    A + B
    :=
    \Bbraces
    {
        a + b:
        a \in A,
        b \in B
    },
    \quad
    A, B \subseteq \R^n
\end{align*}

Die folgenden Mengen $X, Y$ sind zwar abgeschlossen, ihre Summe $X + Y$ aber nicht.

\begin{gather*}
    X := \Bbraces{x \in \R^2: x_1 > 0, x_1 x_2 \geq 1} \subset \R^+ \times \R^+,
    \quad
    Y := \Bbraces{x \in \R^2: y_1 > 0, y_1 y_2 \leq -1} \subset \R^+ \times \R^- \\
    \implies
    X + Y = \R^+ \times \R
\end{gather*}

Damit konstruieren wir unsere Kandidaten.

\begin{align*}
    f(x)
    :=
    e^{-(x_1 + x_2)} \1_X(x),
    \quad
    x \in \R^2,
    \quad
    g(y)
    :=
    e^{-(y_1 - y_2)} \1_Y(y),
    \quad
    y \in \R^2
\end{align*}

Diese sind in der Tat integrierbar, d.h. $f, g \in L^1(\R^2)$.

\begin{align*}
    \norm[L^1(\R^2)]{f}
    & =
    \Int[\R^2]{|f|}{\lambda^2}
    \leq
    \underbrace
    {
        \pbraces
        {
            \Int[0][\infty]{e^{-x_1}}{x_1}
        }
    }_1
    \underbrace
    {
        \pbraces
        {
            \Int[0][\infty]{e^{-x_2}}{x_2}
        }
    }_1
    <
    \infty, \\
    \norm[L^1(\R^2)]{g}
    & =
    \Int[\R^2]{|g|}{\lambda^2}
    \leq
    \underbrace
    {
        \pbraces
        {
            \Int[0][\infty]{e^{y_1}}{y_1}
        }
    }_1
    \underbrace
    {
        \pbraces
        {
            \Int[-\infty][0]{e^{-y_2}}{y_2}
        }    
    }_1
    <
    \infty
\end{align*}

Wir berechnen deren Faltung an der Stelle $\pbraces{\frac{1}{n}, 0}$ für $n \in \N$.

\begin{multline*}
    (f \ast g) \pbraces{\frac{1}{n}, 0}
    =
    \Int[\R^2]{f \pbraces{\frac{1}{n} - y_1} g(y)}{\lambda^2(y)} \\
    =
    \Int[\R^2]
    {
        e^{y_1 + y_2 - \frac{1}{n}}
        \1_X \pbraces{\frac{1}{n} - y_1, -y_2}
        e^{-(y_1 - y_2)}
        \1_Y(y)
    }{\lambda^2(y)}
    =
    \cdots
\end{multline*}

\begin{align*}
    y \in Y
    \iff
    \begin{cases}
        y_1 > 0, \\
        y_1 y_2 \leq -1 \iff y_2 \leq -\frac{1}{y_1}
    \end{cases}
\end{align*}

\begin{multline*}
    \implies
    \cdots
    =
    \Int[0][\infty]
    {
        \Int[-\infty][-\frac{1}{y_1}]
        {
            e^{2 y_2 - \frac{1}{n}}
            \1_X \pbraces{\frac{1}{n} - y_1, y_2}
        }{y_2}
    }{y_1} \\
    =
    \Int[0][\infty]
    {
        \Int[\frac{1}{y_1}][\infty]
        {
            e^{-(2 y_2 + \frac{1}{n})}
            \1_X \pbraces{\frac{1}{n} - y_1, y_2}
        }{y_2}
    }{y_1}
    =
    \cdots
\end{multline*}

\begin{align*}
    \pbraces{\frac{1}{n} - y_1, y_2} \in X
    \iff
    \begin{cases}
        \frac{1}{n} - y_1 > 0 \iff \frac{1}{n} > y_1 \\
        \pbraces{\frac{1}{n} - y_1} y_2 \geq 1 \iff y_2 \geq \frac{n}{1 - n y_1}
    \end{cases}
\end{align*}

\begin{align*}
    \implies
    \cdots
    =
    \Int[0][\frac{1}{n}]
    {
        \Int[\frac{n}{1 - n y_1}][\infty]
        {
            e^{-(2 y_2 + \frac{1}{n})}
        }{y_2}
    }{y_1}
    >
    0
\end{align*}

Damit ist also die ganze Folge $\pbraces{\frac{1}{n}, 0}_{n \in \N} \subset \supp(f \ast g)$.
Weil der $\supp$ abgeschlossen ist, liegt auch $0 \in \supp(f \ast g)$.

$X$ und $Y$ sind, wie gesagt, abgeschlossen.

\begin{align*}
    \implies
    \supp f \subset X,
    \quad
    \supp g \subset Y
    \implies
    \supp f + \supp g \subset X + Y = \R^+ \times \R
\end{align*}

$0$ liegt allerdings in keiner Mengen-Summe.

\begin{align*}
    0 \not \in \R^+ \times \R = X + Y
    \implies
    0
    \begin{cases}
        \not \in \supp f + \supp g, \\
        \in \supp(f \ast g)
    \end{cases}
    \implies
    \supp f + \supp g
    \not \subset
    \supp(f \ast g)
\end{align*}

\end{solution}

% --------------------------------------------------------------------------------
