% --------------------------------------------------------------------------------

\begin{exercise}

Sei $(M, \metric)$ ein metrischer Raum.
Dann ist $M \times M$ mit der Maximumsmetrik $\metric_m((x_1, x_2), (y_1, y_2)) := \max(\metric(x_1, y_1), \metric(x_2, y_2))$ ein metrischer Raum und für eine Vervollständigung $(\tilde M, \tilde \metric)$ von $M$ ist $\tilde M \times \tilde M$ mit der Maximumsmetrik eine Vervollständigung von $M \times M$.

\end{exercise}

% --------------------------------------------------------------------------------

\begin{solution}

\phantom{}

\begin{enumerate}[label = \arabic*.]

    \item Teil:
    
    Wir zeigen, dass $\metric_{\max} := \metric_m$ eine Metrik ist.
    Dazu überprüfen (bzw. behaupten) wir $\Forall x, y, z \in M^2$ die folgenden 4 Eigenschaften.
    
    \begin{enumerate}[label = \arabic*.]

        \item Eigenschaft (Nicht-Negativität):
        
        \begin{align*}
            \text{d.h.}~
            \metric_{\max}(x, y) \geq 0
        \end{align*}

        \item Eigenschaft (Definitheit):
        
        \begin{align*}
            \text{d.h.}~
            \metric_{\max}(x, y) = 0 \iff x = y
        \end{align*}

        \item Eigenschaft (Symmetrie):
        
        \begin{align*}
            \text{d.h.}~
            \metric_{\max}(x, y) = \metric_{\max}(y, x)
        \end{align*}

        \item Eigenschaft (Dreiecksungleichung):
        
        \begin{align*}
            \text{d.h.}~
            \metric_{\max}(x, y)
            +
            \metric_{\max}(y, z)
            \leq
            \metric_{\max}(x, z)
        \end{align*}

    \end{enumerate}

    \item Teil:
    
    \includegraphicsboxed{Ana1&2/Ana1&2 - 6.7.1 Definition.png}

    Wir zeigen also, dass folgender metrischer Raum vollständig ist und es eine passende Isometrie gibt.
    Laut dem 1. Teil ist er ja bereits ein metrischer Raum, weil $(\tilde M, \tilde \metric)$ einer ist.

    \begin{align*}
        (\tilde M^2, \tilde \metric_{\max}),
        \quad
        \metric_{\max}(x, y)
        :=
        \max
        (
            \tilde \metric(x_1, y_1),
            \tilde \metric(x_2, y_2)
        )
    \end{align*}

    \begin{enumerate}[label = 2.\arabic*.]

        \item Teil (Vollständigkeit):
        
        Sei dazu $(x_n)_{n \in \N} \in (\tilde M^2)^\N$ eine Cauchy-Folge bzgl. $\tilde \metric_{\max}$, d.h.

        \begin{align*}
            \Forall \varepsilon > 0:
            \Exists N \in \N:
            \Forall n, m \geq N:
            \varepsilon
            >
            \tilde \metric_{\max}(x_n, x_m)
            =
            \max
            (
                \tilde \metric(x_{n, 1}, x_{m, 1}),
                \tilde \metric(x_{n, 2}, x_{m, 2})
            ).
        \end{align*}

        Offenbar sind dann $(x_{n, 1})_{n \in \N}, (x_{n, 2})_{n \in \N} \in (\tilde M^2)^\N$ Cauchy-Folgen bzgl. $\tilde \metric$.
        Weil $(\tilde M, \tilde \metric)$ ja vollständig ist, konvergieren die gegen Grenzwerte $x_1 \in \tilde M$ bzw. $x_2 \in \tilde M$.

        Unseren Grenzwert-Kandidat für $(x_n)_{n \in \N} \in (\tilde M^2)^\N$ setzen wir also gleich $x := (x_1, x_2)$.
        Dieser ist nicht nur ein Kandidat, weil

        \begin{align*}
            \tilde \metric_{\max}(x_n, x)
            =
            \max
            (
                \tilde \metric(x_{n, 1}, x_1),
                \tilde \metric(x_{n, 2}, x_2)
            )
            \xrightarrow{n \to \infty}
            0.
        \end{align*}

        \item Teil (Isometrie):
        
        Weil $(\tilde M, \tilde \metric)$ eine Vervollständigung von $(M, \metric)$ ist,

        \begin{align*}
            \Exists \iota:
            (M, \metric) \to (\tilde M, \tilde \metric) ~\text{Isometrie}:
            \overline{\iota(M)} = \tilde M.
        \end{align*}

        Daraus können wir einen Kandidaten für $(\tilde M^2, \tilde \metric_{\max})$ basteln.

        \begin{align*}
            \iota^2:
            (M^2, \metric_{\max})
            \to
            (\tilde M^2, \tilde \metric_{\max}):
            (x_1, x_2)
            \mapsto
            (
                \iota(x_1),
                \iota(x_2)
            )
        \end{align*}

        $\iota^2$ ist tatsächlich isometrisch, weil $\iota$ eine ist, also $\Forall x, y \in M^2:$

        \begin{align*}
            \metric_{\max}(x, y)
            =
            \max
            (
                \metric(x_1, y_1),
                \metric(x_2, y_2)
            )
            =
            \max
            (
                \tilde \metric(\iota(x_1), \iota(y_1)),
                \tilde \metric(\iota(x_2), \iota(y_2))
            )
            =
            \tilde \metric_{\max}(\iota^2(x), \iota^2(y)).
        \end{align*}

        Für die Dichtheit des Bildes von $\iota^2$, d.h. $\overline{\iota^2(M^2)} = \tilde M^2$, sind 2 Inklusionen zu zeigen.

        \begin{enumerate}[label = \arabic*.]

            \item Inklusion (\enquote{$\subseteq$}):

            \item Inklusion (\enquote{$\supseteq$}):
            
            Sei $y \in \tilde M^2$, also dessen Komponenten $y_1, y_2 \in \tilde M$.
            Weil ja $\overline{\iota(M)} = \tilde M$, finden wir $2$ Folgen aus $\iota(M)$, die gegen $y_1$ bzw. $y_2$ konvergieren, d.h. $\Exists (y_{1, n})_{n \in \N}, (y_{2, n})_{n \in \N} \in \iota(M):$

            \begin{align*}
                y_{1, n} \xrightarrow[n \to \infty]{\tilde \metric} y_1,
                \quad
                y_{2, n} \xrightarrow[n \to \infty]{\tilde \metric} y_2.
            \end{align*}

            Daraus können wir den Kandidaten $(y_{1, n}, y_{2, n})_{n \in \N}$ basteln.
            Diese Folge muss nun aus $\iota^2(M^2)$ sein und gegen $y$ konvergieren.
            
            Laut Definition des Bildes $\iota(M)$, $\Exists (x_{1, n})_{n \in \N}, (x_{2, n})_{n \in \N} \in M^\N: \Forall n \in \N:$

            \begin{align*}
                & \iota(x_{1, n}) = y_{1, n},
                \quad
                \iota(x_{2, n}) = y_{2, n} \\
                \implies
                & (x_{1, n}, x_{2, n}) \in M^2 \\
                \implies
                & \iota^2(M^2)
                \ni
                \iota^2((x_{1, n}, x_{2, n}))
                =
                (
                    \iota(x_{1, n}),
                    \iota(x_{2, n})
                )
                =
                (y_{1, n}, y_{2, n})
                \xrightarrow[n \to \infty]{\tilde \metric_{\max}}
                (y_1, y_2)
                =
                y.
            \end{align*}

        \end{enumerate}

    \end{enumerate}

\end{enumerate}

\end{solution}

% --------------------------------------------------------------------------------
