% -------------------------------------------------------------------------------- %

\begin{exercise}

Berechnen SIe die Faltung $f \ast g$ der Funktion

\begin{align*}
    f(x)
    & =
    \begin{cases}
        e^{-\alpha x} & x > 0 \\
        0             & \text{sonst}
    \end{cases} \\
    g(x)
    & =
    \begin{cases}
        e^{-\beta x} & x > 0 \\
        0            & \text{sonst}
    \end{cases}
\end{align*}

für $\alpha, \beta > 0$.

\end{exercise}

% -------------------------------------------------------------------------------- %

\begin{solution}

\phantom{}

\begin{align*}
    (f \ast g)(x)
    =
    \Int[\R]{f(x - y) g(y)}{\lambda(y)}
    =
    \cdots
\end{align*}

\begin{align*}
    f(x - y) \neq 0
    & \iff
    x - y > 0
    \iff
    x > y \\
    g(y) \neq 0
    & \iff
    y > 0 \\
    & \implies
    \Bbraces{y \in \R: f(x - y) g(y) \neq 0}
    =
    \Bbraces{y \in \R: x > y > 0}
    =
    (0, x)
\end{align*}

\begin{multline*}
    \implies
    \cdots
    =
    \1_{\R^+}(x)
    \Int[0][x]
    {
        e^{-\alpha (x - y)}
        e^{-\beta y}
    }{y}
    =
    \1_{\R^+}(x)
    e^{-\alpha x}
    \Int[0][x]
    {
        e^{(\alpha - \beta) y}
    }{y} \\
    =
    \1_{\R^+}(x)
    e^{-\alpha x}
    \begin{cases}
        x,
        & \alpha = \beta, \\
        \frac
        {
            e^{(\alpha - \beta) x - 1}
        }{
            \alpha - \beta
        },
        & \text{sonst}
    \end{cases}
\end{multline*}

\end{solution}

% -------------------------------------------------------------------------------- %
