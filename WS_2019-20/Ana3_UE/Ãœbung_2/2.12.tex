% -------------------------------------------------------------------------------- %

\begin{exercise}

\phantom{}

\begin{align*}
    \Int[0][1]
    {
        \frac{1 - t}{1 - a t^3}
    }{t}
    =
    \sum_{n=0}^\infty
    \frac
    {
        a^n
    }{
        (3 n + 1)
        (3 n + 2)
    }
\end{align*}

für $|a| < 1$ und zeigen Sie damit

\begin{align*}
    \frac{\pi}{3 \sqrt 3}
    =
    \sum_{n=0}^\infty
    \frac{1}
    {
        (3 n + 1)
        (3 n + 2)
    }
\end{align*}

\end{exercise}

% -------------------------------------------------------------------------------- %

\begin{solution}

\phantom{}

\begin{enumerate}[label = \arabic*.]

    \item Teil:

    Um die spätere Vertauschung von Integration mit Summation zu rechtfertigen definieren wir zuerst folgende Funktionen.

    \begin{gather*}
        f_{0, N}(t)
        :=
        \sum_{n=0}^N
        (a t^3)^n,
        \quad
        f_{1, N}(t)
        :=
        t
        \sum_{n=0}^N
        (a t^3)^n,
        \quad
        N \in \N,
        \quad
        t \in (0, 1), \\
        f_0
        :=
        \lim_{N \to \infty}
        f_{0, N},
        \quad
        f_1
        :=
        \lim_{N \to \infty}
        f_{1, N}
    \end{gather*}

    Wir wollen eine, von $N$ unabhängige, Majorange für $(f_{0, N})_{N \in \N}$ bzw. $(f_{1, N})_{N \in \N}$ finden.
    $\Forall N \in \N: \Forall t \in (0, 1):$

    \begin{multline*}
        |f_{1, N}(t)|
        =
        \abs
        {
            t
            \sum_{n=0}^N
            (a t^3)^n
        }
        =
        t
        \abs
        {
            \sum_{n=0}^N
            (a t^3)^n
        }
        \leq
        |f_{0, N}(t)|
        =
        \abs
        {
            \sum_{n=0}^N
            (a t^3)^n
        } \\
        \leq
        \sum_{n=0}^N
        |a|^n |t^3|^n
        \leq
        \sum_{n=0}^N
        |a|^n
        \leq
        \sum_{n=0}^\infty
        |a|^n
        \stackrel
        {
            \mathrm{GR}
        }{=}
        \frac{1}{1 - |a|}
        =:
        g
    \end{multline*}

    $g$ ist als Konstante auf dem Kompaktum $[0, 1]$ eine integrierbare Majorante.

    \begin{align*}
        \implies
        \mathrm{lhs}
        & =
        \Int[0][1]
        {
            \frac{1}{1 - a t^3}
        }{t}
        -
        \Int[0][1]
        {
            t
            \frac{1}{1 - a t^3}
        }{t} \\
        & \stackrel
        {
            \mathrm{GR}
        }{=}
        \Int[0][1]
        {
            \sum_{n=0}^\infty
            (a t^3)^n
        }{t}
        -
        \Int[0][1]
        {
            t
            \sum_{n=0}^\infty
            (a t^3)^n
        }{t} \\
        & =
        \Int[0][1]
        {
            f_0(t)
        }{t}
        -
        \Int[0][1]
        {
            f_1(t)
        }{t} \\
        & \stackrel
        {
            \mathrm{DK}
        }{=}
        \lim_{N \to \infty}
        \Int[0][1]
        {
            f_{0, N}(t)
        }{t}
        -
        \lim_{N \to \infty}
        \Int[0][1]
        {
            f_{1, N}(t)
        }{t} \\
        & =
        \lim_{N \to \infty}
        \Int[0][1]
        {
            \sum_{n=0}^N
            (a t^3)^n
        }{t}
        -
        \lim_{N \to \infty}
        \Int[0][1]
        {
            t
            \sum_{n=0}^N
            (a t^3)^n
        }{t} \\
        & =
        \lim_{N \to \infty}
        \sum_{n=0}^N
        a^n
        \Int[0][1]
        {
            t^{3 n}
        }{t}
        -
        \lim_{N \to \infty}
        \sum_{n=0}^N
        a^n
        \Int[0][1]
        {
            t^{3 n + 1}
        }{t} \\
        & =
        \sum_{n=0}^\infty
        a^n
        \pbraces
        {
            \Int[0][1]
            {
                t^{3 n}
            }{t}
            -
            \Int[0][1]
            {
                t^{3 n + 1}
            }{t}
        } \\
        & =
        \sum_{n=0}^\infty
        a^n
        \pbraces
        {
            \frac{1}{3 n + 1}
            -
            \frac{1}{3 n + 2}
        } \\
        & =
        \sum_{n=0}^\infty
        a^n
        \frac
        {
            (3 n + 2)
            -
            (3 n + 1)
        }{
            (3 n + 1)
            (3 n + 2)
        } \\
        & =
        \mathrm{rhs}
    \end{align*}

    \item Teil:
    
    Wir können nicht direkt für $a := 1$ einsetzen.
    Stattdessen betrachten wir den Grenzwert $a \to 1$.
    Um diesen in die $\text{lhs}$ hineinzuziehen, müssen wir zeigen, dass diese in $a$ stetig ist.

    \begin{align*}
        g_n(a)
        :=
        \frac
        {
            a^n
        }{
            (3 n + 1)
            (3 n + 2)
        },
        \quad
        |a| < 1
    \end{align*}

    \begin{align*}
        \implies
        \sum_{n=0}^\infty
        \norm[\infty]{g_n}
        =
        \sum_{n=0}^\infty
        \frac{1}
        {
            (3 n + 1)
            (3 n + 2)
        }
        \sum_{n=0}^\infty
        \frac{1}
        {
            9 n^2 + 7 n + 3
        }
        \leq
        \frac{1}{9}
        \sum_{n=0}^\infty
        \frac{1}{n^2}
        <
        \infty
    \end{align*}

    \includegraphicsboxed{Ana1&2/Ana1&2 - 6.8.4 Korollar (Weierstraß-Kriterium).png}
    \includegraphicsboxed{Ana1&2/Ana1&2 - 6.6.14 Korollar.png}

    Laut 6.8.4 Korollar (Weierstraß-Kriterium) und (anschließend) 6.6.14 Korollar, ist $\text{rhs}$ stetig in $a$.

    \includegraphicsboxed{Ana1&2/Ana1&2 - 8.7.9 Korollar.png}

    $\text{lhs}$ ist ein Parameterintegral in $a$.
    Laut Korollar 8.7.9, dürfen wir also $\lim_{a \to 1}$ dort hineinziehen.

    \begin{multline*}
        \sum_{n=0}^\infty
        \frac{1}
        {
            (3 n + 1)
            (3 n + 2)
        }
        =
        \lim_{a \to 1}
        \sum_{n=0}^\infty
        \frac
        {
            a^n
        }{
            (3 n + 1)
            (3 n + 2)
        }
        \stackrel
        {
            \mathrm{1. Teil}
        }{=} \\
        \lim_{a \to 1}
        \underbrace
        {
            \Int[0][1]
            {
                \frac{1 - t}{1 - a t^3}
            }{t}    
        }_\text{
            Parameterintegral
        }
        =
        \Int[0][1]
        {
            \frac{1 - t}{1 - t^3}
        }{t}
        \stackrel
        {
            \mathrm{Integralrechner}
        }{=}
        \mathrm{rhs}
    \end{multline*}

\end{enumerate}

\end{solution}

% -------------------------------------------------------------------------------- %
