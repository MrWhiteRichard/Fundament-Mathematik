% --------------------------------------------------------------------------------

\begin{exercise}

\phantom{}

\begin{align*}
    \Int[0][1]
    {
        \frac{x^{p-1}}{1 + x^q}
    }{x}
    =
    \sum_{n=0}^\infty
    \frac{(-1)^n}{p + n q},
    \quad
    p, q > 0
\end{align*}

und zeigen Sie damit

\begin{align*}
    \frac{\pi}{4}
    =
    \sum_{n=0}^\infty
    \frac{(-1)^n}{2 n + 1}
\end{align*}

\end{exercise}

% --------------------------------------------------------------------------------

\begin{solution}

\phantom{}

\begin{enumerate}[label = \arabic*.]

    \item Teil:

    Um die spätere Vertauschung von Integration mit Summation zu rechtfertigen definieren wir zuerst folgende Funktionen.

    \begin{align*}
        f_N(x)
        :=
        x^{p-1}
        \sum_{n=0}^N
        (-x^q)^n,
        \quad
        N \in \N,
        \quad
        x \in (0, 1),
        \quad
        f
        :=
        \lim_{N \to \infty}
        f_N
    \end{align*}

    Wir finden eine Majorante für

    \begin{multline*}
        |f_N(x)|
        =
        \abs
        {
            x^{p-1}
            \sum_{n=0}^N
            (-x^q)^n    
        }
        \stackrel
        {
            \mathrm{GS}
        }{=}
        x^{p-1}
        \abs
        {
            \frac
            {
                1 - (-x^q)^{N+1}
            }{
                1 - (-x^q)
            }    
        } \\
        =
        x^{p-1}
        \abs
        {
            \frac
            {
                1 + (-1)^n (x^q)^{N+1}
            }{
                1 - (-x^q)
            }
        }
        =
        x^{p-1}
        \underbrace
        {
            \frac
            {
                1 + (x^q)^{N+1}
            }{
                1 + x^q
            }    
        }_{
            \in (0, 1)
        }
        \leq
        x^{p-1}
        =:
        g(x).
    \end{multline*}
    
    Dabei haben wir die Formel für die geometrische Summe verwendet.
    Man beachte auch, dass

    \begin{align*}
        \frac
        {
            1 + (x^q)^{N+1}
        }{
            1 + x^q
        }    
        \leq
        1
        \iff
        1 + (x^q)^{N+1}
        \leq
        1 + x^q
        \iff
        (x^q)^{N+1}
        \leq
        x^q
        \iff
        (x^q)^N
        \leq
        1.
    \end{align*}

    Die Majorante $g$ ist auch integrierbar.

    \begin{align*}
        \Int[0][1]{g(x)}{x}
        =
        \Int[0][1]{x^{p-1}}{x}
        =
        \frac{1}{p}
        x^p \Big |_{x=0}^n
        =
        \frac{1}{p}
        <
        \infty
    \end{align*}

    Mit Satz 5.7 (Satz von der dominierten Konvergenz) folgt nun die Behauptung.

    \begin{align*}
        \implies
        \mathrm{lhs}
        & =
        \Int[0][1]
        {
            \sum_{n=0}^\infty
            \frac{x^{p-1}}{1 + x^q}
        }{x} \\
        & \stackrel
        {
            \mathrm{GR}
        }{=}
        \Int[0][1]
        {
            x^{p-1}
            \sum_{n=0}^\infty
            (-x^q)^n
        }{x} \\
        & =
        \Int[0][1]
        {
            f(x)
        }{x} \\
        & \stackrel
        {
            \mathrm{DK}
        }{=}
        \lim_{N \to \infty}
        \Int[0][1]
        {
            f_N(x)
        }{x} \\
        & =
        \lim_{N \to \infty}
        \Int[0][1]
        {
            x^{p-1}
            \sum_{n=0}^N
            (-x^q)^n
        }{x} \\
        & =
        \lim_{N \to \infty}
        \sum_{n=0}^N
        (-1)^n
        \Int[0][1]
        {
            x^{p-1}
            x^{n q}
        }{x} \\
        & =
        \sum_{n=0}^\infty
        (-1)^n
        \Int[0][1]
        {
            x^{p + n q - 1}
        }{x} \\
        & =
        \sum_{n=0}^\infty
        (-1)^n
        \frac{1}{p + n q}
        x^{p + n q} \Big |_{x=0}^1 \\
        & =
        \mathrm{rhs}
    \end{align*}

    Dabei dürfen wir die Formel für die geometrische Reihe verwenden, weil

    \begin{align*}
        0 < x < 1
        \implies
        0 < -x^q < 1.
    \end{align*}

    \item Teil:
    
    Wir setzen $p := 1 > 0$ und $q := 2$.

    \begin{align*}
        \implies
        \mathrm{rhs}
        =
        \Int[0][1]
        {
            \frac{1}{1 + x^2}
        }{x}
        =
        \arctan x \Big |_{x=0}^1
        =
        \frac{\pi}{4}
    \end{align*}

\end{enumerate}

\end{solution}

% --------------------------------------------------------------------------------
