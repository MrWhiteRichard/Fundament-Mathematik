% --------------------------------------------------------------------------------

\begin{exercise}

\phantom{}

\begin{align*}
    \Int[0][\infty]{e^{-x} \cos \sqrt x}{x}
    =
    \sum_{n=0}^\infty
    (-1)^n
    \frac{n!}{(2 n)!}
\end{align*}

\end{exercise}

% --------------------------------------------------------------------------------

\begin{solution}

Wir verwenden die Potenzreihenentwicklung vom $\cos$ (vgl. Lemma 6.9.2).

\includegraphicsboxed{Ana1&2/Ana1&2 - 6.9.2 Lemma.png}

Um die Vertauschung von Integration mit Summation zu rechtfertigen definieren wir zuerst folgende Funktionen.

\begin{gather*}
    f_N(x)
    :=
    \sum_{n=0}^N
    \frac{(-x)^n}{(2 n)!}
    e^{-x},
    \quad
    g_N(x)
    :=
    f_N(-x)
    =
    \sum_{n=0}^N
    \frac{x^n}{(2 n)!}
    e^{-x},
    \quad
    x \in (0, \infty),
    \quad
    N \in \N, \\
    f := \lim_{N \to \infty} f_N,
    \quad
    g := \lim_{N \to \infty} g_N
\end{gather*}

\includegraphicsboxed{MassWHT1&2/MassWHT1&2 - Satz 5.3.png}

Offenbar ist $(g_N)_{N \in \N}$ eine monoton steigende Folge nicht-negativer Funktionen.
Mit Satz 5.3 (monotone Konvergenz, Beppo Levi-Theorem) können wir also den folgenden $\lim$ mit dem $\int$ vertauschen.

\begin{multline*}
    \implies
    \Int[0][\infty]{g}{x}
    =
    \Int[0][\infty]
    {
        \lim_{N \to \infty}
        g_N
    }{x}
    \stackrel
    {
        \mathrm{MK}
    }{=}
    \lim_{N \to \infty}
    \Int[0][\infty]
    {
        g_N
    }{x} \\
    =
    \sum_{n=0}^\infty
    \frac{1}{(2 n)!}
    \Int[0][\infty]{x^n e^{-x}}{x}
    =
    \sum_{n=0}^\infty
    \frac{1}{(2 n)!}
    \Gamma(n+1)
    =
    \sum_{n=0}^\infty
    \frac{n!}{(2 n)!}
\end{multline*}

\includegraphicsboxed{Ana1&2/Ana1&2 - 3.10.3 Satz (Quotientenkriterium).png}

Laut Satz 3.10.3 (Quotientenkriterium) konvergiert die letztere Reihe.

\begin{align*}
    \frac
    {
        \abs
        {
            \frac{(n + 1)!}{(2 (n + 1))!}
        }
    }{
        \abs
        {
            \frac{n!}{(2 n)!}
        }
    }
    =
    \frac{(n + 1)!}{n!}
    \frac{(2 n)!}{(2 n + 2)!}
    =
    \frac{n + 1}{(2 n + 2) (2 n + 1)}
    \xrightarrow{n \to \infty}
    0
\end{align*}

\includegraphicsboxed{MassWHT1&2/MassWHT1&2 - Satz 5.7.png}

Laut der Dreiecksungleichung, bildet $g$ eine gleichmäßige Majorante von $(f_N)_{N \in \N}$.
Mit Satz 5.7 (Satz von der dominierten Konvergenz) können wir also den folgenden $\lim$ mit dem $\int$ vertauschen.

\begin{multline*}
    \implies
    \mathrm{lhs}
    =
    \Int[0][\infty]
    {
        f(x)
    }{x}
    =
    \Int[0][\infty]
    {
        \lim_{N \to \infty}
        f_N(x)
    }{x}
    \stackrel
    {
        \mathrm{DK}
    }{=}
    \lim_{N \to \infty}
    \Int[0][\infty]
    {
        f_N(x)
    }{x} \\
    =
    \lim_{N \to \infty}
    \sum_{n=0}^N
    \frac{(-1)^n}{(2 n)!}
    \Int[0][\infty]
    {
        x^n
        e^{-x}
    }{x}
    =
    \sum_{n=0}^\infty
    \frac{(-1)^n}{(2 n)!}
    \Gamma(n+1)
    =
    \mathrm{rhs}
\end{multline*}

\end{solution}

% --------------------------------------------------------------------------------
