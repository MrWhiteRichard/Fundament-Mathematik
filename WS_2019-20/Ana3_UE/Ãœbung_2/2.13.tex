% -------------------------------------------------------------------------------- %

\begin{exercise}

Berechnen Sie

\begin{align*}
    \Int[0][\infty]
    {
        \frac
        {
            \arctan(x)
        }{
            (1 + x^2) x
        }
    }{x}
\end{align*}

indem Sie für $\lambda > -1$ die Funktion

\begin{align*}
    F(\lambda)
    :=
    \Int[0][\infty]
    {
        \frac
        {
            \arctan(\lambda x)
        }{
            (1 + x^2) x
        }
    }{x}
\end{align*}

differenzieren.

\end{exercise}

% -------------------------------------------------------------------------------- %

\begin{solution}

\phantom{}

\begin{enumerate}[label = \arabic*.]

    \item Versuch (naiv):
    
    Wir betrachten die Folgende Familie von Funktionen.

    \begin{align*}
        f_\lambda(x)
        :=
        \frac
        {
            \arctan(\lambda x)
        }{
            (1 + x^2) x
        },
        \quad
        x \in (0, 1),
        \quad
        \lambda > -1
    \end{align*}

    Dafür finden wir eine, von $\lambda$, unabhängige Majorante.

    \begin{align*}
        \abs
        {
            \pderivative[][f_\lambda(x)]{\lambda}
        }
        =
        \abs
        {
            \pderivative{\lambda}
            \frac
            {
                \arctan(\lambda x)
            }{
                (1 + x^2) x
            }    
        }
        =
        \abs
        {
            \frac{x}
            {
                1 + (\lambda x)^2
            }    
            \frac{1}
            {
                (1 + x^2) x
            }    
        }
        =
        \underbrace
        {
            \frac{1}
            {
                1 + (\lambda x)^2
            }        
        }_{
            \leq 1
        }
        \frac{1}
        {
            1 + x^2
        }    
        \leq
        \frac{1}
        {
            1 + x^2
        }
        =:
        g(x)
    \end{align*}

    Die Majorante $g$ ist auf $(0, \infty)$ auch integrierbar.

    \begin{align*}
        \Int[0][\infty]{g(x)}{x}
        =
        \Int[0][\infty]
        {
            \frac{1}
            {
                1 + x^2
            }    
        }{x}
        =
        \arctan x \Big |_{x=0}^\infty
        =
        \frac{\pi}{2}
        <
        \infty
    \end{align*}

    Wir können daher die folgende Vertauschung des Differenzialoperators mit dem Integral durchführen.

    \begin{align*}
        \derivative{\lambda}
        F(\lambda)
        =
        \derivative{\lambda}
        \Int[0][\infty]{f_\lambda(x)}{x}
        =
        \Int[0][\infty]
        {
            \pderivative{\lambda}
            f_\lambda(x)
        }{x}
        =
        \Int[0][\infty]
        {
            \frac{1}
            {
                1 + (\lambda x)^2
            }    
            \frac{1}
            {
                1 + x^2
            }    
        }{x}
        \stackrel
        {
            \mathrm{Integralrechner}
        }{=}
        \frac{\pi}{2 (\lambda + 1)}
    \end{align*}

    Wir benutzen nun den Hauptsatz der Differezial- und Integral-Rechnung.

    \begin{align*}
        \Int[0][\infty]
        {
            \frac
            {
                \arctan(x)
            }{
                (1 + x^2) x
            }
        }{x}
        =
        F(1) - \underbrace{F(0)}_0
        \stackrel
        {
            \mathrm{HS}
        }{=}
        \Int[0][1]
        {
            \derivative{\lambda}
            F(\lambda)
        }{\lambda}
        =
        \frac{\pi}{2}
        \Int[0][1]
        {
            \frac{1}{\lambda + 1}
        }{\lambda}
        =
        \frac{\pi}{2}
        \ln(\lambda + 1) \Big |_{\lambda = 0}^1
        =
        \frac{\pi}{2}
        \ln 2
    \end{align*}

    \item Versuch (rigoros):
    
    Wir wollen nun noch zeigen, dass $F$ wohldefiniert, d.h. endlich, ist.
    Weil der $\arctan$ eine ungerade Funktion ist, d.h. das $-$ rutscht durch, betrachten wir o.B.d.A. bloß $\lambda \geq 0$.
    Wir setzen den Integranden $f_\lambda$ von $F$ stetig auf $0$ fort.

    \begin{align*}
        f_\lambda(0)
        :=
        \lim_{x \to 0}
        f_\lambda(x)
        \stackrel
        {
            \mathrm{L'Hospital}
        }{=}
        \lim_{x \to 0}
        \frac
        {
            \frac{1}{1 + x^2}
            \lambda
        }{
            1 + 3 x^2
        }
        =
        \lambda
    \end{align*}

    Wir wollen nun zeigen, dass $f_\lambda(x)$ monoton fallend in $x$ ist.

    \begin{align*}
        & \iff
        \pderivative{x}
        f_\lambda(x)
        \stackrel
        {
            \mathrm{Ableitungsrechner}
        }{=}
        \frac{\lambda}
        {
            (\lambda^2 x^2 + 1)
            (x^3 + x)
        }
        -
        \frac
        {
            (3 x^2 + 1)
            \arctan(\lambda x)
        }{
            (x^3 + x)^2
        }
        \leq 0 \\
        & \iff
        \frac{\lambda}
        {
            (\lambda^2 x^2 + 1)
            (x^3 + x)
        }
        \leq
        \frac
        {
            (3 x^2 + 1)
            \arctan(\lambda x)
        }{
            (x^3 + x)^2
        } \\
        & \impliedby
        \lambda (x^3 + x)
        \stackrel{!}{\leq}
        (\lambda^2 x^2 + 1) (3 x^2 + 1)
        \stackrel{!}{\leq}
        (\lambda^2 x^2 + 1) (3 x^2 + 1) \arctan(\lambda x) \\
    \end{align*}

    Die linke Seite ist ein Polynom $3$-ten und die Mitte ein Polynom $4$-ten Grades in $x$.
    Daher glit die $1$-te Ungleichung ab einem gewissen $x_{\lambda, 1} \in \N$.
    Weil $\lambda > 0$, ist $\arctan(\lambda x)$ in $x$ monoton steigend gegen $\frac{\pi}{2}$.

    \begin{align*}
        \implies
        \Exists x_{\lambda, 1} \in \N:
        \Forall x \geq x_{\lambda, 1}:
        \arctan(\lambda x) \geq 1
    \end{align*}

    Die $1$-te und $2$-te Ungleichung gelten also beide ab $x_\lambda := \max(x_{\lambda, 1}, x_{\lambda, 2})$.
    $f_\lambda$ ist also in $[x_\lambda, \infty)$ monoton fallend.

    \begin{align*}
        F(\lambda)
        :=
        \Int[0][\infty]
        {
            f_\lambda(x)
        }{x}
        =
        \Int[0][x_\lambda]
        {
            f_\lambda(x)
        }{x}
        +
        \Int[x_\lambda][\infty]
        {
            f_\lambda(x)
        }{x}
        \stackrel{!}{<}
        \infty
    \end{align*}

    Das $1$-te $\int$ ist, als $\int$ einer stetigen Funktion auf dem Kompaktum $[0, x_\lambda]$, endlich;
    das $2$-te untersuchen wir jetzt-dann.

    Zuerst beweisen wir noch, dass

    \begin{align*}
        \Forall x \in [0, \infty): \arctan x \leq x.
    \end{align*}

    Für $x = 0$ gilt sogar Gleichheit.
    Die Ungleichung geht für $x > 0$ nicht verlohren, weil wenn sie für $x = 0$ stimmt, dann ist

    \begin{align*}
        x - \arctan x \geq 0
        & \iff
        x - \arctan x ~\text{monoton steigend}~ \\
        & \iff
        \derivative{x}
        (x - \arctan x)
        =
        1 - \frac{1}{1 + x^2}
        \geq
        0 \\
        & \iff
        1 + x^2 \geq 1.
    \end{align*}

    Wir kommen nun aber endlich zu unseren finalen Abschätzung.

    \begin{align*}
        \Int[x_\lambda][\infty]
        {
            \frac
            {
                \arctan(\lambda x)
            }{
                (1 + x^2) x
            }
        }{x}
        \leq
        \sum_{n = x_\lambda}^\infty
        f_\lambda(n)
        =
        \sum_{n = x_\lambda}^\infty
        \frac
        {
            \arctan(\lambda n)
        }{
            (1 + n^2) n
        }
        \leq
        \sum_{n = x_\lambda}^\infty
        \frac{n}
        {
            (1 + n^2) n
        }
        \leq
        2
        \sum_{n = x_\lambda}^\infty
        \frac{1}{n^2}
        <
        \infty
    \end{align*}

\end{enumerate}

\end{solution}

% -------------------------------------------------------------------------------- %
