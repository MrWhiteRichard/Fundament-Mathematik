\begin{exercise}
    Sei $f$ eine ganze nichtkonstante Funktion. Man beweise, dass die Bildmenge $f(\C)$ dicht in $\C$ liegt.
\end{exercise}
\begin{solution}
    Wir führen einen Widerspruchsbeweis, nehmen also an $f(\C)$ wäre nicht dicht in $\C$. Dann gibt es ein $w \in \C$ und $M \in \R^+$ so, dass für alle $z \in \C: \vbraces{f(z) - w} \geq M$. Dann ist $\frac{1}{f(z)- w}$ also eine ganze Funktion und für alle $z \in \C$ gilt, dass $\vbraces{\frac{1}{f(z) - w}}\leq \frac{1}{M}$ und damit gibt es nach dem Satz von Liouville ein $c \in \C \setminus \{0\}$ so, dass für alle $z \in \C$ die Gleichheit $\frac{1}{f(z) - w} = c$, also auch $f(z) = \frac{1}{c} + w$ gilt. Das ist ein Widerspruch dazu, dass $f$ nicht konstant ist.
\end{solution}