\section{Leicht aber unverzichtbar}

\begin{itemize}

  \item
  [1.1.6]
  Leicht aber unverzichtbar

  \item
  [1.3.1]
  Lineare (Un-)Abhängigkeit

  \item
  [1.3.2]
  Das Austauschlemma und seine Konsequenzen

  \item
  [2.1.1]
  Notation und Terminologie

  \item
  [2.1.2]
  Grundbegriffe der Ordnungstheorie

  \item
  [2.1.6]
  Strukturverträgliche Abbildungen zwischen relationalen Strukturen

  \item
  [2.2.1]
  Kategorien

  \item
  [2.2.2]
  Beispiele von Kategorien

  \item
  [2.3.5]
  Triviale und nichttriviale Varietäten

  \item
  [3.1.1]
  Potenzen und Inverse

  \item
  [3.3.3]
  Charakteristik

  \item
  [3.3.4]
  Die binomische Forme

  \item
  [3.5.1]
  Grundlegende Definitionen

  \item
  [3.4.2]
  Geordnete Gruppen

  \item
  [3.6.1]
  Elementare Eigenschaften

  \item
  [3.6.2]
  Unterverbände

  \item
  [3.6.3]
  Kongruenzrelationen; Filter und Ideale

  \item
  [3.6.4]
  Vollständige Verbände

  \item
  [3.6.5]
  Distributive und modulare Verbände

  \item
  [3.6.6]
  Boolesche Ringe

  \item
  [3.6.7]
  Einfache Rechenregeln

  \item
  [3.6.8]
  Atome

  \item
  [5.1.2]
  Teilbarkeit als Quasiordnung auf kommutativen Monoiden

  \item
  [5.1.3]
  Teilbarkeit in Integritätsbereichen

  \item
  [6.1.1]
  Primkörper

  \item
  [6.1.2]
  Das Vektorraumargument

  \item
  [6.2.4]
  Mehrfache Nullstellen und formale Ableitung

\end{itemize}
