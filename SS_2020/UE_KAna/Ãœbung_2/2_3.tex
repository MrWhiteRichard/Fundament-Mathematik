\begin{exercise}
    Hier könnte Ihre Werbung stehen!
    \begin{enumerate}[label = (\roman*)]
        \item Berechnen Sie $\int_\gamma z \cos\pbraces{z^2} \rm{d}z$, wenn
        \begin{enumerate}[label = (\arabic*)]
            \item $\gamma$ die Verbindungsstrecke von $\rm{i}$ und $-\rm{i} + 2$ ist.
            \item $\gamma$ die Punkte $0$ und $1 + \rm{i}$ entlang der Kurve $y = x^2$ verbidnet.
        \end{enumerate}
        \item Berechnen Sie $\int_\gamma \overline{z} \rm{d}z$ längs der Steckenzüge
        \begin{enumerate}[label = (\alph*)]
            \item $0 \to 1 \to 1 + \rm{i}$
            \item $0 \to \rm{i} \to 1 + \rm{i}$
            \item $0 \to 1 + \rm{i}$
        \end{enumerate}
    \end{enumerate}
\end{exercise}

\begin{solution}
    Hier könnte Ihre Werbung stehen!
    \begin{enumerate}[label = (\roman*)]
        \item Schnell erkennt man, dass $\frac{\sin\pbraces{z^2}}{2}$ eine Stammfunktion des Integranden ist, deshalb gilt in
        \begin{enumerate}[label = (\arabic*)]
            \item für das Integral
            \begin{align*}
                \int_\gamma z \cos\pbraces{z^2} \rm{d}z = \frac{1}{2}\pbraces{\sin\pbraces{(2 - \rm{i})^2} - \sin\pbraces{i^2}}
            \end{align*}
            und in 
            \item gilt
            \begin{align*}
                \int_\gamma z \cos\pbraces{z^2} \rm{d}z = \frac{\sin\pbraces{(1 + i)^2}}{2}.
            \end{align*}
        \end{enumerate}
        \item Hier gilt
        \begin{enumerate}[label = (\alph*)]
            \item 
            \begin{align*}
                \int_\gamma \overline{z} \rm{d}z = \int_0^1 t \rm{d}t +  \int_0^1 \rm{i}(1 - \rm{i}t) \rm{d}t = 2 + \rm{i}
            \end{align*}
            \item 
            \begin{align*}
                \int_\gamma \overline{z} \rm{d}z = \int_0^1 t \rm{d}t + \int_0^1 (t - \rm{i}) \rm{d}t = 2 - \rm{i}
            \end{align*}
            \item 
            \begin{align*}
                \int_\gamma \overline{z} \rm{d}z = \int_0^1 (1 - \rm{i})t(1 + \rm{i}) \rm{d}t = 2
            \end{align*}
        \end{enumerate}
    \end{enumerate}
    
\end{solution}