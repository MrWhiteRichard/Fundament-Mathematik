Ich kann mir gut vorstellen, dass Sie lieber eine handschriftlich verfasste Lösung hätten und kann das auch gut nachvollziehen, da sie an der Schrift einer Person wohl besser festestellen können ob der oder die Studierende selbst Verfasser der Abgabe ist. Nun habe ich aber diese Übung bereits am Computer geschrieben und mich entschlossen das Risiko einzugehen und die Lösung so abzugeben in der Hoffnung, dass Sie mich, sollten Sie nicht zufrieden damit sein, lediglich ermahnen und dieses eine Mal darüber hinwegsehen. In diesem Fall werde ich ab dem nächsten Mal selbstverständlich wieder eine handschriftlich verfasste Lösung abgeben. Wenn Ihnen eine Abgabe in der jetzigen Form recht ist, dann werde ich mir überlegen auch in Zukunft meine Lösung in dieser Form abzugeben. Ich denke so ist die Lesbarkeit höher als bei einer handschriftlichen Abgabe und Sie tun sich leichter beim Verbessern. 
\begin{exercise}
    Berechnen Sie den Konvergenzradius $R$ der folgenden Potenzreihen. 
    \begin{enumerate}[label = \alph*)]
        \item 
        \begin{align*}
            \sum_{n = 1}^\infty \pbraces{\frac{\log(n)}{n}}^n \rm{i}^n z^n
        \end{align*}
        \item 
        \begin{align*}
            \sum_{n = 1}^\infty \pbraces{\frac{n!}{\log\pbraces{n^n}}} z^{2n}    
        \end{align*}
        \item 
        \begin{align*}
            \sum_{n = 1}^\infty \frac{n!}{n^{n-1}}(z - 3\rm{i})^n
        \end{align*}
    \end{enumerate}
\end{exercise}

\begin{solution}
    Hier könnte Ihre Werbung stehen!
    \begin{enumerate}[label = \alph*)]
        \item Es gilt 
            \begin{align*}
                \limsup_{n \to \infty} \frac{\log(n)}{n} = \lim_{n \to \infty} \frac{\log(n)}{n} = \lim_{n \to \infty} \frac{1}{n} = 0
            \end{align*}
            und damit $R = \infty$.
        \item Wegen $n! \geq n^{\frac{n}{2}}$ gilt
        \begin{align*}
            \lim_{n \to \infty} \pbraces{\frac{n!}{\log\pbraces{n^n}}}^{\frac{1}{n}} \geq \lim_{n \to \infty} \frac{\sqrt{n}}{n^{\frac{1}{n}} \pbraces{\log(n)}^{\frac{1}{n}}} = \infty
        \end{align*}
        und damit $R = 0$.
        \item Es gilt
            \begin{align*}
                \lim_{n \to \infty} \frac{(n + 1)! n^{n - 1}}{(n + 1)^n n!} = \lim_{n \to \infty} \pbraces{\frac{n}{n + 1}}^{n - 1} = \rm{e}^{-1}
            \end{align*}
            und damit $R = \rm{e}$.
    \end{enumerate}
\end{solution}