\begin{algebraUE}{324}
Sei $R = \Z[\sqrt{-5}] := \{a + b\sqrt{-5} | a,b \in \Z\} \subseteq \C$.
\begin{enumerate}
  \item Zeigen Sie, dass die Elemente $2$ und $3$ in $R$ irreduzibel, aber nicht prim
  sind.
  \item Finden Sie eine Primzahl $p \in \Z$, die im Ring $Z[-\sqrt{5}]$ nicht irreduzibel ist.
\end{enumerate}
\textit{Hinweis:} Verwenden Sie die Normfunktion $N$ aus 5.1.3. und zeigen Sie,
dass genau jene $x \in R$ Einheiten sind, die $N(x) = 1$ erfüllen.
\end{algebraUE}
\begin{solution}
\leavevmode \\
\begin{enumerate}
  \item Wir betrachten die Normfunktion
  \begin{align*}
    N: \begin{cases}
      \Z[\sqrt{-5}] &\to \Z \\
      (a + b\sqrt{-5}) &\mapsto a^2 + 5b^2
    \end{cases}.
  \end{align*}
  Die Normfunktion ist wohldefiniert, da für $(a + b\sqrt{-5}) = (c + d\sqrt{-5})$
  folgt, dass $a = b$ und $c = d$.
  Wir zeigen nun, dass für $D < 0: R^* = \{r \in R: N(r) = 1\}$ ist.
  Gilt $N((a + b\sqrt{-5}) ) = 1$, dann folgt aufgrund
  \begin{align*}
    (a + b\sqrt{-5})(a - b\sqrt{-5}) = a^2 + 5b^2 = N(a + b\sqrt{-5}) = 1,
  \end{align*}
  dass $(a + b\sqrt{-5})$ eine Einheit von $R$ ist.

  Für die Rückrichtung sei $x = a + b\sqrt{-5} \neq 0$ eine beliebige Einheit. Das heißt, es gibt ein $x^{-1}$ mit $xx^{-1}=1$.
  Wir nutzen wieder die Multiplikativität der Normfunktion aus und wissen:
  \begin{align*}
    N(xx^{-1}) = N(x)N(x^{-1}) = 1
  \end{align*}
  Da die Normfunktion nur positive ganze Werte annehmen kann, gilt $N(x) = 1$.

  Sei nun $(a + b\sqrt{-5}) \in R$ ein Teiler von $2$, also gibt es ein $(c + d\sqrt{-5}) \in R$, sodass
  \begin{align*}
    (a + b\sqrt{-5})(c + d\sqrt{-5}) = (ac - 5bd) + (ad + bc)\sqrt{-5} = 2.
  \end{align*}
  Da $N$ ein multiplikativer Homomorphismus ist, folgt
  \begin{align*}
    N((a + b\sqrt{-5}))N(c + d\sqrt{-5}) = N(2) = 4
  \end{align*}
  und damit können die einzelnen Faktoren nur Werte in $\{-4,-2,-1,1,2,4\}$ annehmen.
  Da $a^2 + 5b^2 \geq 0$, können sogar nur die positiven Werte auftreten.
  Der einzig verbleibende Fall, in dem keine Einheit im Produkt auftritt ist also $(2,2)$,
  welcher aufgrund
  \begin{align*}
    \forall a, b \in \Z: a^2 + 5b^2 \neq 2
  \end{align*}
  nicht auftreten kann.
  Für $n = 3$ sieht man analog
  \begin{align*}
    N((a + b\sqrt{-5}))N(c + d\sqrt{-5}) = N(3) = 9.
  \end{align*}
  Diesmal ist die kritische Faktorkombination $(3,3)$, welche aber auch nicht auftreten kann. \\
  Wir haben also gezeigt, dass jeder Teiler von $2$ oder $3$ eine Einheit, oder assoziiert
  mit $2$ oder $3$ sein muss. Also sind $[2]_{\sim},[3]_{\sim}$ obere Nachbarn
  von $[1]_{\sim}$ und per Definition damit irreduzibel. \\
  Um einzusehen, dass $2$ nicht prim sein kann, betrachte
  \begin{align*}
    2 | 6 = (1 + \sqrt{-5})(1 - \sqrt{-5}).
  \end{align*}
  $2$ teilt aber sicher nicht $1 + \sqrt{-5}$, da
  \begin{align*}
    \frac{1 + \sqrt{-5}}{2} = \frac{1}{2} + \frac{1}{2}\sqrt{-5} \notin R.
  \end{align*}
  Das selbe Gegenbeispiel funktioniert auch für $3$.
  \item
  \begin{align*}
    5 = -\sqrt{-5}\sqrt{-5}.
  \end{align*}
\end{enumerate}
\end{solution}
