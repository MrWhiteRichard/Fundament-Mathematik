\begin{algebraUE}{346}
Ein Polynom $f$ über einem Körper $K$ vom Grad $2$ oder $3$ ist genau dann
irreduzibel, wenn $f$ in $K$ keine Nullstelle hat.

\end{algebraUE}

\begin{solution}
Man erinnere sich, dass die Einheiten in $K[x]$ genau die Polynome von Grad $0$ sind. \\
Habe $f$ in $K$ keine Nullstelle. Dann gilt
\begin{itemize}
  \item Fall 1: $\grad(f) = 2$: \\
  Nach der Formel $\grad(pq) = \grad(p) + \grad(q)$ muss jede Darstellung $f = pq$
  entweder zwei Polynome vom Grad 1 oder eine Einheit enthalten. Jedes Polynom vom Grad
  1 hat eine Nullstelle, also muss $f$ irreduzibel sein.
  \item Fall 2: $\grad(f) = 3$: \\
  Mit der selben Argumentation wie oben haben wir in der Darstellung $f = pq$
  nur zwei Fälle, die eintreten können. Entweder einer der Faktoren ist eine
  Einheit oder wir haben ein Produkt mit einem Polynom von Grad $1$ und einem
  Polynom vom Grad $2$. Wieder kann der zweite Fall nicht eintreten, da jedes
  Polynom vom Grad $1$ eine Nullstelle hat.
\end{itemize}
Sei $f$ irreduzibel. Angenommen $f$ hätte eine Nullstelle $x_0$.
Es folgt
\begin{align*}
  f = q(x)(x - x_0),
\end{align*}
wobei $\grad(q) = \grad(f) - 1 \geq 1$. Widerspruch!
\end{solution}
