\begin{exercise}
    Folgendes ist zu tun.
    \begin{enumerate}[label = (\roman*)]
        \item Seien $C,D$ Algebren, $\varphi:C \to D$ und $\varphi^\prime:C \to D$ Homomorphismen. Sei $B \subseteq C$ und $\langle B \rangle = C$ sowie für alle $b \in B: \varphi(b) = \varphi^\prime(b)$. Zeigen Sie, dass dann $\varphi = \varphi^\prime$ gilt.

        \item Seien $C,D$ Körper und  $\varphi:C \to D$ und $\varphi^\prime:C \to D$ Homomorphismen. Sei $B \subseteq C$ und $\langle B \rangle = C$ sowie für alle $b \in B: \varphi(b) = \varphi^\prime(b)$. Ist dann $\varphi = \varphi^\prime$ ?
    \end{enumerate}
\end{exercise}
\textbf{Version 1: ``von unten''}
\begin{solution}
    Hier könnte Ihre Werbung stehen!
    \begin{enumerate}[label = (\roman*)]
        \item Wir verwenden die rekursive Konstruktion von $\langle B \rangle$ durch $B_0 := B$ und
        \begin{align*}
            B_{k + 1} := B_k \cup \Bbraces{\omega(b_1, \dots, b_l) \mid b_1, \dots, b_l \in B_k \land \omega \in \Omega}.
        \end{align*}
        Für alle $b \in B_0$ gilt nach Voraussetzung $\varphi(b) = \varphi^\prime(b)$. Sei nun $n \in \N \setminus \Bbraces{0}$ und gelte für alle $x \in B_n: \varphi(x) = \varphi^\prime(x)$. Wählen wir nun ein beliebiges $c \in B_{n + 1}$.
        \begin{enumerate}[label = Fall \arabic*:]
            \item Sei $c \in B_n$. Dann gilt nach Voraussetzung $\varphi(c) = \varphi^\prime(c)$.
            \item Sei $c \notin B_n$. Dann gibt es $\omega \in \Omega$ und $c_1, \dots, c_l \in B_n$ mit $c = \omega(c_1, \dots, c_l)$. Es gilt also
            \begin{align*}
                \varphi(c) &= \varphi(\omega(c_1, \dots, c_l)) = \omega(\varphi(c_1), \dots, \varphi(c_l)) \\
                &= \omega(\varphi^\prime(c_1), \dots. \varphi^\prime(c_l)) = \varphi^\prime(\omega(c_1, \dots, c_l)) = \varphi^\prime(c)
            \end{align*}
        \end{enumerate}
        \item Man kann wohl für Körper eine ähnliche Charakterisierung vornehmen wie für Algebren. Wir definieren rekursiv $B_0 := B$ und
        \begin{align*}
            B_{k+1} := B_k \cup \Bbraces{\omega(b_1, \dots, b_l) \mid b_1, \dots, b_l \in B_k \land \omega \in \Omega} \cup \Bbraces{b^{-1} \mid b \in B_k \setminus \Bbraces{0}}
        \end{align*}
        Für $b \in B_0$ gilt nach Voraussetzung $\varphi(b) = \varphi^\prime(b)$. Sei nun $n \in \N \setminus \Bbraces{0}$ und gelte für alle $x \in B_n: \varphi(x) = \varphi^\prime(x)$. Wählen wir nun ein beliebiges $c \in B_{n + 1}$.
        \begin{enumerate}[label = Fall \arabic*:]
            \item Sei $c \in B_n$. Dann gilt nach Voraussetzung $\varphi(c) = \varphi^\prime(c)$.
            \item Sei $c \in \Bbraces{\omega(b_1, \dots, b_l) \mid b_1, \dots, b_l \in B_k \land \omega \in \Omega}$. Dann gibt es $\omega \in \Omega$ und $c_1, \dots, c_l \in B_n$ mit $c = \omega(c_1, \dots, c_l)$. Es gilt also
            \begin{align*}
                \varphi(c) &= \varphi(\omega(c_1, \dots, c_l)) = \omega(\varphi(c_1), \dots, \varphi(c_l)) \\
                &= \omega(\varphi^\prime(c_1), \dots. \varphi^\prime(c_l)) = \varphi^\prime(\omega(c_1, \dots, c_l)) = \varphi^\prime(c)
            \end{align*}
            \item Sei $c \in \Bbraces{b^{-1} \mid b \in B_k \setminus \Bbraces{0}}$. Dann gilt
            \begin{align*}
                \varphi(c) = \varphi(b^{-1}) = (\varphi(b))^{-1} = (\varphi^\prime(b))^{-1} = \varphi^\prime(b^{-1}) = \varphi^\prime(c)
            \end{align*}
        \end{enumerate}
    \end{enumerate}
\end{solution}
\textbf{Version 2: ``von oben''} \\
\begin{solution}
Definiere $A := \{x \in C: \varphi(x) = \varphi^{\prime}(x)\}$.
Wir zeigen, dass $A$ eine Unteralgebra von $(C,\Omega_C)$ ist.
Sei $i \in I, (x_1,\dots,x_{n_i}) \in A^{n_i}$ beliebig.
\begin{align*}
  \varphi(\omega_{i,C}(x_1,\dots,x_{n_i})) =
  \omega_{i,D}(\varphi(x_1),\dots,\varphi(x_{n_i})) =
  \omega_{i,D}(\varphi^{\prime}(x_1),\dots,\varphi^{\prime}(x_{n_i})) =
  \varphi^{\prime}(\omega_{i,C}(x_1,\dots,x_{n_i}))
\end{align*}
Damit ist $A$ abgeschlossen bezüglich $(\omega_i)_{i \in I}$ und somit eine
Unteralgebra von $(C,\Omega_C)$.
Es gilt
\begin{align*}
A \supseteq \bigcap\{T : T \supseteq B  ~\text{und}~ T
~\text{ist Unteralgebra von}~ (C,\Omega_C)\} = \langle B \rangle = C
\end{align*}
und damit $\varphi = \varphi^{\prime}$ auf $C$. \\
Für Körper lautet die modifizierte Aussage folgendermaßen: \\
Seien $C, D$ Körper, $\varphi: C \mapsto D$ und $\varphi^{\prime}: C \mapsto D$
Homomorphismen. Sei $B \subseteq C$. Wenn $C = \langle B \rangle$, also der
Durchschnitt aller Unterkörper, die $B$ enthalten, und
$\varphi(b) = \varphi^{\prime}(b)$ für alle $b \in B$, dann ist $\varphi =
\varphi^{\prime}$. \\
Dann lässt sich analog zu zuvor zeigen, dass $\varphi = \varphi^{\prime}$ auf
einer Unteralgebra $U$ von $(C,\Omega_C)$, die $B$ enthält, gelten muss. Diese Unteralgebra muss allerdings
nicht zwingend ein Unterkörper sein. Es gilt aber für
alle Körperhomomorphismen $f$
\begin{align*}
1 = f(1) = f(x*x^{-1}) = f(x)*f(x^{-1}).
\end{align*}
Es folgt also für alle $0 \neq x \in U$
\begin{align*}
  \varphi(x^{-1}) = \varphi(x)^{-1} = \varphi^{\prime}(x)^{-1} = \varphi^{\prime}(x^{-1})
\end{align*}
Damit ist $U$ auch unter der Inversenbildung abgeschlossen, somit ein Unterkörper
und damit wiederum gleich ganz $C$.
\end{solution}
