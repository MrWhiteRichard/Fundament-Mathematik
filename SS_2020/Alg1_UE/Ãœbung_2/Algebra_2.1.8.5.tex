\begin{exercise}
    Beweisen Sie folgende Aussagen, wobei die Notation von Satz 2.1.8.4 gilt. 
    \begin{enumerate}[label = \arabic*.]
        \item In jedem Term kommen  nur endlich viele Variablen vor.
        \item Der Wert $\bar{\alpha}(t)$ eines Terms $t$ für eine Variablenbelegung $\alpha: X \to A$ hängt von den $\alpha(x)$ nur für jene $x \in X$ mit $x \in v(t)$, die also in $t$ vorkommen, ab.
    \end{enumerate}
\end{exercise}

\begin{solution} 
    Hier könnte Ihre Werbung stehen!
    \begin{enumerate}[label = \arabic*.]
        \item Wir wollen die Aussage induktiv für alle $T_n$ zeigen. Zuerst betrachten wir $t \in T_0 := X$. Das heißt es gibt ein $x \in X$ mit $t = x$ und die Menge der Variablen ist $v(t) = \Bbraces{x}$, also ist die Menge endlich. Sei nun $n \in \N$ und für alle $k < n$ seien die Terme in $T_k$ alle endlich. Wählen wir nun ein beliebiges $t \in T_n$. 
        \begin{enumerate}[label = Fall \arabic*:]
            \item Es sei $t \in T_{n-1}$. Dann ist nach Voraussetzung $v(t)$ endlich.
            \item Es sei $t \notin T_{n-1}$. Das heißt es gibt ein $j \in I$ und $t_1, \dots, t_{n_j} \in T_{n-1}$ so, dass $t = \omega_j(t_1, \dots, t_{n_j})$ ist. Die Menge der Terme ist dann $v(t) = v(t_1) \cup \dots \cup v(t_{n_j})$ und da für alle $l \in \Bbraces{1, \dots, n_j}$ die Menge $v(t_l)$ nach Voraussetzung endlich ist, ist auch $v(t)$ endlich. 
        \end{enumerate}
        \item Wir zeigen die Behauptung induktiv. Sei $t \in T_0$. Das heißt es gibt ein $x \in X$ so, dass $t = x$. Daraus folgt schließlich, dass $\bar{\alpha}(t) = \alpha(x)$ und $x \in v(t) = {x}$. Sei nun $k \in \N$ und hänge für alle $l < k+1$ für alle $s \in T_l$ der Wert $\bar{\alpha}(s)$ nur von $\alpha(x)$ für jene $x \in X$ mit $x \in v(s)$ ab. Sei nun $t \in T_{k+1}$. Es können zwei Fälle auftreten.
        \begin{enumerate}[label = Fall \arabic*:]
            \item Sei $t \in T_k$. Nach Voraussetzung gilt dann unsere Behauptung.
            \item Sei $t \notin T_k$. Das heißt es gibt ein $i \in I$ und $t_1, \dots, t_{n_i}$ mit $t = \omega_i(t_1, \dots, t_{n_i})$. 
            \begin{enumerate}[label = Fall 2.\arabic*:]
                \item Sei $n_i = 0$. Dann ist $\bar{\alpha}(t) = \omega_{i, \mathfrak{A}}$, also hängt $\bar{\alpha}(t)$ von gar keinen Variablen ab und unsere Behauptung stimmt.
                \item Sei $n_i \neq 0$. Dann ist $\bar{\alpha}(t) = \omega_{i, \mathfrak{A}}(\bar{\alpha}(t_1), \dots, \bar{\alpha}(t_{n_i}))$. Für alle $x \in X$ von welchen der Wert von $\bar{\alpha}(t)$ abhängt gilt nach Voraussetzung also $\exists l \in \Bbraces{1, \dots, n_i}$ mit $x \in v(t_l)$. Damit folgt natürlich auch $x \in v(t) = v(t_1) \cup \dots \cup v(t_{n_i})$.
            \end{enumerate} 
        \end{enumerate}
    \end{enumerate}
\end{solution}