\begin{algebraUE}{150}
Zeigen Sie, dass es zu gegebenen $U \leq G$ ($G$ Gruppe) gleich viele Links-
wie auch Rechtsnebenklassen gibt. (Beachten Sie, dass Ihr Beweis auch im unendlichen
Fall gelten sollte.)
\end{algebraUE}
\begin{solution}
Sei $G$ eine beliebige Gruppe, $U$ eine beliebige Untergruppe davon.
Bezeichne mit $\mathcal{L},\mathcal{R}$ die Menge
aller Links-, beziehungsweise Rechtsnebenklassen bezüglich $U$.
Definiere die Abbildung $\varphi$ wie folgt
\begin{align*}
  \varphi: \begin{cases}
    \mathcal{L} \rightarrow \mathcal{R} \\
    gU \mapsto Ug^{-1}
  \end{cases}.
\end{align*}
Sei nun $gU = hU$. Wir wissen, dass dies äquivalent ist zu $g^{-1}h \in U$,
was wiederum äquivalent ist zu $Ug^{-1} = Uh^{-1}$, womit die Abbildung
wohldefiniert ist.
Um die Bijektivität dieser Abbildung einzusehen, betrachte man die Inverse
\begin{align*}
  \varphi^{-1}: \begin{cases}
    \mathcal{R} \rightarrow \mathcal{L} \\
    Ug \mapsto g^{-1}U
  \end{cases}.
\end{align*}
\end{solution}
