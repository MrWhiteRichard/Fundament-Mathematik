\begin{algebraUE}{196}
\leavevmode \\
\begin{enumerate}
  \item Zeigen Sie: Ist $V$ ein endlich-dimensionaler Vektorraum über einem
  Körper, so ist $\End(V)$ als Ring einfach (besitzt also nur die trivialen Ideale).
  \item Zeigen Sie, dass dies für unendlich-dimensionale Vektorräume nicht gilt. \\
  \textit{Hinweis:} Betrachten Sie alle Endomorphismen mit endlich-dimensionalem Bild.
\end{enumerate}
\end{algebraUE}
\begin{solution}
\leavevmode \\
\begin{enumerate}
  \item Sei $V$ ein endlich-dimensionaler Vektorraum über einem Körper $K$, dann ist
  \begin{align*}
    R := \End(V) = \{f: V \rightarrow V: f \text{ linear}\}
  \end{align*}
  nach Proposition 3.3.8.1. ein Ring mit Eins.
  Sei $\{0\} \neq I \vartriangleleft \End(V)$ ein Ideal, dann existieren
  $g \in I, b_1 \in V$, sodass
  \begin{align*}
    g(b_1) = c_1 \neq 0.
  \end{align*}
  Nun können wir $b_1,c_1$ zu zwei Basen $B = \{b_1,\dots,b_n\}$ und $C = \{c_1,\dots,c_n\}$ von $V$
  fortsetzen.
  Dann werden durch
  \begin{align*}
    &\alpha_k:\begin{cases} V  \rightarrow & V \\
      b_j  \mapsto & \begin{cases}
        b_1 & \text{für } j = k \\
        0 & \text{für } j \neq k.
        \end{cases}
        \end{cases} \\
    &\beta_k: \begin{cases} V \rightarrow & V \\
      c_j \mapsto & \begin{cases}
        b_k & \text{für } j = 1 \\
        0 & \text{für } j \neq 1.
        \end{cases}
        \end{cases}
  \end{align*}
  lineare Abbildungen definiert und
  \begin{align*}
    f_{k,l} := \beta_l \circ g \circ \alpha_k: \begin{cases}
      V \rightarrow & V \\
      b_j \mapsto & \begin{cases}
      b_l & \text{für } j = k\\
      0 & \text{für } j \neq k.
      \end{cases}
    \end{cases}
  \end{align*}
  wegen $\End(V)I \subset I$ und $\End(V) \subset I$, ist $f \in I$. Nun definiere
  für $c \in K$
  \begin{align*}
    f_{k,l,c} := c f_{k,l} \in I.
  \end{align*}
  Sei nun $f \in \End(V)$ beliebig. Dann lässt sich $f$ darstellen als
  \begin{align*}
    f(b_k) = \sum_{l=1}^n c_{k,l}b_l, \qquad k = 1,\dots,n.
  \end{align*}
  Es gilt also für $j \in \{1,\dots,n\}$ beliebig gilt
  \begin{align*}
    \sum_{k = 1}^n \sum_{l = 1}^n f_{k,l,c_{k,l}}(b_j) = \sum_{l=1}^n c_{j,l}f_{j,l}(b_j)
    = \sum_{l = 1}^n c_{j,l} b_l = f(b_j).
  \end{align*}
  und damit
  \begin{align*}
    f = \sum_{k=1}^n\sum_{l=1}^n f_{k,l,c_{k,l}}.
  \end{align*}
  Also ist, da $I$ insbesondere eine Untergruppe von $\End(V)$ ist, $f \in I$ und $I = \End(V)$.
  \item Sei $V$ nun unendlich-dimensional und definiere
  \begin{align*}
    I := \{f: V \rightarrow V: \dim f(V) < \infty \} \neq \End(V).
  \end{align*}
  Es gilt vorerst für $f_1,f_2 \in I$
  nach dem Dimensionssatz
  \begin{align*}
    \dim((f_1 + f_2)(V)) \leq \dim(f_1(V) + f_2(V)) =
    \dim(f_1(V)) + \dim(f_2(V)) - \dim(f_1(V) \cap f_2(V)) < \infty
  \end{align*}
  Nach der Rangformel gilt auch für $f \in I, g \in \End(V)$
  \begin{align*}
    \dim(g(f(V))) &= \dim(f(V)) - \dim(\ker (g|_{f(V)})) < \infty \\
    \dim(f(g(V))) &\leq \dim(f(V)) < \infty
  \end{align*}
  Es ist also $I \leq \End(V), \End(V)I \subset I, I\End(V) \subset$ und $I$
  ist damit ein nichttriviales Ideal.
\end{enumerate}
\end{solution}
