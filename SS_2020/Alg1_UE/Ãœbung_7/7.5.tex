\begin{algebraUE}{210}
Sei $p$ eine Primzahl.
\begin{enumerate}
  \item Wie viele Untergruppen hat $C_p \times C_p$?
  \item Wie viele Untergruppen hat $C_{p^2}$?
\end{enumerate}
\textit{Hinweis:} Überlegen Sie sich zuerst, dass alle nichttrivialen Untergruppen
zyklisch sein müssen.
\end{algebraUE}
\begin{solution}
Wir wissen, dass jede Untergruppe einer zyklischen Gruppe wieder zyklisch ist.
\begin{enumerate}
  \item Damit hat $C_p$ klarerweise nur die beiden trivialen Untergruppen.
  \begin{align*}
    C_p \times C_p = \{(a + p\Z, b + p\Z): a,b \in \Z\}
  \end{align*}
  Wir wissen nach dem Satz von Lagrange, dass für jede Untergruppe
  $|U|$ ein Teiler von $|C_p \times C_p| = p^2$ sein muss. Also gibt es neben den
  trivialen Untergruppen nur noch Untergruppen von der Mächtigkeit $p$ und da
  jede endliche Gruppe von Primzahlordnung zyklisch sein muss, sind diese
  Untergruppen allesamt zyklisch.
  Also können wir uns auf
  \begin{align*}
    \langle (a,b) \rangle, \qquad a,b \in C_p
  \end{align*}
  einschränken. Für alle $(0,0) \neq (a,b) \in C_p \times C_p$ gilt
  \begin{align*}
    \langle (a,b) \rangle = \{k(a,b): 0 \leq k \leq p-1\}.
  \end{align*}
  Da laut Proposition 3.2.4.9 jedes Element aus dieser Menge ein erzeugendes Element
  ist, kann der Schnitt zweier solcher Untergruppen nur das Null-Element erhalten.
  Gleichzeitig liegt aber jedes Element in einer Untergruppe der Mächtigkeit $p$,
  also haben wir genau $(p^2-1)/(p-1) = (p+1)$ nichttriviale Untergruppen
  und insgesamt $(p+3)$ Untergruppen.
  \item Da $1,p,p^2$ die drei einzigen Teiler von $p^2$ sind, sind laut
  Satz 3.2.4.8.
  \begin{align*}
    \langle\overline{1}\rangle &= C_{p^2} \\
    \langle\overline{p}\rangle &= \{\overline{kp}: 0 \leq k < p\} \\
    \langle\overline{p^2}\rangle &= \{0\} \\
  \end{align*}
  die einzigen 3 Untergruppen von $C_{p^2}$.
\end{enumerate}
\end{solution}
\begin{algebraUE}{212}
  Sei $p$ eine Primzahl. Wie viele Automorphismen hat $C_p \times C_p$? \\
  \textit{Hinweis:} Verwenden Sie Ihr Wissen aus der Linearen Algebra.
\end{algebraUE}
\begin{solution}
  Wir fassen $C_p \times C_p$ als zwei-dimensionalen Vektorraum über dem Körper $C_p$
  auf und betrachten die Basis $\{(0,1),(1,0)\}$.
  Sei $\phi: C_p \times C_p \rightarrow C_p \times C_p$ ein Automorphismus.
  Für das Bild des ersten Basisvektors können wir alles außer der Null wählen.
  \begin{align*}
    \varphi((0,1)) := (a,b) \neq (0,0)
  \end{align*}
  Also haben wir $p^2 - 1$ Möglichkeiten.
  Da $|\spn\{(a,b)\}| = p$ gibt es in $C_p \times C_p$ insgesamt $p^2 - p$
  linear unabhängige Vektoren auf die $(1,0)$ abgebildet werden kann.
  Also erhalten wir $(p^2 - 1)(p^2 - p)$ verschiedene Automorphismen.
\end{solution}
