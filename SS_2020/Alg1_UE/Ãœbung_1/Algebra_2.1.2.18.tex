\begin{exercise}
    Sei $(P,\leq)$ eine Halbordnung, in der jede Teilmenge ein Infimum hat.
    Dann hat auch jede Teilmenge von $P$ ein Supremum. Insbesondere liegt ein
    vollständiger Verband vor.
\end{exercise}
\begin{solution}
Sei $M \subseteq P$ beliebig. \\
Mit $O := \{p \in P: \forall m \in M: p \geq m\}$
bezeichne die Menge aller oberen Schranken von M. Laut Voraussetzung wissen wir,
dass diese Menge ein Infimum besitzt. Nun behaupten wir $\inf(O)  = \min(O) = \sup(M)$. \\
Dafür müssen wir nur noch zeigen, dass $\inf(O) \in O$. Sei nun $m \in M$ beliebig. \\
$\forall o \in O: m \leq O$. Also ist $m$ eine untere Schranke von O. \\
Da $\inf(O)$ allerdings die größte untere Schranke ist, gilt: \\
$\forall m \in M: m \leq \inf(O)$ und somit $O \ni \inf(O) = \min(O) = \sup(M)$



Die Halbordnung $(\mathbb{N}, \leq)$ ist kein Gegenbeispiel, da sie die Voraussetzung
nicht erfüllt. \\
Dazu betrachte man die leere Menge: $\emptyset$\\
Die Menge aller oberen Schranken M = $\{n \in \mathbb{N}: \forall x \in \emptyset: n \leq x \}$\\
der leeren Menge ist klarerweise ganz $\mathbb{N}$.
Sei nun angenommen, dass diese Menge ein Maximum hätte. \\
Im Fundament finden sich folgende Defintionen von $<, \leq$:
\begin{align*}
  n < m &: \iff (\exists t \in \mathbb{N}: n + t = m) \\
  n \leq m &: \iff (n = m) ~\text{oder}~ (n < m)
\end{align*}
Bezeichne nun $m \in \mathbb{N}$ das Maximum von $\mathbb{N}$.
Dann folgt $\forall n \in \mathbb{N}: n \leq m$. \\
Insbesondere gilt: $v(m) \leq m$. \\
Wir wissen aber: $m + 1 = v(m)$, also: $m \leq v(m)$. \\
Mit der Antisymmetrie von $\leq$ folgt schließlich: $m = v(m)$
und mit der Kürzungsregel für die Addition $1 = 0$. \\
Damit haben wir einen Widerspruch und $(N, \leq)$ hat kein Maximum.
\end{solution}
