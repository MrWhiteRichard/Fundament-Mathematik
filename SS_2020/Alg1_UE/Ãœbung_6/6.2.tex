\begin{algebraUE}{164}
Sei $G$ eine Gruppe. Für alle $g \in G$ definieren wir $\pi_g: G \rightarrow G, x \mapsto gxg^{-1}$
und betrachten die Abbildung $\Phi: g \mapsto \pi_g$. Dann gilt
\begin{enumerate}
  \item Für $g,h \in G$ gilt $\pi_g \circ \pi_h = \pi_{gh}$. Somit ist $\Phi$
  ein Homomorphismus von $G$ in die Automorphismengruppe $\Aut(G)$.
  \item Für alle $g \in G$ ist $\pi_g$ ein Automorphismus (genannt der durch
  Konjugation mit $g$ induzierte innere Automorphismus von $G$).
  \item Für den Kern von $\Phi$ gilt
  \begin{align*}
    \ker(\Phi) = Z(G) = \{g \in G: \forall h \in G: gh = hg\}.
  \end{align*}
  Insbesondere ist $\Phi: G \rightarrow \Aut(G)$ eine isomorphe Einbettung genau dann,
  wenn das Einselement $e \in G$ das einzige ist, das mit allen $g \in G$ vertauscht.
  \item Die inneren Automorphismen bilden einen Normalteiler $\Phi(G) \vartriangleleft \Aut(G)$
  der Automorphismengruppe von $G$. (Die Faktorgruppe $\Aut(G)/\Phi(G)$ nennt man
  auch die äußere Automorphismengruppe von $G$.)
\end{enumerate}
\end{algebraUE}
\begin{solution}
\leavevmode \\
\begin{enumerate}
  \item Seien $g,h,x \in G$ beliebig. Dann gilt aufgrund $(gh)^{-1} = h^{-1}g^{-1}$
  \begin{align*}
    (\pi_g \circ \pi_h)(x) = \pi_g(hxh^{-1}) = ghxh^{-1}g^{-1} = \pi_{gh}(x)
  \end{align*}
  \item Sei $g \in G$ beliebig. Dann gilt
  \begin{itemize}
    \item \begin{align*}
      \pi_g(1_G) = g1_Gg^{-1} = gg^{-1} = 1_G
    \end{align*}
    \item \begin{align*}
      \pi_g(xy^{-1}) = g(xy^{-1})g^{-1} =  gxg^{-1}gy^{-1}g^{-1} =  (gxg^{-1})(gyg^{-1})^{-1} = \pi_g(x)\pi_g(y)^{-1}
    \end{align*}
    Also ist $\pi_G$ zumindest ein Endomorphismus auf $G$.
    \item Aus Punkt 1 haben wir mit $\pi_{g^{-1}}$ aufgrund
    \begin{align*}
      \id_G = \pi_{1_G} = \pi_{gg^{-1}} = \pi_g \circ \pi_{g^{-1}}
    \end{align*}
    bereits eine Inverse gefunden und $\pi_g$ ist damit bijektiv und ein Automorphismus auf $G$.
  \end{itemize}
\item Es gelten folgend Äquivalenzen für alle $g \in G$:
\begin{align*}
  g \in \ker (\Phi) \Leftrightarrow \pi_g = \Phi (g) = \id_G \Leftrightarrow \Forall h \in G: ghg^{-1} = h \Leftrightarrow \Forall h \in G: gh = hg 
\end{align*}
Wenn $Z(G) = \{1_G\}$ gilt, folgt mit
\begin{align*}
  \pi_g = \pi_h \implies \pi_{gh^{-1}} = \id_G \implies gh^{-1} \in Z(G) = \{1_G\} \implies g = h.
\end{align*}
direkt die Injektivität von $\Phi$. Klarerweise ist die Bedingung auch notwendig für die Injektivität.
\item Aufgrund der Homomorphie-Eigenschaft von $\Phi$ ist $\Phi(G)$ eine Untergruppe
der Automorphismengruppe von $G$. Weiters gilt für $\psi \in \Aut(G), \pi_g \in \Phi(G), x\in G$ beliebig
\begin{align*}
  (\psi \circ \pi_g) (x)= \psi(gxg^{-1}) = \psi(g)\psi(x)\psi(g)^{-1} = \pi_{\psi(g)}(\psi(x)) = (\pi_{\psi(g)} \circ \psi)(x).
\end{align*}
Also folgt $\forall \psi \in \Aut(G): \psi\Phi(G) \subseteq \Phi(G)\psi$ und $\Phi(G)$ ist somit ein Normalteiler von $\Aut(G)$.
\end{enumerate}

\end{solution}
