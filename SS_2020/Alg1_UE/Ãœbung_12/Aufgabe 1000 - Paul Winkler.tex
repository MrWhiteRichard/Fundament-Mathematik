\documentclass[a4paper,11pt]{article}

\usepackage[german]{babel}
\usepackage{amsmath}
\usepackage{amsfonts}
\usepackage{amssymb}
\usepackage{fullpage}
\usepackage{mathtools}
\usepackage{ngerman}
\begin{document}
\begin{itemize}

\begin{center}
  \textbf{Aufgabe 1000 - Paul Winkler}
\end{center}
\item
 \textit{Welches Konzept, das in der Algebra eine besonders wichtige Rolle spielt, hat zur Erweiterung Ihres Verst\"andnisses von Mathematik am meisten beigetragen?}

 Sehr lehrreich waren die zahlreichen Anwendungen von Faktorisierungen / Kongruenzrelationen / Normalteilern.
 \item
 \textit{Unter den Inhalten der Vorlesung fanden Sie welches Resultat am beeindruckendsten?}

 Den Darstellungssatz von Stone.
 \item
 \textit{Was war am schwierigsten zu verstehen?}

 Freie Algebren, Koprodukte, Variet\"aten. Aus mehreren Gr\"unden: Die Thematik ist sehr abstrakt und es gibt zu viele Definitionen und Resultate, um alle gleichzeitig im Kopf zu behalten. Ich habe meistens Teilresultate verstanden, aber sofort wieder Verst\"andnis- oder Wissensl\"ucken in den vorherigen Abschnitten gehabt. Der Nutzen dieser Konzepte ist mir \"uberhaupt nicht klar, das gesamte Kapitel wirkt irgendwie isoliert vom Rest der Vorlesung.
 \item
 \textit{Was ist die interessanteste mathematische Frage, die durch die Inhalte der Vorlesung zwar nahegelegt, in der Lehrveranstaltung (VO+UE+RE) aber nicht zufriedenstellend behandelt, geschweige denn beantwortet wurde?}

 Einige interessante zahlentheoretische Aussagen wie die Transzendenz von $\pi$ oder $e.$ Zumindest den Beweis der Transzendenz irgendeiner konkreten Zahl h\"atte ich mir gew\"unscht.

 \item
 \textit{Was an den Inhalten der Vorlesung hat Sie am wenigsten angesprochen?}

 Das gesamte Kapitel 4 (au\ss{}er vielleicht 4.1.1 und 4.1.2) aus oben genannten Gr\"unden und aus fehlendem Interesse daran.

 \item
 \textit{Was an Organisation, Stil etc. der Vorlesung ist Ihrem Geschmack nach am ehesten missgl\"uckt?}

 Die Aufteilung des Stoffs im Hinblick auf die \"Ubung war etwas unausgeglichen. In manchen Wochen gab es schwierige \"Ubungsaufgaben und viele Stunden Vorlesungsmaterial dazu, in anderen waren die Beispiele einfach und nur wenige Vorlesungen anzuschauen.

 Des Weiteren k\"onnte man eventuell ein paar Abschnitte und Definitionen weglassen und sich daf\"ur genauer mit der restlichen Materie besch\"aftigen. Ich hatte das Gef\"uhl, dass die Vorlesung zu viele Themen auf einmal anschneiden wollte (auch wenn ich zugeben muss, dass Fun Facts wie die Konstruktion mit Zirkel und Lineal die mitunter spannendsten Elemente der Vorlesung waren).

 Ich finde auch, dass die Beweisf\"uhrung im Skript teilweise un\"ubersichtlich ist (z. B. Seite 325: eine Seite schwer lesbarer Flie\ss{}text und die Proposition dann als Zusammenfassung davon).
\end{itemize}
\end{document}
