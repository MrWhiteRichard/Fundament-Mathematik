\begin{algebraUE}{361}
Gib das Minimialpolynom von $\sqrt{2} + \sqrt{3}$ und $\sqrt{3} + i$ über $\Q$ an.
\end{algebraUE}

\begin{solution}
  Gib das Minimialpolynom von $\sqrt{2} + \sqrt{3}$ und $\sqrt{3} + i$ über $\Q$ an.

  \begin{itemize}
  \item Folgendes Polynome hat $\sqrt{2} + \sqrt{3}$ als Nullstelle:
  \begin{align*}
    m_1(x) &:= (x^2 - 5)^2 - 24 = x^4 - 10x^2 + 1 \\
    &= \left(x + \sqrt{5+\sqrt{24}}\right)\left(x + \sqrt{5-\sqrt{24}}\right)
    \left(x - \sqrt{5+\sqrt{24}}\right)\left(x - \sqrt{5-\sqrt{24}}\right).
  \end{align*}
  \begin{align*}
    \left(x + \sqrt{5+\sqrt{24}}\right)\left(x + \sqrt{5-\sqrt{24}}\right) &= \left(x^2 + \left(\sqrt{5+\sqrt{24}} + \sqrt{5-\sqrt{24}}\right)x + 1\right) \\
    \left(x + \sqrt{5+\sqrt{24}}\right)\left(x - \sqrt{5+\sqrt{24}}\right) &= \left(x^2 - 5 - \sqrt{24}\right) \\
    \left(x + \sqrt{5+\sqrt{24}}\right)\left(x - \sqrt{5-\sqrt{24}}\right) &= \left(x^2 + \left(\sqrt{5+\sqrt{24}} - \sqrt{5-\sqrt{24}}\right)x - 1\right)
  \end{align*}
  Wir berechnen weiter
  \begin{align*}
     \sqrt{5+\sqrt{24}} + \sqrt{5-\sqrt{24}} &= \sqrt{5+2\sqrt{6}} + \sqrt{5-2\sqrt{6}}
     = \sqrt{\frac{(\sqrt{6}+2)^2}{2}} + \sqrt{\frac{(\sqrt{6}-2)^2}{2}} \\
     &= \frac{(\sqrt{6}+2) + (\sqrt{6}-2)}{\sqrt{2}} = \sqrt{6}\sqrt{2} = 2\sqrt{3} \notin \Q\\
     \sqrt{5+\sqrt{24}} + \sqrt{5-\sqrt{24}} &= \frac{(\sqrt{6}+2) - (\sqrt{6}-2)}{\sqrt{2}} \\
     &= \frac{4}{\sqrt{2}} = 2\sqrt{2} \notin \Q.
  \end{align*}
  Damit ist es nicht möglich die Linearfaktoren über $\R$ zu quadratischen Faktoren in $\Q$
  zusammenzufassen und das Polynom muss über $\Q$ irreduzibel sein.
  \item Folgendes Polynom hat $\sqrt{3} + i$ als Nullstelle:
  \begin{align*}
    m_2(x) &:= (x^2 -2)^2 + 12 = x^4 - 4x^2 + 16 = (x - \sqrt{3} + i)(x - \sqrt{3} - i)(x + \sqrt{3} + i)(x + \sqrt{3} - i) \\
    &= \left(x^2 - 2\sqrt{3}x + 4\right)\left(x^2 + 2\sqrt{3}x + 4\right)
  \end{align*}
  Hier können wir jeweils die beiden Faktoren, deren Nullstellen sich nur durch Konjugation unterscheiden, zu einem quadratischen Polynom über $\R$ zusammenfassen, welches dann klarerweise irreduzibel ist. In beiden Fällen erhält man Polynome, die nichtrationale Koeffizienten haben. Daher sind beide Polynome irreduzibel über $\Q$.
  \end{itemize}
\end{solution}
