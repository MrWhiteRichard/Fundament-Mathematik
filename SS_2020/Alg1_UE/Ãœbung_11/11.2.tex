\begin{algebraUE}{354}
Für $K \leq E \leq L$ (als Körper oder auch als Schiefkörper/Divisionsringe) gilt
\begin{align*}
  [L:K] = [L:E]\cdot[E:K].
\end{align*}
Beweisen Sie den Gradsatz, indem Sie zeigen: Ist die Familie $(a_i)_{i \in I}$
eine Basis von $E$ über $K$, die Familie $(b_j)_{j \in J}$ eine Basis von $L$ über $E$,
so ist die Familie $(a_ib_j)_{(i,j) \in I \times J}$ eine Basis von $L$ über $K$.
(Achtung: Ihre Argumentation muss auch für unendliches $I$ und $J$ gelten.)

\end{algebraUE}

\begin{solution}
\leavevmode \\
  \begin{itemize}
      \item \textbf{Erzeugendensystem:} Sei $x \in L:K$ beliebig. Fassen wir $L$ als Vektorraum über $E$ auf, erhalten wir für $x$
      die Darstellung als definitionsgemäß endliche Linearkombination
  \begin{align*}
      x = \sum_{j=1}^n \underbrace{\beta_j}_{\in E} b_j = \sum_{j=1}^n \left(\sum_{i=1}^m \underbrace{\alpha_{ij}}_{\in K} a_i \right) b_j = \sum_{\substack{i \in \{1, ..., m\} \\ {j\in \{1,...,n\}}}} \alpha_{ij} (a_i b_j).
  \end{align*}
  \item \textbf{Linear unabhängig:} Wir betrachten folgende Darstellung des Nullvektors als Linearkombination:
  \begin{align*}
      0 = \sum_{j=1}^n \sum_{i=1}^m \alpha_{ij} (a_i b_j) = \sum_{j=1}^n \left(\sum_{i=1}^m \alpha_{ij} a_i\right) b_j.
  \end{align*}
  Weil $(b_j)_{j \in J}$ eine Basis von $L:E$ ist, folgt $\sum_{i=1}^m \alpha_{ij} a_i = 0$ für alle $j = 1,\dots,n.$

  Weil $(a_i)_{i \in I}$ eine Basis von $E:K$ ist, gilt daher für alle $i = 1,\dots,m$: $\alpha_{ij} = 0$.
  \end{itemize}
\end{solution}
