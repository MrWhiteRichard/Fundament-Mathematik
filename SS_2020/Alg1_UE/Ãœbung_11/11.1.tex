\begin{algebraUE}{349}

Gegeben sei das Polynom $f(x) = x^4 + x^3 + x^2 +x^1 + 1 = \frac{x^5 - 1}{x - 1}.$
\begin{enumerate}
  \item Zerlegen Sie $f$ in seine irreduzible Faktoren über $\Q,\R$ und $\C$.
  \item Finden Sie Wurzelausdrücke für die reellen Zahlen $\cos(\frac{k\pi}{5}), k = 1,2,3,4$.
\end{enumerate}

\end{algebraUE}

\begin{solution}
  $f(x) = 0$ genau dann wenn $x \neq 1$ und $x^5 = 1$. Über $\C$ zerfällt das
  Polynom nach dem Fundamentalsatz der Algebra in Linearfaktoren, welche dann klarerweise irreduzibel sind:
  \begin{align*}
    f(x) &= \left(x - \exp\left(\frac{2\pi i}{5}\right)\right)\left(x - \exp\left(\frac{4\pi i}{5}\right)\right)
    \left(x - \exp\left(\frac{6\pi i}{5}\right)\right)\left(x - \exp\left(\frac{8\pi i}{5}\right)\right) \\
    &= \left(x - \cos\left(\frac{2\pi}{5}\right) - i\sin\left(\frac{2\pi}{5}\right)\right)
    \left(x - \cos\left(\frac{4\pi}{5}\right) - i\sin\left(\frac{4\pi}{5}\right)\right)\\
    &\left(x - \cos\left(\frac{6\pi}{5}\right) - i\sin\left(\frac{6\pi}{5}\right)\right)
    \left(x - \cos\left(\frac{8\pi}{5}\right) - i\sin\left(\frac{8\pi}{5}\right)\right)
  \end{align*}
  Über $\R$ fassen wir einfach die komplex konjugierten Linearfaktoren zu quadratischen
  Faktoren zusammen.
  \begin{align*}
    f(x) = \left(x^2 - 2\cos\left(\frac{2\pi}{5}\right)x + 1\right)\left(x^2 - 2\cos\left(\frac{4\pi}{5}\right)x + 1\right)
  \end{align*}
  Nun können wir ausmultiplizieren und die Koeffizienten vergleichen:
  \begin{align*}
    x^4 + x^3 + x^2 +x^1 + 1 &= x^4 - 2\left(\cos\left(\frac{2\pi}{5}\right) + \cos\left(\frac{4\pi}{5}\right)\right)x^3
    + \left(2 + 4\cos\left(\frac{2\pi}{5}\right)\cos\left(\frac{4\pi}{5}\right)\right)x^2\\
    &- 2\left(\cos\left(\frac{2\pi}{5}\right) + \cos\left(\frac{4\pi}{5}\right)\right)x
    + 1
  \end{align*}
  Wir erhalten somit die beiden Gleichungen
  \begin{align*}
    \cos\left(\frac{2\pi}{5}\right) + \cos\left(\frac{4\pi}{5}\right) &= -\frac{1}{2} \\
    \cos\left(\frac{2\pi}{5}\right)\cos\left(\frac{4\pi}{5}\right) &= -\frac{1}{4}.
  \end{align*}
  Setzen wir die obige in die untere ein, erhalten wir die quadratische Gleichung
  \begin{align*}
    \cos\left(\frac{4\pi}{5}\right)^2 + \frac{1}{2}\cos\left(\frac{4\pi}{5}\right) - \frac{1}{4} = 0
  \end{align*}
  mit der Lösung $\cos\left(\frac{4\pi}{5}\right) = -\frac{1}{4} \pm \sqrt{\frac{1}{16} + \frac{1}{4}} = \frac{1}{4}(-1\pm\sqrt{5})$.
  Da der Kosinus im Intervall $[\frac{\pi}{2},\frac{3\pi}{2}]$ negativ ist
  und um den Punkt $\pi/2$ schiefsymmetrisch, gilt
  \begin{align*}
    \cos\left(\frac{4\pi}{5}\right) &= \frac{1}{4}(-1-\sqrt{5}) \\
    \cos\left(\frac{2\pi}{5}\right) &= \frac{1}{4}(-1+\sqrt{5}) \\
    \cos\left(\frac{1\pi}{5}\right) &= \frac{1}{4}(1+\sqrt{5}) \\
    \cos\left(\frac{3\pi}{5}\right) &= \frac{1}{4}(1-\sqrt{5}).
  \end{align*}
  Die Zahlen können auch als Nullstellen des Polynoms
  \begin{align*}
    f(x) = ((4x)^2 - 6)^2 - 20.
  \end{align*}
  beschrieben werden. \\
  Also liegen die quadratischen Faktoren nicht in $\Q(x)$ und somit ist das Polynom über $\Q$ irreduzibel.

\end{solution}
