\begin{algebraUE}{372'}
Zeigen Sie, dass das regelmäßige Siebeneck mit Radius 1 nicht konstruierbar ist.
\end{algebraUE}

\begin{solution}
  Wir definieren eine komplexe Zahl $z \in \C$ sei genau dann konstruierbar, wenn ihr
  Real- und ihr Imaginärteil es sind. Dann behaupten wir, dass jedes
  aus einer Menge von Punkten $A \supseteq \{(0,0),(0,1)\}$ konstruierbare $z := a + ib \in \C$,
  in einer Quadratwurzelerweiterung von $\Q(B)$ liegt, wobei
  $B$ die Menge der Koordinaten aus $A$ bezeichnet.
  Mit Satz 6.1.6.8 erhalten wir, dass $a,b$ jeweils in einer Quadratwurzelerweiterung $L_a,L_b$ von $\Q(B)$ liegen.
  Jetzt können wir die Quadratwurzelerweiterungsschritte einfach aneinanderreihen
  und erhalten eine Quadratwurzelerweiterung $L$, welche $a$ und $b$ erhält.
  Diese können wir dann gegebenfalls noch mit $i$ erweitern. Da $i^2 = -1$
  sicher in $L$ liegt, ist auch $L(i)$ eine Quadratwurzelerweiterung. \\
  Wenn wir das regelmäßige Siebeneck konstruieren können, können wir es auch um
  den Mittelpunkt $(0,0)$ konstruieren. In diesem Fall stimmen die Koordinaten der Eckpunkte genau
  mit den Real- und Imaginärteilen siebten komplexen Einheitswurzeln $\zeta_7^i, i = 1,\dots,7$ überein.
  Wir zeigen, dass die primitive siebte Einheitswurzel $\zeta_7$ nicht in einer
  Quadratwurzelerweiterung liegen kann und mit der Kontraposition der vorigen Aussage
  nicht konstruierbar sein kann. Nach Definition ist $\zeta_7$ eine Nullstelle des Polynoms $x^7-1.$ Durch Polynomdivision erhält man die Faktorisierung
  \begin{align}
      x^7-1 = (x-1)\underbrace{(x^6+x^5+x^4+x^3+x^2+x+1)}_{=: m(x)}
  \end{align}
  und erkennt, dass $\zeta_7$ Nullstelle des zweiten Faktors sein muss.

  Wäre $m(x)$ reduzibel, so wäre auch $m(x+1)$ reduzibel. Allerdings gilt
  \begin{align}
      m(x+1) = x^6+7x^5+21x^4+35x^3+35x^2+21x+7,
  \end{align}
  mit $7\nmid a_6, 7 | a_i, i = 0,\dots,5, 7^2 \nmid a_0$,
  was nach dem Eisensteinkriterium irreduzibel in $\Z[x]$ ist.
  Laut Proposition 5.3.2.9 ist $f$ dann auch irreduzibel über $\Q[x]$, da
  $\Q$ der Quotientenkörper von $\Z$ ist.
  $m(x)$ ist also ebenfalls irreduzibel über $\Q$ und daher das Minimalpolynom von $\zeta_7$ über $\Q$. 

  Damit gilt $[\mathbb{Q}(\zeta_7):\mathbb{Q}] = \mathrm{grad}(m(x)) = 6$; der Grad der Körpererweiterung ist also nicht Teiler einer Zweierpotenz und $\zeta_7$ folglich nicht konstruierbar. Laut Definition ist folglich Real- oder Imaginärteil nicht konstruierbar und auf jeden Fall
  ist ein Eckpunkt unseres regelmäßigen Siebenecks nicht konstruierbar.


  *** Alternative***

  Wir zeigen, dass wir $\sin(\frac{2 \pi }{7})$ nicht konstruieren können,
  und damit auch nicht das regelmäßige Siebeneck mit Radius 1.

  Es gilt:

  \begin{align*}
    \cos(2 \pi) + i \sin(2 \pi) = (\cos(\frac{2 \pi }{7}) + i \sin(\frac{2 \pi }{7}))^{7}
  \end{align*}

  Ausmultiplizieren und Vergleichen der Imaginärteile (mit Python) führt zu

  \begin{align*}
    0 = \sin(2 \pi) = \sin(\frac{2 \pi }{7}) (-64 \sin(\frac{2 \pi }{7})^{6} + 112 \sin(\frac{2 \pi }{7})^4 - 56 \sin(\frac{2 \pi }{7})^2) + 7).
  \end{align*}

  Es ist also $\sin(\frac{2 \pi }{7})$ Nullstelle vom Polynom

  \begin{align*}
    p(x) = x\underbrace{(-64x^6 + 112x^4 - 56x^2 + 7)}_{=:\tilde{p}(x)}
  \end{align*}

  Das Polynom $\tilde{p}(x)$ ist nach dem Eisensteinkriterium (mit der Primzahl $7$) irreduzibel und somit Minimalpolynom von $\sin(\frac{2 \pi }{7})$.

  Damit gilt $[\mathbb{Q}(\sin(\frac{2 \pi }{7})):\mathbb{Q}] = \mathrm{grad}(\tilde{p}(x)) = 6$; der Grad der Körpererweiterung ist also nicht Teiler einer Zweierpotenz und $\sin(\frac{2 \pi }{7})$ folglich nicht konstruierbar.

\end{solution}
