\begin{algebraUE}{386}
  Die Automorphismen von GF($p^n$) sind genau die Abbildungen der Form $a \mapsto a^{p^k}$
  mit $k = 0,1,\dots,n-1$ (die sogenannten Frobeniusautomorphismen). Sie bilden
  eine zyklische Gruppe, die vom Automorphismus $a \mapsto a^p$ erzeugt wird.


Beweisen Sie Satz 6.3.3.3, indem Sie folgendes zeigen:
\begin{enumerate}
  \item Ist $p$ eine Primzahl, $k,n \in \N, n \geq 1$, so ist die Abbildung
  $\varphi: a \mapsto a^p$ ein Automorphismus von GF($p^n$).
  \item Die in der Automorphismengruppe Aut(GF($p^n$)) von $\varphi$ erzeugte
  Untergruppe besteht aus allen \\
  $\varphi^k: a \mapsto a^{p^k}$ mit $k = 0,\dots,n-1$.
  \item Jeder Automorphismus $\varphi$ eines Körpers $K$ lässt den Primkörper $P$
  von $K$ punktweise fest. \\
  \textit{Hinweis:} $P$ wird als Ring mit $1$ von der leeren Menge erzeugt.
  \item Jeder Automorphismus von GF($p^n$) ist von der Form $a \mapsto a^{p^k}$. \\
  \textit{Hinweis:} Jeder Automorphismus ist eindeutig durch seinen Wert
  für ein primitives Element $\alpha$ bestimmt. Als mögliche Werte kommen genau
  die Konjugierten von $\alpha$ in Frage. Davon gibt es $n$ Stück, genauso viele
  wie Frobeniusautomorphismen.
\end{enumerate}
\end{algebraUE}
\begin{solution}
  \begin{itemize}
      \item[1.] $\varphi$ ist ein Körperhomomorphismus wegen
      \begin{align}
          \varphi(a + b) = (a + b)^p = a^p + b^p = \varphi(a) + \varphi(b), \\
          \varphi(ab) = (ab)^p = a^p b^p = \varphi(a) \varphi(b), \\
          \varphi(0) = 0.
      \end{align}

  Die Injektivität gilt wegen
  \begin{align}
      \varphi(a) = \varphi(b) \Longleftrightarrow a^p = b^p \Longleftrightarrow (a-b)^p = 0 \Longleftrightarrow a - b = 0.
  \end{align}
  Weil GF($p^n$) eine endliche Menge ist, folgt aus Injektivität schon Surjektivität.

  \item[2.] Es gilt $\langle\varphi\rangle = \{\varphi^k~|~ 0 \leq k \leq \mathrm{ord}(\varphi)\}.$ Nach Proposition 6.3.3.2 (5) gibt es genau $n$ Automorphismen von GF($p^n$), es muss also $\mathrm{ord}(\varphi) < n$ gelten.

  \item[3.] $P$ wird als $\text{Ring}_1$ von der leeren Menge erzeugt. Da ein Automorphismus bereits durch seine Einschränkung auf ein Erzeugendensystem eindeutig bestimmt ist, kann es damit nur einen Automorphismus geben. Die Identität ist ein solcher und folglich auch der einzige.

  \item[4.] Wir zeigen noch, dass die Automorphismen $\varphi^k, k \in \{0, ..., n-1\}$ alle verschieden sind. Angenommen, es gäbe $i, j < n, i \neq j$ mit $a^{p^i} = a^{p^j}$ für alle $a.$ Die multiplikative Gruppe GF($p^n$)* hat ein erzeugendes Element $\alpha.$ Insbesondere gilt für dieses
  \begin{align}
      1 = \alpha^{p^j - p^i}
  \end{align}
  im Widerspruch zu $\mathrm{ord}(\alpha) = p^n - 1$. Es gibt also genau $n$ Frobeniusautomorphismen.

  Ein beliebiger Automorphismus $\psi$ ist eindeutig durch seinen Wert für $\alpha$ festgelegt. Sei $f$ das Minimalpolynom von $\alpha$ über $P.$ Nach Proposition 6.3.3.2 (4) kommen als Werte für $\alpha$ unter $\psi$ genau die $n$ verschiedenen Nullstellen von $f$ infrage \textbf{(warum kann keine Nullstelle doppelt vorkommen?)}. Es gibt also $n$ Automorphismen, genauso viele wie Frobeniusautomorphismen.
\end{itemize}
\end{solution}
