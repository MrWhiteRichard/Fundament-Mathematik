\begin{algebraUE}{391}
Sei $K$ ein Körper der Charakteristik $p > 0$. Man zeige: \\
$x^p + a \in K[x]$ ist entweder irreduzibel, oder $p$-te Potenz eines linearen
Polynoms.
\end{algebraUE}

\begin{solution}\leavevmode \\
Fall 1: $\exists b \in K: b^p + a = 0$ (In einem endlichen Körper gibt es
so ein Element immer aufgrund der Frobeniusautomorphismen.) \\
Dann gilt nach Satz 3.3.4.3.
\begin{align*}
  (x - b)^p = x^p + (-b)^p = x^p - (-1)^pa.
\end{align*}
Im Spezialfall $p = 2$ gilt $(-1)^2 = 1 = -1$, in allen anderen Fällen ist $p$ ungerade
und es gilt $(-1)^p = -1$. Also folgt
\begin{align*}
  (x - b)^p = x^p + a.
\end{align*}
Fall 2: $\nexists b \in K: b^p + a = 0$. \\
Wir betrachten den Zerfällungskörper $E$ von $x^p + a$. In diesem zerfällt
das Polynom in Linearfaktoren. Insbesondere hat $x^p + a$ in $E$ eine Nullstelle
und nach Fall 1 gilt
\begin{align*}
  x^p + a = (x - b)^p
\end{align*}
mit einem $b \in E\backslash K$.
Sei nun angenommen, $x^p + a$ wäre nicht
irreduzibel über $K$, also exisitiert ein $0 < k < p$, sodass
\begin{align*}
  x^p - a = \underbrace{(x - b)^k}_{\in K[x]}(x - b)^{p - k}.
\end{align*}
Nach der binomischen Formel gilt
\begin{align*}
  (x - b)^k = x^k - kbx^{k-1} + \dots + (-b)^k.
\end{align*}
Es folgt also $-kb \in K$ und damit auch $b \in K$. Widerspruch!
\end{solution}
