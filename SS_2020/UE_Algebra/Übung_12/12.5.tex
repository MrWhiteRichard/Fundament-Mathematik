\begin{algebraUE}{394}
  Begründen Sie, warum der Faktorring $K := \mathbb{Z}_2[x]/(x^3 + x + 1)$ ein Körper ist und
   berechnen Sie die multiplikative Inverse von $x + (x^3 + x + 1)$ mit dem euklidischen Algorithmus.

\end{algebraUE}

\begin{solution}
  Das Polynom $x^3 + x + 1$ ist irreduzibel über $\mathbb{Z}_2,$ da es Grad 3 und keine Nullstellen hat. Deshalb ist $K$ nach dem Satz von Kronecker ein Körper.

  Da $\mathbb{Z}_2[x]$ als Polynomring in einer Variablen über einem Körper ein euklidischer Ring ist, können wir den Euklidischen Algorithmus anwenden und erhalten in einem Schritt
  \begin{align}
x^3 + x + 1 = x(x^2 + 1) + 1
\end{align}
und somit $\mathrm{ggT}(x^3 + x + 1, x) = 1.$ Wir können 1 als Linearkombination dieser Polynome darstellen:
\begin{align}
    1 = (x^3 + x + 1) - (x^2 + 1)x \Longleftrightarrow x(x^2 + 1) \equiv 1 \mathrm{~mod~} (x^3 + x + 1).
\end{align}

Die Restklasse von $-(x^2+1)$ ist im Faktorring also multiplikativ invers zur Restklasse von $x$.
\end{solution}
