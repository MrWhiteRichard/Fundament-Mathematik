\begin{algebraUE}{168}
Sei $G := S_4$. Wir geben die Elemente von $G$ in Zyklenschreibweise an. Sei
$U$ die vom Element $(1234)$ erzeugte Untergruppe und $N = \{\id, (12)(34),(13)(24),(14)(23)\}$.
Begründen Sie, warum $N \vartriangleleft S_4$ ein Normalteiler ist. Bestimmen
Sie die Gruppen $NU, N\cap U, NU/N, U/(N \cap U)$ und geben Sie den kanonischen
Isomorphismus zwischen $NU/N$ und $U/(N \cap U)$ explizit an.
\end{algebraUE}
\begin{solution}
\begin{align*}
  U = \{\id, (1234),(13)(24),(4321)\}
\end{align*}
Zuerst zeigen wir, dass $N$ eine Untergruppe von $S_4$ ist.
\begin{align*}
  (12)(34) \circ (13)(24) &= (14)(23) \\
  (13)(24) \circ (12)(34) &= (14)(23) \\
  (12)(34) \circ (14)(23) &= (13)(24) \\
  (14)(23) \circ (12)(34) &= (13)(24) \\
  (13)(24) \circ (14)(23) &= (12)(34) \\
  (14)(23) \circ (13)(24) &= (12)(34)
\end{align*}
Zusätzlich bemerkt man leicht, dass jedes Element aus $N$ zu sich selbst invers ist.
Laut Proposition 3.2.5.11.4. ist die Untergruppe $N$ genau dann ein Normalteiler von $S_4$
wenn sie von jedem Permutationstyp entweder kein oder alle $\pi \in S_n$ dieses Typs
enthält. Unter dem Permutationstyp von $\pi \in S_n$ sei dabei die Familie $(v_k(\pi))_{k \in \N}$
verstanden, wobei $v_k(\pi)$ die Anzahl der Zyklen der Länge $k$ in einer beliebigen
Darstellung von $\pi$ als Produkt elementfremden Zyklen ist. \\
$N$ enthält mit der Identität alle Zyklen vom Typ $(0,4,\dots)$, sowie alle
Zyklen vom Typ $(0,0,2)$, wie man leicht überprüfen kann.
Also ist $N$ ein Normalteiler.

\begin{align*}
  NU &= \{\id,(12)(34),(13)(24),(14)(23),(1234),(24),(4321),(13)\} \\
  N \cap U &= \{\id, (13)(24) \} \\
\end{align*}
Die durch $N$ induzierte Kongruenzrelation ist charakterisiert durch
$f \sim g :\iff f^{-1}\circ g \in N$.
Also ist
\begin{align*}
  NU/N = \{ \{\id, (12)(34), (13)(24), (14)(23)\}, \{(1234),(24),(4321),(13)\}\}
\end{align*}
und
\begin{align*}
  U/(N \cap U) = \{\{\id, (13)(24)\}, \{(1234), (4321)\}\}
\end{align*}
Der kanonische Isomorphismus lautet daher
\begin{align*}
  \varphi: \begin{cases}
    NU/N \rightarrow U/(N \cap U) \\
    \{\id, (12)(34), (13)(24), (14)(23)\} \mapsto \{\id, (13)(24)\} \\
    \{(1234),(24),(4321),(13)\} \mapsto \{(1234), (4321)\}
  \end{cases}.
\end{align*}
\end{solution}
