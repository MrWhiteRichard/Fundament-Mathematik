\begin{algebraUE}{394}
  Begründen Sie, warum der Faktorring $K := \mathbb{Z}_2[x]/(x^3 + x + 1)$ ein Körper ist und
   berechnen Sie die multiplikative Inverse von $x + (x^3 + x + 1)$ mit dem euklidischen Algorithmus.

\end{algebraUE}

\begin{solution}
  Das Polynom $x^3 + x + 1$ ist irreduzibel über $\mathbb{Z}_2,$ da es Grad 3 und keine Nullstellen hat. Deshalb ist $K$ nach dem Satz von Kronecker ein Körper.

  Um die Inverse eines Elements $p(x) + (x^3 + x + 1) \neq 0$ aus dem Faktorring zu bestimmen,
  können wir den euklidischen Algorithmus in dem euklidischen Ring $\mathbb{Z}_2[x]$
  anwenden, um den ggT der Elemente $p(x), x^3 + x + 1$ zu bestimmen. Der Algorithmus
  liefert uns eine Darstellung des ggT als Linearkombination der beiden Elemente.
  Da $x^3 + x + 1$ bereits irreduzibel ist, kommen als ggT nur Einheiten oder
  $x^3 + x + 1$ selbst in Frage. Da wir nur Elemente ungleich $0$ betrachten, kann
  der zweite Fall nicht eintreten und wir erhalten somit immer eine Darstellung
  der $1$ als Linearkombination von $p(x)$ und $x^3 + x + 1$. Aus dieser Darstellung
  können wir dann unmittelbar die Inverse ablesen.
  \begin{align*}
    x^3 + x + 1 = x(x^2 + 1) + 1 \implies -x(x^2 + 1) \equiv 1 \mathrm{~mod~} (x^3 + x + 1).
  \end{align*}
Die Restklasse von $-(x^2+1)$ ist im Faktorring also multiplikativ invers zur Restklasse von $x$.
\end{solution}
