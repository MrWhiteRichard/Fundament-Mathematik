\begin{exercise}
    Zeigen Sie, dass jede abzählbare, dichte Kette $(M, \prec)$ ohne größtes und ohne kleinstes Element ordnungsisomorph zu $(\Q, <)$ ist.
\end{exercise}
\begin{solution}
    Wir betrachten Abzählungen $(m_n)_{n \in \N}$ von $M$ und $(q_k)_{k \in \N}$ von $\Q$. Wir definieren nun rekursiv den Isomorphismus $\varphi: M \to \Q$. Dieser soll $m_0 \mapsto q_0$ schicken. Für alle $n \in \N \setminus \Bbraces{0}$ existiert $j_n := \min\Bbraces{j \in \N \mid \forall l \in \N: (l < n \Rightarrow (m_l \prec m_n \Rightarrow \varphi(m_l) < q_j) \land (m_n \prec m_l \Rightarrow q_j < \varphi(m_l) ) )}$ und ist wohldefiniert. Für die Wohldefiniertheit braucht man, dass $(M, \prec)$ eine Kette ist, also jedes Element sich in den vorherigen endlich vielen eindeutig einordnen lässt. Für die Existenz ist natürlich wichtig, dass $\Q$ dicht ist und wir kein größtes und kleinstes Element finden, was garantiert, dass die Menge von der wir das Minimum nehmen nicht leer ist. Also schicken wir dann $m_n \mapsto q_{j_n}$.
    
    Natürlich gilt es nachzuweisen, dass $\varphi$ ein Isomorphismus ist. Schauen wir zuerst, ob die Abbildung die Ordnung erhält, wählen wir also $a,b \in M$ mit $a \prec b$. Nach unserer Konstruktion von $\varphi$ gilt $\varphi(a) < \varphi(b)$. Wählen wir nun umgekehrt $c,d \in M$ mit $\varphi(c) < \varphi(d)$. Auch hier gilt aufgrund unserer Konstruktion $c \prec d$. 

    Jetzt müssen wir uns noch mit der Bijektivität auseinandersetzen. Die Injektivität sieht man unmittelbar an der Definition von $\varphi$. Die Surjektivität wollen wir induktiv beweisen. Es gilt zu beginn $\varphi(m_0) = q_0$. Nehmen wir jetzt an, dass $n \in \N \setminus \Bbraces{0}$ und für alle $i < n$ gibt es bereits ein $m_{k_i} \in M$ mit $\varphi(m_{k_i}) = q_i$. Nun definieren wir analog zur obigen Konstruktion
    \begin{align*}
        k_n :=  \min\Bbraces{j \in \N \mid \forall l \in \N: (l < n \Rightarrow (q_l < q_n \Rightarrow m_{k_l} \prec m_j) \land (q_n < q_l \Rightarrow m_j \prec m_{k_l} ) )} .
    \end{align*}
    Nun muss also $\varphi(m_{k_n}) = q_n$ gelten und damit ist $\varphi$ auch surjektiv.
\end{solution}