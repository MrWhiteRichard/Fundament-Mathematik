\begin{algebraUE}{320}
Sei $D$ eine ganze Zahl und $R := \Z[\sqrt{D}]$ der von $\Z$ und $\sqrt{D}$
erzeugte Unterring von $\C$.
\begin{enumerate}
  \item Man zeige, dass $R$ mit der gewöhnlichen Addition und Multiplikation von $\C$
  einen Integritätsbereich bildet.
  \item Man bestimme für $D < 0$ die Einheiten in $R$.
  \item Man zeige, dass für $D = 2$ unendlich viele Einheiten in $\Z[\sqrt{D}] \subseteq $
  existieren.
\end{enumerate}
\end{algebraUE}
\begin{solution}
\leavevmode \\
\begin{itemize}
  \item $\Z[\sqrt{D}] := \{a + b\sqrt{D} | a,b \in \Z\}$ ist unter Addition, Multiplikation,
  additiver Inversenbildung abgeschlossen und enthält $0$, ist also ein Unterring von $\C$.
  Exemplarisch rechnen wir die Multiplikation nach:
  \begin{align*}
    (a + b\sqrt{D})(c + d\sqrt{D}) = \underbrace{(ac + bdD)}_{\in \Z} + \underbrace{(ad + bc)}_{\in \Z}\sqrt{D}.
  \end{align*}
  Weiters vererbt sich die Kommutativität der Multiplikation von $\C$ und $R$
  enthält auch die $1$.
  Da $\C$ als Körper insbesondere auch ein Integritätsbereich ist, ist auch
  jeder Unterring davon klarerweise nullteilerfrei. Damit ist $R$ für alle $D \in \Z$
  ebenso ein Integritätsbereich.
  \item
  In der nächsten Aufgabe (324) werden wir zeigen, dass für $D < 0$ genau jene $x \in R$ Einheiten sind, die $N(x) = 1$ erfüllen.
  Wir machen eine Fallunterscheidung:
  \begin{itemize}
    \item $D = -1$: \\
    Es gilt $N(x) = a^2 + b^2$. In dem Fall sind also $\pm1,\pm i$ alle Einheiten von $R$.
    \item $D < -1$: \\
    Es gilt $N(x) = a^2 - D b^2$. In dem Fall sind also $\pm1$ alle Einheiten von $R$.
  \end{itemize}
  \item
  Seien $a,b \in \Z$ beliebig. Die Inverse von $a + b\sqrt{2}$ lautet
  \begin{align*}
    \frac{a}{a^2 - 2b^2} - \frac{b}{a^2-2b^2}\sqrt{2}.
  \end{align*}
  Für $a^2 - 2b^2 = \pm 1$ sind diese Koeffizienten sicher in $\Z$ und somit die Inverse
  auch in $R$.
  Man sieht leicht, dass $a = 1 + \sqrt{2}$ eine Einheit ist mit $a^{-1} = -1 + \sqrt{2}$. Nun gilt für alle $n \in \N$:
  \begin{align*}
    a^n {(a^{-1})}^n = 1
  \end{align*}

  Dass es sich hierbei um unendlich viele unterschiedliche Einheiten handelt, sieht man leicht, da $|a|>1$ und somit für $m>n$ gilt $|a^m|>|a^n|$. Insbesondere gilt nicht Gleichheit.

\end{itemize}

\end{solution}
