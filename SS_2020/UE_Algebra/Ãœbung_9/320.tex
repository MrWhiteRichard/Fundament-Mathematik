\begin{algebraUE}{320}
Sei $D$ eine ganze Zahl und $R := \Z[\sqrt{D}]$ der von $\Z$ und $\sqrt{D}$
erzeugte Unterring von $\C$.
\begin{enumerate}
  \item Man zeige, dass $R$ mit der gewöhnlichen Addition und Multiplikation von $\C$
  einen Integritätsbereich bildet.
  \item Man bestimme für $D < 0$ die Einheiten in $R$.
  \item Man zeige, dass für $D = 2$ unendlich viele Einheiten in $\Z[\sqrt{D}] \subseteq $
  existieren.
\end{enumerate}
\end{algebraUE}
\begin{solution}
\leavevmode \\
\begin{itemize}
  \item $\Z[\sqrt{D}] := \{a + b\sqrt{D} | a,b \in \Z\}$ ist unter Addition, Multiplikation,
  additiver Inversenbildung abgeschlossen und enthält $0$, ist also ein Unterring von $\C$.
  Exemplarisch rechnen wir die Multiplikation nach:
  \begin{align*}
    (a + b\sqrt{D})(c + d\sqrt{D}) = \underbrace{(ac + bdD)}_{\in \Z} + \underbrace{(ad + bc)}_{\in \Z}\sqrt{D}.
  \end{align*}
  Weiters vererbt sich die Kommutativität der Multiplikation von $\C$ und $R$
  enthält auch die $1$.
  Da $\C$ als Körper insbesondere auch ein Integritätsbereich ist, ist auch
  jeder Unterring davon klarerweise nullteilerfrei. Damit ist $R$ für alle $D \in \Z$
  ebenso ein Integritätsbereich.
  \item Sei nun $D < 0 \in \Z$ beliebig. Mit Einheiten in $R$ bezeichnen wir die Einheiten
  des multiplikativen Monoids $(R,\cdot,1)$. Es gilt $a + b\sqrt{D} \in E(R)$, genau dann wenn
  \begin{align*}
    \exists c + d\sqrt{D} \in E(R): (a + b\sqrt{D})(c + d\sqrt{D}) =
    (ac + bdD) + (ad + bc)\sqrt{D} = 1.
  \end{align*}
  Wir suchen also genau jene Elemente aus $R$, welche in $R$ eine multiplikative Inverse besitzen.
  Sei uns nun ein beliebiges $a + b\sqrt{D} \neq 0$ aus $R$ vorgegeben, zu dem wir eine
  Inverse finden wollen. Dazu finden wir zuerst die Inverse in $\C$,
  welche sicher existiert und überprüfen, ob sie auch in $R$ liegt.
  Die Inverse lautet
  \begin{align*}
    \frac{a - b\sqrt{D}}{a^2 - b^2D}
  \end{align*}
  und liegt genau dann in $R$, wenn es $k_1,k_2 \in \Z$ gibt mit $(a^2 - b^2D)k_1 = a$ und $(a^2 - b^2D)k_2 = b$.
  Sei nun $a \notin \{-1,0,1\}$. Dann folgt
  \begin{align*}
    |a^2 - b^2D| \geq a^2 > a
  \end{align*}
  und somit kann es kein $k_1 \in \Z$ geben, sodass $(a^2 - b^2D)k_1 = a$.
  Für $b \notin \{-1,0,1\}$ folgt analog
  \begin{align*}
    |a^2 - b^2D| \geq -b^2D \geq b^2 > b
  \end{align*}
  und somit kann es kein $k_2 \in \Z$ geben, sodass $(a^2 - b^2D)k_2 = b$.
  Also bleiben nur noch acht mögliche Einheitskandidaten: $\pm1, \pm\sqrt{D}, \pm(1+\sqrt{D}),\pm(1-\sqrt{D})$.
  Nun unterscheiden wir zwei Fälle:
  \begin{itemize}
    \item $D = -1$: \\
    In dem Fall sind $\pm1,\pm i, \pm (1+i), \pm (1-i)$ alle Einheiten von $R$.
    \item $D < -1$: \\
    In diesem Fall folgt
    \begin{align*}
      |a^2 - b^2D| \geq -b^2D > b^2 > b
    \end{align*}
    für alle $b \neq 0$. Also bleiben als einzige Einheiten $\pm 1$ übrig.
  \end{itemize}
  \item Seien $a,b \in \Z$ beliebig. Die Inverse von $a + b\sqrt{2}$ lautet
  \begin{align*}
    \frac{a}{a^2 - 2b^2} - \frac{b}{a^2-2b^2}\sqrt{2}.
  \end{align*}
  Für $a^2 = 2b^2 \pm 1$ sind diese Koeffizienten in $\Z$ und somit die Inverse
  auch in $R$. Wir müssen also unendlich viele Lösungen der Gleichung
  \begin{align*}
    a^2 = 2b^2 \pm 1
  \end{align*}
  finden. Die ersten Lösungen dieser Gleichung lauten unter anderem $(1,0),(1,1),(3,2)$.
  Also sind $1, 1 + \sqrt{2}, 3 + 2\sqrt{2}$ Einheiten in $R$, deren Inversen ebenfalls in $R$
  liegen. Damit müssen auch alle Potenzen dieser Einheiten wieder Inversen in $R$ haben
  und somit ebenso Einheiten sein. Konkret können wir also mit $\left((1 + \sqrt{2})^n\right)_{n \in \N}$
  unendlich viele Einheiten angeben.
\end{itemize}

\end{solution}
