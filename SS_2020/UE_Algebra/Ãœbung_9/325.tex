\begin{algebraUE}{325}
Zeigen Sie, dass für einen Integritätsbereich Zerlegbarkeit in
irreduzible Elemente im Allgemeinen nicht eindeutige Zerlegbarkeit in
irreduzible Elemente impliziert.
\end{algebraUE}
\begin{solution}
Wir betrachten wieder den Integritätsbereich $R = \Z[\sqrt{-5}]$.
Zuerst müssen wir zeigen, dass hier Zerlegbarkeit in irreduzible Elemente gilt.
Dazu zeigen wir, dass es eine Zerlegung in Primelemente gibt.
Sei $(a + b\sqrt{-5}) \in R$ beliebig. Wenn $(a + b\sqrt{-5})$ prim oder eine Einheit
ist, sind wir damit schon fertig. Anderenfall existieren $(c + d\sqrt{-5}),(e + f\sqrt{-5})$
mit $(a + b\sqrt{-5})|(c + d\sqrt{-5})(e + f\sqrt{-5})$, aber
$(a + b\sqrt{-5})\nmid (c + d\sqrt{-5}), (a + b\sqrt{-5})\nmid (e + f\sqrt{-5})$.
\end{solution}
