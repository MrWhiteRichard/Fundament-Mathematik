\begin{algebraUE}{328}
Sei $R$ ein Integritätsbereich. Der Polynomring $R[x]$ ist genau dann ein
Hauptidealring, wenn $R$ ein Körper ist. \\
\textit{Hinweis:} Betrachten Sie das von $a$ und $x$ erzeugte Ideal, wo $a \neq 0$
eine Nichteinheit von $R$ ist.
\end{algebraUE}
\begin{solution}
Sei zuerst $R$ ein Körper, $I \vartriangleleft R[x]$ ein beliebiges Ideal von $R[x]$.
Dann ist laut Proposition 5.2.3.3. $R[x]$ ein eukldischer Ring und nach Satz 5.2.3.4.
insbesondere ein Hauptidealring. \\
Sei also $R$ kein Körper. Dann existiert ein Element $a$, dass in $R$ kein
multiplikatives Inverses besitzt. Das erzeugte Ideal von $a$ und $x$ lässt sich darstellen als
\begin{align*}
  (a,x) = \{r_1a + r_2x: r_1,r_2 \in R[x]\}.
\end{align*}
Ich schätze, wir wollen damit zeigen, dass das erzeugte Ideal nicht von einem
Element erzeugt werden kann und wir somit keinen Hauptidealring haben.
Wähle also mit $r_1a + r_2x$ ein beliebiges Element von $(a,x)$
und betrachte das davon erzeugte Ideal.
\begin{align*}
  (r_1a + r_2x) = \{rr_1a + rr_2x: r \in R[x]\}
\end{align*}
Um damit wieder das gesamte ursprünglich Ideal zu erhalten muss insbesondere
\begin{align*}
  \exists r \in R[x]: rr_1a + rr_2x = x.
\end{align*}
Wir stellen $p,r$ dar durch
\begin{align*}
  p = \sum_{k = 0}^n p_kx^k \\
  r = \sum_{k = 0}^m r_kx^k.
\end{align*}
\end{solution}
