\begin{algebraUE}{331}
Zeigen Sie, dass sich in euklidischen Ringen der $\ggT$ nicht nur von zwei
Elementen als deren Linearkombination schreiben lässt, sondern für eine
beliebige endliche Anzahl. Beschreiben Sie, wie man diese Darstellung
algorithmisch erhalten kann. Wie verhält es sich mit dem $\ggT$ unendlich
vieler Elemente?
\end{algebraUE}
\begin{solution}
Sei $A = \{a_1,\dots,a_n\} \subset R$ eine beliebe endliche Teilmenge von $R$.
Nun haben wir im Skript bereits eine algorithmische Darstellung vom
$\ggT$ zweier Elemente hergeleitet.
Daher berechne man zuerst $b_1 := \ggT(a_1,a_2)$.
Induktiv erhalten wir mit $b_k := \ggT(b_{k-1},a_{k+1})$
in $n - 1$ Schritten $b_{n-1} = \ggT(b_{n-1},a_{k+1})$.
Wir behaupten nun, dass $b_{n-1} = \ggT(A)$.
Es gilt zunächst $b_{n-1} | b_k | a_{k+1}, k = 1,\dots,n-1$ und $b_{n-1} | b_1 | a_1$.
Damit ist $b_{n-1}$ zumindest ein Teiler von $A$. \\
Induktiv sieht man, dass $b_{n-1}$ bereits der größte Teiler sein muss,
da klarerweise jede weitere Teiler $\widetilde{b}$ von $A: \widetilde{b} | b_1$
erfüllen muss.
Wenn $\widetilde{b} | b_k$, dann folgt aber auch nach Konstruktion $\widetilde{b} | b_{k+1}$
und schließlich $\widetilde{b} | b_{n-1}$. \\
Sei $A$ nun eine unendliche Teilmenge von $R$. Nun können wir unseren Algorithmus
zwar weiter so durchführen, allerdings terminiert er nicht.
Aufgrund der Teilerkettenbedingung wissen wir aber zumindest, dass es einen
Index $N$ geben muss, ab dem $b_{n+1} \sim b_{n}, n \geq N$ gilt.
Damit ist die Existenz des $\ggT$s zwar sichergestellt,
aber für die algorithmische Beschreibung ist die Tatsache leider nutzlos, da
wir nicht wissen können, wann wir den Index erreicht haben, ab dem wir
unseren Algorithmus terminieren können.
\end{solution}
