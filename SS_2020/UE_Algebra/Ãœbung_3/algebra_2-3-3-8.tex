\begin{exercise}
    Zeigen Sie die folgenden Aussagen.
    \begin{enumerate}[label = \alph*)]
        \item \label{kongru} Seien $\mathfrak{A} = (A, \Omega)$ eine Algebra und für alle $j \in J$ die Menge $\sim_j$ eine Kongruenzrelation. Sei weiters $\sim := \bigcap_{j \in J} \sim_j$, falls $J \neq \emptyset$ und $\sim := A \times A$, falls $J = \emptyset$. Die Menge $\sim$ ist dann eine Kongruenzrelation auf $\mathfrak{A}$.
        \item Ist $\mathfrak{A} = (A, (\omega_i)_{i \in I})$ eine Algebra und $Con(\mathfrak{A}) := \Bbraces{\sim \subseteq A \times A \mid \sim \text{ist Kongruenzrelation auf } \mathfrak{A}}$, so ist $(Con(\mathfrak{A}), \subseteq)$ ein vollständiger Verband im ordnungstheoretischen Sinn mit dem mengentheoretischen Durchschnitt als Infimum.
    \end{enumerate}
\end{exercise}

\begin{solution} 
    Wir müssen endlich wen finden der bei uns Werbung machen will!
    \begin{enumerate}[label = \alph*)]
        \item Wir unterscheiden zwei Fälle.
        \begin{enumerate}[label = Fall \arabic*:]
            \item Sei $J \neq \emptyset$.
            \begin{enumerate}[label = (\roman*)]
                \item Wissen wir schon irgendwoher, dass der Durchschnitt von Äquivalenzrelationen wieder eine ist oder müssen wir das hier durcharbeiten? (Ich hab nach kurzer Suche im Skriptum nichts gefunden)
                \item Sei $\omega \in \Omega$ beliebig eine $n-stellige$ Operation, $n \in \N$ und seien entsprechend $(a_1, b_1), \dots, (a_n, b_n) \in \sim$ beliebig. Für alle $j \in J$ sind also $(a_1, b_1), \dots, (a_n, b_n) \in \sim_j$ und da $\sim_j$ schließlich eine Kongruenzrelation ist, gilt auch $(\omega(a_1, \dots, a_n), \omega(b_1, \dots, b_n)) \in \sim_j$. Da das für alle $j \in J$ der Fall ist, können wir $(\omega(a_1, \dots, a_n), \omega(b_1, \dots, b_n)) \in \sim$ schließen. 
            \end{enumerate}
            \item Sei $J 0\emptyset$.
            \begin{enumerate}[label = (\roman*)]
                \item Wissen wir schon irgendwoher, dass $A \times A$ eine Äquivalenzrelationen ist?
                \item Sei $\omega \in \Omega$ beliebig eine $n-stellige$ Operation, $n \in \N$ und seien entsprechend $(a_1, b_1), \dots, (a_n, b_n) \in \sim$ beliebig. Es ist $(\omega(a_1, \dots, a_n), \omega(b_1, \dots, b_n)) \in \sim$, weil $\omega: A^n \to A$ und $\sim = A \times A$.
            \end{enumerate}
        \end{enumerate}
        \item Folgt unmittlebar aus \ref{kongru} und Korollar 2.1.2.19. 
    \end{enumerate}
\end{solution}