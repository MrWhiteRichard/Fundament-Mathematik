\begin{exercise}
    Zeigen Sie, wie und warum die Halbgruppe $(\Z \setminus \{0\}, \cdot)$ isomorph ist zum direkten Produkt von $(\N^+, \cdot)$ und einer zweiten Halbgruppe. Welcher?
\end{exercise}
\begin{solution}
    Wir behaupten die zweite Halbgruppe ist mit $H := \{-1, 1\}$ gegeben durch $(H, \cdot)$, wobei $\cdot$ das normale Produkt ist. Direktes Produkt heißt nun für beliebige $(u, m), (v, n) \in H \times \N^+$ gilt $(u, m) \cdot (v, n) = (u \cdot v, m \cdot n)$. Der Isomophismus ist
    \begin{align*}
        \varphi: H \times \N^+ \to \Z \setminus \{0\}: (v, n) \mapsto 
        \begin{cases}
            -n &, v = -1 \\
            n &, v = 1
        \end{cases}
    \end{align*} 
    Es ist leicht einzusehen, dass $\varphi((u,m) \cdot (v, n)) = \varphi(u, m) \cdot \varphi(v, n)$. 

    Sei $\varphi(u, m) = \varphi(v, n)$. Es folgt $(u, m) = (v, n)$. 

    Sei $k \in \Z \setminus \{0\}$ beliebig. $\exists (v, n) \in H \times \N^+: \phi(v, n) = k$. 
\end{solution}