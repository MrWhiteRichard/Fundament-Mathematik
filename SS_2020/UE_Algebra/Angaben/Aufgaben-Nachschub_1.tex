Nach den vier noch vor Beginn der Ausnahmesituation ursprünglich für vorgestern bekanntgegebenen Übungsaufgaben folgen nun weitere. Insgesamt haben wir bis jetzt die folgenden 15 Aufgaben zur Vorbereitung durch Sie ausgewählt:

7, 14, 43, 51, 59, 65, 68, 80, 86, 96, 103, 110, 115, 118, 123

Wer sich gerne darüber hinaus mit Algebra beschäftigen möchte, dem empfehlen wir als lehrreiche und hoffentlich reizvolle Zusatzaufgaben zum Beispiel 57 (nicht sehr schwer) und 73 (anspruchsvoll).

Wann genau Abgabefrist für die Pflichtaufgaben sein wird und in welcher Form Sie diese Aufgaben präsentieren sollen bzw. können, hängt von den weiteren Entwicklungen ab. Orientieren Sie sich aber bitte an einem Wochenpensum von 6 Aufgaben, das ursprünglich 11mal zu leisten gewesen wäre. Insgesamt sollen es also 66 Aufgaben werden. Daran soll sich nichts ändern, auch wenn der Übungsmodus an die gegenwärtige Ausnahmesituation angepasst werden muss. Jedenfalls werden wir Ihnen noch vor Ostern weitere Aufgaben nennen, die zu bearbeiten sein werden.

Reinhard Winkler (auch für Martin Goldstern)
