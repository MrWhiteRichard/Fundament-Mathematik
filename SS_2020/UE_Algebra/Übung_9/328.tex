\begin{algebraUE}{328}
Sei $R$ ein Integritätsbereich. Der Polynomring $R[x]$ ist genau dann ein
Hauptidealring, wenn $R$ ein Körper ist. \\
\textit{Hinweis:} Betrachten Sie das von $a$ und $x$ erzeugte Ideal, wo $a \neq 0$
eine Nichteinheit von $R$ ist.
\end{algebraUE}
\begin{solution}
Sei zuerst $R$ ein Körper, $I \vartriangleleft R[x]$ ein beliebiges Ideal von $R[x]$.
Dann ist laut Proposition 5.2.3.3. $R[x]$ ein eukldischer Ring und nach Satz 5.2.3.4.
insbesondere ein Hauptidealring. \\
Sei also $R$ kein Körper. Dann existiert ein Element $a \neq 0$, das in $R$ kein
multiplikatives Inverses besitzt. Das erzeugte Ideal von $a$ und $x$ lässt sich darstellen als
\begin{align*}
  (a,x) = \{r_1a + r_2x: r_1,r_2 \in R[x]\}.
\end{align*}
Wir zeigen nun durch einen Widerspruch, dass $(a,x)$ kein Hauptideal ist (und somit $R[x]$ kein Hauptidealring).

Wir nehmen also an, es gibt ein $p \in R[x]$ mit $(a,x) = (p) = {rp: r \in R[x]}$. Man sieht sofort, $p$ vom Grad $0$ sein muss, da wir sonst keine konstanten Polynome erzeugen können (also auch nicht $a$).

Aufgrund von $(a,x)=(p)$ folgt:
\begin{align*}
  a \in (a,x) \Rightarrow \Exists r \in R: rp = a \\
  p \in (p) \Rightarrow \Exists r_1 \in R: r_1 a = p \\
  \Rightarrow a \sim p \\
  1\cdot x \in (a,x) \Rightarrow \Exists r \in R: (rx)p = 1\cdot x \Rightarrow rp = 1 \\
  \Rightarrow p \sim 1
\end{align*}
Insgesamt gilt also auch $a \sim 1$, im Widerspruch zu $a$ Nichteinheit.
\end{solution}
