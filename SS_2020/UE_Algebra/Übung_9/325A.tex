\begin{algebraUE}{325A}
Für $n \in \N$ sei $E_n := \{\frac{k}{2^n}: k \in \N\}$, und $E := \bigcup_{n \in \N}E_n$.
Mit der üblichen Addition sind diese Mengen Monoide. \\
Sei $K$ ein Körper. In 4.2.4. haben wir den Monoidring $K(E)$ definiert, als die
Menge aller formalen Summen $\sum_{e \in E}r_ee$, wobei die $r_e$ Elemente von $K$
sind, aber $\{e \in E: r_e \neq 0\}$ endlich ist. Um die Notation an die von den
Polynomen bekannte Notation anzugleichen, führen wir eine formale Variable
(oder ``Unbestimmte'') $x$ ein und ersetzen (wie in 4.2.4.4.) die Menge $E$
durch die Menge aller formalen Potenzen $x^e,e \in M$. Die Menge $\{x^e : e \in E\}$
trägt nun eine multiplikative Struktur: $x^e\cdot x^{e^{\prime}} := x^{e+e^{\prime}}$,
und die Elemente des Monoidrings $K(E)$ schreiben wir nun als endliche Summen
$\sum_{e \in E}r_e x^e$, die wie Polynome aussehen, in denen aber als Exponenten
nicht nur natürliche Zahlen erlaubt sind, sondern beliebige Elemente von $E$.
Addition und Multiplikation sind wie bei gewöhnlichen Polynomen definiert
(siehe 4.2.4.1.). Analog definieren wir $K(E_n)$. Zeigen Sie:
\begin{enumerate}
  \item Für alle $n \in \N$ in $K(E_n)$ ein Unterring von $K(E_{n+1})$ und von
  $K(E)$.
  \item $K(E) = \bigcup_{n \in \N}K(E_n)$
  \item Für alle $n \in \N$ gibt es einen Isomorphismus $\varphi_n: K[x] \to K(E_n)$. \\
  \textit{Hinweis:} Wähle $\varphi_{n+1}$ so, dass $\varphi_{n+1}(x^e) = \varphi_{n}(x^{2e})$
  für alle $e \in E_n$ gilt.
  \item Für einen beliebigen Ring $R$ und Elemente $p,q \in R$ schreiben wir $R \vDash p|q$
  für die Aussage ``$p$ teilt $q$ in $R$'', also: es gibt ein $r \in R: p\cdot r = q$. \\
  Für alle $n \in \N$ und alle $p,q \in K(E_n)$ gilt: $K(E_n) \vDash p|q$
  genau dann, wenn $K(E) \vDash p|q$.
  \item Die Einheiten von $K(E_n)$ und von $K(E)$ sind genau die konstanten Polynome,
  also genau die Bilder von konstanten Polynomen unter $\varphi_0$.
  \item Für alle $p \in K(E_n)$ gilt:
  \begin{align*}
    K(E) \vDash p \text{ ist prim } \iff \forall k \geq n: K(E_k) \vDash p \text{ ist prim}.
  \end{align*}
  \item Für alle $p \in K(E_n)$ gilt:
  \begin{align*}
    K(E) \vDash p \text{ ist irreduzibel } \iff \forall k \geq n: K(E_k) \vDash p \text{ ist irreduzibel}.
  \end{align*}
  \item Schließen Sie daraus, dass in $K(E)$ die irreduziblen Element genau die primen Elemente sind.
  \item Finden Sie eine echt absteigende Teilerkette in $K(E)$ und folgern Sie, dass $K(E)$
  kein faktorieller Ring ist.
\end{enumerate}
\end{algebraUE}
\begin{solution}
\leavevmode \\
\begin{enumerate}
  \item Es gilt klarerweise $E_n \subset E_{n+1} \subset E$. Daher können wir $K(E_n)$
  als Teilmenge von $K(E_{n+1})$ auffassen.
  \begin{align*}
    K(E_n) = \left\{\sum_{e \in E_n}r_ex^e: r_e \in K\right\}
  \end{align*}
  $K(E_n)$ enthält klarerweise Null- und Eins-Element.
  \begin{align*}
    \sum_{e \in E_n}r_ex^e + \sum_{e \in E_n}s_ex^e &= \sum_{e \in E_n}(r_e + s_e)x^e \in K(E_n) \\
    \left(\sum_{e \in E_n}r_ex^e\right)\cdot\left(\sum_{e \in E_n}s_ex^e\right) &=
    \sum_{e \in E_n}\sum_{\stackrel{(e_1,e_2) \in E_n^2}{e_1 + e_2 = e}}r_{e_1}s_{e_2}x^e \in K(E_n)
  \end{align*}
  Dabei gilt die letzte Gleichung, da $E_n$ unter der Addition abgeschlossen ist.
  Damit ist $K(E_n)$ auch unter den binären Operationen in $K(E_{n+1})$
  abgeschlossen und somit ein Unterring von $K(E_{n+1})$ und ebenso von $K(E)$.
  \item
  Sei $\sum_{e \in E}r_ex^e \in K(E)$ beliebig. Es sind nach Definition nur endlich
  viele $r_e \neq 0_K$, also existiert ein $N \in \N: \{e \in E: r_e \neq 0_K\} \subset E_N$
  und damit $\sum_{e \in E}r_ex^e = \sum_{e \in E_N}r_ex^e \in K(E_N)$. \\
  Umgekehrt sei $\sum_{e \in E_N}r_ex^e \in \bigcup_{n \in \N}K(E_n)$ beliebig.
  Dann definiere $r_e = 0$ für $e \in E\backslash E_N$ und es folgt
  \begin{align*}
    \sum_{e \in E_N}r_ex^e = \sum_{e \in E}r_ex^e \in K(E).
  \end{align*}
  \item $E_0 = \N$, also müssen wir zuerst einen Isomorphismus $\varphi_0: K(\N) \to K[x]$
  finden.
  \begin{align*}
    \varphi_0: \begin{cases}
      K(\N) &\to K[x] \\
      \sum_{i = 0}^nk_ix^i &\mapsto \sum_{i = 0}^nk_ix^i
    \end{cases}
  \end{align*}
  Induktiv fahren wir fort: Haben wir den Isomorphismus $\varphi_n$ bereits gefunden,
  dann definiere
  \begin{align*}
    \varphi_{n+1}(x^e) = \varphi_n(x^{2e}), \qquad e \in E_{n+1}.
  \end{align*}
  $\varphi_{n+1}$ ist aufgrund der Homomorphie-Bedingung dadurch bereits eindeutig
  festgelegt ($K(E_{n+1})$ wird von $(x^e)_{e \in E_{n+1}}$ erzeugt). Weiters ist die Festlegung wohldefiniert, da aus $e \in E_{n+1}$
bereits $2e \in E_n$ folgt. Die Injektivität überträgt sich von $\varphi_n$ und da zu jedem $e \in E_n: \frac{e}{2} \in E_{n+1}$
überträgt sich auch die Surjektivität.
  \begin{align*}
    \varphi_{n+1}(x^{e_1}x^{e_2}) = \varphi_{n+1}(x^{e_1 + e_2}) = \varphi_n(x^{2e_1 + 2e_2})
    = \varphi_n(x^{2e_1})\varphi_n(x^{2e_2}) = \varphi_{n+1}(x^{e_1})\varphi_{n+1}(x^{e_2})
  \end{align*}
  \item Seien $n \in \N, p,q \in K(E_n)$ beliebig fest.
  Sei zuerst $K(E_n) \vDash p|q$, also existiert $r \in K(E_n) \subset K(E): rp = q$.
  Damit folgt $ K(E) \vDash p|q$. \\
  Gelte umgekehrt $K(E) \vDash p | q$. Sei O.B.d.A. $q \neq 0$.
  Dann existiert ein $r \in K(E): rp = q $.
  Es gilt sogar $\exists r \in K(E_n): rp = q$, da anderenfalls
  \begin{align*}
    \exists e^* := \min \{e \in E\backslash E_n: r_e \neq 0_K\}
  \end{align*}
  und aufgrund $p \neq 0$
  \begin{align*}
    \exists e^{\prime} := \min \{e \in E_n: p_e \neq 0_K\} \text{ und }
    \widetilde{e} := e^* + e^{\prime} \notin E_n.
  \end{align*}
  Damit folgt
  \begin{align*}
    q_{e^* + e^{\prime}} = \sum_{\stackrel{(e_1,e_2) \in E_n \times E}{e_1 + e_2 = \widetilde{e}}}r_{e_1}p_{e_2}x^{\widetilde{e}}
    = \underbrace{r_{e^*}p_{e^{\prime}}}_{\neq 0}x^{\widetilde{e}}
  \end{align*}
  und daher $q \notin K(E_n)$. Widerspruch!
  \item Sei $p = \varphi_0^{-1}(q) \in K(E_0) \subset K(E)$ mit $q \in K[x]: \grad(q) = 0$.
  Wir wissen aus Proposition 3.3.6.5, dass $q$ in $K[x]$ eine Inverse besitzt, genau
  dann wenn $\grad(q) = 0$.
  \begin{align*}
    \varphi_0^{-1}(q)\varphi_0^{-1}(q^{-1}) = \varphi_0^{-1}(1) = 1.
  \end{align*}
  Sei umgekehrt $p := \sum_{e \in E}p_ex^e \in K(E)$ eine beliebige Einheit
  mit $p^{-1} = \sum_{e \in E}q_ex^e$. Es gibt also ein $n \in \N: p,p^{-1} \in K(E_n)$ und es gilt
  \begin{align*}
    1 = \varphi_n(1) = \varphi_n(pp^{-1}) = \varphi_n(p)\varphi_n(p^{-1})
    = \left(\varphi_0(\sum_{e \in E}p_ex^{2^ne})\right)\left(\varphi_0(\sum_{e \in E}q_ex^{2^ne})\right)
  \end{align*}
  Wiederum ist das nur möglich, wenn $\grad(\varphi_0(\sum_{e \in E}p_ex^{2^ne})) = 0$
  und $p$ ist daher Bild eines konstanten Polynoms unter $\varphi_0^{-1}$.
  \item Seien $n \in \N, p \in K(E_n)$ beliebig fest. \\
  Gelte zuerst $K(E) \vDash p$ ist prim. $p$ ist also keine Einheit in $K(E)$
  und $\neq 0$. Weiters gilt für alle $a,b \in K(E): p|ab \implies p|a \lor p|b$.
  Die Einheiten in $K(E_k)$ sind genau die Einheiten in $K(E)$ und die Implikation
  gilt klarerweise auch, wenn wir sie auf $K(E_k)$ einschränken. \\
  Gelte umgekehrt $\forall k \geq n; K(E_k) \vDash p$ ist prim.
  Wieder ist klarerweise $p$ auch in $K(E)$ keine Einheit.
  Seien $a,b \in K(E)$ mit $p|ab$ beliebig. Es existiert ein $k \in N: a,b \in K(E_k)$
  und damit $p|a \lor p|b$. Also gilt auch $K(E) \vDash p$ ist prim.
  \item Seien $n \in \N, p \in K(E_n)$ beliebig fest. \\
  Gelte zuerst $K(E) \vDash p$ ist irreduzibel, also
  $\forall a,b \in K(E): p = ab \implies a \sim 1 \lor b \sim 1$.
  Die Implikation gilt sicher auch in jedem $K(E_k)$. \\
  Umgekehrt folgt aus
  \begin{align*}
    \forall k \geq n: \forall a,b \in K(E_k): p = ab \implies a \sim 1 \lor b \sim 1
  \end{align*}
  für $a,b \in K(E): \exists k \in \N: a,b \in K(E_k) \implies a \sim 1 \lor b \sim 1$.
  \item $\forall n \in \N: K(E_n)$ ist isomorph zum faktoriellen Ring $K[x]$ und damit ebenso faktoriell. Daher sind
  die irreduziblen Elemente in $K(E_n)$ genau die primen Elemente und es gilt:
  \begin{align*}
    K(E) \vDash p \text{ ist prim } &\iff \forall k \geq n: K(E_k) \vDash p \text{ ist prim} \\
    \iff \forall k \geq n: K(E_k) \vDash p \text{ ist irreduzibel}
    &\iff K(E) \vDash p \text{ ist irreduzibel}.
  \end{align*}
  \item $(x^{\nicefrac{1}{2n}})_{n \in \N}$ und Satz 5.2.1.7.
\end{enumerate}
\end{solution}
