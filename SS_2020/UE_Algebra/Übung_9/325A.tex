\begin{algebraUE}{325A}
Für $n \in \N$ sei $E_n := \{\frac{k}{2^n}: k \in \N\}$, und $E := \bigcup_{n \in \N}E_n$.
Mit der üblichen Addition sind diese Mengen Monoide. \\
Sei $K$ ein Körper. In 4.2.4. haben wir den Monoidring $K(E)$ definiert, als die
Menge aller formalen Summen $\sum_{e \in E}r_ee$, wobei die $r_e$ Elemente von $K$
sind, aber $\{e \in E: r_e \neq 0\}$ endlich ist. Um die Notation an die von den
Polynomen bekannte Notation anzugleichen, führen wir eine formale Variable
(oder ``Unbestimmte'') $x$ ein und ersetzen (wie in 4.2.4.4.) die Menge $E$
durch die Menge aller formalen Potenzen $x^e,e \in M$. Die Menge $\{x^e : e \in E\}$
trägt nun eine multiplikative Struktur: $x^e\cdot x^{e^{\prime}} := x^{e+e^{\prime}}$,
und die Elemente des Monoidrings $K(E)$ schreiben wir nun als endliche Summen
$\sum_{e \in E}r_e x^e$, die wie Polynome aussehen, in denen aber als Exponenten
nicht nur natürliche Zahlen erlaubt sind, sondern beliebige Elemente von $E$.
Addition und Multiplikation sind wie bei gewöhnlichen Polynomen definiert
(siehe 4.2.4.1.). Analog definieren wir $K(E_n)$. Zeigen Sie:
\begin{enumerate}
  \item Für alle $n \in \N$ in $K(E_n)$ ein Unterring von $K(E_{n+1})$, und von
  $K(E)$, und $K(E) = \bigcup_{n \in \N}K(E_n)$.
  \item Für alle $n \in \N$ gibt es einen Isomorphismus $\varphi_n: K[x] \to K(E_n)$. \\
  \textit{Hinweis:} Wähle $\varphi_{n+1}$ so, dass $\varphi_{n+1}(x^e) = \varphi_{n}(x^{2e})$
  für alle $e \in E_n$ gilt.
  \item Für einen beliebigen Ring $R$ und Elemente $p,q \in R$ schreiben wir $R \vDash p|q$
  für die Aussage ``$p$ teilt $q$ in $R$'', also: es gibt ein $r \in R: p\cdot r = q$. \\
  Für alle $n \in \N$ und alle $p,q \in K(E_n)$ gilt: $K(E_n) \vDash p|q$
  genau dann, wenn $K(E) \vDash p|q$.
  \item Die Einheiten von $K(E_n)$ und von $K(E)$ sind genau die konstanten Polynome,
  also genau die Bilder von konstanten Polynomen unter $\varphi_0$.
  \item Für alle $p \in K(E_n)$ gilt:
  \begin{align*}
    K(E) \vDash p \text{ ist prim } \iff \forall k \geq n: K(E_k) \vDash p \text{ ist prim}.
  \end{align*}
  \item Für alle $p \in K(E_n)$ gilt:
  \begin{align*}
    K(E) \vDash p \text{ ist irreduzibel } \iff \forall k \geq n: K(E_k) \vDash p \text{ ist irreduzibel}.
  \end{align*}
\end{enumerate}
Schließen Sie daraus, dass in $K(E)$ die irreduziblen Element genau die primen Elemente sind.
Finden Sie eine echt absteigende Teilerkette in $K(E)$ und folgern Sie, dass $K(E)$
kein faktorieller Ring ist.
\end{algebraUE}
\begin{solution}
\leavevmode \\
\begin{enumerate}
  \item
\end{enumerate}
\end{solution}
