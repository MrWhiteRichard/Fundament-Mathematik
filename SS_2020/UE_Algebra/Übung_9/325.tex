\begin{algebraUE}{325}
Zeigen Sie, dass für einen Integritätsbereich Zerlegbarkeit in
irreduzible Elemente im Allgemeinen nicht eindeutige Zerlegbarkeit in
irreduzible Elemente impliziert.
\end{algebraUE}
\begin{solution}
Wir betrachten wieder den Integritätsbereich $\Z[\sqrt{-5}]$.
Zuerst müssen wir zeigen, dass hier Zerlegbarkeit in irreduzible Elemente gilt.
Dafür zeigen wir, dass $\Z[\sqrt{-5}]$ die Teilerkettenbedingung erfüllt.
Sei also $(r_n)_{n \in \N}$ eine Folge aus $\Z[\sqrt{-5}]$ mit $r_{n+1} | r_n$
für alle $n \in \N$. Bezeichne zusätzlich mit $(a_n)_{n \in N}$ die Folge aus $\Z[\sqrt{-5}]$
mit $r_{n+1}a_n = r_n$.
Betrachte $N(r_0) = k \in \Z$. Wenn $r_{n+1}$ ein echter Teiler
von $r_n$ ist, gilt $r_{n+1}a_n = r_n$, wobei $a \in \Z[\sqrt{-5}]\backslash \Z[\sqrt{-5}]^*$ und es folgt
\begin{align*}
  N(r_n) = N(r_{n+1}a_n) = N(r_{n+1})N(a_n) > N(r_{n+1}).
\end{align*}
Da wir wissen, dass $\forall r \in \Z[\sqrt{-5}]: N(r) \geq 1$ muss es also ein $n_0 \in \N$
geben, ab dem die Folge $N(r_n)_{n \geq n_0}$ konstant bleibt.
Aus $N(r_n) = N(r_{n+1})$ folgt weiters, dass $N(a_n) = 1$ und daher $a_n$
eine Einheit sein muss. Daher gilt $\forall n \geq n_0: r_n \sim r_{n+1}$. \\
$\Z[\sqrt{-5}]$ erfüllt also die Teilerkettenbedingung und damit gilt Zerlegbarkeit
in irreduzible Elemente. In der Aufgabe davor haben wir gesehen, dass nicht jedes
irreduzible Element prim ist und mit Satz 5.2.1.7. kann hiermit kein faktorieller
Ring vorliegen, also keine eindeutige Zerlegbarkeit in irreduzible Elemente.
\end{solution}
