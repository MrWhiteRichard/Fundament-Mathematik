\begin{exercise}
    Zeigen Sie, dass für alle natürlichen Zahlen $k,l,j$ 
    \begin{itemize}
        \item[(1)] $(k^j)^l = k^{j \cdot l}$ 
        \item[(2)] $(k^j) \cdot (k^l) = k^{j + l}$
    \end{itemize}
    gilt.
\end{exercise}

\begin{solution}
    Wir definieren die Funktion 
    \begin{align*}
        b: K^{J \times L} \to \pbraces{K^J}^L : f \mapsto \varphi: 
        \begin{cases}
            L \to K^J \\
            l \mapsto \psi: 
            \begin{cases}
                J \to K \\
                j \mapsto f\pbraces{(j,l)}
            \end{cases}
        \end{cases}  
    \end{align*}
    und behaupten, dass diese bijektiv ist.

    Um die Injektivität zu zeigen wählen wir also $f,g \in K^{J \times L}$ mit $b(f) = b(g)$. Sei weiters $(j,l) \in J \times L$ beliebig. Dann gilt 
    \begin{align*}
        f(j,l) = \pbraces{ \pbraces{b(f) }(l) }(j) = \pbraces{ \pbraces{b(g) }(l) }(j) = g(j,l)
    \end{align*}
    und damit folgt $f = g$, also die Injektivität.

    Um die Surjektivität zu zeigen wählen wir $\varphi \in \pbraces{K^J}^L$ beliebig und definieren 
    \begin{align*}
        f \in K^{J \times L}: (j, l) \mapsto \pbraces{\varphi(l)}(j).
    \end{align*}
    Damit gilt für beliebiges $(j,l) \in J \times L$, dass
    \begin{align*}
        \pbraces{\pbraces{b(f) }(l) }(j) = f\pbraces{(j,l)} = \pbraces{\varphi(l)}(j)
    \end{align*}
    und damit gilt $b(f) = \varphi$ also ist die Surjektivität von $b$ gezeigt.
    
    Nun ist noch der zweite Teil zu beweisen. Dafür definieren wir die Funktion
    \begin{align*}
        b: \pbraces{K^J} \times \pbraces{K^L} \to K^{J \cup L}: (f,g) \mapsto \varphi:
        \begin{cases}
            J \cup L \to K \\
            \alpha \mapsto 
            \begin{cases}
                f(\alpha) &, \alpha \in J \\
                g(\alpha) &, \alpha \in L
            \end{cases}
        \end{cases} .
    \end{align*}
    Die Funktion ist wohldefiniert, da $L$ und $J$ disjunkt sind. Nun wollen wir wieder die Bijektivität beweisen.

    Für die Injektivität seien $(f,g), (u,v) \in \pbraces{K^J} \times \pbraces{K^L}$ und $b\pbraces{(f,g)} = b\pbraces{(u,v)}$. Weiters sei $\alpha \in J \cup L$, wobei wir ohne Beschränkung der Allgemeinheit $\alpha \in J$ annehmen können. Es gilt 
    \begin{align*}
        f(\alpha) = \pbraces{b\pbraces{(f,g)}}(\alpha) = \pbraces{b\pbraces{(u,v)}}(\alpha) = u(\alpha)
    \end{align*}
    Damit folgt also $(f,g) = (u,v)$ und damit die Injektivität.

    Um die Surjektivität von $b$ zu zeigen wählen wir $\varphi \in K^{J \cup L}$ beliebig und definieren $(f,g) \in \pbraces{K^J} \times \pbraces{K^L}$ mit
    \begin{align*}
        f: x \mapsto \varphi(x) \\
        g: x \mapsto \varphi(x) 
    \end{align*} 
    Es gilt dann für beliebiges $\alpha \in J \cup L$
    \begin{align*}
        \pbraces{b\pbraces{(f,g)}}(\alpha) = \varphi(\alpha)
    \end{align*}
    und damit $b\pbraces{(f,g)} = \varphi$, also folgt die Surjektivität.
\end{solution}