\begin{algebraUE}{334}

Beweisen Sie Proposition 5.2.3.10. (Proposition muss noch eingefügt werden!)

\end{algebraUE}

\begin{solution}

Die euklidische Bewertung

\begin{align*}
  H(a+ib) = a^2 + b^2
\end{align*}

ist sicher eine Abbildung von $\Z [i] \setminus \{0\} \to \N$.

Wir müssen noch zeigen, dass Division mit Rest möglich ist, Also

\begin{align*}
  \Forall a \in \Z [i] \setminus \{0\}, b \in \Z [i] \Exists q,r \in \Z [i]: b = aq + r \text{~mit~} H(r) < H(a).
\end{align*}

Seien also $a \in \Z [i] \setminus \{0\}, b \in \Z [i]$ beliebig, dann gilt mit der Definition

\begin{align}\label{def}
  c_r + i c_i := \frac{b}{a} = \frac{b\overline{a}}{a\overline{a}},
\end{align}

dass $c_r, c_i \in \Q$. Wir können in jedem Fall ganze Zahlen $q_r, q_i$ finden mit $|c_r-q_r| \leq \frac{1}{2}$ und $|c_i-q_i| \leq \frac{1}{2}$.

Wir schreiben ein wenig um

\begin{align*}
  c_r + i c_i = \underbrace{q_r + i q_i}_{\in \Z [i]} + (c_r- q_r) + i (c_i - q_i)
\end{align*}

Multiplizieren wir \eqref{def} mit $a$, so erhalten wir

\begin{align*}
  b = a(q_r + i q_i) + a((c_r- q_r) + i (c_i - q_i))
\end{align*}

wobei der erste Summand in $\Z [i]$ ist, und somit auch der zweite. Der zweite Summand ist also ein Kandidat für das $r$ aus der Definition. Prüfen wir die Bedingung noch nach.

Zunächst gilt

\begin{align*}
  H((c_r- q_r) + i (c_i - q_i)) = (c_r- q_r)^2 + (c_i - q_i)^2 \leq \frac{1}{2}^2 + \frac{1}{2}^2 < 1.
\end{align*}

Mit der Multiplikativität des Betrages gilt Also

\begin{align*}
  H(a(c_r- q_r) + i (c_i - q_i)) = H(a) \cdot H((c_r- q_r) + i (c_i - q_i)) < H(a)
\end{align*}


\end{solution}
