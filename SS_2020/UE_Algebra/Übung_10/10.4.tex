\begin{algebraUE}{342}
Sei $R$ ein faktorieller Ring. Ist $f = \sum_{i=0}^na_ix^i \in R[x]$ mit Grad $\geq 1$
ein primitives Polynom und $p \in R$ irreduzibel mit
\begin{align*}
  p \nmid a_n, p | a_i, \text{ für } i = 0,\dots,n-1, \text{ und } p^2 \nmid a_0,
\end{align*}
dann ist $f$ irreduzibel in $R[x]$.
\end{algebraUE}

\begin{solution}
Angenommen $f$ wäre nicht irreduzibel, also existieren Nicht-Einheiten $q,r \in R[x]$ mit $f = qr$.
Da $f$ ein primitives Polynom ist gilt $\ggT\{a_i: i = 1,\dots,n\} = 1_R$ und $q,r$
müssen ebenso primitiv sein. Es gilt also
\begin{align*}
  f = \sum_{i=0}^na_ix^i = \sum_{i = 0}^n \sum_{k= 0}^i q_kr_{i-k}x^i.
\end{align*}
Durch Koeffizientenvergleich erhalten wir
\begin{align*}
  a_0 &= q_0r_0 \\
  a_1 &= q_0r_1 + q_1r_0 \\
  &\vdots \\
  a_n &= q_0r_n + \dots + q_nr_0.
\end{align*}
Da $p$ ein irreduzibles Element eines faktoriellen Ringes ist, ist $p$ prim und es folgt
aus $p| a_0 = q_0r_0$, dass $p|q_0 \lor p|r_0$. Da zusätzlich $p^2 \nmid a_0$ gilt,
kann $p$ nur genau einen der Faktoren $q_0,r_0$ teilen. Gelte also o.B.d.A. $p | q_0, p \nmid r_0$.
Jetzt gehen wir induktiv vor und zeigen $p | q_k, k = 0,\dots,n$. Der Anfang ist bereits getan,
gelte nun $p | q_{i}, i = 0,\dots,k-1$:
\begin{align*}
  p | a_k = q_kr_n + \dots + q_{k-1}r_1 + q_kr_0.
\end{align*}
Mit der Induktionsvoraussetzung erhalten wir $p | q_kr_n + \dots + q_{k-1}r_1$
und daraus folgt $p | q_kr_0$. Da $p \nmid r_0$ gilt damit schon $p | q_k$. \\
Also können wir $p$ aus $q$ herausheben und $q$ kann nicht primitiv sein. Widerspruch!

Alternativ: (erster Teil fast identisch)

Angenommen $f$ wäre nicht irreduzibel, also existieren Nicht-Einheiten $q,r \in R[x]$ mit $f = qr$.
Es gilt
\begin{align*}
  f = \sum_{i=0}^na_ix^i = \sum_{i = 0}^n \sum_{k= 0}^i q_kr_{i-k}x^i.
\end{align*}
Durch Koeffizientenvergleich erhalten wir
\begin{align*}
  a_0 &= q_0r_0 \\
  a_1 &= q_0r_1 + q_1r_0 \\
  &\vdots \\
  a_n &= q_0r_n + \dots + q_nr_0.
\end{align*}
Da $p$ ein irreduzibles Element eines faktoriellen Ringes ist, ist $p$ prim und es folgt
aus $p| a_0 = q_0r_0$, dass $p|q_0 \lor p|r_0$. Da zusätzlich $p^2 \nmid a_0$ gilt,
kann $p$ nur genau einen der Faktoren $q_0,r_0$ teilen. Gelte also o.B.d.A. $p | q_0, p \nmid r_0$.

Nun können sicher nicht alle Koeffizienten $q_0,...,q_n$ durch $p$ teilbar sein, da dies sonst auch $a_n$ wäre. Es gibt also ein minimales $s$, sodass $q_s$ nicht durch $p$ teilbar ist mit

\begin{align*}
  s \leq grad(q) < grad(f)
\end{align*}

da $r$ einen Grad größer $0$ hat.

Nun gilt aber für $a_s$

\begin{align*}
  a_s = q_0 r_s + \dots + q_{s-1}r_1 + q_s r_0
\end{align*}

Nach Voraussetzung sind alle bis auf den letzten Summanden durch $p$ teilbar (da $p \nmid q_s \land p \nmid r_0$). Das ist aber ein Widerspruch zu $p|a_s$.

\end{solution}
