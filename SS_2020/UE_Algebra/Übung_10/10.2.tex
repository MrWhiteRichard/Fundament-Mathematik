\begin{algebraUE}{335}

Begründen Sie folgende Aussagen über den (nach der vorherigen Aufgabe) euklidischen Ring $\Z [i]$.

\begin{enumerate}
  \item Für die Einheitengruppe von $\Z [i]$ gilt $E(\Z [i]) = \{1,-1,i,-i\}$.

  \item Ist $p$ prim in $\Z [i]$ und eine natürliche Zahl, so auch eine Primzahl.

  \item Die Umkehrung gilt nicht: Es gibt Primzahlen, die nicht prim in $\Z [i]$ sind.

  \item Lässt sich $p = a^2 + b^2 = (a+ib)(a-ib) \in \P$ als Summe zweier Quadrate positiver ganzer Zahlen $a,b$ darstellen, so sind die Faktoren $a+ib$ und $a-ib$ prim in $\Z [i]$.

  \item Man bestimmt alle primen Elemente $z \in \Z[i]$ mit $|z|^2 \leq 10$.

  \item Man bestimme in $\Z [i]$ die Primfaktorzerlegungen von $27+6i$ und $-3+4i$.

  \item Man bestimme in $\Z [i]$ einen ggT der Elemente $a = 7+i$ und $b=5$ und stelle ihn in der Form $ax + by$ mit $x,y \in \Z [i]$ dar.

\end{enumerate}

\end{algebraUE}

\begin{solution}

\begin{enumerate}
  \item Siehe Hinweis Aufgabe 324. Dazu bemerke, dass nur die angegebenen Elemente Norm $= 1$ haben.

  \item Zunächst einemal können wir im folgenden äquivalent von primen und irreduziblen Elementen sprechen (Euklidischer Ring $\Rightarrow$ Faktorieller Ring).

  Wäre $p$ keine Primzahl, dann hätte es zwei Teiler $a,b \in \N$ ungleich $1$ und $p$. Insbesondere sind $a,b \in \Z [i]$ und keine Einheiten (vorherige Aufgabe), also $p$ nicht irreduzibel.

  \item Betrachte $p = 5 = (2+i)(2-i)$. Dieses ist als Primzahl jedoch nicht irreduzibel in $\Z [i]$.

  \item Sei $p = (a+ib)(a-ib) \in \P$ beliebig und angenommen $a+ib$ nicht irreduzibel (prim).

  \begin{align*}
    \Rightarrow \Exists c,d \in \Z [i] \setminus E(\Z [i]):  cd = a+ib \Rightarrow a - ib = \overline(a+ib) = \overline(cd) = \overline(c) \overline(d)
  \end{align*}

  Dann würde folgen

  \begin{align*}
    p = a^2 + b^2 = (a+ib)(a-ib) = cd\overline(c) \overline(d) = c\overline(c)d \overline(d) = H(c)H(d)
  \end{align*}

  welche beide natürliche Zahlen $\neq 1$ sind im Widerspruch zu $p$ Primzahl.

  \item Wir wollen zunächst einen Zusammenhang zwischen gewissen primen Elementen herstellen.

  Mit $E:= E(\Z [i])$ überlegt man sich sehr leicht, dass für ein beliebiges $p$ prim in $\Z [i]$ auch jedes Element aus $pE$ prim ist. (genau dann)

  Mit ähnlichen Überlegungen und der Tatsache, dass die Konjugierten von unseren Einheiten wieder Einheiten sein müssen, sieht man auch, dass genau $\overline{p}$ prim ist.

  Insgesamt ist für $p$ prim also die gesamte Menge
  \begin{align*}
    P_p := pE \cup \overline{pE}
  \end{align*}
  prim. (Diese Menge muss nicht immer 8-elementig sein, sondern kann für $p$ auf den Achsen oder Diagonalen auch nur 4-elementig sein).

  Nun zum eigentlichen Teil der Aufgabe. Wie gerade gesehen, müssen wir nicht wirklich zwischen $a$ und $b$ unterscheiden, können uns also einfach anschauen, welche zwei Quadratzahlen in Summe weniger gleich 10 ergeben.

  \begin{align*}
    0+0, 1+0, 4+0, 9+0, 1+1, 4+1, 9+1, 4+4
  \end{align*}

  kommen in Frage. Die ersten beiden entsprechen der Null und den Einheiten, können also femäß Definition nicht prim sein.

  Der dritte Fall ist z.B. $p = 2 = (1+i)(1-i)$, also nicht irreduzibel.

  Der vierte Fall ist z.B. $p = 3$. Für eine beliebige Zerlegung von $3 = cd$ gilt $9 = H(3) = H(cd) = H(c)H(d)$, im nichttrivialen Fall also $H(c)=H(d) = 3$. Das ist allerdings offensichtlich nicht im Wertebereich der Normfunktion. Also ist $P_3 $ prim.

  Den Fall $(1+1) = 2$ und $(4+1) = 5$ behandeln wir gleich gemeinsam, da nach Aufgabenteil (4) z.B. $1+i$ und $2+i$ prim sind.

  Für den vorletzten Fall gilt z.B. $p = 3+i = (2-i)(1+i)$.

  Für den letzten Fall gilt z.B. $p = 2+2i = 2(1+i)$.

  Insgesamt erhalten wir für die gesuchte Menge also

  \begin{align*}
    P_3 \cup P_{1+i} \cup P_{2+i}
  \end{align*}

  \item Die Primfaktorzerlegungen kann man sich mit dem Ansatz

  \begin{align*}
    H(27+6i) = 765 = 17 \cdot 5 \cdot 9 \\
    H(-3+4i) = 25 = 5 \cdot 5
  \end{align*}

  leicht überlegen.

  \begin{align*}
    (27+6i) = (4-i)(6+3i) = (4-i)(2+i)3 \\
    (-3+4i) = (1+2i)(1+2i)
  \end{align*}

  Wir wissen von (fast) allen Elementen, dass sie prim sind, bei $4-i$ geht wieder das Argument aus (4).

  \item Zur Bestimmung wenden wir den Euklid-Algorithmus an.

  Im ersten Schritt und zweiten Schritt gilt

  \begin{align*}
    (7+i) = 5 q_1 + r_1 = 5 \cdot 1 + (2+i) \\
    5 = (2+i)q_2 + r_2 = (2+i)(2-i) + 0
  \end{align*}

  Wir terminieren also schon nach zwei Schritten und erhalten als ggT $r_1 = 2+i = (7+i) \cdot 1 + 5 \cdot (-1)$.


\end{enumerate}

\end{solution}
