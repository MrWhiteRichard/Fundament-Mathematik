\begin{algebraUE}{141}
In dieser Übungsaufgabe interessieren wir uns für Unteralgebren und Kongruenzrelationen
auf $\mathbb{N}$ bezüglich additiver und/oder multiplikativer Struktur. Versuchen
Sie jeweils alle Objekte der angegebenen Art zu beschreiben. Wenn Ihnen das zu
schwierig erscheint (was in der Mehrzahl der Fälle wahrscheinlich ist), ermitteln
Sie, wieviele es davon gibt. Unterscheiden Sie dabei verschiedene unendliche
Kardinalitäten, insbesondere $|\mathbb{N}|$ und $|\mathbb{R}|$.
\begin{itemize}
  \item [1.] Unteralgebren von $(\mathbb{N},+,0)$
  \item [2.] Kongruenzrelationen von $(\mathbb{N},+,0)$
  \item [3.] Unteralgebren von $(\mathbb{N},\cdot,1)$
  \item [4.] Kongruenzrelationen von $(\mathbb{N},\cdot,1)$
  \item [5.] Unteralgebren von $(\mathbb{N},+,0,\cdot,1)$
  \item [6.] Kongruenzrelationen von $(\mathbb{N},+,0,\cdot,1)$
\end{itemize}
\end{algebraUE}
\begin{solution}
\leavevmode \\
\begin{itemize}
  \item [1.] Bezeichne mit $\mathcal{U}$ die Menge aller Unteralgebren
  von $(\mathbb{N},+,0)$. \\
  Einige Beispiele an Unteralgebren lauten
  \begin{itemize}
    \item Die beiden trivialen Unteralgebren: $\{0\}$ und $\mathbb{N}$.
    \item Für jede natürliche Zahl $n \geq 2: n\cdot\mathbb{N}$.
    \item Für jede natürliche Zahl $n \geq 1: \mathbb{N}~\backslash\{1,\dots,n\}$.
  \end{itemize}
Im allgemeinen können wir zu jeder beliebiger Teilmenge $X \subset \mathbb{N}$
die erzeugte Unteralgebra mit $\{\sum_{j=1}^nx_jn_j: n, n_j \in \mathbb{N}, x_j \in X \}$
bestimmen. Also haben wir eine Abbildung
\begin{align*}
  \varphi: \begin{cases}
    2^\N \rightarrow \mathcal{U} \\
    X \mapsto \{\sum_{j=1}^nx_jn_j: n, n_j \in \mathbb{N}, x_j \in X \}
  \end{cases}
\end{align*}
gefunden, welche $2^\N$ surjektiv auf $\mathcal{U}$ abbildet.
Damit haben wir ein System um alle Unteralgebren vollständig zu beschreiben.
Natürlich werden wir damit öfters die gleichen Unteralgebren erzeugen,
allerdings haben wir mit $|2^{\mathbb{N}}| = |\mathbb{R}|$ zumindest eine obere
Schranke für die Kardinalität der Unteralgebren gefunden. Bleibt noch die Frage,
ob wir die Schranke noch auf abzählbar unendlich herunterschrauben können. \\
Dafür zeigen wir zunächst, dass jede Unteralgebra endlich erzeugt ist.
Sei $U \in \mathcal{U}$ beliebig und sei angenommen,
dass $U$ nicht endlich erzeugt ist.
Dazu definieren wir uns die Folge $(a_n)_{n \in \mathbb{N}}$ induktiv wie folgt
\begin{align*}
  a_0 := \min U \\
a_{n+1} := \min U \backslash \langle a_0,\dots,a_n\rangle
\end{align*}
Betrachte nun die endliche Menge $(a_0,\dots,a_m)$ mit einem $m > a_0$.
Nun folgt mit dem Schubfachprinzip $\exists k,n \leq m: a_k \equiv a_n (\mod a_0)$.
Wir nehmen o.B.d.A. an $a_k < a_n$. Dann gilt also
\begin{align*}
  \exists c \in \mathbb{N}: a_n = a_k + ca_0 \in \langle a_0,\dots,a_m \rangle
\end{align*}
Dies ist ein Widerspruch zu unserer Konstruktion und $U$ muss somit endlich erzeugt sein.
Also können wir $\mathcal{U}$ als Bild einer nicht notwendigerweise injektiven
Abbildung von der Menge $\mathcal{E}(\N)$ aller endlichen Teilmengen von $\N$ darstellen.
Die Menge $\mathcal{E}(\N)$ können wir als
abzählbare Vereinigung endlicher Mengensysteme anschreiben
\begin{align*}
  \mathcal{E}(\N) = \bigcup_{n \in \N} \{A \subset 2^\N: |A| = n \}
\end{align*}
und ist damit ebenfalls abzählbar. Also gilt $|\mathcal{U}| \leq |\N|$.
\item [2.] Wieder haben wir zwei triviale Kongruenzrelationen am Anfang:
Die Allrelation $\mathbb{N} \times \mathbb{N}$ und die Identitätsrelation
$\{(n,n): n \in \mathbb{N}\}$. \\

Sei nun $\sim$ eine beliebige Kongruenzrelation ungleich der Identitätsrelation.
Dann gibt es ein kleinstes $a_{0}$, das mit mindestens einem anderen Element in Relation steht. In der Menge aller $b$, die mit $a_{0}$ in
Relation stehen, gibt es wieder ein kleinstes Element, das wir $b_{0}$ nennen. Da $\sim$ eine Kongruenzrelation bezüglich $+$ ist und $1
\sim 1$, gilt (mit $m := b_{0}-a_{0} > 0$)
\begin{align*}
  a_{0} \sim b_{0} \Rightarrow a_{0}+1 \sim b_{0}+1 \Rightarrow \dots \Rightarrow a_{0}+(m-1) \sim b_{0}+(m-1) \Rightarrow b_{0} \sim
  b_{0}+m \Rightarrow \dots
\end{align*}
Zusammengefasst gilt also
\begin{align*}
  \forall n \geq a_0: n \sim n + m
\end{align*}
Also ist $\sim$ eine Kongruenzrelation mit $a_{0}$ einelementigen Äquivalenzklassen und maximal $m$ weiteren Quasi-Modulo-Äquivalenzklassen
(beginnend ab $a_{0}$).
Für jedes $n \in \N,a_0 \leq n < b_0$ ist also
\begin{align*}
  [n]_{\sim} \supseteq \{n + km: k \in \N\}.
\end{align*}
Angenomenn es gäbe zusätzlich $a_1 \sim a_1 + m^{\prime}$ mit $m^{\prime} \neq m \in \N$. \\
\begin{itemize}
\item Fall 1: $a_1 < a_0$:
Widerspruch zur Annahme, dass $a_0$ das kleinste Element ist, welches nicht nur
mit sich selbst in Relation steht.
  \item Fall 2: $a_1 \geq a_0, m^{\prime} < m$:
  Es existiert ein kleinstes $k \in \N$, sodass $a_0 + km \geq a_1$.
  Es folgt $a_0 + km \sim a_0 + km + m^{\prime} \sim a_0 + m^{\prime}$
  im Widerspruch dazu, dass $b_0$ das kleinste Element ungleich $a_0$ ist, welches mit
  $a_0$ in Relation steht.
  \item Fall 3: $a_1 \geq a_0, m^{\prime} > m$:
  Es gibt ein kleinstes $k \in \N: mk < m^{\prime}$.
  Es folgt $a_1 + mk \sim a_1 \sim a_1 + m^{\prime}$ und wir erhalten einen
  Widerpruch mittels Fall 1 angewandt auf das Paar $(a_1 + mk, a_1 + m^{\prime})$.
\end{itemize}
Insgesamt gilt also, dass jede Kongruenzrelation auf $(\mathbb{N},+,0)$ einer Partition
der Form
\begin{align*}
  P_{(a_0,m)} := \{\{1\},\dots,\{a_0 - 1\},\{a_0 + km: k \in \N\},\dots,\{a_0 + m - 1 + km: k \in \N\}\}
\end{align*}
mit $a_0,m \in N$ entspricht. Damit haben wir eine bijektive Abbildung
\begin{align*}
  \varphi: \begin{cases}
    \N^2 \rightarrow \mathcal{K} \\
    (a_0,m) \mapsto P_{(a_0,m)}
  \end{cases}
\end{align*}
gefunden und es gilt $|K| = |\N|$. \\
*************************** (andere Argumentation) \\
Dass Gleichheit gilt, erkennt man im Folgenden. Sei angenommen, es gibt zwei Zahlen $n_1 < n_2$ zwischen $a_0$ und $b_0$ (also $n_1 = a_0 + d_1$ und $n_2 = a_0 + d_2$ mit $d_1,d_2 \in {0,...,m-1}$), die in Relation stehen.

Dann folgt (wiederum durch induktives Anwenden der Kongruenz bzgl. $+$ und $1 \sim 1$), dass
\begin{align*}
n_1 + (m-d_2) = \underbrace{a_0 + m}_{\sim a_0} \sim a_0 + \underbrace{(m + d_1 - d_2)}_{<m} = n_2 + (m-d_2),
\end{align*}
was ein Widerspruch zur Minimalität von $b$ wäre.\\
********************* \\
Umgekehrt kann man zeigen, dass für alle $a_0 \in \N$ und $b_0 > a_0$ diese Konstruktion eine Kongruenzrelation ergibt. Die Äquivalenzklassen sind wie oben festgelegt, die Verträglichkeit mit der Addition zeigt man durch Fallunterscheidung.
Wir wollen zeigen, dass für alle $x_1,x_2,y_1,y_2 \in \N$ gilt:
\begin{align*}
  x_1 \sim y_1,x_2 \sim y_2 \Rightarrow x_1 + x_2 \sim y_1 + y_2
\end{align*}

\begin{enumerate}[label = \textit{\arabic*.}]
\item Fall ($x_1 < a_0, x_2 < a_0$):
\begin{align*}
  &\Rightarrow y_1 = x_1, y_2 = x_2 \\
  &\Rightarrow x_1 + x_2 = y_1 + y_2
\end{align*}
\item Fall ($x_1 < a_0, x_2 \geq a_0$):
\begin{align*}
  &\Rightarrow y_1 = x_1, y_2 = x_2 + nm \\
  &\Rightarrow  y_1 + y_2 = x_1 + x_2 + nm \sim x_1 + x_2
\end{align*}
\item Fall ($x_1 \geq a_0, x_2 \geq a_0$):
\begin{align*}
  &\Rightarrow y_1 = x_1 + n_1 m, y_2 = x_2 + n_2 m \\
  &\Rightarrow  y_1 + y_2 = x_1 + x_2 + (n_1+n_2) m \sim x_1 + x_2
\end{align*}
\end{enumerate}

Die Kongruenzrelationen von $(\mathbb{N},+,0)$ sind also genau festgelegt durch ein beliebiges $a_0 \in \N$ und $b_0 > a_0$. Insgesamt gibt es also abzählbar unendlich viele.

Insbesondere haben wir (für $a_0 = 0$) die Modulo-Kongruenzrelationen. Für jede natürliche Zahl
$n \geq 1$ ist $a \sim b: \iff a \equiv b \mod n$ eine Kongruenzrelation, welche
die Partition in die jeweiligen Restklassen induziert.
\item [3.] Jede Teilmenge $A$ der Potenzmenge der Primzahlen induziert eine
Unteralgebra von $(\mathbb{N},\cdot,1)$ mit
\begin{align*}
  U_A := \left\{\prod_{j=1}^na_j^{n_j}: n, n_j \in \mathbb{N}, a_j \in A\right\}.
\end{align*}
Diese Unteralgebren sind aufgrund der Eindeutigkeit
der Primfaktorzerlegung allesamt paarweise verschieden.
Damit können wir zwar noch nicht alle möglichen Unteralgebren
vollständig beschreiben, allerdings ist aufgrund $|2^\mathbb{P}| = |2^\mathbb{N}|
= |\mathbb{R}|$ die Kardinalität der Unteralgebren auf jeden
Fall überabzählbar. \\
Sei $X \subset \mathbb{N}$ beliebig.
Dann ist die Menge
\begin{align*}
  U_X := \left\{\prod_{j=1}^nx_j^{n_j}: n, n_j \in \mathbb{N}, x_j \in X\right\}.
\end{align*}
eine Unteralgebra und offensichtlicherweise können damit auch alle möglichen
Unteralgebren beschrieben werden, allerdings ist diese Darstellung im Allgemeinen
nicht mehr eindeutig.
Also haben wir sowohl eine injektive, als auch eine surjektive Abbildung von
einer Teilmenge der Potenzmenge der natürlichen Zahlen in die Menge aller
Unteralgebren von $(\mathbb{N},\cdot,1)$ gefunden. \\
Ich denke, dass es nicht möglich ist, eine elementare bijektive Abbildung zu finden.
\item [4.] Wieder gibt es die beiden trivialen Relationen: Allrelation + Identitätsrelation.
Weiters ist auch die Modulo 2 Relation eine Kongruenzrelation.
Wiederum eine weitere Kongruenzrelation ist durch
\begin{align*}
  a = \prod_{p \in \mathbb{P}}p^{a_p} \sim b = \prod_{p \in \mathbb{P}}p^{b_p}: \iff \sum_{p \in \mathbb{P}} a_p = \sum_{p \in \mathbb{P}} b_p
\end{align*}
definiert.
\item [5.] Jede Unteralgebra muss auf jeden Fall $\{0,1\}$ enthalten. Aufgrund
der Abgeschlossenheit unter der Addition und dem Induktionsprinzip
muss jede Unteralgebra bereits ganz $\mathbb{N}$ enthalten.
\item [6.] Wieder gibt es zwei triviale Relationen mit der Allrelation und der
Identitätsrelation.

Es kommen nur die in Punkt 2. beschriebenen Kongruenzrelationen in Frage. Tatsächlich sind sogar alle davon auch mit der Multiplikation verträglich, wie sich wiederum durch Unterscheidung in die 3 Fälle zeigt:

Wir wollen zeigen, dass für alle $x_1,x_2,y_1,y_2 \in \N$ gilt:
\begin{align*}
  x_1 \sim y_1,x_2 \sim y_2 \Rightarrow x_1 \cdot x_2 \sim y_1 \cdot y_2
\end{align*}

\begin{enumerate}[label = \textit{\arabic*.}]
\item Fall ($x_1 < a_0, x_2 < a_0$):
\begin{align*}
  &\Rightarrow y_1 = x_1, y_2 = x_2 \\
  &\Rightarrow x_1 \cdot x_2 = y_1 \cdot y_2
\end{align*}
\item Fall ($x_1 < a_0, x_2 \geq a_0$):
\begin{align*}
  &\Rightarrow y_1 = x_1, y_2 = x_2 + nm \\
  &\Rightarrow  y_1 \cdot y_2 = x_1 \cdot x_2 + (nx_1)m \sim x_1 \cdot x_2
\end{align*}
\item Fall ($x_1 \geq a_0, x_2 \geq a_0$):
\begin{align*}
  &\Rightarrow y_1 = x_1 + n_1 m, y_2 = x_2 + n_2 m \\
  &\Rightarrow  y_1 \cdot y_2 = x_1 \cdot x_2 + (n_1 x_2 +n_2 x_1 + n_1 n_2 m) m \sim x_1 \cdot x_2
\end{align*}
\end{enumerate}

\end{itemize}
\end{solution}
