\begin{algebraUE}{147}
Auf der zweielementigen Menge $H := \{0,1\}$ definiert $xy := x$ eine
nichtkommutative Halbgruppenoperation. Adjunktion eines neutralen Elements $e$
ergibt nach Proposition 3.1.1.9 ein Monoid $M = \{e,0,1\}$. Das einelementige
Monoid $\{e\}$ zusammen mit der konstanten Abbildung $\kappa: M \rightarrow \{e\}$
erweist sich als Quotientenmonoid von $M$. Es liegt also ein Beispiel eines
Monoids vor, das erstens nicht isomorph in sein Quotientenmonoid eingebettet
wird, und zweitens ein (in diesem Fall auf triviale Weise) kommutatives
Quotientenmonoid hat, obwohl es selbst nicht kommutativ ist.
\end{algebraUE}
\begin{solution}
Zuerst zeigen wir, dass $(\{e\}, \kappa)$ ein Objekt der Kategorie $\mathcal{C}(M,M)$ ist. \\
$\kappa$ ist klarerweise ein Monoidhomomorphismus und für jedes $x \in M$
besitzt $\kappa(x) = e$ mit dem neutralen Element $e$ auch ein Inverses.
Jetzt gilt es noch zu zeigen, dass $(\{e\}, \kappa)$ auch initial in dieser
Kategorie ist. Sei dazu $(A,\varphi)$ ein beliebiges Objekt der Kategorie.
Dann existiert genau ein Monoidhomomorphismus $\psi: \{e\} \rightarrow A; e \mapsto e_A$,
der das neutrale Element wieder auf das neutrale Element abbildet.
Wir zeigen, dass $\varphi$ ebenfalls die konstante Abbildung
$\psi \circ \kappa: M \rightarrow A; x \mapsto e_A$
sein muss.
Angenommen
\begin{align*}
  e_A \neq a = \varphi(0).
\end{align*}
Dann folgt
\begin{align*}
  a = \varphi(0) = \varphi(01) = \varphi(0)\varphi(1) = a\varphi(1) \\
  a^{-1}a = a^{-1}a\varphi(1) \\
  e_A = e_A\varphi(1) = \varphi(10) = \varphi(1)\varphi(0) = e_Aa
\end{align*}
im Widerspruch zu $a \neq e_A$. Analog zeigt man $\varphi(1) = e_A$. \\
Also gilt $\varphi = \psi \circ \kappa$ und $(\{e\}, \kappa)$ ist somit initial
in $\mathcal{C}(M,M)$.
\end{solution}
