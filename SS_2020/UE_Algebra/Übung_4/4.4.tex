\begin{algebraUE}{151}
Seien $G,H$ Gruppen, $\sim$ eine Äquivalenzrelation auf $G$ und $f: G \rightarrow H$
eine Abbildung. Dann gilt: \\
Ist $\sim$ eine Kongruenzrelation auf $G$ bezüglich der binären Operation, dann
sogar bezüglich der Gruppenstruktur.
\end{algebraUE}
\begin{solution}
Es gilt also:
\begin{align*}
  \forall a,b,c,d \in G: a_1 \sim b_1, a_2 \sim b_2 \implies a_1a_2 \sim b_1b_2
\end{align*}
Für die 0-stellige Operation des neutralen Elements wird die Kongruenzbedingung
aufgrund der Reflexivität von jeder Äquivalenzrelation erfüllt.
Gelte nun $a \sim b$, dann folgt aufgrund $a^{-1} \sim a^{-1}, b^{-1} \sim b^{-1}$
\begin{align*}
  a \sim b, a^{-1} \sim a^{-1} \implies aa^{-1} = e \sim ba^{-1} \\
  b^{-1} \sim b^{-1}, e \sim ba^{-1} \implies b^{-1} =  b^{-1}e \sim bb^{-1}a^{-1} = a^{-1}
\end{align*}
Damit ist $\sim$ sogar eine Kongruenzrelation bezüglich der Gruppenstruktur.
\end{solution}
