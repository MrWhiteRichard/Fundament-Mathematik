\begin{exercise}
    Sei $(P,\leq)$ eine Halbordnung, in der jede Teilmenge ein Infimum hat.
    Dann hat auch jede Teilmenge von $P$ ein Supremum. Insbesondere liegt ein
    vollständiger Verband vor.
\end{exercise}
\begin{solution}
Sei $M \subseteq P$ beliebig. \\
Mit $O := \{p \in P: \forall m \in M: p \geq m\}$
bezeichne die Menge aller oberen Schranken von M. Laut Voraussetzung wissen wir,
dass diese Menge ein Infimum besitzt. Nun behaupten wir $\inf(O)  = \sup(M)$. 

Zuerst weisen wir dafür nach, dass $\inf(O)$ eine obere Schranke von $M$ ist. Sei also $m \in M$ beliebig. Wir wissen, dass $m$ eine untere Schranke von $O$ ist, weil $\forall o \in O: m \leq o$. Da $\inf(O)$ die größte untere Schranke ist, gilt also $m \leq \inf(O)$. Da $m \in M$ beliebig war ist also $\inf(O)$ eine obere Schranke von $M$.

Nun gilt es noch nachzuweisen, dass $\inf(O)$ die kleinste obere Schranke von $M$ ist. Dies folgt aber unmittelbar daraus, dass für eine beliebige obere Schranke $o \in O$ gilt, dass $\inf(O) \leq o$ ist.  \newline


Die Halbordnung $(\mathbb{N}, \leq)$ ist kein Gegenbeispiel, da sie die Voraussetzung
nicht erfüllt. \\
Dazu betrachte man die leere Menge: $\emptyset$\\
Die Menge aller oberen Schranken M = $\{n \in \mathbb{N}: \forall x \in \emptyset: n \leq x \}$\\
der leeren Menge ist klarerweise ganz $\mathbb{N}$ und besitzt kein Maximum. \\
Formaler Beweis davon bräuchte eine formale Definition von $\leq$, welche ich
im Algebra-Skript leider nicht finden konnte.
\end{solution}
