\begin{algebraUE}{183}
\begin{enumerate}
  \item Sei $R$ ein Unterring mit $1$ eines Körpers $K$. Dann ist
  \begin{align*}
    K^{\prime} := \left\{\frac{p}{q}: p,q \in R, q \neq 0 \right\}
  \end{align*}
  ein Unterkörper von $K$.
  \item Der Körper $K^{\prime}$ aus dem ersten Teil ist der kleinster Unterkörper
  von $K$, der $R$ enthält, symbolisch $K^{\prime} = \langle R \rangle_{\text{Körper}}$.
  Explizit bedeutet das: Jeder Unterkörper $K^{\primeprime}$ von $K$ mit $R \subseteq K^{\primeprime}$
  umfasst $K^{\prime}$.
  \item In derselben Situation ist $K$ (zusammen mit der Inklusionsabbildung) genau
  dann ein Quotientenkörper von $R$, wenn $K = K^{\prime}$ gilt.
  \item Ist $\iota: R \rightarrow K$ eine isomorphe Einbettung des Integritätsbereichs $R$
  in einen Körper $K$ und $Q$ der von $\iota(R)$ erzeugte Unterkörper von $K$,
  so ist $Q$ zusammen mit $\iota$ ein Quotientenkörper von $R$.
\end{enumerate}
\end{algebraUE}
\begin{solution}
\leavevmode \\
\begin{enumerate}
  \item
\end{enumerate}

\end{solution}
