\begin{algebraUE}{164}
Sei $G$ eine Gruppe. Für alle $g \in G$ definieren wir $\pi_g: G \rightarrow G, x \mapsto gxg^{-1}$
und betrachten die Abbildung $\Phi: g \mapsto \pi_g$. Dann gilt
\begin{enumerate}
  \item Für $g,h \in G$ gilt $\pi_g \circ \pi_h = \pi_{gh}$. Somit ist $\Phi$
  ein Homomorphismus von $G$ in die Automorphismengruppe $\Aut(G)$.
  \item Für alle $g \in G$ ist $\pi_g$ ein Automorphismus (genannt der durch
  Konjugation mit $g$ induzierte innere Automorphismus von $G$).
  \item Für den Kern von $\Phi$ gilt
  \begin{align*}
    \ker(\Phi) = Z(G) = \{g \in G: \forall h \in G: gh = hg\}.
  \end{align*}
  Insbesondere ist $\Phi: G \rightarrow \Aut(G)$ eine isomorphe Einbettung genau dann,
  wenn das Einselement $e \in G$ das einzige ist, das mit allen $g \in G$ vertauscht.
  \item Die inneren Automorphismen bilden einen Normalteiler $\Phi(G) \vartriangleleft \Aut(G)$
  der Automorphismengruppe von $G$. (Die Faktorgruppe $\Aut(G)/\Phi(G)$ nennt man
  auch die äußere Automorphismengruppe von $G$.)
\end{enumerate}
\end{algebraUE}
\begin{solution}
\leavevmode \\
\begin{enumerate}
  \item
\end{enumerate}

\end{solution}
