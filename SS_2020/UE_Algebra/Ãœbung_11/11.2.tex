\begin{algebraUE}{335}
Beweisen Sie den Gradsatz, indem Sie zeigen: Ist die Familie $(a_i)_{i \in I}$
eine Basis von $E$ über $K$, die Familie $(b_j)_{j \in J}$ eine Basis von $L$ über $E$,
so ist die Familie $(a_ib_j)_{(i,j) \in I \times J}$ eine Basis von $L$ über $K$.
(Achtung: Ihre Argumentation muss auch für unendliches $I$ und $J$ gelten.)

\end{algebraUE}

\begin{solution}
  \begin{itemize}
      \item \textbf{Erzeugendensystem:} Sei $x \in L/K$ beliebig. Als Element von $L/E$ hat $x$ eine Darstellung der Form
  \begin{align}
      x = \sum_{j=1}^n \underbrace{\beta_j}_{\in E/K} b_j = \sum_{j=1}^n \left(\sum_{i=1}^m \alpha_{ij} a_i \right) b_j = \sum_{\substack{i \in \{1, ..., n\} \\ {j\in \{1,...,m\}}}} \alpha_{ij} (a_i b_j).
  \end{align}
      \item \textbf{Linear unabhängig:} Es gelte $\sum_{(i,j) \in E} \alpha_{ij} (a_i b_j) = 0.$  Aus
      \begin{align}
          0 = \sum_{(i,j) \in E} \alpha_{ij} (a_i b_j) = \sum_{(i,j) \in E} \underbrace{(\alpha_{ij} a_i)}_{\in E} b_j
      \end{align}
      und weil $(b_j)_{j \in J}$ eine Basis von $L/E$ ist, folgt $\alpha_{ij} a_i = 0$ für alle $(i,j) \in E.$

      Weil $(a_i)_{i \in I}$ als Basis sicher nicht den Nullvektor enthält, gilt für alle $(i,j) \in E$: $a_i \neq 0$ und somit $a_{ij} = 0$.
  \end{itemize}
\end{solution}
