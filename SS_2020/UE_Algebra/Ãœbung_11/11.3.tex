\begin{algebraUE}{336}

Zeigen Sie, dass der Ring $K[[x]]$ der formalen Potenzreihen über einem Körper $K$ euklidisch, folglich auch ein Hauptidealring und faktoriell ist. Bestimmen Sie alle irreduziblen Elemente modulo Assoziiertheit und geben Sie sämtliche Ideale durch Erzeugende an, jedes genau einmal.

\end{algebraUE}

\begin{solution}

Für die euklidische Bewertung wählen wir in dem Fall

\begin{align*}
  H:K[[x]] \setminus \{0\} \to \N: p \mapsto \ord(p)
\end{align*}

wohldefiniert, da für $p \in K[[x]] \setminus \{0\}$, $\ord(p) \geq 0$.

Wir müssen noch zeigen, dass Division mit Rest möglich ist, Also

\begin{align*}
  \Forall g \in K[[x]] \setminus \{0\}, f \in K[[x]] \Exists q,r \in K[[x]]: f = gq + r \text{~mit~} H(r) < H(g).
\end{align*}

Seien also $g \in K[[x]] \setminus \{0\}, f \in K[[x]]$ beliebig, dann brauchen wir eine Fallunterscheidung

Fall 1: $\ord(f) < \ord(g) \lor f = 0$:

Dann gilt

\begin{align*}
  f = g \cdot 0 + f
\end{align*}

Mit $H(r) = \ord(f) < \ord(g) = H(g) \lor f = r = 0$.

Fall 2: $\underbrace{\ord(f)}_{m} \geq \underbrace{\ord(g)}_{n}$:

Wir wissen schon, dass wir $f$ und $g$ umschreiben können als

\begin{align*}
  f = x^m\tilde{f}, \ord(\tilde{f}) = 0 \\
  g = x^n\tilde{g}, \ord(\tilde{g}) = 0
\end{align*}

Nach Proposition 3.3.6.5 (6) wissen wir bereits, dass $\tilde{g}$ eine Inverse besitzt, also wählen wir $q = \tilde{g}^{-1}x^{m-n}\tilde{f}$ und erhalten

\begin{align*}
  f = x^m \tilde{f} = x^n\tilde{g} \cdot \tilde{g}^{-1}x^{m-n}\tilde{f} = gq + 0
\end{align*}

also $r = 0$.

Der Ring ist also euklidisch und somit bekanntlich auch Hauptideal- und faktorieller Ring.
Zusätzlich haben wir gezeigt, dass $\ord(f) \geq \ord(g) \implies g | f$.

Die Einheiten sind, wie bereits vorhin erwähnt $E := E(K[[x]]) = \{p \in K[[x]]: \ord(p) = 0\}$. Wir behaupten, dass alle irreduziblen Elemente gegeben sind durch

\begin{align*}
  \{p \in K[[x]]: \ord(p) = 1\} = xE ( = [x]_{\sim})
\end{align*}

Zunächst zur Gleichheit der beiden Ausdrücke. Wie oben schon verwendet kann man ein $p \in K[[x]]$ genau dann als $x\tilde{p}$ mit $\tilde{p} \in E$ schreiben, wenn $\ord(p) = 1$.

Dass $x$ (und somit alle assoziierten) irreduzibel sind, folgt für $x = pq$ mit beliebigen Potenzreihen $p,q$, da $1 = \ord(x) = \ord(pq) = \ord(p) + \ord(q)$. Also muss $\ord(p) = 0 \lor \ord(q) = 0$ und somit ein Faktor eine Einheit.

Angenommen, es gäbe ein irreduzibles Element $p \neq xe, e \in E$, also $\ord(p) = n \geq 2$. Dann können wir $p = x^n\tilde{p} = x^{n-1} (x\tilde{p})$ schreiben, wobei beide Faktoren Ordnung größer 0 haben und somit keine Einheiten sind.

Nun zu den Idealen. Wir wissen schon, dass $K[[x]]$ ein Hauptidealring ist, und somit können wir nach Proposition 5.2.2.6 jedes Ideal mit einer Äquivalenzklasse
bezüglich der Assoziiertheitsrelation aus der Teilerhalbordnung (in dem Fall sogar Totalordnung) $(K[[x]]/_{\sim},|)$ identifizieren.

Diese besteht genau aus den Äquivalenzklassen der Monome (bzw. Potenzreihen mit Ordnung $n$, Fall $n=1$ oben schon behandelt). Alle Ideale sind also eindeutig gegeben durch

\begin{align*}
  (x^n) = \{px^n: p \in K[[x]]\} = \{q \in K[[x]]: \ord(q) \geq n\}, n \in \N
\end{align*}
Alternative Argumentation: \\
Wie man leicht erkennt haben für alle $n \in \N$ mit
\begin{align*}
  (x^n) = \{px^n: p \in K[[x]]\} = \{q \in K[[x]]: \ord(q) \geq n\}, n \in \N
\end{align*}
unterschiedliche Ideale gegeben. Weiters muss jedes Ideal in der Form darstellbar sein,
da für $r \in K[[x]]$ beliebig mit $\ord(r) = n$ gilt $x^n | r$, also $\frac{x^n}{r} \in K[[x]]$ und
\begin{align*}
  x^n = \frac{x^n}{r}r \in (r) \implies (x^n) \subseteq (r).
\end{align*}
Gleichzeitig gilt für alle $p \in K[[x]]: \ord(pr) = \ord(p) + \ord(r) \geq n$.
Also gilt sogar $(x^n) = (r)$.
\end{solution}
