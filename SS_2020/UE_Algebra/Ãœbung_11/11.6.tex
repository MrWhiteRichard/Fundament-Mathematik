\begin{algebraUE}{382}
  Sei $L := \Q(x)$ der Körper der gebrochen rationalen Funktionen über $\Q$.
  \begin{itemize}
      \item Berechnen Sie $[L:K]$ für $K:= \Q(x^3) \leq L$, indem Sie das Minimalpolynom von $x$ über $K$ finden.
      \item Wie Teil 1, nur mit $K := \Q(x+\frac{1}{x}).$
      \item Z. z.: $[L:K] < \infty$ für jeden Körper $K := \Q(\alpha)$ mit $\alpha \in \Q(x) \backslash \Q.$
  \end{itemize}
\end{algebraUE}

\begin{solution}
\leavevmode \\
\begin{itemize}
  \item $\Q(x^3)$ ist per Definition der kleinste Körper, welcher $\Q \cup \{x^3\}$
  enthält.
  \begin{align*}
    K = \left\{\frac{\sum_{k=0}^np_kx^{3k}}{\sum_{k=0}^mq_kx^{3k}}: p_k,q_k \in \Q, k = 0,\dots,n\right\}
  \end{align*}
  faktorisiert nach $\sim$ ist mit Sicherheit ein Körper der $\Q \cup \{x^3\}$ enthält. Klarerweise muss
  auch jeder Körper, welcher $\Q$ und $\{x^3\}$ enthält, bereits $K/_{\sim}$ enthalten,
  also ist $K/_{\sim} = \Q(x^3)$. \\
  Wir suchen zuerst ein Polynom $m \in \Q(x^3)[y]$ mit $m(x) = 0$.
    \begin{align*}
        m(y) = y^3 - x^3.
    \end{align*}
    Wir zeigen nun, dass $m$ auch irreduzibel über $\Q(x^3)$ ist. Angenommen $m$
    hätte eine Nullstelle  $\frac{p(x^3)}{q(x^3)} \in \Q(x^3)$. Dann gilt
    \begin{align*}
      \frac{p(x^3)^3}{q(x^3)^3} = x^3
      \iff \sum_{k=0}^np_kx^{3k} = p(x^3) = q(x^3)x = \sum_{k=0}^mq_kx^{3k+1}.
    \end{align*}
    Dies ist aufgrund der linearen Unabhängigkeit der Familie $(x^k)_{k \in \N}$
    über $\Q$ nur in trivialer Weise möglich, was ein Widerspruch zu $q(x^3) \neq 0$ ist.
    Also ist $m$ irreduzibel über $\Q(x)$ und somit das Minimalpolynom von $x$.
  \item \begin{align*}
    m(y) = y^2 + 1 - \left(\frac{x^2 + 1}{x}\right)y
  \end{align*}
  erfüllt $m(x) = 0$. Um zu zeigen, dass $m$ das gewünschte Minimalpolynom ist,
  zeigen wir, dass $x$ nicht in $\Q(x + \frac{1}{x})$ liegt und somit keine
  Nullstelle eines linearen Polynoms sein kann. Angenommen dem wäre so, dann gäbe
  es $p(x+\frac{1}{x}),q(x+\frac{1}{x}) \in \Q[x+\frac{1}{x}]$ mit
  \begin{align*}
    &\frac{p(x+\frac{1}{x})}{q(x+\frac{1}{x})} = x \\
    &\iff p(x+\frac{1}{x}) = q(x + \frac{1}{x})x \\
    &\iff \sum_{k=0}^np_k\frac{(x^2+1)^k}{x^k} = \sum_{k=0}^nq_k\frac{(x^2+1)^k}{x^{k-1}} \\
    &\iff \sum_{k=0}^np_k(x^2+1)^kx^{n-k} = \sum_{k=0}^nq_k(x^2+1)^kx^{n-k+1} \\
    &\iff \sum_{k=0}^np_k\sum_{j=0}^k\binom{k}{j}x^{2(k-j)}x^{n-k} = \sum_{k=0}^nq_k\sum_{j=0}^k\binom{k}{j}x^{2(k-j)}x^{n-k+1} \\
    &\iff \sum_{k=0}^np_k\sum_{j=0}^k\binom{k}{j}x^{n+k-2j}= \sum_{k=0}^nq_k\sum_{j=0}^k\binom{k}{j}x^{n+k-2j+1} \\
  \end{align*}
  Jetzt steh ich da an.
  \item Man kann die gebrochen rationale Funktion mit dem Nenner multiplizieren
  und dann den Zähler abziehen und erhält mit dieser Vorschrift stets ein Polynom,
  dass $x$ als Nullstelle hat:
  Sei $\alpha = \frac{\sum_{i=1}^n a_i x^i}{\sum_{j=1}^m b_j x^j}$. Das Polynom
  \begin{align}
  m_\alpha(y) := \alpha \left(\sum_{j=1}^m b_j y^j\right) - \sum_{i=1}^n a_i y^i
  \end{align}

  erfüllt $m_\alpha(x) = 0.$
\end{itemize}



\end{solution}
