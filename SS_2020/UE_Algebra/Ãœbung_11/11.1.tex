\begin{algebraUE}{349}

Gegeben sei das Polynom $f(x) = x^4 + x^3 + x^2 +x^1 + 1 = \frac{x^5 - 1}{x - 1}.$
\begin{enumerate}
  \item Zerlegen Sie $f$ in seine irreduzible Faktoren über $\Q,\R$ und $\C$.
  \item Finden Sie Wurzelausdrücke für die reellen Zahlen $\cos(\frac{k\pi}{5}), k = 1,2,3,4$.
\end{enumerate}

\end{algebraUE}

\begin{solution}
\begin{enumerate}
  \item $f(x) = 0$ genau dann wenn $x \neq 1$ und $x^5 = 1$. Über $\C$ zerfällt das
  Polynom in Linearfaktoren, welche dann klarerweise irreduzibel sind:
  \begin{align*}
    f(x) = \left(x - \exp\left(\frac{2\pi i}{5}\right)\right)\left(x - \exp\left(\frac{4\pi i}{5}\right)\right)
    \left(x - \exp\left(\frac{6\pi i}{5}\right)\right)\left(x - \exp\left(\frac{8\pi i}{5}\right)\right)
  \end{align*}
  Über $\R$ fassen wir einfach die komplex konjugierten Linearfaktoren zu quadratischen
  Faktoren zusammen.
  \begin{align*}
    f(x) = \left(x^2 + \left(\frac{1 + \sqrt{5}}{2}\right)x + 1\right)\left(x^2 + \left(\frac{1 - \sqrt{5}}{2}\right)x + 1\right)
  \end{align*}
  Diese beiden Faktoren liegen nicht in $\Q(x)$ und somit ist das Polynom über $\Q$ irreduzibel.
  \item
  \begin{align*}
    \cos\left(\frac{\pi}{5}\right) &= \frac{1}{4}\left(1 + \sqrt{5}\right) \\
    \cos\left(\frac{2\pi}{5}\right) &= \frac{1}{4}\left(- 1 + \sqrt{5}\right) \\
    \cos\left(\frac{3\pi}{5}\right) &= \frac{1}{4}\left(1 - \sqrt{5}\right) \\
    \cos\left(\frac{4\pi}{5}\right) &= \frac{1}{4}\left(-1 - \sqrt{5}\right) \\
  \end{align*}
  sind Nullstellen von
  \begin{align*}
    f(x) = ((4x)^2 - 6)^2 - 20.
  \end{align*}
\end{enumerate}


\end{solution}
