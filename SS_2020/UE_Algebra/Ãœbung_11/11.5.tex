\begin{algebraUE}{372'}
Zeigen Sie, dass das regelmäßige Siebeneck mit Radius 1 nicht konstruierbar ist.
\end{algebraUE}

\begin{solution}
  Wir müssen zeigen, dass die primitive siebente Einheitswurzel $\zeta_7$ nicht konstruierbar ist. Nach Definition ist $\zeta_7$ eine Nullstelle des Polynoms $x^7-1.$ Durch Polynomdivision erhält man die Faktorisierung
  \begin{align}
      x^7-1 = (x-1)\underbrace{(x^6+x^5+x^4+x^3+x^2+x+1)}_{=: m(x)}
  \end{align}
  und erkennt, dass $\zeta_7$ Nullstelle des zweiten Faktors sein muss.

  Wäre $m(x)$ reduzibel, so wäre auch $m(x+1)$ reduzibel. Allerdings gilt
  \begin{align}
      m(x+1) = x^6+7x^5+21x^4+35x^3+35x^2+21x+7,
  \end{align}

  was nach dem Eisensteinkriterium irreduzibel ist. $m(x)$ ist also ebenfalls irreduzibel und daher das Minimalpolynom von $\zeta_7.$

  Damit gilt $[\mathbb{Q}(\zeta_7):\mathbb{Q}] = \mathrm{grad}(m(x)) = 6$; der Grad der Körpererweiterung ist also nicht Teiler einer Zweierpotenz und $\zeta_7$ folglich nicht konstruierbar.
\end{solution}
