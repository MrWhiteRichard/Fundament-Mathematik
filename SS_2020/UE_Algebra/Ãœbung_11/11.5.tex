\begin{algebraUE}{372'}
Zeigen Sie, dass das regelmäßige Siebeneck mit Radius 1 nicht konstruierbar ist.
\end{algebraUE}

\begin{solution}
  Wir müssen zeigen, dass die primitive siebente Einheitswurzel $\zeta_7$ nicht konstruierbar ist. Nach Definition ist $\zeta_7$ eine Nullstelle des Polynoms $x^7-1.$ Durch Polynomdivision erhält man die Faktorisierung
  \begin{align}
      x^7-1 = (x-1)\underbrace{(x^6+x^5+x^4+x^3+x^2+x+1)}_{=: m(x)}
  \end{align}
  und erkennt, dass $\zeta_7$ Nullstelle des zweiten Faktors sein muss.

  Wäre $m(x)$ reduzibel, so wäre auch $m(x+1)$ reduzibel. Allerdings gilt
  \begin{align}
      m(x+1) = x^6+7x^5+21x^4+35x^3+35x^2+21x+7,
  \end{align}

  was nach dem Eisensteinkriterium irreduzibel ist. $m(x)$ ist also ebenfalls irreduzibel und daher das Minimalpolynom von $\zeta_7.$

  Damit gilt $[\mathbb{Q}(\zeta_7):\mathbb{Q}] = \mathrm{grad}(m(x)) = 6$; der Grad der Körpererweiterung ist also nicht Teiler einer Zweierpotenz und $\zeta_7$ folglich nicht konstruierbar.


  *** Alternative***

  Wir zeigen, dass wir $\sin(\frac{2 \pi }{7})$ nicht konstruieren können, somit nicht $\cos(\frac{2 \pi }{7})$, also auch nicht $\frac{1}{\cos(\frac{2 \pi }{7})}$ und damit auch nicht das regelmäßige Siebeneck mit Radius 1. (SKIZZE!!)

  Es gilt:

  \begin{align*}
    \cos(2 \pi) + i \sin(2 \pi) = (\cos(\frac{2 \pi }{7}) + i \sin(\frac{2 \pi }{7}))^{7}
  \end{align*}

  Ausmultiplizieren und Vergleichen der Imaginärteile (mit Python) führt zu

  \begin{align*}
    0 = \sin(2 \pi) = \sin(\frac{2 \pi }{7}) (-64 \sin(\frac{2 \pi }{7})^{6} + 112 \sin(\frac{2 \pi }{7})^4 - 56 \sin(\frac{2 \pi }{7})^2) + 7).
  \end{align*}

  Es ist also $\sin(\frac{2 \pi }{7})$ Nullstelle vom Polynom

  \begin{align*}
    p(x) = x\underbrace{(-64x^6 + 112x^4 - 56x^2 + 7)}_{=:\tilde{p}(x)}
  \end{align*}

  Das Polynom $\tilde{p}(x)$ ist nach dem Eisensteinkriterium (mit der Primzahl $7$) irreduzibel und somit Minimalpolynom von $\sin(\frac{2 \pi }{7})$.

  Damit gilt $[\mathbb{Q}(\sin(\frac{2 \pi }{7})):\mathbb{Q}] = \mathrm{grad}(\tilde{p}(x)) = 6$; der Grad der Körpererweiterung ist also nicht Teiler einer Zweierpotenz und $\sin(\frac{2 \pi }{7})$ folglich nicht konstruierbar.

\end{solution}
