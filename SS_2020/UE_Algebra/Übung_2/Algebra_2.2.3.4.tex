\begin{exercise}
    Sei $\iota: \N \to \Z$ die Identität auf $\N$. Geben Sie eine Kategorie $\mathfrak{D}$ so an, dass erstens
    \begin{align}
        (\Z, \iota) \text{ ist initial in } \mathfrak{D} \label{firstCondition}
    \end{align}
    gilt und zweitens die Aussage \eqref{firstCondition} eine Umformulierung von Theorem 1.2.1.1(3) ist.
\end{exercise}

\begin{solution}
    Die Kategorie $\mathfrak{D}$ besteht aus einer Klasse $Ob(\mathfrak{D})$ von Objekten $(G, \iota_G)$, mit einer Gruppe $G$ und einer zugehörigen isomorphen Einbettung $\iota_G:\N \to G$ der additiven Halbgruppe $\N$. 
    
    Für alle $(G, \iota_G), (H, \iota_H) \in Ob(\mathfrak{D})$ ist die Menge 
    \begin{align*}
        Hom((G, \iota_G) \to (H, \iota_H)) := \Bbraces{\varphi:G \to H \mid \varphi \text{ ist isomorphe Einbettung } \land \iota_H = \varphi \circ \iota_G}
    \end{align*}
    Diese Mengen bilden ebenfalls eine Kalsse paarweise disjunkter Mengen, wobei Disjunktheit gilt, weil die Funktionen in verschiedenen Mengen entweder unterschiedlichen Definitionsbereich oder unterschiedlichen Zielbereich haben. 
    
    Der dritte Bestandteil von $\mathfrak{D}$ ist die Klasse der Kompositionen $Hom(B,C) \times Hom(A,B) \to Hom(A,C): (g,f) \mapsto g \circ f$ die als die gewöhnliche Hintereinanderausführung von Funktionen verstanden wird. 

    Klarerweise gilt das Assoziativgesetz. Die Identität in der Kategorie ist schlicht die Identität als Funktion einer Menge auf sich selbst.

    Nach Theorem 1.2.1.1. gibt es für alle $(G, \iota_G)$ genau eine isomorphe Einbettung $\varphi:\Z \to G$ mit $\iota_G = \varphi \circ \iota$. Deshalb ist $(\Z, \iota)$ initial.
\end{solution}