\begin{exercise}
    Folgendes ist zu tun.
    \begin{enumerate}[label = (\roman*)]
        \item Seien $C,D$ Algebren, $\varphi:C \to D$ und $\varphi^\prime:C \to D$ Homomorphismen. Sei $B \subseteq C$ und $\langle B \rangle = C$ sowie für alle $b \in B: \varphi(b) = \varphi^\prime(b)$. Zeigen Sie, dass dann $\varphi = \varphi^\prime$ gilt.
        
        \item Seien $C,D$ Körper und  $\varphi:C \to D$ und $\varphi^\prime:C \to D$ Homomorphismen. Sei $B \subseteq C$ und $\langle B \rangle = C$ sowie für alle $b \in B: \varphi(b) = \varphi^\prime(b)$. Ist dann $\varphi = \varphi^\prime$ ?
    \end{enumerate}
\end{exercise}

\begin{solution}
    Hier könnte Ihre Werbung stehen.
    \begin{enumerate}[label = (\roman*)]
        \item Wir verwenden die rekursive Konstruktion von $\langle B \rangle$ durch $B_0 := B$ und
        \begin{align*}
            B_{k + 1} := B_k \cup \Bbraces{\omega(b_1, \dots, b_l) \mid b_1, \dots, b_l \in B_k \land \omega \in \Omega}.
        \end{align*}
        Für alle $b \in B_0$ gilt nach Voraussetzung $\varphi(b) = \varphi^\prime(b)$. Sei nun $n \in \N \setminus \Bbraces{0}$ und gelte für alle $x \in B_n: \varphi(x) = \varphi^\prime(x)$. Wählen wir nun ein beliebiges $c \in B_{n + 1}$. 
        \begin{enumerate}[label = Fall \arabic*:]
            \item Sei $c \in B_n$. Dann gilt nach Voraussetzung $\varphi(c) = \varphi^\prime(c)$. 
            \item Sei $c \notin B_n$. Dann gibt es $\omega \in \Omega$ und $c_1, \dots, c_l \in B_n$ mit $c = \omega(c_1, \dots, c_l)$. Es gilt also
            \begin{align*}
                \varphi(c) &= \varphi(\omega(c_1, \dots, c_l)) = \omega(\varphi(c_1), \dots, \varphi(c_l)) \\
                &= \omega(\varphi^\prime(c_1), \dots. \varphi^\prime(c_l)) = \varphi^\prime(\omega(c_1, \dots, c_l)) = \varphi^\prime(c)
            \end{align*}
        \end{enumerate}
        \item Ausständig!
    \end{enumerate}
\end{solution}