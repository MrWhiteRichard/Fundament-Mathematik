\begin{algebraUE}{279}
Sei $B$ eine unendliche Boolesche Algebra. Zeigen Sie, dass es einen Ultrafiler gibt,
der kein Atom enthält. \\
\textit{Hinweis:} Zeigen Sie, dass die Menge $I$ aller $b \in B$, zu denen es
Atome $a_1,\dots,a_k$ von $B$ mit $b \leq a_1 \lor \dots \lor a_k$ gibt,
ein echtes Ideal von $B$ ist, und betrachten Sie $B/I$.
\end{algebraUE}
\begin{solution}
Zuerst behandeln wir den Fall, dass $\At(B) = \emptyset$. Dann gilt für ein beliebiges $b \in B\setminus \{ 0\}$, dass $\{ a \in B: a \geq b\}$ ein echter Filter ist. Somit gibt es nach dem Ultrafiltersatz einen Ultrafilter (der keine Atome enthält).
Wir zeigen zunächst, dass die Menge $I$ tatsächlich ein Ideal ist. \\

Sei ab nun $\At(B) \neq \emptyset$.
Seien $x, y \in I$ beliebig, d.h.
\begin{align*}
  \exists a_1,...,a_n,b_1,...,b_m \in At(B): &x \leq a_1 \lor \dots \lor a_n , y \leq b_1 \lor \dots \lor b_m \\
  \Rightarrow &x \lor y \leq a_1 \lor \dots \lor a_n \lor b_1 \lor \dots \lor b_m \\
  \Rightarrow &x\lor y \in I
\end{align*}
aufgrund der Monotonie von $\lor$.
Sei $x \in I$ beliebig und $b \in B$ mit $b \leq x$. Dann folgt offensichtlich aus der Transitivität von $\leq$, dass $b \in I$.
Ebenso ist $I \neq \emptyset$, da wir die Existenz eines Atoms $a \in \At(B)$
vorausgesetzt haben und damit klarerweise $a \in I$ folgt.
Nun wollen wir zeigen, dass $I$ ein echtes Ideal ist, also $1 \notin I$.
Unter der Annahme $1 \in I$ gibt es also $a_1,...,a_k$, sodass $1 \leq a_1 \lor \dots \lor a_k$.

Für ein beliebiges $b \in B$ gilt nun
\begin{align*}
  b = b \land 1 = b \land (a_1 \lor \dots \lor a_k) = (b \land a_1) \lor \dots \lor (b \land a_k).
\end{align*}
Für alle $i=1 \dots k$ gilt nun $(b \land a_i) = 0$ oder $ (b \land a_i) = a_i$. Also gilt $|B| = 2^k < \infty$ und somit ein Widerspruch.

Nach Definition 3.6.6.9 ist durch das Ideal $I$ eine Äquivalenzrelation $\sim_I$ definiert, wobei nach Lemma 3.6.6.10 gilt
\begin{align*}
  b \sim_I c \Leftrightarrow \Exists i \in I: b \land i' = c \land i' \Leftrightarrow \Exists i \in I: b \lor i = c \lor i
\end{align*}

Da $I$ auf dem zugehörigen Booleschen Ring eine Kongruenzrelation induziert,
haben wir auch eine Kongruenzrelation auf der Booleschen Algebra.

Seien $x,y \in [1]_{\sim_I}$ beliebig, d.h.
\begin{align*}
  \Exists i,j \in I: &x \lor i = 1 \lor i = 1, y \lor j = 1 \lor j = 1\\
  \Rightarrow &(x \land y) \lor \underbrace{(i \lor j)}_{\in I} = (x \lor (i \lor j)) \land (y \lor (i \lor j)) = 1 \land 1 = 1 = 1 \lor (i \lor j) \\
  \Rightarrow &x \land y \in [1]_{\sim_I}
\end{align*}

Für $x \in [1]_{\sim_I}$ beliebig und $y \in B: y \geq x$ gilt
\begin{align*}
  y \lor i \geq x \lor i = 1 \lor i \\
  \Rightarrow y \in [1]_{\sim_I}
\end{align*}
Weiters gilt, da $1 \notin I$
\begin{align*}
  \forall i \in I: 0 \lor i  = i \neq 1 = 1 \lor i.
\end{align*}
Es ist also $[1]_{\sim_I}$ ein echter Filter und es gilt
\begin{align*}
  \Forall a \in At(B): a \in I \text{ und } a'  \lor a = 1 \lor a \Rightarrow a' \in [1]_{\sim_I}.
\end{align*}

Nach dem Ultrafiltersatz wissen wir, es gibt einen echten Ultrafilter $U \supseteq [1]_{\sim_I}$. Es gilt also auch, dass alle $a'$ in $U$ sind und somit kann kein Atom $a$ im Ultrafilter liegen.

\end{solution}
