\begin{algebraUE}{334}

Der von $\Z$ und der imaginären Einheit erzeugte Ring $Z[i] = \{a + ib: a,b \in \Z\}$
(genannt der Ring der ganzen Gauß'schen Zahlen) ist euklidisch vermittels der
euklidischen Bewertung $H(z) := |z|^2$, folglich auch ein Hauptidealring und faktoriell.

\end{algebraUE}

\begin{solution}

Die euklidische Bewertung

\begin{align*}
  H(a+ib) = a^2 + b^2
\end{align*}

ist sicher eine Abbildung von $\Z [i] \setminus \{0\} \to \N$.

Wir müssen noch zeigen, dass Division mit Rest möglich ist, also

\begin{align*}
  \Forall a \in \Z [i] \setminus \{0\}, b \in \Z [i]: \Exists q,r \in \Z [i]: b = aq + r \text{~mit~} H(r) < H(a) \text{ oder } r = 0.
\end{align*}

Seien also $a \in \Z [i] \setminus \{0\}, b \in \Z [i]$ beliebig, dann gilt mit der Definition

\begin{align}\label{def}
  c_r + i c_i := \frac{b}{a} = \frac{\overbrace{b\overline{a}}^{\in \Z[i]}}{\underbrace{a\overline{a}}_{\in \Z}},
\end{align}

dass $c_r, c_i \in \Q$. Wir können in jedem Fall ganze Zahlen $q_r, q_i$ finden mit $|c_r-q_r| \leq \frac{1}{2}$ und $|c_i-q_i| \leq \frac{1}{2}$
($q_r := \lfloor |c_r| + \nicefrac{1}{2}\rfloor\sgn(c_r), q_i := \lfloor |c_i| + \nicefrac{1}{2}\rfloor\sgn(c_i)$).

Wir schreiben ein wenig um

\begin{align*}
  c_r + i c_i = \underbrace{q_r + i q_i}_{\in \Z [i]} + (c_r- q_r) + i (c_i - q_i)
\end{align*}

Multiplizieren wir \eqref{def} mit $a$, so erhalten wir

\begin{align*}
  b = a(c_r + ic_i) = a(q_r + i q_i) + \underbrace{a((c_r- q_r) + i (c_i - q_i))}_{=: r}
\end{align*}

wobei der erste Summand in $\Z [i]$ ist, und somit auch der zweite. Der zweite Summand ist also ein Kandidat für das $r$ aus der Definition. Prüfen wir die Bedingung noch nach.

Zunächst gilt

\begin{align*}
  H((c_r- q_r) + i (c_i - q_i)) = (c_r- q_r)^2 + (c_i - q_i)^2 \leq \frac{1}{4} + \frac{1}{4} < 1.
\end{align*}

Mit der Multiplikativität des Betrages gilt also

\begin{align*}
  H(a((c_r- q_r) + i (c_i - q_i))) = H(a) \cdot H((c_r- q_r) + i (c_i - q_i)) < H(a).
\end{align*}


\end{solution}
