\begin{algebraUE}{141}
In dieser Übungsaufgabe interessieren wir uns für Unteralgebren und Kongruenzrelationen
auf $\mathbb{N}$ bezüglich additiver und/oder multiplikativer Struktur. Versuchen
Sie jeweils alle Objekte der angegebenen Art zu beschreiben. Wenn Ihnen das zu
schwierig erscheint (was in der Mehrzahl der Fälle wahrscheinlich ist), ermitteln
Sie, wieviele es davon gibt. Unterscheiden Sie dabei verschiedene unendliche
Kardinalitäten, insbesondere $|\mathbb{N}|$ und $|\mathbb{R}|$.
\begin{itemize}
  \item [1.] Unteralgebren von $(\mathbb{N},+,0)$
  \item [2.] Kongruenzrelationen von $(\mathbb{N},+,0)$
  \item [3.] Unteralgebren von $(\mathbb{N},\cdot,1)$
  \item [4.] Kongruenzrelationen von $(\mathbb{N},\cdot,1)$
  \item [5.] Unteralgebren von $(\mathbb{N},+,0,\cdot,1)$
  \item [6.] Kongruenzrelationen von $(\mathbb{N},+,0,\cdot,1)$
\end{itemize}
\end{algebraUE}
\begin{solution}
\leavevmode \\
\begin{itemize}
  \item [1.] Als erstes haben wir die beiden trivialen Unteralgebren: $\{0\}$ und $\mathbb{N}$. \\
  Dann haben wir zu jeder natürlichen Zahl $n \geq 2: n\cdot\mathbb{N}$. \\
Ebenfalls für jede natürliche Zahl $n \geq 1: \mathbb{N}\backslash\{1,\dots,n\}$. \\
Im allgemeinen können wir zu jeder beliebiger Teilmenge $X \subset \mathbb{N}$
die erzeugte Unteralgebra mit $\{\sum_{j=1}^nx_jn_j: n, n_j \in \mathbb{N}, x_j \in X \}$
bestimmen. Damit haben wir ein System um alle Unteralgebren vollständig zu beschreiben.
Natürlich werden wir damit öfters die gleichen Unteralgebren erzeugen,
allerdings haben wir mit $|2^{\mathbb{N}}| = |\mathbb{R}|$ zumindest eine obere
Schranke für die Kardinalität der Unteralgebren gefunden. Bleibt noch die Frage,
ob wir die Schranke noch auf abzählbar unendlich herunterschrauben können. \\
Wir können zumindest zeigen, dass jede Unteralgebra endlich erzeugt ist.
Sei $U$ eine beliebige Unteralgebra von $(\mathbb{N},+,0)$ und sei angenommen,
dass $U$ nicht endlich erzeugt ist.
Dazu definieren wir uns die Folge $(a_n)_{n \in \mathbb{N}}$ induktiv wie folgt
\begin{align*}
  a_0 := \min U \\
a_{n+1} := \min U \backslash \langle a_0,\dots,a_n\rangle
\end{align*}
Betrachte nun die endliche Menge $(a_0,\dots,a_m)$ mit einem $m > a_0$.
Nun folgt mit dem Schubfachprinzip $\exists k,n \leq m: a_k \mod a_0 = a_n \mod a_0$.
Wir nehmen o.B.d.A. an $a_k < a_n$. Dann gilt also
\begin{align*}
  \exists c \in \mathbb{N}: a_n = a_k + ca_0 \in \langle a_0,\dots,a_m \rangle
\end{align*}
Dies ist ein Widerspruch zu unserer Konstruktion und $U$ muss somit endlich erzeugt sein.
Laut Proposition 2.3.1.22. wissen wir damit jetzt, dass der Unteralgebrenverband
mit der Mengeninklusion eine Noethersche Halbordnung ist, also dass es keine
unendlich aufsteigenden Ketten gibt.
Ebenfalls äquivalent dazu ist, dass jede nichtleere Teilmenge ein Maximum hat.
\item [2.] Wieder haben wir zwei triviale Kongruenzrelationen am Anfang:
Die Allrelation $\mathbb{N} \times \mathbb{N}$ und die Identitätsrelation
$\{(n,n): n \in \mathbb{N}\}$. \\
Dann können wir die Modulo-Kongruenzrelationen definieren. Für jede natürliche Zahl
$n \geq 1$ ist $a \sim b: \iff a \mod n = b \mod n$ eine Kongruenzrelation, welche
die Partition in die jeweiligen Restklassen induziert.
Also haben wir zumindest abzählbar unendlich viele Kongruenzrelationen.
\item [3.] Jede Teilmenge $A$ der Potenzmenge der Primzahlen induziert eine
Unteralgebra von $(\mathbb{N},\cdot,1)$ mit
\begin{align*}
  U_A := \left\{\prod_{j=1}^na_j^{n_j}: n, n_j \in \mathbb{N}, a_j \in A\right\}.
\end{align*}
Diese Unteralgebren sind aufgrund der Eindeutigkeit
der Primfaktorzerlegung allesamt paarweise verschieden.
Damit können wir zwar noch nicht alle möglichen Unteralgebren
vollständig beschreiben, allerdings ist aufgrund $|2^\mathbb{P}| = |2^\mathbb{N}|
= |\mathbb{R}|$ die Kardinalität der Unteralgebren auf jeden
Fall überabzählbar. \\
Sei $X \subset \mathbb{N}$ beliebig.
Dann ist die Menge
\begin{align*}
  U_X := \left\{\prod_{j=1}^nx_j^{n_j}: n, n_j \in \mathbb{N}, x_j \in X\right\}.
\end{align*}
eine Unteralgebra und offensichtlicherweise können damit auch alle möglichen
Unteralgebren beschrieben werden, allerdings ist diese Darstellung im Allgemeinen
nicht mehr eindeutig.
Also haben wir sowohl eine injektive, als auch eine surjektive Abbildung von
einer Teilmenge der Potenzmenge der natürlichen Zahlen in die Menge aller
Unteralgebren von $(\mathbb{N},\cdot,1)$ gefunden. \\
Ich denke, dass es nicht möglich ist, eine elementare bijektive Abbildung zu finden.
\item [4.] Wieder gibt es die beiden trivialen Relationen: Allrelation + Identitätsrelation.
Weiters ist auch die Modulo 2 Relation eine Kongruenzrelation.
Wiederum eine weitere Kongruenzrelation ist durch
\begin{align*}
  a = \prod_{p \in \mathbb{P}}p^{a_p} \sim b = \prod_{p \in \mathbb{P}}p^{b_p}: \iff \sum_{p \in \mathbb{P}} a_p = \sum_{p \in \mathbb{P}} b_p
\end{align*}
definiert.
\item [5.] Jede Unteralgebra muss auf jeden Fall $\{0,1\}$ enthalten. Aufgrund
der Abgeschlossenheit unter der Addition und dem Induktionsprinzip
muss jede Unteralgebra bereits ganz $\mathbb{N}$ enthalten.
\item [6.] Wieder gibt es zwei triviale Relationen mit der Allrelation und der
Identitätsrelation.
\newline
***Ab hier bitte mal drüberschauen, bin mir nicht sicher, ob irgendwo ein Fehler drin ist...falls nicht, sollte das für 2. eigenltich auch schon gelten oder?*** Sei $\sim$ eine Kongruenzrelation ungleich der Identitätsrelation. Dann gibt es ein kleinstes $a_{0}$, das mit mindestens einem anderen Element in Relation steht. In der Menge aller $b$, die mit $a_{0}$ in Relation stehen, gibt es wieder ein kleinstes Element, das wir $b_{0}$ nennen. Da $\sim$ eine Kongruenzrelation bezüglich $+$ ist und $1 \sim 1$, gilt (mit $m := b_{0}-a_{0} > 0$)
\begin{align*}
  a_{0} \sim b_{0} \Rightarrow a_{0}+1 \sim b_{0}+1 \Rightarrow \dots \Rightarrow a_{0}+(m-1) \sim b_{0}+(m-1) \Rightarrow b_{0} \sim b_{0}+m \Rightarrow \dots
\end{align*}
Also ist $\sim$ eine Kongruenzrelation mit $a_{0}$ einelementigen Äquivalenzklassen und maximal $m$ weiteren Quasi-Modulo-Äquivalenzklassen (beginnend ab $a_{0}$).
Da also für alle solchen minimalen $a_{0}$ und $b_{0}$ also mindestens eine (die eben genannte (feinste) unter der Voraussetzung $a_{0} \sim b_{0}$ - die Verträglichkeit dieser mit der Multiplikation ist aufgrund der modulo-Homomorphie gegeben), aber maximal endlich viele verschiedene Kongruenzrelationen in Frage kommen, sieht man, dass es auf jeden Fall unendlich viele, aber höchstens abzählbar viele Kongruenzrelationen von $(\mathbb{N},+,0,\cdot,1)$ gibt.
\end{itemize}
\end{solution}
