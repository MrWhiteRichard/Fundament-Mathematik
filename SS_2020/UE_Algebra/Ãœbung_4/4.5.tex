\begin{algebraUE}{152}
Seien $G$ eine Gruppe und $A,B,\dots \subseteq G$ Teilmengen.
\begin{itemize}
  \item [1.] Aus $A,B \leq G$ folgt im Allgemeinen nicht $AB \leq G$.
  \item [2.] Aus $A \vartriangleleft G$ und $B \vartriangleleft G$ folgt $BA = AB \leq G$.
  \item [3.] Aus $A,B \vartriangleleft G$ folgt $AB \vartriangleleft G$.
  \item [4.] Im Normalteilerverband ist das Supremum zweier oder, allgemeiner,
  endlich vieler Normalteiler $N_1,\dots,N_k \vartriangleleft G$ gegeben durch
  das Komplexprodukt $N_1N_2\dots N_k$.
  \item [5.] Sind $N_i \vartriangleleft G, i \in I \neq \emptyset$, Normalteiler
  der Gruppe $G$, so ist ihr Supremum $N := \sup_{i \in I}N_i$ im Verband aller
  Normalteiler gegeben durh die Vereinigung aller endlichen Komplexprodukte
  $N_{i_1}\dots N_{i_n}, n \in \mathbb{N}$ und $i_1,\dots,i_n \in I$.
\end{itemize}
\end{algebraUE}
\begin{solution}
\leavevmode \\
\begin{itemize}
  \item [1.] Wähle als Gruppe die symmetrische Gruppe $S_3 = \{e,d,d^2,s_1,s_2,s_3\}$
  mit den Permutationen
  \begin{align*}
  e &= \begin{pmatrix}
      1 & 2 & 3 \\
      1 & 2 & 3
    \end{pmatrix} \qquad
  d = \begin{pmatrix}
    1 & 2 & 3 \\
    2 & 3 & 1
  \end{pmatrix} \qquad
  d^2 = \begin{pmatrix}
    1 & 2 & 3 \\
    3 & 1 & 2
  \end{pmatrix} \\
  s_1 &= \begin{pmatrix}
    1 & 2 & 3 \\
    1 & 3 & 2
  \end{pmatrix} \qquad
  s_2 = \begin{pmatrix}
    1 & 2 & 3 \\
    3 & 2 & 1
  \end{pmatrix} \qquad
  s_3 = \begin{pmatrix}
    1 & 2 & 3 \\
    2 & 1 & 3
  \end{pmatrix}
  \end{align*}
  Dann existieren als Untergruppen davon $A := \{e,s_1\}, B := \{e,s_2\}$. \\
  Mit $s_1 \circ s_2 = d$ ist $AB = \{e,s_1,s_2, d\}$ allerdings keine Untergruppe, da $d \circ d = d^2 \notin AB$.
  \item [2.] Sei $A$ beliebiger Normalteiler, $B$ beliebige Untergruppe von $G$.
  Aufgrund der Normalteilereigenschaft von $A$ gilt $xA = Ax$ für alle
  $x \in G$ und daher
  \begin{align*}
    AB = \bigcup_{b \in B}Ab = \bigcup_{b \in B}bA = BA.
  \end{align*}
  Da das neutrale Element sowohl in $A$, als auch in $B$ enthalten ist, gilt auch
  $ee = e \in AB$. \\
  Seien nun $x= a_1b_1, y = a_2b_2 \in AB$ beliebig mit $a_1,a_2 \in A, b_1,b_2 \in B$. \\
  \begin{align*}
    xy = a_1b_1a_2b_2 = \widetilde{a}b_1b_2 \in AB.
  \end{align*}
  Nun gilt $b_1a \in b_1 A = A b_1$ also gibt es $\tilde{a} \in A: b_1a = \tilde{a}b_1$.
  \begin{align*}
    xy = a_1\tilde{a}b_1b_2 \in AB.
  \end{align*}
  Sei weiters $x = ab \in AB$ beliebig. Da $B$ und $A$ beide Gruppen sind, gilt
  auch $b^{-1}a^{-1} \in BA = AB$ und $AB$ ist somit Untergruppe von $G$.
  \item [3.] Da $A,B$ beide Normalteiler sind, gilt
  \begin{align*}
    xAB = AxB = ABx.
  \end{align*}
  Damit ist $AB$ ebenso Normalteiler von $G$.
  \item [4.] Seien $N_1, ..., N_n$ beliebige Normalteiler von $G$. Durch
  induktive Fortführung von Punkt 3 zeigt man, dass $N_1\cdots N_n$ auch Normalteiler ist.
  Jede obere Schranke $O$ der Menge $\{N_i : 1 \leq i \leq n\}$ im Normalteilerverband
  ist insbesondere eine Untergruppe von $G$. Daher gilt
  \begin{align*}
    \forall (n_1,\dots, n_n) \in N_1 \times \dots \times N_n: n_1\cdots n_n \in O,
  \end{align*}
  also $N_1\cdots N_n \subseteq O$;
  klarerweise gilt auch $N_i \subseteq N_1\cdots N_n$ für alle $i \in \{1,\dots,n\}$.
  $N_1\cdots N_n$ ist also die kleinste obere Schranke im Normalteilerverband und damit das gesuchte Supremum.
  \item [5.] Seien $N_i \vartriangleleft G, i \in I \neq \emptyset$ beliebig,
  $N$ die Vereinigung aller endlichen Komplexprodukte davon.
  Sei $y \in xN$ beliebig. Also existiert ein endliches Komplexprodukt $N_I^*$,
  welches nach Punkt 4 wieder ein Normalteiler ist, sodass $y \in xN_I^* = N_I^*x$.
  Damit folgt
  \begin{align*}
    xN \subseteq Nx
  \end{align*}
  und $N$ ist damit ein Normalteiler. Weiters gilt für jedes $i \in I: N_i \subseteq N$.
  Sei nun $O$ beliebige obere Schranke der Menge $N_i, i\in I$. Dann folgt
  \begin{align*}
    \forall n \in \N, x_{i_1} \in N_{i_1},\dots, x_{i_n} \in N_{i_n}: x_{i_1}\cdots x_{i_n} \in O.
  \end{align*}
  Es gilt also für jedes beliebige endliche Komplexprodukt $N_I^* \subseteq O$
  und somit auch für die Vereinigung $N \subseteq O$.
  Damit ist $N$ also das gesuchte Supremum.
\end{itemize}
\end{solution}
