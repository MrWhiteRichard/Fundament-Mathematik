\begin{algebraUE}{196}
\begin{enumerate}
  \item Zeigen Sie: Ist $V$ ein endlich-dimensionaler Vektorraum über einem
  Körper, so ist $\End(V)$ als Ring einfach (besitzt also nur die trivialen Ideale).
  \item Zeigen Sie, dass dies für unendlich-dimensionale Vektorräume nicht gilt. \\
  \textit{Hinweis:} Betrachten Sie alle Endomorphismen mit endlich-dimensionalem Bild.
\end{enumerate}
\end{algebraUE}
\begin{solution}
\leavevmode \\
\begin{enumerate}
  \item Sei $I \neq \{0\}$ ein Ideal von $\End(V)$. Dann existiert ein $f \neq 0 \in I$.
  Dann ist $\ker(f) = f^{-1}(\{0\}) \neq V$ ein Normalteiler auf $V$.
  Sei $x \notin \ker(f)$ beliebig. Ich gebe mich vorerst geschlagen.
\end{enumerate}
\end{solution}
