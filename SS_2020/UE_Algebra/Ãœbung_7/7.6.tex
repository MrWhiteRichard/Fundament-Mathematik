\begin{algebraUE}{215}
Jede archimedisch angeordnete abelsche Gruppe $G$ lässt sich als solche isomorph
in die geordnete additive Gruppe $\R$ der reellen Zahlen einbetten. (Umgekehrt
ist jede additive Untergruppe von $\R$ archimedisch angeordnet.)
Ist $\iota: G \mapsto \R$ eine solche isomorphe Einbettung, so sind alle anderen
gegeben durch sämtliche Abbildungen $\lambda_l$ mit reellem $\lambda > 0$. \\
\textit{Anleitung für die Existenz von $\iota$}: Gehen Sie für nichttriviales
$G$ von einem positiven Element $g \in G$ aus, das Sie auf die reelle Zahl
$1 = \iota(g)$ abbilden. Wegen der archimedischen Eigenschaft definiert das
einen eindeutigen ordnungsverträglichen Homomorphismus $\iota$, der
(wieder wegen der archimedischen Eigenschaft) sogar injektiv sein muss.
\end{algebraUE}
\begin{solution}
Wir setzen o.B.d.A. $G \neq \{0\}$, also $\exists 0 \neq g \in G^+$.
Wir setzen $\iota(g) = 1$ und haben damit
\begin{align*}
  \forall k \in \Z: \iota(kg) = k\iota(g) = k.
\end{align*}
Für $G = \Z g$ sind wir damit bereits fertig.
Anderenfalls, wähle $h \in G^+\backslash \Z g$ und setze
\begin{align*}
  \forall n \in \N\backslash \{0\}: m_n^h &:= \min\{k \in \N: 2^nh \leq kg\}.
\end{align*}
Damit $\iota$ ein Ordnungisomorphismus werden kann, muss gelten
\begin{align*}
  2^n \iota(h) &= \iota(2^nh) \leq \iota(m_n^hg) = m_n^h \iff \iota(h) \leq \frac{m_n^h}{2^n} \\
  2^n \iota(h) &= \iota(2^nh) \geq \iota((m_n^h - 1)g) = m_n^h - 1 \iff \iota(h) \geq \frac{m_n^h - 1}{2^n},
\end{align*}
also
\begin{align*}
  \forall n \in \N: \iota(h) \in \left[\frac{m_n^h-1}{2^n},\frac{m_n^h}{2^n}\right]
  \iff \iota(h) \in \bigcap_{n \in \N} \left[\frac{m_n^h-1}{2^n},\frac{m_n^h}{2^n}\right].
\end{align*}
Es gilt
\begin{align*}
  (m_n - 1)g \leq 2^nh \leq m_ng.
\end{align*}
Es folgt
\begin{align*}
  2^{n+1}h &\leq 2\cdot2^nh \leq 2 m_n g \implies m_{n+1} \leq 2m_n
  \iff \frac{m_{n+1}}{2^{n+1}} \leq \frac{m_n}{2^n} \\
  2^{n+1}h &\geq 2 (m_n - 1)g \implies m_{n+1} - 1 \geq 2 (m_n - 1)
  \iff \frac{m_{n+1}-1}{2^{n+1}} \geq \frac{m_n-1}{2^n}\\
  &\iff \left[\frac{m_{n+1} - 1}{2^{n+1}},\frac{m_{n+1}}{2^{n+1}}\right] \subset \left[\frac{m_{n} - 1}{2^{n}},\frac{m_{n}}{2^{n}}\right]
\end{align*}
Induktiv folgt dann, dass mit $[\frac{m_n^h-1}{2^n},\frac{m_n^h}{2^n}]_{n \in \N}$
eine absteigende Mengenfolge vorliegt und der Schnitt darüber ist somit nicht leer.
Aufgrund der Vollständigkeit der reellen Zahlen existiert also
\begin{align*}
  \iota(h) = x = \lim_{n \rightarrow \infty} \frac{m_n^h}{2^n} = \lim_{n \rightarrow \infty} \frac{m_n^h-1}{2^n}
\end{align*}
und somit wird $\iota$ bereits auf ganz $G$ eindeutig festgelegt. \\
Jetzt müssen wir noch zeigen, dass diese Konstruktion auch ein Homomorphismus ist.
\begin{align*}
  \iota(0) &= \iota(0g) = 0\iota(g) = 0 \\
  \iota(-h) &= -\iota(h)
\end{align*}
Für die Verträglichkeit mit der Addition müssen wir etwas mehr arbeiten:
\begin{align*}
  \forall n \in \N:& 2^nh_1 \leq m_n^{h_1}g \land 2^nh_2 \leq m_n^{h_2}g \\
  &\implies 2^n(h_1 + h_2) \leq (m_n^{h_1} + m_n^{h_2})g \implies m_n^{h_1} + m_n^{h_2} \geq m_n^{h_1 + h_2} \\
  \forall n \in \N:& 2^nh_1 \geq (m_n^{h_1}-1)g \land 2^nh_2 \geq (m_n^{h_2}-1)g \\
  &\implies 2^n(h_1 + h_2) \geq (m_n^{h_1} + m_n^{h_2} - 2)g \implies m_n^{h_1} + m_n^{h_2} \leq m_n^{h_1 + h_2} + 1
\end{align*}
Daraus folgt
\begin{align*}
  \iota(h_1) + \iota(h_2) = \lim_{n \rightarrow \infty} \frac{m_n^{h_1} + m_n^{h_2}}{2^n}
  = \lim_{n \rightarrow \infty} \frac{m_n^{h_1 + h_2}}{2^n} = \iota(h_1 + h_2).
\end{align*}
Nun zur Injektivität von $\iota$: Sei $h \in G^+\backslash\{0\}$ beliebig.
Dann existiert ein $n \in \N: g < nh$, weil die Gruppe $G$
archimedisch angeordnet ist. Daraus folgt $0 < 1 = \iota(g) < \iota(nh) = n\iota(h)$.
Daraus erhalten wir schließlich die Injektivität
und gleichzeitig auch die Ordnungsverträglichkeit, denn für $a > b \in G$ gilt
\begin{align*}
  \iota(a) - \iota(b) = \iota(a-b) > 0 \implies \iota(a) > \iota(b).
\end{align*}
Schließlich noch zu Eindeutigkeit von $\iota$: \\
Sei nun $\alpha$ eine weiter isomorphe Einbettung mit
\begin{align*}
  \alpha(g) = \lambda = \lambda \iota(g) = \iota(\lambda g).
\end{align*}
Wie zuvor ist dadurch schon
\begin{align*}
  \forall n \in \N: \lambda \frac{m_n^h - 1}{2^n} = \frac{m_n^h - 1}{2^n}\alpha(g) \leq \alpha(h) \leq \frac{m_n^h}{2^n}\alpha(g)
  = \lambda \frac{m_n^h}{2^n}
\end{align*}
$\alpha(h)$ eindeutig festgelegt und es gilt $\alpha = \lambda \iota$.
\end{solution}
