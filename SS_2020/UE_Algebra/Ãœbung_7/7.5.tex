\begin{algebraUE}{210}
Sei $p$ eine Primzahl.
\begin{enumerate}
  \item Wie viele Untergruppen hat $C_p \times C_p$?
  \item Wie viele Untergruppen hat $C_{p^2}$?
\end{enumerate}
\textit{Hinweis:} Überlegen Sie sich zuerst, dass alle nichttrivialen Untergruppen
zyklisch sein müssen.
\end{algebraUE}
\begin{solution}
Wir wissen, dass jede Untergruppe einer zyklischen Gruppe wieder zyklisch ist.
\begin{enumerate}
  \item Damit hat $C_p$ klarerweise nur die beiden trivialen Untergruppen.
  \begin{align*}
    C_p \times C_p = \{(a + p\Z, b + p\Z): a,b \in \Z\}
  \end{align*}
  Sei $U \leq C_p \times C_p$ beliebig.
  \begin{itemize}
    \item Fall 1: $\exists (\overline{a},\overline{b} \in U: \overline{a} \neq \overline{b})$
    \begin{itemize}
      \item Fall 1.1: $\overline{a} \neq 0$
      \begin{itemize}
        \item Fall 1.1.1: $\overline{b} = 0$ \\
        Nach Proposition 3.2.4.9.ist $\overline{a}$ ein erzeugendes Element von $C_p$ und
        \begin{align*}
          \langle (\overline{a}, \overline{0})\rangle  = C_p \times \{0\}
        \end{align*}
        ist eine Untergruppe von $U$ und also auch von $C_p \times C_p$.
        \item Fall 1.1.2: $\overline{b} \neq 0$. \\
        Dann sind $\overline{a},\overline{b}$ erzeudende Elemente und es gibt $\alpha, \beta \in \N$:
        \begin{align*}
          \alpha \overline{a} = 0 \land \beta \overline{b} = 0
          \land \alpha \overline{b} \neq 0 \land \beta \overline{a} \neq 0,
        \end{align*}
        weil
        \begin{align*}
          \alpha (\overline{a},\overline{b}) &= (0,\alpha \overline{b}) \in U \\
          \beta (\overline{a}, \overline{b}) &= (\beta \overline{a}, 0) \in U.
        \end{align*}
        Es gilt also $U = C_p \times C_p$.
      \end{itemize}
      \item Fall 1.2: $\overline{a} = 0, \overline{b} \neq 0$. \\
      Analog zu Fall 1.1
    \end{itemize}
    \item Fall 2: $\exists (\overline{a},\overline{b}) \in U: \overline{a} = \overline{b}$. \\
    \begin{itemize}
      \item Fall 2.1: $U = \{0\}$
      \item Fall 2.2: $U \neq \{0\}$. \\
      Also existiert $(\overline{a},\overline{a}) \in U: \overline{a} \neq 0$ und
      \begin{align*}
        \{(\overline{a},\overline{a}); \overline{a} \in C_p\} \leq U \leq C_p \times C_p.
      \end{align*}
    \end{itemize}
  \end{itemize}
  Insgesamt gibt es also 5 Untergruppen: $\{(\overline{0},\overline{0})\},
  \{\overline{0}\} \times C_p, C_p \times \{\overline{0}\},
  \{(\overline{a},\overline{a}); \overline{a} \in C_p\}, C_p \times C_p$.
  \item Da $1,p,p^2$ die drei einzigen Teiler von $C_{p^2}$ sind, hat
  $C_{p^2}$ genau 3 Untergruppen.
\end{enumerate}
\end{solution}


\begin{algebraUE}{212}
  Sei $p$ eine Primzahl. Wie viele Automorphismen hat $C_p \times C_p$? \\
  \textit{Hinweis:} Verwenden Sie Ihr Wissen aus der Linearen Algebra.
\end{algebraUE}
\begin{solution}
  Sei $\phi: C_p \times C_p \rightarrow C_p \times C_p$ ein Automorphismus.
  Dann gilt
  \begin{align*}
    \varphi(0,0) &= (0,0) \\
    \varphi(1,0) &= (a,b) \neq (0,0)
  \end{align*}
  Wieder machen wir eine Fallunterscheidung:
  \begin{itemize}
    \item Fall 1: $a \neq 0$
    \begin{itemize}
      \item Fall 1.1: $b \neq 0, b \neq a$. \\
      Es ist $\varphi(C_p \times \{0\})$ eine Untergruppe von $C_p \times C_p$
      mit gleich vielen Elementen wie $C_p \times \{0\}$, was einen Widerspruch ergibt-
      \item Fall 1.2: $b = 0$. \\
      Dann ist $\varphi(C_p \times \{0\}) = C_p \times \{0\}$, es gibt also $p$ Möglichkeiten.
      \item Fall 1.3: $b = a$. \\
      Dann ist $\varphi(C_p \times \{0\}) = \{(a,a): a \in C_p\}$, es gibt also wieder
      $p$ Möglichkeiten.
    \end{itemize}
    \item Fall 2: $a = 0$. \\
    Dann ist $b \neq 0$ und $\varphi(C_p \times \{0\}) = \{0\} \times C_p$.
    Für $\varphi(0,1)$ funktioniert alles analog, außer
    \begin{align*}
      \forall (x,y) \in \varphi(C_p \times \{0\}): (x,y) \notin \varphi(\{0\} \times C_p).
    \end{align*}
    Also: Für $\varphi(1,0)$ gibt es $3p$ Möglichkeiten und für $\varphi(1,0)$
    gibt es unabhängig davon $2p$ Möglichkeiten. Mit diesen beiden Werten ist
    der Automorphismus bereits eindeutig bestimmt und wir erhalten insgesamt
    $6p^2$ mögliche Automorphismen.
  \end{itemize}
\end{solution}
