\begin{algebraUE}{183}
\begin{enumerate}
  \item Sei $R$ ein Unterring mit $1$ eines Körpers $K$. Dann ist
  \begin{align*}
    K^{\prime} := \left\{\frac{p}{q}: p,q \in R, q \neq 0 \right\}
  \end{align*}
  ein Unterkörper von $K$.
  \item Der Körper $K^{\prime}$ aus dem ersten Teil ist der kleinster Unterkörper
  von $K$, der $R$ enthält, symbolisch $K^{\prime} = \langle R \rangle_{\text{Körper}}$.
  Explizit bedeutet das: Jeder Unterkörper $K^{\primeprime}$ von $K$ mit $R \subseteq K^{\primeprime}$
  umfasst $K^{\prime}$.
  \item In derselben Situation ist $K$ (zusammen mit der Inklusionsabbildung) genau
  dann ein Quotientenkörper von $R$, wenn $K = K^{\prime}$ gilt.
  \item Ist $\iota: R \rightarrow K$ eine isomorphe Einbettung des Integritätsbereichs $R$
  in einen Körper $K$ und $Q$ der von $\iota(R)$ erzeugte Unterkörper von $K$,
  so ist $Q$ zusammen mit $\iota$ ein Quotientenkörper von $R$.
\end{enumerate}
\end{algebraUE}
\begin{solution}
\leavevmode \\
\begin{enumerate}
  \item Da der Körper $K$ kommutativ ist, ist auch $K^{\prime}$ bereits kommutativ.
  Mit den Rechenregeln
  \begin{align*}
    \frac{p_1}{q_1} + \frac{p_2}{q_2} = \frac{p_1q_2 + p_2q_1}{q_1q_2} \\
    \frac{p_1}{q_1}   \frac{p_2}{q_2} = \frac{p_1p_2}{q_1q_2}
  \end{align*}
  sieht man auch die Abgeschlossenheit bezüglich der Addition, sowie Multiplikation.
  Das Distributivgesetz in $K^{\prime}$ folgt ebenso aus dem Distributivgesetz in $K$ selbst.
  Damit bleiben nur noch die Nullteilerfreiheit, sowie die Existenz der multiplikativen Inversen zu überprüfen.
  Seien also $\frac{p_1}{q_1}, \frac{p_2}{q_2} \in K^{\prime}$ mit $\frac{p_1}{q_1}   \frac{p_2}{q_2} = 0$ beliebig.
  Dann gilt
  \begin{align*}
    0 = \frac{p_1}{q_1}   \frac{p_2}{q_2} = \frac{p_1p_2}{q_1q_2} \implies p_1p_2 = 0
  \end{align*}
  und aufgrund der Nullteilerfreiheit in $K:$
  \begin{align*}
    p_1 = 0 \lor p_2 = 0 \implies \frac{p_1}{q_1} = 0 \lor \frac{p_2}{q_2} = 0.
  \end{align*}
  Nun zur Existenz multiplikativer Inversen: Sei $\frac{p}{q} \in \K^{\prime}: p \neq 0$ beliebig.
  Dann ist $\frac{q}{p} \in \K^{\prime}$ und es gilt
  \begin{align*}
    \frac{p}{q}\frac{q}{p} = \frac{pq}{qp} = 1.
  \end{align*}
  Damit ist $K^{\prime}$ ein Unterkörper von $K$.
  \item Sei $K^{\primeprime}$ nun ein beliebiger Körper, welcher $R$ enthält.
  Damit muss vorerst $K^{\primeprime} \subseteq R$ gelten. Aufgrund der Abgeschlossenheit
  unter der multiplikativen Inversenbildung muss auch gelten
  \begin{align*}
    K^{\primeprime} \supseteq \{\frac{1}{q}: q \neq 0 \in R\} \cup \{\frac{p}{1}: p \in R\}
  \end{align*}
  Aufgrund der Abgeschlossenheit bezüglich der Multiplikation gilt schließlich auch
  \begin{align*}
    K^{\primeprime} \supseteq  \{\frac{p}{q}: q \neq 0 \in R\} = K^{\prime}.
  \end{align*}
  \item Gelte zuerst $K = K^{\prime}$. \\
  Damit $(K,\iota)$ der bis auf Isomorhie eindeutig bestimmte Quotientenkörper ist, muss
  noch gelten, dass für jeden anderen Körper $K^{\primeprime}$ mit isomorpher Einbettung
  $\iota^{\prime}: R \rightarrow K^{\primeprime}$ eine eindeutige isomorphe Einbettung
  $\varphi: K \rightarrow K^{\primeprime}$ gibt, sodass $\iota^{\prime}(r) = \varphi \circ \iota$.
  Im vorherigen Schritt haben wir gezeigt, dass $K = K^{\prime} \subseteq K^{\primeprime}$ erfüllt.
  Damit erfüllt die triviale Einbettungsabbildung $\varphi: K \rightarrow K^{\primeprime}, x \mapsto x$
  die Bedingung $\iota^{\prime}(r) = \varphi \circ \iota$. Ich nehme stark an, dass
  durch die Homomorphie-Bedingungen $\varphi$ auf $K$ bereits eindeutig festgelegt wird,
  kanns aber grad nicht zeigen. \\
  Die Rückrichtung zeigen wir mittels Kontraposition, gelte also $K \supsetneq K^{\prime}$.
  Nachdem $\varphi(K^{\prime}) = K^{\prime}$ gelten muss, kann $\varphi$ nicht injektiv
  und somit keine isomorphe Einbettung sein.
  \item Sei $(Q^{\prime},\iota^{\prime})$ ein weiterer Körper mit Einbettungsabbildung
  $\iota: R \rightarrow Q^{\prime}$.
\end{enumerate}

\end{solution}
