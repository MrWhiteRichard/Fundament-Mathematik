\begin{algebraUE}{171}
Sei $R$ ein Ring mit $1$ und $A \subseteq R$. Bezeichne $I$ den Schnitt aller Ideale,
$J \vartriangleleft R$ mit $A \subseteq J$. ($I$ ist also das kleinste $A$ umfassende
Ideal in $R$, genannt das von $A$ erzeugte Ideal, symbolisch $I = (A)$, im Fall
$A = \{a_1,\dots,a_n\}$ auch $I = (a_1,\dots,a_n)$.) Dann gilt:
\begin{enumerate}[label = (\arabic*)]
  \item $I$ ist die Menge aller
  \begin{align*}
    \sum_{i = 1}^n r_ia_is_i + \sum_{j = 1}^{m^{\prime}}r_j^{\prime}b_j +
    \sum_{k = 1}^{n^{\prime}}c_ks_k^{\prime} + \sum_{l = 1}^{m}d_l
  \end{align*}
  mit $n,m^{\prime}, n^{\prime}, k \in \N, a_i, b_j, c_k, d_l \in A$ und
  $r_i, s_i, r_j^{\prime}, s_k^{\prime} \in R$.
  \item Hat $R$ ein Einselement, so ist $I$ auch darstellbar als die Menge aller
  \begin{align*}
    \sum_{i = 1}^n r_ia_is_i
  \end{align*}
  mit $n \in \N, a_i \in A$ und $r_i \in R$.
  \item Ist $R$ kommutativ mit $1$, so ist $I$ darstellbar als die Menge aller
  Summen (Linearkombinationen)
  \begin{align*}
    \sum_{i=1}^n r_ia_i
  \end{align*}
  mit $n \in \N, a_i \in A, r_i \in R$. Ist außerdem $A = \{a\}$ einelementig, so ist
  \begin{align*}
    I = (a) = \{ra: r \in R\}.
  \end{align*}
\end{enumerate}
\end{algebraUE}
\begin{solution}
\leavevmode \\
\begin{enumerate}[label = (\arabic*)]
  \item
\end{enumerate}
\end{solution}
