\begin{algebraUE}{168}
Sei $G := S_4$. Wir geben die Elemente von $G$ in Zyklenschreibweise an. Sei
$U$ die vom Element $(1234)$ erzeugte Untergruppe und $N = \{\id, (12)(34),(13)(24),(14)(23)\}$.
Begründen Sie, warum $N \vartriangleleft S_4$ ein Normalteiler ist. Bestimmen
Sie die Gruppen $NU, N\cap U, NU/N, U/(N \cap U)$ und geben Sie den kanonischen
Isomorphismus zwischen $NU/N$ und $U/(N \cap U)$ explizit an.
\end{algebraUE}
\begin{solution}
\begin{align*}
  U = \{\id, (1234),(13)(24),(4321)\}
\end{align*}
Zuerst zeigen wir, dass $N$ eine Untergruppe von $S_4$ ist.
\begin{align*}
  (12)(34) \circ (13)(24) &= (14)(23) \\
  (13)(24) \circ (12)(34) &= (14)(23) \\
  (12)(34) \circ (14)(23) &= (13)(24) \\
  (14)(23) \circ (12)(34) &= (13)(24) \\
  (13)(24) \circ (14)(23) &= (12)(34) \\
  (14)(23) \circ (13)(24) &= (12)(34)
\end{align*}
Zusätzlich bemerkt man leicht, dass jedes Element aus $N$ zu sich selbst invers ist.
Wir wissen, dass $N$ genau dann ein Normalteiler ist, wenn $\forall g \in G: \pi_g(N) = N$ gilt,
also wenn $N$ invariant unter allen inneren Automorphismen von $G$ ist.

\begin{align*}
  NU = \{\id,(12)(34),(13)(24),(14)(23),(24),
\end{align*}
\end{solution}
