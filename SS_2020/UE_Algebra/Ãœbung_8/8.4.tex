\begin{algebraUE}{291}
Seien $t(x_1,\dots,x_n)$ und $t^{\prime}(x_1,\dots,x_n)$ Terme (in einer festen
Sprache $L$), in denen jeweils nur die Variablen $x_1,\dots,x_n$ (oder Teilmengen
davon) vorkommen. Sei $\mathcal{V}$ eine Varietät (zur Sprache $L$). Für $C \in \mathcal{V}$
schreiben wir $C \vDash t \approx t^{\prime}$ (gelesen: ``Das Gesetz $t = t^{\prime}$
gilt in $C$'') als Abkürzung für
\begin{align*}
  \forall c_1,\dots,c_n \in C: t(c_1,\dots,c_n) = t^{\prime}(c_1,\dots,c_n).
\end{align*}
Sei $F \in \mathcal{V}$ frei über der $n$-elementigen Menge $\{b_1,\dots,b_n\}$ in $\mathcal{V}$.
Zeigen Sie, dass die folgenden Aussagen äquivalent sind, und schließen Sie daraus, dass 4.1.3.4. gilt:
\begin{enumerate}[label = (\alph*)]
  \item In $F$ gilt $t(b_1,\dots,b_n) = t^{\prime}(b_1,\dots,b_n)$.
  \item Für alle $C \in \mathcal{V}$ gilt $C \vDash t \approx t^{\prime}$.
  \item Es gilt $F \vDash t \approx t^{\prime}$.
\end{enumerate}
\end{algebraUE}
\begin{solution}
Beweis.
\end{solution}
