\begin{algebraUE}{228}
Auf dem Körper $\Q(x)$ der gebrochen rationalen Funktionen sei eine Relation $\leq$
definiert durch $q_1(x) \leq q_2(x)$, falls $q_1 = q_2$ oder es ein $r_0 > 0$ gibt
mit $\forall r \in R: r > r_0 \implies q_1(r) < q_2(r)$. Zeigen Sie:
\begin{enumerate}
  \item Die so definierte Relation $\leq$ macht $\Q(x)$ zu einem nichtarchimedisch
  angeordneten Körper.
  \item $Q(x)$ lässt sich als angeordneter Körper solcher in jeden anderen
  nichtarchimedisch angeordneten Körper isomorph einbetten.
\end{enumerate}
\end{algebraUE}
\begin{solution}
\begin{enumerate}
  \item Zuerst ist also zu zeigen, dass die gegebene Relation eine Totalordnung
  auf $\Q(x)$ definiert. Dazu betrachte $p \neq q \in \Q(x)$ beliebig und sei $n := \grad(p - q)$
  \begin{align*}
    \lim_{x \to \infty} p(x) - q(x) = \lim_{x \to \infty}\underbrace{x^n}_{> 0}
    \underbrace{\left((p_n - q_n) + \sum_{k \in \Z}(p_k - q_k)x^{k-n}\right)}_{\to (p_n - q_n)}.
  \end{align*}
  Für hinreichend großes $r$ hat $p(x) - q(x)$ also das selbe Vorzeichen wie $(p_n - q_n)$.
  Da für $p \neq q$ immer $p_n - q_n \neq 0$ gilt, liegt hiermit eine Totalordnung vor und
  \begin{align*}
    p \geq q &\iff (p_n > q_n) \\
    q \geq p &\iff (q_n > p_n)
  \end{align*}
  und damit haben wir eine Totalordnung auf $\Q(x)$.
  Aus $p \leq q$ folgt ebenso für beliebiges $c \in \Q(x)$ klarerweise $p + c \leq q + c$.
  Für $c \geq 0$ folgt auch
  \begin{align*}
    \exists r_0 > 0&: \forall r > r_0: c(r) > 0 \\
    \exists r_1 > 0&: \forall r > r_1: q(r) - p(r) > 0 \\
    \forall r > r_0 + r_1&: (q(r) - p(r))c(r) > 0
  \end{align*}
  und damit $pc \leq qc$. Somit haben wir auf jeden Fall einen angeordneten Körper
  gegeben.
  Um einzusehen, dass $\Q(x)$ mit der gegebenen Relation kein archimedisch angeordneter
  Körper sein kann, betrachte die Funktionen $f(x) = x, g(x) = 1$. Es gilt
  \begin{align*}
    \forall k \in \N: f \geq kg.
  \end{align*}
  \item Sei $K$ nun ein beliebiger, nicht archimedisch angeordneter Körper.
  Aufgrund der Monotoniegesetze folgt $\Char(K) = 0$.
  Der Primkörper $\Q_K$, also der Durchschnitt sämtlicher Unterkörper von $K$
  ist isomorph zum Körper der rationalen Zahlen. Also existiert ein
  \begin{align*}
    \varphi: \Q \to \Q_K
  \end{align*}
  Damit ist also klar, wohin
  alle Polynome vom Grad $n = 0$ abgebildet werden.
  Da $\Q_K$ isomorph zum archimedisch angeordneten Körper $\Q$ ist,
  muss es also Elemente aus $K \backslash \Q_K$ geben.
  Angenommen,
  \begin{align*}
    \forall k \in K: \exists q \in \Q_K: k \geq q
  \end{align*}
  dann folgt
  \begin{align*}
    \forall p \in Q_K: \exists b \in \Z: bp > k \geq q
  \end{align*}
  und somit für $k_1, k_2 \in K$ beliebig
  \begin{align*}
    \exists q_1,q_2 \in Q_K: q_1 \geq k_1, q_2 \geq k_2 \\
    \exists a,b \in \Z: aq_1 > q_2 \geq k_2, bq_2 > g_1 \geq k_1.
  \end{align*}
  im Widerspruch zu $K$ nicht archimedisch angeordnet.
  Sei uns nun ein Element $x \in K$ mit $x > \Q_K$ gegeben. Es folgt
  \begin{align*}
    \Q_K < x < x^2 < \dots.
  \end{align*}
  Analog folgt
  \begin{align*}
    \Q_K > x^{-1} > x^{-2} > \dots.
  \end{align*}
  Also definieren wir unsere Abbildung
  \begin{align*}
    \psi: \begin{cases}
      \Q(x) &\to K \\
      (q_k)_{k \in \Z} \mapsto
      \begin{cases}
        \sum_{k \in \Z}\varphi(q_k)x^k
      \end{cases}
    \end{cases},
  \end{align*}
\end{enumerate}
von der wir nun nachweisen müssen, dass sie die gewünschte isomorphe Einbettung liefert.
\begin{itemize}
  \item Injektivität: Seien $p, \neq q \in \Q(x)$ beliebig. \\
  Dann existiert ein größtes $k \in \Z$, sodass $p_k \neq q_k$. Ss folgt
  \begin{align*}
    \psi(p) - \psi(q) = \sum_{n = -\infty}^kr_kx^k
  \end{align*}
  und aufgrund $\forall n < k: x^k > -r_nx^n$
  \begin{align*}
    \psi(p) - \psi(q) > 0.
  \end{align*}
  Damit ist $\psi$ injektiv.
  \item Homomorphie:
  \begin{align*}
    \psi(0_{\Q(x)}) &= \sum_{k \in \Z}\varphi(0)x^k = 0_K \\
    \psi(1_{\Q(x)}) &= \varphi(1) = 1_K \\
    \psi(q + p) &= \sum_{k \in \Z}\varphi(q_k + p_k)x^k =
    \sum_{k \in \Z}\varphi(q_k)x_k + \sum_{k \in \Z}\varphi(p_k)x_k
    = \psi(q) + \psi(p) \\
    \psi(pq) &= \sum_{k \in \Z}\varphi(\sum_{i \in \Z}p_iq_{k-i})x^k =
    \sum_{k \in \Z}\sum_{i \in \Z}\varphi(p_i)\varphi(q_{k-i})x^k
    = \psi(p)\psi(q) \\
    1_K &= \psi(1_{\Q(x)}) = \psi(pp^{-1}) = \psi(p)\psi(p^{-1})
    \implies \psi(p)^{-1} = \psi(p^{-1}).
  \end{align*}
\end{itemize}
\end{solution}
