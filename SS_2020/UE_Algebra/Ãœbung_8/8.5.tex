\begin{algebraUE}{296}
Sei $(\mathcal{G},i_1,i_2)$ ein Koprodukt von $C_2$ und $C_2$. Wir schreiben
$\mathcal{G} = (G,\ast,1,^{-1})$. Sei $F_2$ die von 2 Elementen frei erzeugte Gruppe.
\begin{enumerate}
  \item Beschreiben Sie die Elemente von $G$, sowie die Gruppenoperation $\ast$
  möglichst explizit (zum Beispiel durch eine ``Normalform'', ähnlich wie wir
  die Element von $F_2$ beschrieben haben); jedenfalls so explizit, dass die
  nächste Teilaufgabe trivial wird.
  \item Zeigen Sie, dass $G$ unendlich viel Elemente hat, indem sie explizit
  unendlich viele (verschiedene) Elemente angeben. Zeigen Sie, dass $\mathcal{G}$
  nicht abelsch ist, indem sie 2 Elemente $x,y \in \mathcal{G}$ angeben mit
  $x \ast y \neq y \ast x$.
  \item Zeigen Sie, dass $\mathcal{G}$ nicht zu $F_2$ isomorph ist.
\end{enumerate}
\end{algebraUE}
\begin{solution}
\begin{enumerate}
  \item Betrachte die Menge aller maximal gekürzten endlichen Wörter, welche aus den beiden
  selbstinversen Elementen $1_0$ und $1_1$ gebildet werden können.
  \begin{align*}
  G = \{g_1\dots g_n: n \in \N, g_j \in\{1_0,1_1\}, g_j \neq g_{j+1}, j = 0,\dots,n-1\}
  \end{align*}
  und die Einbettungen
  \begin{align*}
    \iota_1&: C_2 \to G; \iota_1(0) = \epsilon, \iota_1(1) = 1_0 \\
    \iota_2&: C_2 \to G; \iota_2(0) = \epsilon, \iota_2(1) = 1_1,
  \end{align*}
  wobei $\epsilon$ das leere Wort bezeichnet.
  Zusammen mit der Konkatenation $*$ als Gruppenoperation wird $G$ zu einer Gruppe
  und $\iota_1,\iota_2$ zu Gruppenhomomorphismen.
  Für eine beliebige Gruppe $G^{\prime}$ mit Einbettungsabbildungen $j_1: C_2 \to G,
  j_2: C_2 \to G$ folgt, dass
  \begin{align*}
    f(g_1\dots g_n) := h(g_1)\circ\dots h(g_n),
  \end{align*}
  mit
  \begin{align*}
    h(g) = \begin{cases}
      j_1(1), & g = 1_0 \\
      j_2(1), & g = 1_1
    \end{cases}
  \end{align*}
  der eindeutig bestimmte Gruppenhomomorphismus von $G$ nach $G^{\prime}$
  ist, der nach Definition
  \begin{align*}
    j_1 &= f \circ \iota_1 \\
    j_2 &= f \circ \iota_2
  \end{align*}
  erfüllt, wobei wir zusätzlich $f(\epsilon) = 0_{G^\prime}$ setzen.
  Da jeder weitere Homomorphismus auf dem Erzeugnis $\{1_0,1_1\}$ von $G$
  mit $\varphi$ übereinstimmen muss, folgt mit Proposition 2.3.1.13., dass
  $\varphi$ eindeutig bestimmt ist. \\
  Damit ist $(G,\iota_1,\iota_2)$ ein Koprodukt von $(C_2,C_2)$ in der Varietät
  der Gruppen.
  \item Eine Aufzählung von Elementen aus $G$ lautet
  \begin{align*}
    1_0,1_01_1,1_01_11_0,\dots
  \end{align*}
  G ist nicht kommutativ, da $1_0*1_1 = 1_01_1 \neq 1_11_0 = 1_1*1_0$
  \item $F_2$ bezeichnet die freie Gruppe über einer beliebigen zwei-elementigen
  Variablenmenge $\{a,b\}$ mit den zugehörigen Inversen $\{\overline{a},\overline{b}\}$.
  \begin{align*}
    F_2 = \{f_1^{k_1}f_2^{k_2}\dots f_n^{k_n}: n \in \N,k_j \in \N, f_j \in
    \{a,b,\overline{a},\overline{b}\}, f_j \neq f_{j+1}\}
  \end{align*}
  Um die Nicht-Isomorphie zu zeigen, bemerken wir zuerst, dass kein Element aus $F_2$
  zu sich selbst invers sein kann. Angenommen $gg = \eps$, dann unterscheiden wir
  \begin{itemize}
    \item Fall 1: $g = g_1\dots g_{2n}$.\\
    Es folgt $g_i = \overline{g_{2n+1-i}}$ und damit $g_{n} = \overline{g_{n+1}}$
    im Widerspruch zur Gekürztheit von $g$
    \item Fall 2: $f = g_1\dots g_{2n-1}$. \\
    Es folgt $g_i = \overline{g_{2n-i}}$ und damit $g_{n} = \overline{g_n}$
    im Widerspruch zu $B \cap \overline{B} \emptyset$.
  \end{itemize}
  Angenommen, wir hätten nun einen Isomorphismus $\varphi$ zwischen $F_2$ und $G$
  gegeben, dann wäre $\varphi$ insbesondere surjektiv und es exisitiert ein $g \in G$,
  sodass wir mit
  \begin{align*}
    1_0 = \varphi(a) \neq \varphi(a^{-1}) = \varphi(a)^{-1} = 1_0
  \end{align*}
  einen Widerspruch erhalten.
  \begin{align*}
    \varphi(a^{-1}) = \varphi(a)^{-1} = \varphi(a)
  \end{align*}
\end{enumerate}
Betrachte das erzeugte Monoid $G = \{a,b\}^*$

ein Koprodukt über $C_2, C_2$ mit den Morphismen $\iota_1,\iota_2$.
Sei
\end{solution}
