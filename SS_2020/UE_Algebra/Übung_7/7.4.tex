\begin{algebraUE}{208}
Beweisen Sie folgende Variante des Hauptsatzes 3.4.5.2.: \\
Jede endliche abelsche Gruppe $A$ ist direkte Summe zyklischer Gruppen $(C_{m_i})_{i=1}^n$,
deren Ordnungen $m_i > 1$ eine Teilerkette $m_1 | m_2 | \dots | m_n$ bilden.
Die $m_i$ sind durch $A$ eindeutig bestimmt. \\
\textit{Hinweis:} Aus Lemma 3.4.4.1. folgt leicht, dass direkte Summen zyklischer
Gruppen mit teilerfremden Ordnungen wieder zyklisch sind. Damit lässt sich die
hier zu beweisende Variante ohne große Mühe aus dem Hauptsatz in der Version
von 3.4.5.2. ableiten.
\end{algebraUE}
\begin{solution}
Wir wissen aus der letzen Übung, dass für jedes $n = \prod_{p \in P}p^{e_p}\in \N$
\begin{align*}
  C_n \cong \bigoplus_{p \in \P}C_{p^{e_p}}.
\end{align*}
Insbesondere ist die direkte Summe teilerfremder zyklischer Gruppen damit wieder zyklisch.
Aus dem Hauptsatz in 3.4.5.2. wissen wir bereits, dass sich eine beliebige endliche
abelsche Gruppe als direkte Summe
\begin{align*}
  A \cong \bigoplus_{p \in \P}\bigoplus_{n=1}^{\infty}C_{p^n}^{e_{p,n}}
  =  \bigoplus_{n=1}^{\infty}\bigoplus_{p \in \P}C_{p^n}^{e_{p,n}}
\end{align*}
von zyklischen Gruppen von Primzahlordnung darstellen lässt.
Definiere nun für $p \in \P$ die Folge
\begin{align*}
  M_p &:= (\underbrace{p^n,\dots,p^n}_{e_{p,n} \text{ viele }},\dots,
  \underbrace{p^1,\dots,p^1}_{e_{p,1} \text{ viele}},1,\dots)
\end{align*}
Jetzt definiere
\begin{align*}
  m_i := \prod_{p \in \P}p^{M_{p,i}}
\end{align*}
Da nur endlich viele $e_{p,n} \neq 0$, existiert ein $N > 0: \forall n > N: m_n = 1$.
Nach Definition folgt $m_N | \dots | m_1$ und es gilt
\begin{align*}
  \bigoplus_{i = 1}^{N} C_{m_i} &\cong \bigoplus_{i = 1}^{N}\bigoplus_{p \in \P}C_{M_{p,i}}
  \cong \bigoplus_{n=1}^{\infty}\bigoplus_{p \in \P}C_{p^n}^{e_{p,n}} \cong A.
\end{align*}
Für die Eindeutigkeit dieser Darstellung betrachte eine weitere Folge $(k_j)_{j=1}^M$
mit $k_M | \dots | k_1$ und $A \cong \bigoplus_{j = 1}^{N} C_{k_j}$.
Dann gilt für
\begin{align*}
  m_i &= \prod_{p \in \P}p^{M_{p,i}}, \qquad e^m_{p,n} := |\{i: M_{p,i} = n\}| \\
  k_j &= \prod_{p \in \P} p^{K_{p,j}}, \qquad e^k_{p,n} := |\{j: K_{p,j} = n\}|
\end{align*}
\begin{align*}
\bigoplus_{i = 1}^{N}\bigoplus_{p \in \P}C_{p^n}^{e^m_{p,n}} &\cong
\bigoplus_{i = 1}^{N}\bigoplus_{p \in \P}C_{M_{p,i}} \cong
\bigoplus_{i = 1}^{N} C_{m_i} \cong A \cong \bigoplus_{j = 1}^{N} C_{k_j}
\cong \bigoplus_{i = 1}^{N}\bigoplus_{p \in \P}C_{K_{p,j}}
\cong \bigoplus_{i = 1}^{N}\bigoplus_{p \in \P}C_{p^n}^{e^k_{p,n}}
\end{align*}
Nach der Eindeutigkeitsaussage des Hauptsatzes über abelsche Gruppen folgt daraus
\begin{align*}
  \forall p \in P, n \in N: e^m_{p,n} = e^k_{p,n}
\end{align*}
und aufgrund $p^n | m_k \implies (p^n | m_n, n \leq k)$ folgt
\begin{align*}
  m_1 &= \prod_{p \in \P}p^{\max\{n \in N: e^m_{p,n} \neq 0\}} = k_1.
\end{align*}
Weitergeführt folgt damit auch
\begin{align*}
  m_i = k_i, \qquad i = 1,\dots,N
\end{align*}
und damit die Eindeutigkeit der Darstellung.
\end{solution}
