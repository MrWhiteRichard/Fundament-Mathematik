\begin{algebraUE}{208}
Beweisen Sie folgende Variante des Hauptsatzes 3.4.5.2.: \\
Jede endliche abelsche Gruppe $A$ ist direkte Summe zyklischer Gruppen $(C_{m_i})_{i=1}^n$,
deren Ordnungen $m_i > 1$ eine Teilerkette $m_1 | m_2 | \dots | m_n$ bilden.
Die $m_i$ sind durch $A$ eindeutig bestimmt. \\
\textit{Hinweis:} Aus Lemma 3.4.4.1. folgt leicht, dass direkte Summen zyklischer
Gruppen mit teilerfremden Ordnungen wieder zyklisch sind. Damit lässt sich die
hier zu beweisende Variante ohne große Mühe aus dem Hauptsatz in der Version
von 3.4.5.2. ableiten.
\end{algebraUE}
\begin{solution}
Seien $C_p, C_q$ zwei zyklische Gruppen mit teilerfremden Ordnungen, also
$\ggT(p,q) = 1$. Dann ist $C_p \bigoplus C_q$ eine abelsche Gruppe und mit
Lemma 3.4.4.1. folgt
\begin{align*}
  \ord((1,1)) = \ord((1,0) + (0,1)) = \ord(1,0) \cdot \ord(0,1) = pq.
\end{align*}
Da $|C_p \bigoplus C_q| = pq$ gilt $\langle (1,1) \rangle = C_p \bigoplus C_q$
und $C_p \bigoplus C_q$ ist somit zyklisch.
Aus dem Hauptsatz in 3.4.5.2. wissen wir bereits, dass sich eine beliebige endliche
abelsche Gruppe als direkte Summe
\begin{align*}
  A \cong \bigoplus_{p \in \P}\bigoplus_{n=1}^{\infty}C_{p^n}^{e_{p,n}}
  =  \bigoplus_{n=1}^{\infty}\bigoplus_{p \in \P}C_{p^n}^{e_{p,n}}
\end{align*}
von zyklischen Gruppen von Primzahlordnung darstellen lässt.
Da nur endlich viele $e_{p,n} \neq 0$ sind, lässt sich zu jedem $p \in \P$
das kleinste $p_1 \in \N$ finden, sodass $e_{p_1,k} \neq 0$ und wir können
\begin{align*}
  \bigoplus_{p \in \P}C_{p^{p_1}}^{e_{p,p_1}} \cong C_{m_1}
\end{align*}
schreiben. Wenn wir dieses Schema analog fortführen haben wir in endlich vielen
Schritten alle $e_{p,n} \neq 0$ abgearbeitet und wir können
\begin{align*}
  A \cong \bigoplus_{k=1}^{n} C_{m_k}.
\end{align*}
schreiben, wobei für alle $0 < i < j \leq n $ gilt: $m_i | m_j$.

\end{solution}
