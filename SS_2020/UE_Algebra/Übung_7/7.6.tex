\begin{algebraUE}{215}
Jede archimedisch angeordnete abelsche Gruppe $G$ lässt sich als solche isomorph
in die geordnete additive Gruppe $\R$ der reellen Zahlen einbetten. (Umgekehrt
ist jede additive Untergruppe von $\R$ archimedisch angeordnet.)
Ist $\iota: G \mapsto \R$ eine solche isomorphe Einbettung, so sind alle anderen
gegeben durch sämtliche Abbildungen $\lambda_l$ mit reellem $\lambda > 0$. \\
\textit{Anleitung für die Existenz von $\iota$}: Gehen Sie für nichttriviales
$G$ von einem positiven Element $g \in G$ aus, das Sie auf die reelle Zahl
$1 = \iota(g)$ abbilden. Wegen der archimedischen Eigenschaft definiert das
einen eindeutigen ordnungsverträglichen Homomorphismus $\iota$, der
(wieder wegen der archimedischen Eigenschaft) sogar injektiv sein muss.
\end{algebraUE}
\begin{solution}
Exzellenter Einfall.
\end{solution}
