\begin{algebraUE}{187}
Sei $R$ ein kommutativer Ring mit $1$, außerdem $p,q \in R[[x]]$ mit
$p(x) = \sum_{n=0}^{\infty}a_nx^n$ und $q(x) = \sum_{n=0}^{\infty}b_nx^n$.
Mit $R^*,R[[x]]^*$ und $R[x]^*$ seien die Einheitengruppen der Ringe
$R, R[[x]]$, beziehungsweise $R[x]$ bezeichnet. Dann gilt:
\begin{enumerate}
  \item $\ord(p + q) \geq \min\{\ord(p),\ord(q)\}$
  \item $\grad(p + q) \leq \max\{\grad(p),\grad(q)\}$
  \item $\ord(pq) \geq \ord(p) + \ord(q)$. Ist $R$ ein Integritätsbereich, so gilt
  sogar Gleichheit.
  \item $\grad(pq) \leq \grad(p) + \grad(q)$. Ist $R$ ein Integritätsbereich, so gilt
  sogar Gleichheit.
  \item Ist $R$ ein Integritätsbereich, so auch $R[[x]]$ und $R[x]$.
  \item Genau dann ist $p \in R[[x]]^*$, wenn $a_0 \in R^*$. Ist $q = p^{-1}$
  das multiplikative Inverse von $p$, dann erfüllen die Koeffizienten $b_0 = a_0^{-1}$
  und für alle $n = 1,2,\dots$ die Rekursion
  \begin{align*}
    b_n = -b_0(a_1b_{n-1} + a_2b_{n_2} + \dots a_nb_0).
  \end{align*}
  Ist speziell $R$ ein Körper, so ist $p \in R[[x]]^*$ genau dann, wenn $\ord(p) = 0$.
  \item Wenn $R$ Integritätsbereich ist, dann gilt für alle $p \in R[x]:$ Genau dann
  ist $p \in R[x]^*$, wenn $\grad(p) = 0$ und $p = a_0$ mit $a_0 \in R^*$.
\end{enumerate}
\end{algebraUE}
\begin{solution}
Beeindruckender Beweis.
\begin{enumerate}
  \item
\end{enumerate}
\end{solution}
