\begin{algebraUE}{189}
Ist $R$ ein Körper, so bilden die formalen Laurentreihen (zusammen mit der
Inklusionsabbildung $\iota: R[[x]] \rightarrow R[[[x]]], (a_n)_{n \in \N}
\mapsto (a_n)_{n \in \Z}$ mit $a_n = 0$ für alle $n < 0$, als isomorpher
Einbettung) einen Quotientenkörper von $R[[x]]$. \\
Im Beweis dieses Satzes sind genauer folgende Schritte zu tun:
\begin{enumerate}
  \item Definieren Sie sorgfältig die fundamentalen Operationen von $R[[[x]]]$
  (Addition, Nullelement, additive Inverse, Multiplikation, Einselement).
  \item Zeigen Sie, dass es sich dabei um einen kommutativen Ring mit $1$ handelt.
  \item Zeigen Sie, dass sich jedes $q \in R[[[x]]]\backslash \{0\}$ eindeutig in
  der Form $x_n\overline{q}(x)$ schreiben lässt, mit $n \in \Z$ und $\overline{q}(x) \in R[[x]]^*$.
  (Mit $R[[x]]^*$ bezeichnen wir die Einheiten von $R[[x]]$.)
  \item Zeigen Sie, dass jedes $q \in R[[[x]]]\backslash\{0\}$ ein multiplikatives
  Inverses in $R[[[x]]]$ hat. (Somit ist $R[[[x]]]$ ein Körper.)
  \item Zeigen Sie, dass $R[[[x]]]$ mit $\iota$ tatsächlich ein Quotientenkörper
  von $R[[x]]$ ist.
\end{enumerate}
\end{algebraUE}
\begin{solution}
\leavevmode \\
\begin{enumerate}
  \item Wir definieren die Operationen auf $R[[[x]]]$ folgendermaßen:
  \begin{align*}
    (a_k)_{k \in \Z} + (b_k)_{k \in \Z} &:= (a_k + b_k)_{k \in \Z} \\
    0_{R[[[x]]]} &:= (0_R)_{k \in \Z} \\
    -(a_k)_{k \in \Z} &:= (-a_k)_{k \in \Z} \\
    (a_k)_{k \in \Z} \cdot (b_k)_{k \in \Z} &:= \left(\sum_{i \in \Z}a_ib_{k-i}\right)_{k \in \Z} \\
    1_{R[[[x]]]} &:= (\delta_{k0})_{k \in \Z}
  \end{align*}
  \item Die Addition ist wohldefiniert auf $R[[[x]]]$, da für $N := \ord(p), M := \ord(q)$
  und $p+q(x) = \sum_{k \in \Z}c_kx^k$
  \begin{align*}
    \forall k < \min{N,M}: c_k = a_k + b_k = 0.
  \end{align*}
  $(R[[[x]]],+,0,-)$ ist klarerweise eine abelsche Gruppe, da sich die
  entsprechenden Eigenschaften direkt von $R$ übertragen lassen.
  Die Multiplikation ist wohldefiniert auf $R[[[x]]]$, da für $N := \ord(p), M := \ord(q)$
  und $pq(x) = \sum_{k \in \Z}c_kx^k$
  \begin{align*}
    \forall k < N + M: c_k = \sum_{i \in \Z}a_ib_{k-i}
    = \sum_{i < N}\underbrace{a_i}_{=0}b_{k-i} + \sum_{i \geq N}a_i\underbrace{b_{k-i}}_{=0} = 0.
  \end{align*}
  Ebenso ist sie klarerweise assoziativ, kommutativ, distributiv bezüglich $+$
  und hat das Einselement $1_{R[[[x]]]}$.
  Also stellt $(R[[[x]]],+,0,-1,\cdot,1)$ tatsächlich einen kommutativer Ring mit Eins dar.
  \item Sei $q \in R[[[x]]]\backslash \{0\}$ beliebig. Sei $N := \ord(q)$, dann gilt
  aufgrund der Tatsache, dass $R$ ein Körper ist.
  \begin{align*}
    q(x) = \sum_{k = N}^{\infty}a_kx^k = \sum_{k = N}^{\infty}a_kx^k
    = x^{N}\sum_{k = 0}^{\infty}a_{k+N}x^k
  \end{align*}
  mit $\overline{q}(x) := \sum_{k = 0}^{\infty}a_{k+N}x^k \in R[[x]]^*$, da $0 \neq a_{k+N} \in R^*$.
  Sei nun $N^{\prime} \in \Z, \overline{q^{\prime}} \in R[[x]]^*: q(x) = x^{N^{\prime}}\overline{q}^{\prime}(x)$,
  mit Koeffizienten $(a_k^{\prime})_{k \in \N}$
  Dann folgt
  \begin{align*}
    \overline{q}^{\prime}(x) = x^{N-N^{\prime}}\sum_{k = 0}^{\infty}a_{k+N}x^k
  \end{align*}
  \begin{itemize}
    \item Fall 1: $N-N^{\prime} < 0$: Dann ist $\overline{q}^{\prime} \notin R[[x]]$.
    \item Fall 2: $N-N^{\prime} > 0$: Dann ist aufgrund
    $\ord(x^{N-N^{\prime}}\overline{q}(x)) \leq N - N^{\prime} > 0: a_0^{\prime} = 0$
    und $\overline{q}^{\prime} \notin R[[x]]^*$.
    \item Fall 3: $N-N^{\prime} = 0$: Dann ist $\overline{q}^{\prime}(x) = \overline{q}(x)$.
  \end{itemize}
  Damit ist diese Darstellung für beliebiges $q \in R[[[x]]]\backslash \{0\}$ eindeutig.
  \item Sei $q \in R[[[x]]]\backslash \{0\}$ beliebig. Sei $N := \ord(q)$, dann gilt
  \begin{align*}
    p(x) = x^{-N}\overline{q}(x).
  \end{align*}
  $\overline{q}$ hat in $R[[x]]$ ein Inverses: $\overline{q}^{-1}$. Damit erhalten wir
  \begin{align*}
    p^{-1}(x) = x^N\overline{q}^{-1}(x)
  \end{align*}
  und somit ist $R[[[x]]]$ ein Körper.
  \item Zuerst bemerken wir, dass $\iota$ eine isomorphe Einbettung von $R[[x]]$
  als Ring mit Eins in $R[[[x]]]$ ist. Mit Punkt 3 und Punkt 4 folgt, dass
  \begin{align*}
    R[[[x]]] = \left\{\frac{p}{q}: p, q \in R[[x]], q \neq 0 \right\}
  \end{align*}
  und aus Proposition 3.3.5.10 folgt, dass $R[[[x]]]$ somit ein Quotientenkörper
  von $R[[x]]$ ist.
\end{enumerate}
\end{solution}
