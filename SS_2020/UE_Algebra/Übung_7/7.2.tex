\begin{algebraUE}{189}
Ist $R$ ein Körper, so bilden die formalen Laurentreihen (zusammen mit der
Inklusionsabbildung $\iota: R[[x]] \rightarrow R[[[x]]], (a_n)_{n \in \N}
\mapsto (a_n)_{n \in \Z}$ mit $a_n = 0$ für alle $n < 0$, als isomorpher
Einbettung) einen Quotientenkörper von $R[[x]]$. \\
Im Beweis dieses Satzes sind genauer folgende Schritte zu tun:
\begin{enumerate}
  \item Definieren Sie sorgfältig die fundamentalen Operationen von $R[[[x]]]$
  (Addition, Nullelement, additive Inverse, Multiplikation, Einselement).
  \item Zeigen Sie, dass es sich dabei um einen kommutativen Ring mit $1$ handelt.
  \item Zeigen Sie, dass sich jedes $q \in R[[[x]]]\backslash \{0\}$ eindeutig in
  der Form $x_n\overline{q}(x)$ schreiben lässt, mit $n \in \Z$ und $\overline{q}(x) \in R[[x]]^*$.
  (Mit $R[[x]]^*$ bezeichnen wir die Einheiten von $R[[x]]$.)
  \item Zeigen Sie, dass jedes $q \in R[[[x]]]\backslash\{0\}$ ein multiplikatives
  Inverses in $R[[[x]]]$ hat. (Somit ist $R[[[x]]]$ ein Körper.)
  \item Zeigen Sie, dass $R[[[x]]]$ mit $\iota$ tatsächlich ein Quotientenkörper
  von $R[[x]]$ ist.
\end{enumerate}
\end{algebraUE}
\begin{solution}
\leavevmode \\
\begin{enumerate}
  \item 
\end{enumerate}
\end{solution}
