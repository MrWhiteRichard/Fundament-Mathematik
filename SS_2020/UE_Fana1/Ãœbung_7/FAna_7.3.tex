\begin{exercise}[34/3]

Sei $E$ ein Spektralmaß auf den Borelmengen von $\R$ welches kompakten Träger hat, und sei $[a, b]$ ein kompaktes Intervall dessen Komplement eine $E$-Nullmenge ist.
Definiere $E_\lambda: \R \to \mathcal{B}(H)$ als $E_\lambda := E((-\infty, \lambda])$. \\

Sei $\phi \in \mathrm{BM}(\R, \C)$ mit $\phi |_{[a, b]}$ stetig.
Zeige, dass der Limes (das Riemann-Stieltjes Integral)

\begin{align*}
  \lim_{|\mathcal{R}| \to 0}
  \sum_{j=1}^{n(\mathcal{R})}
  \phi(\alpha_j)
  (E_{\xi_j} - E_{\xi_{j-1}}),
\end{align*}

wobei die Riemannzerlegung $\mathcal{R}$ des Intervalles $[a, b]$ die Stützstellen $\xi_j$ und die Zwischenstellen $\alpha_j$ hat, bezüglich der Operatornorm existiert und gleich der
(wie in der Vorlesung definierte) Operator \\
$\Int{\phi|_{[a,b]}}{E} - \phi(a) E(\{a\})$ ist.

\end{exercise}

\begin{solution}
Wir werden wohl oder übel zeigen müssen, dass obiges Netz ein Cauchy-Netz in der
Operatornorm ist. Seien $\mathcal{R}_1,\mathcal{R}_2$ also zwei Riemannzerlegungen.
Wir machen zuerst eine Vorberechnung
\begin{align*}
  E_{\xi_j} - E_{\xi_{j-1}} = E((-\infty, \xi_j]) - E((-\infty, \xi_{j-1}])
  = E((\xi_{j-1},\xi_j]).
\end{align*}
Sei nun $\epsilon > 0$ beliebig und wähle $\delta > 0$, sodass für $\forall x,y \in [a,b]: |x - y| < \delta \implies |\phi(x) - \phi(y)| < \frac{\epsilon}{2}$. Dies ist möglich da $\phi$
auf dem Kompaktum $[a,b]$ stetig und somit sogar gleichmäßig stetig ist. Seien
$\mathcal{R}_1,\mathcal{R}_2$ zwei beliebige Riemannzerlegungen mit
$|\mathcal{R}_1|,|\mathcal{R}_2| < \delta$. \\
Damit gilt nun für alle $x \in \mathcal{H}$ mit der Notation
$P(\mathcal{R}) := \sum_{j=1}^{n(\mathcal{R})}\phi(\alpha_j)(E_{\xi_j} - E_{\xi_{j-1}})$
\begin{align*}
  \|(P(\mathcal{R}_1) - P(\mathcal{R}_2))x\| &= \left\|\sum_{j=1}^{n(\mathcal{R}_1)}
  \phi(\alpha_{1,j})
  (E_{\xi_{1,j}} - E_{\xi_{1,j-1}})x -
  \sum_{j=1}^{n(\mathcal{R}_2)}
  \phi(\alpha_{2,j})
  (E_{\xi_{2,j}} - E_{\xi_{2,j-1}})x\right\| \\
  &= \left\|\sum_{j=1}^{n(\mathcal{R}_1)}
  \phi(\alpha_{1,j})
  E((\xi_{1,j-1},\xi_{1,j}])x -
  \sum_{j=1}^{n(\mathcal{R}_2)}
  \phi(\alpha_{2,j})
  E((\xi_{2,j-1},\xi_{2,j}])x\right\|.
\end{align*}
Sei nun $\mathcal{R} = ((\eta_{k})_{k=0}^{n(\mathcal{R})},(\beta_k)_{k=1}^{n(\mathcal{R})})$
eine Riemannzerlegung mit
\begin{align*}
  \{\xi_{1,j}: j = 0,\dots,n(\mathcal{R}_1)\} \cup \{\xi_{2,j}: j = 0,\dots,n(\mathcal{R}_2)\}
  \subseteq \{\eta_k: k = 0,\dots,n(\mathcal{R})\}.
\end{align*}
Ist $j \in \{1,\dots,n(\mathcal{R}_1)\}$, so gibt es Indizes $k(j-1) < k(j)$, sodass
\begin{align*}
  \xi_{1,j-1} = \eta_{k(j-1)} < \dots < \eta_{k(j)} = \xi_{1,j}.
\end{align*}
Dann gilt
\begin{align*}
  \|(P(\mathcal{R}_1) - P(\mathcal{R}))x\| &= \left\|
  \sum_{j=1}^{n(\mathcal{R}_1)}\sum_{k =k(j-1)+1}^{k(j)}\phi(\alpha_{1,j})
  E((\eta_{k-1},\eta_{k}])x -
  \sum_{j=1}^{n(\mathcal{R})}
  \phi(\beta_j)
  E((\eta_{j-1},\eta_{j}])x\right\| \\
  &= \left\|
  \sum_{j=1}^{n(\mathcal{R}_1)}\sum_{k =k(j-1)+1}^{k(j)}
  E((\eta_{k-1},\eta_{k}])
  \underbrace{(\phi(\alpha_{1,j}) - \phi(\beta_k))}_{<\nicefrac{\epsilon}{2}}x\right\| \\
  &\leq \frac{\epsilon}{2}\left\|
  \sum_{j=1}^{n(\mathcal{R})}
  E((\eta_{j-1},\eta_{j}])x\right\|
  = \frac{\epsilon}{2}\left\|E((a,b])x\right\| \leq \frac{\epsilon}{2}\|E((a,b])\|\|x\|
  \leq \frac{\epsilon}{2}\|x\|,
\end{align*}
wobei die letzte Ungleichung gilt, da Orthogonalprojektionen Kontraktionen sind, tatsächlich gilt wegen $E([a, b]) = E(\R) = I$ sogar Gleichheit. \\
Analog folgt $\|(P(\mathcal{R}_2) - P(\mathcal{R}))x\| \leq \frac{\epsilon}{2}\|x\|$
und damit erhalten wir für alle $x \in \mathcal{H}$
\begin{align*}
  \|(P(\mathcal{R}_1) - P(\mathcal{R}_2))x\| \leq \epsilon\|x\|.
\end{align*}
Also haben wir gezeigt, dass der Limes in der Operatornorm existiert. \\
Laut Definition gilt für alle $x,y \in \mathcal{H}$
\begin{align*}
  \left(\left(\Int{\phi|_{[a,b]}}{E} - \phi(a) E(\{a\})\right)x,y\right) =
  \Int{\phi|_{[a,b]}}{E_{x,y}} - \phi(a)(E(\{a\})x,y).
\end{align*}
Weiters rechnen wir
\begin{align*}
  \left(\lim_{|\mathcal{R}| \to 0}\sum_{j=1}^{n(\mathcal{R})}
  \phi(\alpha_j)(E_{\xi_j} - E_{\xi_{j-1}})x,y\right) &=
  \lim_{|\mathcal{R}| \to 0}
  \left(\sum_{j=1}^{n(\mathcal{R})}\phi(\alpha_j)(E_{\xi_j} - E_{\xi_{j-1}})x,y\right) \\
  &= \lim_{|\mathcal{R}| \to 0}
  \left(\sum_{j=1}^{n(\mathcal{R})}\phi(\alpha_j)E((\xi_{j-1},\xi_j])x,y\right).
\end{align*}
Für konstantes $\phi|_{[a,b]} \equiv \phi_0$ gilt offensichtlich
\begin{align*}
  \left(\lim_{|\mathcal{R}| \to 0}\sum_{j=1}^{n(\mathcal{R})}
  \phi(\alpha_j)(E_{\xi_j} - E_{\xi_{j-1}})x,y\right) =
  (\phi_0E((a,b])x,y) = \Int{\phi|_{[a,b]}}{E_{x,y}} - \phi(a)(E(\{a\})x,y).
\end{align*}
Wie im Beispiel zuvor können wir die Funktion $\phi$ durch eine monotone Folge
von Treppenfunktionen von unten approximieren, also
\begin{align*}
  \phi = \lim_{m \to \infty}\underbrace{\sum_{k=1}^{k(m)}\alpha_{k,m}
  \1_{[\beta_{k-1,m},\beta_{k,m}]}}_{=: \phi_m}.
\end{align*}
Für diese Treppenfunktionen können wir ein Teilnetz an Riemannzerlegungen
finden, deren Stützstellen die $(\beta_{k,m})_{k=0}^{k(m)}$ umfassen und
es gilt
\begin{align*}
  \left(\lim_{|\mathcal{R}| \to 0}\sum_{j=1}^{n(\mathcal{R})}
  \phi_m(\alpha_j)(E_{\xi_j} - E_{\xi_{j-1}})x,y\right) &=
  \lim_{|\mathcal{R}| \to 0}\left(\sum_{k=1}^{k(m)}
  \alpha_{k,m}\sum_{j=1}^{j(k)}(E_{\xi_j} - E_{\xi_{j-1}})x,y\right) \\
  &= \left(\sum_{k=1}^{k(m)}
  \alpha_{k,m}E((\beta_{k-1,m},\beta_{k,m}])x,y\right) =
  \Int{\phi_m|_{[a,b]}}{E_{x,y}} - \phi_m(a)(E(\{a\})x,y)
\end{align*}
Damit müssen wir jetzt nur noch eine Limes-Vertauschung begründen und wir sind durch:
\begin{align*}
  \left(\lim_{|\mathcal{R}| \to 0}\sum_{j=1}^{n(\mathcal{R})}
  \phi(\alpha_j)(E_{\xi_j} - E_{\xi_{j-1}})x,y\right) =
  \lim_{|\mathcal{R}| \to 0}\lim_{m \to \infty}
  \underbrace{\left(\sum_{j=1}^{n(\mathcal{R})}
  \phi_m(\alpha_j)(E_{\xi_j} - E_{\xi_{j-1}})x,y\right)}_{=: f_{m,\mathcal{R}}}
\end{align*}
Der Grenzwert $\lim_{|\mathcal{R}| \to 0}f_{m,\mathcal{R}}$ für jedes feste $m$
nach der Rechnung am Anfang erreicht
und der Grenzwert $f_{\mathcal{R}} := \lim_{m \to \infty}f_{m,\mathcal{R}} =
\left(\sum_{j=1}^{n(\mathcal{R})}
\phi(\alpha_j)(E_{\xi_j} - E_{\xi_{j-1}})x,y\right)$ existiert gleichmäßig in $\mathcal{R}$, da
\begin{align*}
  \|f_{\mathcal{R}} -f_{m,\mathcal{R}}\| &=
  \left\|\left(\sum_{j=1}^{n(\mathcal{R})}
  (\phi(\alpha_j) - \phi_m(\alpha_j))(E_{\xi_j} - E_{\xi_{j-1}})x,y\right)\right\|
  \leq \max_{j=1}^{n(\mathcal{R})}(\phi(\alpha_j) - \phi_m(\alpha_j))
  \left\|\left(\sum_{j=1}^{n(\mathcal{R})}
  (E_{\xi_j} - E_{\xi_{j-1}})x,y\right)\right\| \\
  &= \max_{j=1}^{n(\mathcal{R})}(\phi(\alpha_j) - \phi_m(\alpha_j))
  \left\|\left(\sum_{j=1}^{n(\mathcal{R})}
  E((a,b])x,y\right)\right\| \xrightarrow{m \to \infty} 0.
\end{align*}
Also dürfen wir vertauschen und erhalten für alle $x,y \in \mathcal{H}$
\begin{align*}
  \left(\lim_{|\mathcal{R}| \to 0}\sum_{j=1}^{n(\mathcal{R})}
  \phi(\alpha_j)(E_{\xi_j} - E_{\xi_{j-1}})x,y\right) &=
  \lim_{|\mathcal{R}| \to 0}\lim_{m \to \infty}
  \left(\sum_{j=1}^{n(\mathcal{R})}
  \phi_m(\alpha_j)(E_{\xi_j} - E_{\xi_{j-1}})x,y\right) \\
  &= \lim_{m \to \infty}\lim_{|\mathcal{R}| \to 0}
  \left(\sum_{j=1}^{n(\mathcal{R})}
  \phi_m(\alpha_j)(E_{\xi_j} - E_{\xi_{j-1}})x,y\right) \\
  &= \lim_{m \to \infty} \Int{\phi_m|_{[a,b]}}{E_{x,y}} - \phi_m(a)(E(\{a\})x,y) \\
  &= \Int{\phi|_{[a,b]}}{E_{x,y}} - \phi(a)(E(\{a\})x,y) \\
  &= \left(\left(\Int{\phi|_{[a,b]}}{E} - \phi(a) E(\{a\})\right)x,y\right).
\end{align*}
Die vorletzte Gleichung gilt aufgrund monotoner Konvergenz. \\
*** Alternative zur Gleichheit: *** \\
Für jedes $n \in \N$ seien 
\begin{align*}
  \forall k \in \{0, \dots, n\}: \xi_k := a + \frac{k (b - a)}{n}.
\end{align*}
die Stützstellen einer Riemannzerlegung und wegen der Stetigkeit von $\phi$ auf $[a, b]$ gilt
\begin{align*}
  \forall k \in \{1, \dots, n\} \exists \alpha_k \in [\xi_{k-1}, \xi_k]: |\phi(\alpha_k)| = \min\{|\phi(x)| : x \in [\xi_{k - 1}, \xi_k]\},
\end{align*}
ein solches $\alpha_k$ nehmen wir jeweils als Zwischenstelle. Wegen der Kompaktheit von $[a, b]$ ist das auf $[a, b]$ stetige $\phi$ sogar gleichmäßig stetig und daher konvergiert die Folge von Treppenfunktionen 
\begin{align*}
  s_n := \phi(a) \1_{\{a\}} + \sum_{k = 1}^n \phi(\alpha_k) \1_{]\xi_{k-1}, \xi_k]}
\end{align*}
punktweise $E$-fast überall gegen $\phi$. Dafür benütigen wir auch, dass $E(\R \setminus [a, b]) = 0$. Wir wollen den bereits im vorigen Schritt gefundenen Operator
\begin{align*}
  A := \lim_{n \to \infty} \sum_{j = 1}^n \phi(\alpha_j) \pbraces{E_{\xi_j} - E_{\xi_{j - 1}}}
\end{align*} 
taufen. Für beliebige $g, h \in \mathcal{H}$ berechnen unter Benützung des Satzes über Konvergenz durch Majorisierung
\begin{align*}
  (Ag, h) &= \pbraces{\lim_{n \to \infty} \sum_{j = 1}^n \phi(\alpha_j) \pbraces{E_{\xi_j}g - E_{\xi_{j - 1}}g}, h} = \lim_{n \to \infty} \sum_{j = 1}^n \phi(\alpha_j) \pbraces{\pbraces{E_{\xi_j}g, h} - \pbraces{E_{\xi_{j - 1}}g, h}} \\
  & = \lim_{n \to \infty} \sum_{j = 1}^n \phi(\alpha_j) E_{g, h}(]\xi_{j - 1}, \xi_j]) = \lim_{n \to \infty} \Int{s_n}{E_{g,h}} - \phi(a) E_{g, h}(\{a\}) = \\
  &= \Int{\phi}{E_{g, h}} - \phi(a) E_{g, h}(\{a\}) = \pbraces{\pbraces{\Int{\phi}{E} - \phi(a)E(\{a\})}g, h}
\end{align*}
Da diese Gleichheit für alle $g, h \in \mathcal{H}$ gilt folgt $A = \Int{\phi}{E} - \phi(a)E(\{a\})$.
\end{solution}
