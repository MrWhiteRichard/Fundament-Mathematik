\begin{exercise}[37/3]

Sei $A$ ein beschränkter selbstadjungierter Operator.
Zeige mit Hilfe des Spektralsatzes, dass für $\lambda > \norm{A}$ gilt

\begin{align*}
  (A - \lambda)^{-1}
  =
  -\Int[0][\infty]{e^{-\lambda t} e^{t A}}{t}.
\end{align*}

\end{exercise}

\begin{solution}

\phantom{}

\includegraphicsboxed{Satz 7.2.1 (Spektralsatz für beschränkte selbstadjungierte Operatoren)}

Wir bemerken vorerst, dass $\lambda \in \rho(A)$.
Damit ist $A - \lambda$ invertierbar und die linke Seite wohldefiniert.

\begin{align*}
  \text{lhs}~
  =
  -\Int{\frac{1}{\lambda - t}}{E(t)}
  =
  -\frac{1}{\lambda}
  \Int{\frac{1}{1 - t / \lambda}}{E(t)}
  =
  -\frac{1}{\lambda}
  \Int{\sum_{n=0}^\infty (t / \lambda)^n}{E(t)}
  =
  -\frac{1}{\lambda}
  \sum_{n=0}^\infty
  \frac{A^n}{\lambda^n}
\end{align*}

\includegraphicsboxed{Korollar 7.2.12}

Laut dem Beweis von Korollar 7.2.12, gilt die erste Gleichheit.
Die geometrische Reihe konviergiert, weil $\Forall t \in \sigma(A):$

\begin{align*}
  \lambda > \norm{A} \geq r(a) \geq |t|
  \implies
  1 > |t / \lambda|.
\end{align*}

\includegraphicsboxed{Beispiel 7.2.8}

Laut Beispiel 7.2.8, gilt die letzte Gleichheit, weil der Konvergenzradius der geometrischen Reihe als Potenzreihe genau $|\lambda|$ ist.
Hier wird der Spektralsatz verwendet.

\begin{align*}
  \text{rhs}~
  =
  -\sum_{n=0}^\infty
  \frac{A^n}{n!}
  \Int[0][\infty]{e^{-\lambda t} t^n}{t}
  =
  -\sum_{n=0}^\infty
  \frac{\Gamma(n+1)}{n!}
  \frac{A^n}{\lambda^{n+1}}
\end{align*}

Das Integral darf man mit der Reihe vertauschen, weil $e^{t A}$ absolut, und damit gleichmäßig, konvergiert.

\begin{align*}
  \sum_{n=0}^\infty
  \norm{\frac{t^n A^n}{n!}}
  \leq
  \sum_{n=0}^\infty
  \frac{(|t| \norm{A})^n}{n!}
  =
  e^{|t| \norm{A}} < \infty
\end{align*}

Die letztere Gleichheit gelingt durch die Substitution $s = \lambda t$.

\begin{align*}
  \Int[0][\infty]{e^{-\lambda t} t^n}{t}
  =
  \frac{1}{\lambda}
  \Int
  [0][\infty]
  {
    e^{-s}
    \frac{s^n}{\lambda^n}
  }{s}
  =
  \frac{1}{\lambda^{n + 1}}
  \Gamma(n+1)
\end{align*}

Nun ist lhs $=$ rhs, weil ja $\Gamma(n+1) = n!$.

\end{solution}
