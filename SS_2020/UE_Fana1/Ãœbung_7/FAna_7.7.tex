\begin{exercise}[38/1*]

Betrachte den Hilbertraum $L^2(0, 1)$, seinen Teilraum

\begin{align*}
  E
  :=
  \Bbraces
  {
    h \in C^1(0, 1):
    h^\prime ~\text{absolut stetig},
    h^\primeprime \in L^2(0, 1),
    h(0) = h(1),
    h^\prime(0) = h^\prime(1)
  },
\end{align*}

und die lineare Abbildung $K: E \to L^2(0, 1)$ die definiert ist als $K h := -h^\primeprime + h$.
Zeige, dass $K$ bijektiv ist, und dass $K^{-1}$ kompakt und selbstadjungiert ist.
Bestimme $\sigma(K^{-1})$ und die Eigenräume $\ker{(K^{-1} - \lambda)}$ für $\lambda \in \sigma_p(K^{-1})$.
(Man beachte den Unterschied im Verhalten dieses periodischen Problems zu dem Verhalten eines Sturm-Liouville Problems mit getrennten Randbedingungen) \\

\textit{Hinweis.}
Bemerke als erstes dass $K$ injektiv ist.
Dann betrachte den Operator $L h := -h^\primeprime$ auf dem durch die Randbedingungen $h(0) = h(1) = 0$ festgelegten Definitionsbereich $D$.
Die Einschränkungen von $L + 1$ bzw. $K$ auf $D \cap E$ sind gleich.
Verwende dies, um zu zeigen dass $K$ surjektiv ist und dass $K^{-1}=(L+1)^{-1}+T$ gilt, wobei $T$ ein Operator mit endlichdimensionalem Bild ist.

\end{exercise}

\begin{solution}

ToDo!

\end{solution}
