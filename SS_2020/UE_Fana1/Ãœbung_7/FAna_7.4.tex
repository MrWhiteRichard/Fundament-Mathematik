\begin{exercise}[37/1]

Sei $k \in C([0, 1]^2)$ und $K \in \mathcal{B}(L^2(0, 1))$ der Integraloperator mit Kern $k$.
Sei weiters $k(s, t) = \overline{k(t, s)}$, sodass $K$ selbstadjungiert ist.
Schließlich sei

\begin{align*}
  K
  =
  \sum_{n \in \N}
  \lambda_n
  (\cdot, e_n)
  e_n
\end{align*}

die Spektralzerlegung von $K$.
Zeige

\begin{enumerate}[label = (\alph*)]

  \item
  Die Eigenfunktionen $e_n$ sind stetig auf $[0, 1]$.

  \item
  Für jedes $f \in L^2(0, 1)$ konvergiert die Reihe $K f = \sum_{n \in \N} \lambda_n (f, e_n) e_n$ absolut für jedes feste $s$ und gleichmäßig auf $[0, 1]$.

\end{enumerate}

\end{exercise}

\begin{solution}

Man erinnere sich an die Definition von $K: L^2(0, 1) \to L^2(0, 1)$.

\begin{align*}
  K(f) :=
  \pbraces
  {
    s
    \mapsto
    \Int[0][1]{k(s, t)f(t)}{\lambda(t)}
  }.
\end{align*}

Wir wissen von letzter Woche (IO), dass

\begin{itemize}

  \item
  $\norm[\infty]{k} = \norm[L^2(\mu \times \mu)]{K}$,

  \item
  die Hilbertraum-Adjugierte $K^\ast$, der Integraloperator mit Kern $\overline{k(y,x)}$ ist, und

  \item
  dass wegen $k \in C([0, 1]^2) \subset L^2(\lambda \times \lambda)$, $K$ sogar ein kompakter Operator ist.

\end{itemize}

\includegraphicsboxed{Spektralsatz}

Also ist Korollar 7.2.13 anwendbar.

\begin{enumerate}[label = (\alph*)]

  \item
  $\Forall n \in \N:$

  \begin{align*}
    K e_n - \lambda_n e_n = 0
    & \iff
    (K e_n)(s)
    =
    \Int[0][1]{k(s, t)e_n(t)}{t}
    =
    \lambda_n e_n(s) \\
    & \iff
    e_n(s)
    =
    \Int[0][1]
    {k(s, t) e_n(t) \lambda_n^{-1}}{t}.
  \end{align*}

  Seien $(s_n)_{n \in \N} \subset [0, 1]$, mit $s_n \to s$, $n \in \N$, und

  \begin{align*}
    h_\ell(t)
    :=
    \frac{k(s_\ell, t) e_n(t)}{\lambda_n},
    \quad
    \ell \in \N,
    \quad
    h(t)
    :=
    \frac{k(s, t) e_n(t)}{\lambda_n}.
  \end{align*}

  Wegen der Stetigkeit von $k$, gilt die, notationell bereits angedeutete, Beziehung

  \begin{align*}
    h_\ell
    \xrightarrow
    [\ell \to \infty]
    {\text{punktweise}}
    h.
  \end{align*}

  $\Forall t \in [0, 1]:$

  \begin{align*}
    |h_\ell(t)|
    =
    \frac
    {
      |k(s_\ell, t)|
      |e_n(t)|
    }
    {|\lambda_n|}
    \leq
    \frac
    {
      \norm[\infty]{k}
      |e_n(t)|
    }
    {|\lambda_n|}
    =:
    g(t) \\
    \implies
    \norm[1]{g}
    =
    \Int[0][1]
    {|g(t)|}{\lambda(t)}
    =
    \frac
    {\norm[\infty]{k}}
    {|\lambda_n|}
    \underbrace
    {
      \Int[0][1]
      {|e_n(t)|}{t}
    }_{
      \leq
      \norm[2]{e_n}
      \norm[2]{1}
    }
    < \infty
  \end{align*}

  Mittels dominierter Konvergenz
  erhalten wir schließlich, dass $e_n$ folgensteig ist.

  \begin{align*}
    \lim_{\ell \to \infty}
    e_n(s_\ell)
    =
    \lim_{\ell \to \infty}
    \Int[0][1]{h_\ell(t)}{t}
    =
    \Int[0][1]{h(t)}{t}
    =
    e_n(s).
  \end{align*}

  \includegraphicsboxed{BLU1_2_4}

  Da $\R$ eine abzählbare Umgebungsbasis $(U_{1/n}(q))_{n \in \N, q \in \Q}$ hat, so auch $[0, 1]$.
  Laut Blümlinger Satz 1.2.4, gilt also sogar Stetigkeit.

  \item
  $\Forall f \in L^2(0, 1):$

  \begin{align*}
    \sum_{n \in \N}
    |(f, e_n)\lambda_n e_n(s)|
     &= \sum_{n \in \N}
     |(f, e_n)(k(s,\cdot),e_n)| \leq
     \left(\sum_{n \in \N}
     |(f, e_n)|^2\right)^{\nicefrac{1}{2}}
     \left(\sum_{n \in \N}|(k(s,\cdot),e_n)|^2\right)^{\nicefrac{1}{2}} \\
    &= \norm[2]{k(s,\cdot)}\norm[2]{f} \leq\norm[\infty]{k}\norm[2]{f}
  \end{align*}

  Die vorletzte Gleichung ist die Parceval'sche Gleichung.
  Also konvergiert die Funktionenreihe absolut für jedes feste $s \in [0,1]$
  und auch gleichmäßig. \\

\end{enumerate}

\end{solution}
