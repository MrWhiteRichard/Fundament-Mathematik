\begin{exercise}[37/1]

Sei $k \in C([0, 1]^2)$ und $K \in \mathcal{B}(L^2(0, 1))$ der Integraloperator mit Kern $k$.
Sei weiters $k(s, t) = \overline{k(t, s)}$, sodass $K$ selbstadjungiert ist.
Schließlich sei

\begin{align*}
  K
  =
  \sum_{n \in \N}
  \lambda_n
  (\cdot, e_n)
  e_n
\end{align*}

die Spektralzerlegung von $K$.
Zeige

\begin{enumerate}[label = (\alph*)]

  \item
  Die Eigenfunktionen $e_n$ sind stetig auf $[0, 1]$.

  \item
  Für jedes $f \in L^2(0, 1)$ konvergiert die Reihe $K f = \sum_{n \in \N} \lambda_n (f, e_n) e_n$ absolut und gleichmäßig auf $[0, 1]$.

\end{enumerate}

\end{exercise}

\begin{solution}

Man erinnere sich an die Definition von $K: L^2(0,1) \to L^2(0,1)$.
\begin{align*}
  K(f) := \left(s \mapsto \int_0^1k(s,t)f(t)d\lambda(t)\right).
\end{align*}
Weiters wissen wir noch von letzter Woche, dass $\|K\| = \|k\|_{L^2(\mu \times \mu)}$ und
dass die Konjugierte der Operator mit Kern $\overline{k(y,x)}$ ist. Ebenso folgt
aus $k \in C([0, 1]^2) \subset L^2(\lambda \times \lambda)$ mit IO/2, dass $K$
sogar ein kompakter Operator ist.
Also ist die neue Version (Korollar 7.2.13) das Spektralsatzes für kompakte,
selbstadjungierte Operatoren anwendbar.
\begin{enumerate}[label = (\alph*)]
  \item Sei $n \in \N$ beliebig. Nach Konstruktion der $e_n$ gilt $e_n \in \ker(K - \lambda_nI)$,
  also
  \begin{align*}
    Ke_n - \lambda_ne_n = 0 &\iff
    (Ke_n)(s) = \int_0^1 k(s,t)e_n(t)dt = \lambda_n e_n(s) \\
    &\iff e_n(s) = \int_0^1 k(s,t)e_n(t)\lambda_n^{-1}dt.
  \end{align*}
  Wähle nun $(s_n)_{n\in\N} \subset [0,1]$ mit $s_n \to s$ beliebig. Definiere
  \begin{align*}
    h_{\ell}(t) := k(s_{\ell},t)e_n(t)\lambda_n^{-1}, \quad
    h(t) := k(s,t)e_n(t)\lambda_n^{-1}.
  \end{align*}
  Wegen der Stetigkeit von $k$ gilt die notationell bereits angedeutete Beziehung
  $h_{\ell}(t) \to h(t)$ punktweise. Weiters gilt für alle $n \in \N, t \in [0,1]$
  \begin{align*}
    |h_{\ell}(t)| = |k(s_{\ell},t)||e_n(t)\lambda_n^{-1}| \leq
    \|k\|_{\infty}|\lambda_n^{-1}||e_n(t)| =: g(t). \\
    \int_0^1|g(t)|d\lambda(t) = \|k\|_{\infty}|\lambda_n^{-1}|\int_0^1|e_n(t)|dt
    \leq \|k\|_{\infty}|\lambda_n^{-1}|\|e_n\|_2\|1\|_2 < \infty,
  \end{align*}
  wobei die vorletzte Ungleichung mit Hölder gilt. Mit der dominierten Konvergenz
  erhalten wir schließlich
  \begin{align*}
    \lim_{\ell \to \infty} e_n(s_{\ell}) = \lim_{\ell \to \infty}\int_0^1h_{\ell}(t)dt
    = \int_0^1h(t)dt = e_n(s).
  \end{align*}
  Also ist $e_n$ folgensteig, da $[0,1]$ eine abzählbare Umgebungsbasis hat,
  folgt nach Blümlinger Satz 1.2.4. sogar die Stetigkeit.
  \item Sei $f \in L^2(0,1)$ beliebig.
  \begin{align*}
    \sum_{n \in \N}\|\lambda_n e_n (f,e_n)\|_{\infty} &=
    \sum_{n \in \N}|(f,e_n)|\|\lambda_ne_n\|_{\infty} =
    \sum_{n \in \N}|(f,e_n)|\|Ke_n\|_{\infty} =
    \sum_{n \in \N}|(f,e_n)|\left\|\int_0^1k(s,t)e_n(t)dt\right\|_{\infty} \\ &\leq
    \sum_{n \in \N}|(f,e_n)|
    \left\|\left(\int_0^1|k(s,t)|^2dt\right)^{\nicefrac{1}{2}}\|e_n\|_{L^2}\right\|_{\infty} \leq
    \|k\|_{\infty}\sum_{n \in \N}|(f,e_n)| = \|k\|_{\infty}\|f\|_{L^2},
  \end{align*}
  wobei die letzte Gleichung die Parsevalsche Gleichung ist. Also konvergiert die
  Funktionenreihe absolut und damit auch gleichmäßig. \\
\end{enumerate}
\end{solution}
