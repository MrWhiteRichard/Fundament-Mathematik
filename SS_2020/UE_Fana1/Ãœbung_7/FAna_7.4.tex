\begin{exercise}[37/1]

Sei $k \in C([0, 1]^2)$ und $K \in \mathcal{B}(\L^2(0, 1))$ der Integraloperator mit Kern $k$.
Sei weiters $k(s, t) = \overline{k(t, s)}$, sodass $K$ selbstadjungiert ist.
Schließlich sei

\begin{align*}
  K
  =
  \sum_{n \in \N}
  \lambda_n
  (\cdot, e_n)
  e_n
\end{align*}

die Spektralzerlegung von $K$.
Zeige

\begin{enumerate}[label = (\alph*)]

  \item
  Die Eigenfunktionen $e_n$ sind stetig auf $[0, 1]$.

  \item
  Für jedes $f \in L^2(0, 1)$ konvergiert die Reihe $K f = \sum_{n \in \N} \lambda_n (f, e_n) e_n$ absolut und gleichmäßig auf $[0, 1]$.

\end{enumerate}

\end{exercise}

\begin{solution}

ToDo!

\end{solution}
