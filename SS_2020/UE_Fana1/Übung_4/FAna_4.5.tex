\begin{exercise}
Man zeige: Sei $H$ ein Hilbertraum, $A: H \to H$ linear und symmetrisch, also
\begin{align*}
  (Ax,y) = (x,Ay), \qquad x,y \in H.
\end{align*}
Dann ist $A$ stetig. \\
Die Vollstädnigkeit spielt hier eine entscheidende Rolle: Betrachte den Vektorraum $X$
aller Funktionen $f: \R \to \C$, die beliebig oft stetig differenzierbar sind und
außerhalb des Intervalls $[0,1]$ verschwinden, und versehe ihn mit dem inneren Produkt
\begin{align*}
  (f,g) := \int_0^1 f(t)\overline{g(t)}dt, \qquad f,g \in X.
\end{align*}
Wir betrachten damit jenen Teilraum von $L^2(0,1)$, der aus $C^{\infty}$-Funktionen
besteht, die am Rand verschwinden. Da $X$ in $L^2(0,1)$ dicht liegt, aber sicher
$X \neq L^2(0,1)$ ist, kann $X$ nicht vollständig sein. \\
Zeige, dass der Operator
\begin{align*}
  A: \begin{cases}
    X &\to X \\
    f(t) &\mapsto if^{\prime}(t)
  \end{cases}
\end{align*}
in diesem Sinne symmetrisch ist, aber nicht stetig.
\end{exercise}
\begin{solution}
Beweis.
\end{solution}
