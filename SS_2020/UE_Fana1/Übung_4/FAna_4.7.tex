\begin{exercise}
Sei $\mu$ das normierte Lebesguemaß $d\mu = \frac{1}{2\pi}dx$. Betrachte die Elemente
beziehungsweise Teilräume im $L^2([0,2\pi], \mu)$, die definiert sind als
\begin{align*}
  e_n(t) &:= \exp(int), n \in \Z, \qquad u_n(t) := \frac{e_{-n}(t) + ne_n(t)}{\sqrt{1+n^2}}, n \in \N, \\
  M &:= \overline{\spn\{e_n: n = 0,1,\dots\}}, \qquad N := \overline{\spn\{u_n: n = 0,1,\dots\}}.
\end{align*}
Zeige die folgenden Aussagen:
\begin{enumerate}[label = (\alph*)]
  \item Die Räume $M$ und $N$ sind, versehen mit dem $L^2(\mu)$-Skalarprodukt, Hilberträume.
  Die Mengen $\{e_n: n = 0,1,\dots\}$, beziehungsweise
  $\{u_n: n = 0,1,\dots\}$ sind Orthonormalbasen von $M$, beziehungsweise $N$.
  \item $M \cap N = \{0\}$.
  \item $M \dot+ N$ ist dicht in $L^2([0,2\pi],\mu)$, aber nicht gleich ganz $L^2([0,2\pi],\mu)$.
  \item Die Projektion des normierten Raumes $X := M \dot + N$ mit Bild $M$ und Kern $N$
  ist nicht stetig.
  \item Finde eine stetige Projektion von $L^2([0,2\pi],\mu)$ auf $M$.
\end{enumerate}
\end{exercise}
\begin{solution}
Beweis.
\end{solution}
