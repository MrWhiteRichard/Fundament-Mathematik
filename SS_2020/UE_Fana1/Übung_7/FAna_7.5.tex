\begin{exercise}[37/2]

Sei $A: L^2([0, 1]) \to L^2([0, 1])$ definiert durch

\begin{align*}
  (A f)(x)
  =
  i
  \Int[{[0, x]}]{f}{\lambda}
  - \frac{i}{2}
  \Int[{[0, 1]}]{f}{\lambda}.
\end{align*}

Zeige, dass $A$ kompakt und selbstadjungiert ist.
Bestimme das Spektralmaß von $A$, und finde eine Orthonormalbasis des $L^2([0, 1])$ die aus aus Eigenvektoren von $A$ besteht.

\end{exercise}

\begin{solution}
Sei $f \in L^2([0,1])$ beliebig. Definiere
\begin{align*}
  A_1:& L^2([0,1]) \mapsto L^2([0,1]): (A_1f)(x) = \int_{[0,x]}f d\lambda  \\
  A_2:& L^2([0,1]) \mapsto L^2([0,1]): (A_2f)(x) = \int_{[0,1]}f d\lambda
\end{align*}
$A_2$ ist offensichtlich selbstadjungiert und nach Aufgabe IO/2 auch kompakt.
Der Operator $A_1$ ist uns auch schon unter dem Namen Volterra-Operator bekannt,
von dem wir auch bereits wissen, dass er kompakt ist und dass
$(A_1^*f)(x) = \int_{[x,1]}f d\lambda$. Da die Menge der kompakten Operatoren
ein linearer Unterraum von $L_b([0,1])$ ist, ist damit auch $A$ kompakt.
Weiters berechnen wir
\begin{align*}
  (Af,g) = (iA_1f - \frac{i}{2}A_2f,g) = i(A_1f,g) - \frac{i}{2}(A_2f,g)
  = i(f,A_1^*g) - \frac{i}{2}(f,A_2g)
  = (f,-iA_1^*g + \frac{i}{2}A_2g).
\end{align*}
Es gilt
\begin{align*}
  -iA_1^*g + \frac{i}{2}A_2g &= \int_x^1 -ig d\lambda + \int_0^1\frac{i}{2}g d\lambda
  = \int_0^1i\left(\frac{1}{2} -\1_{[x,1]}\right)g d\lambda =
  \int_0^1i\left(\1_{[0,x]} -\frac{1}{2}\right)g d\lambda \\
  &= \int_0^x ig d\lambda - \int_0^1 \frac{i}{2}g d\lambda = Ag.
\end{align*}
Also ist $A$ selbstadjungiert. Als nächstes wollen wir
$\sigma(A)\backslash\{0\} = \sigma_p(A)\backslash\{0\} \subset \R$ bestimmen.
Eine Eigenfunktion müsste die Gleichung $\lambda f(x) = A f(x)$ fast überall erfüllen, also
\begin{align*}
  \lambda f(x) = i \int_0^x f d\lambda - \frac{i}{2}\int_0^1 f d\lambda.
\end{align*}
Mit dem Hauptsatz für Lebesgue-Integrale folgt daraus
\begin{align*}
  \lambda f^{\prime}(x) = if(x) \quad \lambda- \text{ f. ü.}
\end{align*}
Wir können die allgemeine Lösung dieser ODE angeben mit
\begin{align*}
  f_{a,\lambda}(t) := a\exp\left(\frac{i}{\lambda}t\right), \quad a \in \C.
\end{align*}
Wir berechnen
\begin{align*}
  A f_{a,\lambda}(x) &=
  i \int_0^x f_{a,\lambda} d\lambda - \frac{i}{2}\int_0^1 f_{a,\lambda} d\lambda =
  i \int_0^x a\exp\left(\frac{i}{\lambda}t\right) d\lambda -
  \frac{i}{2}\int_0^1 a\exp\left(\frac{i}{\lambda}t\right) d\lambda \\
  &= ai \frac{\lambda}{i}\left(\exp\left(\frac{i}{\lambda}x\right) - 1\right)
  - a\frac{i}{2}\frac{\lambda}{i}\left(\exp\left(\frac{i}{\lambda}\right) - 1\right)
  = \frac{a\lambda}{2}\left(2\left(\exp\left(\frac{i}{\lambda}x\right) - 1\right)
  - \left(\exp\left(\frac{i}{\lambda}\right) - 1\right)\right) \\
  &= \lambda f_{a,\lambda}(x) + \frac{a\lambda}{2}
  \left(\left(1 - \exp\left(\frac{i}{\lambda}\right)\right) -  2\right)
  \stackrel{!}{=} \lambda f_{a,\lambda}(x)
\end{align*}
Also muss gelten
\begin{align*}
  \exp\left(\frac{i}{\lambda}\right) = -1 \iff \lambda = \frac{1}{(2k+1)\pi}.
\end{align*}
Nun setze $e_k := \frac{f_{1,\lambda_k}}{\|f_{1,\lambda_k}\|_2},
\lambda_k := \frac{1}{(2k+1)\pi}, k \in \Z$.
Die Vektoren $e_k$ bilden eine Orthonormalbasis des $L^2([0,1])$ und wir erhalten
das Spektralmaß $E$ mit $E(\{\lambda_k\}) = P_k = (\cdot,e_k)e_k$.
\end{solution}
