\begin{exercise}[38/1*]

Betrachte den Hilbertraum $L^2(0, 1)$, seinen Teilraum

\begin{align*}
  E
  :=
  \Bbraces
  {
    h \in C^1(0, 1):
    h^\prime ~\text{absolut stetig},
    h^\primeprime \in L^2(0, 1),
    h(0) = h(1),
    h^\prime(0) = h^\prime(1)
  },
\end{align*}

und die lineare Abbildung $K: E \to L^2(0, 1)$ die definiert ist als $K h := -h^\primeprime + h$.
Zeige, dass $K$ bijektiv ist, und dass $K^{-1}$ kompakt und selbstadjungiert ist.
Bestimme $\sigma(K^{-1})$ und die Eigenräume $\ker{(K^{-1} - \lambda)}$ für $\lambda \in \sigma_p(K^{-1})$.
(Man beachte den Unterschied im Verhalten dieses periodischen Problems zu dem Verhalten eines Sturm-Liouville Problems mit getrennten Randbedingungen) \\

\textit{Hinweis.}
Bemerke als erstes dass $K$ injektiv ist.
Dann betrachte den Operator $L h := -h^\primeprime$ auf dem durch die Randbedingungen $h(0) = h(1) = 0$ festgelegten Definitionsbereich $D$.
Die Einschränkungen von $L + 1$ bzw. $K$ auf $D \cap E$ sind gleich.
Verwende dies, um zu zeigen dass $K$ surjektiv ist und dass $K^{-1}=(L+1)^{-1}+T$ gilt, wobei $T$ ein Operator mit endlichdimensionalem Bild ist.

\end{exercise}

\begin{solution}

\underline{Injektivität:} \\

Elemente $h \in \ker{K}$ sind Lösungen des RWP

\begin{align*}
  h^\primeprime = h,
  \quad
  h(0) = h(1),
  \quad
  h^\prime(0) = h^\prime(1).
\end{align*}

Ein Fundamentalsystem der ODE ist gegeben durch $(e^x, e^{-x})$.
Lösungen haben daher folgende Form.

\begin{align*}
  h(x)
  & =
  h_1 e^x + h_2 e^{-x},
  \quad
  h_1, h_2 \in \C \\
  \implies
  h^\prime(x)
  & =
  (h_1 e^x + h_2 e^{-x})^\prime
  =
  h_1 e^x - h_2 e^{-x}
\end{align*}

Wir rechnen, unter Berücksichtigung der RWB, nach, dass $h_1 = h_2 = 0$.
Die erste RWB führt zu

\begin{align*}
  h_1 + h_2
  =
  h(0)
  =
  h(1)
  =
  h_1 e + h_2 / e
  \implies
  h_1 (1 - e) + h_2 (1 - 1/e) = 0.
\end{align*}

Die zweite RWP führt zu

\begin{align*}
  h_1 - h_2
  =
  h^\prime(0)
  =
  h^\prime(1)
  =
  h_1 e - h_2 / e
  \implies
  h_1 (1 - e) - h_2 (1 - 1/e) = 0.
\end{align*}

Das lässt sich zu einem homogenen LGS zusammenfassen.

\begin{align*}
  \underbrace
  {
    \begin{pmatrix}
    1 - e & 1 - 1/e \\
    1 - e & 1/e - 1
    \end{pmatrix}
  }_{\in \GL{2, \R}}
  \begin{pmatrix}
    h_1 \\ h_2
  \end{pmatrix}
  = 0
\end{align*}

Dieses hat, aufgrund der Invertiebarkeit der Matrix, nur die triviale Lösung $h_1 = h_2 = 0$.
Daher, muss $\ker{K} = \Bbraces{0}$.
Folglich, ist $K$ injektiv.

\end{solution}
