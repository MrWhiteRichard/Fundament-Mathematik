\begin{exercise}[34/2]

Sei $\mu$ ein endliches positives Maß auf $(\Omega, \mathcal{A})$,
und sei $E(\Delta) := M_{\1_\Delta} \in \mathcal{B}(L^2(\mu))$ wobei $M_{\1_\Delta}$
der Multiplikationsoperator mit $\1_\Delta$ ist (d.h. $M_{\1_\Delta} f = \1_\Delta \cdot f$).
Zeige, dass $E$ ein Spektralmaß ist, und berechne $A := \Int{\phi}{E}$ für $\phi \in \mathrm{BM}(\Omega, \C)$.

\end{exercise}

\begin{solution}
Wir zeigen zuerst, dass $E$ ein Spektralmaß ist, also
\begin{itemize}
  \item $\forall \Delta \in \mathcal{A}: E(\Delta)$ ist eine Orthogonalprojektion: \\
  Zuallererst bemerken wir, dass $E(\Delta)$ wohldefiniert ist, da für $f \in L^2(\mu)$
  \begin{align*}
    \int_\Omega |E(\Delta)f|^2 d\mu = \int_\Delta |f|^2 \leq \|f\|_{L^2}.
  \end{align*}
  Damit ist $E(\Delta)$ sogar eine Kontraktion und insbesondere ein beschränkter Operator. \\
  Man sieht leicht, dass $E(\Delta)$ eine Projektion ist:
  \begin{align*}
    E(\Delta)(\lambda f + g) &= M_{\1_\Delta}(\lambda f + g) = \lambda M_{\1_\Delta}f
    + M_{\1_\Delta}g = \lambda E(\Delta)(f) + E(\Delta)(g), \\
    E(\Delta)^2 &= M_{\1_\Delta}^2 = M_{\1_\Delta} = E(\Delta).
  \end{align*}
  Wir wissen, dass eine Projektion genau dann orthogonal ist, wenn $E(\Delta) = E(\Delta)^*$.
  Seien $f,g \in L^2(\mu)$ beliebig und berechne
  \begin{align*}
    (E(\Delta)f,g) = \int_\Omega \1_\Delta f\overline{g} d\mu
    = \int_\Omega  f\overline{ \1_\Delta g} d\mu = (f,E(\Delta)g).
  \end{align*}
  \item $E(\emptyset) = 0, E(\Omega) = 1$:
  \begin{align*}
    E(\emptyset) = M_{\1_{\emptyset}} = 0. \\
    E(\Omega) = M_{\1_{\Omega}} = I. \\
  \end{align*}
  \item $\forall \Delta_1,\Delta_2 \in \mathcal{A}:
  E(\Delta_1 \cap \Delta_2) = E(\Delta_1)E(\Delta_2)$: \\
  Seien $\Delta_1,\Delta_2 \in \mathcal{A}, f \in L^2(\mu)$ beliebig. Es gilt
  \begin{align*}
    E(\Delta_1 \cap \Delta_2)f = M_{\1_{\Delta_1 \cap \Delta_2}}f
    = \1_{\Delta_1 \cap \Delta_2}f = \1_{\Delta_1}\1_{\Delta_2}f
    = E(\Delta_1)E(\Delta_2)f.
  \end{align*}
  \item Für alle $(\Delta_n)_{n \in \N} \subset \mathcal{A}$ paarweise disjunkt, $f \in L^2(\mu)$ gilt $E(\dot\bigcup_{n \in \N}\Delta_n)f = \sum_{n \in \N}E(\Delta_n)f$: \\
  Wir müssen also zeigen, dass die Funktion $\sum_{n \in \N}\1_{\Delta_n}f$
  in der $L^2(\mu)$-Norm gegen $\1_{\bigcup_{n \in \N}\Delta_n}f$ konvergiert.
  Die Funktionenfolge $g_k := \sum_{n = 1}^k\1_{\Delta_n}f - \1_{\bigcup_{n \in \N}\Delta_n}f$ ist monoton nichtfallend mit $g_k \geq -f$ für $k \in \N$
  und konvergiert punktweise gegen $0$.
  Mit dem Satz von Beppo Levi folgt daher
  \begin{align*}
  \lim_{N \to \infty}\int_\Omega |\sum_{n = 1}^N\1_{\Delta_n}f - \1_{\bigcup_{n \in \N}\Delta_n}f|^2 d\mu
  &= \lim_{N \to \infty}\int_\Omega \1_{\bigcup_{n = 1}^N\Delta_n}f -
  \1_{\bigcup_{n \in \N}\Delta_n}f
   d\mu \\
  &= \int_\Omega \lim_{N \to \infty} \1_{\bigcup_{n = 1}^N\Delta_n}f -
  \1_{\bigcup_{n \in \N}\Delta_n}f d\mu \xrightarrow{N \to \infty} 0.
  \end{align*}
\end{itemize}
Sei nun $\phi \in BM(\Omega,\C)$ beliebig.
\begin{align*}
  (Ag,h) = \int_{\Omega}\phi dE_{g,h}.
\end{align*}
Die Frage ist jetzt: Was ist $E_{g,h}$ für ein Maß?
\begin{align*}
  E_{g,h}(\Delta) = (E(\Delta)g,h).
\end{align*}
Mittels dem Approximationssatz für messbare Funktionen können wir $\phi$
als punktweisen Grenzwert einer Folge von Treppenfunktionen darstellen.
Da $\phi$ beschränkt ist, können wir diese Folge sogar monoton nichtfallend wählen.
Wieder bemühen wir den alten Beppo Levi:
\begin{align*}
  (Ag,h) &= \int_{\Omega}\phi dE_{g,h} = \int_{\Omega}\lim_{n \to \infty}
  \sum_{k=0}^{k(n)}\alpha_k\1_{\Delta_k} dE_{g,h}
  = \lim_{n \to \infty}\int_{\Omega}\sum_{k=0}^{k(n)}\alpha_k\1_{\Delta_k} dE_{g,h} \\
  &= \lim_{n \to \infty}\sum_{k=0}^{k(n)}\alpha_k\int_{\Omega}\1_{\Delta_k} dE_{g,h}
  = \lim_{n \to \infty}\sum_{k=0}^{k(n)}\alpha_k(E(\Delta_k)g,h)
  = \lim_{n \to \infty}\left(\sum_{k=0}^{k(n)}\alpha_kE(\Delta_k)g,h\right) \\
  &= \left(\lim_{n \to \infty}\sum_{k=0}^{k(n)}\alpha_kE(\Delta_k)g,h\right).
\end{align*}
Also gilt $\Int{\phi}{E} = \lim_{n \to \infty}\sum_{k=0}^{k(n)}\alpha_kE(\Delta_k)$.

**** Alternative Berechnung des Integrals***
Für alle $g, h \in L^2(\mu)$ folgt aus dem Satz von Radon-Nikodym mit
\begin{align}
    E_{g, h}(\Delta) = (E(\Delta) g, h) = (\1_\Delta g, h) = \int_{\Delta} g \overline h~\mathrm{d} \mu,
\end{align}
dass $E_{g, h}$ absolutstetig bezüglich $\mu$ mit Dichte $g\overline h$ ist.

Damit können wir $\int \phi$ d$E$ berechnen:
\begin{align}
    ((\int \phi ~\mathrm{d}E)g, h) = \int \phi ~\mathrm{d}E_{g,h} = \int \phi g \overline h~\mathrm{d} \mu = (\phi g, h).
\end{align}

Diese Gleichung gilt für alle $g$ und $h$, wir erhalten also $\int \phi ~\mathrm{d}E = M_\phi$.
\end{solution}
