\begin{exercise}

Sei $X$ ein topologischer Vektorraum.
Eine Menge $B \subseteq X$ heißt beschränkt, falls es zu jeder Nullumgebung $U$ ein positive Zahl $\lambda_U$ gibt, sodass $B \subseteq \lambda_U U$.
Zeige, dass jede kompakte Teilmenge von $X$ beschränkt ist. Zeige, dass jeder lineare Teilraum $Y \neq \Bbraces{0}$ von $X$ unbeschränkt ist.

\end{exercise}

\begin{solution}

Sei $K \subseteq X$ kompakt, $U \in \mathfrak{U}(0)$ beliebig. \\
Wähle $W \subseteq U$ als kreisförmige Nullumgebung. \\
$W$ ist absorbierend, also: $\forall x \in X: \exists t > 0: x \in tW$. \\
Für alle $ \lambda \neq 0 \in \mathbb{C}$ ist die Muliplikation mit einem komplexen Skalar
ein Homöomorphismus und somit gilt: \\
\[ \bigcup_{t > 0}tW = X \supseteq K \] ist eine offene Überdeckung von K. \\
Nach der Definition einer kompakten Menge existiert davon eine endliche Teilüberdeckung: \\
\[ \bigcup_{i=1}^nt_iW \supseteq K \]
Aufgrund der Kreisförmigkeit von $W$ gilt: $t_j \geq t_i: t_jW \supseteq t_iW$ \\
und somit mit $t_{max} := \max_{i=1}^nt_i: t_{max}U \supseteq t_{max}W \supseteq K$. \\
Damit ist K beschränkt.

\end{solution}
