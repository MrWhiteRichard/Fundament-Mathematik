\begin{exercise}[IO/3]

Der Volterra-Operator ist der Integraloperator $(V f)(x) := \Int[0][x]{f(t)}{t}$.
Zeige, dass $V \in \mathcal{B}(L^2(0, 1))$ mit $\norm{V} = \frac{2}{\pi}$, dass $V$ kompakt ist, und dass

\begin{align*}
  (V^\ast f)(x)
  =
  \Int[x][1]{f(t)}{t}.
\end{align*}

Zeige, dass $\sigma(V) = \sigma_c(V) = \Bbraces{0}$.

\end{exercise}

\begin{solution}

Der Kern unseres Operators ist die Funktion $\1_{(0,x)}(t)$.
Das diese in $L^2((0,1) \times (0,1))$ zeigen wir ganz einfach.

\begin{align*}
  \int_{(0,1) \times (0,1)} \vbraces{\1_{(0,x)}(t)}^2 d\lambda(t) \times \lambda(x)
  =
  \int_0^1 \int_0^1 \vbraces{\1_{(0,x)}(t)} d\lambda(t) d\lambda(x)
  =
  \int_0^1 \int_0^x 1 d\lambda(t) d\lambda(x)
  =
  \int_0^1 x d\lambda(x)
  =
  \frac{1}{2}
\end{align*}

Damit haben wir nach IO/1 schon $V \in \mathcal{B}(L^2(0, 1))$ gezeigt und die
Kompaktheit von $V$ folgt aus IO/2.

Um die Adjungierte von $V$ zu erhalten benutzen wir IO/1. Wir betrachten also den
Integraloperator mit Kern $\1_{(0,t)}(x)$. Für diesen Indikator gilt:

\begin{align*}
  \1_{(0,t)}(x) = 1
  \Leftrightarrow
  0 < x < t < 1
  \Leftrightarrow
  \1_{(x,1)}(t) = 1
\end{align*}

Damit haben wir auch die Darstellung der Adjungierten wie in der Angabe.

Da wir es mit einem kompakten Operator zu tun haben, sind nach Satz 6.5.12 besteht
das Spektrum (ausgenommen der $0$) nur aus Eigenwerten.
\includegraphicsboxed{Satz 6.5.12.png}

Nehmen wir also an, es gibt ein Eigenpaar $\lambda \neq 0, f \neq 0$, also

\begin{align*}
  Vf(x) - \lambda f(x) = 0
  \Leftrightarrow
  \Int[0][x]{f(t)}{t} = \lambda f(x)
\end{align*}

Aus der Maßtheorie wissen wir, dass $Vf(x)$ stetig ist, somit auch $f$. Damit ist
$f$ aber auch schon stetig differenzierbar. Nach dem Hauptsatz der Differential und
Integralrechnung gilt:

\begin{align*}
  f^\prime (x)
  =
  \frac{f(x)}{\lambda}
\end{align*}

Die Lösung dieser Differentialgleichung ist

\begin{align*}
  f(x)
  =
  c \exp^{\frac{x}{\lambda}}
\end{align*}

Um die Konstante $c$ zu bestimmen sehen wir uns zuerst

\begin{align*}
  Vf(0) = \lambda f(0) = 0
\end{align*}

Damit erhalten wir

\begin{align*}
  f(0) = 0
  \Rightarrow
  c = 0
  \Rightarrow
  f \equiv 0
\end{align*}

im Widerspruch zu unserer Annahme. Um nun noch zu zeigen, dass $0 \in \sigma_c(V)$
bemerkten wir zuerst, dass wegen Satz 6.4.14 $\sigma(V) = {0}$ gilt. Außerdem ist,
wie man sich leicht überzeugt, $0 \notin \sigma_p(V)$. Aus Analysis 3 wissen wir, dass
$C_c^{\infty}$ dicht in $L^2$ ist.

\includegraphicsboxed{Blümlinger - Satz 2.5.1.jpg}

Sei nun $f: (0,1) \rightarrow \C$ für die auch $ f \in C^{\infty}$ gilt. Dann ist

\begin{align*}
  f^\prime(x) \in L^2(0,1)
\end{align*}

da die Funktion ja auch in $C_c^{\infty}$ ist. Dann gilt also (nach HS der
Differential und Integralrechnung)

\begin{align*}
  Vf^\prime(x) = f(x)
  \Rightarrow
  C_c^{\infty} \subseteq \ran(V)
\end{align*}

Da diese schon dicht in $L^2(0,1)$ sind also auch insbesondere $\ran(V)$.

Zuguterletzt widmen wir uns noch der Norm von $V$. Dazu wollen wir Proposition
6.6.2 (i) verwenden.

\includegraphicsboxed{Proposition 6.6.2.jpg}

Also $\Vbraces{V} = \sqrt{\Vbraces{V^*V}}$. Mit den Rechenregeln für die Adjungierte
überzeugt man sich leicht, dass $S:=V^*V$ ein normaler Operator ist, deswegen gilt
$r(S) = \Vbraces{S}$ (Proposition 6.6.5). Er ist, wegen der
Kompaktheit von $V$, auch kompakt. Nun ist $S$ definiert durch

\begin{align*}
  (Sf)(x)
  =
  \Int[x][1]{
    \Int[0][y]{f(t)}{t}
  }{y}
\end{align*}

Sehen wir uns wieder Eigenpaare $\lambda \neq 0, f \neq 0$ von $S$ an. Dann gilt

\begin{align*}
  Sf(s) = \lambda f(x)
\end{align*}

Durch zweimaliges differenzieren erhalten wir

\begin{align*}
  \lambda f^\primeprime(x) = -f(x)
\end{align*}

Mit den Nullstellen des charakteristischen Polynoms dieser Differentialgleichung
erhalten wir eine allgemeine Lösung mit $a,b \in \C, \omega = \sqrt{\frac{1}{\lambda}}$

\begin{align*}
  f(x) = a e^{i\omega x} + b e^{-i \omega x}
\end{align*}

Um die Koeffizienten zu bestimmen wenden wir den Operator an

\begin{align*}
  Sf(x)
  =&
  a \Int[x][1]{
    \Int[0][y]{e^{i\omega t}}{t}
  }{y}
  +
  b \Int[x][1]{
      \Int[0][y]{e^{-i\omega t}}{t}
  }{y} \\
  =&
  \frac{1}{\omega^2} f(x)
  +
  \frac{1}{i\omega} (a-b)x
  -
  \frac{1}{\omega^2}(ae^{i\omega} + be^{-i\omega})
  -
  \frac{1}{i\omega}(a-b)
  \stackrel{!}{=}
  \lambda f(x)
\end{align*}

Aus dem zweiten und vierten Term erkennen wir, dass $a=b$ sein muss, also mit
Hilfe der eulerschen Identität $f(x) = 2a \cos(\omega x)$. Aus dem dritten Term,
da ja $a \neq 0$ weil sonst ja $f = 0$ wäre, erhalten wir $2cos(\omega) = 0$. Damit
erhalten wir unsere Eigenwerte

\begin{align*}
  \omega
  =
  \frac{2n+1}{2} \pi     \quad n \in \Z
  \Rightarrow
  \lambda_n
  =
  \frac{4}{(2n+1)^2 \pi^2}
\end{align*}

Um nun $r(S)$ zu erhalten nehmen wir den Betragsgrößten, also $\lambda_0$.
Damit gilt $\Vbraces{S} = \frac{4}{\pi^2}$ womit wir nun Zuguterletzt
$\Vbraces{V} = \frac{2}{\pi}$ erhalten.
\end{solution}
