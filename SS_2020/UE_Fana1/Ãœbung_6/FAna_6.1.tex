\begin{exercise}[23/1]

Sei $S$ der Shift-Operator am $\ell^1(\N)$.

\begin{enumerate}[label = (\alph*)]

  \item
  Zeige, dass

  \begin{align*}
    \sigma_p(S^\ast) = \sigma_r(S) = \Bbraces{\lambda \in \C: |\lambda| < 1}, \\
    \sigma_c(S^\ast) = \sigma_c(S) = \Bbraces{\lambda \in \C: |\lambda| = 1}, \\
    \sigma_r(S^\ast) = \sigma_p(S) = \emptyset
  \end{align*}

  \item
  Bestimme $\sigma_{app}(S)$ und finde zu jedem Punkt $\lambda \in \sigma_{app}(S)$ eine Folge wie in der Definnition des approximativen Punktspektrums verlangt.

  \item
  Für $\lambda \in \C$ mit $|\lambda| \neq 1$ bestimme $\dim{(\ell^2(\N) / \ran{S - \lambda})}$.

\end{enumerate}

\end{exercise}

\begin{solution}

ToDo!

\end{solution}
