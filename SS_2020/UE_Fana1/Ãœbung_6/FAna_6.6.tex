\begin{exercise}[IO/4]

Sei $k \in C([0, 1]^2)$, und betrachte den Integraloperator $(K f)(x) := \Int[0][1]{k(x, t)f(t)}{t}$.
Zeige, dass $K \in \mathcal{B}(C([0, 1]))$ mit

\begin{align*}
  \norm{K}
  =
  \sup_{x \in [0, 1]} \Int[0][1]{|k(x, t)|}{t}
  \leq
  \norm[{C([0, 1]^2)}]{k}.
\end{align*}

Zeige, dass $K$ kompakt ist.

\end{exercise}

\begin{solution}

Die Linearität von $K$ folgt aus der Linearität des Integrals.

Dass für $f \in C([0,1])$ auch $Kf$ wieder stetig ist, sollt auch klar sein.

Sei nun $f \in C([0,1])$ beliebig mit $\|f\|_{\infty} = 1$. Dann gilt

\begin{align*}
  \|Kf\|_{\infty} =& \sup_{x \in (0,1)} | \int_{0}^{1} k(x,t)f(t)dt | \leq \sup_{x \in (0,1)} \int_{0}^{1} |k(x,t)||f(t)|dt \\
  \leq& \|f\|_{\infty} \sup_{x \in (0,1)} \int_{0}^{1} |k(x,t)|dt = \sup_{x \in (0,1)} \int_{0}^{1} |k(x,t)|dt
\end{align*}

Somit erhalten wir also $\|K\| \leq \sup_{x \in (0,1)} \int_{0}^{1} |k(x,t)|dt \leq \|k\|_{\infty}$.

Für die andere Richtung der ersten Ungleichung betrachten wir das $x_0 \in [0,1]$, welches $\int_{0}^{1} |k(x,t)|dt$ maximiert. Dieses existiert, da wir es mit einer stetigen Funktion zu tun haben.

Definiere nun $f(t) := \sgn(k(x_0,t)) \in L^1([0,1])$. Da $C([0,1])$ dicht in $L^1([0,1])$ liegt, können wir eine Folge $(f_n)_{n \in \N}$ finden, mit
\begin{align*}
  f_n \stackrel{\|\cdot\|_1}{\to} f, \Forall t \in [0,1]: |f_n(t)| \leq |f(t)|.
\end{align*}

Damit gilt

\begin{align*}
  \|K\| \geq& \lim_{n \to \infty} \|Kf_n\| \geq \lim_{n \to \infty} |Kf_n(x_0)| = \lim_{n \to \infty} |\int_{0}^{1} k(x_0,t)f_n(t) dt| \\
  =& |\int_{0}^{1} k(x_0,t)f(t) dt| = \int_{0}^{1} |k(x_0,t)| dt = \sup_{x \in (0,1)} \int_{0}^{1} |k(x_0,t)| dt.
\end{align*}

Nun zu der Kompaktheit von $K$.

 Um zu zeigen, dass $K(K_{1}^{\|\cdot\|_{\infty}}(0)) = \{Kf : f \in K_{1}^{\|\cdot\|_{\infty}}(0)\}$ relativ kompakt ist, verwenden wir den Satz von Arzela-Ascoli.

 Es reicht also zu zeigen, dass $K(K_{1}^{\|\cdot\|_{\infty}}(0))$ punktweise beschränkt und gleichgradig stetig ist.

 \begin{enumerate}

 \item Punktweise beschränkt

 Für $x \in [0,1]$ und $f \in K_{1}^{\|\cdot\|_{\infty}}(0)$ beliebig gilt:
\begin{align*}
  |Kf(x)| \leq \|Kf\|_{\infty} \leq \|K\| \|f\|_{\infty} \leq \|K\|
\end{align*}

\item Gleichgradig stetig:

Allgemein gilt für $x_1, x_2 \in [0,1]$ und $f \in K_{1}^{\|\cdot\|_{\infty}}(0)$ beliebig, dass
\begin{align*}
  |Kf(x_2) - Kf(x_1)| =& | \int_{0}^{1} k(x_2,t)f(t) - k(x_1,t)f(t) dt| \leq \int_{0}^{1} |f(t)||k(x_2,t)- k(x_1,t)|dt \\
  \leq& \int_{0}^{1} |k(x_2,t)- k(x_1,t)|dt
\end{align*}

Sei also $\epsilon > 0$ beliebig, dann wähle $\delta$ so, dass
\begin{align*}
  \Forall (x_1,t_1),(x_2,t_2): | (x_1,t_1) - (x_2,t_2)| \leq \delta \Rightarrow | k(x_1,t_1) - k(x_2,t_2)| \leq \epsilon
\end{align*}

Dann gilt für beliebiges $x_1 \in [0,1]$, dass für alle $f \in K_{1}^{\|\cdot\|_{\infty}}(0)$ und $x_2 \in [x_1-\delta,x_1+\delta]$

\begin{align*}
  |Kf(x_2) - Kf(x_1)| \leq \int_{0}^{1} |k(x_2,t)- k(x_1,t)|dt \leq \int_{0}^{1} \epsilon dt = \epsilon
\end{align*}

 \end{enumerate}


\end{solution}
