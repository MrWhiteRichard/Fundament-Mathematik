\begin{exercise}[IO/1]

Sei $X$ eine Menge, $\mu$ ein $\sigma$-endliches Maß auf $X$, sei $k \in L^2(\mu \times \mu)$, und betrachte den Integraloperator $K$ mit Kern $k$ und Maß $\mu$. \\

Zeige, dass $K \in \mathcal{B}(L^2(\mu))$ mit $\norm{K} \leq \norm[L^2(\mu \times \mu)]{k}$.
Zeige, dass seine Hilbertraumadjungierte $K^\ast \in \mathcal{B}(L^2(\mu))$ der Integraloperator mit Kern $k^\ast(x, y) := \overline{k(y, x)}$ ist.

\end{exercise}

\begin{solution}
  Die Linearität von $K$ folgt direkt aus der des Integrals.
  Wir zeigen zuerst $\|K\| \leq \|k\|$; insbesondere ist K ein beschränkter Operator:
  \begin{align}
      \|Kf\|_{L^2(\mu)} = \int_X |Kf(x)|^2 ~\mathrm{d}\mu(x)
      = \int_X \left|\int_X k(x,y) f(y) ~\mathrm{d}\mu(y)\right|^2 ~\mathrm{d}\mu(x) \\
      \leq \int_X \int_X |k(x,y)|^2 |f(y)|^2 ~\mathrm{d}\mu(y)
      ~\mathrm{d}\mu(x) \leq \|f\|_{L^2(\mu)} \int_X \int_X |k(x,y)|^2 ~\mathrm{d}\mu(y) ~\mathrm{d}\mu(x) \\ \stackrel{\text{Fubini}}{=} \|f\|_{L^2(\mu)} \int_{X^2} |k(x,y)|^2 ~\mathrm{d}(\mu\times\mu)(x,y)
      = \|f\|_{L^2(\mu)} \|k\|_{L^2(\mu\times\mu)}.
  \end{align}

  Zur Berechnung der Hilbertraumadjungierten erinnern wir uns an Folgendes:

  \begin{itemize}
      \item $L^2(\mu)$ ist isomorph zu seinem eigenen Dualraum vermöge folgender Abbildung:

      $\phi: L^2(\mu) \rightarrow L^2(\mu)^\prime: f \mapsto \int_X f\overline{g} \mathrm{d}\mu.$
      \item Die Konjugierte von $K$ ist wie folgt definiert:

      $K^\prime: L^2(\mu)^\prime \rightarrow L^2(\mu)^\prime:
      y^\prime \mapsto (x \mapsto y'(K(x))).$
  \end{itemize}

  Damit berechnen wir
  \begin{align}
      K^*(f) := \phi^{-1}(K(\phi(f))) = \phi^{-1}\left(K^\prime\left(g \mapsto \int_X f\overline g ~\mathrm{d}\mu\right)\right)
      = \phi^{-1}\left(z \mapsto \int_X f(x) ~\overline{K(z)}(x) ~\mathrm{d}\mu(x)\right) \\
      = \phi^{-1}\left(z \mapsto \int_X f(x)
      ~\overline{\int_X k(x,y) z(y) \mathrm{d}\mu(y)}
      ~\mathrm{d}\mu(x)\right)
      = \phi^{-1}\left(z \mapsto \int_X
      ~\int_X f(x) \overline{k(x,y)} \overline{z(y)} \mathrm{d}\mu(y)
      ~\mathrm{d}\mu(x)\right) \\
      \stackrel{\text{\textbf{Fubini***}}}{=} \phi^{-1}\left(z \mapsto \int_X
      ~\int_X f(x) \overline{k(x,y)} \overline{z(y)} \mathrm{d}\mu(x)
      ~\mathrm{d}\mu(y)\right)
      = \phi^{-1}\left(z \mapsto \int_X
      \left(\int_X f(x) \overline{k(x,y)} \mathrm{d}\mu(x) \right) \overline{z(y)}
      ~\mathrm{d}\mu(y)\right) \\
      = x \mapsto \int_X f(x) \overline{k(x,y)} \mathrm{d}\mu(x).
  \end{align}

  *** \textbf{Begründung warum das integrierbar ist fehlt}


\end{solution}
