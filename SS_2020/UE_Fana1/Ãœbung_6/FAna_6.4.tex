\begin{exercise}[IO/2]

Sei $X$ eine Menge und $\mu$ ein $\sigma$-endliches Maß auf $X$.
Zeige:

\begin{enumerate}[label = (\alph*)]

  \item
  Seien $a_i, b_i \in L^2(\mu), i = 1, \ldots, n$.
  Setze $k(s, t) := \sum_{i=1}^n a_i(s) b_i(t)$ und betrachte den Integraloperator $K \in \mathcal{B}(L^2(\mu))$ mit Kern $k$.
  Dann ist $\dim{\ran{K}} \leq n$.

  \item
  Sei $k \in L^2(\mu \times \mu)$.
  Dann ist der Integraloperator $K \in \mathcal{B}(L^2(\mu))$ mit Kern $k$ und Maß $\mu$ kompakt.

\end{enumerate}

\end{exercise}

\begin{solution}

\phantom{}

\begin{enumerate}[label = (\alph*)]

  \item
  Zuerst wollen wir uns ansehen, wie $K$ tatsächlich aussieht.

  \begin{align*}
    (K f)(s)
    & =
    \Int[X]
    {
      \sum_{i=1}^n
      a_i(s) b_i(t) f(t)
    }{\mu(t)}
    =
    \sum_{i=1}^n
    a_i(s) \Int[X]{b_i(t) f(t)}{\mu(t)}
    =
    \sum_{i=1}^n
    a_i(s) (b_i, f)
  \end{align*}

  Damit, sieht das Bild von $K$ wie folgt aus.

  \begin{align*}
    \ran{K}
    =
    \Bbraces
    {
      K f:
      f \in L^2(\mu)
    }
    =
    \Bbraces
    {
      \sum_{i=1}^n
      a_i (b_i, f):
      f \in L^2(\mu)
    }
    \subseteq
    \Span \Bbraces{a_1, \ldots, a_n}
    =: A_n
  \end{align*}

  Daher, muss $\dim \ran{K} \leq \dim{A_n} \leq n$.

  \item
  Sei $K_n$ ein Integraloperator der Form
\begin{align}
    K_n f(s) = \int_X \left(\sum_{i=1}^n a_i(s) b_i(t)\right) f(t) \dmu(t).
\end{align}

$K_n$ hat nach Aufgabe (a) endlichdimensionales Bild und ist daher nach Proposition 6.5.4 kompakt. Wir wollen $K$ nun durch eine Folge $(K_n)_{n \in \mathbb{N}}$ solcher Operatoren approximieren. Es gilt
\begin{align}
    \|(K-K_n) f\|_2^2 = \int_X \left| \int_X k(s,t) f(t) \dmu(t) - \int_X \left(\sum_{i=1}^n a_i(s) b_i(t)\right) f(t) \dmu(t) \right|^2 \dmu(s) \\
    \leq \int_X \left( \int_X |k(s,t) - \sum_{i=1}^n a_i(s) b_i(t)| |f(t)| \dmu(t)\right)^2 \dmu(s) \\
    \leq \int_X \left(\int_X |k(s,t) - \sum_{i=1}^n a_i(s) b_i(t)|^2 \dmu(t)\right) \left(\int_X |f(t)|^2 \dmu(t)\right) \dmu(s) \\
    = \|f\|_2 \int_{X^2} |k(s,t) - \sum_{i=1}^n a_i(s) b_i(t)|^2 ~\mathrm{d}(\mu\times\mu)(s,t).
\end{align}

Aus der Maßtheorie wissen wir, dass wir $k$ von unten durch Treppenfunktionen der Form $t_n(s,t) = \sum_{i=1}^{m_n} \alpha_{n_i} \mathds{1}_{C_{n_i}}(s,t)$ approximieren können; seien $(t_n)_{n \in \mathbb{N}}$ also von dieser Gestalt und gelte
\begin{align}
    \forall $n \in \mathbb{N}:~ \|t_n\|_{L^2(\mu\times\mu)} \leq \|k\|_{L^2(\mu\times\mu)}, ~~\lim\limits_{n \rightarrow \infty}{\|k-t_n\|_{L^2(\mu\times\mu)}} = 0.
\end{align}

Sei $C \in \mathfrak{S} \times \mathfrak{S}$ beliebig; weil $\mu$ sigmaendlich ist, können wir nach dem Approximationssatz (vgl. Kusolitsch Satz 2.59) ein $D$ aus dem zugehörigen Ring finden, dass $C$ beliebig genau approximiert. Genauer erhalten wir für beliebiges $\epsilon > 0$ ein $D$ mit
\begin{align}
    D = \bigcup_{i=1}^{l} A_j \times B_j \text{~(wobei~} A_j, B_j \in \mathfrak{S}), ~~\mu\times\mu(C \triangle D) < \epsilon.
\end{align}

Damit gilt
\begin{align}
    \|\mathds{1}_C - \mathds{1}_D\|_2^2 = \int_{X^2} |\mathds{1}_C(s,t) - \mathds{1}_D(s,t)|^2 ~\mathrm{d}(\mu\times\mu)(s,t) = \int_{X^2} \mathds{1}_{C \triangle D}~\mathrm{d}(\mu\times\mu)(s,t) = \mu\times\mu(C \triangle D) < \epsilon.
\end{align}

Wir können also eine Folge $D_n$ wählen mit $\|\mathds{1}_C - \mathds{1}_D\|_2 < \frac{1}{n},$ also $D_n \xrightarrow{\|.\|_2} C.$

Zurück zur Aufgabe: Für gegebenes $\epsilon > 0$ wählen wir für jedes $C_{n_i}$ ein $D_{n_i}$ mit $\|\mathds{1}_{C_{n_i}} - \mathds{1}_{D_{n_i}}\|_2 < \frac{\epsilon}{m_n ~|\alpha_{n_i}|}.$

Mit dieser Wahl ist
\begin{align}
    \left\| \sum_{i=1}^{m_n} \alpha_{n_i} \mathds{1}_{C_{n_i}} - \sum_{i=1}^{m_n} \alpha_{n_i} \mathds{D_{n_i}}\right\|_2
    \leq \sum_{i=1}^{m_n} |\alpha_{n_i}| \|\mathds{1}_{C_{n_i}} - \mathds{1}_{D_{n_i}} \|_2 < \epsilon.
\end{align}

Weiters gilt
\begin{align}
    \sum_{i=1}^{m_n} \alpha_{n_i} \mathds{1}_{D_{n_i}}(s,t)
    = \sum_{i=1}^{m_n} \alpha_{n_i} \sum_{j=1}^{l_{n_i}}  \mathds{1}_{A_{n_{i_j}} \times B_{n_{i_j}}}(s,t)
    = \sum_{i=1}^{m_n} \sum_{j=1}^{l_{n_i}}   \alpha_{n_i} \mathds{1}_{A_{n_{i_j}}}(s) \mathds{1}_{B_{n_{i_j}}}(t).
\end{align}

Wir finden also Funktionen der Form
\begin{align}
    s_{n_q} (s,t) = \sum_{j=1}^{p_{n_q}} \beta_{n_q} \mathds{1}_{A_{n_{i_j}}}(s) \mathds{1}_{B_{n_{i_j}}}(t)
\end{align}
mit $\|s_{n_q} - t_n\| \xrightarrow{q \rightarrow \infty} 0.$
Sei nun $\epsilon > 0$ beliebig. Dann gilt für hinreichend große $n$ und $q$
\begin{align}
    \| k-s_{n_q} \|_2 \leq \|k - t_n\|_2 + \|t_n - s_{n_q}\|_2 < \epsilon.
\end{align}

\end{enumerate}

\end{solution}
