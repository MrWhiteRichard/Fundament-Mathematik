\begin{exercise}[IO/2]

Sei $X$ eine Menge und $\mu$ ein $\sigma$-endliches Maß auf $X$.
Zeige:

\begin{enumerate}[label = (\alph*)]

  \item
  Seien $a_i, b_i \in L^2(\mu), i = 1, \ldots, n$.
  Setze $k(s, t) := \sum_{i=1}^n a_i(s) b_i(t)$ und betrachte den Integraloperator $K \in \mathcal{B}(L^2(\mu))$ mit Kern $k$.
  Dann ist $\dim{\ran{K}} \leq n$.

  \item
  Sei $k \in L^2(\mu \times \mu)$.
  Dann ist der Integraloperator $K \in \mathcal{B}(L^2(\mu))$ mit Kern $k$ und Maß $\mu$ kompakt.

\end{enumerate}

\end{exercise}

\begin{solution}

\phantom{}

\begin{enumerate}[label = (\alph*)]

  \item
  Zuerst wollen wir uns ansehen, wie $K$ tatsächlich aussieht.

  \begin{align*}
    (K f)(s)
    & =
    \Int[X]
    {
      \sum_{i=1}^n
      a_i(s) b_i(t) f(t)
    }{\mu(t)}
    =
    \sum_{i=1}^n
    a_i(s) \Int[X]{b_i(t) f(t)}{\mu(t)}
    =
    \sum_{i=1}^n
    a_i(s) (b_i, f)
  \end{align*}

  Damit, sieht das Bild von $K$ wie folgt aus.

  \begin{align*}
    \ran{K}
    =
    \Bbraces
    {
      K f:
      f \in L^2(\mu)
    }
    =
    \Bbraces
    {
      \sum_{i=1}^n
      a_i (b_i, f):
      f \in L^2(\mu)
    }
    \subseteq
    \Span \Bbraces{a_1, \ldots, a_n}
    =: A_n
  \end{align*}

  Daher, muss $\dim \ran{K} \leq \dim{A_n} \leq n$.

  \item
Aus der Maßtheorie (Lemma 13.37) wissen wir, dass wir $k$ von unten durch Treppenfunktionen der Form $t_n(s,t) = \sum_{i=1}^{m_n} \alpha_{n_i} \1_{C_{n_i}}(s,t)$ approximieren können; seien $(t_n)_{n \in \mathbb{N}}$ also von dieser Gestalt und gelte
\begin{align*}
    \forall n \in \mathbb{N}:~ \|t_n\|_{L^2(\mu\times\mu)} \leq \|k\|_{L^2(\mu\times\mu)}, ~~\lim\limits_{n \rightarrow \infty}{\|k-t_n\|_{L^2(\mu\times\mu)}} = 0.
\end{align*}
Wir zeigen nun, dass für jede Treppenfunktion $T$ der Integraloperator mit Kern $T$ kompakt ist.
\begin{align*}
  T(s,t) := \sum_{i = 1}^n \alpha_i \1_{C_i}(s,t)
\end{align*}
Weil $\mu$ sigmaendlich ist, können wir für $i = 1,\dots,n$ nach dem Approximationssatz für messbare Mengen
$D_i$ aus $\mathfrak{R}(\mathfrak{S} \times \mathfrak{S})$ finden, welche $C_i$ beliebig genau
approximieren. Also gilt
\begin{align*}
    D_i = \bigcup_{k=1}^{l_i} A_{i_k} \times B_{i_k} \text{~(wobei~} A_{i_k}, B_{i_k} \in \mathfrak{S}), ~~\mu\times\mu(C_i \triangle D_i) < \frac{\epsilon}{n|\alpha_i|^2}.
\end{align*}

Damit gilt
\begin{align*}
    \|\1_{C_i} - \1_{D_i}\|_2^2 = \int_{X^2} |\1_{C_i}(s,t) - \1_{D_i}(s,t)|^2
     ~\mathrm{d}(\mu\times\mu)(s,t) = \int_{X^2} \1_{C_i \triangle D_i}~\mathrm{d}(\mu\times\mu)(s,t) =
     \mu\times\mu(C_i \triangle D_i) < \/
     \frac{\epsilon}{n|\alpha_i|^2}.
\end{align*}


Mit dieser Wahl ist
\begin{align*}
    \left\| \sum_{i = 1}^n \alpha_i \1_{C_i} -
    \sum_{i = 1}^n \alpha_i \1_{D_i}\right\|_2^2
    \leq \sum_{i=1}^{n} |\alpha_{i}|^2 \|\1_{C_{i}} - \1_{D_{i}} \|_2^2 < \epsilon.
\end{align*}

Weiters gilt
\begin{align*}
    k_{\epsilon}(s,t) := \sum_{i = 1}^n \alpha_i \1_{D_i}(s,t)
    = \sum_{i=1}^{n} \alpha_{i} \sum_{j=1}^{l_{i}}  \1_{A_{i_k} \times B_{i_k}}(s,t)
    = \sum_{i=1}^{n} \sum_{j=1}^{l_{i}}   \underbrace{\alpha_{i} \1_{A_{i_k}}(s)}_{\in L^2(\mu)}
     \underbrace{\1_{B_{i_k}}(t)}_{\in L^2(\mu)}.
\end{align*}

Sei $K_{\epsilon}$ der dazugehörige Integraloperator
\begin{align*}
  K_{\epsilon} f(s) = \int_X \left(\sum_{i=1}^{n} \sum_{j=1}^{l_{i}}   \alpha_{i} \1_{A_{i_k}}(s) \1_{B_{i_k}}(t)\right) f(t) \dmu(t).
\end{align*}

$K_{\epsilon}$ hat nach Aufgabe (a) endlichdimensionales Bild und ist daher nach Proposition 6.5.4 kompakt. Es gilt mit Cauchy-Schwarz-Bunjakowski
\begin{align*}
    \|(K_t-K_{\epsilon}) f\|_2^2 = \int_X \left| \int_X T(s,t) f(t) \dmu(t) - \int_X \left(\sum_{i=1}^{n} \sum_{j=1}^{l_{i}}   \alpha_{i} \1_{A_{i_k}}(s) \1_{B_{i_k}}(t)\right) f(t) \dmu(t) \right|^2 \dmu(s) \\
    \leq \int_X \left( \int_X \left|T(s,t) - \sum_{i = 1}^n \alpha_i \1D_i(s,t)\right| |f(t)| \dmu(t)\right)^2 \dmu(s) \\
    \leq \int_X \left(\int_X \left|T(s,t) - \sum_{i = 1}^n \alpha_i \1D_i(s,t)\right|^2 \dmu(t)\right) \left(\int_X |f(t)|^2 \dmu(t)\right) \dmu(s) \\
    = \|f\|_2^2\int_{X^2} \left| \sum_{i = 1}^n \alpha_i \1C_i(s,t) -
    \sum_{i = 1}^n \alpha_i \1D_i(s,t)\right|^2 \mathrm{d}(\mu\times\mu)(s,t) \leq \|f\|_2^2\epsilon
\end{align*}
die Konvergenz in der Operatornorm.
Jetzt können wir die selbe Rechnung noch einmal für den Integraloperator mit Kern $k$
und die kompakten Integraloperatoren mit Kern $(t_n)_{n \in \N}$ durchführen und
erhalten, dass $K$ als Grenzwert kompakter Operatoren wieder kompakt sein muss.

\end{enumerate}

\end{solution}
