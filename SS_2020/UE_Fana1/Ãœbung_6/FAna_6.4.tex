\begin{exercise}[IO/2]

Sei $X$ eine Menge und $\mu$ ein $\sigma$-endliches Maß auf $X$.
Zeige:

\begin{enumerate}[label = (\alph*)]

  \item
  Seien $a_i, b_i \in L^2(\mu), i = 1, \ldots, n$.
  Setze $k(s, t) := \sum_{i=1}^n a_i(s) b_i(t)$ und betrachte den Integraloperator $K \in \mathcal{B}(L^2(\mu))$ mit Kern $k$.
  Dann ist $\dim{\ran{K}} \leq n$.

  \item
  Sei $k \in L^2(\mu \times \mu)$.
  Dann ist der Integraloperator $K \in \mathcal{B}(L^2(\mu))$ mit Kern $k$ und Maß $\mu$ kompakt.

\end{enumerate}

\end{exercise}

\begin{solution}

\phantom{}

\begin{enumerate}[label = (\alph*)]

  \item
  Zuerst wollen wir uns ansehen, wie $K$ tatsächlich aussieht.

  \begin{align*}
    (K f)(s)
    & =
    \Int[X]
    {
      \sum_{i=1}^n
      a_i(s) b_i(t) f(t)
    }{\mu(t)}
    =
    \sum_{i=1}^n
    a_i(s) \Int[X]{b_i(t) f(t)}{\mu(t)}
    =
    \sum_{i=1}^n
    a_i(s) (b_i, f)
  \end{align*}

  Damit, sieht das Bild von $K$ wie folgt aus.

  \begin{align*}
    \ran{K}
    =
    \Bbraces
    {
      K f:
      f \in L^2(\mu)
    }
    =
    \Bbraces
    {
      \sum_{i=1}^n
      a_i (b_i, f):
      f \in L^2(\mu)
    }
    \subseteq
    \Span \Bbraces{a_1, \ldots, a_n}
    =: A_n
  \end{align*}

  Daher, muss $\dim \ran{K} \leq \dim{A_n} = n$.
  Tatsächlich, gilt sogar Gleichheit.

  \item
  ToDo!

\end{enumerate}

\end{solution}
