\begin{exercise}[20/1]

Sei $(a_n)_{n \in \N} \in \ell^1(\N)$, und betrachte den Operator auf $\ell^2(\N)$ der durch die Matrix

\begin{align*}
  A
  :=
  (a_{i+j-1})_{i,j = 1}^\infty
  =
  \begin{pmatrix}
    a_1    & a_2    & a_3    & \cdots \\
    a_2    & a_3    & \cdots &        \\
    a_3    & \cdots &        &        \\
    \cdots &        &        &
  \end{pmatrix}
\end{align*}

gegeben ist.
Explizite agiert $A$ also als $(x_n)_{n \in \N} \mapsto \pbraces{\sum_{k=1}^\infty a_{k+n-1} x_k}_{n \in \N}$.
Sei (der Einfachheit halber) weiters voausgesetzt, dass $(a_n)_{n \in \N}$ monoton fallend ist.
Zeige, dass $A$ kompakt ist. \\

\textit{Hinweis.}
Approximiere $A$ mit Operatoren die endlichdimensionales Bild haben.

\end{exercise}

\begin{solution}

Zuerst zeigen wir $A: \ell^2(\N) \rightarrow \ell^2(\N)$. Sei dazu $(x_n)_{n \in \N} \in \ell^2(\N)$
beliebig. Dann ist

\begin{align*}
  \sum_{n=1}^\infty |\sum_{k=1}^\infty a_{k+n-1}x_k|^2 &\leq
  \sum_{n=1}^\infty \Big( \sum_{k=1}^\infty |a_{k+n-1}| |x_k|\Big)^2 \leq
  \sum_{n=1}^\infty \Big( \sum_{k=1}^\infty |a_{k+n-1}|^2 \cdot \sum_{k=1}^\infty |x_k|^2\Big) = \\
  & \Vbraces{(x_n)_{n \in \N}}_{\ell^2(\N)}^2 \sum_{n=1}^\infty \sum_{k=1}^\infty |a_{k+n-1}|^2 \stackrel{(1)}{\leq}
  \Vbraces{(x_n)_{n \in \N}}_{\ell^2(\N)}^2 \sum_{n=1}^\infty \sum_{k=1}^\infty |a_n| |a_k| =
  \Vbraces{(x_n)_{n \in \N}}_{\ell^2(\N)}^2 \Vbraces{(a_n)_{n \in \N}}_{\ell^1(\N)}^2,
\end{align*}

wobei $(1)$ gilt, da die Folge $(a_n)_{n \in \N}$ monoton fallend ist.

Dem Hinweis folgend definieren wir für $j \in \N$ Operatoren $A_j$ folgendermaßen:

\begin{align*}
  A_j((x_n)_{n\in \N}) = (y_n)_{n \in \N} = \begin{cases}
  \sum_{k=1}^{\infty} a_{k+n-1}x_k & \text{für} \quad n \leq j, \\
  0 & \text{sonst}.
  \end{cases}
\end{align*}

Also gilt $A_j: \ell^2(\N) \rightarrow \{(y_n)_{n \in \N} \in \ell^2(\N): \forall n > j:
y_n = 0\}$; damit haben die Operatoren endlichdimensionales Bild. Nach Proposition
$6.5.4$ gilt zu zeigen, dass die $A_j$ linear und beschränkt sind.
\includegraphicsboxed{Prop. 6.5.4}

Die Linearität rechnen wir nach:

\begin{align*}
  A_j((x_n)_{n \in \N} + c (y_n)_{n \in \N}) =& \big(\sum_{k=1}^{\infty}a_{k+n-1}
  (x_k+c y_k) \big)_{n=1}^j = \big(\sum_{k=1}^{\infty}a_{k+n-1}
  x_k+ c \sum_{k=1}^{\infty}a_{k+n-1}y_k \big)_{n=1}^j,\\
  =&\big(\sum_{k=1}^{\infty}a_{k+n-1}
  x_k \big)_{n=1}^j + c\big(\sum_{k=1}^\infty a_{k+n-1}y_k \big)_{n=1}^j
  = A_j((x_n)_{n \in \N})+ c A_j((y_n)_{n \in \N}).
\end{align*}

Weisen wir also noch die Beschränktheit nach:

\begin{align*}
  \Vbraces{A_k(x_n)_{n\in\N}}_{\ell^2(\N)}^2 =&
  \sum_{n=1}^j |\sum_{k=1}^\infty a_{k+n-1}x_k|^2  \leq
  \sum_{n=1}^j \Big( \sum_{k=1}^\infty |a_{k+n-1}| |x_k|\Big)^2 \leq
  \sum_{n=1}^j \Big( \sum_{k=1}^\infty |a_{k+n-1}|^2 \cdot \sum_{k=1}^\infty |x_k|^2\Big) = \\
  =& \Vbraces{(x_n)_{n \in \N}}_{\ell^2(\N)}^2 \sum_{n=1}^j \sum_{k=1}^\infty |a_{k+n-1}|^2 =
  \Vbraces{(x_n)_{n \in \N}}_{\ell^2(\N)}^2 \sum_{n=1}^j \Vbraces{(a_{k+n-1})_{k \in \N}}_{\ell^2(\N)}^2
  < \infty.
\end{align*}

Dabei haben wir einmal die Ungleichung von Schwarz angewendet und die letze Ungleichheit gilt, da
$(a_n)_{n\in \N} \in \ell^1(\N) \subset \ell^2(\N)$. Weil die Folge monoton fallend ist, gilt
für festes $n \in \N$

\begin{align*}
  \Vbraces{(a_k)_{k\in \N}}_{\ell^2(\N)}^2 \geq \Vbraces{(a_{k+n-1})_{k \in \N}}_{\ell^2(\N)}^2.
\end{align*}

Jetzt brauchen wir noch die Konvergenz der $A_j$ gegen $A$ in der Operatornorm:

\begin{align*}
  \Vbraces{A(x_n)_{n \in \N}- A_j(x_n)_{n \in \N}}_{\ell^2(\N)}^2 &=
  \Vbraces{\Big(\sum_{k=1}^\infty a_{k+n-1} x_k\Big)_{n=j+1}^\infty}_{\ell^2(\N)}^2
  = \sum_{n=j+1}^\infty |\sum_{k=1}^\infty a_{k+n-1} x_k|^2 \\
  &\leq \Vbraces{(x_n)_{n \in \N}}_{\ell^2(\N)}^2 \sum_{n=j+1}^\infty \sum_{k=1}^\infty |a_{k+n-1}|^2
  \leq \Vbraces{(x_n)_{n \in \N}}_{\ell^2(\N)}^2 \sum_{n=j+1}^\infty \sum_{k=1}^\infty |a_n||a_k| \\
  &= \Vbraces{(x_n)_{n \in \N}}_{\ell^2(\N)}^2 \Vbraces{(a_k)_{k \in \N}}_{\ell^1(\N)} \sum_{n=j+1}^\infty |a_n|
  \stackrel{j \rightarrow \infty}{\longrightarrow} 0.
\end{align*}

Das gilt, da die $(|a_n|)_{n \in \N}$ eine Nullfolge bilden. Nach der Proposition oben
(Unterpunkt $(iii)$) ist die Menge der kompakten Operatoren bezüglich der Operatornorm in
$L_b(X,Y)$ abgeschlossen. Dabei ist $A$ linear und beschränkt wegen Korollar 4.2.3. (wir haben
ja gerade gezeigt, dass der Grenzwert existiert).
\includegraphicsboxed{Kor. 4.2.3}


\end{solution}
