\begin{exercise}

Ein TVR ohne stetige Funktionale: Sei $0 < p < 1$, und sei $L^p(0, 1)$ der Raum aller (Äquivalenzklassen von) Lebesgue-messbaren komplexwertigen Funktionen definiert auf $(0, 1)$ mit $\Int[(0,1)]{|f(x)|^p}{x} < \infty$.
Weiters sei

\begin{align*}
  d_p(f, g)
  :=
  \Int[(0, 1)]{|f(x) - g(x)|}{x},
  \enspace
  f, g \in L^p(0, 1).
\end{align*}

Zeige:

\begin{enumerate}[label = (\alph*)]
  \item $d_p$ ist eine Metrik auf $L^p(0, 1)$, und $L^p(0, 1)$ wird mit der von $d_p$ induzierten Topologie zu einem topologischen Vektorraum.
  \item Ist $V \subseteq L^p(0, 1)$ eine Umgebung von $0$ und ist $V$ konvex, so folgt $V = L^p(0, 1)$.
  \item $\dim{X} = \infty$ und $X^\prime = \Bbraces{0}$.
\end{enumerate}

Hinweis.
Sei $V$ konvexe Nullumgebung, $r > 0$, sodass $U_r(0) := \Bbraces{g \in L^p(0, 1): \Delta(g) < r} \subseteq V$ wobei $\Delta(f) := d_p(f, 0)$.
Sei $f \in L^p(0, 1)$.
Wähle $n \in \N$ mit $n^{p-1} \Delta(f) < r$, $0 = x_0 < x_1 < \ldots < x_n = 1$ mit $\Int[x_{i-1}][x_i]{|f(t)|^p}{t} = n^{-1} \Delta(f)$ und setze $g_i(t) := n f(t) \1_{[x_{i-1}, x_i]}$, sodass $f = n^{-1}(g_1 + \ldots + g_n)$.

\end{exercise}

\begin{solution}

(a)
Damit $d_p$ eine Metrik auf $L^p(0, 1)$ ist, müssen 3 Bedingungen gelten:

\begin{enumerate}[label = (\roman*)]

  \item
  \Quote{Null-Gleichheit}:
  $L^p(0, 1)$ besteht aus Äquivalenzklassen f.ü. gleicher Funktionen. $\Forall f, g \in L^p(0, 1):$

  \begin{align*}
    f = g
    \Leftrightarrow
    |f - g|^p = 0
    \Leftrightarrow
    d_p(f, g)
    =
    \Int[(0, 1)]{|f - g|^p}{\lambda} = 0
  \end{align*}

  \item
  \Quote{Symmetrie}:
  $\Forall f, g \in L^p(0, 1):$

  \begin{align*}
    d_p(f, g)
    =
    \Int[(0, 1)]{|f - g|^p}{\lambda}
    =
    \Int[(0, 1)]{|g - f|^p}{\lambda}
    =
    d_p(g, f)
  \end{align*}

  \item
  \Quote{Dreiecksungleichung}:
  Dazu brauchen wir zuerst, dass $\Forall a, b \in \R:$

  \begin{align*}
    |a|^p + |b|^p \geq |a + b|^p.
  \end{align*}

  Für $a + b = 0$, stimmt die Aussage.
  Ansonsten, genügt, aufgrund der Dreiecksungleichung für $|\cdot|$,

  \begin{align*}
    |a|^p + |b|^p \geq (|a| + |b|)^p
    \Leftarrow
    \pbraces{\frac{|a|}{|a| + |b|}}^p +
    \pbraces{\frac{|b|}{|a| + |b|}}^p
    \geq
    \frac{|a|}{|a| + |b|} +
    \frac{|b|}{|a| + |b|} = 1.
  \end{align*}

  Wegen der Monotonie des Integrals, folgt somit aber $\Forall f, g, h \in L^p(0, 1):$

  \begin{align*}
    d_p(f, g) + d_p(g, h)
    & =
    \Int[(0, 1)]{|f - g|^p}{\lambda} +
    \Int[(0, 1)]{|g - h|^p}{\lambda} \\
    & =
    \Int[(0, 1)]{|f - g|^p + |g - h|^p}{\lambda}
    \geq
    \Int[(0, 1)]{|f - h|^p}{\lambda}
    =
    d_p(f, h).
  \end{align*}

\end{enumerate}

Analog zu Beispiel 2.1.2, prüfen wir 3 Bedingungen nach.

\begin{itemize}

  \item
  \Quote{$+$ stetig}:
  In der Tat gilt wegen

  \begin{align*}
    d_p((f_1 + f_2), (g_1 + g_2))
    & =
    \Int[(0, 1)]{|(f_1 + f_2) - (g_1 + g_2)|^p}{\lambda} \\
    & \leq
    \Int[(0, 1)]{|f_1  - g_1|^p}{\lambda} +
    \Int[(0, 1)]{|f_2  - g_2|^p}{\lambda}
    =
    d_p(f_1, g_1) + d_p(f_2, g_2)
  \end{align*}

  $U_\epsilon^{L^p(0, 1)}(f_1) + U_\epsilon^{L^p(0, 1)}(f_2) \subseteq U_{2 \epsilon}^{L^p(0, 1)}(f_1 + f_2)$, womit die Addition stetig ist.

  \item
  \Quote{$\cdot$ stetig}:
  Aus

  \begin{align*}
    d_p(\alpha f, \beta g)
    & =
    \Int[(0, 1)]{|\alpha f - \beta g|^p}{\lambda}
    \leq
    \Int[(0, 1)]{|\alpha (f - g)|^p}{\lambda} +
    \Int[(0, 1)]{|(\alpha - \beta) g|^p}{\lambda} \\
    & =
    |\alpha|^p
    \Int[(0, 1)]{|f - g|^p}{\lambda} +
    |\alpha - \beta|^p
    \Int[(0, 1)]{|g|^p}{\lambda}
    =
    |\alpha|^p
    d_p(f, g) +
    |\alpha - \beta|^p
    \Delta(g)
  \end{align*}

  folgt $U_\epsilon^\C(\alpha) \cdot U_\epsilon^{L^p(0, 1)}(f) \subseteq U_{\epsilon (|\alpha| + \Delta(f) + \epsilon)}^{L^p(0, 1)}(\alpha f)$ und damit die Stetigkeit der Skalarmultiplikation.

  \item
  \Quote{$T_2$}:
  Schließlich bemerke man, dass jeder metrische Raum Hausdorff ist.

\end{itemize}

(b)
Dem Hinweis fügen wir noch Folgendes hinzu. $\Forall i = 1, \ldots, n:$

\begin{align*}
  \Delta(g_i)
  =
  \Int[(0, 1)]{|g_i|^p}{\lambda}
  =
  n^p
  \Int[(x_{i-1}, x_i)]{|f|^p}{\lambda}
  =
  n^{p-1} \Delta(f) < r
  \Rightarrow
  g_i \in U_r(0)
\end{align*}

Nachdem $U_r(0)$ konvex ist und $f = n^{-1}(g_1 + \cdots + g_n)$ eine Konvexkombination, muss $f \in U_r(0) \subseteq V$. \\

(c)
Alle Polynomfunktionen liegen in $L^p(0, 1)$ und bilden bereits einen unendlichdimensionalen linearen Teilraum. \\

Sei $D \subseteq \C$ offen und konvex und $f \in X^\prime$ linear und stetig. Dann ist $f^{-1}(D) \in L^p(0, 1)$ offen und konvex, weil $\Forall x, y \in f^{-1}(D), \Forall \alpha \in (0, 1):$

\begin{align*}
  f(x), f(y) \in D
  \Rightarrow
  f(\alpha x + (1 - \alpha) y)
  =
  \alpha f(x) + (1 - \alpha) f(y) \in D
  \Rightarrow
  \alpha x + (1 - \alpha) y \in f^{-1}(D).
\end{align*}

Wenn nun, für $\epsilon > 0$, $D = U_\epsilon(0)$, dann ist wegen der Linearität von $f$, $f^{-1}(D)$ eine offene konvexe $0$-Umgebung und somit ganz $L^p(0, 1)$. Damit liegt $f(L^p(0, 1))$ in jeder offenen $\epsilon$-Kugel um $0$, also in ihrem Schnitt $\Bbraces{0} \subseteq \C$, d.h. $f = 0$.

\end{solution}
