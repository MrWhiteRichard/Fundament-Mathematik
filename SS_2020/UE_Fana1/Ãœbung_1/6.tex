\begin{exercise}

Sei $X$ ein topologischer Vektorraum.
Eine Menge $B \subseteq X$ heißt beschränkt, falls es zu jeder Nullumgebung $U$ ein positive Zahl $\lambda_U$ gibt, sodass $B \subseteq \lambda_U U$.
Zeige, dass jede kompakte Teilmenge von $X$ beschränkt ist. Zeige, dass jeder lineare Teilraum $Y \neq \Bbraces{0}$ von $X$ unbeschränkt ist.

\end{exercise}

\begin{solution}

\Quote{kompakte Teilmenge beschränkt}:
Sei $K \subseteq X$ kompakt, $U \in \mathfrak{U}(0)$ beliebig.
Wähle $W \subseteq U$ als kreisförmige Nullumgebung.
$W$ ist absorbierend, d.h. $\Forall x \in X: \Exists t > 0: x \in tW$.
Nach Lemma 2.1.3, ist $M_\lambda: x \mapsto \lambda x$ homöomorph, für alle $ \lambda \in \C \setminus \Bbraces{0}$.
$(tW)_{t > 0}$ ist damit eine offene Überdeckung von $K$.

\begin{align*}
  \bigcup_{t > 0} tW = X \supseteq K
\end{align*}

Nach der Definition einer kompakten Menge existiert davon eine endliche Teilüberdeckung $(t_i W)_{i=1}^n$.
Aufgrund der Kreisförmigkeit von $W$ gilt $\Forall i = 1, \ldots, n:$

\begin{align*}
  t_j \geq t_i
  \Rightarrow
  t_j W \supseteq t_i W.
\end{align*}

Mit $t_\Text{max} := \max_{i=1}^n t_i$, folgt also $t_\Text{max} U \supseteq t_\Text{max} W \supseteq K$.
Damit ist $K$ beschränkt. \\

\Quote{Linearer Teilraum unbeschränkt}:
Weil $Y \neq \Bbraces{0}$, muss $\Exists y \in Y: y \neq 0$. Weil $(X, \mathcal{T})$ Hausdorff ist, $\Exists V \text{ Umgebung von } y, \Exists U \text{ Umgebung von } 0: V \cap U = \emptyset$.
Damit muss aber $y \notin U \in \mathfrak{U}(0)$.
Weil $Y$ ein linearer Teilraum ist, gilt $\Forall \lambda > 0: \lambda y \in Y$, aber $\lambda y \notin \lambda U$.
Daher ist $Y$ unbeschränkt, weil $\Exists U \in \mathfrak{U}(0): \Forall \lambda > 0: Y \not \subseteq \lambda U$.

\end{solution}
