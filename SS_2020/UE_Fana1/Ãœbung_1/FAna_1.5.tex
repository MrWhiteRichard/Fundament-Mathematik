\begin{exercise}

Sei $X$ ein Vektorraum mit $X \neq \Bbraces{0}$. Ist $X$ mit der diskreten Topologie ein topologischer Vektorraum?
Finde eine Topologie auf $X$ mit der $X$ ein topologischer Vektorraum wird und sodass $X^\prime = X^\ast$ ist.

\end{exercise}

\begin{solution}
$X$ ist mit der diskreten Topologie kein topologischer Vektorraum. \\
Dazu betrachten wir die Stetigkeit der Skalarmultiplikation bei der $0$.
Sei $x \in X\backslash\{0\}$ beliebig. \\
Dann ist $\{0\}$ eine Umgebung von $\cdot(0,x) = 0$.
Sei nun $V$ eine beliebige Umgebung von $(0,x)$ in der Produkttopologie von $\C \times X$.
Dann lässt sich $V = \bigcup_{i \in I}O_i^{\mathbb{C}}\times O_i^X$, mit
$O_i^{\mathbb{C}}$ offen in $\mathbb{C}$, versehen mit der euklidischen Topologie
und $O_i^X$ offen bezüglich der diskreten Topologie auf $X$ darstellen.
Dann gilt:
\begin{align*}
  \exists (a,x) \in V: a \neq 0 \implies \cdot(a,x) \neq 0 \implies \cdot(V) \nsubseteq \{0\}
\end{align*}
Also ist die Skalarmultiplikation bei 0 nicht stetig und $X$ somit kein topologischer
Vektorraum.
\\
Für den zweiten Teil der Aufgabe betrachte die Menge aller linearen Funktionale $(f_i)_{i \in I}$ auf X.
\begin{align*}
  f_i: X \rightarrow \mathbb{C}, i \in I
\end{align*}
Es gilt klarerweise $\bigcap_{i\in I}\ker f_i = \{0\}$.
Ebenso ist $\mathbb{C}$, versehen mit der euklidischen Topologie ein topologischer Vektorraum.
Damit sind alle Voraussetzungen für Proposition 2.4.1 erfüllt, die besagt, dass
damit $X$ mit der von $(f_i)_{i\in I}$ induzierten Initialtopologie zu einem topologischen Vektorraum wird.
Aus der Konstruktion der Topologie folgt sofort, dass $X^\prime = X^\ast$.
\end{solution}
