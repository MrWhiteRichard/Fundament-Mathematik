\begin{exercise}

Sei $X$ ein Vektorraum mit $X \neq \Bbraces{0}$.
Ist $X$ mit der diskreten Topologie ein topologischer Vektorraum?
Finde eine Topologie auf $X$ mit der $X$ ein topologischer Vektorraum wird und sodass $X^\prime = X^\ast$ ist.

\end{exercise}

\begin{solution}

$X$ ist mit der diskreten Topologie $\mathcal{T}$ kein topologischer Vektorraum.
Um das zu zeigen, widerlegen wir die Stetigkeit der Skalarmultiplikation bei $0$. \\


Sei dazu $x \in X \setminus \Bbraces{0}$.
Offenbar, gilt $\cdot((0, x)) = 0 \cdot x = 0$.
Weil $\mathcal{T} = \mathcal{P}(X)$, ist $\Bbraces{0} \in \mathfrak{U}(0)$.
Sei weiters $V \in \mathfrak{U}((0, x))$ im topologischen Raum $(\C \times X, \mathcal{E} \times \mathcal{T})$.
Dann $\Exists (O_i^\C)_{i \in I} \in \mathcal{E}^I, \Exists (O_i^X)_{i \in I} \in \mathcal{T}^I:$

\begin{align*}
  V = \bigcup_{i \in I} O_i^\C \times O_i^X.
\end{align*}

Es muss dabei aber $\Exists i \in I: O_i^\C \setminus \Bbraces{0} \neq \emptyset$.
Daher $\Exists z \in \C \setminus \Bbraces{0}: (z, x) \in V$.
Damit gilt aber auch $a \cdot x \neq 0$ und somit $\cdot(V) \nsubseteq \Bbraces{0}$.
Also, ist $V \not \subseteq \cdot^{-1}(\Bbraces{0}) \notin \mathfrak{U}((0, x))$, die Skalarmultiplikation bei $0$ nicht stetig, und $X$ schließlich kein topologischer Vektorraum. \\

Für den zweiten Teil der Aufgabe betrachte die Initialtopologie $\mathcal{T}_\Text{init}$, welche durch alle $f \in X^\ast$ induziert wird.
Es gilt klarerweise

\begin{align*}
  \bigcap_{f \in X^\ast} \ker f = \Bbraces{0}.
\end{align*}
<<<<<<< HEAD

Ebenso, ist $(\C, \mathcal{E})$ ein topologischer Vektorraum.
Laut Proposition 2.4.1, ist $(X, \mathcal{T}_\Text{init})$ also ein topologischer Vektorraum.
Aus der Konstruktion Topologie von $\mathcal{T}_\Text{init}$ folgt sofort, dass $X^\prime = X^\ast$.

=======
Es gilt klarerweise $\bigcap_{i\in I}\ker f_i = \{0\}$.
Ebenso ist $\mathbb{C}$, versehen mit der euklidischen Topologie ein topologischer Vektorraum.
Damit sind alle Voraussetzungen für Proposition 2.4.1 erfüllt, die besagt, dass
damit $X$ mit der von $(f_i)_{i\in I}$ induzierten Initialtopologie zu einem topologischen Vektorraum wird.
Aus der Konstruktion der Topologie folgt sofort, dass $X^\prime = X^\ast$.
>>>>>>> 1b7d0d0b6e8895e73b1d9ca16132d8822292577d
\end{solution}
