\begin{exercise}

Sei $X$ ein topologischer Vektorraum.
Zeige

\begin{align*}
  \Forall A \subseteq X, \enspace \text{kreisförmig}:
  \pbraces
  {
    A^\circ \enspace \text{kreisförmig}
    \Leftrightarrow
    (A^\circ = \emptyset \lor 0 \in A^\circ)
  }
\end{align*}

Finde ein Beispiel eines topologischen Vektorraumes $X$ und einer kreisförmigen Menge $A \subseteq X$, deren
Inneres nicht kreisförmig ist.

\end{exercise}

\begin{solution}
Hier könnte Ihre Werbung stehen!
\begin{enumerate}[label = (\roman*)]
  \item Wir nehmen ein kreisörmiges $A \subseteq X$.
  \begin{enumerate}
    \item[``$\Rightarrow$''] Sei $A^\circ \neq \emptyset$ kreisförmig. Das heißt es gibt ein $x \in A^\circ$ und wegen der Kreisförmigkeit ist auch $0x = 0 \in A^\circ$.
    \item[``$\Leftarrow$'']  Wir unterscheiden zwei Fälle.
    \begin{enumerate}[label = Fall \arabic*:]
      \item Sei $A^\circ = \emptyset$. Dann ist natürlich $A^\circ$ kreisförmig.
      \item Sei $0 \in A^\circ$. Wir wählen ein beliebiges $x \in A^\circ$ und ein $\lambda \in \C$ mit $\vbraces{\lambda} \leq 1$.
      \begin{enumerate}[label = Fall 2.\arabic*:]
        \item Sei $\lambda = 0$. Dann ist $0x = 0 \in A^\circ$.
        \item Sei $\lambda \neq 0$. Es gibt ein offenes $U \subseteq A$ mit $x \in U$. Da $A$ kreisförmig ist und $M_\lambda: X \to X : x \mapsto \lambda x$ ein Homöomorphismus ist $\lambda U \subseteq A$ und $\lambda x \in \lambda U$ sowie $\lambda U$ offen. Also ist $A$ eine Umgebung von $\lambda x$ und damit $\lambda x \in A^\circ$.
      \end{enumerate}
    \end{enumerate}
    Insgesamt ist also $A^\circ$ kreisförmig.
  \end{enumerate}
  \item Wir betrachten den Raum $\C^2$ mit der einzigen Topologie, welche diesen zu einem topologischen Vektorraum macht.
  \begin{align*}
    A := \Bbraces{
      \begin{pmatrix}
        v \\ w
      \end{pmatrix}
      \in \C^2 : \vbraces{v} \leq \vbraces{w} }
  \end{align*}
  Zuerst gilt es nachzuweisen, dass $A$ kreisförmig ist. Dazu wählen wir $\pbraces{v,w}^T \in A$ und $\lambda \in \C$ mit $\vbraces{\lambda} \leq 1$. Es gilt
  \begin{align*}
    \vbraces{v} \leq \vbraces{w} \Rightarrow \vbraces{\lambda} \vbraces{v} \leq \vbraces{\lambda} \vbraces{w} \Rightarrow \vbraces{\lambda v} \leq \vbraces{\lambda w}
  \end{align*}
  und damit $\lambda \pbraces{v, w}^T \in A$. Also ist $A$ kreisförmig.

  Nun behaupten wir, dass $A^\circ$ nicht kreisförmig ist. Das ist äquivalent dazu, dass $A^\circ \neq \emptyset \land (0,0)^T \notin A^\circ$.

  Es ist $(0,1)^T \in A^\circ$ also $A^\circ \neq \emptyset$.

  Sein nun $U \subseteq \C^2$ eine offene Menge mit $(0,0)^T \in U$. Nun gibt es $\epsilon \in \mathbb{R}^+$ so, dass mit $V:= \Bbraces{z \in \C : \vbraces{z} < \epsilon}$ die Inklusion $V \times V \subseteq U$ gilt. Der Punkt $\pbraces{\frac{\epsilon}{2}, \frac{\epsilon}{4} }^T \in V \times V$ liegt nicht in $A$ weshalb also $A$ keine Umgebung von $(0,0)^T$ ist und damit ist $(0,0)^T \notin A^\circ$.

\end{enumerate}

\end{solution}
