\begin{exercise}

Sei $X$ ein topologischer Vektorraum und $\mathfrak{W}$ eine Basis des Umgebungsfilters der Null in $X$.
Zeige

\begin{align*}
  \Forall A \subseteq X:
  \overline{A}
  =
  \bigcap_{W \in \mathfrak{W}} (A + W)
\end{align*}

\end{exercise}

\begin{solution}

\phantom{}

\begin{itemize}

  \item[\Quote{$\subseteq$}:]
  Sei $x \in \overline{A}$ beliebig, also $\Forall U$ Umgebung von $x: A \cap U \neq \emptyset$.
  Da $\mathfrak{W}$ eine Umgebungbasis von $0$ ist, ist $x + \mathfrak{W}$ eine Umgebungsbasis von $x$.
  Sei $W \in \mathfrak{W}$ beliebig und wähle $W_0 \subset W$ kreisförmige Umgebung der $0$.
  $W_0$ ist somit insbesondere symmetrisch, $x + W_0$ eine Umgebung von $x$ und es gilt daher

  \begin{align*}
    \emptyset
    \neq
    (x + W_0) \cap A
    =
    (x - W_0) \cap A.
  \end{align*}

  Damit, $\Exists w \in W_0 \subseteq W, \Exists a \in A: x - w = a$, also $x = a + w$ und somit $x \in \bigcap_{W \in \mathfrak{W}}(A + W)$.

  \item[\Quote{$\supseteq$}:]
  Umgekehrt betrachte $y \in \bigcap_{W \in \mathfrak{W}}(A + W)$, sowie $ U \in \mathfrak{U}(0)$.
  Dann $\Exists W \in \mathfrak{W}: W \subseteq U$ und $\Exists W_0 \in \mathfrak{U}(0) \enspace \text{kreisförmig}: W_0 \subseteq W$.
  Weil $\Exists W_1 \in \mathfrak{W}: W_1 \subseteq W_0$, muss $y \in (A + W_1) \subseteq (A + W_0)$.
  $W_0$ ist insbesondere symmetrisch, und es gilt

  \begin{align*}
    \emptyset
    \neq
    \Bbraces{y} \cap (A + W_0)
    \stackrel{!}{=}
    (y - W_0) \cap A = (y + W_0) \cap A
    \subseteq
    (y + W) \cap A
    \subseteq
    (y + U) \cap A.
  \end{align*}

  (Für \Quote{!}, betrachte die Überlegung am Ende von \Quote{$\subseteq$}.)
  Also haben wir $y \in \overline{A}$, da $A$ mit jeder Umgebung aus $y + \mathfrak{U}(0) = \mathfrak{U}(y)$ nichtleeren Schnitt hat.

\end{itemize}

\end{solution}
