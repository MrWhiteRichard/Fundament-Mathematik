\begin{exercise}[24/1$^\ast$]

Sei $S$ der Shift-Operator am $\ell^2(\N)$, und sei $M \in \mathcal{B}(\ell^2(\N))$ der Operator definiert als $(\alpha_n)_{n \in \N} \mapsto (\frac{1}{n} \cdot \alpha)_{n \in \N}$.
Betrachte $T := M S$.

\begin{enumerate}[label = (\alph*)]

  \item
  Für $k \in \N$ bestimme $\norm{T^k}$ und berechne $\lim_{n \to \infty} \norm{T^k}^\frac{1}{k}$.

  \item
  Zeige dass $T$ kompakt ist.

  \item
  Zeige dass $\sigma_p(T) = \emptyset$ und $\sigma(T) = \Bbraces{0}$.

\end{enumerate}

\end{exercise}

\begin{solution}

\phantom{}

\begin{enumerate}[label = (\alph*)]

  \item
  Zuerst wollen wir uns ansehen, wie $T$ tatsächlich aussieht.

  \begin{align*}
    T \vec x
    & =
    \pbraces
    {
      0,
      \frac{1!}{2!} x_1,
      \frac{2!}{3!} x_2,
      \frac{3!}{4!} x_3,
      \ldots
    }, \\
    T^2 \vec x
    & =
    \pbraces
    {
      0, 0,
      \frac{1!}{3!} x_1,
      \frac{2!}{4!} x_2,
      \frac{3!}{5!} x_3,
      \ldots
    }, \\
    T^3 \vec x
    & =
    \pbraces
    {
      0, 0, 0,
      \frac{1!}{4!} x_1,
      \frac{2!}{5!} x_2,
      \frac{3!}{6!} x_3,
      \ldots
    }, \\
    \vdots \\
    T^k \vec x
    & =
    \bigg (
      \underbrace{0, \ldots, 0}_{k \text{-mal}},
      \frac{1!}{(k+1)!} x_1,
      \frac{2!}{(k+2)!} x_2,
      \frac{3!}{(k+3)!} x_3,
      \ldots,
      \frac{n!}{(k+n)!} x_n,
      \ldots
    \bigg )
  \end{align*}

  Wir wollen nun eine geeignete Abschätzung für $\norm{T^k}$ finden.
  $\Forall \vec x \in \ell^2(\N):$

  \begin{align*}
    \norm[2]{T^k \vec x}^2
    & =
    \sum_{n=1}^\infty
    \pbraces
    {
      \frac{n!}{(k+n)!}
      x_n
    }^2
    \leq
    \sum_{n=1}^\infty
    \pbraces
    {
      \frac{1}{(k+1)!}
      x_n
    }^2
    =
    \pbraces{\frac{1}{(k+1)!}}^2
    \sum_{n=1}^\infty x_n^2
    =
    \pbraces{\frac{1}{(k+1)!}}^2
    \norm[2]{\vec x}^2 \\
    \implies
    \norm{T^k}
    & \leq
    \frac{1}{(k+1)!}
  \end{align*}

  Das $\sup$ wird tatsächlich auch angenommen.

  \begin{align*}
    \norm[2]{(\delta_{(k+1) n})_{n \in \N}} = 1,
    \quad
    \norm[2]{T (\delta_{(k+1) n})_{n \in \N}}
    =
    \frac{1}{(k+1)!}
    \implies
    \norm{T^k}
    =
    \frac{1}{(k+1)!}
  \end{align*}

  Um den Grenzwert zu berechnen, benützen wir einen alt-bekannten Trick.

  \begin{align*}
    \norm{T^k}^{1/k}
    =
    \pbraces{\frac{1}{(k+1)!}}^{1/k}
    =
    \exp \ln \frac{1}{(k+1)!}^{1/k}
    =
    \exp \pbraces
    {
      -\frac{1}{k}
      \sum_{i=1}^{k+1}
      \ln{i}
    }
  \end{align*}

  \includegraphicsboxed{Blümlinger - Cesaro-Mittel}

  Wir schätzen den Ausdruck im $\exp$ separat ab.

  \begin{align*}
    \frac{1}{k}
    \sum_{i=1}^{k+1}
    \ln{i}
    \geq
    \frac{1}{k}
    \sum_{i=1}^k
    \ln{i}
    \xrightarrow{k \to \infty}
    \lim_{i \to \infty}
    \ln{i}
    =
    \infty
  \end{align*}

  Weil $\exp$ stetig ist, können wir den $\lim$ hineinziehen.

  \begin{align*}
    \implies
    \lim_{k \to \infty}
    \norm{T^k}^{1/k}
    =
    0
  \end{align*}

  \item
  Nachdem $T = M S$, können wir auch, laut Proposition 6.5.4 (iv), zeigen, dass $M$ oder $S$ kompakt ist.

  \includegraphicsboxed{Proposition 6.5.4}

  $S$ ist auf jeden Fall mal NICHT kompakt!
  Sonst wäre, laut Proposition 6.5.4 (iv) $S^\ast S = I$ kompakt!
  Man erinnere sich, dass, laut Aufgabe ???, $S^\ast$ der umgekehrte Shift-Operator ist. \\

  Wir hoffen also, dass $M$ kompakt ist.
  Das zeigen wir so, wie in Aufabe ???.
  Wir approximieren $M$ mit kompakten Operatoren.

  \begin{align*}
    M_j:
    \vec \alpha
    \mapsto
    \pbraces
    {
      \frac{\alpha_1}{1},
      \ldots,
      \frac{\alpha_j}{j},
      0, 0, 0, \ldots
    }
  \end{align*}

  Diese Operatoren haben endlichdimensionales Bild, sind also, laut Proposition 6.5.4 (i), kompakt.
  Wir benötigen also nur noch die konvergenz von $(M_j)_{j \in \N}$ gegen $M$ in der Operator-Norm.

  \begin{align*}
    \norm[2]{(M - M_j) \vec \alpha}^2
    & =
    \sum_{n = j+1}^\infty
    \pbraces
    {
      \frac{\alpha_n}{n}^2
    }
    \leq
    \norm[2]{\vec \alpha}
    \sum_{n = j+1}^\infty
    \frac{1}{n^2}, \\
    \implies
    \norm{M - M_j}
    & \leq
    \sum_{n = j+1}^\infty
    \xrightarrow{j \to \infty} 0
  \end{align*}

  Also, ist $M$, laut Proposition 6.5.4 (iii) kompakt.
  Damit, ist auch $T$ kompakt.

  \item
  \phantom{}

  \includegraphicsboxed{Satz 6.4.14}

  Laut Satz 6.4.14, ist Also

  \begin{align*}
    0
    =
    \lim_{k \to \infty} \norm{T^k}
    =
    r(T)
    =
    \max_{\lambda \in \sigma(T)} |\lambda|.
  \end{align*}

  Weil $T$ nur links-invertierbar ist, ist $0 \in \sigma(T)$.
  Insgesamt, erhalten wir also $\sigma(T) = \Bbraces{0}$. \\

  Aus der links-Invertierbarkeit von $T$, folgt Injektivität, und damit $\ker{(T - 0 I)} = \ker{T} = \Bbraces{0}$.
  Also, ist $0 \notin \sigma_p(T)$.

\end{enumerate}

\end{solution}
