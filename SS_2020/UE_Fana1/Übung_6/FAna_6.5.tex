\begin{exercise}[IO/3]

Der Volterra-Operator ist der Integraloperator $(V f)(x) := \Int[0][x]{f(t)}{t}$.
Zeige, dass $V \in \mathcal{B}(L^2(0, 1))$ mit $\norm{V} = \frac{2}{\pi}$, dass $V$ kompakt ist, und dass

\begin{align*}
  (V^\ast f)(x)
  =
  \Int[x][1]{f(t)}{t}.
\end{align*}

Zeige, dass $\sigma(V) = \sigma_c(V) = \Bbraces{0}$.

\end{exercise}

\begin{solution}

Wir bringen den Operator zunächst auf Standard-Form.
Dazu, betrachten wir das $\sigma$-endliche Lebesgues-Maß
$\nu := \lambda$, die Menge $X := (0, 1)$ und die Indikator-Funktion $k: (x, t) \mapsto \1_{(0, x)}(t)$.

\begin{align*}
  \implies
  (V f)(x)
  =
  \Int[0][x]{f(t)}{t}
  =
  \Int[0][1]{\1_{(0, x)}(t) f(t)}{t}
  =
  \Int[X]{k(x, t) f(t)}{\lambda(t)}
\end{align*}

Es gilt übrigens tatsächlich $k \in L^2$.

\begin{align*}
  \int_{(0,1) \times (0,1)} \vbraces{\1_{(0,x)}(t)}^2 d\lambda(t) \times \lambda(x)
  =
  \int_0^1 \int_0^1 \vbraces{\1_{(0,x)}(t)} d\lambda(t) d\lambda(x)
  =
  \int_0^1 \int_0^x 1 d\lambda(t) d\lambda(x)
  =
  \int_0^1 x d\lambda(x)
  =
  \frac{1}{2}
\end{align*}

\underline{Beschränktheit:} \\

Laut Aufgabe 3 (IO/1). \\

\underline{Hilbertraumadjungierte:} \\

Laut Aufgabe 3 (IO/1), können wir die Hilbertraumadjungierte sofort hinschreiben als Integraloperator mit Kern

\begin{align*}
  k^\ast(x, t)
  =
  \overline{k(t, x)}
  =
  \1_{(0, t)}(x).
\end{align*}

Zunächst noch eine kleine Bemerkung zu unserer Indikator-Funktion.

\begin{align*}
  \1_{(0, t)}(x) = 1
  \iff
  x \in (0, t)
  \iff
  0 < x < t < 1
  \iff
  t \in (x, 1)
  \iff
  \1_{(x, 1)}(t) = 1
\end{align*}

Jetzt kann uns nichts mehr aufhalten.

\begin{align*}
  (V^\ast f)(x)
  =
  \Int[0][1]{k^\ast(x, t) f(t)}{t}
  =
  \Int[0][1]{\1_{(0, t)}(x) f(t)}{t}
  =
  \Int[0][1]{\1_{(x, 1)}(t) f(t)}{t}
  =
  \Int[x][1]{f(t)}{t}
\end{align*}

\underline{Kompaktheit:} \\

Offensichtlich, ist $k \in L^2(\lambda \times \lambda)$.
Laut Aufgabe 4 (IO/2), ist $V$ also kompakt. \\

\underline{Spektrum:} \\

Da wir es mit einem kompakten Operator zu tun haben, sind nach Satz 6.5.12 besteht
das Spektrum (ausgenommen der $0$) nur aus Eigenwerten.
\includegraphicsboxed{Satz 6.5.12.png}

Nehmen wir also an, es gibt ein Eigenpaar $\lambda \neq 0, f \neq 0$, also

\begin{align*}
  Vf(x) - \lambda f(x) = 0
  \Leftrightarrow
  \Int[0][x]{f(t)}{t} = \lambda f(x)
\end{align*}

Aus der Maßtheorie wissen wir, dass $V f$ stetig ist, somit auch $f$.
Laut dem Hauptsatz der Differential- und Integral-Rechnung, ist $f$ aber damit auch schon stetig differenzierbar.
Wir leiten auf beiden Seiten ab

\begin{align*}
  \implies
  f^\prime (x)
  =
  \frac{f(x)}{\lambda}
\end{align*}

Die Lösung dieser Differentialgleichung ist

\begin{align*}
  f(x)
  =
  c \exp{\frac{x}{\lambda}},
  \quad
  c \in \R.
\end{align*}

Um die Konstante $c$ zu bestimmen sehen wir uns zuerst

\begin{align*}
  0 = Vf(0) = \lambda f(0)
\end{align*}

Damit, und $\lambda \neq 0$, erhalten wir

\begin{align*}
  f(0) = 0
  \Rightarrow
  c = 0
  \Rightarrow
  f \equiv 0
\end{align*}

im Widerspruch zu unserer Annahme. \\

Wir zeigen jetzt, dass $0 \in \sigma_c(V)$.
Laut Satz 6.4.14, ist das Spektrum nichtleer.
Weil $\lambda \neq 0$ anscheinend nicht im Spektrum enthalten ist, muss $\sigma(V) = \Bbraces{0}$. \\

Der Ableitungsoperator ist eine Linksinverse von $V$.
Also ist $V$ injektiv und $\ker{V} = \Bbraces{0}$.
Also ist $0 \notin \sigma_p(V)$. \\

\includegraphicsboxed{Blümlinger - Satz 2.5.1.jpg}

Aus Analysis 3 wissen wir, dass $C_c^{\infty}$ dicht in $L^2$ ist.
Sei nun $f: (0,1) \rightarrow \C$ für die auch $ f \in C^{\infty}$ gilt. Dann ist

\begin{align*}
  f^\prime(x) \in L^2(0,1)
\end{align*}

da die Funktion ja auch in $C_c^{\infty}$ ist. Dann gilt also (nach HS der
Differential und Integralrechnung)

\begin{align*}
  Vf^\prime(x) = f(x)
  \Rightarrow
  C_c^{\infty} \subseteq \ran(V)
\end{align*}

Da diese schon dicht in $L^2(0,1)$ sind also auch insbesondere $\ran(V)$.
Das heißt ja genau, dass $0$ im stetigen Spektrum liegt. \\

\underline{Norm:} \\

Zuguterletzt widmen wir uns noch der Norm von $V$. Dazu wollen wir Proposition
6.6.2 (i) verwenden.

\includegraphicsboxed{Proposition 6.6.2.jpg}

Also $\Vbraces{V} = \sqrt{\Vbraces{V^*V}}$.
Mit den Rechenregeln für die Adjungierte überzeugt man sich leicht, dass $S:=V^*V$ ein selbstadjungierter, also normaler Operator ist.

\includegraphicsboxed{Proposition 6.6.5}

Laut Proposition 6.6.5, gilt $r(S) = \Vbraces{S}$.
$S$ ist, wegen der Kompaktheit von $V$, und weil $K(L^2(0, 1))$ ein Ideal ist (vgl. Proposition 6.5.4), auch kompakt.

\begin{align*}
  (Sf)(x)
  =
  \Int[x][1]{
    \Int[0][y]{f(t)}{t}
  }{y}
\end{align*}

Sehen wir uns wieder Eigenpaare $\lambda \neq 0, f \neq 0$ von $S$ an. D.h.

\begin{align*}
  Sf(x) = \lambda f(x)
\end{align*}

Das Punkt-Spektrum von $S$ zu betrachten reicht ja, weil $S$ kompakt ist.
Durch zweimaliges differenzieren von $S f$ erhalten wir

\begin{align*}
  \lambda f^\primeprime(x) = -f(x)
\end{align*}

Mit den Nullstellen des charakteristischen Polynoms dieser Differentialgleichung
erhalten wir eine allgemeine Lösung mit $a,b \in \C, \omega = \sqrt{\frac{1}{\lambda}}$

\begin{align*}
  f(x) = a e^{i\omega x} + b e^{-i \omega x}
\end{align*}

Um die Koeffizienten zu bestimmen wenden wir den Operator an

\begin{align*}
  Sf(x)
  & =
  a \Int[x][1]{
    \Int[0][y]{e^{i\omega t}}{t}
  }{y}
  +
  b \Int[x][1]{
      \Int[0][y]{e^{-i\omega t}}{t}
  }{y} \\
  & =
  \frac{1}{\omega^2} f(x)
  +
  \frac{1}{i\omega} (a-b)x
  -
  \frac{1}{\omega^2}(ae^{i\omega} + be^{-i\omega})
  -
  \frac{1}{i\omega}(a-b)
  \stackrel{!}{=}
  \lambda f(x)
\end{align*}

Aus dem zweiten Term erkennen wir, dass $a=b$.
Also fällt auch der vierte Term weg.
Also muss auch der dritte Term wegfallen.

\begin{align*}
  \implies
  0
  =
  \frac{1}{\omega^2}(ae^{i\omega} + be^{-i\omega})
  =
  \frac{a}{\omega^2}
  (e^{i\omega} + e^{-i\omega})
  =
  \frac{a}{\omega^2}
  2 \cos{\omega}
\end{align*}

Damit erhalten wir unsere Eigenwerte

\begin{align*}
  \implies
  \sqrt{\frac{1}{\lambda_n}}
  =
  \omega_n
  =
  \frac{2n+1}{2} \pi,
  \quad n \in \Z
  \implies
  \lambda_n
  =
  \frac{4}{(2n+1)^2 \pi^2}
\end{align*}

Um nun $r(S)$ zu erhalten nehmen wir den Betragsgrößten, also $\lambda_0$.
Damit gilt $\Vbraces{S} = \frac{4}{\pi^2}$, womit wir nun schließlich $\Vbraces{V} = \frac{2}{\pi}$ erhalten.

\end{solution}
