\begin{exercise}[19/1]

Betrachte den Shift-Operator $S$ am $\ell^2(\N)$, das ist

\begin{align*}
  S:
  \begin{cases}
    \ell^2(\N)              & \to     \ell^2(\N) \\
    (x_1, x_2, x_3, \ldots) & \mapsto (0, x_1, x_2, \ldots)
  \end{cases}
\end{align*}

\begin{enumerate}[label = (\alph*)]

  \item
  Zeige dass $S$ isometrisch ist, bestimme $\ran{S}$ und zeige dass $\ran{S}$ abgeschlossen ist, und zeige $\bigcap_{n=1}^\infty \ran{S^n} = \Bbraces{0}$.

  \item
  Bestimme die Hilbertraumadjungierte $S^\ast$ von $S$, und bestimme $\ker{(S^\ast)}$, $\ran{(S^\ast)}$, und $\bigcap_{n=1}^\infty \ran{[S^\ast]^n}$.

\end{enumerate}

\end{exercise}

\begin{solution}
  \leavevmode \\
  \begin{itemize}
      \item $S$ ist isometrisch: Klar.
      \item $\ran{S}$ abgeschlossen:
      \begin{align*}
        \ran(S) = \{(x_n)_{n \in \N} \in \ell^2(\N): x_0 = 0\}
      \end{align*}
      Sei $(\vv{x_n})_{n=0}^\infty =
      (S(\vv{y_n}))_{n=0}^\infty$ eine gegen $\vv{x}$ konvergente Folge aus $\ran{S}$. Weil $S$ isometrisch und linear ist, gilt
      $\| S(\vv{y_n}) - S(\vv{y_m})\|
      = \| S(\vv{y_n} - \vv{y_m}) \|
      = \| \vv{y_n} - \vv{y_m} \|$, also ist auch $(\vv{y_n})_{n=0}^\infty$ eine Cauchyfolge in $\ell^2$ und damit konvergent gegen ein $\vv{y} \in \ell^2.$ Nun gilt
      \begin{align*}
       \| S(\vv{y}) - \vv{x} \| \leq
       \| S(\vv{y}) - S(\vv{y_n}) \| + \| S(\vv{y_n}) - \vv{x} \|
       \stackrel{n \rightarrow \infty}{\longrightarrow} 0,
      \end{align*}
      also $\vv{x} = S(\vv{y}) \in \ran{S}$.
      \item $\bigcap_{n=0}^\infty \ran{S^n} = \{0\}:$ Für ein $\vv{x} \in \ran{S^n}$ gilt $\vv{x}_n = 0.$ Liegt $\vv{x}$ im Schnitt aller $\ran{S^n},$ ist also jede Komponente gleich Null.
      \item Wir bestimmen nun die Hilbertraumadjungierte $S^*$ von $S.$

      $\ell^2$ ist isomorph zu seinem eigenen Dualraum vermöge folgender Abbildung $\Phi$:
      \begin{align*}
            \Phi: \begin{Bmatrix}
       \ell^2 & \rightarrow & (\ell^2)^\prime \\
      (x_n)_{n=0}^\infty & \mapsto & ((y_n)_{n=0}^\infty \mapsto \sum_{n=0}^\infty y_n \overline{x_n})
        \end{Bmatrix}.
        \end{align*}

      Weiters sei an folgende Definitionen erinnert:
      \begin{align*}
        S^{\prime}: \begin{Bmatrix}
       (\ell^2)^{\prime} &\rightarrow & (\ell^2)^{\prime} \\
       y^{\prime} &\mapsto &x \mapsto y^{\prime}(S(x)),
        \end{Bmatrix}
      \end{align*}

      \begin{align*}
        S^*: \begin{Bmatrix}
       \ell^2 & \rightarrow & \ell^2 \\
       (x_n)_{n=0}^\infty & \mapsto & \Phi^{-1}(S^\prime(\Phi((x_n)_{n=0}^\infty)))).
        \end{Bmatrix}
      \end{align*}

       Durch Anwendung dieser Definitionen erhalten wir
       \begin{align*}
           S^*((x_n)_{n=0}^\infty) &= \Phi^{-1}(S^\prime(\Phi((x_n)_{n=0}^\infty))) =
           \Phi^{-1}\left(S^\prime\left( (y_n)_{n=0}^\infty \mapsto \sum_{n=0}^\infty y_n \overline{x_n}\right)\right) \\
           &= \Phi^{-1}\left((z_n)_{n=0}^\infty \mapsto \sum_{n=0}^\infty z_n \overline{x_{n+1}}\right) = (x_{n+1})_{n=0}^\infty.
       \end{align*}

       Die Hilbertraumadjungierte $S^*$ verschiebt eine Folge also um einen Eintrag nach links.

      \item $\ker{S^*} = \{(c, 0, 0, 0, ...)~|~c \in \mathbb{C}\}.$
      %\item $\ran{S^*} = \ell^2,$ denn es gilt $\sum_{n=0}^\infty |x_n|^2 < \infty \Longleftrightarrow \sum_{n=1}^\infty |x_n|^2 < \infty$. \textbf{Das folgt eh aus dem nächsten}
      \item $\bigcap_{n=0}^\infty \ran{(S^*)^n} = \ell^2$, weil für alle $n \in \mathbb{N}$ gilt: $\ran{(S^*)^n} = \ell^2.$

      Sei dazu $(x_k)_{k=0}^\infty \in \ell^2$ beliebig. Wir konstruieren eine weitere Folge $(y_k)_{k=0}^\infty \in \ell^2,$ die die erste als Bild unter $(S^*)^n$ hat, wie folgt:
      \begin{align*}
       y_k = \left\{\begin{array}{ll}
       0, \text{~~~falls~} k < n, \\
       x_{k-n}, \text{~~~falls~} k \geq n.
       \end{array}\right\}
      \end{align*}
  \end{itemize}
\end{solution}
