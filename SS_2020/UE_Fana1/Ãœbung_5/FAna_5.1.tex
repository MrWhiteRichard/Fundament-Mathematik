\begin{exercise}[18/1]

Betrachte den Operator $T$ der für $f \in L^1(0, 1)$ definiert ist als

\begin{align*}
  T f
  :=
  \pbraces{\Int[0][1]{f(t) t^n}{t}}_{n \in \N_0}.
\end{align*}

Die Zahlen $\Int[0][1]{f(t) t^n}{t}$ heißen auch die Momente von $f$.
Zeige, dass $T$ zu $\mathcal{B}(L^1(0, 1), C_0(\N_0))$ gehört, und bestimme $T^\prime \in \mathcal{B}(\ell^1(\N_0), L^\infty(0, 1))$.

\end{exercise}

\begin{solution}

Seien $f \in L^1(0, 1)$ und $ n \in \N$.

\begin{align*}
  \Int[0][1]{f(t)t^n}{t}
  \leq
  \Int[0][1]{|f(t) t^n|}{t}
  =
  \Int[0][1]{|f(t)| \cdot |t^n|}{t}
  \leq
  \Int[0][1]{|f(t)|}{t} < \infty
\end{align*}

Damit existiert $Tf$.
Weil $\int$ linear ist, ist es auch $T$.

\begin{align*}
  \Forall t \in [0, 1]:
  f(t) t^n
  \xrightarrow{n \to \infty} 0
\end{align*}

Also folgt mittels majorisierter Konvergenz, mit $1$ als Majorante, dass

\begin{align*}
  \lim_{n \to \infty}
  (T f)_n
  =
  \lim_{n \to \infty}
  \Int[0][1]{f(t) t^n}{t}
  =
  \Int[0][1]{\lim_{n \to \infty} f(t) t^n}{t}
  =
  \Int[0][1]{0}{t} = 0.
\end{align*}

$T$ bildet also wirklich nach $C_0(\N_0)$ ab.
$C_0(\N_0)$ ist mit der Supremumsnorm $\norm[\infty]{\cdot}$ versehen.

\begin{align*}
  \norm[\infty]{T f}
  =
  \norm[\infty]
  {\pbraces{\Int[0][1]{f(t) t^n}{t}}_{n \in \N_0}}
  =
  \vbraces{\Int[0][1]{f(t)}{t}}
  \leq
  \norm[1]{f}
  \implies
  \norm{T} \leq 1
\end{align*}

Also ist $T$ ein beschränkter linearer Operator.
Der Operator $T^\prime$ wird charakteristiert durch

\begin{align*}
  \Forall x \in X, y^\prime \in Y^\prime:
  y^\prime(T x)
  =
  \abraces{T x, y^\prime}
  =
  \abraces{x, T^\prime y^\prime}
  =
  T^\prime y^\prime(x) \\
  \quad
  T^\prime:
  C_0(\N_0)^\prime = L^1(\N_0)
  \to
  L^1(0, 1)^\prime = L^\infty(0, 1).
\end{align*}

Folgende Abbildungen sind isometrische Isomorphismen.

\begin{align*}
  \varphi_1:
  &
  \begin{Bmatrix}
    L^1(\N_0)        & \to     & C_0(\N_0)^\prime \\
    (a_n)_{n \in \N} & \mapsto &
    ((x_n)_{n \in \N} \mapsto \sum_{n=0}^\infty x_n a_n)
  \end{Bmatrix}, \\
  \varphi_2:
  &
  \begin{Bmatrix}
    L^\infty(0, 1) & \to    & L^1(0, 1)^\prime \\
    g              &\mapsto & (f \mapsto \Int[0][1]{fg}{t})
  \end{Bmatrix}.
\end{align*}

$T^\prime$ aufgefasst als Abbildung zwischen (noch nicht isomorph identifizierten Dualräumen) leistet also

\begin{align*}
  T^\prime \pbraces
  {
    \varphi_1((a_n)_{n \in \N})
  }(f)
  & =
  \varphi_1((a_n)_{n \in \N})(T f)
  =
  \pbraces{(x_n)_{n \in \N} \mapsto \sum_{n=0}^\infty x_n a_n}
  \pbraces{\Int[0][1]{f(t) t^n}{t}}_{n \in \N_0} \\
  & =
  \sum_{n=0}^\infty a_n
  \Int[0][1]{f(t) t^n}{t}
  =
  \lim_{n \to \infty} \sum_{k= 0}^n a_k
  \Int[0][1]{f(t) t^k}{t}
  =
  \lim_{n \to \infty}
  \Int[0][1]{\sum_{k=0}^n a_k t^k f(t)}{t} = \cdots.
\end{align*}

Wir dominieren den Integranden.

\begin{align*}
  \Int[0][1]{\sum_{k=0}^n a_k t^k f(t)}{t}
  \leq
  \Int[0][1]{\sum_{n=0}^\infty |a_n| \cdot t^n \cdot |f(t)|}{t}
  \leq
  \Int[0][1]{\norm[L^1(\N_0)]{a} |f(t)|}{t}
  \leq
  \norm[L^1(\N_0)]{a} \norm[L^1(0, 1)]{f}
\end{align*}

Also, können wir den $\lim$ ins $\int$ ziegen.

\begin{align*}
  \implies
  \cdots
  =
  \Int[0][1]{\sum_{n=0}^\infty a_n t^n f(t)}{t}
\end{align*}

Es gilt schließlich nach kanonischer Identifikation

\begin{align*}
  T^\prime((a_n)_{n \in \N})
  =
  \sum_{n=0}^\infty a_n t^n
  \leq
  \sum_{n=0}^\infty |a_n|
  \in
  L^\infty(0, 1).
\end{align*}

\end{solution}
