\begin{exercise}[18/1]

Betrachte den Operator $T$ der für $f \in L^1(0, 1)$ definiert ist als

\begin{align*}
  T f
  :=
  \pbraces{\Int[0][1]{f(t) t^n}{t}}_{n \in \N_0}.
\end{align*}

Die Zahlen $\Int[0][1]{f(t) t^n}{t}$ heißen auch die Momente von $f$.
Zeige, dass $T$ zu $\mathcal{B}(L^1(0, 1), C^0(\N_0))$ gehört, und bestimme $T^\prime \in \mathcal{B}(\ell^1(\N_0), L^\infty(0, 1))$.

\end{exercise}

\begin{solution}
Sei $f \in L^1(0,1), n \in \N$ beliebig. Dann ist
\begin{align*}
  \int_0^1 f(t)t^n dt \leq   \int_0^1|f(t)t^n|dt \leq \int_0^1 |f(t)| dt < \infty
\end{align*}
und damit existiert $Tf_n$ für alle $n \in \N$. Weiters folgt mittels majorisierter
Konvergenz und der Tatsache, dass $f(t)t^n$ punktweise für alle $t$ gegen $0$ konvergiert
\begin{align*}
  \lim_{n \to \infty} Tf_n = \lim_{n \to \infty} \int_0^1 f(t)t^n dt
  =  \int_0^1 \lim_{n \to \infty}f(t)t^n dt = \int_0^1 0 dt= 0,
\end{align*}
dass $T$ wirklich nach $C^0(\N_0)$ abbildet. Da $C^0(\N_0)$ mit der Supremumsnorm
versehen ist, folgt auch
\begin{align*}
  \|Tf\|_{\infty} = \left\|\pbraces{\Int[0][1]{f(t) t^n}{t}}_{n \in \N_0}\right\|_{\infty} =
  \left|\Int[0][1]{f(t)}{t}\right| \leq \|f\|_{L^1},
\end{align*}
dass $T$ ein beschränkter linearer Operator ist mit Norm $\leq 1$.
Der Operator $T^{\prime}$ wird durch
\begin{align*}
  \forall x \in X, y^{\prime} \in Y^{\prime}:
  y^{\prime}(Tx) = \langle Tx,y^{\prime} \rangle =
  \langle x, T^{\prime}y^{\prime}\rangle = T^{\prime}y^{\prime}(x)
\end{align*}
charakteristiert. Also ist $T^{\prime}$ eine Abbildung von $C^0(\N_0)^{\prime} = L^1(\N_0)$
nach $L^1(0, 1)^{\prime} = L^\infty(0, 1)$.
Der Isomorphismus zwischen $C^0(\N_0)^{\prime}$ und $L^1(\N_0)$ lautet
\begin{align*}
  \varphi_1:& \begin{Bmatrix}
    L^1(\N_0) & \to & C^0(\N_0)^{\prime} \\
    (a_n)_{n \in \N} &\mapsto & ((x_n)_{n \in \N} \mapsto \sum_{n \in \N}x_na_n)
  \end{Bmatrix}. \\
  \varphi_2:& \begin{Bmatrix}
    L^\infty(0, 1) & \to & L^1(0, 1)^{\prime} \\
    g &\mapsto & (f \mapsto \int_0^1 fg dt)
  \end{Bmatrix}.
\end{align*}
$T^{\prime}$ aufgefasst als Abbildung zwischen (noch nicht isomorph identifizierten
Dualräumen) leistet
\begin{align*}
  T^{\prime}\left((x_n)_{n \in \N} \mapsto \sum_{n \in \N}x_na_n\right)(f)
  &= \sum_{n \in \N}a_n\int_0^1 f(t)t^n dt \\
  &= \lim_{n \in N} \sum_{k= 0}^n a_k \int_0^1 f(t)t^k dt
  = \lim_{n \in N}  \int_0^1\sum_{k= 0}^n a_k t^k f(t) dt
\end{align*}

Wieder folgt aufgrund majorisierter Konvergenz mit der Majorante
\begin{align*}
  \int_0^1 \sum_{k=0}^n a_kt^k f(t)dt \leq \int_0^1 \sum_{n \in \N} |a_n|t^nf(t) dt
  \leq \int_0^1 \|a\|_{L^1(\N_0)}f(t) \leq \|a\|_{L^1(\N_0)}\|f\|_{L^1(0,1)},
\end{align*}
dass
\begin{align*}
  T^{\prime}\left((x_n)_{n \in \N} \mapsto \sum_{n \in \N}x_na_n\right)(f)
  = \int_0^1\sum_{n \in \N} a_n t^nf(t) dt
\end{align*}
Es gilt schließlich nach kanonischer Identifikation
\begin{align*}
  T^{\prime}((a_n)_{n \in\N}) = \sum_{n \in \N}a_n t^n \leq \sum_{n \in \N}|a_n| \in L^\infty(0, 1).
\end{align*}
\end{solution}
