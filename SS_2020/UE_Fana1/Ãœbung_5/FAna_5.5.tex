\begin{exercise}[21/1]

Sei $U: L^2(\R) \to L^2(\R)$ die Fouriertransformation.
Bestimme $\sigma(U)$ und $\sigma_p(U)$. \\

\textit{Hinweis.}
Man erinnere sich, dass $U$ auf dem dichten Teilraum $L^1(\R) \cap L^2(\R)$ gegeben ist durch

\begin{align*}
  (U f)(\zeta)
  =
  \frac{1}{\sqrt{2 \pi}}
  \Int[\R]{f(t) \exp{(-i t \zeta)}}{\lambda(t)}.
\end{align*}

Verwende den Spektralabbildungssatz und betrachte Funktionen der Bauart $p(t) e^{-\frac{t^2}{2}}$ mit einem Polynom $p(t)$.

\end{exercise}

\begin{solution}

Wir beginnen, indem wir ein paar Eigenwerte ausrechnen.
Davor sei aber bemerkt, dass Funktionen der Form $p(t)\exp(-\frac{t^2}{2})$ mit $p \in \Pi_n$ integrierbar sind, da für alle $k \in \N$
ein $t_0 \in \R$ exisitiert, sodass $t^k < \exp(\frac{t^2}{4})$. Damit folgt
\begin{align*}
  \int_\R t^k \exp\left(\frac{-t^2}{2}\right) dt &= \int_{-t_0}^{t_0} t^k \exp\left(\frac{-t^2}{2}\right) dz
  + \int_{-\infty}^{-t_0} t^k \exp\left(\frac{-t^2}{2}\right) dt + \int_{t_0}^{+\infty} t^k \exp\left(\frac{-t^2}{2}\right) dt\\
  &\leq \int_{-t_0}^{t_0} t^k \exp\left(\frac{-t^2}{2}\right) dt + 2\int_{t_0}^{+\infty}\exp\left(\frac{-t^2}{4}\right)  dt,
\end{align*}
wobei das erste Integral wegen stetiger Funktion auf Kompaktum existiert und
das zweite mittels Substitution $u = \frac{t}{\sqrt{2}}$ auf das Integral von $\exp(-\nicefrac{t^2}{2})$
zurückgeführt werden kann.
\includegraphicsboxed{Blümlinger - Beispiel 2.1.10}
\FloatBarrier
\begin{itemize}
  \item
  \begin{align*}
    U(e^{-\frac{t^2}{2}})(\zeta)
    & =
    \frac{1}{\sqrt{2 \pi}}
    \Int[\R]
    {
      \exp \pbraces{-\frac{t^2}{2}}
      \exp{(-i t \zeta)}
    }
    {\lambda(t)} \\
    & =
    \frac{1}{\sqrt{2 \pi}}
    \Bigg (
      \underbrace
      {
      \Int[\R]
      {
      \exp \pbraces{-\frac{t^2}{2}}
      \cos{(\zeta t)}
      }
      {\lambda(t)}
      }_{
      \sqrt{2 \pi}
      \exp \pbraces{-\frac{\zeta^2}{2}}
      } -
      i
      \underbrace
      {
      \Int[\R]
      {
      \exp \pbraces{-\frac{t^2}{2}}
      \sin{(\zeta t)}
      }
      {\lambda(t)}
      }_0
    \Bigg )
    =
    \exp \pbraces{-\frac{\zeta^2}{2}}
  \end{align*}

  Das erste \Quote{Underbrace} gilt wegen Blümlinger - Beispiel 2.1.10 und das zweite, weil der Integrand ungerade ist.
  Wir erhalten also $1 \in \sigma_p(U) \subseteq \sigma(U)$.
  \FloatBarrier
  \includegraphicsboxed{Blümlinger - Proposition 3.2.2}
  \FloatBarrier
  \item
  \begin{align*}
    \frac{d}{dt}
    \exp \pbraces{-\frac{t^2}{2}}
    =
    -t \exp \pbraces{-\frac{t^2}{2}}
  \end{align*}



  \begin{align*}
    U \pbraces{-t \exp \pbraces{-\frac{t^2}{2}}}(\zeta)
    =
    U \pbraces{\frac{d}{dt} \exp \pbraces{-\frac{t^2}{2}}}(\zeta)
    =
    i \zeta U \pbraces{\exp \pbraces{-\frac{t^2}{2}}}(\zeta)
    =
    i \zeta \exp \pbraces{-\frac{\zeta^2}{2}}
    =
    (-i) (-\zeta \exp \pbraces{-\frac{\zeta^2}{2}})
  \end{align*}

  Wir erhalten also $-i \in \sigma_p(U) \subseteq \sigma(U)$.
  \item
  \begin{align*}
    \frac{d^2}{dt^2}
    \exp \pbraces{-\frac{t^2}{2}}
    =
    \frac{d}{dt}
    \pbraces{-t \exp \pbraces{-\frac{t^2}{2}}}
    =
    -\exp \pbraces{-\frac{t^2}{2}} +
    t^2 \exp \pbraces{-\frac{t^2}{2}}
  \end{align*}

  \begin{align*}
    -\zeta^2 \exp \pbraces{-\frac{\zeta^2}{2}}
    & =
    U \pbraces
    {
      \frac{d^2}{dt^2}
      \exp \pbraces{-\frac{t^2}{2}}
    }
    =
    U \pbraces
    {
      -\exp \pbraces{-\frac{t^2}{2}} +
      t^2 \exp \pbraces{-\frac{t^2}{2}}
    } \\
    & =
    U \pbraces
    {
      -\frac{1}{2}
      \exp \pbraces{-\frac{t^2}{2}}
    } +
    U \pbraces
    {
      -\frac{1}{2}
      \exp \pbraces{-\frac{t^2}{2}} +
      t^2 \exp \pbraces{-\frac{t^2}{2}}
    } \\
    & =
    -\frac{1}{2} \exp \pbraces{-\frac{\zeta^2}{2}} +
    U \pbraces
    {
      -\frac{1}{2}
      \exp \pbraces{-\frac{t^2}{2}} +
      t^2 \exp \pbraces{-\frac{t^2}{2}}
    }
    =
    -\zeta^2 \exp \pbraces{-\frac{\zeta^2}{2}}
  \end{align*}

  \begin{align*}
    U \pbraces
    {
      -\frac{1}{2}
      \exp \pbraces{-\frac{t^2}{2}} +
      t^2 \exp \pbraces{-\frac{t^2}{2}}
    }
    =
    (-1) \pbraces
    {
      -\frac{1}{2}
      \exp \pbraces{-\frac{\zeta^2}{2}} +
      \zeta^2
      \exp \pbraces{-\frac{\zeta^2}{2}}
    }
  \end{align*}

  Wir erhalten also $-1 \in \sigma_p(U) \subseteq \sigma(U)$.
  \item
  \begin{align*}
    \frac{d^3}{dt^3}
    \exp \pbraces{-\frac{t^2}{2}}
    =
    \frac{d}{dt}
    \pbraces
    {
      -\exp \pbraces{-\frac{t^2}{2}} +
      t^2 \exp \pbraces{-\frac{t^2}{2}}
    }
    =
    t \exp \pbraces{-\frac{t^2}{2}} +
    2 t \exp \pbraces{-\frac{t^2}{2}} -
    t^3 \exp \pbraces{-\frac{t^2}{2}}
  \end{align*}

  \begin{align*}
    -i \zeta^3 \exp \pbraces{-\frac{t^2}{2}}
    & =
    i^3 \zeta^3 \exp \pbraces{-\frac{t^2}{2}}
    =
    U \pbraces
    {
      \frac{d^3}{dt^3}
      \exp \pbraces{-\frac{t^2}{2}}
    }
    =
    U \pbraces
    {
      3 t \exp \pbraces{-\frac{t^2}{2}} -
      t^3 \exp \pbraces{-\frac{t^2}{2}}
    } \\
    & =
    \frac{3}{2}
    U \pbraces{t \exp \pbraces{-\frac{t^2}{2}}} +
    U \pbraces
    {
      \frac{3}{2}
      t \exp \pbraces{-\frac{t^2}{2}} -
      t^3 \exp \pbraces{-\frac{t^2}{2}}
    } \\
    & =
    -\frac{3}{2} i \zeta \exp \pbraces{-\frac{t^2}{2}} +
    U \pbraces
    {
      \frac{3}{2}
      t \exp \pbraces{-\frac{t^2}{2}} -
      t^3 \exp \pbraces{-\frac{t^2}{2}}
    }
  \end{align*}

  \begin{align*}
    U \pbraces
    {
      \frac{3}{2}
      t \exp \pbraces{-\frac{t^2}{2}} -
      t^3 \exp \pbraces{-\frac{t^2}{2}}
    }
    =
    i \pbraces
    {
      \frac{3}{2} t \exp \pbraces{-\frac{t^2}{2}} -
      t^3 \exp \pbraces{-\frac{t^2}{2}}
    }
  \end{align*}

  Wir erhalten also $i \in \sigma_p(U) \subseteq \sigma(U)$. \\
\end{itemize}


Wir erhalten also insgesamt $\Bbraces{\pm 1, \pm i} \subseteq \sigma_p(U) \subseteq \sigma(U)$.
\FloatBarrier
\includegraphicsboxed{Blümlinger - Satz 3.3.2}
\includegraphicsboxed{Proposition 6.6.7}
\includegraphicsboxed{Proposition 6.6.8}
\FloatBarrier

Laut Blümlinger - Satz 3.3.2, ist $U: L^2(\R) \to L^2(\R)$ eine surjektive Isometrie.
Laut Proposition 6.6.7, ist $U$ also unitär.
Laut Proposition 6.6.8, gilt also

\begin{align*}
  \Bbraces{\pm 1, \pm i}
  \subseteq
  \sigma_p(U)
  \subseteq
  \sigma(U)
  \subseteq
  \T
  =
  \Bbraces{z \in \C: |z| = 1}.
\end{align*}

\includegraphicsboxed{Blümlinger - Proposition 3.3.1}

Laut Blümlinger - Proposition 3.3.1, ist der Schwartzraum $S \subseteq L^2(\R)$ und $T := U|_S$ ein Automorphismus.

\begin{align*}
  S
  :=
  \Bbraces
  {
    f \in C^\infty(\R):
    \Forall n, m \in \N:
    \sup \Bbraces{|x^n f^{(m)}(x)| : x \in \R} < \infty
  }
\end{align*}

\includegraphicsboxed{Blümlinger - Satz 2.5.1}

Laut Blümlinger - Satz 2.5.1, liegt $C_c^\infty(\R) \subseteq S$ dicht in $L^2(\R)$.
Also liegt auch $S$ dicht in $L^2(\R)$. \\

Weil auch $S \subseteq L^1(\R)$, können wir $T$, wie im Hinweis, explizit anschreiben. \\

\includegraphicsboxed{Blümlinger - MISC}

Wir betrachten den Operator $R f(x) = f(-x)$.
Mit Substitution folgt $\Forall f \in S:$

\begin{align*}
  T R f
  =
  \Int[-\infty][\infty]
  {f(-t) \exp{(-i t \zeta)}}{t}
  \begin{vmatrix}
    s = -t \\
    ds = -dt
  \end{vmatrix}
  =
  -\Int[\infty][-\infty]{f(s) \exp{(i s \zeta)}}{s}
  =
  \Int[-\infty][\infty]{f(s) \exp{(i t \zeta)}}{s}
  =
  R T f.
\end{align*}

Laut Blümlinger - (3.14), erhalten wir also $T^{-1} = R T$.

\begin{align*}
  T^{-1} = R T = T R
  \implies
  R f = R T^{-1} T f = R R T T f = T^2 f
  \implies
  T^2 = R, T^4 = R^2 = I.
\end{align*}

Da $S$ dicht in $L^2(\R)$ liegt, gilt damit auch $U^4 = I$.

\includegraphicsboxed{Satz 6.4.7 (Spektralabbildungssatz)}

Mit dem Spektralabbildungssatz, Satz 6.4.7, folgt also

\begin{align*}
  \implies
  1 = \sigma(I) = \sigma(U^4) = (\sigma(U))^4
  \implies
  \Bbraces{\pm 1, \pm i}
  \subseteq
  \sigma_p(U)
  \subseteq
  \sigma(U)
  \subseteq
  \Bbraces{\pm 1, \pm i}.
\end{align*}

\end{solution}
