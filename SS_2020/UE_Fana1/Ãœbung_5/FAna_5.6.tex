\begin{exercise}[22/1]

Sei $X$ ein Banachraum, $A, T \in \mathcal{B}(X)$ mit $0 \in \rho(A)$, und setze $B := A + T$.
Wir wissen (Neumannsche Reihe):
Ist $\norm{T} < \norm{A^{-1}}^{-1}$, dann ist $0 \in \rho(B)$.
Unter der Voraussetzung dass $A T = T A$, zeige die stärkere Aussage:
Ist $r(T) < r(A^{-1})^{-1}$, dann ist $0 \in \rho(B)$.
Warum ist diese Aussage eigentlich stärker als die obige?

\end{exercise}

\begin{solution}

ToDo!

\end{solution}
