\begin{exercise}[22/1]

Sei $X$ ein Banachraum, $A, T \in \mathcal{B}(X)$ mit $0 \in \rho(A)$, und setze $B := A + T$.
Wir wissen (Neumannsche Reihe):
Ist $\norm{T} < \norm{A^{-1}}^{-1}$, dann ist $0 \in \rho(B)$.
Unter der Voraussetzung dass $A T = T A$, zeige die stärkere Aussage:
Ist $r(T) < r(A^{-1})^{-1}$, dann ist $0 \in \rho(B)$.
Warum ist diese Aussage eigentlich stärker als die obige?

\end{exercise}

\begin{solution}
  Zuerst stellen wir fest, dass $T$ auch mit $A^{-1}$ kommutiert, denn es gilt
  \begin{align}
      A^{-1}T = A^{-1}TAA^{-1} = A^{-1}ATA^{-1} = TA^{-1}.
  \end{align}

  Damit und mit Satz 6.4.14 erhalten wir nun folgende Abschätzung:
  \begin{align}
      r(TA^{-1}) = \lim\limits_{n \rightarrow \infty}{\|(TA^{-1})^n\|^{\frac{1}{n}}} =
      \lim\limits_{n \rightarrow \infty}{\|T^n(A^{-1})^n\|^{\frac{1}{n}}} \\ \leq
      \lim\limits_{n \rightarrow \infty}{\|T^n\|^{\frac{1}{n}}\|(A^{-1})^n\|^{\frac{1}{n}}} =
      r(T) r(A^{-1}) < 1,
  \end{align}

  wobei die letzte Ungleichung nach Voraussetzung gilt.

  Nach der Definition des Spektralradius gilt also $-1 \in \rho(TA^{-1}),$ was gleichbedeutend ist damit, dass $(TA^{-1} + I) = A^{-1} (T + A)$ invertierbar, diese Abbildung also bijektiv ist.
  Da das bei einer Verkettung von Abbildungen genau dann der Fall ist, wenn beide Abbildungen bijektiv sind, ist auch $T + A$ invertierbar, das heißt $0 \in \rho(B).$


  Des Weiteren gilt stets $r(A) \leq \|A\|$; die neue Aussage ist also stärker, weil die Voraussetzung schwächer ist.
  \begin{align*}
    r(T) \leq \norm{T} < \norm{A^{-1}}^{-1} \leq r(A^{-1})^{-1}
  \end{align*}
  \textbf{(wsl will er haben, dass sie echt stärker ist)}.
\end{solution}
