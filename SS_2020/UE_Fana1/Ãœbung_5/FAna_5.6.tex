\begin{exercise}[22/1]

Sei $X$ ein Banachraum, $A, T \in \mathcal{B}(X)$ mit $0 \in \rho(A)$, und setze $B := A + T$.
Wir wissen (Neumannsche Reihe):
Ist $\norm{T} < \norm{A^{-1}}^{-1}$, dann ist $0 \in \rho(B)$.
Unter der Voraussetzung dass $A T = T A$, zeige die stärkere Aussage:
Ist $r(T) < r(A^{-1})^{-1}$, dann ist $0 \in \rho(B)$.
Warum ist diese Aussage eigentlich stärker als die obige?

\end{exercise}

\begin{solution}

\underline{Was ist was?}

\includegraphicsboxed{Definition 6.4.4}

Der Spektralradius von $a \in A$ ist dabei definiert als

\begin{align*}
  r(a)
  :=
  \max_{\lambda \in \sigma(a)} |\lambda|.
\end{align*}

Um zu zeigen, dass $0 \in \rho(B)$, müssen wir also $B \in \Inv{\mathcal{B}(X)}$.
Wir suchen also eine Inverse von $B$. \\

\underline{Argumentation 1:}

\begin{align*}
  B
  =
  A + T
  =
  (I - (-T A^{-1})) A
\end{align*}

$A$ ist bereits invertierbar, weil $0 \in \rho(A)$.
Ein bereits in der Angabe angesprochener Kandidat für die Inverse des ersten Faktors ist die Neumannsche Reihe.
Um zu argumentieren, dass diese existiert, d.h. (absolut) konvergiert, und das Gewünschte leistet, gehen wir wie folgt vor. \\

Wir stellen fest, dass $T$ auch mit $A^{-1}$ kommutiert.

\begin{align}
    A^{-1} T
    =
    A^{-1} T A A^{-1}
    =
    A^{-1} A T A^{-1}
    = T A^{-1}.
\end{align}

\includegraphicsboxed{Satz 6.4.14}

Laut Satz 6.4.14, gilt also $\Forall \epsilon > 0: \Exists N \in \N: \Forall n \geq N:$

\begin{align*}
\begin{array}{ccccc}
  \norm{T^n}^\frac{1}{n} - r(T) < \epsilon &
  \implies &
  \norm{T^n}^\frac{1}{n} < \epsilon + r(T) &
  \implies &
  \norm{T^n} < (\epsilon + r(T))^n, \\
  \norm{A^{-n}}^\frac{1}{n} - r(A^{-1}) < \epsilon &
  \implies &
  \norm{A^{-n}}^\frac{1}{n} < \epsilon + r(A^{-1}) &
  \implies &
  \norm{A^{-n}} < (\epsilon + r(A^{-1}))^n
\end{array}
\end{align*}

Man beachte, dass die $|\cdot|$ wegen des $\inf$ weggelassen werden können.
Für $\epsilon$ hinreichend klein gilt also

\begin{align*}
  \norm{T^n} \norm{A^{-n}}
  <
  (\epsilon + r(T))^n
  (\epsilon + r(A^{-1}))^n
  =
  (
    \underbrace
    {
      (\epsilon + r(T))
      (\epsilon + r(A^{-1}))
    }_{=: q < 1}
  )^n < 1.
\end{align*}

Damit, konvergiert die folgende Neumannsche Reihe absolut.

\begin{align*}
  \norm{\sum_{n = 0}^\infty (-T A^{-1})^n}
  \leq
  \sum_{n = 0}^{N-1} \norm{T^n} \norm{A^{-n}} +
  \sum_{n = N}^\infty \norm{T^n} \norm{A^{-n}}
  \leq
  \sum_{n = 0}^{N-1} \norm{T^n} \norm{A^{-n}} +
  \sum_{n = N}^\infty q^n
  < \infty
\end{align*}

Weil $T$ und $A^{-1}$ kommutieren, rechnet man wie immer nach, dass die Neumannsche Reihe tatsächlich eine Inverse ist. \\

\underline{Argumentation 2:} \\

Weil $T$ und $A^{-1}$ kommutieren, erhalten wir mit Satz 6.4.14 folgende Abschätzung:

\begin{align*}
    r(TA^{-1})
    =
    \lim_{n \to \infty}{\|(TA^{-1})^n\|^{\frac{1}{n}}}
    =
    \lim_{n \to \infty}
    \norm{T^n(A^{-n})}^\frac{1}{n}
    \leq
    \lim_{n \to \infty}
    \norm{T^n}^\frac{1}{n}
    \norm{(A^{-n})}^\frac{1}{n}
    =
    r(T) r(A^{-1}) < 1
\end{align*}

Die letzte Ungleichung gilt nach Voraussetzung. \\

Nach der Definition des Spektralradius gilt also $-1 \in \rho(TA^{-1}),$ was gleichbedeutend ist damit, dass $(TA^{-1} + I) = A^{-1} (T + A)$ invertierbar, diese Abbildung also bijektiv ist. Da das bei einer Verkettung von Abbildungen genau dann der Fall ist, wenn beide Abbildungen bijektiv sind, ist auch $T + A$ invertierbar, das heißt $0 \in \rho(B).$ \\

\underline{Stärkere Aussage:} \\

Wegen Satz 6.4.14, gilt überdies

\begin{align*}
  r(a)
  =
  \inf_{n \in \N}
  \norm{a^n}^\frac{1}{n}
  \leq
  \norm{a}.
\end{align*}
und daher
\begin{align*}
  \|T\| \leq \|A^{-1}\|^{-1} \implies r(T) \leq \|T\| \leq \|A^{-1}\|^{-1} \leq r(A)^{-1}.
\end{align*}

Die neue Aussage ist also stärker, weil die Voraussetzung schwächer ist.

Als Beispiel dafür, dass die schwächere Voraussetzung erfüllt ist, die stärkere aber nicht,
betrachte die Matrizen
\begin{align*}
  T &= \begin{pmatrix}
    0.5 & 1 \\ 0 & 0.5
  \end{pmatrix} \\
  A &= \begin{pmatrix}
    1 & 0 \\ 0 & 1
  \end{pmatrix}
\end{align*}
Dann gilt
\begin{align*}
  r(T)r(A) = 0.5\cdot1 < 1 \\
  \|T\|_{\infty}\|A\|_{\infty} = 1.5\cdot1 > 1
\end{align*}
\end{solution}
