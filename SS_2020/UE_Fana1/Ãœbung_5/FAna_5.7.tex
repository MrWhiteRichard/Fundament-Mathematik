\begin{exercise}[22/2$^\ast$]

Sei $\mathcal{K}(\C)$ die Menge aller kompakten Teilmengen von $C$, und

\begin{align*}
  d_H(M, N)
  :=
  \max \Bbraces
  {
    \sup_{x \in M} \inf_{y \in N} \vbraces{x - y},
    \sup_{y \in N} \inf_{x \in M} \vbraces{x - y}
  },
  \quad
  M, N \in \mathcal{K}(\C).
\end{align*}

Es gilt dass $d_H$ eine Metrik ist, die \textit{Hausdorff-Metrik}.
Sei nun $X$ ein Banachraum.
Zeige:

\begin{enumerate}[label = (\alph*)]

  \item
  Sind $A, B \in \mathcal{B}(X)$ mit $A B = B A$, dann ist $d_H(\sigma(A), \sigma(B)) \leq r(A - B)$.

  \item
  Ist $\mathcal{C}$ eine kommutative Teilalgebra von $\mathcal{B}(X)$, so ist die Funktion

  \begin{align*}
    \Sigma:
    \begin{cases}
      (\mathcal{C}, \norm[\mathcal{B}(X)]{.}) & \to (\mathcal{K}(\C), d_H) \\
      A & \mapsto \sigma(A)
    \end{cases}
  \end{align*}

  stetig.

\end{enumerate}

\end{exercise}

\begin{solution}
Aus Lemma 6.4.10 folgt die Kompaktheit von $\sigma(A)$ und $\sigma(B)$. Damit
ist der Metrik-Abstand zwischen diesen beiden Mengen wohldefiniert.

Sei $x \in \sigma(A)$ beliebig. Betrachte $\inf_{y \in \sigma(B)}|x-y|$.
Also ist $(A - xe),(B - ye) \notin \Inv(\mathcal{B}(X))$.

\begin{align*}
  (A - B) -(x -y)e = wahtever
\end{align*}

\begin{align*}
  |x - y| \leq |x| + |y| \leq r(A) + r(B) = \lim_{n \to \infty}\|A^n\|^{\nicefrac{1}{n}}
  + \|B^n\|^{\nicefrac{1}{n}}
\end{align*}

\begin{align*}
  r(A - B) = \lim_{n \to \infty} \|(A-B)^n\|^{\nicefrac{1}{n}} \leq \|A-B\|
\end{align*}
\end{solution}
