\begin{exercise}[22/2$^\ast$]

Sei $\mathcal{K}(\C)$ die Menge aller kompakten Teilmengen von $C$, und

\begin{align*}
  d_H(M, N)
  :=
  \max \Bbraces
  {
    \sup_{x \in M} \inf_{y \in N} \vbraces{x - y},
    \sup_{y \in N} \inf_{x \in M} \vbraces{x - y}
  },
  \quad
  M, N \in \mathcal{K}(\C).
\end{align*}

Es gilt dass $d_H$ eine Metrik ist, die \textit{Hausdorff-Metrik}.
Sei nun $X$ ein Banachraum.
Zeige:

\begin{enumerate}[label = (\alph*)]

  \item
  Sind $A, B \in \mathcal{B}(X)$ mit $A B = B A$, dann ist $d_H(\sigma(A), \sigma(B)) \leq r(A - B)$.

  \item
  Ist $\mathcal{C}$ eine kommutative Teilalgebra von $\mathcal{B}(X)$, so ist die Funktion

  \begin{align*}
    \Sigma:
    \begin{cases}
      (\mathcal{C}, \norm[\mathcal{B}(X)]{.}) & \to (\mathcal{K}(\C), d_H) \\
      A & \mapsto \sigma(A)
    \end{cases}
  \end{align*}

  stetig.

\end{enumerate}

\end{exercise}

\begin{solution}
\begin{enumerate}[label = (\alph*)]

\item
Wir zeigen zuerst eine Hilfsaussage:
\begin{align}\label{Lemmachen}
  \Forall C \in \mathcal{B}(X): r(C) = r(-C)
\end{align}
Dies gilt, da
\begin{align*}
  C - \lambda I \in \Inv(\mathcal{B}(X)) \Leftrightarrow & -C + \lambda I = -(C - \lambda I) \in \Inv(\mathcal{B}(X)) \\
  \Rightarrow \lambda \in \sigma (C) \Leftrightarrow & -\lambda \in \sigma (-C) \\
  \Rightarrow r(C) = \max_{\lambda \in \sigma (C)} |\lambda| = \max_{\lambda \in \sigma (C)} |-\lambda| = & \max_{\lambda \in \sigma (-C)} |\lambda| = r(-C)
\end{align*}

Wir wissen nach Lemma 6.4.10, dass die Spektren kompakt sind, somit ist $d_H(\sigma(A), \sigma(B))$ wohldefiniert.

Sei o.B.d.A $\sigma(A) \neq \sigma(B)$ (sonst ist die linke Seite $=0$). Wir können also ein $\lambda \in \sigma(A) \setminus \sigma(B)$ wählen (wieder o.B.d.A.) und wollen zeigen
\begin{align}\label{minuslambda}
  \sigma(A-\lambda I) = \sigma(A) - \lambda
\end{align}
(gilt auch für $B$).
1. Sei $\tau \in \sigma(A-\lambda I)$. Dann gilt nach Definition
\begin{align*}
  (A-\lambda I) - \tau I = A - (\lambda + \tau) I \notin \Inv(\mathcal{B}(X)) \\
  \Rightarrow \lambda + \tau \in \sigma(A)
\end{align*}

2. Sei $\tau \in \sigma(A) - \lambda$. Dann gilt
\begin{align*}
  A - (\lambda + \tau) I = (A - \lambda I) -\tau I \notin \Inv(\mathcal{B}(X)) \\
  \Rightarrow \tau \in \sigma(A - \lambda I)
\end{align*}

Wir setzen nun $T := A - B$ und somit auch $A- \lambda I = B - \lambda I + T$.

Wir haben $\lambda$ so gewählt, dass $\lambda \in \sigma(A)$. Also ist auch $0 \in \sigma(A - \lambda I)$ und $0 \notin \rho(A - \lambda I)$.

Ebenso folgt $0 \in \rho(B - \lambda I)$.

Also folgt aus der vorigen Aufgabe, dass
\begin{align*}
  r(A-B) = r(T) \geq r((B-\lambda I)^{-1})^{-1}
\end{align*}

($(B- \lambda I)$ nimmt die Rolle von $A$ ein und $(A- \lambda I)$ nimmt die Rolle von $B$ ein. Offensichtlich vertauschen sie.)

\begin{align*}
  \Rightarrow r(A-B) \geq r((B-\lambda I)^{-1})^{-1} \stackrel{\text{Def. und Lemma 6.4.8}}{=}
  (\max_{\beta \in \sigma((B - \lambda I)^{-1})}) |\beta|)^{-1} =
  (\max_{\tilde{\beta} \in \sigma(B - \lambda I)} |\tilde{\beta}|^{-1})^{-1} \\
  \stackrel{\eqref{minuslambda}}{=} (\max_{b \in \sigma(B)} |b - \lambda|^{-1})^{-1} =
  \min_{b \in \sigma(B)} |b - \lambda| = \inf_{\beta \in \sigma(B)} |\beta - \lambda|
\end{align*}

Da $\lambda$ beliebig war, gilt also auch:
\begin{align*}
  r(A-B) \geq \sup_{\lambda \in \sigma(A) \setminus \sigma(B)} \inf_{\beta \in \sigma(B)} \vbraces{\beta - \lambda} = \sup_{\lambda \in \sigma(A)} \inf_{\beta \in \sigma(B)} \vbraces{\beta - \lambda}
\end{align*}

Die letzte Gleichung, da für $\lambda \in \sigma(B)$ eh das Infimum gleich $0$ ist.

Analog zeigt man
\begin{align*}
  r(B-A) \geq \sup_{\beta \in \sigma(B)} \inf_{\lambda \in \sigma(A)} \vbraces{\beta - \lambda}
\end{align*}

Gemeinsam mit \eqref{Lemmachen} bekommen wir
\begin{align*}
  r(A-B) \geq \max\{\sup_{\lambda \in \sigma(A)} \inf_{\beta \in \sigma(B)} \vbraces{\beta - \lambda},\sup_{\beta \in \sigma(B)} \inf_{\lambda \in \sigma(A)} \vbraces{\beta - \lambda}\} = d_H(\sigma(A), \sigma(B))
\end{align*}

\item
\begin{align*}
  d_H(\Sigma(A), \Sigma(B)) = d_H(\sigma(A), \sigma(B)) \leq r(A-B) \leq \|A-B\|
\end{align*}
\end{enumerate}
\end{solution}
