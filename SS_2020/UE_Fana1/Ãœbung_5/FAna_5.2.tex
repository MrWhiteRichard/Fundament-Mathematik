\begin{exercise}

Seien $X, Y$ kompakte Hausdorff Räume, sei $\tau: Y \to X$ stetig, und $A \in \mathcal{B}(C(X), C(Y))$ der Operator $A f := f \circ \tau$.
Zeige, dass sein Konjugierter $A^\prime \in \mathcal{B}(M(Y), M(X))$ gegeben ist durch

\begin{align*}
  (A^\prime \mu)(\Delta) = \mu(\tau^{-1}(\Delta)),
  \quad
  \mu \in M(Y), \Delta \subseteq X ~\text{Borelmenge}.
\end{align*}

\end{exercise}

\begin{solution}
$M(X),M(Y)$ Menge aller messbaren Funktionen auf $X,Y$? Wohin?
Oder Borelmaße? Oder was?
\begin{align*}
  (A^{\prime}\mu)(\Delta) = \langle \Delta, A^{\prime}\mu \rangle =
  \langle A \Delta, \mu \rangle = \mu(A \Delta) = \mu(\Delta \circ \tau)
\end{align*}


\end{solution}
