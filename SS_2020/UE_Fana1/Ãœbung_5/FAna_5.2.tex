\begin{exercise}

Seien $X, Y$ kompakte Hausdorff Räume, sei $\tau: Y \to X$ stetig, und $A \in \mathcal{B}(C(X), C(Y))$ der Operator $A f := f \circ \tau$.
Zeige, dass sein Konjugierter $A^\prime \in \mathcal{B}(M(Y), M(X))$ gegeben ist durch

\begin{align*}
  (A^\prime \mu)(\Delta) = \mu(\tau^{-1}(\Delta)),
  \quad
  \mu \in M(Y), \Delta \subseteq X ~\text{Borelmenge}.
\end{align*}

\end{exercise}

\begin{solution}
Da $X,Y$ kompakt sind, gilt $C(X) = C_c(X), C(Y) = C_c(Y)$.
Laut dem Darstellungssatz von Riesz-Markov gilt $C_c(L)^{\prime} = M_{reg}(L)$,
wobei $(M_{reg}(L), \|\cdot\|)$ den Raum der komplexen regulären Borelmaße auf $L$
mit der Totalvariation als Norm.
Der Isomorphismus lautet
\begin{align*}
  \Phi(\mu): \begin{Bmatrix}
    M_{reg}(L) & \to & C_c(L)^{\prime} \\
    \mu & \mapsto & (f \mapsto \int_L f d\mu)
  \end{Bmatrix}.
\end{align*}
Seien nun $f \in C_c(X), g^{\prime} \in C_c(Y)^{\prime}$ beliebig mit $\Phi(\mu_g) = g$.
\begin{align*}
  (A^{\prime}g^{\prime})f = \langle f, A^{\prime}g^{\prime} \rangle = \langle A f, g^{\prime} \rangle
  = g^{\prime}(Af) = \Phi(\mu_g)(f \circ \tau)
  = \int_Y f\circ \tau d\mu_g
  = \int_{\tau^{-1}(X)} f \circ \tau d\mu_g.
\end{align*}
Mit dem Transformationssatz für Maße folgt schließlich
\begin{align*}
  (A^{\prime}g^{\prime})f = \int_{\tau^{-1}(X)} f \circ \tau d\mu_g
  = \int_{X} f d\mu_g\tau^{-1},
\end{align*}
also ist $\mu_g\tau^{-1}$ ein reguläres Maß auf $X$.
Setzen wir nun für eine beliebige Borelmenge $f := \1\Delta$ erhalten wir
\begin{align*}
  (A^{\prime}g^{\prime})\1\Delta = \int_{X} \1\Delta d\mu_g\tau^{-1}
  = \mu_g(\tau^{-1})(\Delta)
\end{align*}
\end{solution}
