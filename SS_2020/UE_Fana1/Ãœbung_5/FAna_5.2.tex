\begin{exercise}

Seien $X, Y$ kompakte Hausdorff Räume, sei $\tau: Y \to X$ stetig, und $A \in \mathcal{B}(C(X), C(Y))$ der Operator $A f := f \circ \tau$.
Zeige, dass sein Konjugierter $A^\prime \in \mathcal{B}(M(Y), M(X))$ gegeben ist durch

\begin{align*}
  (A^\prime \mu)(\Delta) = \mu(\tau^{-1}(\Delta)),
  \quad
  \mu \in M(Y), \Delta \subseteq X ~\text{Borelmenge}.
\end{align*}

\end{exercise}

\begin{solution}
Wir nehmen hier an, dass in der Angabe mit $M(X),M(Y)$ bereits $M_{reg}(X), M_{reg}(Y)$ gemeint sind. \\
Da $X,Y$ kompakte Hausdorffräume sind, somit insbesondere lokalkompakt, gilt $C(X) = C_0(X), C(Y) = C_0(Y)$.
\begin{align*}
  C_0(X) := \{f \in C(X): \forall \epsilon > 0: \exists K \text{ kompakt}:
  \forall x \notin K: |f(x)| < \epsilon\}
\end{align*}
Laut dem Darstellungssatz von Riesz-Markov gilt $C_0(L)^{\prime} = M_{reg}(L)$,
wobei $(M_{reg}(L), \|\cdot\|)$ den Raum der komplexen regulären Borelmaße auf $L$
mit der Totalvariation als Norm bezeichnet.
Der Isomorphismus lautet
\begin{align*}
  \Phi_X(\mu)&: \Bigg\{ \begin{matrix}
    M_{reg}(X) & \to & C_0(X)^{\prime} \\
    \mu & \mapsto & (f \mapsto \int_X f d\mu)
  \end{matrix} \\
  \Phi_Y(\mu)&: \Bigg\{ \begin{matrix}
    M_{reg}(Y) & \to & C_0(Y)^{\prime} \\
    \mu & \mapsto & (f \mapsto \int_Y f d\mu)
  \end{matrix}.
\end{align*}
Seien nun $f \in C_0(X), g^{\prime} \in C_0(Y)^{\prime}$ beliebig mit $\Phi_Y(\mu_g) = g^{\prime}$.
\begin{align*}
  (A^{\prime}g^{\prime})f = \langle f, A^{\prime}g^{\prime} \rangle = \langle A f, g^{\prime} \rangle
  = g^{\prime}(Af) = \Phi_Y(\mu_g)(f \circ \tau)
  = \int_Y f\circ \tau d\mu_g
  = \int_{\tau^{-1}(X)} f \circ \tau d\mu_g.
\end{align*}
Mit dem Transformationssatz für Maße folgt schließlich
\begin{align*}
  (A^{\prime}g^{\prime})f = \int_{\tau^{-1}(X)} f \circ \tau d\mu_g
  = \int_{X} f d\mu_g\tau^{-1},
\end{align*}
mit dem regulären Maß $\mu_g\tau^{-1}$ auf $X$.
Es gilt also
\begin{align*}
  (A^{\prime}\Phi_Y(\mu_g)) = \Phi_X(\mu_g\tau^{-1})
\end{align*}
und damit nach isomorpher Identifikation für jede Borelmenge $\Delta \subseteq X$
\begin{align*}
  (A^{\prime}\mu_g)(\Delta) = \mu_g\tau^{-1}(\Delta) = \mu_g((\tau^{-1}(\Delta)).
\end{align*}
\end{solution}
