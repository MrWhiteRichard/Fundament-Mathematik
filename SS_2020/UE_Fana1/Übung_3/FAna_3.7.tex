\begin{exercise}
Sei $\Omega$ eine Menge und $X$ ein linearer Raum, dessen Elemente Funktionen von
$\Omega$ nach $\C$ sind, und dessen lineare Operationen durch punktweise Addition
und skalare Multiplikation gegeben sund. Für $\omega \in \Omega$ bezeichne mit
$\chi_{\omega}: X \rightarrow \C$ das Punktauswertungsfunktional
\begin{align*}
  \chi_{\omega}(f) := f(\omega), \qquad f \in X.
\end{align*}
Dann ist $\chi_{\omega}$ linear. Zeige, dass es (bis auf Äquivalenz der Normen)
höchstens eine Norm $||\cdot||$ auf $X$ geben kann, sodass $(X,||\cdot||)$ ein
Banachraum ist und alle Punktauswertungsfunktionale bezüglich $||\cdot||$ stetig sind. \\
\textit{Hinweis:} Wende den Satz vom abgeschlossenen Graphen auf die identische Abbildung an.
\end{exercise}
\begin{solution}
Beweis.
\end{solution}
