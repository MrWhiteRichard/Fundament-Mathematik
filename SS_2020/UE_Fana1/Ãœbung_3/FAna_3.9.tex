\begin{exercise}
Betrachte $L^2(0,1)$ als Teilmenge von $L^1(0,1)$ und zeige auf drei verschiedene
Arten, dass $L^2(0,1)$ von 1.Kategorie (für die Terminologie siehe Bemerkung 4.1.4
im Skriptum) in $L^1(0,1)$ ist:
\begin{enumerate}[label = (\roman*)]
  \item Zeige $\left\{f \in L^2: \int_0^1 |f(t)|^2 dt \leq n\right\}$ ist
  abgeschlossen (in $L^1$) und hat leeres Inneres.
  \item Setze
  \begin{align*}
    g_n(t) := \begin{cases}
      n, & 0 \leq t \leq \frac{1}{n^3} \\
      0, & \frac{1}{n^3} < t \leq 1
    \end{cases}
  \end{align*}
  und zeige, dass
  \begin{align*}
    \int_0^1 f(t)g_n(t) \rightarrow 0
  \end{align*}
  für jedes $f \in L^2$, aber nicht für jedes $f \in L^1$.
  \item Bemerke, dass die identische Abbildung
  \begin{align*}
    \iota: \begin{cases}
      L^2 &\rightarrow L^1 \\
      f &\mapsto f
    \end{cases}
  \end{align*}
  stetig, aber nicht surjektiv ist.
\end{enumerate}
Argumentiere, warum jede dieser Aussagen (i),(ii),(iii), tatsächlich die
gewünschte Aussage impliziert!

\end{exercise}
\begin{solution}
\begin{enumerate}[label = (\roman*)]
  \item
\end{enumerate}
\end{solution}
