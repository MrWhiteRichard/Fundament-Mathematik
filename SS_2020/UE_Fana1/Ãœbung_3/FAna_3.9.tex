\begin{exercise}

Betrachte $L^2(0, 1)$ als Teilmenge von $L^1(0, 1)$ und zeige auf drei verschiedene Arten, dass $L^2(0, 1)$ von 1. Kategorie (für die Terminologie siehe Bemerkung 4.1.4 im Skriptum) in $L^1(0,1)$ ist:

\begin{enumerate}[label = (\roman*)]

  \item
  Zeige $\Bbraces{f \in L^2: \int_0^1 |f(t)|^2 dt \leq n}$ ist abgeschlossen (in $L^1$) und hat leeres Inneres.

  \item
  Setze

  \begin{align*}
    g_n(t)
    :=
    \begin{cases}
      n, & 0 \leq t \leq \frac{1}{n^3} \\
      0, & \frac{1}{n^3} < t \leq 1
    \end{cases}
  \end{align*}

  und zeige, dass

  \begin{align*}
    \int_0^1 f(t)g_n(t) \rightarrow 0
  \end{align*}

  für jedes $f \in L^2$, aber nicht für jedes $f \in L^1$.

  \item
  Bemerke, dass die identische Abbildung

  \begin{align*}
    \iota:
    \begin{cases}
      L^2 &\rightarrow L^1 \\
      f &\mapsto f
    \end{cases}
  \end{align*}

  stetig, aber nicht surjektiv ist.

\end{enumerate}

Argumentiere, warum jede dieser Aussagen (i),(ii),(iii), tatsächlich die gewünschte Aussage impliziert!

\end{exercise}

\begin{solution}

\begin{enumerate}[label = (\roman*)]
\item \begin{align*}
  M_n := \{ f \in L^2(0,1): \int_0^1 |f(t)|^2dt \leq n \}
  = \{ f \in L^2(0,1): ||f||^2 \leq \sqrt{n}\}
\end{align*}
Sei $f \in \overline{M_n}$ beliebig, dann existiert ein Netz $(f_i)_{i \in I}$ aus $M_n$ mit
\begin{align*}
  \lim_{i \in I}||f_i - f||_{L^2} = 0.
\end{align*}
Aus der Maßtheorie (Kusolitsch Satz 13.15) wissen wir, dass aus der Konvergenz im
p-ten Mittel bereits die Konvergenz im Maß folgt, das heißt
\begin{align*}
  \forall \epsilon > 0: \lim_{i \in I} \lambda(|f_n - f| > \epsilon) = 0.
\end{align*}
Insbesondere liegt eine Cauchy-Folge im Maß vor, woraus wiederum folgt, dass ein
Teilnetz $(f_{i_j})_{j \in J})$ existiert, welches fast überall gegen $f$ konvergiert.
Damit konvergiert auch $(|f_{i_j}^2|)_{j \in J}$ $\lambda$-fast überall gegen $|f^2|$.
Nach dem Lemma von Fatou angewandt auf $|f_{i_j}|^2 \geq g := 0$, folgt
\begin{align*}
  \int_0^1 |f|^2 d\lambda = \int_0^1 \lim_{j \in J}|f_{i_j}|^2 d\lambda
  = \int_0^1 \liminf_{j \in J}|f_{i_j}|^2 d\lambda \leq \liminf_{j \in J} \int_0^1 |f_{i_j}|^2 d\lambda \leq n.
\end{align*}
Also gilt $f \in M_n$ und somit ist $M_n$ abgeschlossen. \\
Nun zeigen wir noch, dass $M_n$ leeres Inneres hat. Dazu sei $g \in L^1\backslash L^2$
und $f \in M_n$, sowie $\forall k \in \N: h_k := f + \frac{1}{k}g$. Dann gilt
\begin{align*}
  \forall k \in \N: h_k \in L^1 \land h_k \notin L^2.
\end{align*}
Außerdem gilt
\begin{align*}
  \lim_{h \rightarrow \infty}\|h_k - f\|_{L^1} = \lim_{k \rightarrow \infty} \|g\|_{L^1} = 0,
\end{align*}
also folgt $h_k \rightarrow^{L^1} f$. Damit ist $M_n^{\circ} = \overline{M_n}^{\circ} = \emptyset$
und $L^2(0,1) = \bigcup_{n \in \N}M_n$ kann als abzählbare Vereinigung nirgends dichter
Mengen geschrieben werden, ist also von 1.Kategorie.
\item Sei $f \in L^2$ beliebig. Mit der Ungleichung von Hölder gilt dann
\begin{align*}
  |\int_0^1 fg_n d\lambda | \leq \|f\|_2\|g\|_2 = \|f\|_2 \int_0^{\frac{1}{n^3}}n^2 dt = \|f\|^2 \frac{1}{n}
  \rightarrow^{n\rightarrow \infty} 0.
\end{align*}
Für das Gegenbeispiel betrachte $f(t) := \frac{1}{\sqrt{t}}$. Es gilt
\begin{align*}
  \int_0^1 \frac{1}{\sqrt{t}} = 2\sqrt{t}\bigg|_0^1 = 2,
\end{align*}
also ist $f \in L^1(0,1)$. Nun berechne man
\begin{align*}
  \int_0^1 f(t)g_n(t) dt = \int_0^{\frac{1}{n^3}}\frac{n}{\sqrt{t}} dt
  = 2\sqrt{t}\bigg|_0^{n^{-3}} = \frac{2}{\sqrt{n}} \stackrel{n \rightarrow \infty}{\longrightarrow} \infty.
\end{align*}
Da $L^1(0,1)$ ein Banachraum, $\C$ ein normierter Raum, $R_n: L^1 \rightarrow \C: f \rightarrow
\int_0^1 fg_n d\lambda$ eine Folge beschränkter linearer Operatoren mit
\begin{align*}
   \|R_n\| = \|g_n\|_{\infty} = n,
\end{align*}
also
\begin{align*}
  \sup_{n \in \N}\|R_n\| = \infty.
\end{align*}
Nach Satz 4.2.1 gibt es also eine dichte $G_{\delta}$-Menge $M \subseteq L^1(0,1)$
mit
\begin{align*}
  \forall f \in M: \sup_{n \in \N}|\int fg_n d\lambda | = \infty.
\end{align*}
Nun gilt, dass
\begin{align*}
  L^2 \subseteq M^{\complement} = \left(\bigcap_{k \in \N}M_k)^{\complement}\right)
  = \bigcup_{k \in \N}M_k^{\complement}.
\end{align*}
Da für alle $k \in \N: M_k$ dicht, gilt $M_k^{\circ} = \overline{M_k}^{\circ} = \emptyset$
und wir haben die Darstellung
\begin{align*}
  L^2 = L^2 \cap \bigcup_{k \in \N} M_k^{\complement} = \bigcup_{k \in \N}(L^2 \cap M_k^{\complement})
\end{align*}
\item Wie bereits gezeigt, ist die Einbettung
\begin{align*}
  \iota: \bigg\{\begin{matrix}
    L_2 & \rightarrow & L_1 \\
    f & \mapsto & f
  \end{matrix}
\end{align*}
nicht surjektiv. Wäre $L^2$ von 2.Kategorie, dann gäbe dass einen Widerspruch zu
Aufgabe 4/3, welche besagen würde, dass $\iota$ surjektiv ist.
\end{enumerate}

\end{solution}
