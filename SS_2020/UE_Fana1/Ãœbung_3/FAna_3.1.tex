\begin{exercise}
Ein topologischer Raum heißt lokalkompakt, wenn jeder Punkt eine Umgebungsbasis
aus kompakten Mengen besitzt. Zeige die folgende Version des Satzes von Baire:
Sei $(X,\mathcal{T})$ ein lokalkompakter Hausdorff-Raum, und seien $(V_n)_{n \in \N}$,
offene, dichte Teilmengen von $X$. Dann ist auch $\bigcap_{n \in \N} V_n$ dicht in $X$.
\end{exercise}
\begin{solution}
Beweis.
\end{solution}
