\begin{exercise}
Ein topologischer Raum heißt lokalkompakt, wenn jeder Punkt eine Umgebungsbasis
aus kompakten Mengen besitzt. Zeige die folgende Version des Satzes von Baire: \\
Sei $(X,\mathcal{T})$ ein lokalkompakter Hausdorff-Raum, und seien $(V_n)_{n \in \N}$,
offene, dichte Teilmengen von $X$. Dann ist auch $\bigcap_{n \in \N} V_n$ dicht in $X$.
\end{exercise}
\begin{solution}
  Sei $W \in \mathcal{T}.$ Zu zeigen ist $W \cap \bigcap_{n \in \mathbb{N}} V_n \neq \emptyset$: \\
  Dazu definieren wir induktiv eine Mengenfolge $K_n$.
  $V_1$ ist offen und dicht, daher ist $W \cap V_1 \neq \emptyset$ offen, also insbesondere Umgebung eines Punktes $w$. Da $w$ eine Umgebungsbasis aus kompakten Mengen besitzt, können wir eine kompakte Umgebung $K_1$ und eine offene Umgebung $U_1$ von $w$ wählen, sodass
  \begin{align*}
    U_1 \subseteq K_1 \subseteq W \cap V_1.
  \end{align*}

  Seien $K_i, i < n$ bereits definiert. Weil $V_n$ dicht liegt, ist $U_{n-1} \cap V_n$ nichtleer und offen. Insbesondere enthält diese Menge eine kompakte Umgebung $K_n$ und eine offene Umgebung $U_n$ mit der Eigenschaft

  \begin{align}
      U_n \subseteq K_n \subseteq \bigcap_{i < n} U_i
      \subseteq \bigcap_{i < n} K_i \subseteq \bigcap_{i < n} V_i \cap W.
  \end{align}

  $K_1$ ist nach unserer Wahl ein kompakter Teilraum von $X$ und nach Konstruktion gilt $K_n \subseteq K_1$ für alle $n \in \mathbb{N}$.

  Weil $X$ ein $T_2$-Raum ist, sind die kompakten Mengen $K_n$ auch abgeschlossen. Die Mengenfolge hat zudem die endliche Durchschnittseigenschaft, das heißt alle endlichen Schnitte sind nichtleer. Aus der Kompaktheit von $K_1$ und der Abgeschlossenheit der $K_n$ folgt daher
  \begin{align}
      \emptyset \neq \bigcap_{n \in \mathbb{N}} K_n \subseteq \bigcap_{n \in \mathbb{N}} V_n \cap W.
  \end{align}
\end{solution}
