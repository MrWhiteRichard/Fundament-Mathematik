\begin{exercise}
\leavevmode \\
\begin{enumerate}[label = (\roman*)]
  \item Zeige, dass es keine Funktion $f: \R \rightarrow \R$ gibt, die an allen rationalen
  Punkten stetig, aber an allen irrationalen Punkten unstetig ist.
  \item Finde eine
 Funktion $f: \R \rightarrow \R$ die an allen irrationalen Punkten stetig,
 aber an allen rationalen Punkten unstetig ist.
\end{enumerate}
\textit{Hinweis:} Ist die Teilmenge $\Q$ von $\R$ (welche ja dicht liegt)
eine $G_{\delta}$-Menge?
\end{exercise}
\begin{solution}
\leavevmode \\
\begin{enumerate}[label = (\roman*)]
  \item Wir führen einen Widerspruchsbeweis und nehmen dafür an $\Q$ ist eine $G_\delta$-Menge. Wir wissen, dass 
  \begin{align*}
    \Q \quad \textrm{und} \quad \R \setminus \Q \quad \textrm{dicht liegen in} \quad \R.
  \end{align*}
  Weiters erkennen wir, dass 
  \begin{align*}
    \R \setminus \Q = \bigcap_{q \in \Q} \R \setminus \{q\} \quad \textrm{eine $G_\delta$-Menge ist.}
  \end{align*}
  Nach Aufgabe 3 ist dann, dann 
  \begin{align*}
    \emptyset = (\R \setminus \Q) \cap \Q \quad \textrm{eine dichte $G_\delta$-Menge.}
  \end{align*}
  Ein Widerspruch!
  
  Nun wissen wir also, dass $\Q$ keine $G_\delta$-Menge ist. Gäbe es nun eine Funktion $f: \R \to \R$ die an allen rationalen Punkten stetig ist und an allen irrationalen Punkten unstetig ist, dann wäre nach Aufgabe 4 die Menge $\Q$ eine $G_\delta$-Menge, also kann es so eine Funktion nicht geben.
  \item Wir kennen so eine Funktion bereits. 
  \begin{align*}
    f: \R \to \R: x \mapsto 
    \begin{cases}
      \frac{1}{q} &\textrm{, falls} \quad x = \frac{p}{q} \in \Q \quad \textrm{, wobei der Bruch vollständig gekürzt ist} \\
      0 &\textrm{, falls} \quad x \in \R \setminus \Q
    \end{cases}
  \end{align*}
  Wir rechnen nach:
  \begin{enumerate}
    \item[\glqq $x \in \Q$ \grqq] Wir schreiben $x$ als vollständig gekürzten Bruch $x = \frac{p}{q}$ und sehen mit 
    \begin{align*}
      \epsilon := \frac{1}{2q} \quad \textrm{gilt} \quad \forall\delta \in \R^+: \exists y \in \R \setminus \Q \cap U_\delta(x) \quad \textrm{und} \quad \vbraces{f(x) - f(y)} = \vbraces{f(x)} = \frac{1}{q} \geq \frac{1}{2q} = \epsilon
    \end{align*} 
    Also ist $f$ im Punkt $x$ nicht stetig.
    \item[\glqq $x \in \R \setminus \Q$ \grqq] Wir wählen $\epsilon \in \R^+$ beliebig und definieren 
    \begin{align*} 
      \delta := \min\Bbraces{\vbraces{\frac{p}{q} - x} : p \in \Z \land q = \max\pbraces{\{1\} \cup \Bbraces{r \in \N \mid \frac{1}{r} \geq \epsilon}} }.
    \end{align*}
    Dann gilt 
    \begin{align*}
      \forall y \in U_\delta(x): \vbraces{f(y) -f(x)} < \epsilon,
    \end{align*}
    also Stetigkeit im Punkt $x$.
  \end{enumerate}
\end{enumerate}
\end{solution}

