\begin{exercise}
Eine Menge heißt $G_{\delta}$-Menge, wenn sie der abzählbare Durchschnitt offener
Mengen ist. Zeige, dass der Durchschnitt von abzählbar vielen dichten $G_{\delta}$-Mengen
eines vollständigen metrischen Raumes wieder eine dichte $G_{\delta}$-Menge ist.
\end{exercise}
\begin{solution}
Sei $(X,d)$ ein beliebiger vollständiger metrischer Raum und sei für alle $n \in \N: M_n$
eine $G_{\delta}$-Menge, also
\begin{align*}
  M_n = \bigcap_{k \in \N}P_{n_k}
\end{align*}
und seien weiters alle $M_n$ dicht in $X$. Dann sind insbesondere auch alle $P_{n_k}$
offene Mengen, die dicht in $X$ liegen und mit dem Satz von Baire folgt, dass
\begin{align*}
  \bigcap_{n \in \N}M_n = \bigcap_{(n,k) \in \N^2}P_{n_k}
\end{align*}
dicht in $X$ liegt und somit eine dichte $G_{\delta}$-Menge ist.
\end{solution}
