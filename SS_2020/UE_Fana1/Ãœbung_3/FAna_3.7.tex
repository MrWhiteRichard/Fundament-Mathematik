\begin{exercise}
Sei $\Omega$ eine Menge und $X$ ein linearer Raum, dessen Elemente Funktionen von
$\Omega$ nach $\C$ sind, und dessen lineare Operationen durch punktweise Addition
und skalare Multiplikation gegeben sund. Für $\omega \in \Omega$ bezeichne mit
$\chi_{\omega}: X \rightarrow \C$ das Punktauswertungsfunktional
\begin{align*}
  \chi_{\omega}(f) := f(\omega), \qquad f \in X.
\end{align*}
Dann ist $\chi_{\omega}$ linear. Zeige, dass es (bis auf Äquivalenz der Normen)
höchstens eine Norm $||\cdot||$ auf $X$ geben kann, sodass $(X,||\cdot||)$ ein
Banachraum ist und alle Punktauswertungsfunktionale bezüglich $||\cdot||$ stetig sind. \\
\textit{Hinweis:} Wende den Satz vom abgeschlossenen Graphen auf die identische Abbildung an.
\end{exercise}
\begin{solution}
Zwei Normen $\|\cdot \|_1$ und $\|\cdot \|_2$ auf $X$ sind nicht äquivalent, wenn
\begin{align}\label{nequ}
  (\Forall \alpha > 0,\Exists x \in X: \alpha \|x \|_1 > \|x \|_2) \lor (\Forall \beta > 0,\Exists x \in X: \|x \|_2 > \beta \|x \|_1)
\end{align}
\newline

Nehmen wir also an, es gibt zwei Normen $\|\cdot \|_1$ und $\|\cdot \|_2$ auf $X$, die nicht äquivalent sind, sodass $(X,\|\cdot \|_1)$ und $(X,\|\cdot \|_2)$ Banachräume sind und alle Punktauswertungsfunktionale bzgl. $\|\cdot \|_i, i = 1,2$ stetig sind.

Sei o.B.d.A. die zweite Bedingung aus \eqref{nequ} erfüllt. Dann ist die Identität als lineare Abbildung $Id:(X,\|\cdot \|_1) \to (X,\|\cdot \|_2)$ nicht stetig, da
\begin{align*}
  \|Id\| = \sup \left \{\frac{\|x\|_2}{\|x\|_1}: x \in X \setminus \{0\} \right \}
\end{align*}
unbeschränkt ist.

Die Menge $M := \{\chi_\omega | \omega \in \Omega\} \subseteq X'$ ist auf jedenfall punktetrennend, da
\begin{align*}
  \Forall f \in X \setminus \{0\}, \Exists \chi_\omega \in M: \chi_\omega(f) \neq 0 \Leftrightarrow \Forall f \in X \setminus \{0\}, \Exists \omega \in \Omega: f(\omega) \neq 0 .
\end{align*}

Gemäß Korollar 4.4.4 (Kontraposition) folgt daraus, dass es ein $\chi_\omega \in M$ gibt, sodass $\chi_\omega \circ Id$ nicht stetig ist. Allerdings ist

\begin{align*}
  \chi_\omega \circ Id = \chi_\omega: (X,\|\cdot \|_1) \to (\C, \|\cdot \|_\C)
\end{align*}

nach Voraussetzung stetig und somit haben wir einen Widerspruch.

\end{solution}
