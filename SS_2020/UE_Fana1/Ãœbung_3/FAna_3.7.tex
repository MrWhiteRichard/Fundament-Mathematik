\begin{exercise}

Sei $\Omega$ eine Menge und $X$ ein linearer Raum, dessen Elemente Funktionen von $\Omega$ nach $\C$ sind, und dessen lineare Operationen durch punktweise Addition und skalare Multiplikation gegeben sind.
Für $\omega \in \Omega$ bezeichne mit $\chi_{\omega}: X \rightarrow \C$ das Punktauswertungsfunktional

\begin{align*}
  \chi_{\omega}(f) := f(\omega), \qquad f \in X.
\end{align*}

Dann ist $\chi_{\omega}$ linear.
Zeige, dass es (bis auf Äquivalenz der Normen) höchstens eine Norm $\norm{\cdot}$ auf $X$ geben kann, sodass $(X, \norm{\cdot})$ ein Banachraum ist und alle Punktauswertungsfunktionale bezüglich $\norm{\cdot}$ stetig sind. \\

\textit{Hinweis:}
Wende den Satz vom abgeschlossenen Graphen auf die identische Abbildung an.

\end{exercise}

\begin{solution}

Angenommen, $(X, \norm[1]{\cdot})$ und $(X, \norm[2]{\cdot})$ seien Banachräume, mit nicht-äquivalenten Normen, und alle Punktauswertungsfunktionale seien, bezüglich derer, stetig.

\begin{align}
\label{nequ}
  \norm[1]{\cdot} \not \equiv \norm[2]{\cdot}
  \iff
  (
    \Forall \alpha > 0:
    \Exists x \in X:
    \alpha \norm[1]{x} > \norm[2]{x}
  )
  \lor
  (
    \Forall \beta > 0:
    \Exists x \in X:
    \norm[2]{x} > \beta \norm[1]{x}
  )
\end{align}

Sei o.B.d.A. die zweite Bedingung aus \eqref{nequ} erfüllt.
Dann ist die Identität als lineare Abbildung $\id: (X, \norm[1]{\cdot}) \to (X, \norm[2]{\cdot})$ nicht stetig, weil sie unbeschränkt ist, also $\Forall \beta > 0: \Exists x_\beta \in X:$

\begin{align*}
  \norm{\id}
  =
  \sup \Bbraces
  {
    \frac{\norm[2]{x}}{\norm[1]{x}}:
    x \in X \setminus \Bbraces{0}
  }
  \geq
  \frac{\norm[2]{x_\beta}}{\norm[1]{x_\beta}}
  >
  \beta.
\end{align*}

Die Menge $M := \Bbraces{\chi_\omega: \omega \in \Omega} \subseteq X^\prime$ ist auf jeden Fall punktetrennend, weil $\Forall f \in X \setminus \Bbraces{0}:$

\begin{align*}
  (
    \Exists \chi_\omega \in M:
    \chi_\omega(f) \neq 0
  )
  \iff
  (
    \Exists \omega \in \Omega:
    f(\omega) \neq 0
  ).
\end{align*}

Laut Korollar 4.4.4 (Kontraposition), folgt daraus, dass es ein $\chi_\omega \in M$ gibt, sodass $\chi_\omega \circ \id$ nicht stetig ist.

\begin{align*}
  \chi_\omega \circ \id
  =
  \chi_\omega:
  (X, \norm[1]{\cdot}) \to (\C, |\cdot|)
\end{align*}

Letzteres ist, laut Voraussetzung, stetig. Widerspruch!

\end{solution}
