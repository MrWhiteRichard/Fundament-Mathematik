\begin{exercise}
Sei $\Omega$ eine Menge und $X$ ein linearer Raum, dessen Elemente Funktionen von
$\Omega$ nach $\C$ sind, und dessen lineare Operationen durch punktweise Addition
und skalare Multiplikation gegeben sund. Für $\omega \in \Omega$ bezeichne mit
$\chi_{\omega}: X \rightarrow \C$ das Punktauswertungsfunktional
\begin{align*}
  \chi_{\omega}(f) := f(\omega), \qquad f \in X.
\end{align*}
Dann ist $\chi_{\omega}$ linear. Zeige, dass es (bis auf Äquivalenz der Normen)
höchstens eine Norm $||\cdot||$ auf $X$ geben kann, sodass $(X,||\cdot||)$ ein
Banachraum ist und alle Punktauswertungsfunktionale bezüglich $||\cdot||$ stetig sind. \\
\textit{Hinweis:} Wende den Satz vom abgeschlossenen Graphen auf die identische Abbildung an.
\end{exercise}
\begin{solution}
Zwei Normen sind nicht äquivalent, sobald eine der beiden Normen "echt feiner" als die andere ist. Die von der feineren Norm induzierte Topologie ist also auch echt feiner als die der anderen. (Ist noch nicht ganz sauber, muss man wohl noch mit der offiziellen Definition von (Nicht-)Äquivalenz zeigen.)
\newline

Nehmen wir also an, es gibt zwei Normen $\|\cdot \|_1$ und $\|\cdot \|_2$ auf $X$, die nicht äquivalent sind, sodass $(X,\|\cdot \|_1)$ und $(X,\|\cdot \|_2)$ Banachräume sind und alle Punktauswertungsfunktionale bzgl. $\|\cdot \|_i, i = 1,2$ stetig sind.

Die Menge $M := \{\chi_\omega | \omega \in \Omega\} \subseteq X'$ ist auf jedenfall punktetrennend, da
\begin{align*}
  \Forall f \in X \setminus \{0\}, \Exists \chi_\omega \in M: \chi_\omega(f) \neq 0 \Leftrightarrow \Forall f \in X \setminus \{0\}, \Exists \omega \in \Omega: f(\omega) \neq 0 .
\end{align*}

Sei o.B.d.A. $\|\cdot \|_1 < \|\cdot \|_2$. Dann ist die Identität $Id:(X,\|\cdot \|_1) \to (X,\|\cdot \|_2)$ nicht stetig.

Gemäß Korollar 4.4.4 (Kontraposition) folgt daraus, dass es ein $\chi_\omega \in M$ gibt, sodass $\chi_\omega \circ Id$ nicht stetig ist. Allerdings ist

\begin{align*}
  \chi_\omega \circ Id = \chi_\omega: (X,\|\cdot \|_1) \to (\C, \|\cdot \|_\C)
\end{align*}

nach Voraussetzung stetig und somit haben wir einen Widerspruch.

\end{solution}
