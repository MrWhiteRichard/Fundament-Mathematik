\begin{exercise}

Sei $(X, \norm{\cdot})$ ein Banachraum.
Zeige, dass die Mächtigkeit einer algebraischen Basis von $X$ als $\C$-Vektorraum entweder endlich oder überabzählbar ist. \\

\textit{Hinweis:}
Zeige, dass ein linearer Teilraum $Y \subsetneq X$ keinen inneren Punkt hat.

\end{exercise}

\begin{solution}

$y \in Y$ ist ein innerer Punkt von $Y$, genau dann, wenn $y \in Y^\circ$.
$Y$ hat also keine inneren Punkte, genau dann, wenn $Y^\circ = \emptyset$. \\

Der Anschauung nach, ist die Aussage im \textit{Hinweis} trivial.
Dazu, betrachte $Y := \R^2 \times 0$ und $X := \R^3$.
Offensichtlich liegen keine $\epsilon$-Kugeln von $y \in Y$ wieder in $Y$, weil die letzte Komponente von $y$ immer $0$ ist.
Diese Argumentation gilt es ins allgemeine Setting zu übertragen. \\

Sei $(b_i)_{i \in J}$ eine Basis von $Y$ und $(b_i)_{i \in I}$ die auf $X$ fortgesetzte.
Weil $Y < X$, muss auch $J \subsetneq I$, also $\Exists i_0 \in I \setminus J$. \\

Angenommen, $Y^\circ \neq \emptyset$, dann $\Exists p \in Y^\circ$.
$Y^\circ$ ist als Vereinigung aller offenen Teilmengen von $Y$ selbst wieder offen.
Daher, $\Exists \epsilon > 0: U_\epsilon(p) \subseteq Y^\circ$.

$p \in Y^\circ \subseteq Y$ besitzt aber auch, eine eindeutige Koordinatendarstellung $(p_i)$ bezüglich der Basis $(b_i)$.

\begin{align*}
  p = \sum_{i \in I} p_i b_i
    = \sum_{i \in J} p_i b_i
\end{align*}

Daher, muss $\Forall i \in I \setminus J: p_i = 0$, also insbesondere auch für das obere $i_0$.
In unserer $U_\epsilon(p) \subseteq Y^\circ$ finden wir aber ein Element $p^\prime \notin Y$, mit $i_0$-ter Komponente von $p^\prime \neq 0$.
Widerspruch!

\begin{align*}
  \| \underbrace{p + \frac{\epsilon}{2 \norm{b_{i_0}}} b_{i_0}}_{=: p^\prime} - p \|
  =
  \frac{\epsilon}{2} \frac{1}{\norm{b_{i_0}}} \norm{b_{i_0}}
  <
  \epsilon
\end{align*}

Damit, zeigen wir die ursprüngliche Aussage.
Wir verwenden zudem ein Korollar (Bemerkung 4.1.3) vom Satz von Baire. \\

Sei $I$ unendlich.
Angenommen, $|I| = \aleph_0$, also o.B.d.A. $I = \N$.
Betrachte die aufsteigende Unterraumkette $(A_n)_{n \in \N}$ von $X$.

\begin{align*}
  A_n := \Span \Bbraces{b_1, \ldots, b_n} < X
\end{align*}
Da laut Satz 2.2.1 endlich-dimensionale lineare Unterräume abgeschlossen sind, ist
$(A_n)_{n \in \N}$ eine Folge abgeschlossener Mengen mit, laut \textit{Hinweis}, komponentenweise leerem Inneren.
Laut Bemerkung 4.1.3, hat auch deren Vereinigung leeres Inneres.

\begin{align*}
  X = \bigcup_{n \in N} A_n
  \implies
  \emptyset = X^\circ = X
\end{align*}

Letztere Gleichheit gilt, weil $X$ in sich selbst wieder offen ist.
$\emptyset$ ist aber kein Vektorraum, also erst recht nicht $\aleph_0$-dimensional.
Widerspruch!

\end{solution}
