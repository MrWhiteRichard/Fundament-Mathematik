\begin{exercise}
Seien $X,Y,Z$ normierte Räume, und seien $A_n,A \in \mathcal{B}(Y,Z),B_n,B\in \mathcal{B}(X,Y)$.
Dann gilt:
\begin{itemize}
  \item [a)] Ist $A_n \stackrel{s}{\rightarrow} A, \sup_{n\in\mathbb{N}} ||A_n|| < \infty$,
  und $B_n \stackrel{s}{\rightarrow} B$, so folgt $A_nB_n \stackrel{s}{\rightarrow} AB$.
  \item [b)] Ist $A_n \stackrel{w}{\rightarrow} A, \sup_{n\in\mathbb{N}} ||A_n|| < \infty$,
  und $B_n \stackrel{s}{\rightarrow} B$, so folgt $A_nB_n \stackrel{w}{\rightarrow} AB$.
  \item [c)] Ist $A_n \stackrel{s}{\rightarrow} A$, so folgt $A_nB \stackrel{s}{\rightarrow} AB$.
  Ist $B_n \stackrel{s}{\rightarrow} B$, so folgt $AB_n \stackrel{s}{\rightarrow} AB$.
  \item [d)] Ist $A_n \stackrel{w}{\rightarrow} A$, so folgt $A_nB \stackrel{w}{\rightarrow} AB$.
  Ist $B_n \stackrel{w}{\rightarrow} B$, so folgt $AB_n \stackrel{w}{\rightarrow} AB$.
\end{itemize}
\end{exercise}

\begin{solution}
\leavevmode \\
  Wir wissen:
\begin{itemize}
    \item $A_n \xrightarrow{s} A \Leftrightarrow \forall x \in Y:~ A_nx \xrightarrow{\mathbb{ \| . \|}_z} Ax,$
    \item $A_n \xrightarrow{w} A \Leftrightarrow \forall x \in Y~ \forall \varphi \in Z':~ \varphi(A_nx) \xrightarrow{\mathbb{C}} \varphi(Ax)$.
\end{itemize}

\begin{itemize}
    \item [a)] Seien $x \in Y, \epsilon > 0$ beliebig. Für hinreichend großes $n$ gilt
    \begin{align*}
        \| A_n B_n x - ABx \| \leq \| A_n B_n x - A_n Bx \| + \| A_n Bx - ABx \| \leq \\
        \underbrace{(\sup{\| A_n \|}) \| B_n x - Bx \|}_{\substack{\leq
        \frac{\epsilon}{2}
        \text{~wegen~} B_n \xrightarrow{s} B}} +
        \underbrace{\| A_n(Bx) - A(Bx) \|}_{\substack{\leq \frac{\epsilon}{2} \text{~wegen~} A_n \xrightarrow{s} A}}
        \leq \epsilon.
    \end{align*}

    \item [b)] Seien $\epsilon > 0, \varphi \in Z', x \in Y$ beliebig. Für hinreichend großes $n$ gilt
    \begin{align*}
        \| \varphi(A_n B_n x) - \varphi(ABx) \|_2
        \leq \| \varphi(A_n B_n x) - \varphi(A_nBx) \|
        +  \underbrace{\| \varphi(A_n (Bx)) - \varphi(A (Bx)) \|}_{\substack{\leq \frac{\epsilon}{2} \text{~wegen~} A_n \xrightarrow{w} A}} \leq \\
        \| \varphi \|~ \| A_n (B_n x - B x) \| + \frac{\epsilon}{2} \leq
        \underbrace{\| \varphi \|~ (\sup{\| A_n \|})
        \| B_n x - Bx \|}_{\substack{\leq \frac{\epsilon}{2} \text{~wegen~} B_n \xrightarrow{s} B}} + \frac{\epsilon}{2} \leq
        \epsilon.
    \end{align*}

    \item [c)]  Die starke Konvergenz von $A_nB$ gegen $AB$ folgt aus der starken Konvergenz von $A_n$ gegen $A$:
    \begin{center} $\| A_n Bx - ABx \|_z = \| A_n(Bx) - A(Bx) \|_z \xrightarrow{n \rightarrow \infty} 0.$ \end{center}
    Gilt hingegen $B_n \xrightarrow{s} B$, dann auch
    \begin{center}
     $\| AB_n x - ABx \|_z = \| A(B_n x - Bx) \| \leq
       \| A \|~ \| B_n x - Bx \| \xrightarrow{n \rightarrow \infty} 0.$
    \end{center}

    \item [d)] Konvergiert $A_n$ schwach gegen $A$, so gilt
    \begin{center}$
    \| \varphi(A_n Bx) - \varphi(ABx) \| = \| \varphi(A_n(Bx)) - \varphi(A(Bx)) \| \xrightarrow{n \rightarrow \infty} 0.
    $\end{center}

    Konvergiert $B_n$ schwach gegen $B$, so gilt
    \begin{center}$
    \| \varphi(AB_n x) - \varphi(ABx) \| = \| \underbrace{(\varphi \circ A)}_\substack{\in Y'}} (B_nx - Bx) \| \xrightarrow{n \rightarrow \infty} 0. $\end{center}
\end{itemize}


\end{solution}
