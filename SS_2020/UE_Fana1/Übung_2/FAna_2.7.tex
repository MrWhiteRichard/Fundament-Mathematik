\begin{exercise}
Kann man in (a) und (b) des vorigen Beispiels die Vorraussetzung
$\sup_{n \in \mathbb{N}} ||A_n|| < \infty$ weglassen?
Falls ja, beweise dies. Falls nein, finde ein Gegenbeispiel.
\end{exercise}
\begin{solution}

Wir zeigen an einem Beispiel, dass man die Vorraussetzung nicht weglassen kann.
Dabei machen wir zuerst die allgemeine Feststellung bezüglich des Zusammenhangs der
Operatortopologien:

\begin{align*}
  \mathcal{T}_w \subseteq \mathcal{T}_s \subseteq \mathcal{T}_{||\cdot||}
\end{align*}

Also ist die schwache Operatortopologie gröber als die starke und diese wiederum
gröber als die Norminduzierte. Daher brauchen wir nur ein Beispiel finden, bei dem
die Voraussetzungen von (a), ohne die aus der Angabe, gelten und wenn wir daraus folgern können, das die Conlusio von (b) falsch ist haben wir gezeigt, dass die Vorraussetzung sowohl
für (a) als auch (b) notwendig ist.

Für die Folge $(A_n)_{n \ in \N}$ gilt:

\begin{align*}
  A_n \stackrel{s}{\rightarrow} A
  \Leftrightarrow
  \forall y \in Y: A_n y \stackrel{||\cdot||_Z}{\rightarrow} Ay
\end{align*}

Wir wählen unsere Räume wie folgt: $Y = (L^1 (0,1), ||\cdot||_1) ; Z=X=\R$. Der
Darstellungssatz von Riesz gibt uns eine Darstellung der $A_n$:

\begin{align*}
  \forall n \in \N: \exists h_n \in L^{\infty}(0,1):
  \forall g \in L^1 (0,1): A_n(g) = \int_{(0,1)}h_n g d\lambda \\
  \text{bzw.} A(g)= \int_{(0,1)} hg d\lambda
\end{align*}

Wir wählen:

\begin{align*}
  h_n (x):= \begin{cases}
    n, & \text{falls\,} x \in (0,\frac{1}{n^2}) \\
    \frac{1}{n} & \text{sonst}
  \end{cases}
\end{align*}

Dann gilt für die Norm der $A_n$ (nach Kusolitsch 13.40):

\begin{align*}
  \norm[]{A_n} = \norm[\infty]{h_n} = n
\end{align*}

Für $g \in L^1(0,1)$ bel. gilt (KU 13.21: $L^1(0,1) \subseteq L^{\infty}(0,1)$):
$\norm[\infty]{g} < \infty$. Damit

\begin{align*}
  \big| \int_{(0,1)}h_n(s) g(s)d\lambda(s) \big|
  \leq \norm[\infty]{g} \int_{(0,1)} h_n(s) d\lambda(s)
  =\norm[\infty]{g}
  \bigg( \int_0^\frac{1}{n^2} n d\lambda + \int_\frac{1}{n^2}^1 \frac{1}{n} d\lambda
  \bigg)
  = \norm[\infty]{g} (\frac{1}{n}+\frac{1}{n}(1-\frac{1}{n}))
  \xrightarrow{n \rightarrow \infty} 0
\end{align*}

Daraus folgern wir, dass $A=0$. Definieren wir nun
$B_n: \R \rightarrow L^1(0,1): x \mapsto xh_n$, diese Funktionen sind nun linear und
beschränkt. Für $x \in \R$ bel. berechnen wir nun:

\begin{align*}
  \int_{(0,1)} |xh_n(s)|d\lambda(s) = |x|
  \bigg( \int_0^\frac{1}{n^2} n d\lambda + \int_\frac{1}{n^2}^1 \frac{1}{n} d\lambda
  \bigg)
  \xrightarrow{n \rightarrow \infty} 0
\end{align*}
Also konvergieren $A_n$ und $B_n$ schwach gegen 0. Wiederum für $x \in \R$ gilt also
$ABx = 0$. Es gilt aber auch:

\begin{align*}
  A_n B_n x =
  \int_{(0,1)} x h_n(s) h_n(s) d\lambda(s) = x
  \bigg(\int_0^\frac{1}{n^2} n^2 d\lambda + \int_\frac{1}{n^2}^1 \frac{1}{n^2}
  \bigg)
  = x(1+\frac{1}{n^2}(1-\frac{1}{n^2}))
  \xrightarrow{n \rightarrow \infty} x
\end{align*}

Wir erinnern uns an eine Beschreibung der Konvergenz bezlüglich der schwachen Operatortopologie:

\begin{align*}
  A_n B_n \stackrel{w}{\rightarrow} AB \Leftrightarrow
  \forall x \in \R, \forall \phi in \R':
  \phi(A_n B_n x) \stackrel{\C}{\rightarrow} \phi(ABx)
\end{align*}

Wählen wir nun $\phi=id_\R$ sehen wir, dass $A_n B_n$ nicht schwach gegen $AB$ konvergiert.
\end{solution}
