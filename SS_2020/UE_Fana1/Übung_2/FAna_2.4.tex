\begin{exercise}
Man betrachte den Banachraum $L^{\infty}(0,1)$, und erinnere sich, dass
$L^{\infty}(0,1) = L^1(0,1)^{\prime}$. Damit haben wir auf $L^{\infty}(0,1)$ drei
in natürlicher Weise gegebene Topologien: die Normtopologie, die $w$-Topologie,
und die $w^*$-Topologie.
\begin{itemize}
  \item [a)] Der Funktionenraum $C([0,1])$ ist ein Teilraum von $L^{\infty}(0,1)$.
  In welcher/n der obigen Topologien ist er abgeschlossen, und in welcher/n nicht?
  \item [b)] Zeige, dass die $w^*$-Topologie verschieden von der $w$-Topologie ist.
  \item [c)] Ein Banachraum heißt \textit{reflexiv}, wenn die kanonische Abbildung
  $\iota: X \rightarrow X^{\primeprime}$ surjektiv ist. Zeige, dass $L^1(0,1)$ nicht
  reflexiv ist.
\end{itemize}

\end{exercise}

\begin{solution}
Beweis.


\end{solution}
