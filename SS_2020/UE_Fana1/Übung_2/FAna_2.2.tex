\begin{exercise}

Ein normierter Raum $Y$ heißt strikt konvex, wenn gilt

\begin{align*}
  x, y \in Y,
  \norm{x} = \norm{y} = 1,
  \norm{\frac{x+y}{2}} = 1
  \implies
  x = y
\end{align*}

Sei nun $X$ ein normierter Raum, $M$ ein linearer Teilraum von $X$, und $f: M \to \C$ ein beschränktes lineares Funktional.
Zeige:
Ist $X\prime$ strikt konvex, so hat $f$ genau eine normerhaltende Fortsetzung.

\end{exercise}

\begin{solution}

Trivial!

\end{solution}
