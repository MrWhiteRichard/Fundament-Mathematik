\begin{exercise}
Sei $(X,||\cdot||)$ ein normierter Raum. Da die kanonische Einbettung $\iota: X
\rightarrow X^{\primeprime}$ den Raum $X$ isometrisch und bijektiv auf $\iota(X)$
abbildet, ist sie ein Homöomorphismus von $(X,||\cdot||_X)$ auf \\
($\iota(X),||\cdot||_{X^{\primeprime}}|_{\iota(X)}$). \\
Zeige auf zwei Arten, dass $\iota$ auch ein Homöomorphismus von ($X,\sigma(X,X^{\prime})$)
auf ($\iota(X),\sigma(X,X^{\primeprime})|_{\iota(X)}$) ist (hier schreiben wir
$\sigma(X^{\primeprime},X^{\prime})$) für $\sigma(X^{\primeprime},\iota_1(X^{\prime}))$,
wobei $\iota_1: X^{\prime} \rightarrow X^{\primeprimeprime}$ die kanonische Einbettung ist).
Nämlich:
\begin{itemize}
  \item [(1)] Betrachte explizit Nullumgebungsbasen der entsprechenden Topologien.
  \item [(2)] Argumentiere mittels der Transitivitätseigenschaft initialer Topologien.
\end{itemize}
\end{exercise}

\begin{solution}

\phantom{}

\begin{enumerate}[label = (\arabic*)]

  \item
  Sei $V \subseteq \iota(X)$ ein beliebiges Element aus der Nullumgebungsbasis von

  \begin{align*}
    (\iota(X),\sigma(X^{\primeprime},\iota_1(X^{\prime}))).
  \end{align*}

  Wir wissen, dass folgende Menge eine Subbasis von $\mathcal{T}$ ist, deren Basiselemente alle die $0$ enthalten.

  \begin{align*}
    \bigcup_{i \in I}f_i^{-1}(\mathcal{T}_{\C})
  \end{align*}

  Insbesondere, liegt damit eine Nullumgebungsbasis vor.
  Wir finden also folgende Darstellung.

  \begin{align*}
    \exists f_1,\dots,f_n \in \iota_1(X^{\prime}), \exists V_1,\dots,V_n \in \mathcal{U}^{\C}(0):
    V = \bigcap_{i=1}^n f_i^{-1}(V_i).
  \end{align*}

  Wir zeigen zuerst, dass $\iota$ stetig ist.
  Wir erinnern uns, dass $f$ stetig im Punkt $x$ ist, genau dann, wenn

  \begin{align*}
    \Forall V \in \mathfrak{U}(f(x)):
    \Exists U \in \mathfrak{U}(x):
    f(U) \subseteq V.
  \end{align*}

  Dabei, können wir uns von $\mathfrak{U}$ auch auf die jeweiligen Umgebungsbasen einschränken.
  Mit unserer Darstellung der Nullumgebung (bzw. Nullumgebungsbasiselement) $U$, läuft das auf Folgendes hinaus.

  \begin{align*}
    \Exists g_1, \dots, g_m \in X^{\prime}:
    \exists W_1, \dots, W_m \in \mathfrak{U}^\C(0):
    \iota\left(\bigcap_{j=1}^m g_j^{-1}(W_j)\right) \subseteq V
  \end{align*}

  Um das einzusehen, sei $k \in \{1,\dots,n\}$ beliebig.
  Nun, wählen wir $g_k := \iota_1^{-1}(f_k) \in X^{\prime}$.
  Wir erinnern uns, dass

  \begin{align*}
    f_k^{-1}(V_k)
    \in
    \sigma(X^{\primeprime}, X^{\prime})
    \subseteq
    \mathcal{T}_{\norm[X^\primeprime|_{\iota(X)}]{\cdot}}.
    % (\subseteq \iota(X)).
  \end{align*}

  Weil $\iota: (X,||\cdot||_X) \rightarrow (\iota(X),||\cdot||_{X^{\primeprime}|_{\iota(X)}})$
  stetig ist gilt tatsächlich

  \begin{align*}
    \Exists S_k \in \mathfrak{U}^{(X,\norm{\cdot})}(0):
    \iota(S_k) \subseteq f_k^{-1}(V_k).
  \end{align*}

  Laut Beispiel 5.3.10, gilt $\mathcal{T}_w \subseteq \mathcal{T}_s \subseteq \mathcal{T}_{\norm{\cdot}}$.
  Weiters, war $g_k \in X^\prime$. Damit, $\Exists W_k \in \mathfrak{U}^\C(0): g_k^{-1}(W_k) \subseteq S_k$.

  Damit folgt
  \begin{align*}
    \iota((\iota_1^{-1}(f_k))^{-1}(W_k)) \subseteq \iota(S_k) \subseteq f_k^{-1}(V_k)
    \implies \iota\left(\bigcap_{i=1}^n (\iota_1^{-1}(f_i))^{-1}(W_i)\right) \subseteq \bigcap_{i = 1}^nf_i^{-1}(V_i).
  \end{align*}
  Daraus folgt, dass $\iota$ stetig im Punkt $0$ und linear, also ist $\iota$ stetig. \\
  Sei nun umgekehrt $\bigcap_{i=1}^n f_i^{-1}(W_i)$ mit $W_i \in \mathcal{U}^{\C}(0)$
  und $g_i \in X^{\prime}$ gegeben, $k \in \{1,\dots,n\}$ beliebig.
  \begin{align*}
    f_k := \iota_1(g_k) \in X^{\primeprime},
  \end{align*}
  wegen $g_i^{-1}(W_i) \in \sigma(X,X^{\prime}) \subseteq \mathcal{T}_{||\cdot||}$
  ist $g_k^{-1}(W_k) \in \mathcal{T}_{||\cdot||}$ und weil $\iota^{-1}$ ein
  Homöomorphismus ist, gibt es ein
  $S_k \in \mathcal{U}^{(\iota(X), ||\cdot||_{X^{\primeprime}|_{\iota(X)}})}$
  mit $\iota^{-1}(S_k) \subseteq g_k^{-1}(W_k)$, wegen $f_k \in X^{\primeprimeprime}$
  gibt es $V_k \in \mathcal{U}^{\C}(0)$ mit
  \begin{align*}
    f_k^{-1}(V_k) \subseteq S_k \implies \iota^{-1}(f_k^{-1}(V_k)) \subseteq g_k^{-1}(V_k)
  \end{align*}
  \item
\end{enumerate}


\end{solution}
