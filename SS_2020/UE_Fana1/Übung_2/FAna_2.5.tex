\begin{exercise}
Sei $(X,||\cdot||)$ ein normierter Raum. Da die kanonische Einbettung $\iota: X
\rightarrow X^{\primeprime}$ den Raum $X$ isometrisch und bijektiv auf $\iota(X)$
abbildet, ist sie ein Homöomorphismus von $(X,||\cdot||_X)$ auf \\
($\iota(X),||\cdot||_{X^{\primeprime}}|_{\iota(X)}$). \\
Zeige auf zwei Arten, dass $\iota$ auch ein Homöomorphismus von ($X,\sigma(X,X^{\prime})$)
auf ($\iota(X),\sigma(X,X^{\primeprime})|_{\iota(X)}$) ist (hier schreiben wir
$\sigma(X^{\primeprime},X^{\prime})$) für $\sigma(X^{\primeprime},\iota_1(X^{\prime}))$,
wobei $\iota_1: X^{\prime} \rightarrow X^{\primeprimeprime}$ die kanonische Einbettung ist).
Nämlich:
\begin{itemize}
  \item [(1)] Betrachte explizit Nullumgebungsbasen der entsprechenden Topologien.
  \item [(2)] Argumentiere mittels der Transitivitätseigenschaft initialer Topologien.
\end{itemize}
\end{exercise}

\begin{solution}
Beweis.


\end{solution}
