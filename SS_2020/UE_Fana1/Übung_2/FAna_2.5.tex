\begin{exercise}
Sei $(X,||\cdot||)$ ein normierter Raum. Da die kanonische Einbettung $\iota: X
\rightarrow X^{\primeprime}$ den Raum $X$ isometrisch und bijektiv auf $\iota(X)$
abbildet, ist sie ein Homöomorphismus von $(X,||\cdot||_X)$ auf \\
($\iota(X),||\cdot||_{X^{\primeprime}}|_{\iota(X)}$). \\
Zeige auf zwei Arten, dass $\iota$ auch ein Homöomorphismus von ($X,\sigma(X,X^{\prime})$)
auf ($\iota(X),\sigma(X,X^{\primeprime})|_{\iota(X)}$) ist (hier schreiben wir
$\sigma(X^{\primeprime},X^{\prime})$) für $\sigma(X^{\primeprime},\iota_1(X^{\prime}))$,
wobei $\iota_1: X^{\prime} \rightarrow X^{\primeprimeprime}$ die kanonische Einbettung ist).
Nämlich:
\begin{itemize}
  \item [(1)] Betrachte explizit Nullumgebungsbasen der entsprechenden Topologien.
  \item [(2)] Argumentiere mittels der Transitivitätseigenschaft initialer Topologien.
\end{itemize}
\end{exercise}

\begin{solution}
\begin{enumerate}[label = (\arabic*)]
  \item Sei $V \subseteq \iota(X)$ ein beliebiges Element aus der Nullumgebungsbasis von
  \begin{align*}
    (\iota(X),\sigma(X^{\primeprime},\iota_1(X^{\prime}))).
  \end{align*}
  Wir wissen, dass
  \begin{align*}
    \bigcup_{i \in I}f_i^{-1}(\mathcal{T}_{\C})
  \end{align*}
  eine Subbasis von $\mathcal{T}$ ist und es folgt
  \begin{align*}
    \exists f_1,\dots,f_n \in \iota_1(X^{\prime}), \exists V_1,\dots,V_n \in \mathcal{U}^{\C}(0):
    V = \bigcap_{i=1}^n f_i^{-1}(V_i).
  \end{align*}
  Wir wollen zeigen, dass
  \begin{align*}
    \exists g_1,\dots,g_m \in X^{\prime}: \exists W_1,\dots,W_m \in \mathcal{U}^{\C}(0):
    \iota\left(\bigcap_{j=1}^m g_j^{-1}(w_j)\right) \subseteq V
  \end{align*}
  Sei $k \in \{1,\dots,n\}$ beliebig.
  \begin{align*}
    g_k &:= \iota_1^{-1} \in X^{\prime} \\
    \mathcal{T}_{||\cdot||_{X^{\primeprime}|_{\iota(X)}}} &\supseteq \sigma(X^{\primeprime},X^{\prime}) \ni f_k^{-1}(V_k)
    (\subseteq \iota(X)).
  \end{align*}
  Es folgt also
  \begin{align*}
    f_k^{-1}(V_k) \in \mathcal{T}_{||\cdot||_{X^{\primeprime}|_{\iota(X)}}}
  \end{align*}
  und weil $\iota: (X,||\cdot||_X) \rightarrow (\iota(X),||\cdot||_{X^{\primeprime}|_{\iota(X)}})$
  stetig ist gilt
  \begin{align*}
    \exists S_k \in \mathcal{U}^{(X,||\cdot||)}(0): \iota(S_k) \subseteq f_k^{-1}(V_k),
  \end{align*}
  weiters
  \begin{align*}
    \exists W_k \in \mathcal{U}^{\C}(0) ~\text{mit}~ (\iota_1^{-1}(f_k))^{-1}(W_k) \subseteq S_k,
  \end{align*}
  weil $\iota_1^{-1}(f_k) \in (X,||\cdot||)^{\prime}$. Damit folgt
  \begin{align*}
    \iota((\iota_1^{-1}(f_k))^{-1}(W_k)) \subseteq \iota(S_k) \subseteq f_k^{-1}(V_k)
    \implies \iota\left(\bigcap_{i=1}^n (\iota_1^{-1}(f_i))^{-1}(W_i)\right) \subseteq \bigcap_{i = 1}^nf_i^{-1}(V_i).
  \end{align*}
  Daraus folgt, dass $\iota$ stetig im Punkt $0$ und linear, also ist $\iota$ stetig. \\
  Sei nun umgekehrt $\bigcap_{i=1}^n f_i^{-1}(W_i)$ mit $W_i \in \mathcal{U}^{\C}(0)$
  und $g_i \in X^{\prime}$ gegeben, $k \in \{1,\dots,n\}$ beliebig.
  \begin{align*}
    f_k := \iota_1(g_k) \in X^{\primeprime},
  \end{align*}
  wegen $g_i^{-1}(W_i) \in \sigma(X,X^{\prime}) \subseteq \mathcal{T}_{||\cdot||}$
  ist $g_k^{-1}(W_k) \in \mathcal{T}_{||\cdot||}$ und weil $\iota^{-1}$ ein
  Homöomorphismus ist, gibt es ein
  $S_k \in \mathcal{U}^{(\iota(X), ||\cdot||_{X^{\primeprime}|_{\iota(X)}})}$
  mit $\iota^{-1}(S_k) \subseteq g_k^{-1}(W_k)$, wegen $f_k \in X^{\primeprimeprime}$
  gibt es $V_k \in \mathcal{U}^{\C}(0)$ mit
  \begin{align*}
    f_k^{-1}(V_k) \subseteq S_k \implies \iota^{-1}(f_k^{-1}(V_k)) \subseteq g_k^{-1}(V_k)
  \end{align*}
  \item
\end{enumerate}


\end{solution}
