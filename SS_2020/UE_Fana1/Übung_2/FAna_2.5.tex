\begin{exercise}
Sei $(X,||\cdot||)$ ein normierter Raum. Da die kanonische Einbettung $\iota: X
\rightarrow X^{\primeprime}$ den Raum $X$ isometrisch und bijektiv auf $\iota(X)$
abbildet, ist sie ein Homöomorphismus von $(X,||\cdot||_X)$ auf \\
($\iota(X),||\cdot||_{X^{\primeprime}}|_{\iota(X)}$). \\
Zeige auf zwei Arten, dass $\iota$ auch ein Homöomorphismus von ($X,\sigma(X,X^{\prime})$)
auf ($\iota(X),\sigma(X^{\primeprime},X^{\prime})|_{\iota(X)}$) ist (hier schreiben wir
$\sigma(X^{\primeprime},X^{\prime})$) für $\sigma(X^{\primeprime},\iota_1(X^{\prime}))$,
wobei $\iota_1: X^{\prime} \rightarrow X^{\primeprimeprime}$ die kanonische Einbettung ist).
Nämlich:
\begin{itemize}
  \item [(1)] Betrachte explizit Nullumgebungsbasen der entsprechenden Topologien.
  \item [(2)] Argumentiere mittels der Transitivitätseigenschaft initialer Topologien.
\end{itemize}
\end{exercise}

\begin{solution}

\phantom{}

\begin{enumerate}[label = (\arabic*)]

  \item
  Wir erinnern uns, dass $f$ stetig im Punkt $x$ ist, genau dann, wenn

  \begin{align*}
    \Forall V \in \mathfrak{U}(f(x)):
    \Exists U \in \mathfrak{U}(x):
    f(U) \subseteq V.
  \end{align*}

  Sei $U \subseteq X$ ein beliebiges Element aus der Nullumgebungsbasis von

  \begin{align*}
    (X,\sigma(X,X^{\prime})).
  \end{align*}

  Gemäß Seite 81, (5.3.1) lässt sich $U$ somit darstellen als

  \begin{align*}
    \{x \in X: \vbraces{f_{j}(x)}<\epsilon_j, j=1,...,m\}
  \end{align*}

  für $m \in \N,f_1,...,f_m \in X^{\prime}$ und $\epsilon_1,...,\epsilon_m > 0$.
  Also gilt

  \begin{align*}
    \iota(U) =& \{\iota(x) \in \iota(X): \vbraces{f_{j}(x)}<\epsilon_j, j=1,...,m\} = \{\iota(x) \in \iota(X): \vbraces{\iota(x)(f_{j})}<\epsilon_j, j=1,...,m\} \\
    =& \{\iota(x) \in \iota(X): |\underbrace{\iota_1 (f_{j})}_{\in \iota_1 (X^{\prime})}(\iota(x))|<\epsilon_j, j=1,...,m\}
  \end{align*}

  und somit ist $V := \iota(U) \subset \iota(X)$ eine Menge aus der Nullumgebungsbasis von $(\iota(X),\sigma(X^{\primeprime},\iota_1(X^{\prime})))$ und es gilt
  \begin{align*}
    \iota^{-1}(V) = U,
  \end{align*}
  also ist $\iota^{-1}$ im Punkt $0$ stetig und aufgrund Proposition 2.1.11
  insgesamt stetig. \\
  Sei umgekehrt $V \subseteq \iota(X)$ ein beliebiges Element aus der Nullumgebungsbasis von

  \begin{align*}
    (\iota(X),\sigma(X^{\primeprime},\iota_1(X^{\prime}))).
  \end{align*}

  nun wollen wir also zeigen, dass auch $\iota^{-1}(V)$ aus der Umgebungsbasis von$(X,\sigma(X,X^{\prime}))$ ist.

  Auch $V$ lässt sich darstellen als

  \begin{align*}
    \{\iota(x) \in \iota(X): \vbraces{w_{j}(\iota(x))}<\epsilon_j, j=1,...,m\}
  \end{align*}

  mit $m \in \N,w_1,...,w_m \in \iota_1(X^{\prime})$ (d.h. $\Forall j = 1,...,m \exists y_j \in X^{\prime} : w_j = \iota_1(y_j))$ und $\epsilon_1,...,\epsilon_m > 0$.

  Also gilt

  \begin{align*}
    U := \iota^{-1}(V) =& \{x \in X: \vbraces{w_{j}(\iota(x))}<\epsilon_j, j=1,...,m\} = \{x \in X: \vbraces{\iota_1(y_{j})(\iota(x))}<\epsilon_j, j=1,...,m\} \\
    =& \{x \in X: \vbraces{\iota(x)(y_j)}<\epsilon_j, j=1,...,m\} = \{x \in X: \vbraces{y_j(x)}<\epsilon_j, j=1,...,m\}
  \end{align*}

  und wir haben mit $U$ eine Nullumgebung gefunden mit $\iota(U) = V$ und damit ist auch $\iota$ stetig.

  ******************

  Wir wissen, dass folgende Menge eine Subbasis von $\mathcal{T}$ ist, deren Basiselemente alle die $0$ enthalten.

  \begin{align*}
    \bigcup_{i \in I}f_i^{-1}(\mathcal{T}_{\C})
  \end{align*}

  Insbesondere, liegt damit eine Nullumgebungsbasis vor.
  Wir finden also folgende Darstellung.

  \begin{align*}
    \exists f_1,\dots,f_n \in \iota_1(X^{\prime}), \exists V_1,\dots,V_n \in \mathcal{U}^{\C}(0):
    V = \bigcap_{i=1}^n f_i^{-1}(V_i).
  \end{align*}

  Wir zeigen zuerst, dass $\iota$ stetig ist.
  Wir erinnern uns, dass $f$ stetig im Punkt $x$ ist, genau dann, wenn

  \begin{align*}
    \Forall V \in \mathfrak{U}(f(x)):
    \Exists U \in \mathfrak{U}(x):
    f(U) \subseteq V.
  \end{align*}

  Dabei, können wir uns von $\mathfrak{U}$ auch auf die jeweiligen Umgebungsbasen einschränken.
  Mit unserer Darstellung der Nullumgebung (bzw. Nullumgebungsbasiselement) $U$, läuft das auf Folgendes hinaus.

  \begin{align*}
    \Exists g_1, \dots, g_m \in X^{\prime}:
    \exists W_1, \dots, W_m \in \mathfrak{U}^\C(0):
    \iota\left(\bigcap_{j=1}^m g_j^{-1}(W_j)\right) \subseteq V
  \end{align*}

  Wir definieren 
  \begin{align*}
    g_k := f_k \circ \iota \in (X,\norm[X]{\cdot})^\prime \quad \textrm{und} \quad W_k := V_k.
  \end{align*}
  Die Funktion $g_k$ ist stetig und linear, weil $\iota: (X, \norm[X]{\cdot}) \to (X, \norm[X^\primeprime]{\cdot})$ und $f_k: (X, \norm[X^\primeprime]{\cdot}) \to \C$ beide stetig und linear sind. 
  Nun gilt 
  \begin{align*}
    y \in \iota(g_k^{-1}(W_k)) \Rightarrow \exists x \in g_k^{-1}(W_k): y = \iota(x).
  \end{align*}
  Wegen $x \in g_k^{-1}(W_k)$ gilt weiter
  \begin{align*}
    g_k(x) = f_k(\iota(x)) \in W_k = V_k \Rightarrow y = \iota(x) \in f_k^{-1}(V_k) 
  \end{align*}
  Es gilt also $\iota(g_k^{-1}(W_k)) \subseteq f_k^{-1}(V_k)$. Nun schneiden wir noch
  \begin{align*}
    \iota\pbraces{\bigcap_{i = 1}^n g_i^{-1}(W_i)} \subseteq \bigcap_{i = 1}^n \iota(g_i^{-1}(W_i)) \subseteq \bigcap_{i=1}^n f_i^{-1}(V_i)
  \end{align*}
  und haben die Stetigkeit von $\iota$ gezeigt.

  Sei nun umgekehrt $\bigcap_{i = 1}^n g_i^{-1}(W_i)$, ein Element aus der Nullumgebungsbasis von $\sigma(X^\primeprime, X^\prime)$ gegeben. Analog wie vorhin definieren wir für $k \in \{1, \dots, n\}$
  \begin{align*}
    f_k := g_k \circ \iota^{-1} \in \iota_1(X^\prime) \quad \textrm{und} \quad V_k := W_k.
  \end{align*}
  Wie schon vorhin sind $\iota^{-1}: (\iota(X), \norm[X^\primeprime]{\cdot}\mid_{\iota(X)}) \to (X, \norm[X]{\cdot})$ und $g_k: (X, \norm[X]{\cdot}) \to \C$ und daher auch $f_k$ stetig und linear, wobei, weil $\iota_1$ die Auswertungsabbildung ist und $g_k \in X^\prime$, sogar $f_k \in \iota_1(X\prime)$. Wie vorhin gilt nun auch 
  \begin{align*}
    \iota \pbraces{\bigcap_{i = 1}^n f_k^{-1}(V_k)} \subseteq \bigcap_{i = 1}^n g_k^{-1}(W_k)
  \end{align*}
  und damit ist $\iota^-1$ stetig.

  ***************

  \item

  Im Sinne einer Abbildung in einen mit der Initialtopologie versehenen Raum ist $\iota$ genau dann stetig, wenn gilt

  \begin{align*}
    \Forall f \in \iota_1(X^{\prime}): f \circ \iota(\cdot) \text{~stetig~} \Leftrightarrow \Forall y \in X^{\prime}: \iota_1 (y) \circ \iota(\cdot) \text{~stetig~} \\
     \Leftrightarrow \Forall y \in X^{\prime}: \iota(\cdot)(y) \text{~stetig~} \Leftrightarrow \Forall y \in X^{\prime}: y(\cdot) \text{~stetig~}.
  \end{align*}

  Wir wissen bereits, dass
  \begin{align*}
    \iota^{-1}: (\iota(X),||\cdot||_{X^{\primeprime}|_{\iota(X)}}) \rightarrow (X,||\cdot||_X)
  \end{align*}
  stetig ist und somit auch für alle $y \in X^{\prime}$
  \begin{align*}
    y \circ \iota^{-1}: (\iota(X),||\cdot||_{X^{\primeprime}|_{\iota(X)}}) \rightarrow (\C,\mathcal{T}_\C)
  \end{align*}
  stetig ist. Also erhalten wir $y \circ \iota^{-1} \in \iota(X)^{\prime} \stackrel{?}{\subseteq} \iota_1(X^{\prime})$

  ***************

  Wir wissen es ist
  \begin{align*}
    \iota \quad \textrm{stetig} \Leftrightarrow \forall f \in \iota_1(X^\prime): f \circ \iota \quad \textrm{stetig}
  \end{align*}
  Wir wählen also 
  \begin{align*}
    f \in \iota_1(X^\prime) \subseteq X^\primeprimeprime 
  \end{align*}
  beliebig. Da $\iota: (X, \norm[X]{\cdot}) \to (X, \norm[X^\primeprime]{\cdot})$ stetig und linear ist erhalten wir auch die Stetigkeit von 
  \begin{align*}
    f \circ \iota \in (X, \norm[X]{\cdot})^\prime = (X, \sigma(X, X^\prime))^\prime
  \end{align*}

  Umgekehrt haben wir
  \begin{align*}
    \iota^{-1} \quad \textrm{stetig} \Leftrightarrow \forall g \in X^{\prime}: g \circ \iota^{-1} \quad \textrm{stetig}
  \end{align*}
  Wir wählen wieder beliebiges $g \in X^\prime$. Da $\iota^{-1}: (\iota(X), \norm[X^\primeprime]{\cdot}\mid_{\iota(X)}) \to (X, \norm[X]{\cdot})$ stetig und linear ist, wissen wir das auch von
  \begin{align*}
    g \circ \iota \in (\iota(X), \norm[X^\primeprime]{\cdot}\mid_{\iota(X)})^\prime = (\iota(X), \sigma(X^\primeprime, \iota_1(X^\prime))\mid_{\iota(X)})^\prime
  \end{align*}

  ***************

\end{enumerate}


\end{solution}
