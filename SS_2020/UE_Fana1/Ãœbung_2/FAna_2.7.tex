\begin{exercise}
Kann man in (a) und (b) des vorigen Beispiels die Vorraussetzung
$\sup_{n \in \mathbb{N}} ||A_n|| < \infty$ weglassen?
Falls ja, beweise dies. Falls nein, finde ein Gegenbeispiel.
\end{exercise}
\begin{solution}

Wir zeigen an einem Beispiel, dass man die Vorraussetzung nicht weglassen kann.
Dabei machen wir zuerst die allgemeine Feststellung bezüglich des Zusammenhangs der
Operatortopologien:

\begin{align*}
  \mathcal{T}_w \subseteq \mathcal{T}_s \subseteq \mathcal{T}_{||\cdot||}
\end{align*}

Also ist die schwache Operatortopologie gröber als die starke und diese wiederum
gröber als die Norminduzierte. Daher brauchen wir nur ein Beispiel finden, bei dem
die Voraussetzungen von (a), ohne die aus der Angabe, gelten und wenn wir daraus folgern können, das die Conlusio von (b) falsch ist haben wir gezeigt, dass die Vorraussetzung sowohl
für (a) als auch (b) notwendig ist.

Für die Folge $(A_n)_{n \ in \N}$ gilt:

\begin{align*}
  A_n \stackrel{s}{\rightarrow} A
  \Leftrightarrow
  \forall y \in Y: A_n y \stackrel{||\cdot||_Z}{\rightarrow} Ay
\end{align*}

Wir wählen unsere Räume wie folgt: $Y = (L^1 (0,1), ||\cdot||_1) ; Z=X=\R$. Der
Darstellungssatz von Riesz gibt uns jetzt vor, wie unsere $A_n$ aussehen:

\begin{align*}
  \forall n \in \N: \exists h_n \in L^{\infty}(0,1):
  \forall g \in L^1 (0,1): A_n(g) = \int_{(0,1)}h_n g d\lambda \\
  \text{bzw.} A(g)= \int_{(0,1)} hg d\lambda
\end{align*}

Wir wählen:

\begin{align*}
  h_n (x):= \begin{cases}
    n, & \text{falls\,} x \in (0,\frac{1}{n^2}) \\
    \frac{1}{n} & \text{sonst}
  \end{cases}
\end{align*}

Dann gilt für die Norm der $A_n$

\end{solution}
