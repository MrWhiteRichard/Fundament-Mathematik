\begin{exercise}

Sei $X$ ein normierter Raum, $M$ ein linearer Teilraum von $X$, und $f: M \to \C$ ein beschränktes lineares Funktional.
Nach dem Satz von Hahn-Banach existiert eine Fortsetzung $F: X \to \C$ mit $\norm{F} = \norm{f}$.
Im allgemeinen muss diese nicht eindeutig sein. \\

Finde ein Beispiel von $X$, $M$, $f$ wie oben, wo es tatsächlich mehrere normerhaltende Fortsetzungen gibt. \\

Hinweis.
Man erinnere sich was $(\ell^1)^\prime$ ist.

\end{exercise}

\begin{solution}

\phantom{}

\begin{enumerate}

  \item
  Wir betrachten die Banachräume $X := (\ell^1, \norm[1]{\cdot})$ und $(\ell^1, \norm[\infty]{\cdot})$, wobei

  \begin{align*}
    \ell^1
    & :=
    \Bbraces
    {
      x \in \C^\N:
      \norm[1]{x} < \infty
    },
    &
    \norm[1]{x}
    & :=
    \sum_{n=1}^\infty |x_n|, \\
    \ell^\infty
    & :=
    \Bbraces
    {
      x \in \C^\N:
      \norm[\infty]{x} < \infty
    },
    &
    \norm[\infty]{x}
    & :=
    \sup_{n=1}^\infty |x_n|.
  \end{align*}

  \item
  $e := (\delta_{1n})_{n \in \N} \in \ell^1$, weil $\norm[1]{e} = 1$.

  \item
  $M :=
  \Span{\Bbraces{e}} =
  \Bbraces{\lambda e: \lambda \in \C}$ ist ein linearer Teilraum von $\ell^1$.

  \item
  $f: M \to \C: \lambda e \mapsto |\lambda|$ ist ein beschränktes Funktional aus $M^\prime$, weil

  \begin{align*}
    \norm[M^\prime]{f}
    =
    \sup
    \{
      \underbrace{|f(\lambda e)|}_{|\lambda|}:
      \lambda \in \C,
      \underbrace{\norm[1]{\lambda e}}_{|\lambda|} = 1
    \} = 1.
  \end{align*}

\end{enumerate}

Laut Beispiel 2.3.3, ist folgendes $\varphi$ ein isometrischer Isomorphismus.

\begin{align*}
  \varphi:
  \ell^\infty \to (\ell^1)^\prime:
  c \mapsto
  \pbraces
  {
    y \mapsto
    \sum_{n = 1}^\infty y_n c_n
  }
\end{align*}

Wir betrachten das Bilde unter $\varphi$ von folgender Teilmenge von $\ell^\infty$

\begin{align*}
  A :=
  \Bbraces
  {
    c \in \ell^1:
    \norm[\infty]{c} = 1,  c_1 = 1
  }
\end{align*}

Wählen wir nun ein beliebiges $F \in \varphi(A) \subseteq (\ell^1)^\prime$, dann $\Exists c \in A: \Forall y \in \ell^1:$

\begin{align*}
  F(y)
  =
  \sum_{n=1}^\infty y_n c_n.
\end{align*}

$F$ setzt $f$ fort, d.h. $F|_M = f$, weil $c_1 = 1$ und damit $\Forall \lambda \in \C:$

\begin{align*}
  F(\lambda e)
  =
  \lambda c_1
  =
  \lambda
  =
  f(\lambda e).
\end{align*}

Da $\varphi$ isometrisch ist, erhalten wir

\begin{align*}
  \norm[X^\prime]{F}
  =
  \norm[\infty]{\varphi^{-1}(F)}
  =
  1
  =
  \norm[M^\prime]{f}.
\end{align*}

D.h. alle $F \in \varphi(A)$ sind normerhaltende Fortsetzungen von $f$.
Es sind $1, e \in A$, also ist $|A| > 1$.
Da $\varphi$ als Isomorphismus insbesondere injektiv ist, gilt daher auch $|\varphi(A)| > 1$.
Also gibt es mindestens zwei verschiedene normerhaltende Fortsetzungen von $f$.
\\
Einfacher: \\
Betrachte die Funktionen
\begin{align*}
  F(x) = |x_1| \\
  G(x) = \sum_{j=1}^n|x_j|,
\end{align*}
welche beide normerhaltende Fortsetzungen von $f$ sind.
\end{solution}
