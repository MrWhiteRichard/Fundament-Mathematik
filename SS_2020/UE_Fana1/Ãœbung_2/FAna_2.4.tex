\begin{exercise}
Man betrachte den Banachraum $L^{\infty}(0,1)$, und erinnere sich, dass
$L^{\infty}(0,1) = L^1(0,1)^{\prime}$. Damit haben wir auf $L^{\infty}(0,1)$ drei
in natürlicher Weise gegebene Topologien: die Normtopologie, die $w$-Topologie,
und die $w^*$-Topologie.
\begin{itemize}
  \item [a)] Der Funktionenraum $C([0,1])$ ist ein Teilraum von $L^{\infty}(0,1)$.
  In welcher/n der obigen Topologien ist er abgeschlossen, und in welcher/n nicht?
  \item [b)] Zeige, dass die $w^*$-Topologie verschieden von der $w$-Topologie ist.
  \item [c)] Ein Banachraum heißt \textit{reflexiv}, wenn die kanonische Abbildung
  $\iota: X \rightarrow X^{\primeprime}$ surjektiv ist. Zeige, dass $L^1(0,1)$ nicht
  reflexiv ist.
\end{itemize}

\end{exercise}

\begin{solution}
$C[0,1]$ ist der Raum aller Äquivalenzklassen von Funktionen auf $[0,1]$ die
einen und sogar genau einen stetigen Repräsentanten besitzen.
  \begin{enumerate}[label = (\alph*)]
    \item Wir gehen die drei Topologien durch.
    \begin{enumerate}
      \item[\glqq $\norm{\cdot}$ \grqq] Wir wählen $f \in \overline{C[0, 1]}^{\norm[\infty]{\cdot}}$ beliebig. Dann gibt es ein Netz $(f_i)_{i \in I}$ aus $C[0,1]$ mit $f_i \stackrel{\norm[\infty]{\cdot}}{\to} f$. Die $f_i$ sind Äquivalenzklassen, wir betrachten sie nun als die stetigen Repräsentaten. Für die stetigen Repräsentanten entspricht
      die $\norm[\infty]{\cdot}$-Konvergenz der gleichmäßigen Konvergenz und
      somit ist $f$ als gleichmäßiger Grenzwert stetiger Funktionen wieder stetig.
      \item [\glqq $w$-Topologie \grqq]
      \begin{align*}
        \sigma(L^{\infty}(0,1),(L^{\infty}(0,1), ||\cdot||_{\infty})^{\prime}) =: \mathcal{T}_w
      \end{align*}
      $C[0,1]$ ist konvex, denn für beliebige $f,g \in C[0,1]$ und $t \in [0,1]:
      tf + (1-t)g \in C[0,1]$, deshalb gilt nach Satz 5.3.8
      $\overline{C[0,1]}^{\mathcal{T}_w} = \overline{C[0,1]}^{||\cdot||_{\infty}} = C[0,1]$.
      Also gilt $C[0,1]$ ist abgeschlossen in $(L^{\infty}(0,1), \mathcal{T}_w)$.
      \item [\glqq $w^*$-Topologie \grqq]
      \begin{align*}
        \sigma(L^{\infty}(0,1),\iota(L^1(0,1))) =: \mathcal{T}_{w^*},
      \end{align*}
      wobei $\iota: L^1(0,1) \rightarrow (L^1(0,1)^{\prime})^*: g \mapsto (T \mapsto T(g))$.
      Nach dem Darstellungsatz von Riesz (vgl. Kusolitsch Satz 13.40) können wir
      jedem $T \in L^1(0,1)^{\prime}$ mit einem eindeutigen $h \in L^{\infty}(0,1)$
      identifizieren, sodass $T(g) = \int_{0}^1 gh d\lambda$ gilt.
      Fassen wir also $\iota(L^1(0,1))$ als Teilmenge von $(L^{\infty}(0,1))^*$ auf,
      so ist
      \begin{align*}
      \iota(L^1(0,1)) = \{S_g: L^{\infty}(0,1) \rightarrow \mathbb{C}: h \mapsto \int_0^1gh d\lambda
      : g \in L^1(0,1)\} \subseteq L^{\infty}(0,1)^{\prime},
      \end{align*}
      wobei die zweite
      Inklusion gilt, da nach Kusolitsch Lemma 13.38 alle $S_g \in \iota(L^1(0,1))$
      stetig sind als Abbildungen $(L^{\infty}(0,1),||\cdot||_{\infty}) \rightarrow \mathbb{C})$.
      Für beliebige $f,g \in C[0,1]$ und beliebige $\alpha \in \mathbb{C}$ ist $f + \alpha g \in C[0,1]$,
      also ist $C[0,1]$ ein linearer Teilraum von $L^{\infty}(0,1)$, $(L^{\infty}(0,1),\mathcal{T}_{w^*})$
      ein lokalkonvexer topologischer Vektorraum und deshalb
      gilt nach Korollar 5.2.6:
      \begin{align*}
        \overline{C[0,1]} = \bigcap_{\stackrel{S_g \in \iota(L^1(0,1))}{C[0,1] \subseteq \ker S_g}} \ker S_g.
      \end{align*}
      Aus $C[0,1] \subseteq \ker S_g$ folgt
      \begin{align*}
        \forall h \in C[0,1]: S_g(h) = \int_0^1 gh d\lambda = 0
        \implies g = 0 ~\text{f. ü.}~ \implies S_g = 0.
      \end{align*}
      Also gilt schlicht
      \begin{align*}
        \overline{C[0,1]}^{\mathcal{T}_w^*} = \ker 0 = L^{\infty}(0,1) \supsetneq C[0,1].
      \end{align*}
      Um einzusehen, dass $L^{\infty}(0,1)$ eine echte Obermenge von $C[0,1]$ ist,
      betrachte die Signum-Funktion, welche nicht fast überall gleich einer stetigen
      Funktion ist.
      $C[0,1]$ ist demnach nicht abgeschlossen in $(L^{\infty}(0,1),\mathcal{T}_{w^*})$.
    \end{enumerate}
    \item
    \begin{align*}
      C[0,1]^{\complement} \in \mathcal{T}_w \land C[0,1]^{\complement} \notin \mathcal{T}_{w^*} \implies
      \mathcal{T}_w \neq \mathcal{T}_{w^*}
    \end{align*}
    \item Angenommen $\iota: L^1(0,1) \rightarrow L^1(0,1)^{\primeprime}$ wäre
    surjektiv, dann wäre
    \begin{align*}
      \iota(L^1(0,1)) = (L^{\infty}(0,1), ||\cdot||_{\infty})^{\prime}
    \end{align*}
    und da laut Korollar 5.2.7 für jeden lokalkonvexen Vektorraum $(X,\mathcal{T})$
    $X^{\prime}$ punktetrennend ist, folgt mit Korollar 5.3.5 auch
    \begin{align*}
    \sigma(L^{\infty}(0,1),\iota(L^1(0,1))) = \mathcal{T}_{w^*} = \mathcal{T}_w =
    \sigma(L^{\infty}(0,1),(L^{\infty}(0,1), ||\cdot||_{\infty})^{\prime}),
    \end{align*}
    was einen Widerspruch zu b) ergibt.
  \end{enumerate}
\end{solution}
