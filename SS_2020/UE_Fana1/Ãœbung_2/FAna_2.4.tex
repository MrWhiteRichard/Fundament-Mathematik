\begin{exercise}

Man betrachte den Banachraum $L^\infty(0, 1)$, und erinnere sich, dass $L^\infty(0, 1) = L^1(0, 1)^\prime$.
Damit haben wir auf $L^\infty(0, 1)$ drei in natürlicher Weise gegebene Topologien:
die Normtopologie, die $w$-Topologie, und die $w^\ast$-Topologie.

\begin{enumerate}[label = \alph*)]

  \item
  Der Funktionenraum $C([0, 1])$ ist ein Teilraum von $L^\infty(0, 1)$.
  In welcher/n der obigen Topologien ist er abgeschlossen, und in welcher/n nicht?

  \item
  Zeige, dass die $w^\ast$-Topologie verschieden von der $w$-Topologie ist.

  \item
  Ein Banachraum heißt \textit{reflexiv}, wenn die kanonische Abbildung $\iota: X \rightarrow X^{\primeprime}$ surjektiv ist.
  Zeige, dass $L^1(0, 1)$ nicht reflexiv ist.

\end{enumerate}

\end{exercise}

\begin{solution}

$C[0, 1]$ ist der Raum aller Äquivalenzklassen $\lambda$-f.ü. gleicher Funktionen auf $[0, 1]$, die (genau) einen stetigen Repräsentanten besitzen.
$\norm[\infty]{\cdot}$ ist die essentielle Supremumsnorm.

\begin{enumerate}[label = \alph*)]

  \item
  \phantom{}

  \begin{itemize}

    \item
    [\Quote{$\norm[\infty]{\cdot}$}:]

    Sei $f \in \overline{C[0, 1]}^{\norm[\infty]{\cdot}}$.
    Dann $\Exists (f_i)_{i \in I} \in C[0, 1]^I:$

    \begin{align*}
      f_i \xrightarrow{\norm[\infty]{\cdot}} f.
    \end{align*}

    Die $f_i$ sind ja Äquivalenzklassen, wir identifizieren sie aber mit ihren stetigen Repräsentaten.
    Für diese entspricht die $\norm[\infty]{\cdot}$-Konvergenz der gleichmäßigen Konvergenz.
    Somit ist $f$ als gleichmäßiger Grenzwert stetiger Funktionen wieder stetig.
    Also, ist $C[0, 1]$ abgeschlossen in $(L^\infty(0, 1), \mathcal{T}_{\norm[\infty]{\cdot}})$.

    \item
    [\Quote{$w$}:]

    Wir erinnern uns an die Definition der $w$-Topologie.

    \begin{align*}
      \mathcal{T}_w
      :=
      \sigma
      (
        L^\infty(0,1),
        (L^\infty(0,1), \norm[\infty]{\cdot})^\prime
      )
    \end{align*}

    $C[0, 1]$ ist konvex, denn $\Forall f, g \in C[0, 1], \Forall \alpha \in (0, 1):$

    \begin{align*}
      \alpha f + (1 - \alpha) g \in C[0, 1].
    \end{align*}

    Außerdem, ist $(L^\infty(0, 1), \mathcal{T}_w)$ lokalkonvex.
    Laut \Quote{$\norm[\infty]{\cdot}$} und Satz 5.3.8, gilt also

    \begin{align*}
      \overline{C[0,1]}^{\mathcal{T}_w}
      =
      \overline{C[0,1]}^{\norm[\infty]{\cdot}}
      =
      C[0, 1].
    \end{align*}

    Also, ist $C[0, 1]$ abgeschlossen in $(L^\infty(0,1), \mathcal{T}_w)$.

    \item
    [\Quote{$w^\ast$}:]

    Wir erinnern uns an die Definition der $w^\ast$-Topologie.

    \begin{align*}
      \mathcal{T}_{w^\ast}
      :=
      \sigma
      (
        L^\infty(0,1),
        \iota(L^1(0,1))
      ) \\
      \iota:
      L^1(0, 1) \rightarrow (L^1(0, 1)^\prime)^\ast:
      g \mapsto
      (
        L^1(0, 1)^\prime \to \C:
        T \mapsto T(g)
      )
    \end{align*}

    Laut Kusolitsch Satz 13.40 (Darstellungsatz von Riesz), gilt $\Forall T \in L^1(0, 1)^\prime:
    \ExistsOnlyOne h \in L^\infty(0, 1): \Forall g \in L^1(0, 1):$

    \begin{align*}
      T(g) = \Int[0][1]{g h}{\lambda}.
    \end{align*}

    Fassen wir also $\iota(L^1(0,1)) \subseteq (L^\infty(0,1))^\ast$ auf, so ist

    \begin{align*}
      \iota(L^1(0,1))
      & =
      \Bbraces
      {
        L^1(0, 1)^\prime \to \C:
        T \mapsto T(g):
        g \in L^1(0, 1)
      } \\
      & =
      \Bbraces
      {
        L^\infty(0, 1) \rightarrow \C:
        h \mapsto \Int[0][1]{g h}{\lambda}:
        g \in L^1(0, 1)
      }
      \subseteq L^\infty(0,1)^\prime.
    \end{align*}

    Die letzte Inklusion, gilt laut Kusolitsch Lemma 13.38.
    $C[0, 1]$ ist ein linearer Teilraum von $L^\infty(0,1)$, und $(L^\infty(0, 1),\mathcal{T}_{w^\ast})$
    ein lokalkonvexer topologischer Vektorraum.

    \begin{align*}
      M := \Bbraces
      {
        S \in \iota(L^1(0, 1)):
        C[0, 1] \subseteq \ker S
      }
    \end{align*}

    Dann gilt laut Korollar 5.2.6, dass

    \begin{align*}
      \overline{C[0, 1]}^{\mathcal{T}_w^\ast}
      =
      \bigcap_{S \in M} \ker S.
    \end{align*}

    Für $S \in M$, folgt aus $C[0,1] \subseteq \ker S$, dass $\Forall h \in C[0, 1]:$

    \begin{align*}
      \Int[0][1]{g_S h}{\lambda} = S(h) = 0
      \implies
      g_S h = 0 ~\text{f.ü.}~
    \end{align*}

    Weil das aber für alle $h \in C[0, 1]$ gilt, muss auch $g_S = 0$ f.ü. und damit $S = 0$.
    Das heißt also $M = \Bbraces{0}$ und damit

    \begin{align*}
      \overline{C[0, 1]}^{\mathcal{T}_w^\ast}
      =
      \ker 0
      =
      L^\infty(0, 1)
      \supsetneq
      C[0, 1].
    \end{align*}

    Um $\supsetneq$ einzusehen, betrachte $\sgn$.
    Diese Funktion ist beschränkt ($\in L^\infty(0, 1)$), aber nicht fast überall gleich einer stetigen
    Funktion ($\notin C[0, 1]$).
    $C[0,1]$ ist demnach nicht abgeschlossen in $(L^\infty(0, 1), \mathcal{T}_{w^\ast})$.
  \end{itemize}

  \item

  Abgeschlossene Mengen sind die Komplemente von offenen Mengen.
  In a) haben wir gesehen, dass $C[0, 1]^\complement \in \mathcal{T}_w$ und $C[0,1]^{\complement} \notin \mathcal{T}_{w^\ast}$.
  Dann muss aber $\mathcal{T}_w \neq \mathcal{T}_{w^\ast}$.

  \item

  Angenommen $\iota: L^1(0, 1) \rightarrow L^1(0, 1)^\primeprime$ wäre surjektiv, d.h. $\iota(L^1(0, 1)) = L^1(0, 1)^\primeprime$.
  Mit unserer Identifikation erhalten wir also

  \begin{align*}
    \iota(L^1(0, 1))
    =
    (L^\infty(0, 1), \norm[\infty]{\cdot})^\prime.
  \end{align*}

  Laut Korollar 5.2.7, ist für jeden lokalkonvexen Vektorraum $(X, \mathcal{T})$, auch $X^\prime$ punktetrennend.
  Laut Korollar 5.3.5, gilt also

  \begin{align*}
    \sigma(L^\infty(0, 1),\iota(L^1(0, 1)))
    =
    \mathcal{T}_{w^\ast}
    =
    \mathcal{T}_w
    =
    \sigma
    (
      L^\infty(0, 1),
      (L^\infty(0, 1), \norm[\infty]{\cdot})^\prime
    ).
  \end{align*}

  Das ist aber ein Widerspruch zu b).

\end{enumerate}

\end{solution}
