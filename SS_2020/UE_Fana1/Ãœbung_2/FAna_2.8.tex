\begin{exercise}
Für $i \in I$ sei $(X_i, \mathcal{T}_i)$ ein lokalkonvexer Vektorraum und
$\{p_{ij} ~|~ j \in J\}$ eine Familie von Seminormen, die $\mathcal{T}_i$ erzeugt.
Finde eine Familie von Seminormen auf $\prod_{i \in I} X_i$, welche die Produkttopologie erzeugt.

\end{exercise}

\begin{solution}



Wir identifizieren den Produktraum als Menge aller Auswahlfunktionen:
\begin{align*}
    \prod_{i \in I} X_i = \{f: I \rightarrow \bigcup_{i \in I} X_i ~|~ f(i) \in X_i\}.
\end{align*}

Auf dieser Menge definieren wir eine Familie von Seminormen:
\begin{align*}
    p_{ij}'(f) := p_{ij}(f(i)).
\end{align*}

$p_{ij}'$ ist eine Seminorm auf dem Produktraum:

\begin{itemize}
    \item $p_{ij}'(\alpha f) = p_{ij}(\alpha f(i)) = |\alpha|~ p_{ij}(f(i)) = |\alpha|~ p_{ij}'(f),$
    
    \item $p_{ij}'(f + g) = p_{ij}(f(i) + g(i)) \leq p_{ij}(f(i)) + p_{ij}(g(i)) = p_{ij}'(f) + p_{ij}'(g).$
\end{itemize}

Die Familie $(p_{ij}')_{(i,j) \in I \times J}$ ist überdies separierend: Seien $f, g \in \prod_{i \in I} X_i$ mit $f \neq g$. Dann existiert ein $i \in I$, sodass $g(i) \neq f(i)$. Für $g(i), f(i) \in X_i$ gibt es nun, weil die Familie $(p_{ij})_{j \in J}$ separierend ist, ein $j \in J$, sodass $p_{ij}(g(i)) \neq p_{ij}(f(i))$, was äquivalent zu $p_{ij}'(f) \neq p_{ij}'(g)$ ist.

Die durch die Familie der Seminormen $(p_{ij}')_{(i,j) \in I \times J}$ induzierte Topologie auf dem Produktraum ist die Initialtopologie der zugehörigen Projektionen (\pi_{ij}')_{(i,j) \in I \times J}:


 \pi_{ij}': ~\begin{cases}

    \prod_{i \in I} X_i \rightarrow (\prod_{i \in I} X_i) / N(p_{ij}')\\
    f \mapsto f + N(p_{ij}'), \text{wobei}~ N(p_{ij}') := \{g \in \prod_{i \in I} X_i ~|~ p_{ij}'(g) = 0\}.
\end{cases}

Jeder Faktorraum $(\prod_{i \in I} X_i) / N(p_{ij}')$ trägt dabei jene Topologie, die von der Norm $\overline{p_{ij}'}$ induziert wird, welche repräsentantenweise durch die Seminorm ${p_{ij}'}$ definiert ist.


\textbf{ab da hab ichs mir noch nicht wirklich durchüberlegt - weiß gar nicht ob es genügt, das zu zeigen}

Wir zeigen nun, dass jede Umgebung der Null in der einen Topologie auch eine solche in der jeweils anderen ist.


Eine Nullumgebungsbasis im Produktraum bezüglich der von den Seminormen induzierten Topologie bildet folgende Menge:
\begin{align*}
    \mathcal{B} = 
    \{\bigcap_{(i,j) \in E} (\pi_{ij}')^{-1}(U^{{\overline{p_{ij}'}}}_{\epsilon}(N(p_{ij}'))
    ~|~ ...
    \}.
\end{align*}

Es gilt
\begin{align*}
    (\pi_{ij}')^{-1}(U^{{\overline{p_{ij}'}}}_{\epsilon}(N(p_{ij}')))
   = \{ f \in \prod_{i \in I} X_i ~|~
   \overline{p_{ij}'}(\pi_{ij}'(f) - N(\pi_{ij}'))
   < \epsilon \}
   = \{ f \in \prod_{i \in I} X_i ~|~
   p_{ij}'(f) < \epsilon \}
   = U_{\epsilon}^{p_{ij}'}(0),
\end{align*}

eine Nullumgebungsbasis besteht also genau aus den endlichen Schnitten von Kugeln bezüglich der Seminormen.


Betrachten wir nun eine Nullumgebungsbasis in der Produkttopologie:

\begin{align*}
    \mathcal{C} = 
    \{\bigcap_{i \in E} (\pi_{i})^{-1}(V)
    ~|~ V ~\text{Nullumgebung in}~ (X_i, \mathcal{T}_i)
    \}.
\end{align*}

Die Räume $(X_i, \mathcal{T}_i)$ sind nun ihrerseits wiederum von den Familien
$\{p_{ij} ~|~ j \in J\}$ von Seminormen erzeugt. Jedes V aus der oberen Menge enthält also einen endlichen Schnitt von Kugeln:
\begin{align*}
 \bigcap_{j \in E} U_{\epsilon}^{p_{ij}}(0) \subseteq V.
\end{align*}

Demnach lassen sich die Mengen aus $\mathcal{C}$ schreiben als

\begin{align*}
    \bigcap_{i \in E_1} (\pi_{i})^{-1}(\bigcap_{j \in E_2} U_{\epsilon}^{p_{ij}}(0))
    = 
    \bigcap_{i \in E_1}
    \bigcap_{j \in E_2}
    (\pi_{i})^{-1}(U_{\epsilon}^{p_{ij}}(0))
    =
    \bigcap_{(i,j) \in E_1 \times E_2} U_{\epsilon}^{p_{ij}'}(0);
\end{align*}

insgesamt gilt also $\mathcal{B} = \mathcal{C}$, was zu zeigen war.

\end{solution}
