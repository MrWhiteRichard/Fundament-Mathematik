\begin{exercise}
    Zeigen Sie, dass es einen normierten Raum $X$ über dem Körper $C$, einen linearen Teilraum $M \subseteq X$ und ein beschränktes lineares Funktional $f:M \to \C$ so gibt, dass es mehr als ein $F: X \to \C$ gibt, das $f$ fortsetzt und $\Vbraces{F} = \Vbraces{f}$ erfüllt.
\end{exercise}

\begin{solution}
    Wir betrachten den Banachraum $ X := l^1 = \Bbraces{(x_n)_{n \in \N} : \Vbraces{(x_n)_{n \in \N}}_1 = \sum_{n = 1}^\infty \vbraces{x_n} < \infty}$. Wir definieren $e := (1, 0, \dots)$ und sehen, dass wegen $\Vbraces{e}_1 = 1$ die Familie $e \in l^1$ ist. Nun definieren wir den linearen Teilraum $M := span\Bbraces{e} = \Bbraces{\lambda e \mid \lambda \in \C}$. Dazu definieren wir $f: M \to \C: \lambda e \mapsto \lambda$. Man rechnet nach, dass $f$ linear ist. Weiters ist $\Vbraces{f} = \sup\Bbraces{\vbraces{f(\lambda e)} : \Vbraces{\lambda e_1} = 1 = \vbraces{\lambda} } = 1$, also ist $f$ beschränkt. 

    Aus der Maßtheorie wissen wir, dass $\varphi: l^\infty \to (l^1)^\prime : (c_n)_{n \in \N} \mapsto \pbraces{(y_n)_{n \in \N} \mapsto \sum_{n = 1}^\infty y_n c_n}$ ein isometrischer Isomorphismus ist (vgl. FAna Skript Beispiel 2.3.3), wobei $l^\infty = \Bbraces{(x_n)_{n \in \N}: \Vbraces{(x_n)_{n \in \N}}_\infty = \sup\Bbraces{\vbraces{x_n} : n \in \N} < \infty}$. Um einzusehen, dass $f$ nun mehr als eine normerhaltende Fortsetzung hat, definieren wir \\ $A := \Bbraces{(c_n)_{n \in \N} : \Vbraces{(c_n)_{n \in \N}}_\infty = 1 \land c_1 = 1} \subseteq l^\infty$. Es sind $(1)_{n \in \N}, e \in A$ also ist $\vbraces{A} > 1$. Da $\varphi$ als Isomorphismus insbesondere injektiv ist, gilt daher auch $\vbraces{\varphi(A)} > 1$. Nehmen wir nun ein beliebiges $F \in \varphi(A) \subseteq (l^1)^\prime$, dann gibt es ein $(c_n)_{n \in \N} \in A$ so, dass für alle $(y_n)_{n \in \N} \in l^1: F\pbraces{(y_n)_{n \in \N}} = \sum_{n = 1}^\infty y_n c_n$. Daraus erhalten wir für beliebiges $\lambda e \in M$ wegen $c_1 = 1$ nach Definition von $A$ die Gleichheit $F(\lambda e) = \lambda c_1 = \lambda = f(\lambda e)$, also $F\mid_M = f$ oder $F$ setzt $f$ fort. Da $\varphi$ isometrisch ist erhalten wir $\Vbraces{F} = \Vbraces{\varphi^{-1}(F)}_\infty = 1 = \Vbraces{f}$. Das heißt alle $F \in \varphi(A)$ sind normerhaltende Fortsetzungen von $f$ und $\varphi(A)$ enthält mindestens zwei Funktionen, also gibt es mindestens zwei normerhaltende Fortsetzungen von $f$.
\end{solution}