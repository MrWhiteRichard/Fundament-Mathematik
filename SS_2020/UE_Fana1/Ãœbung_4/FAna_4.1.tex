\begin{exercise}
Sei $(X,\|\cdot\|)$ ein Banachraum. Finde einen kompakten Hausdorffraum $K$, sodass
$(X,\|\cdot\|)$ isometrisch isomorph zu einem abgeschlossenem Teilraum $A$ von
$(C(K),\|\cdot\|_{\infty})$ ist. \\
\textit{Hinweis:} Betrachte $\iota: X \to X^{\primeprime}$.
\end{exercise}
\begin{solution}
Nach dem Satz von Banach-Alaoglu 5.5.6 ist \includegraphicsboxed{5.5.6}
\begin{align*}
  K := \{f \in X^{\prime}: \|f\| \leq 1\}
\end{align*}
bezüglich $\sigma(X^{\prime}, \iota(X))$ kompakt.
Wir wissen bereits, dass $(X^{\prime}, \sigma(X^{\prime},\iota(X)))$ lokalkonvexer
topologischer Vektorraum ist, insbesondere also ein Hausdorffraum.
Es ist also $(K, \sigma(X^{\prime},\iota(X))|_K)$ ein kompakter Hausdorffraum.
Damit ist $(C(K), \|\cdot\|_{\infty})$ ein normierter Raum.
Nach Lemma 5.5.2 ist $\iota: (X,\|\cdot\|) \to (\iota(X),\|\cdot\|_{X^{\primeprime}}|_{\iota(X)})$
isometrisch und linear.
Aus Bemerkung 5.5.3 entnehmen wir, dass $\iota$ auch bijektiv ist, also ist $\iota$
ein Homöomorphismus.
Definiere
\begin{align*}
  A := \{g|_K: K \to \C: g \in \iota(X)\}
\end{align*}
Es gilt sicher $A \subseteq C(K)$.
Sei $\varphi: X \to A; x \mapsto \iota(x)|_K$.
Es gilt:
\begin{itemize}
  \item Injektivität: Seien $x \neq y \in X$ beliebig. Nach Korollar 5.2.7. (i)
  ist $X^{\prime}$ punktetrennend, also
  \begin{align*}
    \exists g \in X^{\prime}: g(x) \neq g(y).
  \end{align*}
  Die Funktion
  \begin{align*}
    h(x) := \frac{g(x)}{\|g\|_{X^{\prime}}}
  \end{align*}
  erfüllt dann $h \in K$ und $\varphi(x)(h) = \iota(x)(h) = h(x) \neq h(y) = \varphi(y)(h)$.
  Also folgt $\varphi(x) \neq \varphi(y)$.
  \item Surjektivität: Sei $f \in A$ beliebig. \\
  \begin{align*}
    f = g|_K = \iota(x)|_K = \phi(x).
  \end{align*}
  \item Linearität: Seien $x,y \in X, \lambda \in \C, f \in K$ beliebig.
  \begin{align*}
    \varphi(x + \lambda y)(f) = f(x + \lambda y) = f(x) + \lambda f(y) = \varphi(x)(f) + \lambda \varphi(y)(f)
  \end{align*}
  \item Isometrie: Sei $x \in X$ beliebig. Dann gilt für $x \in X$ beliebig fest
  \begin{align*}
    \|\phi(x)\|_{\infty} = \sup\{|f(x)|: f \in K\}
    = \sup\{|f(x)|: f \in X^{\prime}, \|f\| \leq 1\} \\
    \leq \sup\{\|f\|\|x\|_X: f \in X^{\prime}, \|f\| \leq 1\} \leq \|x\|_X
  \end{align*}
  Die andere Ungleichung folgt mit Korollar 5.2.4.
  \item Stetigkeit: $\varphi$ ist als isometrische Bijektion sogar ein Homöomorphismus.
\end{itemize}
Da $X$ ein Banachraum ist, erhalten wir mit Lemma 4.3.6, dass $\varphi(X) = A$
ein abgeschlossener Teilraum von $C(K)$ ist.
\end{solution}
