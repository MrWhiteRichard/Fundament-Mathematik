\begin{exercise}
Sei $n \in \N$, $\R^n[z]$ die Menge aller reellen Polynome vom Grad $\leq n$ und
\begin{align*}
  K := \{p \in \R^n[z]: p^{\primeprime}(x) \geq 0, x \in [-1,1]\}.
\end{align*}
Zeige, dass es für jedes $f \in L^2([-1,1])$ genau ein $p_0 \in K$ gibt, sodass
\begin{align*}
  \int_{-1}^{1}|f(t)-p_0(t)|^2 dt \leq \int_{-1}^1 |f(t) - p(t)|^2 dt
\end{align*}
für alle $p \in K$. \\
\textit{Hinweis:} Die Menge aller komplexen Polynome vom Grad $\leq n$ ist ein
endlich dimensionaler Unterraum und $p \mapsto p(t)$, sowie $p \mapsto p^{\primeprime}(t)$
sind lineare Funktionale auf diesem Unterraum für jedes $t \in [a,b]$. Schließe,
dass $K$ abgeschlossen und konvex ist.
\end{exercise}
\begin{solution}
Wir befinden uns im Hilbertraum $L^2([-1,1])$ mit
\begin{align*}
  (f,g) := \int_{-1}^1 f\overline{g} d\lambda.
\end{align*}
Weiters definieren wir
\begin{align*}
  P_n:= \{p: [-1,1] \to \C: x \mapsto \sum_{i=0}^n a_ix^i | \forall i \in
  \{1,\dots,n\}: a_i \in \C\}
\end{align*}
$P_n$ ist ein $n+1$-dimensionaler Unterraum von $L^2([-1,1])$ weil für $f,g \in P_n$
jedenfalls $\|f\|_{L^2}\|g\|_{L^2} < \infty$ und
\begin{align*}
  \grad(f+g) \leq \max\{\grad(f), \grad(g)\} \leq n,
\end{align*}
also gilt für $\lambda \in \C$ beliebig: $f + \lambda g \in P_n$
und $P_N$ ist als endlich-dimensionaler Unterraum abgeschlossen (Satz 2.2.1).
Sei
\begin{align*}
  \forall t \in [-1,1]: \alpha_t: \begin{cases}
    P_n &\to \C \\
    p &\mapsto p(t)
  \end{cases}
\end{align*}
Wir zeigen nun, dass $\alpha_t$ ein lineares Funktional ist:
\begin{itemize}
  \item Linearität: Seien $p,q \in P_n, \lambda \in \C$ beliebig.
  \begin{align*}
    \alpha_t(p + \lambda q) = p(t) + \lambda q(t) = \alpha_t(p) + \alpha_t(q).
  \end{align*}
  \item Stetigkeit: Da $\dim(P_n) = n + 1 < \infty$ gibt es eine Topologie mit
  der $P_n$ zu einem topologischen Vektorraum wird. Also betrachten wir
  \begin{align*}
    \|\alpha_t\| = \sup\{|p(t)|: \|p\|_{\infty} = 1\} = 1 < \infty,
  \end{align*}
  also ist $\alpha_t$ stetig.
\end{itemize}
Sei weiter $\forall t \in [-1,1]: \beta_t : P_n \to \C: p \mapsto p^{\primeprime}(t)$.
Die Funktionen sind wieder lineare Funktionale, weil
\begin{itemize}
  \item Stetigkeit:
  \begin{align*}
    \forall t \in [-1,1]: p^{\primeprime}(t) = \lim_{h \to 0^+}\frac{p(t+h) - 2p(t) + p(t-h)}{h^2}
  \end{align*}
  Sei nun $(h_k)_{k \in \N}$ eine Nullfolge aus $\R^+$ und setze für $k \in \N$:
  \begin{align*}
    \gamma_{t,k} := h_k^{-2}(\alpha_{t + h_k} - 2\alpha_t + \alpha_{t - h_k})
  \end{align*}
  ist ein stetiges, lineares Funktional.
  Für alle $p \in P_n$ existiert der Grenzwert $\lim_{k \to \infty}\gamma_{t,k}(p) = p^{\primeprime}(t)$-
  $P_n$ ist ein Banachraum, also ist nach Korollar 4.2.3
  \begin{align*}
    \beta_t: \begin{cases}
      P_n &\to \C \\
      p &\mapsto \lim_{k \to \infty}\gamma_{t,k}(p) = p^{\primeprime}(t)
    \end{cases}
  \end{align*}
  ein stetiges lineares Funktional.
\end{itemize}
$\R, \R^+ \cup \{0\} \subset \C$ sind abgeschlossen und konvex, also sind $\forall t \in [-1,1]$:
\begin{align*}
  \alpha_t^{-1}(\R) = \{p \in P_n: p(t) \in \R\}
\end{align*}
und
\begin{align*}
  \beta_t^{-1}(\R^+ \cup \{0\}) = \{p \in P_n: p^{\primeprime}(t) \geq 0\}
\end{align*}
abgeschlossen und konvex, also ist auch
\begin{align*}
  L := \bigcap_{t \in [-1,1]}\alpha_t^{-1}(\R)
\end{align*}
abgeschlossen und konvex. Sei nun $p = \sum_{i = 1}^n a_i x^i \in L$ beliebig.
Dann gilt
\begin{align*}
  \forall t \in [-1,1]: p(t) \in \R.
\end{align*}
Seien nun $k := \min \{ i \in \N: \Im(a_i) \neq 0\}$
und $b := \max\{|\Im(a_i)|: i \in \{k+1,\dots,n\}\}$.
Wir wählen jetzt ein $t \in ]0,1]$ mit $t^{k+1}b < \frac{|\Im(a_k)}{n}t^k$.
Ab hier unvollständig!
\end{solution}
