\begin{exercise}
Sei $(H,(\cdot,\cdot)_H)$ ein Hilbertraum, und $[\cdot,\cdot]$ ein weiteres Skalarprodukt
auf $H$. Sei vorausgesetzt, dass $[\cdot,\cdot]$ eine stetige, koerzive Sesquilinearform
ist, also dass (hier bezeichnet $\|\cdot\|_H$ die von $(\cdot,\cdot)_H$ induzierte Norm)
\begin{align*}
  \exists C > 0 \forall x,y \in H: |[x,y]| \leq C\|x\|_H\|y\|_H, \qquad
  \exists m > 0 \forall x \in H: [x,x] \geq m\|x\|_H^2.
\end{align*}
Weiters bezeichne $G$ den Gram-Operator der Sesquilinearform $[\cdot,\cdot]$
bezüglich $(\cdot,\cdot)_H$. \\
Zeige, dass es einen eindeutigen Operator $T \in \mathcal{B}(H)$ gibt, sodass
$GT = TG = \id_H $ gilt. \\
\textit{Hinweis:} Zeige, dass $(H,[\cdot,\cdot])$ ein Hilbertraum ist.
\end{exercise}
\begin{solution}
  Die positive Definitheit folgt direkt aus der zweiten Voraussetzung.

  Sei $(x_n)_{n \in \mathbb{N}}$ eine Cauchyfolge bezüglich der von $[.,.]$ induzierten Norm. Es gilt nach Voraussetzung
  \begin{align}
      \|x_n - x_k\|_H \leq \frac{\|x_n - x_k\|_{[.,.]}}{\sqrt{m}};
  \end{align}
  die Folge hat also einen Grenzwert in $H$, weil sie auch bezüglich der ursprünglichen Norm eine Cauchyfolge ist.

  Nun ist $(H, [.,.])$ also ein Hilbertraum. Weiters ist das ursprüngliche Skalarprodukt eine beschränkte Sesquilinearform,
  weil gilt
\begin{align}
    (x, y)_H \leq \|x\|_H \|y\|_H \leq \frac{1}{m} \|x\|_{[.,.]} \|y\|_{[.,.]}.
\end{align}
Nach dem Satz von Lax-Milgram existiert damit genau ein Operator $T \in L_b(H)$ mit
  \begin{align}
      (x, y)_H = [Tx, y] ~~~~~~~ \text{für alle~~} x, y \in H.
  \end{align}

  Weil $T$ also der Gramoperator von $(.,.)_H$ bezüglich $[.,.]$ und $G$ jener von $[.,.]$ bezüglich $(.,.)_H$ ist, gilt für alle $x, y \in H$

  \begin{itemize}
      \item $(x, y)_H = [Tx, y] = (GTx, y)_H,$
      \item $[x, y] = (Gx, y)_H = [TGx, y].$
  \end{itemize}

  Die erste Gleichung ist äquivalent zu $(y, x)_H = (y, GTx)_H.$

  Ist $\Phi$ die Abbildung $y \mapsto (x \mapsto (x, y)_H)$, dann gilt also $\Phi(x) = \Phi(GTx)$. Nach Proposition 3.2.5 ist $\Phi$ bijektiv, insbesondere injektiv -- damit ist $x = GTx$ für alle $x$.

  Aus der zweiten Gleichung folgt, mit demselben Argument angewandt auf den Hilbertraum $(H, [.,.])$, dass $x = TGx.$
\end{solution}
