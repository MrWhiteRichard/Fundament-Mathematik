\begin{exercise}

Man zeige:
Sei $H$ ein Hilbertraum, $A: H \to H$ linear und symmetrisch, also

\begin{align*}
  (A x, y) = (x, A y),
  \qquad
  x, y \in H.
\end{align*}

Dann ist $A$ stetig. \\

Die Vollständigkeit spielt hier eine entscheidende Rolle: Betrachte den Vektorraum $X$ aller Funktionen $f: \R \to \C$, die beliebig oft stetig differenzierbar sind und außerhalb des Intervalls $[0, 1]$ verschwinden, und versehe ihn mit dem inneren Produkt

\begin{align*}
  (f, g)
  :=
  \Int[0][1]{f(t) \overline{g(t)}}{t},
  \enspace
  f, g \in X.
\end{align*}

D.h. wir betrachten damit jenen Teilraum von $L^2(0,1)$, der aus $C^\infty$-Funktionen besteht, die am Rand verschwinden.
Da $X$ in $L^2(0,1)$ dicht liegt, aber sicher $X \neq L^2(0,1)$ ist, kann $X$ nicht vollständig sein. \\

Zeige, dass der Operator

\begin{align*}
  A:
  \begin{cases}
    X    & \to X \\
    f(t) & \mapsto if^{\prime}(t)
  \end{cases}
\end{align*}

in diesem Sinne symmetrisch ist, aber nicht stetig. \\

Die Aussage dieses Beispiels ist einer der ersten Sätze der Funktionalanalysis (Hellinger, Töplitz 1910).

\end{exercise}

\begin{solution}

\phantom{}

\begin{enumerate}

  \item
  $H$ ist, mit der vom Skalarprodukt induzierten Norm, ein Banachraum, also auch lokalkonvexer topologischer Vektorraum.
  Laut Korollar 5.2.7, operiert $H^\prime$ damit punktetrennend auf $H$.
  Laut Proposition 3.2.5, ist

  \begin{align*}
    H^\prime
    =
    \Bbraces
    {
      f_y:
      H \to \C:
      x \mapsto (x, y):
      y \in H
    }.
  \end{align*}

  Sei $(x_n)_{n \in \N}$ eine Null-Folge aus $H$ und $y \in H$.
  Wegen der Symmetrie von $A$ und der Stetigkeit von $f_y$, muss

  \begin{align*}
    \lim_{n \to \infty} f_y(A x_n)
    =
    \lim_{n \to \infty} (A x_n, y)
    =
    \lim_{n \to \infty} (x_n, A y)
    =
    (\lim_{n \to \infty} x_n, A y)
    =
    (0, A y) = 0
  \end{align*}

  Laut Korollar 4.4.4, ist $A$ also beschränkt.

  \includegraphicsboxed{4.4.4.jpg}

  \item
  \begin{itemize}

    \item
    \Quote{Symmetrie}:
    $\Forall f, g \in X:$

    \begin{align*}
      (f, A g)
      & =
      \Int[0][1]{f \overline{i g^\prime}}{\lambda}
      =
      \Int[0][1]{f (-i) \overline{g}^\prime}{\lambda}
      =
      -i \pbraces
      {
        \underbrace{f(1) \overline{y}(1)}_0 -
        \underbrace{f(0) \overline{g}(0)}_0 -
        \Int[0][1]{f^\prime \overline{g}}
      } \\
      & =
      \Int[0][1]{i f^\prime \overline{g}}{\lambda}
      =
      \Int[0][1]{A f \overline{g}}{\lambda}
      =
      (A f, g)
    \end{align*}

    \item
    \Quote{Nicht-Stetigkeit}:

    \begin{align*}
      \graph{A} :=
      \Bbraces
      {
        (h, A h) \in X^2:
        h \in X
      }
    \end{align*}

    Angenommen, $A$ wäre stetig.
    Sei $(f_n)_{n \in \N}$ eine Cauchy-Folge in $X$, die nicht konvergiert.
    Sie konvergiert aber gegen ein $f$ in $L^2$.
    Weil $A$ stetig ist, ist auch $(A f_n)_{n \in \N}$ eine Cauchy-Folge, konvergiert also gegen ein $g$ in $L^2$. \\

    Diese Konvergenzen sind äquivalent zur komponentenweise Konvergenz in der Produkt-Topologie.

    \begin{align*}
      \underbrace{(f_n, A f_n)}_{\in \graph{A}}
      \xrightarrow{n \to \infty}
      \underbrace{(f, g)}_{\notin \graph{A}}
    \end{align*}

    Damit, wäre aber der $\graph{A}$ nicht abgeschlossen, und Lemma 4.4.1 widerlegt.
    Widerspruch!
    \includegraphicsboxed{4.4.1}
  \end{itemize}

\end{enumerate}

\end{solution}
