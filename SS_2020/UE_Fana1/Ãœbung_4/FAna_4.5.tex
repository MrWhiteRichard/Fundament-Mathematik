\begin{exercise}
Man zeige: Sei $H$ ein Hilbertraum, $A: H \to H$ linear und symmetrisch, also
\begin{align*}
  (Ax,y) = (x,Ay), \qquad x,y \in H.
\end{align*}
Dann ist $A$ stetig. \\
Die Vollständigkeit spielt hier eine entscheidende Rolle: Betrachte den Vektorraum $X$
aller Funktionen $f: \R \to \C$, die beliebig oft stetig differenzierbar sind und
außerhalb des Intervalls $[0,1]$ verschwinden, und versehe ihn mit dem inneren Produkt
\begin{align*}
  (f,g) := \int_0^1 f(t)\overline{g(t)}dt, \qquad f,g \in X.
\end{align*}
Wir betrachten damit jenen Teilraum von $L^2(0,1)$, der aus $C^{\infty}$-Funktionen
besteht, die am Rand verschwinden. Da $X$ in $L^2(0,1)$ dicht liegt, aber sicher
$X \neq L^2(0,1)$ ist, kann $X$ nicht vollständig sein. \\
Zeige, dass der Operator
\begin{align*}
  A: \begin{cases}
    X &\to X \\
    f(t) &\mapsto if^{\prime}(t)
  \end{cases}
\end{align*}
in diesem Sinne symmetrisch ist, aber nicht stetig.
\end{exercise}
\begin{solution}
Wir zeigen, dass
\begin{align*}
  \graph A := \{(x,Ax) \in H \times H: x \in H\}
\end{align*}
abgeschlossen in $H \times H$ ist. Wir versehen dabei $H \times H$ mit der Summennorm.
Wir müssen also zeigen, dass für jede
Folge $(x_n)_{n \in \N}$ von Punkten $x_n \in H$, für die die beiden Limiten
$x := \lim_{n \to \infty} x_n, y:= \lim_{n \to \infty} Ax_n$ existieren, gilt $y = Ax$.
\begin{align*}
    (x,Ax) = (Ax,x) \\
    (x,y) = (\lim_{n \to \infty} x_n, \lim_{n \to \infty} Ax_n) \\
    = \lim_{n \to \infty}(x_n, Ax_n) \\
    =\lim_{n \to \infty}(Ax_n, x_n) \\
    = \dots = (y,x) \\
   (A\lim_{n \to \infty} x_n, y)(Ax,y) = (x,Ay) = (x, A\lim_{n \to \infty} Ax_n)
\end{align*}
\begin{align*}
  \lim_{n \to \infty} (Ax_n - y,Ax_n -y)
  &= \lim_{n \to \infty} (Ax_n,Ax_n) - (y,Ax_n) - (Ax_n,y) +(y,y)
  = \lim_{n \to \infty} (Ax_n,Ax_n) - (y,Ax_n) - (Ax_n,y) +(y,y) \\
  &= \lim_{n \to \infty} (Ax_n,Ax_n) - (\lim_{m \to \infty} Ax_m,Ax_n) - (Ax_n,\lim_{m \to \infty} Ax_m) +(y,y)
\end{align*}
Wir zeigen zuerst, dass $X$ in $L^2(0,1)$ dicht liegt. Dies folgt mit dem
Approximationssatz (Kusoltisch 30.4.3.52)
a\end{solution}
