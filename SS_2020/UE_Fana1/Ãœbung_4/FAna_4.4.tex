\begin{exercise}

Sei $H$ ein Hilbertraum.
Zeige:
Ist $(P_n)_{n \in \N}$ eine Folge von orthogonalen Projektionen $(P_n \neq 0)$, für die gilt

\begin{align*}
  \ran P_n ~\bot~ \ran P_m, \qquad n \neq m,
\end{align*}

und ist $\alpha = (\alpha_n)_{n \in \N} \in \ell^{\infty}$, mit $\alpha_n \neq 0, n \in \N$, so ist für jedes $x \in H$ die Reihe

\begin{align*}
  Ax := \sum_{n=1}^{\infty} \alpha_n P_n x
\end{align*}

konvergent.
Es gilt $A \in \mathcal{B}(H), \|A\| = \norm[\infty]{\alpha}$.

Bestimme $\ker A$. Ist $\alpha_n \to 0$, so konvergiert die Reihe in der Operatornorm.
Gilt auch die Umkehrung?

\end{exercise}

\begin{solution}

Für $n \in \N$, ist $\ker{P_n}$ das orthogonale Komplement von $\ran P_n$.
Wegen $\ran{P_m} \bot \ran{P_n} \bot \ker{P_n}$ gilt also $\ran P_m \subseteq \ker P_n$.
Somit, muss $P_n P_m = 0$.

Wir wissen, dass für orthogonale Projektionen $P_1$ und $P_2$ mit dieser Eigenschaft $P_1 + P_2$ wieder eine orthogonale Projektion ist, und $\ran{(P_1 + P_2)} = \ran{P_1} + \ran{P_2}$.

Die endlichen Summen der Projektionen $(P_n)_{n \in \N}$ sind also wieder orthogonale Projektionen.

\begin{align*}
    \norm{\sum_{n = M+1}^N \alpha_n P_n x}^2
    \stackrel{(a)}{=}
    \sum_{n = M+1}^N \norm{\alpha_n P_n x}^2
    \stackrel{(b)}{=}
    \sum_{n = M+1}^N |\alpha_n|^2 \norm{P_n x}^2 \\
    \stackrel{(c)}{\leq}
    \max_{M+1 \leq n \leq N} |\alpha_n|^2 \sum_{n = M+1}^N \norm{P_n x}^2
    \stackrel{(d)}{=}
    \max_{M+1 \leq n \leq N} |\alpha_n|^2 \norm{\pbraces{\sum_{n = M+1}^N P_n} x}^2 \\
    \stackrel{(e)}{\leq}
    \max_{M+1 \leq n \leq N} |\alpha_n|^2 \norm{x}^2
    \stackrel{(f)}{\leq}
    \norm[\infty]{\alpha}^2 \norm{x}^2.
\end{align*}

(a), (d), (g) ist der Satz von Pythagoras.

(e), (h) gelten, weil die endliche Summe orthogonaler Projektionen mit der Eigenschaft $P_n P_m = 0$ wieder eine orthogonale Projektion ist, also Abbildungsnorm kleiner gleich $1$ hat. \\

Wir betrachten Folgendes nochmal genauer.

\begin{align*}
    \sum_{n = 1}^N \norm{P_n x}^2
    \stackrel{(g)}{=}
    \norm{\pbraces{\sum_{n = 1}^N P_n} x}^2
    \stackrel{(h)}{\leq}
    \norm{x}^2
\end{align*}

$\sum_{n = 1}^N \| P_n x \|^2$ ist also beschränkt, und somit absolut konvergent.

\begin{align*}
  \norm{\sum_{n = M+1}^N \alpha_n P_n x}^2
  \leq
  \max_{M+1 \leq n \leq N} |\alpha_n|^2 \norm{\pbraces{\sum_{n = M+1}^N P_n} x}^2
  \leq
  \norm[\infty]{\alpha} \norm{\pbraces{\sum_{n = M+1}^N P_n} x}^2
  \xrightarrow{M, N \rightarrow \infty} 0
\end{align*}

Dabei, haben die erste Ungleichung schon oben gezeigt.
Der Grenzübergang folgt aus (g).
Damit, ist die Reihe eine Cauchy-Folge, also konvergent. \\

Die erste Ungleichungskette bis (e) gelesen gibt uns folgende Abschätzung:

\begin{align*}
\left\| \sum_{n = M+1}^N \alpha_n P_n x \right\|^2 \leq \max_{M+1 \leq n \leq N} |\alpha_n|^2 \|x\|^2,
\end{align*}

also gilt in der Operatornorm

\begin{align*}
\left\| \sum_{n = M+1}^N \alpha_n P_n \right\| \leq \max_{M+1 \leq n \leq N} |\alpha_n|.
\end{align*}

Die Folge $(\alpha_n)$ ist aber eine Nullfolge, damit ist die Reihe eine Cauchyfolge in der Operatornorm. Da wir in einem vollständigen Raum sind, haben wir somit die Konvergenz bezüglich der Operatornorm gezeigt.

Mit (a)-(f) gilt

\begin{align*}
    \norm{\sum_{n = 1}^N \alpha_n P_n x}^2
    \leq
    \norm[\infty]{\alpha}^2 \norm{x}^2.
\end{align*}

Also $\|A\| \leq \norm[\infty]{\alpha}$ und somit $A \in \mathcal{B}(H)$. \\

Die Umkehrung gilt ebenfalls: Sei $(\alpha_n)$ keine Nullfolge, das heißt für jedes $\epsilon$ und jedes $n$ gibt es $M, N > n$ und ein $k$ mit $M+1 \leq k \leq N$, sodass gilt $|\alpha_k|^2 > \epsilon.$ Wegen $P_k \neq 0$ gibt es ein $x \in \ran P_k\backslash\{0\}.$ $P_k$ ist auf seinem Bild die Identität, damit gilt
\begin{align}
 \left\| \sum_{n = M+1}^N \alpha_n P_n x \right\|^2 \stackrel{b}{=} \sum_{n = M+1}^N | \alpha_n|^2 \| P_n x \|^2 \geq |\alpha_k|^2 \|x\|^2 > \epsilon \|x\|^2.
\end{align}
Es liegt also keine Konvergenz in der Operatornorm vor.
Zuletzt sehen wir, dass für $x \in \ker A$ gilt
\begin{align*}
    \sum_{n=1}^N |\alpha_n|^2 \|P_n x\|^2 =
    \left\| \sum_{n = 1}^N \alpha_n P_n x \right\|^2 \stackrel{N \rightarrow \infty}{\longrightarrow} 0.
\end{align*}

Das ist wegen $\alpha_n \neq 0$ offenbar nur dann möglich, wenn für alle $n$ gilt $\|P_n x\| = 0$, also $P_n x = 0,$ ergo $x \in \bigcap_{n \in \mathbb{N}} \ker P_n$. Da für Elemente dieses Durchschnitts die Reihe sowieso Null ist, haben wir genau den Kern von $A$ gefunden.

\end{solution}
