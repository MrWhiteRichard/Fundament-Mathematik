\begin{exercise}
Sei $X$ ein Banachraum. Dann sind äquivalent:
\begin{enumerate}[label = \roman*)]
  \item  $X$ ist reflexiv.
  \item $X^{\prime}$ ist reflexiv.
  \item Die abgeschlossene Einheitskugel $B_X$ von $X$ ist $w$-kompakt, also
  kompakt bezüglich $\sigma(X,X^{\prime})$.
\end{enumerate}
\textit{Hinweis:} Zeige mit Hilfe des Satzes von Banach-Alaoglu und der Aufgabe 06/2,
dass aus $X$ relexiv bereits $B_X, B_{X^{\prime}}$ $w$-kompakt folgt. Mit dem Satz
von Goldstine zeige, dass aus $B_X$ $w$-kompakt $\iota(B_X) = B_X$ folgt.
Für (ii) $\implies$ (i) erinnere man sich zusätzlich daran, dass der schwache
Abschluss bei konvexen Mengen gleich dem Norm-Abschluss ist.
\end{exercise}
\begin{solution}
\leavevmode \\
\begin{itemize}
  \item (i) $\implies$ (iii): \\
  Sei $X$ reflexiv. Nach dem Satz von Banach-Alaoglu
  angewandt auf den normierten Raum $(X^{\prime},\|\cdot\|)$ ist $B_{X^{\primeprime}}$
  kompakt bezüglich $\sigma(X^{\primeprime}, \iota_1(X^{\prime})$.
  Nach Aufgabe 6/2 ist $\iota: (X,\sigma(X,X^{\prime})) \to
  (\iota(X),\sigma(X^{\primeprime},\iota_1(X^{\prime}))|_{\iota(X)})$ ein Homöomorphismus
  und wegen der Reflexivität gilt $\iota(X) = X^{\primeprime}$.
  Mit Korollar 5.2.4 gilt
  \begin{align*}
    \iota^{-1}(B_{X^{\primeprime}}) &= \{x \in X: \|\iota(x)\|_{X^{\primeprime} \leq 1}
    = \{x \in X: \sup\{|f(x)|: f \in X^{\prime}, \|f\| \leq 1 \} \\
    &= \{x \in X: \|x\| \leq 1\} = B_X
  \end{align*}
  Damit ist $B_X$ als Bild einer kompakten Menge $B_{X^{\primeprime}}$ unter einer
  stetigen Abbildung wieder kompakt.
  \item (iii) $\implies$ (i): \\
  Sei also $B_X$ kompakt bezüglich $\sigma(X,X^{\prime})$. Nach dem Satz von Goldstine
  ist
  \begin{align*}
    \overline{\iota(B_X)}^{\sigma(X^{\primeprime},\iota_1(X))} = B_{X^{\primeprime}}
  \end{align*}
  und $\iota$ ist ein Homöomorphismus, also ist $\iota(B_X)$ bezüglich
  $\sigma(X^{\primeprime}, \iota_1(X^{\prime}))$ kompakt. Da
  $(X^{\primeprime}, \sigma(X^{\primeprime}, \iota_1(X^{\prime})))$ Hausdorff ist,
  ist $\iota(B_X)$ insbesondere auch abgeschlossen (Kaltenbäck Lemma 12.11.7),
  also ist $\iota(B_X) = \overline{\iota(B_X)} = B_{X^{\primeprime}}$.
  Außerdem ist $\iota$ nach Lemma 5.5.2 und Bemerkung 5.5.3 injektiv und linear.
  Sei nun $y \in X^{\primeprime}$ mit $y \neq 0$ und betrachte
  $z := \frac{y}{\|y\|_{X^{\primeprime}}} \in X^{\primeprime}, \|z\|_{X^{\primeprime}} = 1$.
  Es gilt $z \in B_{X^{\primeprime}}$, also folgt
  \begin{align*}
    \exists x \in B_X: \iota(x) = z
  \end{align*}
  Für $u := \|y\|_{X^{\primeprime}} \in X$ gilt dann
  \begin{align*}
    \iota(u) = \|y\|_{X^{\primeprime}}\iota(x) = \|y\|_{X^{\primeprime}}z
    = \|y\|_{X^{\primeprime}} \frac{y}{\|y\|_{X^{\primeprime}}} = y
  \end{align*}
  und es gilt $\iota(X) = X^{\primeprime}$, also ist $X$ reflexiv.
  \item (i) $\implies$ (ii): \\
  Sei $X$ reflexiv. Nach dem Satz von Banach-Alaoglu ist $B_{X^{\prime}}$ abgeschlossen
  bezüglich $\sigma(X^{\prime}, \iota(X)) = \sigma(X^{\prime},X^{\primeprime})$.
  Nun erhalten wir mit (iii) $\implies$ (i), dass $X^{\prime}$ relexiv ist.
  \item (ii) $\implies$ (i): \\
  Sei $X^{\prime}$ reflexiv. Dann ist mit dem Satz von Banach-Alaoglu $B_{X^{\primeprime}}$
  bezüglich $\sigma(X^{\primeprime}, \iota_1(X^{\prime})) = \sigma(X^{\primeprime}, X^{\primeprimeprime})$
  und mit dem Satz von Goldstine und der Konvexität von $\iota(B_X)$ mit Satz 5.3.8
  \begin{align*}
    \overline{\iota(B_X)}^{\|\cdot\|_{X^{\primeprime}}}
    = \overline{\iota(B_X)}^{\sigma(X^{\primeprime},X^{\primeprimeprime})} = B_{X^{\primeprime}}.
  \end{align*}
  Da $B_X$ abgeschlossen bezüglich $\|\cdot\|$ ist und
  \begin{align*}
    \iota^{-1}: (\iota(X), \|\cdot\|_{X^{\primeprime}}|_{\iota(X)}) \to (X,\|\cdot\|)
  \end{align*}
  laut Lemma 5.5.2 und Bemerkung 5.5.3
  stetig und surjektiv ist, ist $(\iota^{-1})^{-1}(B_X)$ abgeschlossen bezüglich $\|\cdot\|_{X^{\primeprime}}$,
  also ist
  \begin{align*}
    \iota(B_X) = \overline{\iota(B_X)}^{\|\cdot\|_{X^{\primeprime}}} = B_{X^{\primeprime}}.
  \end{align*}
  Wie in der Implikation (iii) $\implies$ (i) folgt dan $\iota(X) = X^{\primeprime}$.
 \end{itemize}
\end{solution}
