\begin{exercise}
Sei $X$ ein Banachraum. Dann sind äquivalent:
\begin{enumerate}[label = \roman*)]
  \item  $X$ ist reflexiv.
  \item $X^{\prime}$ ist reflexiv.
  \item Die abgeschlossene Einheitskugel $B_X$ von $X$ ist $w$-kompakt, also
  kompakt bezüglich $\sigma(X,X^{\prime})$.
\end{enumerate}
\textit{Hinweis:} Zeige mit Hilfe des Satzes von Banach-Alaoglu und der Aufgabe 06/2,
dass aus $X$ relexiv bereits $B_X, B_{X^{\prime}}$ $w$-kompakt folgt. Mit dem Satz
von Goldstine zeige, dass aus $B_X$ $w$-kompakt $\iota(B_X) = B_X$ folgt.
Für (ii) $\implies$ (i) erinnere man sich zusätzlich daran, dass der schwache
Abschluss bei konvexen Mengen gleich dem Norm-Abschluss ist.
\end{exercise}
\begin{solution}
Beweis.
\end{solution}
