\begin{exercise}
Sei $c_1 = 0, c_3 = 1$ und $c_2 \in (0,1)$ beliebig.
\begin{enumerate}[label = \textbf{\alph*)}]
  \item Welche Konvergenzordnung ist für 3-stufige Runge-Kutta-Verfahren erreichbar,
  wenn diese durch Kollokation mit diesen Kollokationspunkten erzeugt werden?
  \item Geben Sie die Butcher Tableaus dieser Verfahren an.
\end{enumerate}
\end{exercise}
\begin{solution}
\leavevmode \\
\begin{enumerate}[label = \textbf{\alph*)}]
  \item Theorem 3.28 sagt uns, dass für $3$-stufige Runge-Kutta-Verfahren
  maximal Konvergenzordnung $6$ erreichen können. Allerdings benötigen wir dafür
  Kollokationspunkte aus einer Gauß-Quadratur, welche immer
  im offenen Intervall $(0,1)$ liegen und nicht auf den Randpunkten.
  Daher können wir nicht die maximale Konvergenzordnung erwarten.
  Also setzen wir die Lagrange Basispolynome vorerst allgemein an.
  \begin{align*}
    L_1(x) &= \frac{(x-c_2)(x-1)}{c_2} \\
    L_2(x) &= \frac{x(x-1)}{c_2(c_2 - 1)} \\
    L_3(x) &= \frac{x(x - c_2)}{1 - c_2}
  \end{align*}
  Die zugehörige Quadraturformel lautet
  \begin{align*}
    \int_0^1 f(x) dx &\approx Q(f) := \int_0^1 L_1(x) dx f(0) + \int_0^1 L_2(x) dx f(c_2) +
    \int_0^1 L_1(x) dx f(1) \\
    &= \frac{3c_2 -1}{6c_2} f(0) + \frac{1}{6c_2(1 - c_2)} f(c_2) +
    \frac{2- 3c_2}{6(1 - c_2)} f(1)
  \end{align*}
  Jetzt wollen wir durch geschickte Wahl von $c_2$ die Ordnung dieser Quadratur maximieren.
  Dafür betrachten wir die Monombasis.
  \begin{align*}
    Q(x^0) &= \frac{3c_2 - 1}{6c_2} + \frac{1}{6c_2(1 - c_2)} +
    \frac{2- 3c_2}{6(1 - c_2)} \stackrel{!}{=} 1 \\
    &\iff \frac{(3c_2 - 1)(1 - c_2) + 1 + (2- 3c_2)c_2}{6c_2(1 - c_2)} \stackrel{!}{=} 1 \\
    &\iff \frac{-3c_2^2 + 3c_2 + c_2 + 2c_2 - 3c_2^2}{6c_2(1 - c_2)} \stackrel{!}{=} 1 \\
    &\iff \frac{-6c_2^2 + 6c_2}{6c_2(1 - c_2)} \stackrel{!}{=} 1 \\
    &\iff c_2 \notin \{0,1\}
  \end{align*}
  \begin{align*}
    Q(x^1) &= \frac{c_2}{6c_2(1 - c_2)} +
    \frac{2- 3c_2}{6(1 - c_2)} \stackrel{!}{=} \frac{1}{2} \\
    &\iff \frac{c_2 + c_2(2 - 3c_2)}{6c_2(1 - c_2)} \stackrel{!}{=} \frac{1}{2} \\
    &\iff \frac{3c_2 - 3c_2^2}{6c_2(1 - c_2)} \stackrel{!}{=} \frac{1}{2} \\
    &\iff c_2 \notin \{0,1\}  \\
    Q(x^2) &= \frac{c_2^2}{6c_2(1 - c_2)} +
    \frac{2- 3c_2}{6(1 - c_2)} \stackrel{!}{=} \frac{1}{3} \\
    &\iff \frac{c_2^2 + c_2(2 - 3c_2)}{6c_2(1 - c_2)} \stackrel{!}{=} \frac{1}{3} \\
    &\iff \frac{2c_2 - 2c_2^2}{6c_2(1 - c_2)} \stackrel{!}{=} \frac{1}{3} \\
    &\iff c_2 \notin \{0,1\}  \\
    Q(x^3) &= \frac{c_2^3}{6c_2(1 - c_2)} +
    \frac{2- 3c_2}{6(1 - c_2)} \stackrel{!}{=} \frac{1}{4} \\
    &\iff \frac{c_2^3 + c_2(2 - 3c_2)}{6c_2(1 - c_2)} \stackrel{!}{=} \frac{1}{4} \\
    &\iff \frac{c_2^2 - 3c_2 + 2}{6(1 - c_2)} \stackrel{!}{=} \frac{1}{4} \\
    &\iff 4c_2^2 - 12c_2 + 8 \stackrel{!}{=} 6(1 - c_2)\\
    &\iff 4c_2^2 - 6c_2 + 2 \stackrel{!}{=} 0\\
    &\iff c_2^2 - \frac{3}{2} c_2 + \frac{1}{2} \stackrel{!}{=} 0\\
    &\iff c_2 = \frac{1}{2}.\\
  \end{align*}
  Also hat unsere Quadratur für $c_2 = \frac{1}{2}$ zumindest Ordnung $4$.
  Polynome von Grad $4$ werden allerdings im Allgemeinen nicht mehr exakt integriert, da
  \begin{align*}
    Q(x^4) &= \frac{1}{24} +
    \frac{1}{6} = \frac{5}{24} \neq \frac{1}{5}.
  \end{align*}
  Also hat die zugehörige Kollokationsmethode ebenso Konsistenzordnung $4$.
  Die Kollokationspunkte stimmen dabei mit denen der abgeschlossenen Newton-Cotes-Formel $Q_2$
  überein.
  \item Das zugehörige Butcher Tableau haben wir bereits letzte Woche bestimmt:
  \begin{align*}
    b &= \begin{pmatrix}
      \int_0^1 L_1(x) \\ \int_0^1 L_2(x) \\ \int_0^1 L_3(x)
    \end{pmatrix}
    = \begin{pmatrix}
      \int_0^1 2x^2 - 3x + 1 dx \\
      \int_0^1 -4x^2 + 4x dx \\
      \int_0^1 2x^2 - x dx
    \end{pmatrix}
    = \begin{pmatrix}
       \frac{2}{3} - \frac{3}{2} + 1 \\-\frac{4}{3} + 2 \\ \frac{2}{3} - \frac{1}{2}
    \end{pmatrix}
    = \frac{1}{6}\begin{pmatrix}
      1 \\ 4 \\ 1
    \end{pmatrix} \\
    A &= \begin{pmatrix}
      \int_0^0 L_1(x) & \int_0^0 L_2(x) & \int_0^0 L_3(x)\\
      \int_0^{1/2} L_1(x) & \int_0^{1/2} L_2(x) & \int_0^{1/2} L_3(x)\\
      \int_0^1 L_1(x) & \int_0^1 L_2(x) & \int_0^1 L_3(x)
    \end{pmatrix}
    = \frac{1}{24}\begin{pmatrix}
      0 & 0 & 0\\
      5 & 8 & -1\\
      4 & 16 & 4
    \end{pmatrix}
  \end{align*}
\end{enumerate}
\end{solution}
