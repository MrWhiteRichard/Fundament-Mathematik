\begin{exercise}

Man spricht von einem autonomen Anfangswertproblem, wenn die rechte Seite der Differentialgleichung
nicht explizit von der Zeit abhängt, d.h. wenn $Y$ Lösung des Anfangswertproblems

\begin{align}
\label{Autonom} \tag{2}
  \begin{array}{lll}
    Y'(t)=F(Y(t)), &t\in J, &Y(t_0)=Y_0
  \end{array}
\end{align}

ist. Jedes normale Anfangswertproblem

\begin{align}
  \label{Normal} \tag{3}
  \begin{array}{lll}
    y=f(t,y(t)), &t\in J, &y(t_0)=y_0
  \end{array}
\end{align}

kann äquivalent in ein autonomes Anfangswertproblem mit
$Y(t):=(t,y(t))^T, Y_0:=(t_0,y_0)^T$ und $F(x):=(1,f(x))^T$
umgeformt werden. Ein Einschrittverfahren heisst invariant gegenüber Autonomisierung,
wenn es angewendet auf (\ref{Autonom}) für beliebiges $f$ exakt die gleichen Approximationen
erzeugt wie bei Anwendung auf (\ref{Normal}). \newline

Zeigen Sie: Ein explizites, $s$-stufiges Runge-Kutta-Verfahren ist genau dann invariant
gegenüber Autonomisierung, wenn gilt

\begin{align} \label{Bedingung} \tag{4}
  \begin{array}{ll}
    c_j = \sum_{i=1}^{j-1} a_{ji}, &j=1,...,s
  \end{array}
\end{align}
\end{exercise}

\begin{solution}

Haben wir zuerst ein $s$-stufiges Runge-Kutta-Verfahren gegeben das (\ref{Bedingung})
erfüllt. Das Verfahren liefert sicher dann die gleichen Approximationen, wenn die Inkremente
des autonomen Verfahrens gleich der des \glqq normalen\grqq Verfahrens sind.

Seien dazu $K$ die Inkremente des autonomen Verfahrens und $k$ die Inkremente
des \glqq normalen\grqq. Wir führen den Beweis durch Induktion, für den Induktionsanfang gilt:

\begin{align*}
  K_1= F(Y) = (1,f(Y))^T = (1,f(t,y(t)))^T
\end{align*}

Die zweite Komponente stimmt also genau mit $k_1$ überein (das dass auch genau das ist
was wir wollen, sieht man, wenn man in das autonome AWP einsetzt:
$(1,f(t,y(t)))^T = (1,y'(t))^T$).
Machen wir nun den Induktionsschritt $n \rightarrow n+1$:

\begin{align*}
  K_{n+1} &= F(Y+h\sum_{j=1}^n A_{n+1,j}K_j) \\
  &= (1,f(Y+h\sum_{j=1}^n A_{n+1,j}(1,k_j)^T)) \\
  &= (1,f(t+h\sum_{j=1}^n A_{n+1,j},y+h\sum_{j=1}^n A_{n+1,j}k_j)) \\
  &= (1,k_{n+1})
\end{align*}

Haben wir nun umgekehrt ein gegenüber Autonomisierung invariantes, explizites,
$s$-stufiges Runge-Kutta-Verfahren gegeben. Die Invarianz muss insbesondere für
die Funktion $f(t,y) \mapsto t$ gelten.
Für diese gilt:

\begin{align*}
  (1,k_i) &= (1,f(t+c_ih,y+h\sum_{j=1}^{i-1}A_{ij}k_j)) \\
  &= (1,t+c_ih) \stackrel{!}{=} (1,t+h\sum_{j=1}^{i-1}A_{ij}) \\
  &= F(Y+h\sum_{j=1}^{i-1} A_{ij}K_j)\\
  &= K_i
\end{align*}

Somit ist diese Bedingung auch notwendig.
\end{solution}
