\begin{exercise}
  Ein explizites Einschrittverfahren mit Inkrementfunktion $\Phi(t,y,h)$
  kann mit der sogenannten diskreten Evolution

  \begin{align}
    \Psi^{t,t+h}y := y + h\Phi(t,y,h)
  \end{align}
  formuliert werden durch
  \begin{align}
    y_{j+1}=\Psi^{t_j,t_j + h_j} y_j
  \end{align}

  Es heißt reversibel, wenn gilt $\Psi^{t+h,t}\Psi^{t,t+h}y = y$
  für alle zulässigen $(t,y)$ und alle hinreichend kleinen $h$. Zeigen Sie, dass
  es kein konsistentes, explizites Runge-Kutta-Verfahren gibt, welches für jedes
  beliebige Anfangswertproblem reversibel ist.

  Hinweis: Beweisen Sie zunächst, dass $\Psi^{0,h} y_0$ für ein $s$-stufiges,
  explizites Runge-Kutta-Verfahren ein Polynom der Ordnung $s$ in $h$ ist, wenn
  das Verfahren auf die Differentialgleichung

  \begin{align}
    y'(t)=y(t), y(0)=y_0,
  \end{align}

  angewendet wird.

  Zusatzinformation: Bei reversiblen Einschrittverfahren führt ein Schritt des
  Verfahrens mit positiver Schrittweite $h$ gefolgt von einem Schritt des Verfahrens
  mit negativer Schrittweite $-h$ wieder auf den Anfangswert.
\end{exercise}

\begin{solution}
  Wir zeigen zunächst den Hinweis. Dazu betrachten wir die Inkremente des Verfahrens
  für unsere Differentialgleichung:

  \begin{align*}
    k_i = y_0 + h \sum_{j=1}^{i-1}A_{ij}k_j
  \end{align*}

  Wir zeigen nun per Induktion, dass die Inkremente $k_i$ in $h$ ein Polynom vom Grad
  $i-1$ sind. Wir sehen sofort, dass die Behauptung für $i=1$ stimmt. Für den
  Induktionsschritt $i \rightarrow i+1$ betrachten wir:

  \begin{align*}
    k_{i+1} = y_0 + h \sum_{j=1}^iA_{ij}k_j
  \end{align*}

  Wir sehen, dass die Summe nach Induktionsvoraussetzung ein Polynom vom Grad
  $i-1$ ist, durch die Multiplikation mit $h$ haben wir es also insgesamt mit einem
  Polynom des Grades $i$ zu tun, womit unsere Induktion abgeschlossen ist.

  Damit ist
  \begin{align*}
    \Phi(t,y,h) = \sum_{j=1}^s b_j k_j
  \end{align*}
  Ein Polynom vom Grad $s-1$ und wenn wir $\Psi$ aus $y_{0}$ anwenden, erhalten wir also
  wirklich ein Polynom vom Grad $s$.

  Aus der Induktion und der Definition von $\Psi$ bemerkt man, dass man den Faktor
  $y_0$ aus dem Polynom herausheben kann, es gibt also ein Polynom $P \in \Pi_s$
  sodass

  \begin{align*}
    \Psi^{0,h} y_0 = P(h)y_0 \\
    \Psi^{0,-h} y_0 = P(-h)y_0
  \end{align*}

  Damit erhalten wir insgesamt für die Reversiblität in $(0,y_0)$:

  \begin{align*}
    \Psi^{h,0}\Psi^{0,h}y_0 = \Psi^{0,-h}\Psi^{0,h}y_0
    = P(-h)P(h)y_0
  \end{align*}

  Somit muss $P(-h)P(h) = 1$ für hinreichend kleine $h$. Das bedeutet, dass $P$
  das konstante $1$-Polynom sein muss.

  Nun soll unser Verfahren konsistent sein, das bedeutet, dass es

  \begin{align*}
    \forall t \in \left[t_0 T\right): \lim_{h\rightarrow 0^+}
    \frac{\tau(t,y,h)}{h} = 0
  \end{align*}
  erfüllt, wobei $\tau$ definiert ist durch

  \begin{align*}
    \tau(t,y,h) := \norm{y(t+h)-[y(t)+h\Phi(t,y(t),h)]}{} =
    \norm{y(t+h)- \Psi^{t,t+h}y}{}
  \end{align*}

  Wenn wir zusätzlich zu unseren bisherigen Voraussetzungen noch annehmen, dass
  $y_0 \neq 0$ ergibt sich ein Wiederspruch zur Konsistenz:

  \begin{align*}
    \lim_{h\rightarrow 0^+} \frac{\tau(0,y_0,h)}{h}
    &=\lim_{h\rightarrow 0^+} \frac{\norm{y(h)-P(h)y_0}{}}{h} \\
    &=\vbraces{y_0} \lim_{h\rightarrow 0^+} \frac{e^h-1}{h} = \vbraces{y_0}
  \end{align*}

  Es kann also kein konsistentes, explizites Runge-Kutta-Verfahren geben welches
  für dieses Anfangswertproblem reversibel ist und damit insbesondere nicht für alle.
\end{solution}
