\begin{exercise}
  Erweitern Sie das Programm zu Aufgabe $6$ um eine Schrittweitensteuerung mit
  eingebettetem RK-Verfahren (RK5(4)) und wenden Sie es auf das Räuber-Beute-
  Modell an. Vergleichen Sie die Schrittweiten mit der Lösung. \\

  Hinweis: Es gibt einen Fehler im Vorlesungsskript zum RK5(4)-Schema. Verwenden
  Sie bitte das folgende Schema.

  \begin{align*}
    \begin{array}{c|ccccccc}
    0 & & & & & & &                                                         \\
    \nicefrac{1}{5} & \nicefrac{1}{5} & & & & & &                                 \\
    \nicefrac{3}{10} & \nicefrac{3}{40} & \nicefrac{9}{40} & & & & &                 \\
    \nicefrac{4}{5} & \nicefrac{44}{45} & -\nicefrac{56}{15} & \nicefrac{32}{9} & & & & \\
    \nicefrac{8}{9} & \nicefrac{19372}{6561} & -\nicefrac{25360}{2187} &
    \nicefrac{64448}{6561} & -\nicefrac{212}{729} & & &                           \\
    1 & \nicefrac{9017}{3168} & -\nicefrac{355}{33} & \nicefrac{46732}{5247} &
    \nicefrac{49}{176} & -\nicefrac{5103}{18656} & &                              \\
    1 & \nicefrac{35}{384} & 0 & \nicefrac{500}{1113} & \nicefrac{125}{192} &
    -\nicefrac{2187}{6784} & \nicefrac{11}{84} &                                  \\
    \hline
     & \nicefrac{35}{384} & 0 & \nicefrac{500}{1113} & \nicefrac{125}{192} &
    -\nicefrac{2187}{6784} & \nicefrac{11}{84} & 0                                \\
     & \nicefrac{5179}{57600} & 0 & \nicefrac{7571}{16695} & \nicefrac{393}{640} &
     -\nicefrac{92097}{339200} & \nicefrac{187}{2100} & \nicefrac{1}{40}
    \end{array}
  \end{align*}

\end{exercise}

\begin{solution}
  Siehe .py!
\end{solution}
