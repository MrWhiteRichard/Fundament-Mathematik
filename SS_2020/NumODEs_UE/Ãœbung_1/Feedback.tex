Beispiel 3: Sehr gut. Beim expliziten Eulerverfahren könnte man zwar noch unterscheiden in ‘Limes existiert nicht für h\lambda = -2’ und ‘divergiert für h\lambda < -2’, der wesentliche Punkt (dass eines immer konvergiert, das andere nur bedingt) ist aber offensichtlich verstanden worden.

Beispiel 5: die Berechnung für die Lipschitzkonstante passt. Allerdings ist A symmetrisch, weshalb ||A||=Spektralradius gilt. Die Lipschitzkonstante ist also der betragsmäßig größte Eigenwert. Wenn Sie die Diagonalmatrix der Eigenwerte nicht mit dem (positiven) Spektralradius abschätzen, sondern mit dem (eventuell negativen) größten Eigenwert, sehen Sie, dass die einseitige Lipschitzkonstante eben dieser größte Eigenwert ist und damit doch L <= L_+ gilt.
