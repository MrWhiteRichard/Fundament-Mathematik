\begin{exercise}
  Sei $\Psi^h$ der diskrete Fluss des expliziten Euler-Verfahrens. Welche bekannten
  Verfahren ergeben sich aus $(\Psi^{\nicefrac{h}{2}})^*\circ\Psi^{\nicefrac{h}{2}}$
  und $\Psi^{\nicefrac{h}{2}}\circ(\Psi^{\nicefrac{h}{2}})^*$? Welche Ordnung
  haben diese Verfahren?
\end{exercise}

\begin{solution}
\begin{align*}
  y_1 &= (\Psi^{\nicefrac{h}{2}})^*\circ\Psi^{\nicefrac{h}{2}}(y_0)
  = (\Psi^{\nicefrac{h}{2}})^*(y_0 + \frac{h}{2}f(y_0))
  = (\Psi^{-\nicefrac{h}{2}})^{-1}(y_0 + \frac{h}{2}f(y_0)) \\
  &\iff \Psi^{-\nicefrac{h}{2}}(y_1) = y_0 + \frac{h}{2}f(y_0) \\
  &\iff y_1 - \frac{h}{2}f(y_1) = y_0 + \frac{h}{2}f(y_0) \\
  &\iff y_1 = y_0 + \frac{h}{2}(f(y_0) + f(y_1)).
\end{align*}
Wir erhalten die implizite Trapezregel, welche bekanntlich Ordnung 2 hat,
was wir auch erwarten würden, da reversible Methoden gerade Konvergenzordnung haben.
\begin{align*}
  y_{0} &= \Psi^{-\nicefrac{h}{2}}(y_{\nicefrac{1}{2}}) = y_{\nicefrac{1}{2}} - \frac{h}{2}f(y_{\nicefrac{1}{2}})
  \implies y_{\nicefrac{1}{2}} = y_0 + \frac{h}{2}f(y_{\nicefrac{1}{2}})\\
  y_1 &= \Psi^{\nicefrac{h}{2}}\circ(\Psi^{\nicefrac{h}{2}})^*(y_0)
  = \Psi^{\nicefrac{h}{2}}((\Psi^{-\nicefrac{h}{2}})^{-1}(y_0))
  = \Psi^{\nicefrac{h}{2}}(y_0 + \frac{h}{2}f(y_{\nicefrac{1}{2}}))
  = y_0 + \frac{h}{2}f(y_{\nicefrac{1}{2}}) + \frac{h}{2}f(y_0 + \frac{h}{2}f(y_{\nicefrac{1}{2}})) \\
  &= y_0 + hf(y_{\nicefrac{1}{2}}) \\
  y_1 + y_0 &= 2y_0 + hf(y_{\nicefrac{1}{2}}) = 2y_{\nicefrac{1}{2}} \implies
  y_1 = y_0 + hf\left(\frac{y_0+y_1}{2}\right)
\end{align*}
und wir erhalten die implizite Mittelpunktsregel mit Ordnung 2.
\end{solution}
