\begin{exercise}
Berechnen Sie sich die Butcher-Schemata der 3-stufigen, impliziten Runge-Kutta-Verfahren,
welche als Kollokationspunkte die Stützstellen einer offenen, beziehungsweise
abgeschlossenen Newton-Cotes Formel verwenden.
\end{exercise}
\begin{solution}
Die offenen, beziehungsweise abgeschlossenen Newton-Cotes Formeln mit jeweils
drei Stützstellen lauten
\begin{align*}
  Q_1(f) &:= \frac{1}{6}(f(0) + 4f(1/2) + f(1))\\
  Q_2(f) &:= \frac{1}{3}(2f(1/4) - f(1/2) + 2f(3/4))\\.
\end{align*}
Die zugehörigen Kollokationspolynome $q_l$ sind also kubisch.
Betrachten wir zunächst die abgeschlossene Newton-Cotes Formel mit den Stützstellen
\begin{align*}
  c_1 = 0, \qquad c_2 = \frac{1}{2}, \qquad c_3 = 1
\end{align*}
Mittels Theorem 3.23 können wir uns daraus schon die Matrix $A$ und den Vektor $b$
nach den Formeln
\begin{align}
  A_{ij} &:= \int_0^{c_i} L_j(x) dx \\
  b_j &:= \int_0^1 L_j(x) dx
\end{align}
berechnen. Die benötigten Lagrange Basispolynome lauten
\begin{align*}
  L_1(x) &= 2(x - 1/2)(x - 1) = 2x^2 - 3x + 1\\
  L_2(x) &= -4x(x - 1) = -4x^2 + 4x\\
  L_3(x) &= 2x(x - 1/2) = 2x^2 - x.
\end{align*}
Also erhalten wir
\begin{align*}
  b &= \begin{pmatrix}
    \int_0^1 L_1(x) \\ \int_0^1 L_2(x) \\ \int_0^1 L_3(x)
  \end{pmatrix}
  = \begin{pmatrix}
    \int_0^1 2x^2 - 3x + 1 dx \\
    \int_0^1 -4x^2 + 4x dx \\
    \int_0^1 2x^2 - x dx
  \end{pmatrix}
  = \begin{pmatrix}
     \frac{2}{3} - \frac{3}{2} + 1 \\-\frac{4}{3} + 2 \\ \frac{2}{3} - \frac{1}{2}
  \end{pmatrix}
  = \frac{1}{6}\begin{pmatrix}
    1 \\ 4 \\ 1
  \end{pmatrix} \\
  A &= \begin{pmatrix}
    \int_0^0 L_1(x) & \int_0^0 L_2(x) & \int_0^0 L_3(x)\\
    \int_0^{1/2} L_1(x) & \int_0^{1/2} L_2(x) & \int_0^{1/2} L_3(x)\\
    \int_0^1 L_1(x) & \int_0^1 L_2(x) & \int_0^1 L_3(x)
  \end{pmatrix}
  = \frac{1}{24}\begin{pmatrix}
    0 & 0 & 0\\
    5 & 8 & -1\\
    4 & 16 & 4
  \end{pmatrix}
\end{align*}
Für die entsprechende offene Newton-Cotes Formel haben wir die Stützstellen
\begin{align*}
  c_1 = \frac{1}{4}, \qquad c_2 = \frac{1}{2}, \qquad c_3 = \frac{3}{4}
\end{align*}
gegeben. Die neuen Lagrange Basispolynome lauten
\begin{align*}
  L_1(x) &= \frac{(x - 2/4)(x - 3/4)}{(1/4-2/4)(1/4-3/4)} = 8x^2 - 10x + 3 \\
  L_2(x) &= \frac{(x - 1/4)(x - 3/4)}{(2/4-1/4)(2/4-3/4)} = -16x^2 + 16x - 3 \\
  L_2(x) &= \frac{(x - 1/4)(x - 1/2)}{(3/4-1/4)(3/4-1/2)} = 8x^2 - 6x + 1
\end{align*}
Analog zu vorhin berechnen wir den Vektor $b$ und die Matrix $A$
\begin{align*}
b &= \begin{pmatrix}
  \int_0^1 L_1(x) \\ \int_0^1 L_2(x) \\ \int_0^1 L_3(x)
\end{pmatrix}
= \begin{pmatrix}
  \int_0^1 8x^2 - 10x + 3 dx \\
  \int_0^1 -16x^2 + 16x - 3 dx \\
  \int_0^1 8x^2 - 6x + 1 dx
\end{pmatrix}
= \frac{1}{3}\begin{pmatrix}
  2 \\ -1 \\ 2
\end{pmatrix} \\
A &= \begin{pmatrix}
  \int_0^{1/4} L_1(x) & \int_0^{1/4} L_2(x) & \int_0^{1/4} L_3(x)\\
  \int_0^{2/4} L_1(x) & \int_0^{2/4} L_2(x) & \int_0^{2/4} L_3(x)\\
  \int_0^{3/4} L_1(x) & \int_0^{3/4} L_2(x) & \int_0^{3/4} L_3(x)
\end{pmatrix}
= \frac{1}{48}\begin{pmatrix}
  23 & -16 & 5\\
  28 & -8 & 4\\
  27 & 0 & 9
\end{pmatrix}.
\end{align*}
\end{solution}
