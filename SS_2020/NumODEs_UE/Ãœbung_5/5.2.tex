\begin{exercise}
Sei $x_0,\dots,x_r \in [0,h]$ mit $x_i \neq x_j$ für $1 \leq i \leq j \leq r$.
Gesucht sei für beliebiges $g \in C^r([a,b])$ die Lösung $p \in \Pi_r$ von
\begin{align}
  p(x_0) = g(x_0), \qquad p^{\prime}(x_j) = g^{\prime}(x_j), \quad j = 1,\dots,r.
\end{align}
\begin{enumerate}[label = \textbf{\alph*)}]
  \item Zeigen Sie, dass dieses Problem eindeutig lösbar ist.
  \item Zeigen Sie die Fehlerabschätzung
  \begin{align}
    \sup_{x \in [0,h]} |p(x) - g(x)| \leq \frac{h^{r+1}}{r!}
    \sup_{x \in [0,h]}\left|g^{(r + 1)}(x)\right|.
  \end{align}
\end{enumerate}
\end{exercise}
\begin{solution}
\leavevmode \\
\begin{enumerate}[label = \textbf{\alph*)}]
  \item Seien $p_1,p_2 \in \Pi_r$ Lösungen, so ist $p := p_1 - p_2$ ein Polynom vom Grad $\leq r$
  mit einer Nullstelle in $x_0$.
  $p^{\prime} \in \Pi_{r-1}$ hat $x_1,\dots,x_r$ als Nullstellen und es folgt mit
  dem Fundamentalsatz der Algebra $p^{\prime} \equiv 0$. Also muss $p$ konstant
  sein und aufgrund $p(x_0) = 0$ auch $p \equiv 0$. Es folgt $p_1 = p_2$ und die
  Eindeutigkeit der Lösung ist gezeigt.
  Zum Nachweis der Existenz einer Lösung betrachten wir die Abbildung
  \begin{align*}
    \mu: \begin{cases}
      \Pi_r \rightarrow \R^{r+1} \\
      p \mapsto (p(x_0),p^{\prime}(x_1),\dots,p^{\prime}(x_r))
    \end{cases}
  \end{align*}
  Aufgrund $\dim \Pi_r = r + 1$ folgt aus der eben bewiesenen Injektivität auch
  die Surjektivität. Also existiert immer genau eine Lösung des Problems.
  \item Auf das Polynom $p^{\prime}$ können wir die Fehlerabschätzung aus der
  Lagrange-Interpolation (Lemma 3.18) anwenden und erhalten
  \begin{align*}
      \sup_{x \in [0,h]} |p^{\prime}(x) - g^{\prime}(x)| \leq \frac{h^{r}}{r!}\sup_{x \in [0,h]} |(g^{\prime})^{(r)}(x)|.
  \end{align*}
  Weiters berechnen wir für $x \in [0,h]$ beliebig
  \begin{align*}
    |p(x) - g(x)| = \left|\int_0^x p^{\prime}(y) - g^{\prime}(y) dy \right| \leq h \sup_{x \in [0,h]} |p^{\prime}(x) - g^{\prime}(x)| =
    \frac{h^{r+1}}{r!}\sup_{x \in [0,h]} |g^{(r + 1)}(x)|
  \end{align*}
  und erhalten schließlich
  \begin{align*}
  \sup_{x \in [0,h]} |p(x) - g(x)| \leq \frac{h^{r+1}}{r!}
  \sup_{x \in [0,h]}\left|g^{(r + 1)}(x)\right|,
  \end{align*}
  was zu zeigen war.
\end{enumerate}

\end{solution}
