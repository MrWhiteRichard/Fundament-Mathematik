\begin{exercise}
  Gegeben sei ein beliebiges $m$-stufiges Runge-Kutta-Verfahren mit Butcher-Tableau
  \renewcommand{\arraystretch}{1.2}
    $\begin{matrix}
    c & \vline & A \\
    \hline
     & \vline & b^{\top}
    \end{matrix}$
    und diskretem Fluss $\Psi^h$. Geben Sie die Butcher-Tableaus
    \begin{enumerate}[label = \textbf{\alph*)}]
      \item des zugehörigen adjungierten Verfahrens mit adjungiertem Fluss $(\Psi^h)^*$ und
      \item des reversiblen Verfahrens $(\Psi^{\nicefrac{h}{2}})^* \circ \Psi^{\nicefrac{h}{2}}$ an.
    \end{enumerate}
\end{exercise}

\begin{solution}
\begin{enumerate}[label = \textbf{\alph*)}]
  \item
  Die Definition des diskreten Flusses liefert uns
  \begin{align*}
    z_1 &= \Psi^h(z_0) = z_0 + h\sum_{j=1}^mb_jf(t_0+c_jh,z_0+h\sum_{\ell = 1}^m A_{j\ell}k_{\ell})
  \end{align*}
  Damit erhalten wir für den adjungierten Fluss
  \begin{align*}
    y_0 &= \Psi^{-h}(y_1) = y_1 - h\sum_{j=1}^mb_jf(t_1-c_jh,y_1-h\sum_{\ell = 1}^m A_{j\ell}k_{\ell}) \\
    \implies& y_1 = y_0 + h\sum_{j=1}^mb_jf(t_1 - c_jh,y_1-h\sum_{\ell = 1}^m A_{j\ell}k_{\ell})
    = y_0 + h\sum_{j=1}^mb_jf(t_0 + (1-c_j)h,y_0 + h\sum_{\ell=1}^mb_{\ell}k_{\ell} - h\sum_{\ell = 1}^m A_{j\ell}k_{\ell})
  \end{align*}
  Das zugehörige Butcher-Tableau lautet daher
    \renewcommand{\arraystretch}{1.2}
    $\begin{matrix}
    \1 - c& \vline & \1 b^{\top} - A\\
    \hline
     & \vline & b^{\top}
    \end{matrix}$
  mit $\1 := (1,\dots,1)^{\top} \in \R^m$.
  \begin{align*}
    \1 b^{\top} = \begin{pmatrix}
      b_1 & \hdots & b_m \\
      \vdots & \ddots & \vdots \\
      b_1 & \hdots & b_m
    \end{pmatrix}
  \end{align*}
  \item \begin{align*}
    \Psi^{-\nicefrac{h}{2}}(y_1) &= \Psi^{\nicefrac{h}{2}}(y_0) \\
    &\iff y_1 -  \frac{h}{2}\sum_{j=1}^mb_j\widetilde{k}_j = y_0 + \frac{h}{2}\sum_{j=1}^mb_jk_j \\
    &\iff y_1 = y_0 + \frac{h}{2}\sum_{j=1}^mb_jf(t_0 + \frac{c_j}{2}h,y_0+\frac{h}{2}\sum_{\ell = 1}^m A_{j\ell}k_{\ell})
    + \frac{h}{2}\sum_{j=1}^mb_jf(t_1 -\frac{c_j}{2}h,y_1-\frac{h}{2}\sum_{\ell = 1}^m A_{j\ell}\widetilde{k_{\ell}}) \\
    &= y_0 + \frac{h}{2}\sum_{j=1}^mb_j\left(f(t_0 + \frac{c_j}{2}h, y_0+\frac{h}{2}\sum_{\ell = 1}^m A_{j\ell}k_{\ell})+
    f(t_0 + (1-\frac{c_j}{2})h,y_1 - \frac{h}{2}\sum_{\ell = 1}^m A_{j\ell}\widetilde{k_{\ell}})\right) \\
    &= y_0 + \frac{h}{2}\sum_{j=1}^mb_j\left(f(t_0+\frac{c_j}{2}h,y_0+\frac{h}{2}\sum_{\ell = 1}^m A_{j\ell}k_{\ell})+
    f(t_0 + (1-\frac{c_j}{2})h,y_0 +\frac{h}{2}\sum_{\ell=1}^m\left(b_{\ell}k_{\ell}+b_{\ell}\widetilde{k}_{\ell}\right) -
    \sum_{\ell = 1}^m A_{j\ell}\widetilde{k_{\ell}})\right)
  \end{align*}
  Definiere
  \begin{align*}
  \widetilde{b} &:= \frac{1}{2}(b_1,\dots,b_m,b_1,\dots,b_m)^{\top}, \\
  \widetilde{c} &:= \frac{1}{2}(c_1,\dots,c_m,2-c_1,\dots,2-c_m)^{\top}, \\
  \widetilde{A} &:= \frac{1}{2}\begin{pmatrix}
    A & 0 \\
    \1b^\top & \1 b^\top - A
  \end{pmatrix}.
  \end{align*}
  Das sind die Zutaten für das benötigte Butcher-Tableau.
\end{enumerate}
\end{solution}
