\begin{exercise}
Zur Konstruktion von linearen $k$-Schritt Verfahren für die Differentialgleichung
$y^{\prime}(t) = f(t,y(t))$ kann man folgendermaßen vorgehen: Sei $\{t_0,\dots,t_N\}$
ein uniformes Gitter mit Gitterweite $h$. Gegeben seien die Werte $y_{\ell},\dots,y_{\ell + 1 - k}$.
Zur Berechnung von $y_{\ell + 1}$ sei $p_{\ell} \in \Pi_k$ das Polynom, welches
die Bedingungen
\begin{subequations}
\begin{align}\label{lagrange}
  p_{\ell}(t_{\ell + 1 - j}) = y_{\ell + 1 - j}, \qquad j = 0,\dots,k,
\end{align}
und
\begin{align}
  p_{\ell}^{\prime}(t_{\ell + 1}) = f(t_{\ell + 1},y_{\ell + 1})
\end{align}
\end{subequations}
erfüllt. Dann definieren wir $y_{\ell + 1} := p_{\ell}(t_{\ell + 1})$.
\begin{enumerate}[label = \textbf{\alph*)}]
  \item Geben Sie die Konstanten in Gleichung (5.2) des Vorlesungsskriptes für
  dieses Verfahren abstrakt an und zeigen Sie, dass diese unabhängig von $h$ sind.
  Unter welchen Voraussetzungen ist das Verfahren wohldefiniert?
  \item Geben Sie die Koeffizienten der Verfahren für $k = 1,2,3$ explizit an.
  \item Beweisen Sie, dass
  \begin{align}
    \tilde{\eta}_{\ell}(p,h) := \sum_{j= 0}^kh\beta_jp^{\prime}(t_{\ell + 1 -k} + jh)
    - \alpha_jp(t_{\ell + 1 - k} + jh)
  \end{align}
  bei diesen Verfahren für alle $p \in \Pi_k$ verschwindet. Welche Konsistenzordnung
  haben diese Verfahren?
\end{enumerate}
\end{exercise}
\begin{solution}
\begin{enumerate}[label = \textbf{\alph*)}]
  Offensichtlich gehen in diesem Verfahren Werte von $f$ nur an der Stelle $t_{l+1}$ ein, also gilt für alle $i<k: \beta_i = 0$ bzw.
  \begin{align}\label{eqallg}
    \sum_{j=0}^k \alpha_{k-j} y_{l+1-j} = h \beta_k f(t_{l+1},y_{l+1}) = h p_l' (t_{l+1})
  \end{align}

  \item Wir setzen anfangs $y_{\ell + 1}$ unbestimmt an und lösen das Lagrange-Interpolationsproblem
  \eqref{lagrange} in Abhängigkeit von $y_{\ell + 1}$.
  Wir erhalten als Darstellung von $p_{\ell}$
  \begin{align*}
    p_{\ell}(t_{\ell} + \tau h) = \sum_{j=0}^k\tilde{L}_j(\tau)y_{\ell + 1 - j}
  \end{align*}
  mit
  \begin{align*}
    \tilde{L}_j(\tau) := L_j(t_{\ell} + \tau h) &= \prod_{\stackrel{i = 0}{i \neq j}}^{k}\frac{t_{\ell} + \tau h - t_{\ell + 1 - i}}{t_{\ell + 1 -j} - t_{\ell + 1 - i}}
    = \prod_{\stackrel{i = 0}{i \neq j}}^{k}\frac{t_0 + h{\ell} + \tau h - (t_0 + h(\ell + 1 - i))}{t_0 + h(\ell + 1 -j) - (t_0
    + h(\ell + 1 - i))} \\
    &=  \prod_{\stackrel{i = 0}{i \neq j}}^{k}\frac{\tau h - h(1 - i)}{h(i - j)}
    = \prod_{\stackrel{i = 0}{i \neq j}}^{k}\frac{\tau  - (1 - i)}{i - j}.
  \end{align*}
  Jetzt können wir $y_{\ell + 1}$ durch die zusätzliche Bedingung
  \begin{align*}
    hf(t_{\ell + 1},y_{\ell + 1}) = hp_{\ell}^{\prime}(t_{\ell + 1}) = \partial_{\tau}p_{\ell}(t_{\ell} + \tau h)|_{\tau = 1} =
    \sum_{j=0}^k\tilde{L}_j^{\prime}(1)y_{\ell + 1 - j}
  \end{align*}
  ausdrücken und erhalten
  \begin{align*}
    \alpha_{k - j} = \tilde{L}_j^{\prime}(1), \qquad j = 0,\dots,k
  \end{align*}
  Damit erhalten wir ein implizites Verfahren, welches laut Remark 5.2 für
  hinreichend kleines $h$ wohldefiniert ist.
  \item $k = 1$:
  \begin{align*}
    hf(t_{\ell + 1},y_{\ell + 1}) &= \partial_{\tau}(\tau)\Bigg|_{\tau = 1}y_{\ell + 1} -
    \partial_{\tau}(\tau + 1)\Bigg|_{\tau = 1}y_{\ell} \\
    &= y_{\ell + 1} - y_{\ell}
  \end{align*}
  $k = 2$:
  \begin{align*}
  hf(t_{\ell + 1},y_{\ell + 1}) &= \partial_{\tau}\left(\frac{\tau(\tau + 1)}{2}\right)
  \Bigg|_{\tau = 1}y_{\ell + 1} -
  \partial_{\tau}\left((\tau - 1)(\tau + 1)\right)
  \Bigg|_{\tau = 1}y_{\ell} +
  \partial_{\tau}\left(\frac{(\tau - 1)\tau}{2}\right)
  \Bigg|_{\tau = 1}y_{\ell - 1} \\
  &= \frac{3}{2}y_{\ell + 1} - 2y_{\ell} + \frac{1}{2}y_{\ell - 1}
  \end{align*}
  $k = 3$:
  \begin{align*}
  hf(t_{\ell + 1},y_{\ell + 1}) &= \partial_{\tau}\left(\frac{\tau(\tau + 1)(\tau + 2)}{6}\right)
  \Bigg|_{\tau = 1}y_{\ell + 1} -
  \partial_{\tau}\left(\frac{(\tau - 1)(\tau + 1)(\tau + 2)}{2}\right)
  \Bigg|_{\tau = 1}y_{\ell} \\
  &+ \partial_{\tau}\left(\frac{(\tau - 1)\tau(\tau + 2)}{2}\right)
  \Bigg|_{\tau = 1}y_{\ell - 1} -
  \partial_{\tau}\left(\frac{(\tau - 1)\tau(\tau + 1)}{6}\right)
  \Bigg|_{\tau = 1}y_{\ell - 2}
  \\
  &= \frac{11}{6}y_{\ell + 1} - 3y_{\ell} + \frac{3}{2}y_{\ell - 1} - \frac{1}{3}y_{\ell - 2}
  \end{align*}
  \item Da $\beta_j$ für alle $j \neq k$ wegfällt, vereinfacht sich der Term
  $\tilde{\eta}(p,h)$ zu
  \begin{align*}
    \tilde{\eta}(p,h) = hp^{\prime}(t_{\ell + 1 - k} + kh) -
    \sum_{j = 0}^k \alpha_jp(t_{\ell + 1 - k} + jh)
  \end{align*}
  Wenn wir nun analog zur Darstellung von $p_{\ell}$ oben $p$ als Summe
  von Lagrangebasispolynomen darstellen, erhalten wir aufgrund
  $\alpha_{k-j} = \tilde{L}_j^{\prime}(1)$
  \begin{align*}
    \tilde{\eta}(p,h) = \sum_{j= 0}^k p(t_{\ell + 1 - j})\tilde{L}_j^{\prime}(1)
    - \sum_{j = 0}^k \alpha_jp(t_{\ell + 1 - (k - j)}) = 0
  \end{align*}
  Nach Theorem 5.15 haben diese Verfahren also Konsistenzordnung $\geq k$.
  Dass $k$ auch eine obere Schranke für die Konsistenzordnung ist, sieht man mit
  dem Polynom
  \begin{align*}
    p(t) = \prod_{j = 0}^k (t- t_{\ell + 1 - j}) \in \Pi_{k+1}.
  \end{align*}
  Es folgt
  \begin{align*}
    \tilde{\eta}(p,h) = hp^{\prime}(t_{\ell + 1}) - \sum_{j = 0}^k \alpha_j \underbrace{p(t_{\ell + 1 - k + j})}_{=0}
    = hp^{\prime}(t_{\ell + 1}) \neq 0,
  \end{align*}
  da $p$ keine doppelten Nullstellen besitzt.
  Damit kann das Verfahren wieder laut Theorem 5.15 nicht Konsistenzordnung $k + 1$ haben.
\end{enumerate}
\end{solution}
