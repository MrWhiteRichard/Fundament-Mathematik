\begin{exercise}
Zeigen Sie, dass das Stabilitätsgebiet eines expliziten, linearen Mehrschrittverfahrens
beschränkt und das Verfahren dadurch nicht A-stabil ist. \\
\textit{Hinweis:} Nehmen Sie an, dass die Beträge der Nullstellen von $\xi \mapsto \rho(\xi) - z\sigma(\xi)$
kleiner $1$ sind und finden Sie eine Abschätzung die gleichmäßig in $z$ ist.
Konstruieren Sie daraus einen Widerspruch.
\end{exercise}
\begin{solution}
Wir nehmen an, dass das Stabilitätsgebiet des Verfahrens unbeschränkt ist. Das heißt es gibt für jedes $C>0$ ein $z \in \C$ mit $|z|>C$, das in $S$ liegt. In $S$ zu liegen heißt, alle Nullstellen von $\rho_z$ erfüllen $|\xi_z| \leq 1$.

Die Funktion $\rho_z$ ist im Fall von expliziten Verfahren ($\beta_k = 0$) gegeben durch:
\begin{align*}
  \rho_z(\xi) := \rho(\xi) - z \sigma(\xi) = \xi^k + \sum_{j=0}^{k-1} (\alpha_j - z \beta_j) \xi^j
\end{align*}
Auf der anderen Seite kann man $\rho_z \in \Pi_k$ auch mithilfe der Nullstellen in die Linearfaktoren zerlegen.
\begin{align*}
  \rho_z(\xi) = \prod_{i = 0}^k(\xi - \xi_{z}^{(i)})
\end{align*}
Für alle $z$ aus dem Stabilitätsgebiet gilt nun:
\begin{align*}
  |\rho_z(0)| = |\prod_{i = 0}^k(0 - \xi_{z}^{(i)})| = \prod_{i = 0}^k|\xi_{z}^{(i)}| \leq 1 \\
\end{align*}
Allerdings auch (aus der Darstellung weiter oben):
\begin{align*}
  |\rho_z(0)| = |\alpha_0 - z \beta_0| \geq |z \beta_0| - |\alpha_0| > 1
\end{align*}
für $|z|$ hinreichend groß. ($|z| > C := \frac{|\alpha_0|}{|\beta_0|}$)

Aus dem Widerspruch erhalten wir also, dass das Stabilitätsgebiet beschränkt ist und somit auch nicht $\C_{Re \leq 0} \subset S$ gilt. Also sind lineare explizite Mehrschrittverfahren nicht A stabil.
\end{solution}
