\begin{exercise}
Gegeben seien zwei Adams-Bashforth-Verfahren mit $k$, beziehungsweise $k + 1$
Schritten und gleicher Gitterweite $h$. Wie in Definition 5.9 des Skriptes sei
$y_{\ell + 1}$ die Lösung des $k$-Schritt-Verfahrens und $\widetilde{y}_{\ell + 1}$
die Lösung des $k + 1$-Schritt-Verfahrens bei jeweils exakten Anfangswerten.
Leiten Sie einen berechenbaren Fehlerschätzer $\mu$ für den Konsistenzfehler $\tau_{\ell}(h)$
des ersten Verfahrens her. Es sollte gelten
\begin{align*}
  \tau_{\ell}(h) = \mu + \Landau{h^{k+3}}.
\end{align*}
Benutzen Sie dazu insbesondere die Entwicklung des Abschneidefehlers aus dem Beweis
von Theorem 5.15 und vollziehen Sie die Konstruktion des Fehlerschätzers in Abschnitt
2.6, beziehungsweise Abschnitt 2.7 nach.
\end{exercise}
\begin{solution}
Seien die Adams-Bashforth-Verfahren gegeben durch
\begin{align*}
  y_{\ell + 1} &= y_{\ell} + h\sum_{i = 0}^{k-1} a_i f_{\ell - i} \\
  \widetilde{y}_{\ell + 1} &= y_{\ell} + h\sum_{i = 0}^{k} b_i f_{\ell - i}.
\end{align*}
Aus Lemma 5.12 wissen wir bereits, dass es Konstanten $C_1,C_2, h_0$ gibt, sodass
\begin{align*}
  \forall h \in (0,h_0), \forall \ell \in \{k-1,\dots,N-1\}: C_1\|\tau_{\ell}(h)\|
  \leq \|\eta_{\ell}(y,h)\| \leq C_2\|\tau_{\ell}(h)\|
\end{align*}
In Theorem 5.15 haben wir für den Abschneidefehler linearer $k$-Schritt-Verfahren,
also insbesondere auch Adams-Bashforth-Verfahren folgende Darstellung hergeleitet:
\begin{align*}
  \eta_{\ell}(y,h) = -y(t_{\ell + 1 - k})\sum_{j= 0}^k\alpha_j + \sum_{i=1}^k \frac{y^{(i)}(t_{\ell + 1 - k})}{i!}
  h^i\sum_{j= 0}^ki\beta_jj^{i-1} - \alpha_j j^{i} + \Landau{h^{p+1}}.
\end{align*}
Aus Korollar 5.18 wissen wir, dass $k$-Schritt Adams-Bashforth-Verfahren auch
Konsistenzordnung $k$ besitzen und daher folgt für den Abschneidefehler
\begin{align*}
\eta_{\ell}(y,h) = -y(t_{\ell + 1 - k})\sum_{j= 0}^k\alpha_j + \sum_{i=1}^k \frac{y^{(i)}(t_{\ell + 1 - k})}{i!}
h^i\sum_{j= 0}^ki\beta_jj^{i-1} - \alpha_j j^{i} + \Landau{h^{k+1}}.
\end{align*}
\end{solution}
