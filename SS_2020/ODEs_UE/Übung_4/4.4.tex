\begin{exercise}
Seien $A,B \in \mathbb{R}^{d \times d}$ mit $AB = BA$.
\begin{itemize}
  \item [\textbf{a)}]Zeigen Sie:
  \begin{align*}
    e^{A + B} = e^Ae^B = e^Be^A
  \end{align*}
  \item [\textbf{b)}]Zeigen Sie für $s,t \in \mathbb{R}$:
  \begin{align*}
    e^{(s + t)A} = e^{sA}e^{tA}
  \end{align*}
\end{itemize}
\end{exercise}
\begin{solution}
\leavevmode \\
\begin{itemize}
  \item [\textbf{a)}]
  In einem kommutativen Ring mit Eins gilt der binomische Lehrsatz.
  Der Matrizenring ist zwar im Allgemeinen nicht kommutativ, aber eingeschränkt
  auf jene Matrizen, welche das Kommutativgesetz erfüllen, können wir somit
  den binomischen Lehrsatz anwenden und aufgrund der absoluten Konvergenz der Exponentialmatrix-Reihe
  die Reihenglieder beliebig umordnen.
  \begin{align*}
    e^{A + B} &= \sum_{n=0}^{\infty}\frac{t^n}{n!}(A + B)^n
    = \sum_{n=0}^{\infty}\sum_{k=0}^n\frac{t^n}{n!}\binom{n}{k}A^{n-k}B^k
    = \sum_{n=0}^{\infty}A^n
    \sum_{k=0}^{n}\frac{t^{n+k}}{(n+k)!}\binom{n+k}{k}B^k \\
    &= \sum_{n=0}^{\infty}\frac{t^n}{n!}A^n
    \sum_{k=0}^{\infty}\frac{t^{k}}{k!}B^k
    = e^Ae^B
  \end{align*}
  Da $(A + B)^n = (B + A)^n$ gilt auch: $e^{A + B} = e^Be^A$.
  \item [\textbf{b)}]
  Wegen $sAtA = tAsA$ gilt:
  \begin{align*}
    \exp(sA)\exp(tA) = \exp(sA + tA) = \exp((s+t)A)
  \end{align*}
\end{itemize}


\end{solution}
