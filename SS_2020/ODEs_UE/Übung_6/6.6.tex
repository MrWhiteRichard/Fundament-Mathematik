\begin{exercise}
  Bestimmen Sie ein Fundamentalsystem für die ODE vom Eulerschen Typ:

  \begin{align*}
    2t^3 y^\primeprimeprime + 10t^2 y^\primeprime - 4ty^\prime - 20y = 0
  \end{align*}

  Hinweis: Substituieren Sie $t = \exp(\tau)$.
\end{exercise}

\begin{solution}
  Führen wir die Substitution aus dem Hinweis durch erhalten wir das System

  \begin{align*}
    2\exp(3\tau) y^{(3)}(\exp(\tau)) + 10\exp(2\tau) y^\primeprime (\exp(\tau))
    -4 \exp(\tau) y^\prime (\exp(\tau)) - 20 y(\exp(\tau)) = 0
  \end{align*}

  Definieren wir $z(\tau)=y(\exp(\tau))$ so gilt für deren Ableitungen:

  \begin{align*}
    z^\prime (\tau) &= y^\prime (\exp(\tau))\exp(\tau) \\
    z^\primeprime (\tau) &= y^\primeprime (\exp(\tau)) \exp(2\tau) +
    y^\prime (\exp(\tau))\exp(\tau) \\
    z^{(3)}(\tau) &= y^{(3)}(\exp(\tau)) \exp(3\tau) + 3y^\primeprime (\exp(\tau))
    \exp(2\tau) + y^\prime (\exp(\tau)) \exp(\tau)
  \end{align*}

  Wollen wir das nun auf obige Form bringen müssen wir die Koeffizienten vergleichen und erhalten

  \begin{align*}
    2z^{(3)}(\tau)+4z^\primeprime (\tau) -10 z^\prime (\tau) - 20z(\tau) = 0
    \Leftrightarrow \\
    z^{(3)}(\tau)+2z^\primeprime (\tau) -5 z^\prime (\tau) - 10z(\tau) = 0
  \end{align*}

  Hier können wir nun Satz 3.18 anwenden, wir suchen also die Nullstellen des
  charakteristischen Polynoms $\chi$:

  \begin{align*}
    \chi(\lambda) = \lambda^3 + 2\lambda^2 -5\lambda - 10
  \end{align*}

  Eine einfache Nullstelle, nämlich $\lambda _1 =-2$ können wir erraten und dann Polynomdivision
  durchführen:

  \begin{align*}
    \lambda^3 + 2\lambda^2 -5\lambda - 10 : (\lambda + 2) = \lambda^2 - 5
  \end{align*}

  Damit erhalten wir die zwei weiteren Nullstellen
  $\lambda_2, \lambda_3 = \pm \sqrt{5}$. Damit bilden

  \begin{align*}
    z_1(\tau)=\exp(-2\tau) \qquad z_2 (\tau)=\exp(-\sqrt{5}\tau) \qquad
    z_3 (\tau)=\exp(\sqrt{5}\tau)
  \end{align*}

  ein Fundamentalsystem. Sehen wir uns die Definition von $z$ an bedeutet das also

  \begin{align*}
    y_1 (\exp(\tau)) = \exp(-2\tau) \qquad y_2 (\exp(\tau))=\exp(-\sqrt{5}\tau) \qquad
    y_3 (\exp(\tau))=\exp(\sqrt{5}\tau)
  \end{align*}

  Wenn wir nun rücksubstituieren, erhalten wir damit ein Fundamentalsystem des originalen
  Problems:

  \begin{align*}
    y_1 (t) = t^{-2} \qquad y_2 (t) = t^{-\sqrt{5}} \qquad y_3 (t) = t^{\sqrt{5}}
  \end{align*}
\end{solution}
