\begin{exercise}

Sei $\mathbf{A} \in C(\R; \R^{d \times d})$. Betrachten Sie die Differentialgleichung

\begin{align*}
  \mathbf{y}^\prime(t) + \mathbf{A}(t) \mathbf{y}(t) = 0 \in \R^d.
\end{align*}

Zeigen Sie: Die Menge aller Lösungen (genauer: zu gegebenem offenen Intervall $I \subset \R$ die Menge der $C^1(I; \R^d)$-Funktionen, die die Differentialgleichung erfüllen) ist ein Vektorraum.

\end{exercise}

\begin{solution}

\begin{align*}
  V := \pbraces{\R^d}^I,
  \quad
  U := \Bbraces
  {
    \mathbf{y} \in C^1(I; \R^d):
    \mathbf{y}^\prime + \mathbf{A} \mathbf{y} = 0
  }
\end{align*}

Wir zeigen, dass $U$ ein linearer Unterraum des Vektorraumes $V$ über $\R$ ist.

\begin{itemize}

  \item $0 \in U
  \Rightarrow
  U \neq \emptyset$

  \item $\Forall \mathbf{y}_1, \mathbf{y}_2 \in U, \Forall x \in \R:$
  \begin{align*}
    (\mathbf{y}_1 + x \mathbf{y}_2)^\prime +
    \mathbf{A}
    (\mathbf{y}_1 + x \mathbf{y}_2)
    =
    \underbrace{\pbraces{
      \mathbf{y}_1^\prime + \mathbf{A} \mathbf{y}_1
    }}_0 +
    x
    \underbrace{\pbraces{
    \mathbf{y}_2^\prime + \mathbf{A} \mathbf{y}_2
    }}_0
    = 0 \\
    \Rightarrow \mathbf{y}_1 + x \mathbf{y}_2 \in U
  \end{align*}

\end{itemize}

\end{solution}
