\begin{exercise}

Sei $\epsilon > 0$. Betrachten Sie die \Quote{skalierte} Lotka-Volterra Gleichung

\begin{align*}
  x^\prime(t) & = x(t) - x(t) y(t) \\
  y^\prime(t) & = -\epsilon y(t) + x(t) y(t).
\end{align*}

Zeigen Sie: Die Funktion $(x, y) \mapsto \phi(x, y) := x + y - \epsilon \ln{x} - \ln{y}$ ist eine Erhaltungsgröße, d.h. für Lösungen $t \mapsto (x(t), y(t))$ der Lotka-Volterra-Gleichung gilt, daß $t \mapsto \phi(x(t), y(t))$ konstant ist.

\end{exercise}

\begin{solution}

Die Ableitung der Funktion $f: t \mapsto \phi(x(t), y(t))$ verschwindet.

\begin{align*}
  f^\prime(t)
  & =
  x^\prime(t) +
  y^\prime(t) -
  \epsilon
  \frac{1}{x(t)} x^\prime(t) -
  \frac{1}{y(t)} y^\prime(t) \\
  & =
  x(t) - x(t) y(t) -
  \epsilon
  y(t) + x(t) y(t) -
  \epsilon
  \frac{1}{x(t)}
  \pbraces
  {
    x(t) - x(t) y(t)
  } -
  \frac{1}{y(t)}
  \pbraces
  {
    -\epsilon
    y(t) + x(t) y(t)
  } \\
  & =
  x(t) - \epsilon y(t) - \epsilon + \epsilon y(t) + \epsilon - x(t) = 0
\end{align*}

\end{solution}
