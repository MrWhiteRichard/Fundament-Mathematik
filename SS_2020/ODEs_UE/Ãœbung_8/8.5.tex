\begin{exercise}
Sei $H: C^2(\R^{2d};\R)$. Das zu $H$ gehörende Hamiltonsche System ist gegeben durch
\begin{align*}
  q^{\prime} &= \partial_p H(q,p) \\
  p^{\prime} &= -\partial_q H(q,p).
\end{align*}
Zeigen Sie, dass $H$ eine Ljapnuovfunktion für das System ist. Geben Sie an, welche
Ruhelagen des Systems stabil und welche asymptotisch stabil sind. \\
Geben Sie die stabilen und asymptotisch stabilen Ruhelagen für die konkrete Funktion
\begin{align*}
  H(p,q) = \frac{1}{2}p^2 + (1 - \cos(q))
\end{align*}
an.
\end{exercise}
\begin{solution}
Mit unserem Kriterium für Ljapnuovfunktionen können wir einfach nachrechnen, dass $H$
wirklich eine soche ist.

\begin{align*}
  \abraces{\nabla H(q,p),f\left(
  \begin{array}{c}
    q \\
    p
  \end{array}
  \right)} &= \abraces{\left(
  \begin{array}{c}
  \partial_q H(q,p) \\
  \partial_p H(q,p)
  \end{array}
  \right), \left(
  \begin{array}{c}
      \partial_p H(q,p) \\
      -\partial_q H(q,p)
  \end{array}
  \right)} = \\
  & \partial_q H(q,p) \partial_p H(q,p) - \partial_q H(q,p) \partial_p H(q,p) = 0
\end{align*}

Hier können wir auch direkt erkennen, dass $H$ auch eine Erhaltungsgröße ist, da die
zweite Zeile ja genau $\partial_t H(q(t),p(t))$ ist. Für die Stabilität der Ruhelagen
können wir zuerst das Prinzip der linearisierten Stabilität anwenden, falls wir dort
keine Aussage erhalten können wir noch überprüfen, ob die Ruhelage ein striktes Minimum
von $H$ ist. Dann ist die Ruhelage nach der direkten Methode von Ljapunov stabil.
Für eine Ruhelage gilt jedenfalls schon $\nabla H = 0$, für die Stabilität muss also
die Hesse-Matrix von $H$ positiv definit sein. Um asymptotischsche Stabilität zu erhalten
müssen die Realteile aller Eigenwerte von $Df(y^*)$ echt negativ sein. Wenden wir uns nun dem
konkreten Beispiel zu. Zuerst suchen wir die Ruhelagen

\begin{align*}
  f((p,q)^T) = \left(
  \begin{array}{c}
    -\partial_q H(p,q) \\
    \partial_p H(p,q)
  \end{array}
  \right) = \left(
  \begin{array}{c}
  -\sin(q) \\
    p
  \end{array}
  \right) \stackrel{!}{=}0
\end{align*}

Also sind unsere Ruhelagen gegeben durch $(p^*,q^*)=(0,n \pi), n \in \Z$. Wenden wir
nun das Prinzip der linearisierten Stabilität.

\begin{align*}
  Df(p^*,q^*)= \left(\begin{array}{cc}
    0 & -\cos(q^*) \\
    1 & 0
  \end{array} \right)
\end{align*}

Wir unterscheiden nun für $q^* \in \pi(2\Z)$:

\begin{align*}
Df(p^*,q^*)= \left(\begin{array}{cc}
  0 & -1 \\
  1 & 0
\end{array} \right)
\end{align*}

Das charakteristische Polynom ist

\begin{align*}
  \chi(\lambda) = \lambda^2 + 1 = (\lambda -i)(\lambda +i)
\end{align*}

Die beiden Eigenwerte sind also echt Imaginär und damit liefert der Satz keine Aussage.
Wir werden darauf gleich noch zurückkommen, zuerst betrachten wir noch den Fall
$q^* \in \pi(2\Z +1)$. Dann ist:

\begin{align*}
Df(p^*,q^*)= \left(\begin{array}{cc}
  0 & 1 \\
  1 & 0
\end{array} \right)
\end{align*}

Das charakteristische Polynom ist

\begin{align*}
  \chi(\lambda) = \lambda^2 - 1 = (\lambda -1)(\lambda +1)
\end{align*}

Diese Ruhelagen sind also instabil. Um noch zu überprüfen ob die Ruhelagen mit
$q^* \in \pi(2\Z)$ zumindest stabil sind sehen wir uns die Hesse-Matrix von H an:

\begin{align*}\left(
  \begin{array}{cc}
    \partial_{pp}H(p^*,q^*) & \partial_{pq}H(p^*,q^*) \\
    \partial_{qp}H(p^*,q^*) & \partial_{qq}H(p^*,q^*)
  \end{array}\right)=
  \left(\begin{array}{cc}
    1 & 0 \\
    0 & \cos(q^*)
  \end{array}\right)=
  \left(
  \begin{array}{cc}
    1 & 0 \\
    0 & 1
  \end{array}\right)
\end{align*}

Diese Einheitsmatrix ist klarerweise positiv definit, damit sind diese Ruhelagen zumindest stabil.
Wir wollen noch zeigen, dass sie nicht attraktiv sein können. Das gilt wenn

\begin{align*}
\forall \delta > 0 : \Exists (\widetilde{p},\widetilde{q}) \in B_{\delta}((p^*,q^*)^T):
\lim_{t \rightarrow \infty} \norm{(p,q)^T_{t_0,(p^*,q^*)^T}(t)-(p,q)^T_{t_0,(\widetilde{p},\widetilde{q})^T}(t)} \neq 0
\end{align*}

Sei nun $\delta$ beliebig und $(\widetilde{p},\widetilde{q})^T\neq (p^*,q^*)^T$ aus
der Umgebung. Weil $(p^*,q^*)^T$ ein striktes Minimum von $H$ ist gilt

\begin{align*}
  H(p^*,q^*) < H(\widetilde{p},\widetilde{q})
\end{align*}

Weil $H$ stetig ist finden wir $\delta > \rho >0$ sodass

\begin{align*}
  \forall (\bar{p},\bar{q})^T \in B_{\rho}((p^*,q^*)^T): \\
  H(p^*,q^*) \leq H(\bar{p},\bar{q}) < H(\widetilde{p}, \widetilde{q}) = H((p,q)^T_{t_0,(\widetilde{p},\widetilde{q})^T})
\end{align*}

Wobei letzte Gleichheit gilt, da $H$ eine Erhaltungsgröße ist und somit entlang Lösungen
konstant. Dann gilt aber auch

\begin{align*}
  \lim_{t \rightarrow \infty} \norm{(p,q)^T_{t_0,(p^*,q^*)^T}(t)-(p,q)^T_{t_0,(\widetilde{p},\widetilde{q})^T}(t)}
  > \rho > 0
\end{align*}

Womit diese Ruhelagen nicht attraktiv sind.
\end{solution}
