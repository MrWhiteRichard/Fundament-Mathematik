\begin{exercise}
Sei $Y \in C^1(J;\mathbb{R}^{d\times d})$ eine Fundamentalmatrix für das lineare
System $y^{\prime}(t) = A(t)y(t)$. Zeigen Sie:
\begin{itemize}
  \item [\textbf{a)}] Die Matrixfunktion $X \in C^1(J;\mathbb{R}^{d \times d})$
  ist genau dann eine Fundamentalmatrix, wenn es eine reguläre Matrix
  $B \in \mathbb{R}^{d\times d}$ gibt mit $\forall t \in J: X(t) = Y(t)B$.
  \item [\textbf{b)}] Die Matrix $X(t) := Y(t)(Y(t_0))^{-1}$ ist eine
  Hauptfundamentalmatrix.
\end{itemize}
\end{exercise}
\begin{solution}
\leavevmode \\
\begin{itemize}
  \item [\textbf{a)}] Sei $B \in \mathbb{R}^{d\times d}$ eine beliebige reguläre
  Matrix, $X(t) := Y(t)B$. Da für alle $b \in \mathbb{R}^{d}$ $Y(t) b$ eine
  Lösung des linearen Systems ist, ist $X(t)$ zumindest eine Lösungsmatrix.
  Aufgrund des Multiplikationssatz für Determinanten ist $Y(t)B$ auch regulär
  und $X(t)$ ist damit eine Fundamentalmatrix. \\
  Sei umgekehrt $X \in C^1(J;\mathbb{R}^{d \times d})$ eine beliebige
  Fundamentalmatrix, $Y(t) = (y^1(t),\dots,y^d(t))$. Da der Lösungsraum des
  linearen Systems Dimension $d$ hat, existiert zu jedem $y^i(t)$ ein $b_i \in
  \mathbb{R}^d$, sodass $X(t) b_i = y^i(t)$. Also gilt $X(t)B = Y(t)$. Da
  $X(t), Y(t)$ beide regulär sind, muss auch $B$ regulär und damit invertierbar
  sein und es gilt: $\forall t \in J: X(t) = Y(t)B^{-1}$
  \item [\textbf{b)}] Nach Lemma 3.3 ist $Y(t)$ für jedes $t \in \mathbb{R}^d$
  $Y(t)$ invertierbar, also existiert insbesondere $(Y(t_0))^{-1}$. \\
  Da $(Y(t_0))^{-1}$ regulär ist, folgt mit a), dass $Y(t)(Y(t_0))^{-1}$ eine
  Fundamentalmatrix ist. Bezüglich
  $t_0$ ist $X(t)$ offensichtlich eine Hauptfundamentalmatrix, da
  \begin{align*}
    X(t_0) = Y(t_0)(Y(t_0))^{-1} = I
  \end{align*}
\end{itemize}

\end{solution}
