\begin{exercise}
Betrachten Sie das RWP
\begin{align}\label{1}
  Ly = -(py^{\prime})^{\prime} + qy = f \text{ auf } (a,b), \qquad R_1y = R_2y = 0.
\end{align}
Es sei $G$ die Greensche Funktion für dieses RWP (Sie fordern also insbesondere
eindeutige Lösbarkeit). Es soll gezeigt werden, dass $LG(\cdot,t) = \delta_t$,
wobei $\delta_t$ die \glqq$\delta$-Distribution\grqq ist. Genauer: Zeigen Sie für
$t \in (a,b):$
\begin{align*}
  \int_a^b G(x,t)(L\varphi)(x) dx = \varphi(t), \qquad
  \forall \varphi \in C_0^{\infty}(a,b) = \{\varphi \in C^{\infty}(a,b) ~|~ \supp \varphi \subset (a,b)\}
\end{align*}
\end{exercise}
\begin{solution}
Da das Randwertproblem eindeutig lösbar ist, gilt das für alle linke Seiten $f \in C([a,b];\R)$
und somit insbesondere für alle $\varphi \in C_0^{\infty}(a,b)$.
Dazu betrachte zwei Lösungen $y_1,y_2$ für \eqref{1}.
\begin{align*}
  L(y_1 - y_2) &= Ly_1 - Ly_2 = 0 \\
  R_1(y_1 - y_2) &= R_2(y_1 - y_2) = 0
\end{align*} und da das Problem laut Voraussetzungen für $f = 0, \rho_1 = 0 = \rho_2$
nur trivial lösbar ist, folgt $y_1 = y_2$.
Aus Satz 6.9 und Satz 6.7 wissen wir, dass
\begin{align*}
  y(t) := \int_a^b G(x,t)(L\varphi)(x) dx = \int_a^b G(t,x)(L\varphi)(x) dx.
\end{align*}
das Randwertproblem
\begin{align*}
  Ly = -(py^{\prime})^{\prime} + qy = L\varphi, \qquad R_1y = R_2y = 0.
\end{align*}
Aus der eindeutigen Lösbarkeit folgt also $y(t) = \varphi(t)$.
\end{solution}
