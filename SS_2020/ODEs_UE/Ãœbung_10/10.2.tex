\begin{exercise}
\leavevmode \\
\begin{enumerate}[label = \textbf{\alph*)}]
  \item Betrachten Sie das RWP (auf $(0,1)$)
  \begin{align*}
    y^{\prime} = \begin{pmatrix}
      0 & 1 \\ 0 & 0
    \end{pmatrix}y, \qquad
    \begin{pmatrix}
      1 & 0 \\ 0 & 0
    \end{pmatrix}y(0)
    + \begin{pmatrix}
      -1 & 0 \\ 0 & 1
    \end{pmatrix}y(1)
    = \begin{pmatrix}
      0 \\ 1
    \end{pmatrix}.
  \end{align*}
  Hat das RWP eine Lösung?
  \item Diskutieren Sie die Lösbarkeit des RWP (auf $(0,\pi)$)
  \begin{align*}
  y^{\prime} = \begin{pmatrix}
    0 & 1 \\ -\alpha^2 & 0
  \end{pmatrix}y, \qquad
  \begin{pmatrix}
    1 & 0 \\ 0 & 1
  \end{pmatrix}y(0)
  + \begin{pmatrix}
    -1 & 0 \\ 0 & -1
  \end{pmatrix}y(\pi)
  = \begin{pmatrix}
    0 \\ 0
  \end{pmatrix}
  \end{align*}
  in Abhängigkeit von $\alpha \in \R$.
\end{enumerate}
\end{exercise}
\begin{solution}
\leavevmode \\
\begin{enumerate}[label = \textbf{\alph*)}]
  \item Zuerst bestimmen wir eine Fundamentalmatrix für das System
  \begin{align*}
    y^{\prime}(x) &= \underbrace{\begin{pmatrix}
      0 & 1 \\ 0 & 0
    \end{pmatrix}}_{=: A}y(x).\\
  Y(x) &= \exp(xA) = \begin{pmatrix}
    1 & x \\ 0 & 1
  \end{pmatrix}
  \end{align*}
  ist unsere Lösung dafür. Da wir hier ein halbhomogenes System vorliegen haben
  mit $\ell = (0,1)^{\top}, b \equiv 0$, ist
  unsere Partikulärlösung $y_p \equiv 0$ und es folgt mit Satz 6.3, dass unser
  System genau dann lösbar ist, wenn für $B := R_0Y(0) + R_1Y(1)$
  \begin{align*}
    \Rang(B) = \Rang\left(B,\ell - R_by_p(b)\right) = \Rang(B,\ell)
  \end{align*}
  gilt, also
  \begin{align*}
    \Rang\left(\begin{pmatrix}
      1 & 0 \\ 0 & 0
    \end{pmatrix}\begin{pmatrix}
      1 & 0 \\ 0 & 1
    \end{pmatrix} +
    \begin{pmatrix}
      -1 & 0 \\ 0 & 1
    \end{pmatrix}
    \begin{pmatrix}
      1 & 1 \\ 0 & 1
    \end{pmatrix}\right) =
    \Rang\left(\begin{pmatrix}
      0 & -1 \\ 0 & 1
    \end{pmatrix}\right) = 1 \neq 2 =
    \Rang\left(\begin{pmatrix}
      0 & -1 & 0\\ 0 & 1 & 1
    \end{pmatrix}\right).
  \end{align*}
  Es folgt, dass unser Problem unlösbar ist.
  \item Zuerst behandeln wir den Fall $\alpha = 0$: \\
    Die Fundamentalmatrix lautet
    \begin{align*}
      Y(x) = \begin{pmatrix}
        1 & x \\ 0 & 1
      \end{pmatrix}
    \end{align*}
    und wir erhalten
    \begin{align*}
      B = R_0Y(0) + R_{\pi}Y(\pi) =
      \begin{pmatrix}
        1 & 0 \\ 0 & 1
      \end{pmatrix}
      \begin{pmatrix}
        1 & 0  \\ 0 & 1
      \end{pmatrix}
      + \begin{pmatrix}
        -1 & 0 \\ 0 & -1
      \end{pmatrix}
      \begin{pmatrix}
        1 & \pi  \\ 0 & 1
      \end{pmatrix}
      = \begin{pmatrix}
        0 & -\pi \\ 0 & 0
      \end{pmatrix}
    \end{align*}
    Wieder ist der Lösungsraum ein-dimensional und man kann leicht nachrechnen, dass
    \begin{align*}
      y(x) = \begin{pmatrix}
        a \\ 0
      \end{pmatrix}, \qquad a \in \R
    \end{align*}
    die Randbedingungen erfüllt. \\
  Für $\alpha \neq 0$ lautet unsere Fundamentalmatrix des homogenen Systems
  \begin{align*}
    Y(x) = \exp(xA) = \begin{pmatrix}
      \cos(\alpha x) & \nicefrac{\sin(\alpha x)}{\alpha}  \\ -\alpha\sin(\alpha x) & \cos(\alpha x)
    \end{pmatrix}
  \end{align*}
  Wir gehen nach dem gleichen Schema wie zuvor vor und berechnen
  \begin{align*}
    B &= R_0Y(0) + R_{\pi}Y(\pi) =
    \begin{pmatrix}
      1 & 0 \\ 0 & 1
    \end{pmatrix}
    \begin{pmatrix}
      \cos(0) & \nicefrac{\sin(0)}{\alpha}  \\ -\alpha\sin(0) & \cos(0)
    \end{pmatrix}
    + \begin{pmatrix}
      -1 & 0 \\ 0 & -1
    \end{pmatrix}
    \begin{pmatrix}
      \cos(\alpha \pi) & \nicefrac{\sin(\alpha \pi)}{\alpha}  \\ -\alpha\sin(\alpha \pi) & \cos(\alpha \pi)
    \end{pmatrix} \\
    &= \begin{pmatrix}
      1 & 0 \\ 0 & 1
    \end{pmatrix}
    \begin{pmatrix}
      1 & 0  \\ 0 & 1
    \end{pmatrix}
    + \begin{pmatrix}
      -1 & 0 \\ 0 & -1
    \end{pmatrix}
    \begin{pmatrix}
      \cos(\alpha \pi) & \nicefrac{\sin(\alpha \pi)}{\alpha}  \\ -\alpha\sin(\alpha \pi) & \cos(\alpha \pi)
    \end{pmatrix} \\
    &= \begin{pmatrix}
      1 - \cos(\alpha \pi) & -\nicefrac{\sin(\alpha \pi)}{\alpha} \\
      \alpha\sin(\alpha \pi) & 1 - \cos(\alpha \pi)
    \end{pmatrix}
  \end{align*}
  Klarerweise löst $y \equiv 0$ unser Problem, damit ist die Existenz einer Lösung
  auf jeden Fall gegeben und wir erhalten mit dem zweiten Teil von Satz 6.3, dass
  unser Lösungsraum Dimension $\ker B$ hat.
  Nun brauchen wir eine Fallunterscheidung:
  \begin{itemize}
    \item $\alpha \in 2\Z\backslash\{0\}$: \\
    Es gilt $1 - \cos(\alpha \pi) = 0 = \sin(\alpha \pi)$ und damit
    \begin{align*}
      B = \begin{pmatrix}
        0 & 0 \\ 0 & 0
      \end{pmatrix}.
    \end{align*}
    Der Kern von $B$ und somit der Lösungsraum ist also zwei-dimensional.
    Damit können wir eine allgemeine Lösung $y$ angeben als
    \begin{align*}
      y(x) = Y(x)(a,b)^{\top} = (a\cos(\alpha x) + b\frac{\sin(\alpha x)}{\alpha}, -a\alpha\sin(\alpha x) + b\cos(\alpha x)), \qquad a,b \in \R.
    \end{align*}
    \item $\alpha \notin 2\Z$: \\
      Wir berechnen die Determinante von $B$:
      \begin{align*}
        \det(B) &= \det\left(\begin{pmatrix}
          1 - \cos(\alpha \pi) & -\nicefrac{\sin(\alpha \pi)}{\alpha} \\
          \alpha\sin(\alpha \pi) & 1 - \cos(\alpha \pi)
        \end{pmatrix}\right)
        = (1 - \cos(\alpha \pi))^2 + \sin(\alpha \pi)^2 \\
        &= 1 - 2\cos(\alpha \pi) + \cos(\alpha \pi)^2 + \sin(\alpha \pi)^2
        = 2(1 - \cos(\alpha \pi)) \neq 0.
      \end{align*}
      $B$ ist also regulär und der Lösungsraum ist null-dimensional, es gibt damit nur die triviale $0$-Lösung.
  \end{itemize}
\end{enumerate}
\end{solution}
