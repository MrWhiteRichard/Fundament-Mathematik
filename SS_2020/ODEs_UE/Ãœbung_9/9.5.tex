\begin{exercise}
\leavevmode \\
\begin{enumerate}[label = \textbf{\alph*)}]
  \item Sei $V \in C(G;\R)$ eine Ljapunovfunktion für $y^{\prime} = f(y)$. Nehmen Sie
  an, dass für jedes $y_0 \in G$ die Lösung $y_{0,y_0}$ auf $(0,\infty)$ existiert.
  Zeigen Sie: Für jedes $\alpha \in \R$ ist die Menge $V^{-1}((-\infty,\alpha])$
  eine invariante Menge für die ODE $y^{\prime} = f(y)$.
  \item Sei $f \in C^1(\R^d;\R^d)$ mit $f_i(y) \leq 0$ für alle $y \in \R^d$
  und ein $i \in \{1,\dots,d\}$. Geben Sie eine Ljapunovfunktion für die ODE
  $y^{\prime} = f(y)$ an.
\end{enumerate}
\end{exercise}
\begin{solution}
\leavevmode \\
\begin{enumerate}
  \item Sei $y_0 \in V^{-1}((-\infty,\alpha])$ beliebig. Wir zeigen, dass
  $\gamma_+(y_0) := \{y_{0,y_0}(t): t \geq 0\} \subset V^{-1}((-\infty,\alpha])$ gilt.
  Wir wissen, dass $V$ entlang Lösungen der ODE monoton fällt, also muss
  \begin{align*}
    \forall t > 0: V(y_{0,y_0}(t)) \leq V(y_0) \leq \alpha.
  \end{align*}
  \item Es bietet sich an, ein $V$ zu wählen, welches
  \begin{align*}
    \nabla V = (\underbrace{0,\dots,0}_{i - 1 \text{ viele}},1,0,\dots,0)
  \end{align*}
  erfüllt, da damit sicher
  \begin{align*}
    \nabla V(y) f(y) = f_i(y) \leq 0.
  \end{align*}
  gilt. Also wähle $V(y) := y_i$.
\end{enumerate}
\end{solution}
