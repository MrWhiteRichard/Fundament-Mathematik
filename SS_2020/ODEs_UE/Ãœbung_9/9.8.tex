\begin{exercise}
Betrachten Sie für $d = 1$ das System $y^{\prime} = f(y)$, wobei $f \in C^1(\R;\R)$
mit $f(0) = f(1) = 0$ und $f(y) > 0$ für $y \in (0,1)$. Geben Sie $\omega_+(y_0)$
für $y_0 \in [0,1]$ an.
\end{exercise}
\begin{solution}
Wir machen eine Fallunterscheidung:
Fall 1: $y_0 \in \{0,1\}$

Dann folgt $f(y_0)=0$ und somit ist $y_0$ ein Ruhelage. Die Bahn von $y_0$ ist also $\{y_0\}$ und auch $\omega_+ (y_0) = {y_0}$.

Fall 2: $y_0 \in (0,1))$

Da $f(y) \geq 0$, ist für jede Folge $(t_n)_{n=0}^{\infty}, t_n \to \infty$ somit $y_{0,y_0}(t_n)$ monoton steigend. Da die Folge außerdem beschränkt ist (durch $1$), existiert ein (eindeutiger) Grenzwert.

Nach Definition der Limesmenge $\omega_+ (y_0)$ muss diese also einelementig sein.
Wir behaupten nun, dass $\omega_+ (y_0)=\{1\}$.

Wäre das nicht so, würde ein $x \in (0,1)$ existieren, sodass  $\omega_+ (y_0)=\{x\}$. Nach Satz 5.24 müsste $\omega_+ (y_0)$ aber eine invariante Menge sein. Das gilt nicht, da $x$ offensichtlich keine Ruhelage ist.
\includegraphicsboxed{Satz 5.24}
\end{solution}
