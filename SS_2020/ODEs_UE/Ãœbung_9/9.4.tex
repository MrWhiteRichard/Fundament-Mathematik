\begin{exercise}
Die Existenz einer (strikten) Ljapunovfunktion erlaubt es, die Konvergenz einer
Lösung gegen die asymptotisch stabile Ruhelage zu quantifizieren. Seien hierzu
$V \in C^2(\R^d;\R)$ und $f \in C^1(\R^d;\R), y^* = 0$ eine Ruhelage für $y^{\prime} = f(y)$
und ein striktes Minimum von $V$. Nehmen Sie an, dass für das Spektrum der Matrix
\begin{align*}
  B := (\nabla f(0))^{\top}\nabla^2 V(0) + \nabla^2V(0)\nabla f(0)
\end{align*}
gilt: $\max\{\Re(\lambda): \lambda \in \sigma(B)\} =: \alpha < 0$.
Zeigen Sie: Es existieren $\beta,C > 0$, sodass für alle $y_0$ hinreichend nahe
bei $y^* = 0$ gilt:
\begin{align*}
  \|y_{0,y_0}(t)\| \leq C\exp(-\beta t).
\end{align*}
Wovon hängt $\beta$ ab?
\end{exercise}

\begin{solution}
Beweis.
\end{solution}
