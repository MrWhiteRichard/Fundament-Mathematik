\begin{exercise}
Bestimmen Sie die allgemeine Lösung der folgenden ODEs mit der
Ansatzmethode:
\begin{enumerate}[label = \textbf{\alph*)}]
\item \begin{align*}
  y^{\primeprime} + y &= \sin(t) + \sin(3t),
\end{align*}
\item \begin{align*}
  y^{\primeprime} + y = t\exp(-2t)\cos(t),
\end{align*}
\item \begin{align*}
  y^{\primeprime} - y = t\exp(-t).
\end{align*}
\end{enumerate}
Untersuchen Sie, ob die Lösungen dieser ODEs stabil, beziehungsweise
asymptotisch stabil sind.
\end{exercise}
\begin{solution}
\leavevmode \\
\begin{enumerate}[label = \textbf{\alph*)}]
  \item Das charakteristische Polynom des homogenen Systems lautet
  \begin{align*}
    \chi(\lambda) = \lambda^2 + 1
  \end{align*}
  mit den komplexen Nullstellen
  \begin{align*}
    \lambda_1 = i, \qquad \lambda_2 = -i.
  \end{align*}
  Also erhalten wir mit Satz 2.18 mit
  \begin{align*}
    y_1(t) = \exp(it), \qquad y_2(t) = \exp(-it)
  \end{align*}
  ein Fundamentalsystem für die homogene Gleichung.
  Eine beliebige Lösung $\widetilde{y}$ des homogenen Systems hat daher die Form
  \begin{align*}
    \widetilde{y}(t) = a_1\exp(it) + a_2\exp(-it)
  \end{align*}
  Eine allgemeine Lösung des inhomogenen Systems, lässt sich durch
  \begin{align*}
    y(t) = \widetilde{y}(t) + y_p(t)
  \end{align*}
  darstellen, wobei $y_p$ eine Partikulärlösung des inhomogenen System ist.
  Um diese nun zu berechnen, schreiben wir zuerst die Differentialgleichung folgendermaßen um:
  \begin{align*}
    y^{\primeprime} + y &= \sin(t) + \sin(3t)
    = -\frac{i}{2}(\exp(it) + \exp(3it) - \exp(-3it) - \exp(-it)).
  \end{align*}
  Jetzt können wir für $b_{1,2} = \mp\frac{i}{2}\exp(\pm it), b_{3,4} = \mp\frac{i}{2}\exp(\pm 3it)$
  jeweils seperat Teil-Partikulärlösungen berechnen und diese anschließend zur gesamten Partikulärlösung
  aufsummieren. Einen Ansatz dafür liefert uns Satz 3.20. Da $\chi(\pm i) = 0$ eine
  einfache Nullstelle ist und $\mp\frac{i}{2}$ ein Polynom vom Grad $0$,
  sind wir im 2.Fall und setzen
  \begin{align*}
    y_{p_1}(t) = ct\exp(it)
  \end{align*}
  an. Einsetzen in die Differentialgleichung liefert
  \begin{align*}
    y_{p_1}^{\primeprime}(t) + y_{p_1}(t) = c(2i - t + t)\exp(it)\stackrel{!}{=}
    -\frac{i}{2}\exp(it) \iff c = -\frac{1}{4}.
  \end{align*}
  Damit erhalten wir die erste Teil-Partikulärlösung $y_{p_1}(t) = -\frac{t}{4}\exp(it)$. \\
  Analog berechnen wir $y_{p_2}(t) = -\frac{t}{4}\exp(-it)$. \\
  Für $b_{3,4} = \mp\frac{i}{2}\exp(\pm 3it)$, sind wir aufgrund $\chi(\pm 3i) \neq 0$
  diesmal im 1.Fall des Satzes. Unser Ansatz lautet daher
  \begin{align*}
    y_{p_3}(t) = c\exp(3it).
  \end{align*}
  Einsetzen in die Differentialgleichung liefert
  \begin{align*}
    -8c\exp(3it) \stackrel{!}{=} \frac{-i}{2}\exp(3it) \iff c = \frac{i}{16}
  \end{align*}
  Die dritte Teil-Partikulärlösung lautet daher $y_{p_3}(t) = \frac{i}{16}\exp(3it)$.
  Wiederum analog berechnet man die letzte verbleibende Teil-Partikulärlösung
  und erhält $y_{p_4}(t) = -\frac{i}{16}\exp(-3it)$.
  Insgesamt haben wir mit
  \begin{align*}
    y_p(t) = -\frac{t}{4}\left(\exp(it) + \exp(-it)\right) +
    \frac{i}{16}\left(\exp(3it) - \exp(-3it)\right)
  \end{align*}
  eine Gesamt-Partikulärlösung gefunden.
  Die allgemeine Form der Lösung lautet daher
  \begin{align*}
    y(t) = y_p(t) + \widetilde{y}(t) = -\frac{t}{4}\left(\exp(it) + \exp(-it)\right) +
    \frac{i}{16}\left(\exp(3it) - \exp(-3it)\right) + a_1\exp(it) + a_2\exp(-it)
  \end{align*}
  Jetzt müssen wir auch noch die Stabilität dieser Lösung überprüfen.
  Aus der Vorlesung ist bekannt, das eine Lösung $y_{t_0,y_0}$ des homogenen Systems
  genau dann (asymptotisch) stabil ist, wenn $y^* \equiv 0$ eine (asymptotisch)
  stabile Lösung des homogenen System ist. Durch Einführung von Variablen können
  wir die gegebene Differentialgleichung in ein System linearer Differentialgleichungen umschreiben:
  \begin{align*}
    y^{\prime} &= z \\
    z^{\prime} &= \sin(t) + \sin(3t) - y
  \end{align*}
  Mit $Y(t) = (y(t),z(t))^{\top}$ erhalten wir also das lineare System
  \begin{align*}
    Y^{\prime}(t) = \begin{pmatrix}
      0 & 1 \\
      -1 & 0
    \end{pmatrix}Y(t)
    + \begin{pmatrix}
      0 \\ \sin(t) + \sin(3t)
    \end{pmatrix}
  \end{align*}
  Eine Fundamentalmatrix $Z$ des homogenen Systems lautet somit
  \begin{align*}
    Z(t) = \exp\begin{pmatrix}
      0 & t \\ -t & 0
    \end{pmatrix} = \begin{pmatrix}
      \cos(t) & \sin(t) \\ -\sin(t) & \cos(t)
    \end{pmatrix}
  \end{align*}
  Wie man leicht erkennt, gilt
  \begin{align*}
    \sup_{t \geq t_0} ||Z(t)|| < \infty
  \end{align*}
  und mit Satz 5.6 folgt damit, dass die Ruhelage $y^*$ für das homogene Gleichungssystem
  stabil ist und somit auch unsere Lösung $y_{t_0,y_0}$ für das inhomogene System.
  Da allerdings
  \begin{align*}
    \lim_{t \rightarrow \infty} ||Z(t)|| \neq 0
  \end{align*}
  ist die Lösung nicht attraktiv und somit nicht asymptotisch stabil.
  \item Das charakteristische Polynom lautet weiterhin
  \begin{align*}
    \chi(\lambda) = \lambda^2 + 1
  \end{align*}
  mit den selben Nullstellen
  \begin{align*}
    \lambda_1 = i, \qquad \lambda_2 = -i.
  \end{align*}
  Wieder formen wir die Differentialgleichung um
  \begin{align*}
    y^{\primeprime} + y = t\exp(-2t)\cos(t) = \frac{t}{2}\exp(-2t)(\exp(it) + \exp(-it))
    = \frac{t}{2}(\exp((i-2)t) + \exp(-(i+2)t))
  \end{align*}
  und wieder teilen wir die Partikulärlösungen auf. Für $b_{1} = \frac{t}{2}\exp((i-2)t)$
  gilt $\chi(i - 2) \neq 0$ und somit sind wir im Fall 1. Der Ansatz lautet
  \begin{align*}
    y_{p_1}(t) = (c_1t + c_0)\exp((i-2)t)
  \end{align*}
  Was passiert jetzt bloß? Wir setzen in die Differentialgleichung ein...
  \begin{align*}
    &y_{p_1}^{\primeprime}(t) + y_{p_1}(t) = [c_1(2i-4 + (i - 2)^2t + t) + c_0((i-2)^2 + 1)]\exp((i-2)t)
    \stackrel{!}{=} \frac{t}{2}(\exp((i-2)t) \\
    &\iff 4c_1(1-i)t + c_1(2i-4) + 4c_0(1 - i) = \frac{t}{2} \\
  \end{align*}
  Wir machen nun den Ansatz im Ansatz
  \begin{align*}
    4c_1(1-i)t = \frac{t}{2} \iff c_1 = \frac{1}{8(1-i)},
  \end{align*}
  und setzen zurück ein
  \begin{align*}
    \frac{2i-4}{8(1-i)} + 4c_0(1-i) = 0
    \iff c_0 = - \frac{i-2}{(4(1-i))^2} = - \frac{(i-2)}{16(1-2i-1)}
    = \frac{(i-2)}{32i} = \frac{1 + 2i}{32}.
  \end{align*}
  Für $b_2 = \frac{t}{2}(\exp(-(i+2)t)$ landen wir dank $\chi(-(i+2)) \neq 0$
  wieder im zweiten Fall.
  \begin{align*}
    y_{p_2}(t) &= (c_1t + c_0)\exp(-(i+2)t) \\
    y_{p_2}^{\primeprime}(t) + y_{p_2}(t) &=
    [c_1(-(2i+4) + (i + 2)^2t + t) + c_0((i + 2)^2 + 1)]\exp(-(i+2)t)
    \stackrel{!}{=} \frac{t}{2}(\exp(-(i+2)t) \\\
    &\iff [c_1(-(4 + 2i) + (i + 2)^2t + t) + c_0((i + 2)^2 + 1)] = \frac{t}{2}
  \end{align*}
  Überraschenderweise wählen wir den Ansatz
  \begin{align*}
    c_1((i + 2)^2+1)t = \frac{t}{2} \iff c_1 = \frac{1}{2(4 + 4i)} = \frac{3 - 4i}{64}
  \end{align*}
  und erhalten
  \begin{align*}
    c_0((i + 2)^2 + 1) = \frac{(3 - 4i)(4+2i)}{64}
    \iff c_0 = \frac{(3 - 4i)(4+2i)}{64(4 + 4i)} = \frac{5 - 15i}{256}
  \end{align*}
  wenig überraschend endlich die finale Partikulärlösung
  \begin{align*}
    y_p(t) = \left(\frac{1}{8(1-i)}t + \frac{1 + 2i}{32}\right)\exp((i-2)t) +
    \left(\frac{3 - 4i}{64}t + \frac{5 - 15i}{256}\right)\exp(-(i+2)t),
  \end{align*}
  womit wir leicht die allgemeine Lösung
  \begin{align*}
  y(t) = \left(\frac{1}{8(1-i)}t + \frac{1 + 2i}{32}\right)\exp((i-2)t) +
  \left(\frac{3 - 4i}{64}t + \frac{5 - 15i}{256}\right)\exp(-(i+2)t) +
  a_1\exp(it) + a_2\exp(-it)
  \end{align*}
  angeben können.
  Zur Untersuchung der Stabilität greifen wir wieder auf eine Variableneinführung
  zurück und erhalten das lineare System
  \begin{align*}
    Y^{\prime}(t) = \begin{pmatrix}
      0 & 1 \\
      -1 & 0
    \end{pmatrix}Y(t)
    + \begin{pmatrix}
      0 \\ t\exp(-2t)\cos(t)
    \end{pmatrix}
  \end{align*}
  Wie in Beispiel a) bereits bemerkt, ist $y_{t_0,y_0}$ stabil, aber nicht asymptotisch stabil.
  \item Zur Abwechslung kommt auch mal ein anderes charakteristisches Polynom dran:
  \begin{align*}
    \chi(\lambda) = \lambda^2 - 1
  \end{align*}
  Die Nullstellen davon sind
  \begin{align*}
    \lambda_1 = 1, \qquad \lambda_2 = -1.
  \end{align*}
  Unsere neues Fundamentalsystem der homogenen Gleichung lautet
  \begin{align*}
    y_1(t) = \exp(t), \qquad y_2(t) = \exp(-t)
  \end{align*}
  Diesmal brauchen wir für $b(t) = t\exp(-t)$ nur eine Partikulärlösung berechnen.
  Da $\chi(-1) = 0$ befinden wir uns im zweiten Fall. Der Ansatz lautet
  \begin{align*}
    y_p(t) = t(c_1t + c_0)\exp(-t).
  \end{align*}
  Wir setzen ein letztes Mal in die Differentialgleichung ein:
  \begin{align*}
    &[c_1(2 - 4t + t^2 - t^2)+ c_0(-2 + t - t)]\exp(-t) = t\exp(-t) \\
    &\iff c_1(2 - 4t) - 2c_0 = t
  \end{align*}
  Wir lesen die Lösungen
  \begin{align*}
    c_0 = c_1 = -\frac{1}{4}
  \end{align*}
  ab. Die allgemeine Lösung lautet damit
  \begin{align*}
    y(t) = -\frac{t}{4}(t + 1)\exp(-t) + a_1\exp(t) + a_2\exp(-t).
  \end{align*}
  Für die Stabilität betrachte wieder das umgeformte, lineare System
  \begin{align*}
    Y^{\prime}(t) = \begin{pmatrix}
      0 & 1 \\
      1 & 0
    \end{pmatrix}Y(t)
    + \begin{pmatrix}
      0 \\ t\exp(-t)
    \end{pmatrix}
  \end{align*}
  mit der zugehörigen Fundamentalmatrix
  \begin{align*}
    Z(t) = \begin{pmatrix}
      1 & \exp(t) \\
      \exp(t) & 1
    \end{pmatrix}
  \end{align*}
  und diesmal haben wir
  \begin{align*}
    \sup_{t \geq t_0} ||Z(t)|| = \infty
  \end{align*}
  und somit Instabilität.
\end{enumerate}
\end{solution}
