\begin{exercise}

Sei $G \subset \R^{d+1}$ ein Gebiet, $f \in C(G; \R^d)$ lokal lipschitzstetig im 2. Argument.
Sei $y \in C^1((a, b); \R^d)$ eine Lösung von $y^\prime = f(t, y)$.
Zeigen Sie:
Falls es eine Folge $(t_n)^\infty_{n=0}$ mit $\lim_{n \to \infty}(t_n, y(t_n)) = (b, y_b) \in G$ gibt, so kann $y$ über $b$ hinaus fortgesetzt werden, d.h. es gibt ein $b^\prime > b$ und ein $y_e \in C^1((a, b^\prime), \R^d)$ mit $y_e^\prime = f(t, y_e)$ auf $(a, b^\prime)$ und $y = y_e$ auf $(a, b)$.
Hinweis:
Zeigen Sie $\lim_{t \to b-} y(t) = y_b$ mittels Widerspruch.
Betrachten Sie hierzu Folgen $(\tau_n)_n$ mit $t_n \leq \tau_n$ und $\norm[\R^d]{y(\tau_n) - y_b} = \epsilon > 0$ sowie $\norm[\R^d]{y(t) - yb} \leq \epsilon$ für $t \in [t_n, \tau_n]$.
Verwenden Sie, daß $f$ in einer Umgebung von $(b, y_b)$ beschränkt ist.

\end{exercise}

\begin{solution}

$f$ lokal lipschitzstetig im 2. Argument heißt genau, dass
$\Forall (t, y) \in G:
\Exists U \text{ Umgebung von } (t, y):
\Exists L_U > 0:
\Forall (\tilde{t}, \tilde{y}_1), (\tilde{t}, \tilde{y}_2) \in U:$

\begin{align*}
  \norm{f(\tilde{t}, \tilde{y}_1) - f(\tilde{t}, \tilde{y}_2)}
  \leq
  L_U \norm{\tilde{y}_2 - \tilde{y}_2}.
\end{align*}

zz:

\begin{align*}
  \Exists (t_n)_{n \in \N} \in (a, b)^\N:
  (t_n, y(t_n)) \xrightarrow{n \to \infty} (b, y_b)
  \Rightarrow
  \Exists b^\prime > b:
  \Exists \tilde{y} \in C^1((a, b^\prime), \R^d):
  \begin{cases}
    \tilde{y}^\prime = f(t, \tilde{y})
    & \text{auf } (a, b^\prime) \\
    \tilde{y} = y
    & \text{auf } (a, b)
  \end{cases}
\end{align*}

\end{solution}
