\begin{exercise}
Sei $c_1 = 0, c_3 = 1$ und $c_2 \in (0,1)$ beliebig.
\begin{enumerate}[label = \textbf{\alph*)}]
  \item Welche Konvergenzordnung ist für 3-stufige Runge-Kutta-Verfahren erreichbar,
  wenn diese durch Kollokation mit diesen Kollokationspunkten erzeugt werden?
  \item Geben Sie die Butcher Tableaus dieser Verfahren an.
\end{enumerate}
\end{exercise}
\begin{solution}
\leavevmode \\
\begin{enumerate}[label = \textbf{\alph*)}]
  \item Theorem 3.28 sagt uns, dass für $m$-stufige Runge-Kutta-Verfahren $2m$ die maximal
  erreichbare Konvergenzordnung darstellt, also erhalten wir in unserem Fall maximal Konvergenzordnung 6.
  Aus Numerik A kennen wir mit
  \begin{align*}
    Q_2(f) = \frac{5}{9}f\left(-\sqrt{\frac{3}{5}}\right) + \frac{8}{9}f\left(0\right)
    + \frac{5}{9}f\left(\sqrt{\frac{3}{5}}\right)
  \end{align*}
  die Gauß-Legendre-Quadraturformel mit drei Stützstellen auf dem Intervall $[0,1]$,
  für die konstante Gewichtsfunktion $\omega \equiv 0$. Um die Quadraturpunkte unseren
  vorgegeben Stützstellen anzupassen führen wir eine affine Transformation nach dem Schema
  \begin{align*}
    \int_a^b f(x)dx \approx \frac{b-a}{2}\sum_{i = 1}^n\alpha_i f\left(\frac{(b-a)x_i + (a+b)}{2}\right)
  \end{align*}
  durch und erhalten das Intervall $\left[\frac{1}{2} - \sqrt{\frac{5}{12}}, \frac{1}{2} + \sqrt{\frac{5}{12}}\right]$
  \begin{align*}
  Q_2(f) = \sqrt{\frac{5}{12}}\left(\frac{5}{9}f(0) + \frac{8}{9}f\left(\frac{1}{2}\right)
  + \frac{5}{9}f(1)\right).
  \end{align*}
  \textit{Die affine Transformation hab ich auf Wikipedia bestätigt gesehen, hab
  aber noch kein stichhaltiges Argument dafür, warum das ohne weiteres durchgeht.
  Ich schätze, man könnte sich anschauen, dass die entsprechenden Polynome genauso exakt
  integriert werden, indem man sie analog transformiert.}
  Aufgrund der Eindeutigkeit der Gaußschen Quadraturformeln sind dies auch die
  einzigen Kollokationspunkte mit denen maximale Konvergenzordnung erzielt werden kann.
  \item Um das zugehörige Butcher Tableau zu bestimmen, benötigen wir vorerst die
  zugehörigen Lagrange Basispolynome
  \begin{align*}
    L_1(x) &= 2(x - 1/2)(x - 1) = 2x^2 - 3x + 1\\
    L_2(x) &= -4x(x - 1) = -4x^2 + 4x\\
    L_3(x) &= 2x(x - 1/2) = 2x^2 - x.
  \end{align*}
  Mit den Formeln
  \begin{align*}
    A_{ij} = \int_{\frac{1}{2} - \sqrt{\frac{5}{12}}}^{c_i} L_j dt \\
    b_j = \int_{\frac{1}{2} - \sqrt{\frac{5}{12}}}^{\frac{1}{2} + \sqrt{\frac{5}{12}}} L_j dt
  \end{align*}
  berechnen wir schließlich
  \begin{align*}
    b &= \frac{1}{18}\sqrt{\frac{5}{3}}
    \begin{pmatrix}
       5 \\ 8 \\ 5
    \end{pmatrix}\\
    A &=
     \begin{pmatrix}
      \frac{5}{36}\sqrt{\frac{5}{3}} & \frac{1}{27}\left(2\sqrt{15} - 9\right)
      & \frac{5}{36}\sqrt{\frac{5}{3}} - \frac{1}{6}\\
      \frac{5}{216}\left(9 + 2\sqrt{15}\right) & \frac{2}{9}\sqrt{\frac{5}{3}}
      & \frac{5}{216}\left(-9 + 2\sqrt{15}\right)\\
      \frac{5}{36}\sqrt{\frac{5}{3}} + \frac{1}{6} & \frac{1}{27}\left(2\sqrt{15} + 9\right)
      & \frac{5}{36}\sqrt{\frac{5}{3}}
    \end{pmatrix}.
  \end{align*}
\end{enumerate}
Wunderschön, nicht wahr?
\end{solution}
