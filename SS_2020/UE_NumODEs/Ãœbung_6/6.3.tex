\begin{exercise}
Sei
\begin{align}
  Mh := \frac{1}{h^2}\begin{pmatrix}
    -2 & 1 & &  \\
    1 & -2 & \ddots & \\
    & \ddots & \ddots & 1 \\
    & & 1 & -2
  \end{pmatrix}
  \in \R^{(N-1)\times (N-1)}
\end{align}
mit $N \in \N$ und $h := \nicefrac{1}{N}$ die Matrix aus Example 4.2 des Vorlesungsskripts.
\begin{enumerate}[label = \textbf{\alph*)}]
  \item Zeigen Sie, dass die Eigenwerte $\lambda_j$ mit zugehörigen Eigenvektoren $v_j$
  von $M_h$ gegeben sind durch
  \begin{subequations}
  \begin{align}\label{ew}
    \lambda_j = \frac{2}{h^2}\left(-1 + \cos\left(\frac{j\pi}{N}\right)\right),
    \qquad j = 1,\dots,N-1,
  \end{align}
  beziehungsweise
  \begin{align}
    v^{(j)} := \left(\sin\left(\frac{j\pi}{N}\right), \sin\left(\frac{2j\pi}{N}\right),\dots,
    \sin\left(\frac{(N-1)j\pi}{N}\right)\right)^{\top}.
  \end{align}
  \end{subequations}
  \item Begründen Sie, warum das zugehörige Anfangswertproblem
  \begin{align}\label{awp}
    U_h^{\prime} = M_hU_h, \qquad U_h(0) = G
  \end{align}
  für $G \in \R^{N-1}$ steif genannt wird. Beweisen Sie dazu, dass
  $\lim_{N \rightarrow \infty} \lambda_1 = -\pi^2$ und
  $\lim_{N \rightarrow \infty} \lambda_{N-1} = -\infty$.
\end{enumerate}
\end{exercise}
\begin{solution}
\leavevmode \\
\begin{enumerate}[label = \textbf{\alph*)}]
\item
\end{enumerate}
\end{solution}
