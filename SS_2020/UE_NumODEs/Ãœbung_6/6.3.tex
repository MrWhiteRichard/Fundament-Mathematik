\begin{exercise}
Sei
\begin{align}
  M_h := \frac{1}{h^2}\begin{pmatrix}
    -2 & 1 & &  \\
    1 & -2 & \ddots & \\
    & \ddots & \ddots & 1 \\
    & & 1 & -2
  \end{pmatrix}
  \in \R^{(N-1)\times (N-1)}
\end{align}
mit $N \in \N$ und $h := \nicefrac{1}{N}$ die Matrix aus Example 4.2 des Vorlesungsskripts.
\begin{enumerate}[label = \textbf{\alph*)}]
  \item Zeigen Sie, dass die Eigenwerte $\lambda_j$ mit zugehörigen Eigenvektoren $v_j$
  von $M_h$ gegeben sind durch
  \begin{subequations}
  \begin{align}\label{ew}
    \lambda_j = \frac{2}{h^2}\left(-1 + \cos\left(\frac{j\pi}{N}\right)\right),
    \qquad j = 1,\dots,N-1,
  \end{align}
  beziehungsweise
  \begin{align}
    v^{(j)} := \left(\sin\left(\frac{j\pi}{N}\right), \sin\left(\frac{2j\pi}{N}\right),\dots,
    \sin\left(\frac{(N-1)j\pi}{N}\right)\right)^{\top}.
  \end{align}
  \end{subequations}
  \item Begründen Sie, warum das zugehörige Anfangswertproblem
  \begin{align}\label{awp}
    U_h^{\prime} = M_hU_h, \qquad U_h(0) = G
  \end{align}
  für $G \in \R^{N-1}$ steif genannt wird. Beweisen Sie dazu, dass
  $\lim_{N \rightarrow \infty} \lambda_1 = -\pi^2$ und
  $\lim_{N \rightarrow \infty} \lambda_{N-1} = -\infty$.
\end{enumerate}
\end{exercise}
\begin{solution}
\leavevmode \\
\begin{enumerate}[label = \textbf{\alph*)}]
\item
Wir verwenden die Additionstheoreme
\begin{align}
  \sin(x_1 + x_2) &= \sin(x_1)\cos(x_2) + \sin(x_2)\cos(x_1) \label{add1} \\
  \sin(2x) &= 2\sin(x)\cos(x) \label{add2} \\
  \sin(x_1 - x_2) &= \sin(x_1)\cos(x_2) - \sin(x_2)\cos(x_1) \label{add3}
\end{align}
und berechnen ganz stumpf
\begin{align*}
  M_h v^{(j)} = \frac{1}{h^2}\begin{pmatrix}
    -2 & 1 & &  \\
    1 & -2 & \ddots & \\
    & \ddots & \ddots & 1 \\
    & & 1 & -2
  \end{pmatrix}
  \begin{pmatrix}
  \sin\left(\frac{j\pi}{N}\right) \\
  \sin\left(\frac{2j\pi}{N}\right) \\
  \vdots \\
  \sin\left(\frac{(N-1)j\pi}{N}\right)
  \end{pmatrix}
  = \frac{1}{h^2}\begin{pmatrix}
    -2\sin\left(\frac{j\pi}{N}\right) + \sin\left(\frac{2j\pi}{N}\right) \\
    \sin\left(\frac{j\pi}{N}\right) - 2\sin\left(\frac{2j\pi}{N}\right) + \sin\left(\frac{3j\pi}{N}\right)\\
    \vdots \\
    \sin\left(\frac{(N-2)j\pi}{N}\right) - 2\sin\left(\frac{(N-1)j\pi}{N}\right)
  \end{pmatrix}.
\end{align*}

Mit den Additionstheoremen erhalten wir die Zwischenresultate
\begin{align*}
  \sin\left(\frac{2j\pi}{N}\right) &\stackrel{\eqref{add2}}{=}
  2\sin\left(\frac{j\pi}{N}\right)\cos\left(\frac{j\pi}{N}\right) \\
  \sin\left(\frac{(k-1)j\pi}{N}\right) &= \sin\left(\frac{kj\pi}{N}- \frac{j\pi}{N}\right)\stackrel{\eqref{add3}}{=}
  \sin\left(\frac{kj\pi}{N}\right)\cos\left(\frac{j\pi}{N}\right)
  - \sin\left(\frac{j\pi}{N}\right)\cos\left(\frac{kj\pi}{N}\right) \\
  \sin\left(\frac{(k+1)j\pi}{N}\right) &= \sin\left(\frac{2j\pi}{N}+ \frac{j\pi}{N}\right)
  \stackrel{\eqref{add1}}{=}
  \sin\left(\frac{kj\pi}{N}\right)\cos\left(\frac{j\pi}{N}\right)
  + \sin\left(\frac{j\pi}{N}\right)\cos\left(\frac{kj\pi}{N}\right).
\end{align*}
Wenn wir in der letzten Zeile noch den Term
$0 = \sin\left(\frac{Nj\pi}{N}\right)$ ergänzen haben wir
\begin{align*}
  M_h v^{(j)} = \frac{1}{h^2}\begin{pmatrix}
    2\sin\left(\frac{j\pi}{N}\right)\left(\cos\left(\frac{j\pi}{N}\right) - 1\right) \\
    2\sin\left(\frac{2j\pi}{N}\right)\left(\cos\left(\frac{j\pi}{N}\right) - 1\right)\\
    \vdots \\
    2\sin\left(\frac{(N-1)j\pi}{N}\right)\left(\cos\left(\frac{j\pi}{N}\right) - 1\right)
  \end{pmatrix}
  = \lambda_j v^{(j)}
\end{align*}
damit die Aussage gezeigt.
\item Wir rechnen die Limiten nach:
\begin{align*}
  \lim_{N \rightarrow \infty} \lambda_1 =
  \lim_{h \rightarrow 0} \frac{2\left(-1 + \cos\left(h\pi\right)\right)}{h^2}
\end{align*}
Da Zähler und Nenner gegen $0$ konvergieren, wenden wir L'Hospital an:
\begin{align*}
  \lim_{h \rightarrow 0} \frac{2\left(-1 + \cos\left(h\pi\right)\right)}{h^2}
  &= \lim_{h \rightarrow 0} \frac{-2\pi\sin(h\pi)}{2h}
  = \lim_{h \rightarrow 0} -\pi^2\cos(h\pi) = -\pi^2.
\end{align*}
Jetzt zum zweiten Limes:
\begin{align*}
\lim_{N \rightarrow \infty} \lambda_{N-1} =
\lim_{h \rightarrow 0} \frac{2\left(-1 + \cos\left((1 - \frac{1}{h})\pi\right)\right)}{h^2}
= \lim_{N \rightarrow \infty} -\frac{4}{h^2} = -\infty.
\end{align*}
\end{enumerate}
\end{solution}
