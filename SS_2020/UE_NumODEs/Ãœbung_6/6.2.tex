\begin{exercise}
Sei zunächst $[a,b] = [-1,1]$ und $\omega(x) \equiv 1$.
\begin{enumerate}[label = \textbf{\alph*)}]
  \item Zeigen Sie, dass die Orthogonalpolynome $(q_s)_{s \in \N_0}$ aus Remark B.13
  des Vorlesungsskripts gerade, beziehungsweise ungerade Polynome sind, falls $s$
  gerade, beziehungsweise ungerade ist. Dazu können Sie die Konstruktion dieser
  Polynome aus der Monombasis mit Hilfe des Orthogonalisierungsverfahrens von
  Gram-Schmidt verwenden.
  \item
  \renewcommand{\arraystretch}{1.2}
  Sei nun ~$\begin{matrix}
    c & \vline & A \\
    \hline
    & \vline & b^{\top}
  \end{matrix}$~
  \renewcommand{\arraystretch}{1}
  das Butcher Tableau eines durch Kollokation erzeugten $m$-stufigen Runge-Kutta-Verfahrens,
  bei dem Gauß-Quadratur zur Grundlage genommen wurde. Beweisen Sie, dass die Kollokationspunkte
  und die Gewichte im folgenden Sinn symmetrisch sind:
  \begin{align}
    \left|c_j - \frac{1}{2}\right| = \left| c_{m + 1 - j} - \frac{1}{2}\right|,
    \qquad b_j = b_{m+1-j}, \qquad j = 1,\dots,m.
  \end{align}
\end{enumerate}
\end{exercise}
\begin{solution}
\leavevmode \\
\begin{enumerate}[label = \textbf{\alph*)}]
\item
Mit dem Orthogonalisierungsverfahren von Gram-Schmidt angewandt auf die Monombasis,
erhalten wir $q_s$ folgendermaßen:
\begin{align*}
q_s(x) = x^s - \sum_{i = 0}^{s-1}\frac{\langle q_i(x), x^s\rangle}{\langle q_i(x), q_i(x) \rangle}q_i(x),
\end{align*}
wobei das Skalarprodukt als
\begin{align*}
  \langle f, g \rangle := \int_a^b w(y)f(y)g(y) dy = \int_{-1}^1 f(y)g(y) dy
\end{align*}
definiert ist. Also haben wir
\begin{align*}
  q_s(x) = x^s - \sum_{i = 0}^{s-1}\frac{\int_{-1}^1 q_i(y) y^s dy}{\int_{-1}^1 q_i^2(y) dy}q_i(x)
\end{align*}
Wir zeigen die Aussage mit Induktion nach $s$. Für $s = 0$ lautet $q_0 \equiv 1$,
was klarerweise ein gerades Polynom darstellt. Für $s = 1$ erhalten wir im ersten
Schritt des Orthogonalisierungsverfahrens
\begin{align*}
  q_1(x) = x - \frac{\int_{-1}^1 q_0(y) y dy}{\int_{-1}^1 q_0^2(y) dy}q_0(x)
  = x - \frac{\int_{-1}^1 y dy}{\int_{-1}^1 1 dy}1 = x - \frac{0}{2} = x,
\end{align*}
was ein ungerades Polynom darstellt. \\
Gelte nun die Aussage für $k < s$.
\begin{itemize}
  \item Fall 1: $s$ gerade:
  \begin{align*}
    q_s(x) = x^s - \sum_{i = 0}^{s-1}\frac{\int_{-1}^1 q_i(y) y^s dy}{\int_{-1}^1 q_i^2(y) dy}q_i(x)
  \end{align*}
  Für $i$ ungerade ist $q_i(x) x^s$ eine ungerade Funktion und das Integral verschwindet somit zu 0.
  Also vereinfacht sich die Summe zu
  \begin{align*}
    q_s(x) = x^s - \sum_{i = 0}^{s/2-1}\frac{\int_{-1}^1q_{2i}(y)y^s dy}{\int_{-1}^1q_{2i}^2(y)dy}q_{2i}(x)
    = (-x)^s - \sum_{i = 0}^{s/2-1}\frac{\int_{-1}^1 q_{2i}(y) y^s dy}{\int_{-1}^1 q_{2i}^2(y) dy}q_{2i}(-x)
    = q_s(-x)
  \end{align*}
  und $q_s$ ist daher eine gerade Funktion.
  \item Fall 2: $s$ ungerade: \\
  In diesem Fall verschwinden genau die Integrale der geraden $i$ und wir erhalten
  \begin{align*}
    q_s(x) &= x^s - \sum_{i = 0}^{s-1}\frac{\int_{-1}^1 q_i(y) y^s dy}{\int_{-1}^1 q_i^2(y) dy}q_i(x)
    = x^s - \sum_{i = 0}^{s/2-1}\frac{\int_{-1}^1q_{2i+1}(y)y^s dy}{\int_{-1}^1q_{2i+1}^2(y)dy}q_{2i+1}(x)\\
    &= -(-x)^s +
    \sum_{i = 0}^{s/2-1}\frac{\int_{-1}^1q_{2i+1}(y)y^s dy}{\int_{-1}^1q_{2i+1}^2(y)dy}q_{2i+1}(-x)
    = -q_s(-x)
  \end{align*}
  eine ungerade Funktion.
\end{itemize}
\item Um aus den Nullstellen $(\widetilde{c}_j)_{j = 1}^m$ des $m$-ten Orthogonalpolynoms,
welche im Intervall $(-1,1)$ liegen, auf die zugehörigen Kollokationspunkte zu kommen,
müssen wir sie auf das Intervall $[0,1]$ transformieren.
Da die Nullstellen des $m$-ten Orthogonalpolynoms symmetrisch um die $0$ liegen
($\widetilde{c}_j = -\widetilde{c}_{m+j-1}$), folgt nach der Transformation
\begin{align*}
  c_j = \frac{\widetilde{c_j} + 1}{2}
\end{align*}
und
\begin{align*}
  \left|c_j - \frac{1}{2}\right| = \left|\frac{\widetilde{c_j}}{2}\right| =
  \left|\frac{\widetilde{c}_{m+1-j}}{2}\right| = \left|c_{m+1-j} - \frac{1}{2}\right|.
\end{align*}
Die Quadraturgewichte berechnen sich folgendermaßen:
\begin{align*}
  b_j = \int_{0}^1 L_j(t) dt, \qquad j = 1,\dots,m.
\end{align*}
Definieren wir
\begin{align*}
  P_j(x) := \left(\prod_{k = 1, k \notin \{j,m+1-j\}}^m\frac{x - c_k}{c_j - c_k}\right),
\end{align*}
erhalten wir

\begin{align*}
  L_j(x) &= P_j(x)\left(\frac{x - c_{m+1-j}}{c_j - c_{m+1-j}}\right)\\
  L_{m+1-j}(x) &= P_{m+1-j}(x)\left(\frac{x - c_{j}}{c_{m+1-j} - c_j}\right)
\end{align*}
Wenn man die Faktoren im Nenner von $P_j, P_{m+1-j}$ vergleicht erhält man zu jedem $(c_j - c_k)$
ein $(c_{m+1-j}- c_{m+1-k}) = (c_k - c_j)$. Also unterscheiden sich die Nenner
insgesamt nur um den Faktor $(-1)^{m-2}$.
\end{enumerate}
\end{solution}
