\begin{exercise}
  Gesucht ist die Lösung $u$ des Poisson Problems
  \begin{align}\label{one}
    -\partial_x (k(x)\partial_x u(x))
    =
    \cos(x),
    \quad
    x \in (0,2\pi)
  \end{align}

  mit $u(0) = u(2\pi) = 1$ und gegebenen Wärmeleitkoeffizienten $k$. Sie
  beschreibt die Temperaturverteilung in einem dünnen Stab.

  \begin{itemize}
  \item[a)] Lösen Sie das Problem exakt bei konstanten $k(x) = k_0$ für alle
  $x \in [0, 2\pi]$.

  \item[b)] Lösen Sie das gleiche Problem numerisch mit Hilfe des Finite-Differenzen
  Verfahrens und untersuchen Sie für unterschiedliche $h$ die Größe des Fehlers.

  \item[c)] Wie könnte man \eqref{one} interpretieren, wenn der Wärmeleitkoeffizient
  nur stückweise konstant ist, d.h. wenn z.B. $k(x) = k_1$ für $x \in [0, \pi)$ und
  $k(x) = k_2 \neq k_1$ für $x \in (\pi, 2\pi]$? Berechnen Sie wiederum die analytische
  Lösung und schlagen Sie ein geeignetes Diskretisiertungsverfahren vor.
  \end{itemize}
\end{exercise}

\begin{solution}
  \begin{enumerate}[label = \textbf{\alph*)}]
    \item Das Problem vereinfacht sich zu
    \begin{align*}
      -k_0u^{\primeprime}(x) = \cos(x), \quad x \in (0,2\pi).
    \end{align*}
  \end{enumerate}
\end{solution}
