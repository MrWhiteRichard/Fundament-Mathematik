\begin{exercise}
  Formulieren Sie, basierend auf dem Newton-Verfahren, einen Algorithmus für das
  Mehrzielverfahren (Ende Abschnitt $7.2$ des Vorlesungsskriptes). Orientieren
  Sie sich dabei an Alg. $7.4$ für das einfache Schießverfahren. Geben Sie auch
  die benötigte Jacobi Matrix an.
\end{exercise}

\begin{solution}
  Die zu lösende Funktion in unserem Fall lautet mit $Y = (y_1,\dots,y_{N-1}),
  S := (s_0,\dots,s_{N-1})$
  \begin{align*}
    G(Y,S) &= (y_{x_1,y_1,s_1}(x_2),\dots,y_{x_{N-1},y_{N-1},s_{N-1}}(x_N),
    y^{\prime}_{x_1,y_1,s_1}(x_2),\dots,y^{\prime}_{x_{N-1},y_{N-1},s_{N-1}}(x_N))^{\top} \\
    &- (y_1,\dots,y_N,s_1,\dots,s_{N-1})^{\top}
    \stackrel{!}{=} 0.
  \end{align*}
  Die Jacobi-Matrix davon lautet
  \begin{align*}
    DG(S,Y) = \begin{pmatrix}
      \partial_{y_1}y_{x_1,y_1,s_1}(x_2) & 0 & \hdots & 0 & & & & \\
      0 & \ddots & 0  & \vdots & & & & \\
      \vdots & 0 & \ddots & 0 & & & &\\
      0 & \hdots & 0 & \partial_{y_{N-1}}y_{x_{N-1},y_{N-1},s_{N-1}}(x_{N}) & & & &\\
       & & & & \partial_{s_1}y^{\prime}_{x_1,y_1,s_1}(x_2) & 0 & \hdots & 0 \\
       & & & & 0 & \ddots & 0  & \vdots \\
       & & & & \vdots & 0 & \ddots & 0 \\
       & & & & 0 & \hdots & 0 & \partial_{s_N}y^{\prime}_{x_N,y_N,s_N}(x_{N+1}) \\
  \end{pmatrix}
  \end{align*}
  \begin{algorithm}[caption={Mehrzielverfahren.}, label={alg1}]
   Input: Problembeschreibung $(f,a,b,y_a,y_b)$, Mesh $T = [x_0,\dots,x_N]$, Startwert $s_0$.
   n := 0
   repeat
     Compute solution $y_{s_n}$ of $y^{\primeprime}(x) = f(x,y(x),y^{\prime}(x)), \quad
     y(a) = y_a, y^{\prime}(a) = s_n$.
     Compute solution $v_{s_n}$ of $v^{\primeprime}(x) = f_y(x,y_s(x),y_s^{\prime}(x))
     + f_{y^{\prime}}(x,y_s(x),y_s^{\prime}(x)), \quad v(a) = 0, v^{\prime}(a) = 1$.
   until Abbruchkriterium des Newton-Verfahren erreicht.
   n = n + 1
   Output: Approximation $y_{s_{n-1}}$
  \end{algorithm}
\end{solution}
