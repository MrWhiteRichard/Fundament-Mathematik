\begin{exercise}
  In der Vorlesung würde für hinreichen glatte Funktionen bewiesen, dass gilt

  \begin{align*}
    u^\primeprime (x) = \frac{1}{h^2}\left[
    \begin{array}{ccc}
      1 & -2 & 1
    \end{array}
    \right] u(x) +
    \Landau{h^2}
  \end{align*}


  mit dem Differenzenstern $[1 \text{ }  -2 \text{ } 1]u(x):= 1u(x-h)-2u(x)+1u(x+h)$.

  mit dem Differenzenstern $\left[
  \begin{array}{ccc}
    1 & -2 & 1
  \end{array}
  \right]u(x):= 1u(x-h)-2u(x)+1u(x+h)$.

  Konstruieren Sie eine Approximation der Form

  \begin{align*}
    u^\primeprime (x) = \frac{1}{h^2} \left[
    \begin{array}{ccccc}
    c_{-2} & c_{-1} & c_0 & c_1 & c_ 2
    \end{array}
    \right]
    u(x) + \Landau{h^4}
  \end{align*}

  mit geeigneten Konstanten $c_{-2}, \dots , c_2 \in \R$ und dem Differenzenstern
  \begin{align*}
    \left[
    \begin{array}{ccccc}
        c_{-2} & c_{-1} & c_0 & c_1 & c_ 2
    \end{array}
    \right]
    u(x)
    :=
    \sum_{i=-2}^2 c_i u(x+ih).
  \end{align*}
\end{exercise}

\begin{solution}
  Für $u \in C^6([a,b])$ liefert uns Taylorentwicklung um $x$

  \begin{align*}
    u(x+h) &= u(x) &+ hu'(x) &+ \frac{h^2}{2}u''(x) &+ \frac{h^3}{6}u^{(3)}(x) &+ \frac{h^4}{24}u^{(4)}(x) &+ \frac{h^5}{120}u^{(5)}(x) &+ \Landau{h^6}\\
    u(x+2h) &= u(x) &+ 2hu'(x) &+ \frac{4h^2}{2}u''(x) &+ \frac{8h^3}{6}u^{(3)}(x) &+ \frac{16h^4}{24}u^{(4)}(x) &+ \frac{32h^5}{120}u^{(5)}(x) &+ \Landau{h^6}\\
    u(x-h) &= u(x) &- hu'(x) &+ \frac{h^2}{2}u''(x) &- \frac{h^3}{6}u^{(3)}(x) &+ \frac{h^4}{24}u^{(4)}(x) &- \frac{h^5}{120}u^{(5)}(x) &+ \Landau{h^6}\\
    u(x-2h) &= u(x) &- 2hu'(x) &+ \frac{4h^2}{2}u''(x) &- \frac{8h^3}{6}u^{(3)}(x) &+ \frac{16h^4}{24}u^{(4)}(x) &- \frac{32h^5}{120}u^{(5)}(x) &+ \Landau{h^6}
  \end{align*}

  Da wir wollen, dass alle Terme, bis auf die zweite Ableitung verschwinden, kommen wir auf folgendes lineares Gleichungssystem:

  \begin{align*}
    \begin{pmatrix}
      1 & 1 & 1 & 1 & 1 \\
      1 & 2 & -1 & -2 & 0 \\
      \frac{1}{6} & \frac{8}{6} & \frac{-1}{6} & \frac{-8}{6} & 0 \\
      \frac{1}{24} & \frac{16}{24} & \frac{1}{24} & \frac{16}{24} & 0 \\
      \frac{1}{120} & \frac{32}{120} & \frac{-1}{120} & \frac{-32}{120} & 0 \\
    \end{pmatrix}
    \begin{pmatrix}
      c_1 \\
      c_2 \\
      c_{-1} \\
      c_{-2} \\
      c_{0}
    \end{pmatrix} =
    \begin{pmatrix}
      0 \\ 0 \\ 0 \\ 0 \\ 0
    \end{pmatrix}
  \end{align*}

  Eine Lösung dieses Systems ist gegben durch
  \begin{align*}
  \begin{pmatrix}
    c_1 \\
    c_2 \\
    c_{-1} \\
    c_{-2} \\
    c_{0}
  \end{pmatrix} =
  \begin{pmatrix}
    16 \\ -1 \\ 16 \\ -1 \\ -30
  \end{pmatrix}
  \end{align*}

  und es folgt

  \begin{align*}
  \left[
  \begin{array}{ccccc}
  c_{-2} & c_{-1} & c_0 & c_1 & c_ 2
  \end{array}
  \right]
  u(x) = \frac{h^2}{2}u''(x)(16 - 1\cdot 4 + 16 - 1 \cdot 4) + \Landau{h^6} = 12 h^2 u''(x) + \Landau{h^6} \\
  \Rightarrow u''(x) = \frac{1}{h^2} \frac{1}{12} \left[
  \begin{array}{ccccc}
  c_{-2} & c_{-1} & c_0 & c_1 & c_ 2
  \end{array}
  \right] u(x) + \Landau{h^4}
  \end{align*}

\end{solution}
