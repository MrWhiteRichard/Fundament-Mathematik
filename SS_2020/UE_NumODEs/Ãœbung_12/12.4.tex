\begin{exercise}
  In der Vorlesung würde für hinreichen glatte Funktionen bewiesen, dass gilt

  \begin{align*}
    u^\primeprime (x) = \frac{1}{h^2}\left[
    \begin{array}{ccc}
      1 & -2 & 1
    \end{array}
    \right] u(x) +
    \Landau{h^2}
  \end{align*}

  mit dem Differenzenstern $[1   -2   1]u(x):= 1u(x-h)-2u(x)+1u(x+h)$.
  Konstruieren Sie eine Approximation der Form

  \begin{align*}
    u^\primeprime (x) = \frac{1}{h^2} \left[
    \begin{array}{ccccc}
    c_{-2} & c_{-1} & c_0 & c_1 & c_ 2
    \end{array}
    \right]
    u(x) + \Landau{h^4}
  \end{align*}

  mit geeigneten Konstanten $c_{-2}, \dots , c_2 \in \R$ und dem Differenzenstern
  \begin{align*}
    \left[
    \begin{array}{ccccc}
        c_{-2} & c_{-1} & c_0 & c_1 & c_ 2
    \end{array}
    \right]
    u(x)
    :=
    \sum_{i=-1}^2 c_i u(x+ih).
  \end{align*}
\end{exercise}

\begin{solution}
  Beweis.
\end{solution}
