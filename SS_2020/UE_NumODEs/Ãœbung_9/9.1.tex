\begin{exercise}
Beweisen Sie, dass die $k$-Schritt BDF-Verfahren aus Aufgabe 37 für $k = 1,\dots,6$
die Wurzelbedingung erfüllen und dass die Verfahren für $k = 7,\dots,10$
die Wurzelbedingung nicht erfüllen. \\
\textit{Hinweis:} Die Koeffizienten der $k$-Schritt BDF-Verfahren können Sie mit
einem Computeralgebrasystem ausrechnen. Die Verfahren haben die Form
\begin{align}
  \sum_{j= 0}^k \alpha_{k - j}y_{\ell + 1 - j} = hf_{\ell + 1}, \qquad
  \alpha_{k - j} = hL_j^{\prime}(t_{\ell + 1}), \qquad
  L_j(t) := \prod_{\stackrel{m = 0}{m \neq j}}^k \frac{t - t_{\ell + 1 -m}}{t_{\ell + 1 - j} - t_{\ell + 1 - m}}
\end{align}
\textit{Zusatzinformation}: Man kann zeigen, dass alle BDF-Verfahren für $k \geq 7$
nicht die Wurzelbedingung erfüllen.
\end{exercise}
\begin{solution}
Für $k = 1,\dots,3$ haben wir die Koeffizienten bereits letzte Woche berechnet.
Also lauten die ersten drei Verfahren
\begin{align*}
  k = 1:& y_{\ell+1} = hf(t_{\ell + 1},y_{\ell + 1}) + y_{\ell} \\
  k = 2:& \frac{3}{2}y_{\ell+1} = hf(t_{\ell + 1},y_{\ell + 1}) + 2y_{\ell} - \frac{1}{2}y_{\ell - 1} \\
  k = 3:& \frac{11}{6}y_{\ell + 1} = hf(t_{\ell + 1},y_{\ell + 1}) + 3y_{\ell} -
  \frac{3}{2}y_{\ell - 1} + \frac{1}{3}y_{\ell - 2}.
\end{align*}
\end{solution}
