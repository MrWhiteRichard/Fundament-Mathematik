\begin{exercise}

Konstruieren Sie alle gegen Autonomisierung invarianten $3$-stufigen Runge-Kutta-Verfahren maximaler Ordnung, welche die Gewichte (nicht die Stützstellen) der Simpson-Regel verwenden.

\end{exercise}

\begin{solution}

Wir suchen also genau alle expliziten Runge-Kutta-Verfahren mit $b = (\frac{1}{6},\frac{2}{3},\frac{1}{6})$, die (4),(5a),(5b),(5c) und (5d) erfüllen.

Die Gleichung (5a) ist durch die Vorgabe von $b$ bereits erfüllt, aus (4) können wir direkt $c_{1}=0$ ablesen.

Dies in (5b) und (5c) eingesetzt ergibt mit

\begin{align*}
    \frac{2}{3}c_{2} + \frac{1}{6}c_{3} = \frac{1}{2} \\
    \frac{2}{3}c_{2}^{2} + \frac{1}{6}c_{3}^{2} = \frac{1}{3}
\end{align*}

die zwei Fälle $c_{2}=\frac{7}{10}, c_{3}=\frac{1}{5}$ und $c_{2}=\frac{1}{2}, c_{3}=1$.
Da mit $c_{1}=0$ die Gleichung (5d)
\begin{align*}
    \frac{1}{6}a_{32}c_{2}=\frac{1}{6}
\end{align*}
lautet, sind in beiden Fällen $a_{32}=c_{2}^{-1}$ und durch Gleichung (4) auch $a_{21}=c_{2}$ und $a_{31}=c_{3}-a_{32}$ eindeutig bestimmt.

Für die beiden Fälle ergeben sich also die Runge-Kutta-Verfahren mit
\begin{align*}
    A = \left( \begin{array}{rrr}
         0 & 0 & 0 \\
        \frac{7}{10} & 0 & 0 \\
        -\frac{43}{35} & \frac{10}{7} & 0
    \end{array} \right) , c = \begin{pmatrix}
    0 \\ \frac{7}{10} \\ \frac{1}{5}
    \end{pmatrix}
\end{align*}
und
\begin{align*}
    A = \left( \begin{array}{rrr}
         0 & 0 & 0 \\
        \frac{1}{2} & 0 & 0 \\
        -1 & 2 & 0
    \end{array} \right) , c = \begin{pmatrix}
    0 \\ \frac{1}{2} \\ 1
    \end{pmatrix}
\end{align*}
(bereits bekannt aus Example 2.26).
\end{solution}
