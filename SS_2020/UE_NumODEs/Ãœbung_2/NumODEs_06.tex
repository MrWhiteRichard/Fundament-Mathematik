\begin{exercise}

Sei $y: [0, T] \to \R^n$ Lösung des Anfangswertproblems $y^\prime = f(y)$ mit $y(0) = y_0$.
Programmieren Sie einen allgemeinen Löser für solche Probleme.
Eingabedaten sollten die Funktion $f$, eine Diskretisierung des Intervalls $[0, T]$, der Startwert $y_0$ und das Butcher-Schema eines beliebigen, expliziten Runge-Kutta-Verfahrens sein. Ausgabedaten wären die Approximationen $y_i$ an $y(t_i)$ aus dem zugehörigen Runge-Kutta-Verfahren.
Verwenden Sie Ihr Programm, um folgendes Räuber-Beute-Modell numerisch zu lösen. \\

Modell:
Sei $y_\mathrm{J}(t)$ zu jedem Zeitpunkt $t$ die Größe einer Jäger-Population und $y_\mathrm{B}(t)$ die Größe einer Beutepopulation.
Die Wachstumsrate der Populationen ergibt sich aus der Differenz der Geburtenrate und der Sterberate.
Dabei nehmen wir an, dass für die Beutepopulation genügend Nahrung vorhanden sei, sodass die Geburtenrate $\alpha > 0$ konstant ist.
Bei jedem Zusammentreffen zwischen Jäger und Beute wird mit einer ebenfalls konstanten Rate $\beta > 0$ die Beute gefressen.
Die natürliche Sterberate der Beute sei im Vergleich dazu vernachlässigbar.
Dann ergibt sich für die Beute-Population die Differentialgleichung

\begin{align}
  \label{Beute-Population}
  y_\mathrm{B}^\prime(t)
  =
  \alpha y_\mathrm{B}(t) - \gamma y_\mathrm{J}(t) y_\mathrm{B}(t).
  \tag{1a}
\end{align}

Für die Jäger-Population nehmen wir an, dass diese proportional mit Faktor $\gamma > 0$ zur Anzahl der Begegnungen mit der Beute wächst.
Die natürliche Sterberate der Jäger $\delta > 0$ ist hier nicht vernachlässigbar.
Insgesamt ergibt sich damit für die Jäger-Population

\begin{align}
  \label{Jäger-Population}
  y_\mathrm{J}^\prime(t)
  =
  \gamma y_\mathrm{J}(t) y_\mathrm{B}(t) - \delta y_\mathrm{J}(t).
  \tag{1b}
\end{align}

Zum Testen können Sie z.B. die Modellparameter $\alpha = 2$, $\beta = \gamma = 0.01$, $\delta = 1$, $y_\mathrm{J}(0) = 150$ und $y_\mathrm{B}(0) = 300$ verwenden.
Zur Diskretisierung könnten Sie mit einer äquidistanten Zerlegung von $t \in [0, 100]$ mit $h = 0.01$ und dem den expliziten Runge-Kutta-Verfahren aus der Vorlesung (Example 2.23 und Example 2.25) starten.
Variieren Sie aber Modellparameter und Diskretisierung
und studieren Sie die Effekte auf die numerischen Ergebnisse.

\end{exercise}

\begin{solution}

Trivial!

\end{solution}
