\begin{exercise}
Untersuchen Sie, welche der folgenden Abbildungen Normen auf dem Raum der Polynome
$\Pi_r$ vom maximalen Grad $r$ sind. Sei dazu $p \in \Pi_r$.
\begin{enumerate}[label = \textbf{\alph*)}]
  \item \begin{align*}
    ||p||_k := \sup_{x \in [a,b]} |p(x)| + \sup_{x \in [a,b]}|p^{(k)}(x)|
    \text{ für } k \in \N
      \end{align*}
  \item \begin{align*}
    ||p|| := \sup_{j = 0,\dots,r} |p(x_j)| \text { mit paarweise verschiedenen }
    x_j \in [a,b] \text{ für } j = 0,\dots,r.
    \end{align*}
  \item \begin{align*}
    ||p||_k := |p(x_0)| + \sup_{j = 1,\dots,r}|p^{(k)}(x_j)| \text{ für } k \in \N,
    x_0 \in [a,b] \text{ und paarweise verschiedenen } x_j \in [a,b] \text{ für }
    j = 1,\dots,r.
    \end{align*}
  \item \begin{align*}
    &||p||_k := |p(x_0)| + |p(x_1)| + \sup_{j = 2,\dots,r} |p^{(k)}(x_j)| \text{ für }
    k \in \N, x_0 \neq x_1 \in [a,b] \\
     &\text{ und paarweise verschiedenen } x_j \in [a,b] \text{ für } j = 2,\dots,r.
  \end{align*}
\end{enumerate}
\textit{Hinweis zum Fall k = 1 bei d):} Zeigen Sie, dass diese Abbildung genau
dann eine Norm ist, wenn
\begin{align*}
  \int_{x_0}^{x_1} \prod_{j=2}^r \left(\xi - x_j\right)d\xi \neq 0
\end{align*}
\end{exercise}
\begin{solution}
Wir wissen
\begin{align}\label{Pol}
  \Forall p \in \Pi_r :(\text{p hat mindestens $r+1$ paarweise verschiedene Nullstellen} \Rightarrow p = 0)
\end{align}
\begin{enumerate}[label = \textbf{\alph*)}]
  \item
  \begin{enumerate}[label = \textit{\roman*)}]
    \item \Quote{Skalare raus} \begin{align*}
      & \Forall p \in \Pi_r, \Forall \lambda \in \C: \\
      & \|\lambda p\|_k = \|\lambda p\|_\infty + \|\lambda p^{(k)}\|_\infty = |\lambda| \|p\|_\infty + |\lambda| \|p^{k}\|_\infty = |\lambda| \|p\|_k
    \end{align*}
    \item Dreiecksungleichung \begin{align*}
      &\Forall p,q \in \Pi_r: \\
      & \|p+q\|_k = \|p+q\|_\infty + \|(p+q)^{(k)}\|_\infty \\
      &\leq \|p\|_\infty + \|q\|_\infty + \|p^{(k)}\|_\infty + \|q^{(k)}\|_\infty = \|p\|_k + \|q\|_k
    \end{align*}
    \item \Quote{Null-Äquivalenz} \begin{align*}
      &\Forall p \in \Pi_r: \\
      & \|p\|_k = 0 \Rightarrow \|p\|_\infty = 0 \Rightarrow p = 0
    \end{align*}
    Umkehrung trivial.
  \end{enumerate}
  \item
  \begin{enumerate}[label = \textit{\roman*)}]
    \item \Quote{Skalare raus} $\Rightarrow \checkmark$
    \item Dreiecksungleichung $\Rightarrow \checkmark$
    \item \Quote{Null-Äquivalenz} \begin{align*}
      &\Forall p \in \Pi_r: \\
      & \|p\|_k = 0 \Rightarrow \Forall j=0,...,r: p(x_j) = 0 \stackrel{\eqref{Pol}}{\Rightarrow} p = 0
    \end{align*}
    Umkehrung trivial.
  \end{enumerate}
  \item
  \begin{enumerate}[label = \textit{\roman*)}]
    \item \Quote{Skalare raus} $\Rightarrow \checkmark$
    \item Dreiecksungleichung $\Rightarrow \checkmark$
    \item \Quote{Null-Äquivalenz} \begin{align*}
      &\Forall p \in \Pi_r: \\
      & \|p\|_k = 0 \Rightarrow p(x_0) = 0 \land \Forall j=1,...,r: p^{(k)}(x_j) = 0
    \end{align*}
    \begin{enumerate}[label = \textit{\arabic*.}]
      \item Fall ($k=1$):
      Laut Aufgabe 22 ist die Lösung dieses Problems eindeutig und offensichtlich ist $p = 0$ eine Lösung.
      Umkehrung trivial.
      \item Fall ($k>1$):
      Definiere das Polynom $p_0 \in \Pi_r$ (o.B.d.A $r \geq 1$) durch
      \begin{align*}
        p_0(x) := x-x_0
      \end{align*}
      Nun gilt $p_0^{(k)}(x) = 0$, insbesondere $p_0^{(k)}(x_j) = 0, j=1,...,r$. Außerdem ist durch Definition auch $p_0(x_0) = 0$.
      Somit gilt $\|p_0\|_k = 0$ aber $p_0 \neq 0$. Also ist $\|\cdot\|_k$ für $k>1$ keine Norm.
    \end{enumerate}
  \end{enumerate}
  \item
  \begin{enumerate}[label = \textit{\roman*)}]
    \item \Quote{Skalare raus} $\Rightarrow \checkmark$
    \item Dreiecksungleichung $\Rightarrow \checkmark$
    \item \Quote{Null-Äquivalenz} \begin{align*}
      &\Forall p \in \Pi_r: \\
      & \|p\|_k = 0 \Rightarrow p(x_0) = p(x_1) = 0 \land \Forall j=2,...,r: p^{(k)}(x_j) = 0
    \end{align*}
    \begin{enumerate}[label = \textit{\arabic*.}]
      \item Fall ($k=1$):
      Wir zeigen zunächst die Rückrichtung des Hinweises, nehmen also an, dass
      \begin{align*}
        \int_{x_0}^{x_1} \prod_{j=2}^r \left(\xi - x_j\right)d\xi \neq 0.
      \end{align*}
      Wir wählen ein belibiges $x_{r+1}$ verschieden von den bisherigen Stützstellen. Der Einfachheit halber definieren wir
      \begin{align*}
        z_j = x_j+1, j = 1,...,r
      \end{align*}
      Sei nun $p \in \Pi_r$ mit $\|p\|_k = 0 \Rightarrow p(x_0) = p(x_1) = 0 \land \Forall j=1,...,r-1: p^{\prime}(z_j) = 0$.
      Also gilt
      \begin{align}\label{int}
        0 = p(x_1) - p(x_0) = \int_{x_0}^{x_1}p^{\prime}(\xi) d\xi
      \end{align}
      Wir verwenden nun die Newton Darstellung des Interpolationspolynoms über die dividierten Differenzen, wobei für $j=1,...,r-1$ gilt $y[z_j] = p^{\prime}(x_{j+1}) = 0$ und somit aufgrund der rekursiven Definition der d. D. auch $y[z_1,...,z_j]=0$ und $y[z_1,...,z_r] = \frac{1}{C}p^{\prime}(z_r)$
      Eingesetzt in \eqref{int}:
      \begin{align*}
        0 =& \int_{x_0}^{x_1}p^{\prime}(\xi) d\xi = \int_{x_0}^{x_1} \sum_{j=1}^{r}y[z_1,...,z_j]N_{j-1}(\xi) d\xi \\
        =& y[z_1,...,z_r]\int_{x_0}^{x_1}N_{r-1}(\xi)d\xi = \frac{1}{C}p^{\prime}(z_r)\underbrace{\int_{x_0}^{x_1}\prod_{k=1}^{r-1}(\xi-z_j)d\xi}_{\neq 0}
      \end{align*}
      Es muss also $p^{\prime}(z_r) = 0$, somit wieder nach \eqref{Pol} $p^{\prime} = 0$ und $p = 0$ konstant, da $p(x_0) = 0$. Also ist in diesem Fall $\|\cdot\|_k$ eine Norm.

      Für die umgekehrte Richtung, nehmen wir an, dass
      \begin{align*}
        \int_{x_0}^{x_1} \prod_{j=2}^r \left(\xi - x_j\right)d\xi = 0.
      \end{align*}
      Wir definieren
      \begin{align*}
        p_0(x) := \int_{x_0}^{x} \prod_{j=2}^r \left(\xi - x_j\right)d\xi
      \end{align*}
      Man überprüft leicht, dass $\|p_0\|_k = 0$, allerdings ist $p_0 \neq 0$. In diesem Fall ist $\|\cdot \|_k$ also keine Norm.

      \item Fall ($k=2$):
      In diesem Fall gilt $p^{(k)} \in \Pi_{r-2}$, und $p^{(k)}$ besitzt $r-1$ Nullstellen $\substack{\Rightarrow}{\eqref{Pol}} p^{(k)} = 0$, also ist $p$ linear und an $x_0$ und $x_1$ gleich Null. Also gilt $p=0$.

      \item Fall ($k>2$):
      (Fast) analog zu \textbf{c)} definiere das Polynom $p_0 \in \Pi_r$ (o.B.d.A $r \geq 2$) durch
      \begin{align*}
        p_0(x) := (x-x_0)(x-x_1)
      \end{align*}
      Nun gilt $p_0^{(k)}(x) = 0$, insbesondere $p_0^{(k)}(x_j) = 0, j=2,...,r$. Außerdem ist durch Definition auch $p_0(x_0) = p_0(x_1) = 0$.
      Somit gilt $\|p_0\|_k = 0$ aber $p_0 \neq 0$. Also ist $\|\cdot\|_k$ für $k>2$ keine Norm.
    \end{enumerate}
  \end{enumerate}
\end{enumerate}
\end{solution}
