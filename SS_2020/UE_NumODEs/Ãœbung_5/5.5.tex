\begin{exercise}
Sei $M \in \R^{n \times n}$ mit $n \in \N$ eine diagonalisierbare Matrix und sei
$y: [0,T] \rightarrow \R^n$ Lösung des Anfangswertproblems
\begin{align*}
  y^{\prime} = My, \qquad y(0) = y_0 \in \R^n.
\end{align*}
Zeigen Sie, dass ein Runge-Kutta-Verfahren angewendet auf dieses Anfangswertproblem
exakt dem gleichen Runge-Kutta-Verfahren angewendet auf die separierten Anfangswertprobleme
\begin{align*}
  \widetilde{y}_j: [0,T] \rightarrow \C: \qquad \widetilde{y}_j = \lambda_j\widetilde{y}_j,
  \qquad \widetilde{y}_j(0) = \widetilde{y}_{j,0} \in \C, \qquad j = 1,\dots,n,
\end{align*}
entspricht (in welchem Sinne?), wobei $\lambda_j$ für $j = 1,\dots,n$ die Eigenwerte
der Matrix $M$ gemäß ihrer Vielfachheit sind. Wie müssen die Anfangsdaten $\widetilde{y}_{j,0}$
gewählt werden?
\end{exercise}
\begin{solution}
Beweis.
\end{solution}
