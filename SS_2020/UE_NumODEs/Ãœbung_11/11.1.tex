\begin{exercise}
  Beweisen Sie, dass die symplektischen Euler-Verfahren die Konvergenzordnung 1
  haben. Konstruieren Sie dazu auch ein Beispiel, welches zeigt, dass sie keine
  höhere Ordnung haben.
\end{exercise}

\begin{solution}
\leavevmode \\
Symplektischer Euler 1:
\begin{align*}
  \begin{pmatrix}
    p_{\ell + 1} \\ q_{\ell + 1}
  \end{pmatrix}
  =
  \begin{pmatrix}
    p_{\ell} + hf_1(p_{\ell},q_{\ell + 1}) \\
    q_{\ell} + hf_2(p_{\ell},q_{\ell + 1})
  \end{pmatrix}.
\end{align*}
Definiere die Funktion
\begin{align*}
  F(t + x) := \begin{pmatrix}
    p(t+x) \\ q(t+h-x)
  \end{pmatrix}
\end{align*}
Taylor:
\begin{align*}
  \begin{pmatrix}
    p(t+h) \\ q(t)
  \end{pmatrix} =
  F(t+h) = F(t) + hF^{\prime}(t) + \Landau{h^2}
  = \begin{pmatrix}
    p(t) \\ q(t+h)
  \end{pmatrix}
  + h
  \begin{pmatrix}
    f_1(p(t),q(t+h)) \\ -f_2(p(t),q(t+h))
  \end{pmatrix}
  + \Landau{h^2}
\end{align*}
Symplektischer Euler 2:
\begin{align*}
  \begin{pmatrix}
    p_{\ell + 1} \\ q_{\ell + 1}
  \end{pmatrix}
  =
  \begin{pmatrix}
    p_{\ell} + hf_1(p_{\ell + 1},q_{\ell}) \\
    q_{\ell} + hf_2(p_{\ell + 1},q_{\ell})
  \end{pmatrix}.
\end{align*}
Taylor:
\begin{align*}
\begin{pmatrix}
  p(t) \\ q(t+h)
\end{pmatrix} =
F(t) = F(t + h) - hF^{\prime}(t + h) + \Landau{h^2}
= \begin{pmatrix}
  p(t+h) \\ q(t)
\end{pmatrix}
- h
\begin{pmatrix}
  f_1(p(t+h),q(t)) \\ -f_2(p(t+h),q(t))
\end{pmatrix}
+ \Landau{h^2}
\end{align*}
Gegenbeispiel: Probieren wir mal was ganz einfaches:
\begin{align*}
  \begin{pmatrix}
    p \\ q
  \end{pmatrix}^{\prime}
  = \begin{pmatrix}
    p \\ q
  \end{pmatrix}, \quad
  \begin{pmatrix}
    p(0) \\ q(0)
  \end{pmatrix} =
  \begin{pmatrix}
    1 \\ 1
  \end{pmatrix}
\end{align*}
mit der analytischen Lösung $p(t) = q(t) = \exp(t)$.
Der erste symplektische Euler 1 liefert für $h > 0$
\begin{align*}
  \begin{pmatrix}
    p_1 \\ q_1
  \end{pmatrix}
  = \begin{pmatrix}
    1+h \\ \frac{1}{1-h}
  \end{pmatrix} \\
  \|p(h) - p_1\| = \|\exp(h) - (1+h)\| = \|\sum_{n = 0}^{\infty}\frac{h^n}{n!} - (1+h)\|
  = \|\sum_{n = 2}^{\infty}\frac{h^n}{n!}\| \geq \|\frac{h^2}{2}\|.
\end{align*}
Damit kann das Verfahren nicht Konsistenzordnung $2$ haben.
\end{solution}
