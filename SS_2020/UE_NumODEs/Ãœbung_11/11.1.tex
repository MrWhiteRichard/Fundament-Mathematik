\begin{exercise}
  Beweisen Sie, dass die symplektischen Euler-Verfahren die Konvergenzordnung 1
  haben. Konstruieren Sie dazu auch ein Beispiel, welches zeigt, dass sie keine
  höhere Ordnung haben.
\end{exercise}

\begin{solution}
\leavevmode \\
Symplektischer Euler 1:
\begin{align*}
  \begin{pmatrix}
    p_{\ell + 1} \\ q_{\ell + 1}
  \end{pmatrix}
  =
  \begin{pmatrix}
    p_{\ell} + hf_1(p_{\ell},q_{\ell + 1}) \\
    q_{\ell} + hf_2(p_{\ell},q_{\ell + 1})
  \end{pmatrix}.
\end{align*}
Symplektischer Euler 2:
\begin{align*}
  \begin{pmatrix}
    p_{\ell + 1} \\ q_{\ell + 1}
  \end{pmatrix}
  =
  \begin{pmatrix}
    p_{\ell} + hf_1(p_{\ell + 1},q_{\ell}) \\
    q_{\ell} + hf_2(p_{\ell + 1},q_{\ell})
  \end{pmatrix}.
\end{align*}
Wir zeigen exemplarisch die Konvergenzordnung vom ersten symplektischen
Euler, im zweiten Fall geht man komplett analog vor.
Wir entwickeln nach Taylor mit Anschlussstelle $t$ und erhalten
\begin{align*}
  y(t+h) = y(t) + hy^{\prime}(t) + \Landau{h^2} = y(t) + hf(y(t)) + \Landau{h^2} =
  y(t) + hf(p(t),q(t)) + \Landau{h^2}.
\end{align*}
Es folgt
\begin{align*}
  \tau(t,y,h) = \|y(t+h) - y(t) - h\Phi(t,y(t),h)\| = \|y(t+h) - y(t) - hf(p(t),q(t+h))\|
  = h\|f(p(t),q(t)) - f(p(t),q(t+h))\|.
\end{align*}
Jetzt verwenden wir die Lipschitzstetigkeit von $f$ und $y$:
\begin{align*}
  \tau(t,y,h) = h\|f(p(t),q(t)) - f(p(t),q(t+h))\| \leq hL_1\|(0,q(t)-q(t+h))^{\top}\|
  \leq hL_1L2\|h\| = L_1L_2h^2
\end{align*}
Gegenbeispiel: Probieren wir mal was ganz einfaches:
\begin{align*}
  \begin{pmatrix}
    p \\ q
  \end{pmatrix}^{\prime}
  = \begin{pmatrix}
    p \\ q
  \end{pmatrix}, \quad
  \begin{pmatrix}
    p(0) \\ q(0)
  \end{pmatrix} =
  \begin{pmatrix}
    1 \\ 1
  \end{pmatrix}
\end{align*}
mit der analytischen Lösung $p(t) = q(t) = \exp(t)$.
Der erste symplektische Euler 1 liefert für $h > 0$
\begin{align*}
  \begin{pmatrix}
    p_1 \\ q_1
  \end{pmatrix}
  &= \begin{pmatrix}
    1+h \\ \frac{1}{1-h}
  \end{pmatrix} \\
  \|p(h) - p_1\| &= \|\exp(h) - (1+h)\| = \left\|\sum_{n = 0}^{\infty}\frac{h^n}{n!} - (1+h)\right\|
  = \left\|\sum_{n = 2}^{\infty}\frac{h^n}{n!}\right\| \geq \left\|\frac{h^2}{2}\right\|.
\end{align*}
Damit existiert für alle $C > 0$ ein $h_0 > 0$, sodass für alle $h < h_0: \|p(h)-p_1\| \geq Ch^3$.
und das Verfahren kann nicht Konsistenzordnung $2$ haben.
\end{solution}
