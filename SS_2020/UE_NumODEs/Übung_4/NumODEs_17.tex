\begin{exercise}

Implementieren Sie das implizite Euler Verfahren unter Verwendung des Newton-Verfahrens zur Lösung des nichtlinearen Gleichungssystems. Der Algorithmus soll als Input-Parameter einen Vektor von Stützstellen $t$, einen Startwert $y_0$, die rechte Seite und Ableitung der rechten Seite $f$ bzw. $\frac{\partial}{\partial t} f$ sowie eine geeignete Abbruchbedinung für das Newton Verfahren (Toleranz und/oder maximale Anzahl an Interationen) akzeptieren.

Testen Sie das Verfahren an folgenden Anfangswertproblemen: Sei $Y = (y_1, y_2)^T$ die Lösung des Anfangswertproblems

\begin{align} \tag{1}
    Y^\prime(t)
    =
    \begin{pmatrix}
        -2 &  1 \\
         1 & -2
    \end{pmatrix}
    Y(t)
    +
    \begin{pmatrix}
        2 \sin{t} \\
        2 (\cos{t} - \sin{t})
    \end{pmatrix},
    \quad
    t \geq 0,
    \quad
    Y(0)
    =
    \begin{pmatrix}
        2 \\
        3
    \end{pmatrix}.
\end{align}

Sei $Z = (z_1, z_2)^T$ die Lösung des Anfangswertproblems

\begin{align} \tag{2}
    Z^\prime(t)
    =
    \begin{pmatrix}
        -2   &  1 \\
         998 & -999
    \end{pmatrix}
    Z(t)
    +
    \begin{pmatrix}
        2 \sin{t} \\
        999 (\cos{t} - \sin{t})
    \end{pmatrix},
    \quad
    t \geq 0,
    \quad
    Z(0)
    =
    \begin{pmatrix}
        2 \\
        3
    \end{pmatrix}.
\end{align}

Vergleichen Sie dabei auch mit den Ergebnissen und Schrittweiten des eingebetteten Runge-Kutta Verfahren RK5(4) aus Aufgabe 15. Verwenden Sie dazu die Parameter $t \in [0, 10]$, $\rho = 0.7$, $\eta = 1.5$, $\text{tol} = 10^{-6}$, $h_{\text{min}} = 10^{-10}$.

\end{exercise}

\begin{solution}

Trivial!

\end{solution}
