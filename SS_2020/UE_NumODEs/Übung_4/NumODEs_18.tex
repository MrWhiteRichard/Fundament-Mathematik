\begin{exercise}
Gegeben sei ein implizites, $s$-stufiges Runge-Kutta-Verfahren der Form
\renewcommand{\arraystretch}{1.5}
\begin{align}
  \begin{matrix}
  c & \vline & A \\
  \hline
  0 & \vline & b^{\top}
  \end{matrix}.
\end{align}
Zeigen Sie: Angewendet auf das Anfangswertproblem $y^{\prime} = \lambda y$ mit
$y(0) = y_0$ und $\lambda \in \mathbb{C}$ gilt für hinreichend kleine $h$
\begin{align}
  y_{i+1} = R(\lambda h)y_i
\end{align}
mit einer rationalen Funktion $R = P/Q$ und Polynomen $P,Q \in \Pi_s$ vom
maximalen Grad $s$.
\end{exercise}
\begin{solution}
In diesem Fall ergibt sich also durch die Definition der Inkremente
\begin{align*}
    \forall j = 1,\dots,s: \text{~~} & k_{j} = \lambda y + \lambda h\sum_{l=1}^{s}A_{jl}k_{l} \\
    \Leftrightarrow \text{~~} & \lambda y =(1- \lambda hA_{jj})k_{j} -\sum_{l=1, l \neq j}^{s}\lambda hA_{jl}k_{l} .
\end{align*}
Diese Gleichungen lassen sich äquivalent umschreiben in
\begin{align*}
    (I-\lambda hA)\cdot k &= \lambda y \cdot \mathbbm{1} \\
    \Leftrightarrow \text{~~} k &= \lambda y \cdot (I-\lambda hA)^{-1} \mathbbm{1}
\end{align*}
wobei die Inverse für hinreichend kleine $h$ existiert.
Für das Ein-Schritt-Verfahren gilt also
\begin{align}
    y_{l+1} = y_{l} + h\cdot b^{T} k = y_{l}\underbrace{(1 + \lambda h \cdot b^{T}(I-\lambda hA)^{-1} \mathbbm{1})}_{=:R} .
\end{align}
Man sieht leicht, dass $R$ eine rationale Funktion der gewünschten Form ist, wenn $\tilde{R}(\lambda h) := b^{T}(I-\lambda hA)^{-1} \mathbbm{1})$ eine rationale Funktion ist mit $\tilde{R} = \tilde{P}/\tilde{Q}$ und $\tilde{P} \in \Pi_{s-1}, \tilde{Q} \in \Pi_{s}$.

Das Element an der Stelle $(l,k)$ der Inversen von $C := (I-\lambda hA)$ ist gegeben durch (s. Havlicek S.$203$)
\begin{align*}
    \frac{c_{lk}^{\#}}{\det C}.
\end{align*}

Dabei ist $\det C$ offensichtlich ein Polynom vom Grad $\leq s$, da die Einträge von $C$ maximal Polynome vom Grad $1$ sind (in $\lambda h$).

Ebenso ist $c_{lk}^{\#}$ als Determinante von $C_{lk}$ ein Polynom vom Grad $\leq s-1$.

Daraus folgt schließlich durch Ausführen der Matrix-Vektor-Multiplikation $\tilde{R} = \frac{\sum_{l=1}^{s}b_{l}\sum_{k=1}^{s}c_{lk}^{\#}}{\det C}$. Es ist also $\tilde{P} = \sum_{l=1}^{s}b_{l}\sum_{k=1}^{s}c_{lk}^{\#}$ und $\tilde{Q} = \det C$.
\end{solution}
