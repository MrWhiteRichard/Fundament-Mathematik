\begin{exercise}
Gegeben sei ein implizites, $s$-stufiges Runge-Kutta-Verfahren der Form
\renewcommand{\arraystretch}{1.5}
\begin{align}
  \begin{matrix}
  c & \vline & A \\
  \hline
  0 & \vline & b^{\top}
  \end{matrix}.
\end{align}
Zeigen Sie: Angewendet auf das Anfangswertproblem $y^{\prime} = \lambda y$ mit
$y(0) = y_0$ und $\lambda \in \mathbb{C}$ gilt für hinreichend kleine $h$
\begin{align}
  y_{i+1} = R(\lambda h)y_i
\end{align}
mit einer rationalen Funktion $R = P/Q$ und Polynomen $P,Q \in \Pi_s$ vom
maximalen Grad $s$.
\end{exercise}
\begin{solution}
Beweis.
\end{solution}
