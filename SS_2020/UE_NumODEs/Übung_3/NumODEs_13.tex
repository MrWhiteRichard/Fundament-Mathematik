\begin{exercise}
  Ein explizites Einschrittverfahren mit Inkrementfunktion $\Phi(t,y,h)$
  kann mit der sogenannten diskreten Evolution

  \begin{align}
    \Psi^{t,t+h}y := y + h\Phi(t,y,h)
  \end{align}
  formuliert werden durch
  \begin{align}
    y_{j+1}=\Psi^{t_j,t_j + h_j} y_j
  \end{align}

  Es heißt reversibel, wenn gilt $\Psi^{t+h,t}\Psi^{t,t+h}y = y$
  für alle zulässigen $(t,y)$ und alle hinreichend kleinen $h$. Zeigen Sie, dass
  es kein konsistentes, explizites Runge-Kutta-Verfahren gibt, welches für jedes
  beliebige Anfangswertproblem reversibel ist.

  Hinweis: Beweisen Sie zunächst, dass $\Psi^{0,h} y_0$ für ein $s$-stufiges,
  explizites Runge-Kutta-Verfahren ein Polynom der Ordnung $s$ in $h$ ist, wenn
  das Verfahren auf die Differentialgleichung

  \begin{align}
    y'(t)=y(t), y(0)=y_0,
  \end{align}

  angewendet wird.

  Zusatzinformation: Bei reversiblen Einschrittverfahren führt ein Schritt des
  Verfahrens mit positiver Schrittweite $h$ gefolgt von einem Schritt des Verfahrens
  mit negativer Schrittweite $-h$ wieder auf den Anfangswert.
\end{exercise}

\begin{solution}
  Trivial!
\end{solution}
