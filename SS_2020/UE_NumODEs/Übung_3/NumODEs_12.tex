\begin{exercise}
  Sei $y$ die Lösung des Anfangswertproblems
  $y'(t) = f(t,y)$ mit $t \in \bbraces{0,T}, \text{ } y(0)=y_0$
  und beliebig glattem $f$. Weiter seien $y_i$ für $i=0,\ldots,N$ die Approximationen
  an $y(t_i)$, welche durch ein Einschrittverfahren der Ordnung $p$ mit den
  Stützstellen $t_0 = 0,\ldots,t_N =T$ berechnet werden. Sei weiter $\tilde{y}$ der
  lineare Spline mit $\tilde{y}(t_i)=y_i$ für $i=0,\ldots,N$.

  Zeigen Sie, dass eine Konstante $C>0$ unabhängig von $h_i$ und $N$ existiert mit

  \begin{align}
    \sup_{t \in \bbraces{0,T}} \vbraces{\tilde{y}(t)-y(t)}
    \leq
    C \max\Bbraces{h_0,\ldots,h_{N-1}}
  \end{align}
\end{exercise}

\begin{solution}
  Wir definieren $h_{\triangle}:=\text{max}\{h_{0},...h_{N-1}\}$.
  Sei \=y der lineare Interpolationsspline an die exakte Lösung y mit den gleichen Interpolationsknoten $t_{0},...,t_{N}$ wie bei $\tilde{y}$.
  Aus Numerik A, Beispiel 3.25 wissen wir, dass der Fehler des Splines abgeschätzt werden kann durch
  \begin{align*}
    |y(t)-\=y(t)| \leq \frac{h_{\triangle}^{2}}{8}\|y''\|_{\infty} = \frac{h_{\triangle}^{2}}{8}\|f'\|_{\infty}
  \end{align*}

  Außerdem existiert wegen der Konvergenzordnung $p$ des Verfahrens eine Konstante $\tilde{C}$, sodass
  \begin{align}\label{conv}
    \underset{i=1,...,N}{\max} |y(t_{i})-y_{i}| \leq \tilde{C}h_{\triangle}^{p}
  \end{align}

  Da die beiden Splines $\=y, \tilde{y}$ die gleichen Knoten haben, und zwei Geraden auf dem gleichen Intervall (hier $[t_{i},t_{i+1}]$) ihren maximalen Abstand jeweils an einem der beiden Randpunkte annehmen, gilt

  \begin{align*}
    |\tilde{y}(t)-\=y(t)| \leq& \underset{i=0,...,N-1}{\max} \underset{t \in [t_{i},t_{i+1}]}{\sup} |\tilde{y}(t)-\=y(t)| \\
    \leq& \underset{i=0,...,N-1}{\max} \max \{|\tilde{y}(t_{i})-\=y(t_{i})|, |\tilde{y}(t_{i+1})-\=y(t_{i+1})| \} \\
    \leq& \underset{i=1,...,N}{\max}|\tilde{y}(t_{i})-\=y(t_{i})| = \underset{i=1,...,N}{\max}|y_{i}- y(t_{i})| \\
    \stackrel{\eqref{conv}}{\leq}& \tilde{C}h_{\triangle}^{p}
  \end{align*}

  Insgesamt ergibt sich also für alle $t \in [0,T]$

  \begin{align*}
    |y(t)-\tilde{y}(t)| \leq& |y(t)-\=y(t)| + |\=y(t)-\tilde{y}(t)| \\
    \leq& \frac{h_{\triangle}^{2}}{8}\|f'\|_{\infty} + \tilde{C}h_{\triangle}^{p} \leq \underbrace{(\frac{T}{8}\|f'\|_{\infty} + \tilde{C}T^{p-1})}_{=:C} h_{\triangle}
  \end{align*}


\end{solution}
