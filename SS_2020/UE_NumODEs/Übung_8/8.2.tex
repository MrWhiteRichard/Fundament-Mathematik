\begin{exercise}
Zur Konstruktion von linearen $k$-Schritt Verfahren für die Differentialgleichung
$y^{\prime}(t) = f(t,y(t))$ kann man folgendermaßen vorgehen: Sei $\{t_0,\dots,t_N\}$
ein uniformes Gitter mit Gitterweite $h$. Gegeben seien die Werte $y_{\ell},\dots,y_{\ell + 1 - k}$.
Zur Berechnung von $y_{\ell + 1}$ sei $p{\ell} \in \Pi_k$ das Polynom, welches
die Bedingungen
\begin{subequations}
\begin{align}
  p_{\ell}(t_{\ell + 1- j}) = y_{\ell + 1 - j}, \qquad j = 0,\dots,k,
\end{align}
und
\begin{align}
  p^{\prime}(t_{\ell + 1}) = f(t_{\ell + 1},y_{\ell + 1})
\end{align}
\end{subequations}
erfüllt. Dann definieren wir $y_{\ell + 1} := p_{\ell}(t_{\ell + 1})$.
\begin{enumerate}[label = \textbf{\alph*)}]
  \item Geben Sie die Konstanten in Gleichung (5.2) des Vorlesungsskriptes für
  dieses Verfahren abstrakt an und zeigen Sie, dass diese unabhängig von $h$ sind.
  Unter welchen Voraussetzungen ist das Verfahren wohldefiniert?
  \item Geben Sie die Koeffizienten der Verfahren für $k = 1,2,3$ explizit an.
  \item Beweisen Sie, dass
  \begin{align}
    \tilde{\eta}_{\ell}(p,h) := \sum_{j= 0}^kh\beta_jp^{\prime}(t_{\ell + 1 -k} + jh)
    - \alpha_jp(t_{\ell + 1 - k} + jh)
  \end{align}
  bei diesen Verfahren für alle $p \in \Pi_k$ verschwindet. Welche Konsistenzordnung
  haben diese Verfahren?
\end{enumerate}
\end{exercise}
\begin{solution}
Beweis.
\end{solution}
