\begin{exercise}
Schreiben Sie ein Programm, welches für $k \in \N$ das explizite, beziehungsweise
implizite lineare $k$-Schritt Verfahren maximaler Konsistenzordnung berechnet. Geben Sie dann speziell
das explizite, beziehungsweise implizite 2- und 3-Schritt-Verfahren maximaler Ordnung an.
\end{exercise}
\begin{solution}
Wir machen uns Theorem 5.15 zunutze. \includegraphicsboxed{5.15}
Unterpunkt $(iii)$ gibt uns einen guten Anhaltspunkt. Betrachte zuerst die impliziten Verfahren.
Wir haben jeweils $k+1$ Unbekannte $\alpha$ und $\beta$ . Erinnern wir uns daran,
dass $\alpha_k = 1$ haben wir schon eine Unbekannte weniger. Um auf die restlichen
zu kommen können wir die Gleichungen wie folgt umformen.

\begin{align*}
  \sum_{j=0}^k \alpha_j = 0 \Leftrightarrow -\alpha_k = \sum_{j=1}^{k-i} \alpha_j \\
  \sum_{j=0}^{k-1}\alpha_j j^i - \sum_{j=0}^k \beta_j ij^{i-1} = -k^i \quad i=1,\dots,p
\end{align*}

Mit diesen Informationen stellen wir ein lineares $2k+1$ Gleichungssystem auf:

\begin{align*}
  \left(\begin{array}{c|c}
    1 \text{ }1\dots & 0\text{ } 0 \dots \\
    \hline
    \bigg( j^i\bigg)_{j=0,\dots,k-1}^{i=1,\dots,2k} & \bigg( -ij^{i-1}\bigg)_{j=0,\dots,k}^{i=1,\dots,2k}
  \end{array}\right)
  \left(\begin{array}{c}
    \alpha_0 \\
    \alpha_1 \\
    \vdots \\
    \alpha_{k-1} \\
    \beta_0 \\
    \vdots \\
    \beta_k
  \end{array}\right) =
  \left(\begin{array}{c}
    -k^0 \\
    -k^1 \\
    \vdots \\
    -k^{2k}
  \end{array}\right)
\end{align*}
Damit sieht man auch, dass das Verfahren zumindest Ordnung $2k$ hat. Wir können nun,
wenn wir die $\alpha$ und $\beta$ berechnen, nachprüfen für welche $p \geq 2k$ die Bedingung
in $(iii)$ erfüllt ist und davon das Größte wählen. (Anscheinend ist die Maximale schon $2k$.)

Für explizite Verfahren vereinfacht sich unser System, da bei diesen $\beta_k = 0$ gilt.
Dann sieht das ganze so aus:
\begin{align*}
  \left(\begin{array}{c|c}
    1 \text{ }1\dots & 0\text{ } 0 \dots \\
    \hline
    \bigg( j^i\bigg)_{j=0,\dots,k-1}^{i=1,\dots,2k} & \bigg( -ij^{i-1}\bigg)_{j=0,\dots,k-1}^{i=1,\dots,2k}
  \end{array}\right)
  \left(\begin{array}{c}
    \alpha_0 \\
    \alpha_1 \\
    \vdots \\
    \alpha_{k-1} \\
    \beta_0 \\
    \vdots \\
    \beta_{k-1}
  \end{array}\right) =
  \left(\begin{array}{c}
    -k^0 \\
    -k^1 \\
    \vdots \\
    -k^{2k-1}
  \end{array}\right)
\end{align*}

Hier hat das Verfahren also zumindest Ordnung $2k-1$ und wir können genauso nachprüfen
welche Ordnung maximal ist. (Wiederum ist diese Ordnung augenscheinlich auch die Maximale)
\end{solution}
