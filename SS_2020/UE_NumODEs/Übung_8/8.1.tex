\begin{exercise}
Konstruieren Sie ein Adamsverfahren mit $k = 2, s = 1$ und $r = 0$ und variablen
Schrittweiten $h_j := t_j - t_{j-1}$ der Form
\begin{align}
  y_{i+1} = y_i + \sum_{j=0}^2 \beta_{i,j}(h_{i-1},h_i,h_{i+1})f_{i-j},
  \qquad f_{i-j} := f(t_{i-j},y_{i-j}).
\end{align}
Zur Bestimmung der Konstanten $\beta_{i,j}$ können Sie ein Computeralgebrasystem
verwenden. Vergleichen Sie Ihr Verfahren mit den Werten aus dem Vorlesungsskript
für äquidistante Schrittweiten.
\end{exercise}
\begin{solution}
Die Anzahl der Schritte $m$ dieses Verfahrens ist $m = \max\{r+1,s+k\}=\min\{1,3\} = 3$.
Eine allgemeine lineare $k$-Schrittmethode hat die Form
\begin{align*}
  \sum_{j = 0}^k \alpha_{k - j}y_{l+1-j} = h \sum_{j = 0}^k \beta_{k-j}f(t_{\ell + 1 - j},y_{\ell + 1 -j}).
\end{align*}
In unserem Fall ergibt das
\begin{align*}
  \alpha_3 y_{\ell + 1} + \alpha_2 y_{\ell} + \alpha_1 y_{\ell - 1} + \alpha_0 y_{\ell - 2} =
  h \left(\beta_3 f(t_{\ell + 1},y_{\ell + 1}) +
  \beta_2 f(t_{\ell},y_{\ell}) +
  \beta_1 f(t_{\ell - 1},y_{\ell - 1}) +
  \beta_0 f(t_{\ell - 2},y_{\ell - 2}))\right),
\end{align*}
wobei bei Adamsverfahren $\alpha_{m-j} = 0$ für $j \in \{0,\dots,m-1\}\backslash\{r+1\}$,
$\alpha_m = 1, \alpha_{m-r-1} = -1$ gilt. \\
Ebenso gilt $\beta_{m-j} = 0$ für $j \in \{0,\dots,s-1\}$
und $\beta_{m-j} = b_{j-s}$ für $j \in \{s,\dots,m\}$. \\
Also vereinfacht sich der Ausdruck zu
\begin{align*}
y_{\ell + 1} - y_{\ell} =
h \left(
b_0 f(t_{\ell},y_{\ell}) +
b_1 f(t_{\ell - 1},y_{\ell - 1}) +
b_0 f(t_{\ell - 2},y_{\ell - 2}))\right),
\end{align*}
wobei man die $b_j$ durch Integration der entsprechenden Lagrange-Interpolationspolynome erhält.
\begin{align*}
  b_j = \int_{0}^1 \prod_{\stackrel{m = 0}{m \neq j}}^{2}\frac{\tilde{t} + m}{m - j}dt.
\end{align*}
Also berechnen wir
\begin{align*}
  b_0  &= \int_{0}^1 (t + 1)\frac{t + 2}{2}dt  = \frac{23}{12}\\
  b_1  &= \int_{0}^1 -t(t + 2)dt = -\frac{16}{12}\\
  b_2  &= \int_{0}^1 \frac{t}{2}(t + 1)dt = \frac{5}{12}\\.
\end{align*}
Unsere Methode lautet daher mit der Notation $f_{\ell} := f(t_{\ell},y_{\ell})$
\begin{align*}
  y_{\ell + 1} = y_{\ell} + \frac{h}{12}\left(23f_{\ell} -16f_{\ell - 1} + 5f_{\ell - 2}\right)
\end{align*}
und der Vergleich mit dem Skript gibt uns recht.
\end{solution}
