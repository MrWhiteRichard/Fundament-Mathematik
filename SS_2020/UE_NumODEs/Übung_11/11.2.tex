\begin{exercise}
  Betrachten Sie die Hamilton-Funktion $H(p,q) = \frac{1}{2}p^2 + \frac{1}{2}q^2$.
  Zeigen Sie, dass das symplektische Euler-Verfahren
  \begin{align}
    \begin{pmatrix}
      p_{\ell + 1} \\ q_{\ell + 1}
    \end{pmatrix}
    = \begin{pmatrix}
      p_{\ell} \\ q_{\ell}
    \end{pmatrix} +
    h\begin{pmatrix}
      -\nabla_q H(p_{\ell + 1},q_{\ell}) \\
      \nabla_p H(p_{\ell + 1},q_{\ell})
    \end{pmatrix}
  \end{align}
  im Allgemeinen \textit{nicht} die Energie $H(p,q)$ erhält. Zeigen Sie dazu,
  dass es Anfangsbedingungen $p_0,q_0$ gibt, sodass $H(p_{\ell},q_{\ell})
  \neq H(p_0,q_0)$ für die numerischen Lösungen $p_{\ell},q_{\ell}$ des
  symplektischen Eulerverfahren gilt. \\
  Betrachten Sie weiter die gestörte Hamilton-Funktion
  \begin{align}
    H_h(p,q) := \frac{1}{2}\begin{pmatrix}
      p \\ q
    \end{pmatrix}^{\top}
    \begin{pmatrix}
      1 & - \nicefrac{h}{2} \\ -\nicefrac{h}{2} & 1
    \end{pmatrix}
    \begin{pmatrix}
      p \\ q
    \end{pmatrix}.
  \end{align}
  Zeigen Sie, dass für alle $p,q$ mit $|p|,|q| \leq R \in \R$ gilt, dass
  $\|H(p,q) - H_h(p,q)\| = \Landau{h}$ und dass das symplektische Eulerverfahren
  $H_h$ erhält, also zeigen Sie $H_h(p_{\ell},q_{\ell}) = H_h(p_0,q_0)$.
\end{exercise}

\begin{solution}
Beweis.
\end{solution}
