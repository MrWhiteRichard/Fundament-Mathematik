\begin{exercise}
  Sei $f \in C^1(\R^{2d},\R^{2d})$ die rechte Seite des autonomen Systems
  \begin{align}\label{aut}
    \begin{pmatrix}
      p^{\prime} \\ q^{\prime}
    \end{pmatrix} = f(p,q).
  \end{align}
  Weiters sei die Abbildung $R$ definiert durch
  \begin{align}
    R(p,q) = \begin{pmatrix}
      -p \\ q
    \end{pmatrix}.
  \end{align}
  Es gelte außerdem
  \begin{align}\label{rf}
    R \circ f = -f \circ R.
  \end{align}
  \begin{enumerate}[label = \textbf{\alph*)}]
    \item Zeigen Sie, dass für den kontinuierlichen Fluss $\Phi^t$ von \eqref{aut}
    gilt, dass
    \begin{align*}
      R \circ \Phi^t = \Phi^{-t} \circ R.
    \end{align*}
    \item Zeigen Sie, dass für den diskreten Fluss $\Psi^h$ eines Runge-Kutta-Verfahrens angewandt auf \eqref{aut} gilt, dass
    \begin{align*}
      R \circ \Psi^h = \Psi^{-h} \circ R.
    \end{align*}
    \item Sei $M$ eine symmetrisch, positiv definite Matrix, $U$ eine zweimal
    stetig differenzierbare Funktion und $H$ eine Hamilton-Funktion mit
    \begin{align*}
      H(p,q) := \frac{1}{2}p^{\top}M^{-1}p + U(q).
    \end{align*}
    Zeigen Sie, dass die Funktion $f$ des zugehörigen Hamilton-Systems \eqref{rf}
    erfüllt.
  \end{enumerate}
\end{exercise}

\begin{solution}
  Beweis.
\end{solution}
