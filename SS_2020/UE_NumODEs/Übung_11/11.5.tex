\begin{exercise}
  Sei $f \in C^1(\R^{2d},\R^{2d})$ die rechte Seite des autonomen Systems
  \begin{align}\label{aut}
    \begin{pmatrix}
      p^{\prime} \\ q^{\prime}
    \end{pmatrix} = f(p,q).
  \end{align}
  Weiters sei die Abbildung $R$ definiert durch
  \begin{align}
    R(p,q) = \begin{pmatrix}
      -p \\ q
    \end{pmatrix}.
  \end{align}
  Es gelte außerdem
  \begin{align}\label{rf}
    R \circ f = -f \circ R.
  \end{align}
  \begin{enumerate}[label = \textbf{\alph*)}]
    \item Zeigen Sie, dass für den kontinuierlichen Fluss $\Phi^t$ von \eqref{aut}
    gilt, dass
    \begin{align*}
      R \circ \Phi^t = \Phi^{-t} \circ R.
    \end{align*}
    \item Zeigen Sie, dass für den diskreten Fluss $\Psi^h$ eines Runge-Kutta-Verfahrens angewandt auf \eqref{aut} gilt, dass
    \begin{align*}
      R \circ \Psi^h = \Psi^{-h} \circ R.
    \end{align*}
    \item Sei $M$ eine symmetrisch, positiv definite Matrix, $U$ eine zweimal
    stetig differenzierbare Funktion und $H$ eine Hamilton-Funktion mit
    \begin{align*}
      H(p,q) := \frac{1}{2}p^{\top}M^{-1}p + U(q).
    \end{align*}
    Zeigen Sie, dass die Funktion $f$ des zugehörigen Hamilton-Systems \eqref{rf}
    erfüllt.
  \end{enumerate}
\end{exercise}

\begin{solution}
\leavevmode \\
\begin{enumerate}[label = \textbf{\alph*)}]
  \item Sei $y_0 := (p_0,q_0)^{\top}$
    \begin{align*}
      \Phi^t(y_0) &:= y_{y_0}(t) \\
      R \circ \Phi^t(y_0) &= R(y_{y_0}(t)) = R\left(
      \begin{pmatrix}
        p_0 \\ q_0
      \end{pmatrix}
      + \int_0^t f(p_{y_0}(\tau),q_{y_0}(\tau)) d\tau\right)
      = R\left(
      \begin{pmatrix}
        p_0 \\ q_0
      \end{pmatrix}\right)
      + \int_0^t R\circ f(p_{y_0}(\tau),q_{y_0}(\tau)) d\tau\\
      &= \begin{pmatrix}
        -p_0 \\ q_0
      \end{pmatrix}
      - \int_0^t f\circ R(p_{y_0}(\tau),q_{y_0}(\tau)) d\tau
    \end{align*}
    Jetzt gilt
    \begin{align*}
      \begin{pmatrix}
        -p_{y_0}(t) \\ q_{y_0}(t)
      \end{pmatrix}^{\prime}
      &= \begin{pmatrix}
        -f_1(p_{y_0}(t),q_{y_0}(t)) \\ f_2(p_{y_0}(t),q_{y_0}(t))
      \end{pmatrix}
      = \begin{pmatrix}
        f_1(-p_{y_0}(t),q_{y_0}(t)) \\ f_2(-p_{y_0}(t),q_{y_0}(t))
      \end{pmatrix} \\
      \begin{pmatrix}
        -p_{y_0}(0) \\ q_{y_0}(0)
      \end{pmatrix}
      &= \begin{pmatrix}
        -p_0 \\ q_0
      \end{pmatrix}
    \end{align*}
    Also ist $(-p_{y_0},q_{y_0})$ die Lösung des Anfangswertproblems mit Startwert
    $(p_0,q_0)$ und es gilt
    \begin{align*}
      R \circ \Phi^t(y_0) &= \begin{pmatrix}
        -p_0 \\ q_0
      \end{pmatrix}
      - \int_0^t f\circ R(p_{y_0}(\tau),q_{y_0}(\tau)) d\tau
      = \begin{pmatrix}
        -p_0 \\ q_0
      \end{pmatrix}
      - \int_0^{t} f\circ y_{R(y_0)}(\tau) d\tau \\
      &= \begin{pmatrix}
        -p_0 \\ q_0
      \end{pmatrix}
      - y_{R(y_0)}(t)
      = y_{R(y_0)}(-t) = \Phi^{-t} \circ R(y_0).
    \end{align*}
  \item
  Seien $\Psi^h(y_0) = y_0 + h\sum_{k=1}^mb_jk_j$ und $\Psi^{-h}(R(y_0)) = y_0 + h\sum_{k=1}^mb_jl_j$.
  Dann gilt
  \begin{align*}
    \|k_j + l_j\| = \|f(y + h\sum_{\ell = 1}^mA_{j\ell}k_{\ell}) + f(R(y) + h\sum_{\ell = 1}^mA_{j\ell}k_{\ell})\|
    \leq L\|y + h\sum_{\ell = 1}^mA_{j\ell}k_{\ell} + R(y) + h\sum_{\ell = 1}^mA_{j\ell}k_{\ell}\|
  \end{align*}
  \begin{align*}
    R \circ \Psi^h (y_0) &= R(y_0 + h\sum_{k=1}^mb_jk_j) = R(y_0)+ h\sum_{k=1}^mb_jR(k_j) \\
    &= R(y_0)+ h\sum_{k=1}^mb_jR \circ f(y_0 + h\sum_{\ell = 1}^mA_{j\ell}k_{\ell})
    = R(y_0)- h\sum_{k=1}^mb_jf \circ R(y_0 + h\sum_{\ell = 1}^mA_{j\ell}k_{\ell}) \\
    &= R(y_0)- h\sum_{k=1}^mb_jf(R(y_0) + h\sum_{\ell = 1}^mA_{j\ell}R(k_{\ell})) =
    \Psi^{-h} \circ R(y_0) = R(y_0) + h\sum_{k=1}^mb_jf(R(y_0) + h\sum_{\ell = 1}^mA_{j\ell}k_{\ell})
  \end{align*}
\end{enumerate}

\end{solution}
