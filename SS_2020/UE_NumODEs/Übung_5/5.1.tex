\begin{exercise}
Das folgebde Anfangswertproblem modelliert die Reaktion dreier Spezies
$x,y,z:$
\begin{align*}
  x^{\prime}(t) = -0.04x(t) + 10^4y(t)z(t), \\
  y^{\prime}(t) = 0.04x(t) - 10^4y(t)z(t) - 3\cdot10^7y(t)^2, \\
  z^{\prime}(t) = 3\cdot10^7y(t)^2,
\end{align*}
mit Anfangswerten $x(0) = 1, y(0) = 0, z(0) = 0$ und $t \in [0,1]$.
Approximieren Sie die Lösung der Gleichung zunächst mit dem
eingebetteten RK5(4) Verfahren mit verschiedenen Fehlertoleranzen und
betrachten Sie die sich ergebenden Schrittweiten. Approximieren Sie nun
die Lösung mit dem expliziten RK4 und dem expliziten Eulerverfahren.
Versuchen Sie dabei die Schrittweiten zu optimieren. Beschreiben Sie
Ihre Beobachtungen. Welches der verwendeten Verfahren erscheint Ihnen
bei diesem Beispiel am effizientesten? \\
\textit{Hinweis:} Sie können die im TUWEL zur Verfügung gestellten
Programme verwenden.
\end{exercise}
\begin{solution}
Lösung.
\end{solution}
