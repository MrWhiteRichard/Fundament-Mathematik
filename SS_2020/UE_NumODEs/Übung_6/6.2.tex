\begin{exercise}
Sei zunächst $[a,b] = [-1,1]$ und $\omega(x) \equiv 1$.
\begin{enumerate}[label = \textbf{\alph*)}]
  \item Zeigen Sie, dass die Orthogonalpolynome $(q_s)_{s \in \N_0}$ aus Remark B.13
  des Vorlesungsskripts gerade, beziehungsweise ungerade Polynome sind, falls $s$
  gerade, beziehungsweise ungerade ist. Dazu können Sie die Konstruktion dieser
  Polynome aus der Monombasis mit Hilfe des Orthogonalisierungsverfahrens von
  Gram-Schmidt verwenden.
  \item
  \renewcommand{\arraystretch}{1.2}
  Sei nun ~$\begin{matrix}
    c & \vline & A \\
    \hline
    & \vline & b^{\top}
  \end{matrix}$~
  \renewcommand{\arraystretch}{1}
  das Butcher Tableau eines durch Kollokation erzeugten $m$-stufigen Runge-Kutta-Verfahrens,
  bei dem Gauß-Quadratur zur Grundlage genommen wurde. Beweisen Sie, dass die Kollokationspunkte
  und die Gewichte im folgenden Sinn symmetrisch sind:
  \begin{align}
    \left|c_j - \frac{1}{2}\right| = \left| c_{m + 1 - j} - \frac{1}{2}\right|,
    \qquad b_j = b_{m+1-j}, \qquad j = 1,\dots,m.
  \end{align}
\end{enumerate}
\end{exercise}
\begin{solution}
\leavevmode \\
\begin{enumerate}[label = \textbf{\alph*)}]
\item
\end{enumerate}
\end{solution}
