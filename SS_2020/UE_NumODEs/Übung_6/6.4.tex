\begin{exercise}
Berechnen Sie die Lösung des Anfangswertproblems \eqref{awp} auf dem Intervall
$[0,1]$ numerisch. Der Anfangswert $G$ soll die Funktion
$g(x) := \exp\left(-30\left(x-\frac{1}{2}\right)^2\right)$ nodal approximieren, also
$G_i := g\left(\frac{i}{N}\right)$ für $i = 1,\dots,N-1$.
\begin{enumerate}[label = \textbf{\alph*)}]
  \item Sei die Ortsschrittweite $h$ gegeben. Berechnen Sie mit Hilfe von \eqref{ew},
  wie groß die Zeitschrittweite $\tau$ abhängig von $h$ maximal sein darf, damit
  das explizite Euler-Verfahren zu exponentiell fallenden Lösungen führt.
  \item Überprüfen Sie dieses Resultat numerisch. Testen Sie dafür das explizite
  Euler-Verfahren für dieses Problem numerisch für unterschiedliche Orts- und
  Zeitschrittweiten (zum Beispiel $h = 2^{-1},\dots,2^{-10}, \\
  \tau = 2^{-1},\dots,2^{-10}$).
  Dazu können Sie zum Beispiel $||U_h(t)||_{\infty}$ über eine gewisse Zeit $t \in [0,T]$
  grafisch darstellen.
  \item Testen Sie nun mit dem impliziten Euler-Verfahren. Zur Lösung des linearen
  Gleichungssystem verwenden Sie bitte $LU$-Faktorisierung und Vorwärts-/Rückwärtssubstitution
  (zum Beispiel in Python durch die Funktionen \texttt{lu\_factor} und
  \texttt{lu\_solve} der
  Bibliothek \texttt{scipy.linalg}).
\end{enumerate}
\end{exercise}
\begin{solution}
\leavevmode \\
\begin{enumerate}[label = \textbf{\alph*)}]
\item
\end{enumerate}
\end{solution}
