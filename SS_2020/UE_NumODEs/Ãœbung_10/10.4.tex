\begin{exercise}
  Sei $H \in C^2(\R^{2d};\R)$ eine Hamilton-Funktion und
  $\Psi \in C^1(\R^{2d};\R^{2d})$ ein Diffeomorphismus,
  d.h. auch die inverse Abbildung $\Psi^{-1}$ sei stetig
  differenzierbar. Weiter sei $\Psi$ symplektisch.

  Zeigen Sie folgende Aussage: Wenn $y \in C^1([0,T];\R^{2d})$
  wie in Remark 6.2 des Vorlsesungsskriptes eine Lösung des
  Hamilton Systems

  \begin{align}
    y^\prime = J^{-1}\nabla H(y)
  \end{align}

  ist, dann ist $u := \Psi \circ y$ die Lösung des Hamilton
  Systems mit Hamilton-Funktion $K := H \circ \Psi^{-1}$.
\end{exercise}

\begin{solution}
  $\Psi$ ist symplektisch, also gilt $\forall z \in \R^{2d}: D\Psi(z)^{\top}J\Psi(z) = J$.
  Weiters gilt
  \begin{align*}
    Id = D(id(y)) = D(\Psi^{-1} \circ \Psi(y)) = D(\Psi^{-1}(\Psi(y)))D\Psi(y).
  \end{align*}
  Also folgt $D\Psi(y)^{-1} =  D(\Psi^{-1}(\Psi(y)))$
  Damit rechnen wir einfach nach ($J^{-1} = -J = J^{\top}$).
  \begin{align*}
    u^{\prime} &= D \Psi(y)y^{\prime} = D\Psi(y)J^{-1}\nabla H(y) =  D\Psi(y)J^{-1}D\Psi(y)^{\top}(D\Psi(y)^{\top})^{-1}\nabla H(y) \\
    &= D\Psi(y)J^{\top}D\Psi(y)^{\top}(D\Psi(y)^{\top})^{-1}\nabla H(y)
    = (D\Psi(y)^{\top}JD\Psi(y))^{\top}(D\Psi(y)^{\top})^{-1}\nabla H(y)
    = J^{\top}(D\Psi(y)^{-1})^{\top}\nabla H(y) \\
    &= J^{-1}((D\Psi(y))^{-1})^{\top}\nabla H(\Psi^{-1}(u))
    = J^{-1}D(\Psi^{-1}(\Psi(y)))^{\top}\nabla H(\Psi^{-1}(u)) =
    J^{-1}D(\Psi^{-1}(u))^{\top}\nabla H(\Psi^{-1}(u)) \\
    &= J^{-1}D(H \circ \Psi^{-1})^{\top}(u) =  J^{-1}\nabla K(u).
  \end{align*}

\end{solution}
