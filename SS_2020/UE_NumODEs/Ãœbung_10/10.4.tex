\begin{exercise}
  Sei $H \in C^2(\R^{2d};\R)$ eine Hamilton-Funktion und
  $\Psi \in C^1(\R^{2d};\R^{2d})$ ein Diffeomorphismus,
  d.h. auch die inverse Abbildung $\Psi^{-1}$ sei stetig
  differenzierbar. Weiter sei $\Psi$ symplektisch.

  Zeigen Sie folgende Aussage: Wenn $y \in C^1([0,T];\R^{2d})$
  wie in Remark 6.2 des Vorlsesungsskriptes eine Lösung des
  Hamilton Systems

  \begin{align}
    y^\prime = J^{-1}\nabla H(y)
  \end{align}

  ist, dann ist $u := \Psi \circ y$ die Lösung des Hamilton
  Systems mit Hamilton-Funktion $K := H \circ \Psi^{-1}$.
\end{exercise}

\begin{solution}
  Beweis.
\end{solution}
