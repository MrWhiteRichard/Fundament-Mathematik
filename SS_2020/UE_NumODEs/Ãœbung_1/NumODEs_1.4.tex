\begin{exercise}

Beweisen Sie folgende Variation des Satzes 1.3 aus der Vorlesung:
$f$ sei bezüglich des zweiten Argumentes nur einseitig Lipschitz-stetig, d.h. es existiert ein $L_+ \in \R$ mit

\begin{align*}
  \abraces{f(t, y) - f(t, z), y - z}
  \leq
  L_+ \norm[2]{y - z}^2,
  \quad
  (t, y), (t, z) \in J \times \Omega.
\end{align*}

Weiter sei auch $z$ eine Lösung der Differentialgleichung $z^\prime = f(t, z)$ (d.h. $\delta = 0$ in Satz 1.3). Dann gilt

\begin{align*}
  \norm[2]{y(t) - z(t)}
  \leq
  \norm[2]{y(t_0) - z(t_0)} e^{L_+ (t - t_0)},
  \quad
  t \geq t_0.
\end{align*}

\end{exercise}

\begin{solution}

Wir wollen für den Beweis wieder das Gronwall-Lemma anwenden und definieren somit erstmal fröhlich:

\begin{align*}
  v(t)&:= \norm[2]{y(t)-z(t)}^{2}\\
  A &:= v(t_{0})
\end{align*}

Nun ist $v$ differenzierbar und es gilt:

\begin{align*}
  v'(t)=\frac{d}{dt}\sum_{i=0}^{n} (y_{i}(t)-z_{i}(t))^{2} =\sum_{i=0}^{n} \frac{d}{dt} (y_{i}(t)-z_{i}(t))^{2} =
  \sum_{i=0}^{n} 2(y_{i}(t)-z_{i}(t))(y'_{i}(t)-z'_{i}(t)) = 2 \abraces{y'(t)-z'(t),y(t)-z(t)}_2
\end{align*}
Nun verwenden wir den Hauptsatz der Differnetial und Integralrechnung sowie die Dreiecksungleichung und die einseitige
Lipschitz-stetigkeit für unsere Abschätzungen und erhalten:

\begin{align*}
  v(t) &\leq A + \int_{t_0}^{t} v'(\tau)d\tau \\
  &= A + 2\int_{t_0}^{t}\abraces{y'(\tau)-z'(\tau),y(\tau)-z(\tau)}_2d\tau \\
  &= A + 2\int_{t_0}^{t}\abraces{f(\tau,y(\tau))-f(\tau,z(\tau)),y(\tau)-z(\tau)}_2d\tau \\
  &\leq A +2L_{+}\int_{t_0}^tv(\tau)d\tau \\
  &\end{align*}

Somit gilt nach dem Gronwall-Lemma:

\begin{align*}
  \norm[2]{y(t)-z(t)}^{2} \leq \norm[2]{y(t_{0})-z(t_{0})}^{2}e^{2L_{+}(t-t_{0})}
\end{align*}

Wurzelziehen gibt uns dann das gewünschte Ergebnis.
\end{solution}
