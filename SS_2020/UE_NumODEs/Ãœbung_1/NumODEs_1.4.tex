\begin{exercise}

Beweisen Sie folgende Variation des Satzes 1.3 aus der Vorlesung:
$f$ sei bezüglich des zweiten Argumentes nur einseitig Lipschitz-stetig, d.h. es existiert ein $L_+ \in \R$ mit

\begin{align*}
  \abraces{f(t, y) - f(t, z), y - z}
  \leq
  L_+ \norm[2]{y - z}^2,
  \quad
  (t, y), (t, z) \in J \times \Omega.
\end{align*}

Weiter sei auch $z$ eine Lösung der Differentialgleichung $z^\prime = f(t, z)$ (d.h. $\delta = 0$ in Satz 1.3). Dann gilt

\begin{align*}
  \norm[2]{y(t) - z(t)}
  \leq
  \norm[2]{y(t_0) - z(t_0)} e^{L_+ (t - t_0)},
  \quad
  t \geq t_0.
\end{align*}

\end{exercise}

\begin{solution}

Zuerst, bemerken wir, dass

\begin{align*}
  \frac{d}{d \tau} \norm[2]{y(\tau) - z(\tau)}^2
  & =
  \sum_{i=1}^n \frac{d}{d \tau} (y_i(\tau) - z_i(\tau))^2
  =
  \sum_{i=1}^n 2 (y_i(\tau) - z_i(\tau))(y_i^\prime(\tau) - z_i^\prime(\tau)) \\
  & =
  2 \abraces
  {
    y^\prime(\tau) - z^\prime(\tau),
    y(\tau) - z(\tau)
  }_2
  =
  2 \abraces
  {
    f(\tau, y(\tau)) - f(\tau, z(\tau)),
    y(\tau) - z(\tau)
  }_2
  \leq
  2 L_+ \norm[2]{y(\tau) - z(\tau)}^2.
\end{align*}

Aus dem Hauptsatz der Differnetial- und Integralrechnung, der Dreiecksungleichung, folgt also

\begin{align*}
  \norm[2]{y(t) - z(t)}^2
  & \leq
  \norm[2]{y(t_0) - z(t_0)}^2
  +
  \Int[t_0][t]
  {\frac{d}{d \tau} \norm[2]{y(\tau) - z(\tau)}^2}{\tau} \\
  & \leq
  \norm[2]{y(t_0) - z(t_0)}^2
  +
  2 L_+ \Int[t_0][t]
  {\norm[2]{y(\tau) - z(\tau)}^2}{\tau}.
\end{align*}

Mit dem Grönwall-Lemma, und den (jetzt nicht unmotivierten) Definitionen

\begin{align*}
  v(t) := \norm[2]{y(t) - z(t)}^2,
  \quad
  A := v(t_0),
  \quad
  B := 0,
\end{align*}

erhalten wir

\begin{align*}
  \norm[2]{y(t) - z(t)}^{2}
  \leq
  \norm[2]{y(t_0) - z(t_0)}^2 e^{2 L_+ (t - t_0)}.
\end{align*}

Wenn man hier noch die $\sqrt{\cdot}$ zieht, folgt die Behauptung.

\end{solution}
