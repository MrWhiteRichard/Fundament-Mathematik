\begin{exercise}

Beweisen Sie folgende Variation des Satzes 1.3 aus der Vorlesung:
$f$ sei bezüglich des zweiten Argumentes nur einseitig Lipschitz-stetig, d.h. es existiert ein $L_+ \in \R$ mit

\begin{align*}
  \abraces{f(t, y) - f(t, z), y - z}
  \leq
  L_+ \norm[2]{y - z}^2,
  \quad
  (t, y), (t, z) \in J \times \Omega.
\end{align*}

Weiter sei auch $z$ eine Lösung der Differentialgleichung $z^\prime = f(t, z)$ (d.h. $\delta = 0$ in Satz 1.3). Dann gilt

\begin{align*}
  \norm[2]{y(t) - z(t)}
  \leq
  \norm[2]{y(t_0) - z(t_0)} e^{L_+ (t - t_0)},
  \quad
  t \geq t_0.
\end{align*}

\end{exercise}

\begin{solution}

Trivial!

\end{solution}
