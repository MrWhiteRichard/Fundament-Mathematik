\begin{exercise}

Sei $y \in C^1(\R_{\geq 0}, \R)$ Lösung des Anfangswertproblems

\begin{align*}
  y^\prime(t) = \lambda y(t),
  \quad
  t > 0,
  \quad
  y(0) = y_0
\end{align*}

mit einem $\lambda < 0$.
Sei $h > 0$ eine konstante Schrittweite, $t_j := jh$, $j \in \N_0$, und $y^e_j$ bzw. $y^i_j$ die Approximationen an $y(t_j)$ aus dem expliziten bzw. impliziten Eulerverfahren.
Untersuchen Sie in Abhängigkeit von $\lambda$ und $h$ das Verhalten von $y^e_j$ bzw. $y^i_j$ für $j \to \infty$ und vergleichen Sie es mit dem der exakten Lösung $y(t_j)$.

\end{exercise}

\begin{solution}

Das Anfangswertproblem kann durch $f(t, y) = \lambda y$ beschrieben werden. Man verifiziert unmittelbar, dass die exakte Lösung folgende ist.

\begin{align*}
  y(t) = e^{\lambda t} y_0
\end{align*}

Damit erhält man

\begin{align*}
  y(t_j) \xrightarrow{j \to \infty} 0.
\end{align*}

Die Approximationen $y_j^{\mathrm{e}}$ und $y_j^{\mathrm{i}}$ sollten also auch für $j \to \infty$ verschwinden. \\

Mit $j \in \N_0$ lautet die explizite Euler Methode

\begin{align*}
  \Phi(t_j, y_j^{\mathrm{e}}, h) := f(t_j, y_j^{\mathrm{e}}),
  \quad
  y_{j+1}^{\mathrm{e}} := y_j^{\mathrm{e}} + h \Phi(t_j, y_j^{\mathrm{e}}, h).
\end{align*}

Man erhält also konkret

\begin{align*}
  y_{j+1}^{\mathrm{e}}
  =
  y_j^{\mathrm{e}} + h \lambda y_j^{\mathrm{e}}
  =
  (1 + h \lambda) y_j^{\mathrm{e}}.
\end{align*}

Damit folgt unmittelbar

\begin{align*}
  y_j^{\mathrm{e}}
  =
  (1 + h \lambda)^j y_0.
\end{align*}

Dieser Ausdruck ist für $y_0 = 0$ offensichtlich konstant $0$ und somit dagegen konvergent.
Er verschwindet aber auch für $|1 + h \lambda| < 1$.
Sollte allerdings $y_0 \neq 0$ und $|1 + h \lambda| \geq 1$ so konvergiert er nicht.
Dabei oszilliert die Approximation um die $t$-Achse, wenn $1 + h \lambda < 0$.
Um das zu vermeiden, muss für gegebenes $\lambda < 0$, $h > 0$ hinreichend klein sein. \\

Mit $j \in \N_0$ lautet die implizite Euler Methode

\begin{align*}
  \Phi(t_j, y_j^{\mathrm{i}}, y_{j+1}^{\mathrm{i}}, h) := f(t_{j+1}, y_{j+1}^{\mathrm{i}}),
  \quad
  y_{j+1}^{\mathrm{i}} := y_j^{\mathrm{i}} + h \Phi(t_j, y_j^{\mathrm{i}}, y_{j+1}^{\mathrm{i}}, h).
\end{align*}

Man erhält also konkret

\begin{align*}
  y_{j+1}^{\mathrm{i}}
  =
  y_j^{\mathrm{i}} + h \lambda y_{j+1}^{\mathrm{i}}
  \Rightarrow
  y_{j+1}^{\mathrm{i}}
  =
  (1 - h \lambda)^{-1} y_j^{\mathrm{i}}.
\end{align*}

Damit folgt unmittelbar

\begin{align*}
  y_j^{\mathrm{i}}
  =
  (1 - h \lambda)^{-j} y_0.
\end{align*}

Dieser Ausdruck ist für $y_0 = 0$ offensichtlich konstant $0$ und somit dagegen konvergent.
Er verschwindet aber auch für

\begin{align*}
  |1 - h \lambda|^{-1} < 1
  \Leftrightarrow
  1 < |1 - \underbrace{h \lambda}_{< 0}| = 1 - h \lambda
  \Leftrightarrow
  h \lambda < 0.
\end{align*}

Schließlich, konvergieren die Approximationen beider Eulerverfahren, für $h \to 0$, gegen die exakte Lösung.

\begin{align*}
  y_j^{\mathrm{e}}
  =
  \pbraces
  {
    (1 + h \lambda)^\frac{1}{h \lambda}
  }^{\lambda t_j}
  \xrightarrow{h \to 0}
  y(t_j),
  \quad
  y_j^{\mathrm{i}}
  =
  -\pbraces
  {
    (1 - h \lambda)^\frac{1}{h \lambda}
  }^{-\lambda t_j}
  \xrightarrow{h \to 0}
  y(t_j)
\end{align*}

\end{solution}
