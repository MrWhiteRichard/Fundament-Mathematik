\begin{exercise}

Sei $A \in \R^{n \times n}$ symmetrisch und $y \in C^1([0, T], \R^n)$ Lösung des Anfangswertproblems

\begin{align*}
  y^\prime(t) = A y(t),
  \quad
  t \in [0, T],
  \quad
  y(0) = y_0.
\end{align*}

Berechnen Sie die Lipschitz-Konstante sowie die einseitige Lipschitz-Konstante der zugehörigen Funktion $f$ und vergleichen Sie die Aussage aus dem Satz 1.3 mit der Aussage aus Aufgabe 4.

Hinweis: Symmetrische Matrizen sind diagonalisierbar.

\end{exercise}

\begin{solution}

Die zugehörige Funktion ist $f: (t, y) \mapsto Ay$.
Also $\Forall (t, y), (t, z) \in \R \times \R^n:$

\begin{align*}
  \norm{f(t, y) - f(t, z)}
  =
  \norm{A (y - z)}
  \leq
  \norm{A} \cdot \norm{y - z}.
\end{align*}

Damit ist die Lipschitz-Konstante $L = \norm{A}$. \\

Weil $A$ diagonalisierbar ist, existiert eine Basis $B \in \R^{n \times n}$ aus Eigenvektoren von $A$, sodass $D := B^{-1} A B$ eine Diagonalmatrix ist.
$\rho(A) := \max \Bbraces{|\lambda|: \lambda \text{ Eigenwert von } A}$ ist der Spektralradius von $A$.
Weil $A = A^T$ symmetrisch ist, folgt $\Forall (t, y), (t, z) \in \R \times \R^n:$

\begin{align*}
  \abraces{f(t, y) - f(t, z), y - z}
  & =
  \abraces{A (y - z), y - z}
  =
  (y - z)^T A (y - z)
  =
  (y - z)^T B D B^{-1} (y - z) \\
  & \leq
  (y - z)^T B \rho(A) B^{-1} (y - z)
  =
  \rho(A) (y - z)^T (y - z)
  =
  \rho(A) \norm[2]{y - z}^2.
\end{align*}

Damit ist die einseitige Lipschitz-Konstante $L_+ = \rho(A)$. \\

Für jede natürliche Matrixnorm $\norm{\cdot}$ und jede Matrix $A \in \K^{n \times n}$, gilt $\rho(A) \leq \norm{A}$. Also ist die einseitige Lipschitz-Konstante \Quote{besser} also die Lipschitz-Konstante, d.h. $L_+ \leq L$. \\

Gemeinsam mit der Aufgabe 4, erhält man also dieselbe Konklusio, wie in Satz 1.3, mit $\delta = 0$, also einer exakten Lösung $z$.

\end{solution}
