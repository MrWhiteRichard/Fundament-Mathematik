\begin{exercise}
Gegeben seien zwei Adams-Bashforth-Verfahren mit $k$, beziehungsweise $k + 1$
Schritten und gleicher Gitterweite $h$. Wie in Definition 5.9 des Skriptes sei
$y_{\ell + 1}$ die Lösung des $k$-Schritt-Verfahrens und $\widetilde{y}_{\ell + 1}$
die Lösung des $k + 1$-Schritt-Verfahrens bei jeweils exakten Anfangswerten.
Leiten Sie einen berechenbaren Fehlerschätzer $\mu$ für den Konsistenzfehler $\tau_{\ell}(h)$
des ersten Verfahrens her. Es sollte gelten
\begin{align*}
  \tau_{\ell}(h) = \mu + \Landau{h^{k+3}}.
\end{align*}
Benutzen Sie dazu insbesondere die Entwicklung des Abschneidefehlers aus dem Beweis
von Theorem 5.15 und vollziehen Sie die Konstruktion des Fehlerschätzers in Abschnitt
2.6, beziehungsweise Abschnitt 2.7 nach.
\end{exercise}
\begin{solution}
Beweis.
\end{solution}
