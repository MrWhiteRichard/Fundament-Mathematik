\begin{exercise}
Verallgemeinern Sie den Beweis von Theorem 5.35 des Skriptums für den allgemeinen
Fall $n \in \N$. Dazu müssen Sie im Wesentlichen folgendes zeigen:
\begin{enumerate}[label = \textbf{\alph*)}]
  \item Mit den angepassten Definitionen aus dem Skript gilt
  \begin{align}
    E_{\ell + 1} = \left(A_{\rho}^{\top} \otimes I\right)E_{\ell} + F_{\ell},
  \end{align}
  wobei $I \in \R^{n \times n}$ die Einheitsmatrix und $A \otimes B$ das Kroneckerprodukt
  zweier Matrizen $A \in \R^{k \times k}$ und $B \in \R^{n \times n}$ ist, also
  \begin{align}
    A \otimes B := \begin{pmatrix}
      A_{11}B & \hdots & A_{1k}B \\
      \vdots & & \vdots \\
      A_{k1}B & \hdots & A_{kk}B
    \end{pmatrix}
    \in \R^{kn \times kn}.
  \end{align}
  \item Aus der Wurzelbedingung folgt
  \begin{align}
    \sup_{k \in \N_0} \left\| (A_{\rho}^{\top} \otimes I)^k\right\|_{\infty} < \infty.
  \end{align}
\end{enumerate}
Sie müssen auch erklären können, warum dies die beiden wesentlichen Änderungen
zum skalaren Fall sind!
\end{exercise}
\begin{solution}
Beweis.
\end{solution}
