\begin{exercise}
Wir verwenden die implizite Trapezregel aus Example 3.6.
\begin{enumerate}[label = \textbf{\alph*)}]
  \item Zeigen Sie, dass sie die gleiche Stabilitätsfunktion wie die implizite
  Mittelpunktsregel aus Example 3.5 besitzt.
  \item Zeigen Sie, dass die implizite Trapezregel A stabil ist.
  \item Zeigen Sie, dass die implizite Trapezregel nicht B stabil ist.
\end{enumerate}
\textit{Hinweis (zu c):} Zeigen Sie zunächst, dass die Funktion
\begin{align*}
  f(y) := \begin{cases}
    -y^3, & y \leq 0 \\
    -y^2, & y > 0
  \end{cases}
\end{align*}
dissipativ ist. Wenden Sie dann die implizite Trapezregel auf ein Anfangswertproblem
mit dieser rechten Seite und den Anfangswerten $y_0 = 0$, beziehungsweise $\widetilde{y}_0$ an.
Verwenden Sie zum Beispiel die Schrittweite $h = 0$ und finden Sie dann ein $\widetilde{y}_0$,
sodass $\widetilde{y}_1 > -\widetilde{y}_0$.
\end{exercise}
\begin{solution}
Lösung.
\end{solution}
