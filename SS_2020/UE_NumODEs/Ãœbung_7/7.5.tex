\begin{exercise}
Betrachten Sie ein $m$-stufiges Kollokationsverfahren mit Kollokationspunkten
$c_1,\dots,c_m$. Wir definieren das Polynom
\begin{align*}
  M(x) := \frac{1}{m!}\prod_{i = 1}^m (x - c_i).
\end{align*}
\begin{enumerate}[label = \textbf{\alph*)}]
  \item Zeigen Sie, dass sich für dieses Verfahren die Stabilitätsfunktion $R(z)$
  mit $z = \lambda h$ schreiben lässt als das rationale Polynom $R(z) = P(z)/Q(z)$,
  wobei $P,Q \in \Pi_m$ gegeben sind durch
  \begin{align*}
    P(z) &= M^{(m)}(1) + M^{(m-1)}(1)z + \dots + M(1)z^m, \\
    Q(z) &= M^{(m)}(0) + M^{(m-1)}(0)z + \dots + M(0)z^m.
  \end{align*}
  \item Verwenden Sie diese explizite Darstellung von $R(z)$ um zu zeigen, dass
  Gauß-Verfahren nicht L-stabil sind.
\end{enumerate}
\textit{Hinweis (zu a):} Um die Darstellung von $R(z)$ zu erhalten, betrachten Sie
das übliche Modellproblem mit $h = 1$(dies impliziert $z = \lambda$). Aus der
Definition der Kollokationspolynome $q \in \Pi_m$ schließen Sie nun, dass
\begin{align*}\label{a}
  q^{\prime}(x) - zq(x) = KM(x)
\end{align*}
für eine Konstante $K \neq 0$. Leiten Sie Gleichung \eqref{a} $s=0,\dots,m$
mal ab, um einen Ausdruck für $q(x)$ zu erhalten. Schließlich gilt
$R(z) = \nicefrac{q(1)}{q(0)}$.
\end{exercise}
\begin{solution}
Die Kollokationspolynome $q_l \in \Pi_m$ werden durch
\begin{align*}
  q_l(t_l) &= y_l, \\
  q_l^{\prime}(t_l + c_jh_l) = f(t_l + c_jh_l, q_l(t_l + c_jh_l)), \qquad j = 1,\dots,m \\
  y_{l+1} &:= q_l(t_l + h_l)
\end{align*}
eindeutig festgelegt. Wir wählen nun $f(y) = \lambda y, t_0 = 0, h_0 = 1$
und erhalten also
\begin{align*}
  q(0) &= y_0 \\
  q^{\prime}(c_j) &= \lambda q(c_j), \qquad j = 1,\dots,m \\
  y_1 := q(1).
\end{align*}
Aus der Definition der Stabilitätsfunktion folgt
\begin{align*}
  y_1 = R(\lambda)y_0 \implies  R(z) = R(\lambda) = \frac{y_1}{y_0} = \frac{q(1)}{q(0)}.
\end{align*}
Als nächstes zeigen wir
\begin{align*}
  q^{\prime}(x) - zq(x) - KM(x) \equiv 0
\end{align*}
für geeignetes $K$. Dafür bemerken wir, dass für $c_j, j=1,\dots,m$
\begin{align*}
  q^{\prime}(c_j) - zq(c_j) - KM(c_j) = zq(c_j) - zq(c_j) = 0
\end{align*}
gilt. Nun wähle $x_0: q^{\prime}(x_0) - zq(x_0) \neq 0$ beliebig. Dies ist sicher möglich,
da anderenfalls $\forall x: q^{\prime}(x) = zq(x)$ und somit $q(x) = C\exp(zx)$ folgen würde.
Nun setze
\begin{align*}
  K := q^{\prime}(x_0) - zq(x_0)
\end{align*}
und es folgt
\begin{align*}
  q^{\prime}(x_0) - zq(x_0) - KM(x_0) = 0.
\end{align*}
Damit hat das Polynom $q^{\prime} - zq - KM \in \Pi_m$ insgesamt $m+1$ Nullstellen
und muss damit das Nullpolynom sein.
Jetzt können wir $KM(x), s = 1,\dots,m$ Mal ableiten
und erhalten für $q(x) = \sum_{k=1}^m a_kx^k$
\begin{align*}
  \partial_x^s KM(x) &= \partial_x^s (q^{\prime}(x) + zq(x))
  = \partial_x^{s+1}q(x) - z\partial_x^s q(x)
  = \sum_{k=0}^{m-s-1}a_{k+s+1}\frac{(k+s+1)!}{k!}x^k - z\sum_{k=0}^{m-s}a_{k+s}\frac{(k+s)!}{k!}x^k.
\end{align*}
Nun setzen wir für $x = 1$ ein.
\begin{align*}
  KM^{(s)}(1) &= \sum_{k=0}^{m-s-1}a_{k+s+1}\frac{(k+s+1)!}{k!}
  - z\underbrace{\sum_{k=0}^{m-s}\frac{(k+s)!}{k!}}_{=: b_s} \\
  KP(z) &= \sum_{s=0}^m KM^{(s)}(1)z^{m-s} = \sum_{s=0}^m (b_{s+1} - zb_s)z^{m-s} \\
  &= \sum_{s=0}^m z^{m-s}b_{s+1} - z^{m-s+1}b_s
  = -z^{m+1}b_0 + b_{m+1}
  = -z^{m+1}\sum_{i=1}^m a_i
  = -z^{m+1}q(1)
\end{align*}
Für $x=0$ erhalten wir wiederum
\begin{align*}
  KM^{(s)}(0) = \begin{cases}
    a_{s+1}(s+1)! - za_ss, & s < m \\
    -za_mm!, & s = m
  \end{cases} \\
  KQ(z) &= K\sum_{s=0}^m M^{(s)}(0)z^{m-s} = \sum_{s=0}^{m-1}(b_{s+1}-zb_s)z^{m-s} - zb_m
  = \sum_{s=0}^{m-1}z^{m-s}b_{s+1} - z^{m-s+1}b_s - zb_m
  = -z^{m+1}b_0
  = -z^{m+1}a_0
  = -z^{m+1}q(0)
\end{align*}
Daraus folgt
\begin{align*}
  \frac{KP(z)}{KQ(z)} = \frac{-z^{m+1}q(1)}{-z^{m+1}q(0)} = \frac{q(1)}{q(0)} = R(z).
\end{align*}
\item 
\end{solution}
