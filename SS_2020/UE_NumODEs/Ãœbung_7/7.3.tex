\begin{exercise}
Zeigen Sie mithilfe des Anfangswertproblems
\begin{align*}
  y^{\prime} = f_{\epsilon}(y), \qquad y(0) = 1
\end{align*}
mit der glatten, nicht steigenden Funktion
\begin{align*}
  f_{\epsilon}(y) = \begin{cases}
    -1, & |y - 1| \leq \epsilon \\
    -y, & |y - 1| \geq 2\epsilon
  \end{cases},
\end{align*}
dass die linear-impliziten Runge-Kutta-Verfahren aus Aufgabe 20 nicht B-stabil sein können.
\end{exercise}
\begin{solution}
Lösung.
\end{solution}
