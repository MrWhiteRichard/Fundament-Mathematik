\begin{exercise}
Lösen Sie das Anfangswertproblem
\begin{align*}
  u^{\prime} &= u + v \\
  \epsilon v^{\prime} &= 2u -v
\end{align*}
mit Anfangswerten $(u(0),v(0)) = (1,4)$ mithilfe das Radau-IIA Verfahrens, eines
Gauß-Verfahrens und des RK4-Verfahrens für unterschiedliche $\epsilon > 0$.
Verwenden Sie dabei Verfahren, die vergleichbare Konvergenzordnungen besitzen.
Untersuchen Sie die Abhängigkeit des komponentenweisen Fehlers an der Stelle
$t = 0.1$ vom Parameter $\epsilon$.
\end{exercise}
\begin{solution}
Lösung.
\end{solution}
