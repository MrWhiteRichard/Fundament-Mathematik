\begin{exercise}
Betrachten Sie ein $m$-stufiges Kollokationsverfahren mit Kollokationspunkten
$c_1,\dots,c_m$. Wir definieren das Polynom
\begin{align*}
  M(x) := \frac{1}{m!}\prod_{i = 1}^m (x - c_i).
\end{align*}
\begin{enumerate}[label = \textbf{\alph*)}]
  \item Zeigen Sie, dass sich für dieses Verfahren die Stabilitätsfunktion $R(z)$
  mit $z = \lambda h$ schreiben lässt als das rationale Polynom $R(z) = P(z)/Q(z)$,
  wobei $P,Q \in \Pi_m$ gegeben sind durch
  \begin{align*}
    P(z) &= M^{(m)}(1) + M^{(m-1)}(1)z + \dots + M(1)z^m, \\
    Q(z) &= M^{(m)}(0) + M^{(m-1)}(0)z + \dots + M(0)z^m.
  \end{align*}
  \item Verwenden Sie diese explizite Darstellung von $R(z)$ um zu zeigen, dass
  Gauß-Verfahren nicht L-stabil sind.
\end{enumerate}
\textit{Hinweis (zu a):} Um die Darstellung von $R(z)$ zu erhalten, betrachten Sie
das übliche Modellproblem mit $h = 1$(dies impliziert $z = \lambda$). Aus der
Definitiond der Kollokationspolynome $q \in \Pi_m$ schließen Sie nun, dass
\begin{align*}\label{a}
  q^{\prime}(x) - zq(x) = KM(x)
\end{align*}
für eine Konstante $K \neq 0$. Leiten Sie Gleichung \eqref{a} $s=0,\dots,m$
mal ab, um einen Ausdruck für $q(x)$ zu erhalten. Schließlich gilt
$R(z) = \nicefrac{q(1)}{q(0)}$.
\end{exercise}
\begin{solution}
Lösung.
\end{solution}
