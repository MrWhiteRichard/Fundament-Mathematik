\begin{exercise}
Sei
\begin{align}
  \rho(\lambda) := (\lambda - \lambda_1)(\lambda - \lambda_2)^2(\lambda - \lambda_3)^3
\end{align}
mit paarweise verschiedenen $\lambda_1,\lambda_2,\lambda_3$.
\begin{enumerate}[label = \textbf{\alph*)}]
  \item Geben Sie alle Lösungen der zugehörigen linearen Differenzengleichung explizit an.
  \item Zeigen Sie, dass die in a) gefundenen Lösungen wirklich alle Lösungen
  der Differenzengleichung sind. Führen Sie dazu die Schritte im Beweis von
  Theorem 5.25 des Vorlesungsskriptes explizit für dieses Problem durch.
\end{enumerate}
\end{exercise}
\begin{solution}
Durch Ausmultiplizieren kann man das charakteristische Polynom auf die Form $\rho(\lambda) = \sum_{j=0}^{6}\alpha_j \lambda^j$ bringen, somit lautet die zugehörige Lineare Differenzengleichung
\begin{align}\label{diffeq}
  \sum_{j=0}^{6} \alpha_j y_{l+j}, \qquad \ell \in \N_0
\end{align}
\begin{enumerate}[label = \textbf{\alph*)}]
  \item Da die Nullstellen des charakteristischen Polynoms genau $\lambda_1,\lambda_2,\lambda_3$
  mit Vielfachheiten $1,2,3$ sind, sind
  alle Lösungen der Differenzengleichung gegeben durch Linearkombinationen der Folgen
  \begin{align*}
    (\lambda_{l}^{(n,m)})_{l \in \N_0} \text{~~mit~~} n = 1,2,3, \qquad m = 0,...,n-1
  \end{align*}
  die wie folgt definiert sind:
  \begin{align*}
    \lambda_{l}^{(1,0)} =& \lambda_1^l \\
    \lambda_{l}^{(2,0)} =& \lambda_2^l \\
    \lambda_{l}^{(3,0)} =& \lambda_3^l \\
    \lambda_{l}^{(2,1)} =& \begin{cases}
0 & l = 0 \\
l \lambda_2^{l-1} & \, l > 0
\end{cases} \\
    \lambda_{l}^{(3,1)} =& \begin{cases}
0 & l = 0 \\
l \lambda_3^{l-1} & \, l > 0
\end{cases} \\
    \lambda_{l}^{(3,2)} =& \begin{cases}
0 & l \leq 1 \\
l(l-1) \lambda_3^{l-2} & \, l > 1
\end{cases} \\
  \end{align*}

  \item Wir zeigen zunächst, dass es sich bei den Folgen tatsächlich um Lösungen der linearen Differenzengleichung handelt, indem wir sie in \eqref{diffeq} einsetzen. Das machen wir für jedes $m=1,...,3$ jeweils nur einmal exemplarisch.

  \begin{align*}
    \sum_{j=0}^{k} \alpha_j \lambda_{l+j}^{(1,0)} =& \sum_{j=0}^{k} \alpha_j \lambda_1^{l+j} = \lambda_1^l \rho(\lambda_1) = 0 \\
    \sum_{j=0}^{k} \alpha_j \lambda_{l+j}^{(2,1)} =& \sum_{j=0}^{k} \alpha_j (l+j)\lambda_2^{l+j-1} = \lambda_2^l\rho^{(1)}(\lambda_2) = 0 \\
    \sum_{j=0}^{k} \alpha_j \lambda_{l+j}^{(3,2)} =& \sum_{j=0}^{k} \alpha_j (l+j)(l+j-1)\lambda_3^{l+j-2} = \lambda_3^l\rho^{(2)}(\lambda_3) = 0.
  \end{align*}


  Nachdem wir gezeigt haben, dass es sich um Lösungen handelt, stellen wir fest, dass es $k=6$ verschiedene Lösungen sind (den Fundamentalsatz brauchen wir hier also gar nicht). Nun wollen wir noch zeigen, dass die Lösungen linear unabhängig sind und somit tatsächlich den $k$-dimensionalen Lösungsraum aufspannen.

  Lineare Unabhängigkeit zeigen wir, indem wir annehmen
  \begin{align*}
    \Forall \ell \in \N_0: \sum_{n=1}^3 \sum_{m=0}^{n-1} \alpha_{n,m} \lambda_{\ell}^{(n,m)} = 0 .
  \end{align*}
  Für $\ell=0,...,k-1$ führt das zu folgendem linearen Gleichungssystem
  \begin{align*}
    \underbrace{\begin{pmatrix}
      \lambda_1^0 & \lambda_2^0 & 0 & \lambda_3^0 & 0 & 0\\
      \lambda_1^1 & \lambda_2^1 & \lambda_2^0 & \lambda_3^1 & \lambda_3^0 & 0\\
      \lambda_1^2 & \lambda_2^2 & 2\lambda_2^1 & \lambda_3^2 & 2\lambda_3^1 & 2\lambda_3^0\\
      \lambda_1^3 & \lambda_2^3 & 3\lambda_2^2 & \lambda_3^3 & 3\lambda_3^2 & 6\lambda_3^1\\
      \lambda_1^4 & \lambda_2^4 & 4\lambda_2^3 & \lambda_3^4 & 4\lambda_3^3 & 12\lambda_3^2\\
      \lambda_1^5 & \lambda_2^5 & 5\lambda_2^4 & \lambda_3^5 & 5\lambda_3^4 & 20\lambda_3^3
    \end{pmatrix}}_{:=A}
    \begin{pmatrix}
      \alpha_{1,0} \\ \alpha_{2,0} \\ \alpha_{2,1} \\ \alpha_{3,0}  \\ \alpha_{3,1}  \\ \alpha_{3,2}
    \end{pmatrix} =
    \begin{pmatrix}
      0 \\ 0 \\ 0 \\ 0 \\ 0 \\ 0
    \end{pmatrix}.
  \end{align*}
  Die Lösung für das gesuchte Polynom $p \in \P_5$ ($p(x) = \sum_{j=0}^5 a_j x^j$) einer Hermite-Interpolation mit den Stützstellen $\lambda_1, \lambda_2, \lambda_2, \lambda_3, \lambda_3, \lambda_3$ lässt sich durch folgendes Gleichungssystem bestimmen
  \begin{align*}
    A^T  \begin{pmatrix}
      a_0 \\ a_1 \\ a_2 \\ a_3 \\ a_4 \\ a_5
    \end{pmatrix} =
    \begin{pmatrix}
      p(\lambda_1) \\ p(\lambda_2) \\ p^{(1)}(\lambda_2) \\ p(\lambda_3) \\ p^{(1)}(\lambda_3) \\ p^{(2)}(\lambda_3)
    \end{pmatrix}
  \end{align*}

  Da dieses Problem eindeutig lösbar ist, ist $A^T$ regulär und somit auch $A$ und es müssen alle Koeffizienten $\alpha_{n,m} = 0$ sein. Also bekommen wir lineare Unabhängigkeit.
\end{enumerate}
\end{solution}
