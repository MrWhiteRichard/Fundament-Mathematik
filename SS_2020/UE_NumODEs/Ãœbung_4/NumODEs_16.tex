\begin{exercise}
Wenden Sie die Methode der (Zweischritt-)Richardson Extrapolation aus Abschnitt
2.7 auf das explizite Euler-Verfahren an.
\begin{itemize}
  \item [\textbf{a)}] Welches Verfahren entsteht dabei?
  \item [\textbf{b)}] Weisen Sie unabhängig von Abschnitt 2.7 nach, dass dieses
  Verfahren die Konvergenzordnung 2 besitzt.
\end{itemize}
\end{exercise}
\begin{solution}
\leavevmode \\
\begin{itemize}
  \item [\textbf{a)}] Das explizite Euler-Verfahren stellt mit
  $\Phi(t,y,h) = f(t,y)$ ein stabilies, explizites Einschrittverfahren
  mit Konvergenzordnung 1 dar.
  Für hinreichend glattes $f$ erhalten wir nach Taylor die asymptotische Entwicklung
  \begin{align*}
    z(t+h) = z(t) + hf(t,z(t)) + \frac{h^2}{2}y^{\primeprime} + \Landau{h^3}
  \end{align*}
  Ergo können wir die Richardson-Extrapolation für $p = 1$ anwenden und erhalten damit
  \begin{align*}
    z_1 &= z(t) + hf(t,z(t)) \\
    \widetilde{z}_{1/2} &= z(t) + \frac{h}{2}f(t,z(t)) \\
  \widetilde{z_1} &= \widetilde{z}_{1/2} + \frac{h}{2}f(t + \frac{h}{2},\widetilde{z}_{1/2})
  = z(t) + \frac{h}{2}f(t,z(t)) + \frac{h}{2}f(t + \frac{h}{2},z(t) + \frac{h}{2}f(t,z(t))) \\
    z(t + h) &\approx 2\widetilde{z_1} - z_1
    = 2z(t) + h\left(f(t,z(t))+ f(t+\frac{h}{2},z(t)+\frac{h}{2}f(t,z(t)))\right) -
    (z(t) + hf(t,z(t))) \\
    &= z(t) + hf(t+\frac{h}{2},z(t)+\frac{h}{2}f(t,z(t)))
  \end{align*}
  die modifizierte Eulermethode.
  \item [\textbf{b)}] Nach Proposition 2.17 hat das Einschrittverfahren der Form
  \begin{align*}
    \phi(t,y,h) := b_1f(t,y) + b_2f(t+ch,y+ahf(t,y))
  \end{align*}
  genau dann Konsistenzordnung 2, wenn $a = c$ $b_1 + b_2 = 1$ und $2b_2a = 1$. \\
  In unserem Fall haben wir $b_1 = 0, b_2 = 1, a = c = 1/2$ und die Bedingungen
  werden offensichtlichlerweise erfüllt.
\end{itemize}
\end{solution}
