\begin{exercise}
  Ein explizites Runge-Kutta-Verfahren wurde in $(2.27)$ und $(2.28)$ im Skript
  definiert. Für ein implizites Runge-Kutta-Verfahren wird die Gleichung in
  $(2.28)$ ersetzt durch

  \begin{align} \label{R-K_impl}
    k_i = f \Bigg( t + c_ih, y + h\sum_{j=1}^m A_{ij} k_j
    \Bigg),\text{ } i=1,\ldots, m
  \end{align}

  Die Matrix $A$ ist dabei keine strikte untere Dreiecksmatrix mehr sondern eine
  vollbesetzte Matrix.

  Verallgemeinern Sie Theorem $2.27$ auf implizite Runge-Kutta-Verfahren. Beweisen
  Sie dazu, dass solche Verfahren in dem dortigen Sinne stabil sind.

  Hinweis: Sie dürfen annehmen, dass das implizite Runge-Kutta-Verfahren eindeutig
  durchführbar ist. Diese Frage ist nicht trivial, da $(\ref{R-K_impl})$ ein
  nichtlineares Gleichungssystem ist.
\end{exercise}

\begin{solution}
  Trivial!
\end{solution}
