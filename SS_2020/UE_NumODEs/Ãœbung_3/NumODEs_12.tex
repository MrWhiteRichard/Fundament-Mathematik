\begin{exercise}
  Sei $y$ die Lösung des Anfangswertproblems
  $y'(t) = f(t,y)$ mit $t \in \bbraces{0,T}, \text{ } y(0)=y_0$
  und beliebig glattem $f$. Weiter seien $y_i$ für $i=0,\ldots,N$ die Approximationen
  an $y(t_i)$, welche durch ein Einschrittverfahren der Ordnung $p$ mit den
  Stützstellen $t_0 = 0,\ldots,t_N =T$ berechnet werden. Sei weiter $\tilde{y}$ der
  lineare Spline mit $\tilde{y}(t_i)=y_i$ für $i=0,\ldots,N$.

  Zeigen Sie, dass eine Konstante $C>0$ unabhängig von $h_i$ und $N$ existiert mit

  \begin{align}
    \sup_{t \in \bbraces{0,T}} \vbraces{\tilde{y}(t)-y(t)}
    \leq
    C \max\Bbraces{h_0,\ldots,h_{N-1}}
  \end{align}
\end{exercise}

\begin{solution}
  Trivial!
\end{solution}
