\begin{exercise}
  Erweitern Sie das Programm zu Aufgabe $6$ um eine Schrittweitensteuerung mit
  eingebettetem RK-Verfahren (RK5(4)) und wenden Sie es auf das Räuber-Beute-
  Modell an. Vergleichen Sie die Schrittweiten mit der Lösung. \\

  Hinweis: Es gibt einen Fehler im Vorlesungsskript zum RK5(4)-Schema. Verwenden
  Sie bitte das folgende Schema.

  \begin{align*}
    \begin{array}{c|ccccccc}
    0 & & & & & & &                                                         \\
    \nfrac{1}{5} & \nfrac{1}{5} & & & & & &                                 \\
    \nfrac{3}{10} & \nfrac{3}{40} & \nfrac{9}{40} & & & & &                 \\
    \nfrac{4}{5} & \nfrac{44}{45} & -\nfrac{56}{15} & \nfrac{32}{9} & & & & \\
    \nfrac{8}{9} & \nfrac{19372}{6561} & -\nfrac{25360}{2187} &
    \nfrac{64448}{6561} & -\nfrac{212}{729} & & &                           \\
    1 & \nfrac{9017}{3168} & -\nfrac{355}{33} & \nfrac{46732}{5247} &
    \nfrac{49}{176} & -\nfrac{5103}{18656} & &                              \\
    1 & \nfrac{35}{384} & 0 & \nfrac{500}{1113} & \nfrac{125}{192} &
    -\nfrac{2187}{6784} & \nfrac{11}{84} &                                  \\
    \hline
     & \nfrac{35}{384} & 0 & \nfrac{500}{1113} & \nfrac{125}{192} &
    -\nfrac{2187}{6784} & \nfrac{11}{84} & 0                                \\
     & \nfrac{5179}{57600} & 0 & \nfrac{7571}{16695} & \nfrac{393}{640} &
     -\nfrac{92097}{339200} & \nfrac{187}{2100} & \nfrac{1}{40}
    \end{array}
  \end{align*}

\end{exercise}

\begin{solution}
  Siehe .py!
\end{solution}
