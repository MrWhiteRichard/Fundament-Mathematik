\begin{exercise}
  Seien $a,b,c \in \R$  und $y$ die Lösung des Anfangswertproblems

  \begin{align}
    y'(t)=\vbraces{a-y(t)} + b, \text{} t\geq 0,\text{} y(0) = c.
  \end{align}

  \textbf{a) }Lösen Sie das Anfangswertproblem analytisch. Welches Verhalten der Lösung
  erhalten Sie für unterschiedliche Parameter $a,b,c$? Wie glatt ist die Lösung?

  \textbf{b) }Lösen Sie das Anfangswertproblem numerisch mit expliziten Runge-Kutta-Verfahren
  unterschiedlicher Ordnung. Welche Konvergenzordnung erhalten Sie bei unterschiedlichen
  Parametern $a,b,c$? Begründen Sie Ihre Ergebnisse.

  Hinweis: Sie können das vom Übungsleiter im TUWEL zur Verfügung gestellte Programm zur
  Aufgabe 6 verwenden.
\end{exercise}

\begin{solution}
  To do!
\end{solution}
