\begin{algebraUE}{217}
Zeigen Sie, dass es überabzählbar viele Ordnungsrelationen gibt, die die Gruppe
$\Z \times \Z$ zu einer geordneten Gruppe machen. \\
\textit{Hinweis:} Betrachten Sie für irrationales $\alpha$ die Abbildung
$\varphi_{\alpha}: \Z \times \Z \to \R; \varphi_{\alpha}(a,b) := a + \alpha b$,
und übertragen Sie damit die Struktur von $\R$ als geordnete Gruppe auf $\Z \times \Z$.
\end{algebraUE}
\begin{solution}
$\varphi_{\alpha}$ induziert folgendermaßen eine Ordnung auf $\Z \times \Z$:
\begin{align*}
  (a,b) \leq_{\alpha} (c,d) :\iff a + \alpha b \leq c + \alpha d.
\end{align*}
Nun müssen wir zeigen, dass $(\Z \times \Z, \leq_{\alpha})$ zu einer geordneten Gruppe wird.
Reflexivität und Transitivität sind klar.
Für die Antisymmetrie betrachte $(a,b) \leq_{\alpha} (c,d) \leq_{\alpha} (a,b)$. Es folgt
\begin{align*}
  a + \alpha b= c + \alpha d \iff (a-c) = \alpha (d-b).
\end{align*}
Da für $(d-b)\neq 0: \alpha (d-b) \notin \Z$ muss daraus bereits $a=c, b=d$ folgen.
Die Totalität ist wiederum klar.
Zur Überprüfung des Monotoniegesetzes sei $(c_1,c_2) \in \Z \times \Z$ beliebig,
weiters seien $(a_1,a_2) \leq_{\alpha} (b_1,b_2) \in \Z \times \Z$ beliebig.
Es gilt also $a_1 + \alpha a_2 \leq b_1 + \alpha b_2$ und es folgt
\begin{align*}
  a_1 + c_1 + \alpha (a_2 + c_2) \leq b_1 + c_1 + \alpha (b_2 + c_2) \iff
  (a_1,a_2) + (c_1,c_2) \leq_{\alpha} (b_1,b_2) + (c_1,c_2).
\end{align*}
Also wird $\Z \times \Z$ mit $\leq_{\alpha}$ tatsächlich für jedes $\alpha \in \R\backslash\Q$
zu einer geordneten Gruppe. Nun gilt es noch zu überprüfen, dass durch
verschiedene $\alpha$ auch verschiedene Ordnungsrelationen induziert werden.
Seien also $\alpha_1 \neq \alpha_2 \in \R\backslash\Q$ beliebig
(o.B.d.A.:$\alpha_1 > \alpha_2$).
Da die additive Struktur auf $\R$ archimedisch angeordnet ist, existiert ein $b \in \Z$, sodass
\begin{align*}
  b(\alpha_1 - \alpha_2) > 1.
\end{align*}
Also gibt es ein $k \in \Z$, sodass
\begin{align*}
  b\alpha_1 > k > b\alpha_2.
\end{align*}
Es folgt
\begin{align*}
  (0,b) <_{\alpha_2} (k,0) \leq_{\alpha_1} (0,b)
\end{align*}
und somit werden in der Tat unterschiedliche Ordnungsrelationen festgelegt.
\end{solution}
