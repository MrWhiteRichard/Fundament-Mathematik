\begin{algebraUE}{344}
Seien $R$ ein faktorieller Ring, $f \in R[x]$ mit führendem Koeffizienten $a_n$
und konstantem Koeffizienten $a_0$ und $p,q \in R$ teilerfremd und das Element
$\frac{p}{q}$ des Quotientenkörpers $Q$ eine Nullstelle von $f$. Dann gilt $p|a_0$
und $q | a_n$.
\end{algebraUE}

\begin{solution}
\begin{align*}
  a_n \frac{p^n}{q^n} + \dots + a_0 = 0 \\
  \iff a_np^n + a_{n-1}p^{n-1}q^1 + \dots + a_1p^1q^{n-1} + a_0q^n = 0 \\
  \iff a_np^n + a_{n-1}p^{n-1}q^1 + \dots + a_1p^1q^{n-1} = -a_0q^n.
\end{align*}
Jetzt gilt $p | p(a_np^{n-1} + a_{n-1}p^{n-2}q^1 + \dots + a_1q^{n-1}) = -a_0q^n$.
Da $p,q$ teilerfremd sind folgt daraus bereits $p | a_0$.
Um das einzusehen, betrachte die eindeutige Zerlegung in irreduzible Elemente $p = p_1\cdots p_n$.
Für alle $i = 1,\dots,n$ gilt $p_i | p$ und damit $p_i \nmid q$, also folgt aus $p_i | p = -a_0q_n$
aufgrund der primen Eigenschaft von $p_i$, dass $p_i | -a_0 \iff p_i | a_0$, also insgesamt $p | a_0$.
Analog sieht man $q |q(a_{n-1}p^{n-1} + \dots + a_1p^1q^{n-2} + a_0q^{n-1}) = -a_np^n$
und damit $q | a_n$.

\end{solution}
