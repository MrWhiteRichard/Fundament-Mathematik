\begin{algebraUE}{328}
Sei $R$ ein Integritätsbereich. Der Polynomring $R[x]$ ist genau dann ein
Hauptidealring, wenn $R$ ein Körper ist. \\
\textit{Hinweis:} Betrachten Sie das von $a$ und $x$ erzeugte Ideal, wo $a \neq 0$
eine Nichteinheit von $R$ ist.
\end{algebraUE}
\begin{solution}
Sei zuerst $R$ ein Körper, $I \vartriangleleft R[x]$ ein beliebiges Ideal von $R[x]$.
Dann ist laut Proposition 5.2.3.3. $R[x]$ ein euklidischer Ring (mit der euklidischen
Bewertung $H(p) = \grad(p)$) und nach Satz 5.2.3.4.
insbesondere ein Hauptidealring. \\
Sei also $R$ kein Körper. Dann existiert ein Element $a \neq 0$, das in $R$ kein
multiplikatives Inverses besitzt. Das erzeugte Ideal von $a$ und $x$ lässt sich darstellen als
\begin{align*}
  (a,x) = \{r_1a + r_2x: r_1,r_2 \in R[x]\}.
\end{align*}
Wir zeigen nun durch einen Widerspruch, dass $(a,x)$ kein Hauptideal ist (und somit $R[x]$ kein Hauptidealring).

Wir nehmen also an, es gibt ein $p \in R[x]$ mit $(a,x) = (p) = \{rp: r \in R[x]\}$. Man sieht sofort, $p$ vom Grad $0$ sein muss, da wir sonst keine konstanten Polynome erzeugen können (also auch nicht $a$).

Aufgrund von $(a,x)=(p)$ folgt:
\begin{align*}
  a \in (p) \Rightarrow \Exists r \in R: rp = a \\
  p \in (a,x) \Rightarrow \Exists r_1 \in R: r_1 a + r_2 x = p
\end{align*}
Da $p$ ein konstantes Polynom ist, muss $r_2 = 0$ und daher $a \sim p$. Weiters gilt
\begin{align*}
  1\cdot x \in (p) \Rightarrow \Exists r \in R: rp = 1\cdot x.
\end{align*}
Da $1 = \grad(x) = \grad(rp) = \grad(r) + \grad(p) = \grad(r)$, können wir
$r(x) = ax + b$ mit $a,b \in R$ schreiben und es folgt
\begin{align*}
(ax+ b)p = 1\cdot x \implies b = 0 \implies apx = 1\cdot x \implies ap = 1
  \Rightarrow p \sim 1.
\end{align*}
Insgesamt gilt also auch $a \sim 1$, im Widerspruch zu $a$ Nichteinheit.
\end{solution}
