\begin{exercise}
  Bestimmen Sie ein Fundamentalsystem für die ODE

  \begin{align*}
    y^{(3)} - 3y' + 2y = 0
  \end{align*}

  und suchen Sie eine spezielle Lösung der ODE

  \begin{align*}
    y^{(3)} - 3y' + 2y = 9 e^t
  \end{align*}

  auf 2 Arten:  mittels Vatiation der Konstanten und mittels der Ansatzmethode.
\end{exercise}

\begin{solution}
Für die Bestimmung des Fundamentalsystems sehen wir uns das charakteristische Polynom
$\chi(\lambda)$ an. Für dieses gilt:

\begin{align*}
  \chi(\lambda) = \lambda^3 - 3\lambda + 2
\end{align*}

Wir sehen unmittelbar, dass $\lambda_1 = 1$ eine Nullstelle ist. Um die weiteren
Nullstellen zu berechnen führen wir also Polynomdivision durch. Dabei erhalten wir

\begin{align*}
  \lambda^3 - 3\lambda + 2 : (\lambda - 1) = \lambda^2 + \lambda - 2
\end{align*}

Lösen dieser quadratischen Gleichung gibt uns die weiteren Nullstellen
$\lambda_2 = 1, \lambda_3 = -2$, also ist $1$ Nullstelle mit Vielfachheit $2$.
Nach Satz 3.18 sind damit folgende Lösungen linear unabhängig und bilden damit ein
Fundamentalsystem:

\begin{align*}
t\exp^t, \exp^t, \exp^{-2t}
\end{align*}

Um jetzt die Spezielle Lösung mit der Vatiation der Konstanten zu berechnen, machen
wir das Vorgehen wie in 3.4 (iv) im Skript. Wir machen also den Ansatz:

\begin{align*}
  y_p (t)= y^1 (t) c_1 (t) + y^2 (t) c_2 (t) + y^3 (t) c_3 (t)
\end{align*}

Notatiell geschickter
\begin{align*}
  Y(t) \cdot c(t) \quad \text{mit} \quad
  Y(t) = \left(
  \begin{array}{ccc}
    y^1 & y^2 & y^3 \\
    (y^1)^\prime & (y^2)^\prime & (y^3)^\prime \\
    (y^1)^\primeprime & (y^2)^\primeprime & (y^3)^\primeprime
  \end{array}
  \right)  =
  \left(
  \begin{array}{ccc}
    \exp^t & t\exp^t & \exp^{-2t} \\
    \exp^t & t\exp^t \exp^t & -2\exp^{-2t} \\
    \exp^t & 2\exp^t + t\exp^t & 4\exp^{-2t}
  \end{array}
  \right)
\end{align*}

Dann ist

\begin{align*}
  c(t) = \int Y(t)^{-1} \cdot
  \left(
  \begin{array}{c}
    0 \\
    0 \\
    9\exp^t
  \end{array}
  \right) dt
\end{align*}

Die Inverse lassen wir uns berechnen:

\begin{align*}
  Y(t)^{-1} = \left(
  \begin{array}{ccc}
    \frac{6t+8}{9\exp^t} & \frac{-3t+2}{9\exp^t} & \frac{-3t-1}{9\exp^t} \\
    \frac{-2}{3\exp^t} & \frac{1}{3\exp^t} & \frac{1}{3\exp^t} \\
    \frac{\exp^{2t}}{9} & \frac{-2\exp^{2t}}{9} & \frac{\exp^{2t}}{9}
  \end{array}
  \right)
\end{align*}

Damit:

\begin{align*}
  \int Y(t)^{-1} \cdot
  \left(
  \begin{array}{c}
    0 \\
    0 \\
    9\exp^t
  \end{array}
  \right)dt = \left(
  \begin{array}{c}
    -\frac{3t^2}{2} - t + k_1 \\
    3t + k_2 \\
    \frac{\exp^{3t}}{3} + k_3
  \end{array}
  \right)
\end{align*}

Wobei $k_1,k_2,k_3$ die Integrationskonstanten sind. Setzen wir diese Null haben
also mit $y_p(t) = \frac{3t^2 \exp^t}{2} - t\exp^t + \frac{\exp^t}{3}$ eine
Partikulärlösung und damit eine allgemeine Lösung
\begin{align*}
  y(t) = a_1 \exp^t + a_2 t\exp^t + a_3\exp^{-2t} + \frac{3t^2 \exp^t}{2} - t\exp^t + \frac{\exp^t}{3}
\end{align*}

Wenn wir die (in diesem Fall etwas praktischere) Ansatzmethode anwenden wollen
benutzen wir erstmal Satz 3.20. Unser Polynom $q(t)=1$ ist vom Grad 0 und $\beta=1$.
Da $\beta$ eine 2-fache Nullstelle von $\chi$ ist, wissen wir, dass es eine
Konstante $c$ gibt sodass

\begin{align*}
  y_p (t) = ct^2 \exp^t
\end{align*}

Setzen wir nun in unsere Differentialgleichung ein:

\begin{align*}
  y_p^\primeprimeprime -3y_p^\prime +2y_p =
  c6\exp^t \stackrel{!}{=} 9\exp^t
  \Rightarrow c = \frac{3}{2}
\end{align*}

Damit sind Lösungen mit $a_1 ,a_2 ,a_3 \in \R$ gegeben durch:

\begin{align*}
  y(t) = a_1 \exp^t + a_2 t\exp^t + a_3\exp^{-2t} + \frac{3t^2 \exp^t}{2}
\end{align*}

\end{solution}
