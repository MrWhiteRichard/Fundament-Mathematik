\begin{exercise}
Versuchen Sie, die Lösung des AWP
\begin{align*}
  y^{\primeprime} + y + \epsilon y^3 = 0, \qquad y(0) = 1, \qquad y^{\prime}(0) = 0
\end{align*}
für kleine $\epsilon$ anzunähern. Bestimmen Sie hierzu Funktionen $t \mapsto y_0(t)$
und $t \mapsto y_1(t)$, sodass $y(t) = y_0(t) + \epsilon y_1(t) + \Landau{\epsilon^2}$ ist.
\end{exercise}
\begin{solution}
Lösung.
\end{solution}
