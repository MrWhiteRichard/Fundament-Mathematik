\begin{exercise}
Wir sagen, dass für eine ODE der Form \eqref{ode} die Funktion $\mu$ ein
\textit{integrierender} Faktor ist, falls die (äquivalente) ODE
\begin{align*}
  \mu(t,y)p(t,y)y^{\prime} + \mu(t,y)q(t,y) = 0
\end{align*}
exakt ist. Lösen Sie die ODE
\begin{align*}
  ty^{\prime} + 3t - 2y = 0,
\end{align*}
indem Sie einen integrierenden Faktor der Form $\mu(t,y) = \mu(t)$ suchen. \\
\end{exercise}
\begin{solution}
  Wir haben
  \begin{align*}
    p(t, y) = t \quad \textrm{und} \quad q(t,y) = 3t - 2y
  \end{align*}
  und definieren
  \begin{align*}
    F(t, y) := yt^{-2} - 3t^{-1} \quad \textrm{und} \quad \mu(t,y) := t^{-3}.
  \end{align*}
  Wir berechnen
  \begin{align*}
    \frac{\partial}{\partial y} F(t, y) = t^{-2} = t^{-3}t = \mu(t, y) p(t, y) \quad \textrm{und} \quad \frac{\partial}{\partial t} F(t, y) = -2yt^{-3} + 3t^{-2} = t^{-3}(3t - 2y) = \mu(t, y) q(t, y).
  \end{align*}
  Als nächstes lösen wir für beliebiges $c \in \R$
  \begin{align*}
    F(t, y) = c \Leftrightarrow yt^{-2} - 3t^{-1} = c \Leftrightarrow yt^{-2} = c + 3t \Leftrightarrow y = t^2 (c + 3t^{-1}) = ct^2 + 3t
  \end{align*}
  und haben so mit
  \begin{align*}
    y(t) := ct^2 + 3t \quad \textrm{mit Ableitung} \quad y^\prime(t) = 2ct + 3
  \end{align*}
  die Lösung der ODE gefunden. Auch die Probe
  \begin{align*}
    ty^\prime(t) + 3t - 2y = t (2ct + 3) + 3t - 2(ct^2 + 3t) = 2ct^2 + 6t - 2ct^2 -6t = 0
  \end{align*}
  bestätigt uns das.
\end{solution}
