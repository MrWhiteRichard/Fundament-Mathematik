\begin{exercise}
Wir sagen, dass für eine ODE der Form \eqref{ode} die Funktion $\mu$ ein
\textit{integrierender} Faktor ist, falls die (äquivalente) ODE
\begin{align*}
  \mu(t,y)p(t,y)y^{\prime} + \mu(t,y)q(t,y) = 0
\end{align*}
exakt ist. Lösen Sie die ODE
\begin{align*}
  ty^{\prime} + 3t - 2y = 0,
\end{align*}
indem Sie einen integrierenden Faktor der Form $\mu(t,y) = \mu(t)$ suchen. \\
\end{exercise}
\begin{solution}
  Wir haben
  \begin{align*}
    p(t, y) = t \quad \textrm{und} \quad q(t,y) = 3t - 2y
  \end{align*}

Eine ODE ist exakt, wenn $(q(t,y), p(t,y))$ ein Gradientenfeld ist. Wir kennen eine
Bedingung für (lokale) Gradientenfelder $\phi$ die lautet:

\begin{align*}
  \frac{\partial}{\partial x_i} \phi(.) e_j = \frac{\partial}{\partial x_j} \phi(.) e_i
  \quad \text{für alle} \quad i, j \in \{1,...,n\}
\end{align*}

Für exakte ODE's bedeutet  das also $\frac{\partial}{\partial t} p(t,y) =
\frac{\partial}{\partial y} q(t,y)$
Wir können Nachrechnen, dass die gegebene ODE keine exakte ist, da

  \begin{align*}
    \frac{\partial}{\partial t} p(t,y) = 1 \neq -2 = \frac{\partial}{\partial y} q(t,y)
  \end{align*}

Unsere Funkion $\mu$ soll also erfüllen:

  \begin{align*}
    \frac{\partial}{\partial y} (\mu(t)(3t-2y)) &\stackrel{!}{=}
    \frac{\partial}{\partial t} (\mu(t) t)
    \quad \Leftrightarrow \\
    -2\mu(t) &\stackrel{!}{=} \mu^\prime (t)t+ \mu(t)
   \end{align*}

Also muss unser integrierender Faktor der separablen ODE

\begin{align*}
  \mu^\prime (t) = -\frac{3}{t} \mu(t)
\end{align*}

Wir wissen wie man diese löst:

\begin{align*}
  &\ln(\mu) =\int \frac{1}{\mu} d\mu = -3 \int \frac{1}{t} = -3 \ln(t) \quad \Leftrightarrow \\
  &\mu(t) = t^{-3}
\end{align*}

Wir erhalten also die (äquivalente) exakte ODE

\begin{align*}
  t^{-2}y^\prime + 3t^{-2} -2yt^{-3} = 0
\end{align*}

Wie man diese löst haben wir beim vorherigen Bsp gesehen. Dazu setzen wir also die
Funktion $F(t,y)$ an.

\begin{align*}
  F(t,y) = \int t^{-2} dy + \varphi(t) = t^{-2} y + \varphi(t)
\end{align*}

Um nun $\varphi(t)$ zu bestimmen:

\begin{align*}
  \frac{\partial}{\partial t} F(t,y) = -2yt^{-3} + \varphi^\prime (t)
  \stackrel{!}{=} 3t^{-2} -2yt^{-3} \Rightarrow
  \varphi^\prime (t) = 3t^{-2}
\end{align*}

Also haben wir nun $F(t,y) = yt^{-2} - 3t^{-1}$.
%  und definieren
%  \begin{align*}
%    F(t, y) := yt^{-2} - 3t^{-1} \quad \textrm{und} \quad \mu(t,y) := t^{-3}.
%  \end{align*}
%  Wir berechnen
%  \begin{align*}
%    \frac{\partial}{\partial y} F(t, y) = t^{-2} = t^{-3}t = \mu(t, y) p(t, y) \quad \textrm{und} \quad \frac{\partial}{\partial t} F(t, y) = -2yt^{-3} + 3t^{-2} = t^{-3}(3t - 2y) = \mu(t, y) q(t, y).
%  \end{align*}
Als nächstes lösen wir für beliebiges $c \in \R$
  \begin{align*}
    F(t, y) = c \Leftrightarrow yt^{-2} - 3t^{-1} = c \Leftrightarrow yt^{-2} = c + 3t^{-1} \Leftrightarrow y = t^2 (c + 3t^{-1}) = ct^2 + 3t
  \end{align*}
  und haben so mit
  \begin{align*}
    y(t) := ct^2 + 3t \quad \textrm{mit Ableitung} \quad y^\prime(t) = 2ct + 3
  \end{align*}
  die Lösung der ODE gefunden. Auch die Probe
  \begin{align*}
    ty^\prime(t) + 3t - 2y = t (2ct + 3) + 3t - 2(ct^2 + 3t) = 2ct^2 + 6t - 2ct^2 -6t = 0
  \end{align*}
  bestätigt uns das.
\end{solution}
