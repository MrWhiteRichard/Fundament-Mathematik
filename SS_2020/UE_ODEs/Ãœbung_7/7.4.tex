\begin{exercise}
Betrachten Sie das System
\begin{align*}
  x^{\prime} &= y \\
  y^{\prime} &= x^2 + x.
\end{align*}
Zeigen Sie, dass $h(x,y) := y^2 - x^2 -\frac{2}{3}x^3$ eine Erhaltungsgröße ist.
Benutzen Sie diese Information, um das Phasenportrait zu skizzieren.
Was können Sie über die Stabilität der beiden Ruhelagen aussagen?
\end{exercise}
\begin{solution}
  Die Funktion $h$ ist eine Erhaltungsgröße, denn
  \begin{align*}
      \frac{\partial}{\partial t}h\prime(x(t), y(t)) = 2yy^\prime - 2xx^\prime = 2y(x^2 + x) - 2xy. -2x^2y = 0
  \end{align*}
  Wir definieren die Funktion
  \begin{align*}
    f\pbraces{(x, y)^T} := (y, x^2 + x)^T
  \end{align*}
  und bestimmen die Ruhelagen
  \begin{align*}
    f((x,y)^T) = (0, 0)^T \Leftrightarrow y = 0 \land x^2 + x = 0 \Leftrightarrow y = 0 \land x(x + 1) = 0 \Leftrightarrow y = 0 \land (x = 0 \lor x = 1)
  \end{align*}
  also haben wir zwei Ruhelagen
  \begin{align*}
    f \pbraces{
    \begin{pmatrix}
      x \\ y
    \end{pmatrix}}
    = 0 \Leftrightarrow
    \begin{pmatrix}
      x \\ y
    \end{pmatrix}
    =
    \begin{pmatrix}
      0 \\ 0
    \end{pmatrix}
    \lor
    \begin{pmatrix}
      x \\ y
    \end{pmatrix}
    =
    \begin{pmatrix}
      -1 \\ 0
    \end{pmatrix}.
  \end{align*}
  Nun wollen wir noch die Stabilität der Ruhelagen untersuchen.
  \begin{enumerate}
    \item[\glqq$(0, 0)^T$\grqq] Wir berechnen die Ableitungsmatrix von $f$
    \begin{align*}
      Df\pbraces{\begin{pmatrix}
        x \\ y
      \end{pmatrix}}
      =
      \begin{pmatrix}
        0 & 1 \\
        2x+1 & 0
      \end{pmatrix}
      \quad \textrm{,also} \quad
      Df\pbraces{\begin{pmatrix}
        0 \\ 0
      \end{pmatrix}}
      =
      \begin{pmatrix}
        0 & 1 \\
        1 & 0
      \end{pmatrix}
    \end{align*}
    Von dieser Matrix ist das charakteristische Polynom
    \begin{align*}
      \chi(\lambda) = \det \begin{pmatrix}
        -\lambda & 1 \\
        1 & -\lambda
      \end{pmatrix}
      = \lambda^2 - 1 \quad \textrm{und daher} \quad \chi(\lambda) = 0 \Leftrightarrow \lambda^2 - 1 = 0 \Leftrightarrow \lambda = \pm 1.
    \end{align*}
    Da ein Eigenwert positiv ist wissen wir nach Satz 5.8, dass es sich bei $(0, 0)^T$ um eine instabile Ruhelage handelt.
    \item[\glqq$(-1, 0)^T$\grqq] \textbf{Fehlt!!}
  \end{enumerate}
\end{solution}
