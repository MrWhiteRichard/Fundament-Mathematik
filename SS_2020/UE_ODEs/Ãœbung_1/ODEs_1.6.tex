\begin{exercise}

Das \Quote{mathematische Pendel} wird beschrieben durch die ODE
\begin{align*}
  \varphi^\primeprime(t) + \omega^2 \sin{\varphi(t)} = 0,
  \quad
  \omega = \frac{g}{\ell},
\end{align*}

wobei $g$ die Erdbeschleunigung und $\ell$ die Fadenlänge des Pendels ist. Die Funktion $\varphi: t \mapsto \varphi(t)$ beschreibt die Auslenkung des Pendels.

\begin{enumerate}[label = \textbf{\alph*)}]

  \item Schreiben Sie die ODE in ein System 1. Ordnung um. Hat Ihre neue Variable eine physikalische
  Bedeutung?

  \item Geben Sie eine Erhaltungsgröße an. Begründen Sie, warum es sich um solche handelt.

\end{enumerate}

\end{exercise}

\begin{solution}

(a)

\begin{align*}
  \psi(t) & := \varphi^\prime(t) \\
  \psi^\prime(t) + \omega^2 \sin{\varphi(t)} & = 0
\end{align*}

$\psi$ ist die Winkelgeschwindigkeit des Pendels.

\begin{figure}[h!]
  \centering
  \begin{figure}
\centering

\begin{tikzpicture}



\end{tikzpicture}

\caption{title}
\end{figure}

  \caption{Pendel}
\end{figure}

(b) Die Energie ist eine Erhaltungsgröße. Um das nachzuweisen, stellen wir folgendes Modell auf.

\begin{itemize}
  \item
  $E(t) := E_\Text{pot}(t) + E_\Text{kin}(t)
  \ldots \text{Energie des Pendels}$
  \item
  $E_\Text{pot} := m \cdot g \cdot h(t)
  \ldots \text{potentielle Energie des Pendels}$
  \item
  $E_\Text{kin} := \Frac{2}{m \cdot v(t)^2}
  \ldots \text{kinetische Energie des Pendels}$
  \item
  $m \in \R^+
  \ldots \text{Masse des Pendels}$
  \item
  $h(t) := \ell - \ell \cdot \cos{\varphi(t)} + h_0
  \ldots \text{Höhe des Pendels}$
  \item
  $h_0 \in \R^+
  \ldots \text{Höhe des Pendels im Ruhezustand (d.h. minimale Höhe)}$
  \item
  $v(t) := \psi(t) \cdot \ell
  \ldots \text{Tangentialgeschwindigkeit des Pendels}$
\end{itemize}

Die Ableitung der Funktion $E$ verschwindet.

\begin{multline*}
  E^\prime(t)
  =
  E_\Text{pot}^\prime(t) + E_\Text{kin}^\prime(t)
  =
  m g h^\prime(t) + \Frac{2}{m \pbraces{v(t)^2}^\prime}
  =
  m g \ell \sin{\varphi(t) \varphi^\prime(t)} +
  \Frac{2}{2 m v(t) v^\prime(t)} \\
  =
  m g \ell \sin{\varphi(t) \varphi^\prime(t)} +
  m \varphi^\prime(t) \ell \varphi^\primeprime(t) \ell
  =
  m g \ell \sin{\varphi(t) \varphi^\prime(t)} -
  m \varphi^\prime(t) \ell \frac{g}{\ell} \sin{\varphi(t)} \ell = 0
\end{multline*}

\end{solution}
