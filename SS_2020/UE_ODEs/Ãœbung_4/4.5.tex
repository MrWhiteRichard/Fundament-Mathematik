\begin{exercise}
Im allgemeinen gilt \textit{nicht} $e^{A + B} = e^Ae^B$. Geben Sie ein
Gegenbeispiel an.
\end{exercise}
\begin{solution}
Betrachte die Matrizen
\begin{align*}
  A &= \begin{pmatrix*}[c]
    1 & 1 \\
    0 & 0 \\
  \end{pmatrix*} \\
  B &=  \begin{pmatrix*}[c]
    1 & 0 \\
    1 & 0 \\
  \end{pmatrix*}
\end{align*}
Dann gilt
\begin{align*}
  AB &= \begin{pmatrix*}[c]
    1 & 0 \\
    0 & 0 \\
  \end{pmatrix*}\\
  BA &= \begin{pmatrix*}[c]
    1 & 1 \\
    1 & 1 \\
  \end{pmatrix*} \\
  A^n &= A \\
  B^n &= B \\
\end{align*}
und es folgt
\begin{align*}
  e^A &= \sum_{n=0}^{\infty}\frac{t^n}{n!}(A)^n = \sum_{n=0}^{\infty}\frac{t^n}{n!}A
  = \begin{pmatrix*}[c]
    \sum_{n=0}^{\infty}\frac{t^n}{n!} & \sum_{n=0}^{\infty}\frac{t^n}{n!} \\
    0 & 0 \\
  \end{pmatrix*} \\
  e^B &= \begin{pmatrix*}[c]
    \sum_{n=0}^{\infty}\frac{t^n}{n!} & 0 \\
    \sum_{n=0}^{\infty}\frac{t^n}{n!} & 0 \\
  \end{pmatrix*} \\
  e^Ae^B &= \begin{pmatrix*}[c]
    \left(\sum_{n=0}^{\infty}\frac{t^n}{n!}\right)^2 & 0 \\
    0 & 0 \\
  \end{pmatrix*} \neq
  \begin{pmatrix*}[c]
    \left(\sum_{n=0}^{\infty}\frac{t^n}{n!}\right)^2 &
    \left(\sum_{n=0}^{\infty}\frac{t^n}{n!}\right)^2 \\
    \left(\sum_{n=0}^{\infty}\frac{t^n}{n!}\right)^2 &
    \left(\sum_{n=0}^{\infty}\frac{t^n}{n!}\right)^2 \\
  \end{pmatrix*}
  = e^Be^A.
\end{align*}
\end{solution}
