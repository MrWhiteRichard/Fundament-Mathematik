\begin{exercise}
Sei $X \in C^1(J;\mathbb{R}^{d \times d})$ und
$Y \in C^1(J;\mathbb{R}^{d \times d})$.
\begin{itemize}
  \item [\textbf{a)}] Zeigen Sie die Produktregel
  \begin{align*}
    \frac{d}{dt}(XY) = \left(\frac{d}{dt}X\right)Y + X\left(\frac{d}{dt}Y\right)
  \end{align*}
  \item [\textbf{b)}] Falls $X(t)$ für jedes $t \in J$ invertierbar ist, dann ist
  die Abbildung $t \mapsto (X(t))^{-1}$ in $C^1(J;\mathbb{R}^{d \times d})$.
  Geben Sie $(X^{-1})^{\prime}$ an.
  \item [\textbf{c)}] Sei $Y$ eine Fundamentalmatrix für das lineare System
  $y^{\prime} = A(t)y$. Welche Differentialgleichung wird von $Y^{-1}$ erfüllt?
\end{itemize}
\end{exercise}
\begin{solution}
\leavevmode \\
\begin{itemize}
  \item [\textbf{a)}]
  \begin{align*}
    \frac{d}{dt}(X(t)Y(t)) &= \frac{d}{dt}\begin{pmatrix*}[c]
    \sum_{j=1}^d X_{1i}(t)Y_{i1}(t) & \cdots & \sum_{j=1}^d X_{1i}(t)Y_{id}(t) \\
    \vdots & \ddots & \vdots \\
    \sum_{j=1}^d X_{di}(t)Y_{i1}(t) & \cdots & \sum_{j=1}^d X_{di}(t)Y_{id}(t)
    \end{pmatrix*}\\
    &=
    \begin{pmatrix*}[c]
    \frac{d}{dt}\left(\sum_{j=1}^d X_{1i}(t)Y_{i1}(t)\right) & \cdots &
    \frac{d}{dt}\left(\sum_{j=1}^d X_{1i}(t)Y_{id}(t)\right) \\
    \vdots & \ddots & \vdots \\
    \frac{d}{dt}\left(\sum_{j=1}^d X_{di}(t)Y_{i1}(t)\right) & \cdots &
    \frac{d}{dt}\left(\sum_{j=1}^d X_{di}(t)Y_{id}(t)\right)
    \end{pmatrix*} \\
    &= \begin{pmatrix*}[c]
    \sum_{j=1}^d \left(\frac{d}{dt}X_{1i}(t)\right)Y_{i1}(t) +
    \left(\frac{d}{dt}Y_{i1}(t)\right)X_{1i}(t) & \cdots &
    \sum_{j=1}^d \left(\frac{d}{dt}X_{1i}(t)\right)Y_{id}(t) +
    \left(\frac{d}{dt}Y_{id}(t)\right)X_{1i}(t) \\
    \vdots & \ddots & \vdots \\
    \sum_{j=1}^d \left(\frac{d}{dt}X_{di}(t)\right)Y_{i1}(t) +
    \left(\frac{d}{dt}Y_{i1}(t)\right)X_{di}(t) & \cdots &
    \sum_{j=1}^d \left(\frac{d}{dt}X_{di}(t)\right)Y_{id}(t) +
    \left(\frac{d}{dt}Y_{id}(t)\right)X_{1d}(t)
    \end{pmatrix*} \\
    &= \left(\frac{d}{dt}X(t)\right)Y(t) + X(t)\left(\frac{d}{dt}Y(t)\right)
  \end{align*}
  \item [\textbf{b)}]
  \begin{align*}
    X^{-1}(t) = \frac{1}{\det X(t)} X^{\#}(t),
  \end{align*}
wobei $A^{\#}$ die Kofaktor-Matrix bezeichnet mit $(A^{\#})_{lk} = \det A_{kl}$ mit
\begin{align*}
A_{kl} =
  \begin{pmatrix*}[c]
  a_{11} & \dots & 0 & \dots & a_{1n} \\
  \vdots & \ddots & \vdots & \iddots & \vdots \\
  0 & \dots & 1 & \dots & 0 \\
  \vdots & \iddots & \vdots & \ddots & \vdots \\
  a_{n1} & \dots & 0 & \dots & a_{nn}
  \end{pmatrix*}
\end{align*}
Die Determinantenbildung ist von $\mathbb{R}^{d \times d} \rightarrow \mathbb{R}$
eine Polynomfunktion und somit stetig differenzierbar.
Die Funktion $t \mapsto X^{\#}(t)$ ist also komponentenweise stetig differenzierbar
und damit auch insgesamt stetig differenzierbar.
Damit ist $X^{-1}$ als Verknüpfung und Multiplikation stetig differenzierbarer Funktionen ebenso
stetig differenzierbar. \\
Mithilfe der eben bewiesenen Produktregel können wir nun
$(X^{-1})^{\prime}$ berechnen.
\begin{align*}
  0 = I^{\prime} = (X X^{-1})^{\prime} = X^{\prime}(X^{-1}) + X(X^{-1})^{\prime}
\end{align*}
Es folgt
\begin{align*}
  (X^{-1})^{\prime} = -X^{-1}X^{\prime}X^{-1}
\end{align*}
\item [\textbf{c)}]
\begin{align*}
  (Y^{-1})^{\prime} = -Y^{-1}Y^{\prime}Y^{-1} = -Y^{-1}A(t)YY^{-1} = -Y^{-1}A(t)
\end{align*}
$Y^{-1}$ erfüllt also die Differentialgleichung
\begin{align*}
  y^{\prime}(t) = -A^{\top}(t)y(t)
\end{align*}
\end{itemize}

\end{solution}
