\begin{exercise}
    Sei $Y \in C^1(J; \R^{d \times d})$ eine Fundamentalmatrix für das lineare System $y^\prime = A(t) y$. Zeigen Sie
    \begin{enumerate}[label = \alph*)]
        \item Die Matrixfunktion $X \in C^1(J, \R^{d \times d})$ ist genau dann eine Fundamentalmatrix, wenn es eine reguläre Matrix $B \in \R^{d \times d}$ so gibt, dass für alle $t \in J$ die Gleichheit $X(t) = Y(t)B$ erfüllt ist.
        \item Die Matrix $X(t) := Y(t) \pbraces{Y(t_0)}^{-1}$ ist eine Hauptfundamentalmatrix.
    \end{enumerate}
\end{exercise}

\begin{solution}
    Hier könnte Ihre Werbung stehen!
    \begin{enumerate}[label = \alph*)]
        \item \label{fundam} Wir zeigen zwei Inklusionen. Anmerkung: Ich habe mir die folgende Aufgabe 4.2 nicht angeschaut bevor ich die geschrieben habe, bin aber im Nachhinein draufgekommen, dass das hilft!
        \begin{enumerate}
            \item[``$\Rightarrow$''] Es sei also $X$ eine Fundamentalmatrix. Wir wissen, dass für alle $t \in J$ die Matrix $Y(t)$ regulär ist und dürfen deshalb $B(t) := (Y(t))^{-1} X(t) $ definieren und da $X(t)$ ebenfalls regulär ist, wissen wir auch, dass $B(t)$ regulär ist. Für eine reguläre Matrix $A \in \R^{n \times n}$ ist $\det A = \sum_{\sigma \in S_n} \sgn \sigma \prod_{k = 1}^n a_{\sigma(k), k}$ (siehe LinAlg Buch Kapitel 7) differenzierbar und wegen $A^{-1} = \frac{1}{\det(A)} adj A$ ist auch die Inversenbildung differenzierbar. Da $Y$ differenzierbar ist gilt das auch für $B$ und wir erhalten
            \begin{align*}
                X^\prime(t) &= Y^\prime(t) B(t) + Y(t) B^\prime(t) = A(t) Y(t) B(t) + Y(t) B^\prime(t) \\
                &= A(t) X(t) + Y(t) B^\prime(t) = X^\prime(t) + Y(t) B^\prime(t)
            \end{align*} 
            und daraus folgt $Y(t) B^\prime(t) = 0$ und da $Y(t)$ regulär ist folgt $B^\prime(t) = 0$ also ist $B(t)$ konstant.
            \item[``$\Leftarrow$''] Sei also $B \in \R^{d \times d}$ regulär und für alle $t \in J$ sei $X(t) = Y(t)B$. Es gilt für alle $t \in J$
            \begin{align*}
                X^\prime(t) = Y^\prime(t) B = A(t) Y(t) B = A(t) X(t)
            \end{align*}  
            also ist $X$ eine Lösungsmatrix für das gegebene lineare System. Nach Korollar 3.9. gilt für alle $t \in J$
            \begin{align*}
                \det(X(t)) = \det(Y(t)B) = \det(Y(t)) \det(B) \neq 0
            \end{align*}
            und damit ist $X$ eine Fundamentalmatrix.
        \end{enumerate}
        \item Aus der linearen Algebra wissen wir, dass $(Y(t_0))^{-1}$ regulär ist und aus \ref{fundam} wissen wir bereits, dass $X(t) = Y(t) (Y(t_0))^{-1}$ eine Fundamentalmatrix ist. Es gilt $X(t_0) = I$ also ist $X$ eine Hauptfundamentalmatrix.
    \end{enumerate}
\end{solution}