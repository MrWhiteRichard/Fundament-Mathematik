\begin{exercise}
  Sei $I$ ein abgeschlossenes (endliches) Intervall, $L \in \R$.
  Versehen Sie den Raum $C(I;\R^d)$ der stetigen Funktionen auf $I$ mit der Norm
  \begin{align*}
    \norm[X]{z} := \max_{t\in I}e^{-2Lt}\norm[\R^d]{z(t)}
  \end{align*}
  Zeigen Sie, dass dieser Raum ein Banachraum ist (Sie dürfen verwenden, dass der Raum
  $C(I;\R^d)$ versehen mit der üblichen Norm ein Banachraum ist). Geben Sie einen Teilraum
  von $C(\R;\R^d)$ (und eine Norm) an, so dass ein Banachraum entsteht.
\end{exercise}

\begin{solution}
  Wir zeigen, dass $\norm[X]{z}$ und $\norm[\infty]{z}$ über dem Vektorraum $C(I;\R^d)$
  äquivalent sind. Sei dazu $I:=[a,b]$ und weiteres o.B.d.A $L \leq 0$, dann gilt für bel.
  $z \in C(I;\R^{d})$:

  \begin{align*}
    e^{-2La}\norm[\infty]{z}
    \leq
    \norm[X]{z}
    \leq
    e^{-2Lb}\norm[\infty]{z}
  \end{align*}

  Aufgrund der Äquivalenz der Normen konvergiert eine Cauchyfolge genau dann in einer der beiden,
  wenn sie es in der anderen tut. Da $C(I;\R^d)$ versehen mit $\norm[\infty]{\cdot}$ schon ein Banachraum ist,
  folgt daraus dass er auch mit $\norm[X]{\cdot}$ ein Banachraum ist. \\

  Für den zweiten Teil betrachten wir den Raum $C_{b}(\R,\R^d)$ versehen mit $\norm[\infty]{\cdot}$.
  Dass dieser Raum wirklich einen Unterraum bildet, lässt sich elementar nachrechnen.
  Dazu betrachten wir die Unterraumkriterien:

  \begin{align}
    U \neq \emptyset
  \end{align}

  Da die Nullfunktion klarerweise beschränkt ist ist das erfüllt.

  \begin{align}
    \forall a,b \in U, x \in \R \Rightarrow a+xb \in U
  \end{align}

  Das ist sicherlich auch erfüllt, da für festes $x \in \R$ natürlich auch
  $a+xb$ beschränkt ist.

  Damit haben wir unseren vollständigen Unterraum gefunden.


\end{solution}
