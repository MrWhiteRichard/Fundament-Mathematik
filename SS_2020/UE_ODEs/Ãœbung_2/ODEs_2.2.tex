\begin{exercise}
    Hier könnte Ihre Werbung stehen!
    \begin{enumerate}[label = \alph*)]
        \item Wenden Sie die Picardsche Fixpunktiteration mit Startfunktion $y_0 = 0$ auf das AWP 
            \begin{align*}
                y^\prime = y, \quad y(0) = 1
            \end{align*}
            an. Gegen welche Funktion konvergiert die Iteration? Wie groß ist das Existenzintervall der Lösung?

        \item Wenden Sie die Picardsche Fixpunktiteration mit Startfunktion $y_0 = 0$ auf das AWP 
        \begin{align*}
            y^\prime(t) = 2t - 2\sqrt{\max{0, y(t)}}, \quad y(0) = 0
        \end{align*}
        an. Was beobachten Sie?
    \end{enumerate}
\end{exercise}

\begin{solution}
    Hier könnte Ihre Werbung stehen!
    \begin{enumerate}[label = \alph*)]
        \item Die erste Behauptung ist
        \begin{align*}
        y_n(t) = \sum_{k = 0}^{n-1} \frac{t^k}{k!} .
        \end{align*}
        Man sieht unmittelbar, dass die Behauptung für $y_0$ richtig ist. Sei sie nun für $y_n$ richtig.
        Dann gilt
        \begin{align*}
            y_{n+1}(t) &= 1 + \int_0^t \sum_{k = 0}^{n-1} \frac{\tau^k}{k!} d\tau = 1 +  \sum_{k = 0}^{n-1}\frac{1}{k!} \int_0^t \tau^k d\tau = 1 + \sum_{k = 0}^{n-1} \frac{\tau^{k + 1}}{(k+1)!} = \sum_{k = 0}^n \frac{\tau^k}{k!}
        \end{align*}
        und wir sehen unmittlebar, dass die Folge gegen $\exp$ konvergiert. Die Funktion $\exp$ ist eine Lösung der Differentialgleichung und zwar auf ganz $\R$.

        \item Im zweiten Teil der Aufgabe gilt
        \begin{align*}
            y_n(t) = 
            \begin{cases}
                0 &, n \in 2 \pbraces{\N \cup \Bbraces{0}} \\
                t^2 &, n \in 2 \pbraces{\N \cup \Bbraces{0}} + 1
            \end{cases} .
        \end{align*}
        Für $y_0$ ist diese Behauptung offensichtlich richtig. Gelte die Behauptung nun für ein $n \in \N$.
        \begin{enumerate}[label = \roman*)]
            \item Im ersten Fall sei $n \in 2 \pbraces{\N \cup \Bbraces{0}}$. damit gilt $y_n = 0$, also 
            \begin{align*}
                y_{n+1}(t) = \int_0^t \pbraces{2\tau - 2 \sqrt{\max\Bbraces{0, y_n(\tau)}}} d\tau = \int_0^t 2\tau d\tau = t^2
            \end{align*}

            \item Im zweiten Fall ist $n \in 2 \pbraces{\N \cup \Bbraces{0}} + 1$. Dann ist $y_n(t) = t^2$ und wir erhalten
            \begin{align*}
                y_{n+1}(t) = \int_0^t \pbraces{2\tau - 2 \sqrt{\max\Bbraces{0, y_n(\tau)}}} d\tau = \int_0^t \pbraces{2\tau - 2\tau} d\tau = 0
            \end{align*}
        \end{enumerate}
        Fragt man sich nun, welche Bedingung des Satzes von Pricard-Lindelöf verletzt ist, so erkennt man, dass die Funktion $f: \R^2 \to \R: (t,y) \mapsto 2t - 2\sqrt{\max\Bbraces{0,y}}$ jedenfalls im Punkt $(t,y) = (s,0)$ für ein beliebiges $s \in \R$ nicht lipschitzstetig ist. Denn nehmen wir an es gäbe eine Umgebung $U$ von $(s,0)$ und ein $L \in \R^+$, für die für alle $(u,x), (u,z) \in U$ gilt, dass $\vbraces{f\pbraces{u,x}- f\pbraces{u,z}} \leq L\vbraces{x - z}$. Das heißt insbesondere für alle $(s,x) \in U$
        \begin{align*}
            \vbraces{f\pbraces{s,x}- f\pbraces{s,0}} \leq L\vbraces{x} \Leftrightarrow 2\sqrt{\max\Bbraces{0, x}} \leq L\vbraces{x}
        \end{align*}
        Wählen wir nun $\epsilon \in \R^+$ so klein, dass mit $z := \frac{\epsilon^2}{16L^2}$ der Punkt $(s, z) \in U$ liegt und jedenfalls $\epsilon < 4$ erfüllt ist, so gilt 
        \begin{align*}
            2\sqrt{\max\Bbraces{0, z}} = \frac{\epsilon}{2L} \leq \frac{\epsilon^2}{16 L} \Rightarrow 8 \leq 2\epsilon < 8
        \end{align*}
        und das ist ein Widerspruch, weshalb $f$ im zweiten Argument im Punkt $(s,0)$ nicht lipschitzstetig sein kann.
    \end{enumerate}
    
\end{solution}