\begin{exercise}
    Ein iteratives Verfahren um die Lösung eines Anfangswertproblems zu finden ist es beim Startpunkt $t_0$ eine Taylorreihe der Lösung $y$ zu finden.
    \begin{enumerate}[label = \alph*)]
        \item Geben Sie für die Differentialgleichung
        \begin{align*}
            y^\prime(t) = f\pbraces{t,y(t)}
        \end{align*}
        eine Formel zur Berechnung von $y^{(n)}(t_0)$ für $n \in \Bbraces{1,2,3}$ an.

        \item Geben Sie für das AWP
        \begin{align*}
            y^\prime = y^2, \quad y(0) = 1
        \end{align*}
        und $y(t) = \sum_{n = 0}^\infty y_n t^n$ eine Rekurrenz der Koeffizienten $y_n$ an. Konvergiert die Taylorreihe? Was ist das Konvergenzintervall?
    \end{enumerate}
\end{exercise}

\begin{solution}
    \begin{enumerate}[label = \alph*)]
        \item Man wendet die Kettenregel und die Produktregel an und erhält mit dem Satz von Schwarz
        \begin{align*}
            y^\prime(t_0) &= f\pbraces{t_0, y(t_0)} \\
            y^{\prime\prime}(t_0) &= \partial_t f\pbraces{t_0, y(t_0)} + \partial_y f\pbraces{t_0, y(t_0)} f\pbraces{t_0, y(t_0)} \\
            y^{(3)}(t_0) &= \partial_t^2 f\pbraces{t_0, y(t_0)} + 2 \partial_y \partial_t f\pbraces{t_0,y(t_0)} f\pbraces{t_0, y(t_0)} + \partial_y f\pbraces{t_0, y(t_0)} \partial_t f\pbraces{t_0,y(t_0)} \\
            & \quad + \pbraces{f\pbraces{t_0, y(t_0)}}^T \partial_y^2 f(t_0, y(t_0)) f(t_0, y(t_0)) + \pbraces{\partial_y f\pbraces{t_0, y(t_0)}}^2 f\pbraces{t_0, y(t_0)} .
        \end{align*}

        \item Mit dem Startwert erhält man unmittelbar $y(0) = y_0 = 1$. Sei nun für alle $k < n$ der Koeffizient $y_k = 1$. Nun wollen wir zeigen, dass dann auch $y_n = 1$ gilt. In jedem Fall gilt
        \begin{align*}
            \sum_{j = 0}^\infty (j+1) y_{j + 1} t^j = \pbraces{\sum_{l = 0}^\infty y_l t^l}^2 = \sum_{m = 0}^\infty t^m \sum_{l = 0}^m y_l y_{m-l} .
        \end{align*}
        Durch Koeffizientenvergleich des $t^{n-1}$ Terms erhält man
        \begin{align*}
            n y_n = \sum_{l = 0}^{n - 1} y_l y_{n - 1 -l} = n ,
        \end{align*}
        also $y_n = 1$. Damit gilt für $\vbraces{t} < 1$, wobei das der Konvergenzradius der Reihe ist, $y(t) = \sum_{n = 0}^\infty t^n = \frac{1}{1 - t}$. Man rechnet auch leicht nach, dass diese Funktion tatsächlich die Differentialgleichung erfüllt.
    \end{enumerate}
\end{solution}
