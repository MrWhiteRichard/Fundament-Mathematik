\begin{exercise}
Eine (skalare) ODE der Form
\begin{align}\label{ode}
  p(t,y)y^{\prime} + q(t,y) = 0
\end{align}
heißt \textit{exakt}, falls es eine Funktion $F$ gibt, sodass
\begin{align*}
  \frac{\partial}{\partial y}F(t,y) = p(t,y), \qquad \frac{\partial}{\partial t}F(t,y) = q(t,y).
\end{align*}
\begin{enumerate}[label = \textbf{\alph*)}]
  \item Zeigen Sie: Falls $p(t_0,y_0) \neq 0$, dann kann das AWP
  \begin{align*}
    p(t,y)y^{\prime} + q(t,y) = 0, \qquad y(t_0) = y_0
  \end{align*}
  für eine exakte ODE durch Lösen der impliziten Gleichung $F(t,y(t)) = c$
  für geeignetes $c$ gelöst werden.
  \item Lösen Sie das AWP
  \begin{align*}
    (4bty + 3t + 5)y^{\prime} + 3t^2 + 8at + 2by^2 + 3y = 0, \qquad y(t_0) = y_0
  \end{align*}
\end{enumerate}
\end{exercise}
\begin{solution}
  \phantom{}
\leavevmode \\
\begin{enumerate}[label = \textbf{\alph*)}]
\item Erfülle $y: F(t,y(t)) = c$. Dann folgt nach dem Hauptsatz über implizite
Funktionen, dass
\begin{align*}
  y^{\prime} = -\left(\frac{\partial F}{\partial y}(t,y(t))\right)^{-1}
  \frac{\partial F}{\partial t}(t,y(t)) = \frac{q(t,y(t))}{p(t,y(t))}.
\end{align*}
Daraus folgt
\begin{align*}
  y^{\prime}p(t,y) + q(t,y) = 0.
\end{align*}
Also ist $y$ eine Lösung der exakten Differentialgleichung. Um die Anfangswertbedingung
zu erfüllen, wählen wir $c = F(y_0)$.
\item Wir suchen also $F$, sodass
\begin{align*}
  \frac{\partial}{\partial y}F(t,y) &= 4bty + 3t + 5 \\
  \frac{\partial}{\partial t}F(t,y) &= 3t^2 + 8at + 2by^2 + 3y
\end{align*}
gilt. Aus der ersten Gleichung folgt
\begin{align*}
  F(t,y) = \int 4bty + 3t + 5 dy + c(t) = 2bty^2 + 3ty + 5y + c(t)
\end{align*}
mit einer Funktion $c(t)$, welche nur von $t$ abhängt.
Nun leiten wir den Ausdruck nach $t$ ab, verwenden die zweite Gleichung von oben
und erhalten
\begin{align*}
  2by^2 + 3y + 5 + c^{\prime}(t) = \frac{\partial}{\partial t}F(t,y) &= 3t^2 + 8at + 2by^2 + 3y.
\end{align*}
Es folgt
\begin{align*}
  c^{\prime}(t) = 3t^2 + 8at - 5
\end{align*}
und
\begin{align*}
  c(t) = t^3 + 4at^2 - 5t.
\end{align*}
Insgesamt erhalten wir also
\begin{align*}
  F(t,y) = 2bty^2 + 3ty + 5y + t^3 + 4at^2 - 5t.
\end{align*}
Lösen wir jetzt
\begin{align*}
  F(t,y(t)) &= 2bty(t)^2 + 3ty(t) + 5y(t) + t^3 + 4at^2 - 5t \stackrel{!}{=}
  bt_0y_0^2 + 3t_0y_0 + 5y_0 + t_0^3 + 4at_0^2 - 5t_0\\
  &\iff 2bty(t)^2 + (3t + 5)y(t) + t^3 + 4at^2 - 5t -
  (bt_0y_0^2 + 3t_0y_0 + 5y_0 + t_0^3 + 4at_0^2 - 5t_0) \stackrel{!}{=} 0
\end{align*}
Durch die quadratische Lösungsformel kann man nun die Lösung berechnen.
\end{enumerate}
\end{solution}
