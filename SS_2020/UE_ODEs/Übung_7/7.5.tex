\begin{exercise}
Eine (skalare) ODE der Form
\begin{align}\label{ode}
  p(t,y)y^{\prime} + q(t,y) = 0
\end{align}
heißt \textit{exakt}, falls es eine Funktion $F$ gibt, sodass
\begin{align*}
  \frac{\partial}{\partial y}F(t,y) = p(t,y), \qquad \frac{\partial}{\partial t}F(t,y) = q(t,y).
\end{align*}
\begin{enumerate}[label = \textbf{\alph*)}]
  \item Zeigen Sie: Falls $p(t_0,y_0) \neq 0$, dann kann das AWP
  \begin{align*}
    p(t,y)y^{\prime} + q(t,y) = 0, \qquad y(t_0) = y_0
  \end{align*}
  für eine exakte ODE durch Lösen der impliziten Gleichung $F(t,y(t)) = c$
  für geeignetes $c$ gelöst werden.
  \item Lösen Sie das AWP
  \begin{align*}
    (4bty + 3t + 5)y^{\prime} + 3t^2 + 8at + 2by^2 + 3y = 0, \qquad y(t_0) = y_0
  \end{align*}
\end{enumerate}
\end{exercise}
\begin{solution}
  \phantom{}
\begin{enumerate}[label = \textbf{\alph*)}]
\item
\end{enumerate}
\end{solution}
