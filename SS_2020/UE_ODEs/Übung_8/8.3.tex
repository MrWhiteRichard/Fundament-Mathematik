\begin{exercise}
Eine skalare ODE der Form
\begin{align*}
  y^{\prime} = g(t)y + h(t)y^2 + k(t)
\end{align*}
heißt Riccatigleichung. Sei $y_1$ eine Lösung dieser Gleichung.
\begin{enumerate}[label = \textbf{\alph*)}]
  \item Überprüfen Sie, dass jede Lösung $x$ der Bernoullischen ODE
  \begin{align*}
    x^{\prime} = (g(t) + 2y_1(t)h(t))y + h(t)x^2
  \end{align*}
  eine Lösung $y = y_1 + x$ der Riccatischen Gleichung erzeugt.
  \item Geben Sie die allgemeine Lösung der ODE
  \begin{align*}
    y^{\prime} = 3\left(2(t+1)^2 - \frac{1}{t+1}\right)y - 3(t+1)y^2 - 3(t+1)^3 + 4
  \end{align*}
  an. \\
  \textit{Hinweis:} Versuchen Sie ein lineares Polynom als spezielle Lösung.
\end{enumerate}
\end{exercise}
\begin{solution}
Lösung.
\end{solution}
