\begin{exercise}
Ziel dieser Aufgabe ist eine Beziehung zwischen der linearisierten Stabilität und
dem Konzept der Ljapunovfunktion. Genauer: Wir zeigen die Äquivalenz der folgenden
beiden Aussagen für eine Matrix $A \in R^{d \times d}$:
\begin{enumerate}
  \item Die Ruhelage $y^* = 0$ der ODE $y^{\prime} = Ay$ ist asymptotisch stabil.
  \item Es gibt eine symmetrisch positiv definite Lösung der Matrixgleichung
  (\glqq Ljapunovgleichung\grqq)
  \begin{align} \label{ljapunov}
    A^{\top}Q + QA = -I.
  \end{align}
\end{enumerate}
\begin{enumerate}[label = \textbf{\alph*)}]
  \item Sei die Ruhelage $y^* = 0$ asymptotisch stabil. Definieren Sie die Matrix
  \begin{align*}
    Q := \int_{t = 0}^{\infty} \exp(tA)^{\top}\exp(tA) dt
  \end{align*}
  Zeigen Sie: $Q$ ist symmetrisch positiv definit (insbesondere also $(Qx,x)_2 > 0$
  für $x \neq 0$) und erfüllt \eqref{ljapunov}. Wie Teilaufgabe b) zeigen wird,
  ist die Funktion $V(x) := (Qx,x)_2$ eine strikte Ljapunovfunktion für die obige ODE.
  \item Sei $Q$ eine symmetrisch positiv definite Lösung von \eqref{ljapunov}.
  Zeigen Sie: $V(x) := (Qx,x)_2$ ist eine strikte Ljapunovfunktion für die obige ODE.
  Zeigen Sie: Die Ruhelage $y^* = 0$ ist asymptotisch stabil.
\end{enumerate}
\end{exercise}
\begin{solution}
\begin{enumerate}[label = \textbf{\alph*)}]
  \item Wir zeigen zuerst die Symmetrie:
  \begin{align*}
    Q^{\top} = \left(\int_{t = 0}^{\infty} \exp(tA)^{\top}\exp(tA) dt\right)^{\top}
    = \int_{t = 0}^{\infty} \left(\exp(tA)^{\top}\exp(tA)\right)^{\top} dt
    = \int_{t = 0}^{\infty} \exp(tA)^{\top}\exp(tA) dt = Q.
  \end{align*}
  Für $x \neq 0$ folgt
  \begin{align*}
    x^{\top}Qx &= x^{\top}\int_{t = 0}^{\infty} \exp(tA)^{\top}\exp(tA) dt~x
    = \int_{t = 0}^{\infty} x^{\top}\exp(tA)^{\top}\exp(tA) x~ dt
    = \int_{t = 0}^{\infty} (\exp(tA)x)^{\top}(\exp(tA)x) dt \\
    &= \int_{t = 0}^{\infty} (\exp(tA)x,\exp(tA)x)_2 dt > 0
  \end{align*}
  auch die positive Definitheit. \\
  Aber die entscheidende Frage fehlt noch! Warum existiert das Integral?
  Es wird wohl damit zusammenhängen, dass aufgrund der asymptotischen Stabilität
  von $y^* = 0$ alle Eigenwerte von $A$ negativen Realteil haben müssen.
  Bezeichne also mit $\lambda_1$ den Eigenwert mit größtem Realteil.
  Weiters sei $\omega := \frac{\lambda_1}{2} > s(A) := \max \{\Re(\lambda): \lambda \in \sigma(A)\}$
  Daher gilt mit Aufgabe 8.1 die Abschätzung:
  \begin{align*}
    \|\exp(tA)\|_{\infty} &\leq  M\exp(\omega t)
  \end{align*}
  mit einer Konstante $M \in \R.$
  Analoges gilt für $\|\exp(tA)^{\top}\|$, da die Eigenwerte gleich bleiben. Insgesamt erhalten wir also
  \begin{align*}
    \|\exp(tA)^{\top}\exp(tA)\| \leq \widetilde{M}\exp(\lambda_1 t)
  \end{align*}
  und damit die Existenz des Integrals, da die Abschätzung der Matrixnorm auch
  eine Abschätzung jeder Komponente ist.
  \begin{align*}
    A^{\top}Q + QA &= \int_{t = 0}^{\infty} A^{\top}\exp(tA)^{\top}\exp(A) + \exp(tA)^{\top}\exp(tA)Adt \\
    &= \int_{t = 0}^{\infty} \frac{d}{dt}(\exp(tA)^{\top}\exp(tA)) dt
    = \lim_{t \to \infty} \exp(tA)^{\top}\exp(tA) - I
  \end{align*}
  Aus Satz 5.6 folgt aufgrund der Tatsache, dass $\exp(tA)$ ein Fundamentalsystem
  und der Attraktivität der Ruhelage $y^* = 0$ für $y^{\prime} = Ay$
  ist, dass $\lim_{t \to \infty}\|\exp(tA)\| = 0$ und damit
  \begin{align*}
    A^{\top}Q + QA = \lim_{t \to \infty} \exp(tA)^{\top}\exp(tA) - I = -I.
  \end{align*}
  \item Wir rechnen das hinreichende Kriterium für eine strikte Ljapunovfunktion nach:
  \begin{align*}
    \nabla V(y) \cdot f(y) &= \nabla (Qy,y)_2 Ay = 2y^{\top}Q Ay =
    y^{\top}Q^{\top} Ay + y^{\top}Q Ay = y^{\top}(Q^{\top}A + QA)y \\
    &\stackrel{?????}{=} y^{\top}(A^{\top}Q + QA)y = - y^{\top}y = -(y,y)_2 < 0
  \end{align*}
  Aufgrund der positiven Definitheit gilt für alle $y \neq 0$
  \begin{align*}
    V(y) = (Qy,y)_2 > 0,
  \end{align*}
  also ist $y^* = 0$ ein striktes Minimum von $V$. Wenn jetzt noch zusätzlich $y^*$
  eine isolierte Ruhelage ist, folgt daraus die asymptotische Stabilität.
  Dies ist genau dann der Fall, wenn $A$ regulär ist.
\end{enumerate}
\end{solution}
