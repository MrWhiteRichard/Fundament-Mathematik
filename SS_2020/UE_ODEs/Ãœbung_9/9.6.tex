\begin{exercise}
Betrachten Sie die folgende Verallgemeinerung eines mathematischen Pendels (ohne Reibung):
\begin{align*}
  y^{\primeprime} + g(y) = 0,
\end{align*}
wobei die Funktion $g$ auf $(-a,a)$ definiert ist und $g(0) = 0, g(x) > 0$ für $x > 0$
und $g(x) < 0$ für $x < 0$ erfüllt ($xg(x) > 0$ für $x \neq 0$). Überführen Sie
diese ODE 2.Ordnung in ein System erster Ordnung. Geben Sie eine Ljapunovfunktion
an. Zeigen Sie, dass $(y,y^{\prime}) = (0,0)$ eine stabile Ruhelage ist.
\end{exercise}
\begin{solution}
System erster Ordnung:
\begin{align*}
  y^{\prime} &=  z\\
  z^{\prime} &= -g(y)
\end{align*}
Unsere Ljapunovfunktion lautet
\begin{align*}
  V(y,z) = \frac{z^2}{2} + \int_0^y g(x) dx
\end{align*}
mit
\begin{align*}
  \nabla V(y,z)f(y,z) = (g(y),z)(z,-g(y)) = 0
\end{align*}
Nun betrachte
\begin{align*}
  \nabla V(0,0) = 0
\end{align*}
Weiters gilt $V(y,z) > 0$ für $(y,z) \neq (0,0)$, da
\begin{align*}
  \int_0^y g(x) dx > 0
\end{align*}
für $y \neq 0$.
Damit ist $(0,0)$ ein striktes Minimum von $V$ und damit nach der direkten Methode
von Ljapunov eine stabile Ruhelage.
\end{solution}
