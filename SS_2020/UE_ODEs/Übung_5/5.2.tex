\begin{exercise}
Sei $A \in \mathbb{R}^{d\times d}$.
\begin{itemize}
  \item [\textbf{a)}]Zeigen Sie: Ist $(v,\lambda) \in \mathbb{C}^d
  \times \mathbb{C}$ ein Eigenpaar von $A$, so sind die Funktionen
  $y_1(t) = \exp(\lambda t)v$ und $y_2(t) = \exp(\overline{\lambda}t)\overline{v}$
  Lösungen der ODE $y^{\prime} = Ay$ an.
  \item [\textbf{b)}] Nehmen Sie an, dass $A \in \mathbb{R}^
  {d \times d}$ (komplex) diagonalisierbar ist. Geben Sie ein
  \textit{reelles} Fundamentalsystem der ODE $y^{\prime} = Ay$ an.
\end{itemize}
\end{exercise}
\begin{solution}
\leavevmode \\
\begin{itemize}
  \item [\textbf{a)}]
  Mit der Rechenregel $\overline{a}\overline{b} = \overline{ab}$ und der Tatsache, dass $\overline{A} = A$
  sehen wir ein, dass
  \begin{align*}
    y_1^{\prime}(t) &= \exp(\lambda t)\lambda v = \exp(\lambda t)A v = Ay_1(t) \\
    y_2^{\prime}(t) &= \exp(\overline{\lambda}t)\overline{\lambda}\overline{v} = \exp(\overline{\lambda}t) \overline{\lambda v} =
    \exp(\overline{\lambda}t) \overline{Av} =  \exp(\overline{\lambda}t) \overline{A}\overline{v} = A\exp(\overline{\lambda}t)\overline{v} = Ay_2(t)
  \end{align*}
  gilt.
  \item [\textbf{b)}]
  Seien $(\lambda_i,\dots,\lambda_n,\overline{\lambda_1},\dots,\overline{\lambda_n},a_1,\dots,a_m)$
  mit $a_i,\dots,a_m \in \mathbb{R}$ die Eigenwerte, \\
  $(v_i,\dots,v_n,\overline{v_1},\dots,\overline{v_n},x_1,\dots,x_m)$, $x_1,\dots,x_m \in \mathbb{R}$
  die zugehörigen Eigenvektoren von $A$.
  Für jedes komplexe Eigenpaar $(\lambda_i, v_i), i = 1,\dots,d$ stellen die Funktionen
  \begin{align*}
    y_{i1}(t) &= \exp(\lambda t)v_i, \\
    y_{i2}(t) &= \exp(\overline{\lambda} t)\overline{v_i}
  \end{align*}
  Lösungen der Differentialgleichung dar. Damit sind sowohl
  \begin{align*}
    y_i(t) := y_{i1}(t) + y_{i2}(t) = \exp(\lambda t)v_i + \exp(\overline{\lambda} t)\overline{v_i}
    = \exp(\lambda t)v_i + \overline{\exp(\lambda t)}\overline{v_i}
    = \exp(\lambda t)v_i + \overline{\exp(\lambda t)v_i}
    = 2\mathfrak{R}(\exp(\lambda t)v_i),
  \end{align*}
  als auch
  \begin{align*}
    y_{n+1}(t) := i(y_{i1}(t) + y_{i2}(t)) = 2\mathfrak{I}(\exp(\lambda t)v_i)
  \end{align*}
  reelle Lösungen der Differentialgleichung.
  Für die verbleibenden reellen Eigenvektoren können wir $y_i$ einfach als
  \begin{align*}
    y_i(t) := \exp(a_i t)x_i
  \end{align*}
  definieren.
  Die Funktionen $(y_i)_{i=1}^d$ sind aufgrund der linearen Unabhängigkeit der $v_i$
  ebenso linear unabhängig und
  bilden somit ein reelles Fundamentalsystem.
\end{itemize}
\end{solution}
