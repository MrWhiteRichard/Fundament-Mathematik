\begin{exercise}
Betrachten Sie das RWP
\begin{align*}
  Ly &:= -(py^{\prime})^{\prime} + qy = f \text{ auf } (a,b), \\
  R_1y &:= \alpha_1y(a) + \alpha_2p(a)y^{\prime}(a) = \rho_1, \\
  R_2y &:= \beta_1y(b) + \beta_2p(b)y^{\prime}(b) = \rho_2,
\end{align*}
wobei $p,q$ hinreichend glatt sind, $p > 0$ auf $[a,b]$. Sei $(\alpha_1,\alpha_2) \neq (0,0)$
und $(\beta_1,\beta_2) \neq (0,0)$. Nehmen Sie an, dass das RWP für $f = 0, \rho_1,\rho_2 = 0$
nur trivial lösbar sei. Seien weiters $y_1,y_2$ zwei linear unabhängige Lösungen
von $Ly = 0$ mit $R_1y_1 = 0$ und $R_2y_2 = 0$. Dann gilt:
\begin{align*}
  \kappa(x) &:= p(x)(y_1^{\prime}y_2 - y_1y_2^{\prime}) \equiv \text{const} \neq 0 \\
  G(x,t) &:= \frac{1}{\kappa}\Bigg\{\begin{matrix}
    y_1(t)y_2(x), & a \leq t \leq x \leq b \\
    y_1(x)y_2(t), & a \leq x \leq t \leq b
  \end{matrix}
\end{align*}
\end{exercise}

\begin{solution}
Berechne
\begin{align*}
  \kappa^{\prime}(x) &= p^{\prime}(x)(y_1^{\prime}y_2 - y_1y_2^{\prime})
  + p(x)(y_1^{\prime}y_2^{\prime} + y_1^{\primeprime}y_2 - y_1^{\prime}y_2^{\prime} - y_1y_2^{\primeprime})
  = p^{\prime}(x)(y_1^{\prime}y_2 - y_1y_2^{\prime})
  + p(x)(y_1^{\primeprime}y_2  - y_1y_2^{\primeprime}) \\
  &= y_2(p^{\prime}(x)y_1^{\prime} + p(x)y_1^{\primeprime}) -
  y_1(p^{\prime}(x)y_2^{\prime} + p(x)y_2^{\primeprime})
  = y_2(py_1^{\prime})^{\prime} - y_1(py_2^{\prime})^{\prime} \\
  &= y_2qy_1 - y_1qy_2 = 0.
\end{align*}
Angenommen $\kappa(x) \equiv 0$, dann folgt aufgrund $p(x) > 0$
\begin{align*}
  y_1^{\prime}y_2 = y_1y_2^{\prime} \iff
  \frac{y_1^{\prime}}{y_1} = \frac{y_2^{\prime}}{y_2} \iff
  (\ln(y_1))^{\prime} = (\ln(y_2))^{\prime} \iff
  \ln(y_1) = \ln(y_2) + C \iff
  y_1 = y_2\exp(C)
\end{align*}
im Widerspruch zur linearen Unabhängigkeit von $y_1,y_2$. \\
Um zu zeigen, dass $G$ die Greensche Funktion aus Satz 6.7 ist, müssen wir drei
Bedingungen nachrechnen:
\begin{itemize}
  \item $LG(\cdot,t) = 0$ auf $(a,t) \cup (t,b)$: \\ Für jedes feste $t$ ist $G(\cdot,t)$
  einfach ein Vielfaches von $y_1$ oder $y_2$ und erfüllt damit sicher $LG(\cdot,t) = 0$.
  \item $G(\cdot,t)$ ist stetig bei $x = t$: \\
  Folgt einfach aus der Stetigkeit von $y_1,y_2$, da
  \begin{align*}
    \kappa G(t^-,t) = \lim_{\tau \to t^-}y_1(t)y_2(\tau) = y_1(t)y_2(t)
    = \lim_{\tau \to t^+}y_1(\tau)y_2(t) = \kappa G(t^+,t).
  \end{align*}
  \item $\partial_x G(t^+,t)- \partial_x G(t^-,t) = - \frac{1}{p(t)}$:
  \begin{align*}
    \kappa(\partial_x G(t^+,t)- \partial_x G(t^-,t)) =
    \lim_{x \to t^+}y_1^{\prime}(x)y_2(t)- \lim_{x \to t^-}y_1(t)y_2^{\prime}(x)
    = y_1^{\prime}(t)y_2(t) - y_1(t)y_2^{\prime}(t) = - \frac{\kappa}{p(t)}
  \end{align*}
  \item $R_1G(\cdot,t) = R_2G(\cdot,t) = 0$: \\
  Folgt wieder daraus, dass für festes $t: G(\cdot,t)$ ein Vielfaches einer
   der Lösungen $y_1,y_2$ ist und aufgrund der Linearität der Randbedingungen
   folgt $R_1G(\cdot,t) = R_2G(\cdot,t) = 0$.
\end{itemize}
Also wissen wir, dass die Funktion $x \mapsto \int_a^b G(x,t)f(t) dt$ die Lösung
unseres RWP mit $\rho_1 = \rho_2 = 0$ ist.
\end{solution}
