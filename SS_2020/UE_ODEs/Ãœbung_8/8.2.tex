\begin{exercise}
Betrachten Sie eine skalare ODE $y^{\prime} = f(t,y)$ mit $f: \R \times \R \to \R$
stetig und lokal Lipschitz im 2.Argument. Sei $t \mapsto y(t), t \geq t_0$
die Lösung des AWP $y(t_0) = y_0$. Es seien $t \mapsto y_1(t)$ und $t \mapsto y_2(t)$
zwei differenzierbare Funktionen $\R \to \R$, für die gilt
\begin{align*}
  y_1(t_0) &\leq y_0, \qquad y_1^{\prime} \leq f(t,y_1), \qquad t \geq t_0 \\
  y_2(t_0) &\geq y_0, \qquad y_2^{\prime} \geq f(t,y_2), \qquad t \geq t_0.
\end{align*}
\begin{enumerate}[label = \textbf{\alph*)}]
  \item Zeigen Sie, dass für $t \geq t_0$ gilt
  \begin{align*}
    y_1(t) \leq y(t) \leq y_2(t).
  \end{align*}
  \textit{Hinweis:} Verwenden Sie Aufgabe 3.3 und die stetige Abhängigkeit von
  AWPs von der rechten Seite $f$.
  \item Zeigen Sie damit, dass für die Lösung des AWP
  \begin{align*}
    y^{\prime} = -y^3 + \sin(t), \qquad y(0) = y_0, \qquad -2 \leq y_0 \leq 2
  \end{align*}
  gilt $-2 \leq y(t) \leq 2$ für $t \geq 0$.
  \item Zeigen Sie, dass diese ODE eine $2\pi$-periodische Lösung hat. \\
  \textit{Hinweis:} Brouwerscher Fixpunktsatz.
\end{enumerate}
\end{exercise}
\begin{solution}
\leavevmode \\
\begin{enumerate}[label = \textbf{\alph*)}]
  \item Sei $\epsilon \in \R^+$ beliebig,
  und $|y_2 - z_2| < \frac{\epsilon}{L}$, wobei
  $L$ die Lipschitz-Konstante von $f$ in $y$ bezeichnet.
  Dann gilt
  \begin{align*}
    |f(t,y_2) - f(t,z_2)| \leq L |y_2 - z_2| < \epsilon.
  \end{align*}
  Wähle $\delta := \frac{\epsilon}{2L}, z_1(t) := y_1(t) - \delta, z_2(t) := y_2(t) + \delta$, dann folgt
  \begin{align*}
    z_1^{\prime}(t) &= y_1^{\prime}(t) = f(t,y_1) > f(t,z_1) - \epsilon \\
    z_2^{\prime}(t) &= y_2^{\prime}(t) = f(t,y_2) > f(t,z_2) - \epsilon
  \end{align*}
  Sei nun $z_{\epsilon}$ eine Lösung von $z_{\epsilon}^{\prime} = f(t,z) - \epsilon, z_{\epsilon}(t_0) = y_0$.
  Es gilt
  \begin{align*}
    z_1(t_0) = y_1(t_0) - \delta < y_0 = z_{\epsilon}(t_0) < y_2(t_0) + \delta = z_2(t_0)
  \end{align*}
  und nach Aufgabe 3.3 folgt
  \begin{align*}
    \forall t \geq t_0: z_1(t) < z_{\epsilon}(t) < z_2(t).
  \end{align*}
  Mit Satz 4.1 folgt aufgrund $|z_{\epsilon}^{\prime}(t) - f(t,z)| = |f(t,z) - \epsilon - f(t,z)| \leq \epsilon$
  \begin{align*}
  |y(t) - z_{\epsilon}(t)| \leq \underbrace{|y(t_0) - z_{\epsilon}(t_0)|}_{=0}\exp(L|t - t_0|) + \frac{\epsilon}{L}\left(\exp(L|t-t_0|) - 1\right)
  = \frac{\epsilon}{L}\left(\exp(L|t-t_0|) - 1\right)
  \stackrel{\epsilon \to 0}{\longrightarrow} 0.
  \end{align*}
  Wir erhalten $\lim_{\epsilon \to 0} z_{\epsilon}(t) = y(t)$. Insgesamt gilt
  \begin{align*}
    y_1(t) = \lim_{\epsilon \to 0}y_1(t) - \frac{\epsilon}{2L} \leq \underbrace{\lim_{\epsilon \to 0} z_{\epsilon}(t)}_{= y(t)} \leq
    \lim_{\epsilon \to 0}y_2(t) + \frac{\epsilon}{2L} = y_2(t).
  \end{align*}
  \item Definiere $y_1(t) = -2, y_2(t) = 2$. Dann gilt
  \begin{align*}
    -2 = y_1(0) \leq y(0) \leq y_2(0) \leq 2
  \end{align*}
  und
  \begin{align*}
    0 &= y_1^{\prime}(t) \leq f(t,y_1) = -(-2)^3 + \sin(t) \leq 7 \\
    0 &= y_2^{\prime}(t) \geq f(t,y_2) = -2^3 + \sin(t) \geq -7.
  \end{align*}
  Damit folgt nach Punkt a)
  \begin{align*}
    -2 \leq y(t) \leq 2.
  \end{align*}
  \item Der Fixpunktsatz von Brouwer besagt, dass jede stetige Abbildung $f: K \to K$
  von einer kompakten und konvexen Teilmenge $K \subset \R^n$ in sich einen Fixpunkt hat.
  Betrachte also die Funktion
  \begin{align*}
    \varphi: \begin{cases}
      [-2,2] &\to [-2,2] \\
      y_0 &\mapsto y_{0,y_0}(2\pi)
    \end{cases},
  \end{align*}
  welche aufgrund der stetigen Abhänigkeit von den Daten stetig ist.
  Aufgrund Punkt b) ist diese Funktion wohldefiniert und mit dem Brouwerschen Fixpunktsatz
  existiert ein $y_1 \in [-2,2]$, sodass
  \begin{align*}
    y_{0,y_1}(0) = y_1 = y_{0,y_1}(2\pi).
  \end{align*}
  $\widetilde{y}(t) = y_{0,y_1}(2\pi + t)$ erfüllt klarerweise das Anfangswertproblem $\widetilde{y}^{\prime}(t) =
  \widetilde{f}(t,\widetilde{y}(t)) = f(2\pi+t,\widetilde{y}(t)),
  \widetilde{y}(0) = y_1$. Es gilt
  \begin{align*}
    f(2\pi + t,y) = -y^3 + \sin(2\pi + t) = f(t,y)
  \end{align*}
  Jetzt gilt aufgrund der Eindeutigkeit von Anfangswertproblemen
  $y_{0,y_1}(2\pi + t) = \widetilde{y}(t) = y_{0,y_1}(t)$.
\end{enumerate}
\end{solution}
