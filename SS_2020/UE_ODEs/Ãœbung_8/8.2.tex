\begin{exercise}
Betrachten Sie eine skalare ODE $y^{\prime} = f(t,y)$ mit $f: \R \times \R \to \R$
stetig und lokal Lipschitz im 2.Argument. Sei $t \mapsto y(t), t \geq t_0$
die Lösung des AWP $y(t_0) = y_0$. Es seien $t \mapsto y_1(t)$ und $t \mapsto y_2(t)$
zwei differenzierbare Funktionen $\R \to \R$, für die gilt
\begin{align*}
  y_1(t_0) &\leq y_0, \qquad y_1^{\prime} \leq f(t,y_1), \qquad t \geq t_0 \\
  y_2(t_0) &\geq y_0, \qquad y_2^{\prime} \geq f(t,y_2), \qquad t \geq t_0.
\end{align*}
\begin{enumerate}[label = \textbf{\alph*)}]
  \item Zeigen Sie, dass für $t \geq t_0$ gilt
  \begin{align*}
    y_1(t) \leq y(t) \leq y_2(t).
  \end{align*}
  \textit{Hinweis:} Verwenden Sie Aufgabe 3.3 und die stetige Abhängigkeit von
  AWPs von der rechten Seite $f$.
  \item Zeigen Sie damit, dass für die Lösung des AWP
  \begin{align*}
    y^{\prime} = -y^3 + \sin(t), \qquad y(0) = y_0, \qquad -2 \leq y_0 \leq 2
  \end{align*}
  gilt $-2 \leq y(t) \leq 2$ für $t \geq 0$.
  \item Zeigen Sie, dass diese ODE eine $2\pi$-periodische Lösung hat. \\
  \textit{Hinweis:} Brouwerscher Fixpunktsatz.
\end{enumerate}
\end{exercise}
\begin{solution}
Beweis.
\end{solution}
