\begin{exercise}
Betrachten Sie eine skalare ODE $y^{\prime} = f(t,y)$ mit $f: \R \times \R \to \R$
stetig und lokal Lipschitz im 2.Argument. Sei $t \mapsto y(t), t \geq t_0$
die Lösung des AWP $y(t_0) = y_0$. Es seien $t \mapsto y_1(t)$ und $t \mapsto y_2(t)$
zwei differenzierbare Funktionen $\R \to \R$, für die gilt
\begin{align*}
  y_1(t_0) &\leq y_0, \qquad y_1^{\prime} \leq f(t,y_1), \qquad t \geq t_0 \\
  y_2(t_0) &\geq y_0, \qquad y_2^{\prime} \geq f(t,y_2), \qquad t \geq t_0.
\end{align*}
\begin{enumerate}[label = \textbf{\alph*)}]
  \item Zeigen Sie, dass für $t \geq t_0$ gilt
  \begin{align*}
    y_1(t) \leq y(t) \leq y_2(t).
  \end{align*}
  \textit{Hinweis:} Verwenden Sie Aufgabe 3.3 und die stetige Abhängigkeit von
  AWPs von der rechten Seite $f$.
  \item Zeigen Sie damit, dass für die Lösung des AWP
  \begin{align*}
    y^{\prime} = -y^3 + \sin(t), \qquad y(0) = y_0, \qquad -2 \leq y_0 \leq 2
  \end{align*}
  gilt $-2 \leq y(t) \leq 2$ für $t \geq 0$.
  \item Zeigen Sie, dass diese ODE eine $2\pi$-periodische Lösung hat. \\
  \textit{Hinweis:} Brouwerscher Fixpunktsatz.
\end{enumerate}
\end{exercise}
\begin{solution}
\leavevmode \\
\begin{enumerate}[label = \textbf{\alph*)}]
  \item Angenommen
  \begin{align*}
    \emptyset \neq \{t > t_0: y_2(t) < y(t) \} =: M.
  \end{align*}
  Dann betrachte $t_{1} := \inf M$. Es gilt
  \begin{align*}
    y_2(t_1) = \lim_{h \to 0^+} y_2(t_1 + h) \leq \lim_{h \to 0^+} y(t_1 + h) = y(t_1).
  \end{align*}
  Es muss sogar Gleichheit gelten, da anderenfalls aufgrund der Stetigkeit von $y$
  ein $t_2 < t_1$ existiert, sodass
  \begin{align*}
    y_2(t_2) < y(t_2).
  \end{align*}
  Nach Voraussetzung gilt
  \begin{align}\label{1}
    y_2^{\prime}(t_1) \geq f(t,y_2(t_1)) = f(t,y(t_1)) = y^{\prime}(t_1)
  \end{align}

  Gleichzeitig existiert aufgrund der Stetigkeit von $y_2- y$ ein $h_0 > 0$, sodass
  \begin{align}\label{2}
    y_2 - y < 0 \text{ auf } ]t_1,t_1+h_0[.
  \end{align}
    \begin{align*}
      0 \stackrel{\eqref{1}}{\leq} y_2^{\prime}(t_1) - y^{\prime}(t_1)
      = \lim_{h \to 0^+}\frac{y_2(t_1 + h) - y_2(t_1)}{h} - \lim_{h \to 0^+}\frac{y(t_1 + h) - y(t_1)}{h}
      = \lim_{h \to 0^+}\frac{y_2(t_1 + h) - y(t_1 + h)}{h}
      \stackrel{\eqref{2}}{<} 0.
    \end{align*}
    Widerspruch! \\
  Für $y_1$ geht der Beweis analog.
  \item Definiere $y_1(t) = -2, y_2(t) = 2$. Dann gilt
  \begin{align*}
    -2 = y_1(0) \leq y(0) \leq y_2(0) \leq 2
  \end{align*}
  und
  \begin{align*}
    0 &= y_1^{\prime}(t) \leq f(t,y_1) = -(-2)^3 + \sin(t) \leq 7 \\
    0 &= y_2^{\prime}(t) \geq f(t,y_2) = -2^3 + \sin(t) \geq -7.
  \end{align*}
  Damit folgt nach Punkt a)
  \begin{align*}
    -2 \leq y(t) \leq 2.
  \end{align*}
  \item Der Fixpunktsatz von Brouwer besagt, dass jede stetige Abbildung $f: K \to K$
  von einer kompakten Teilmenge $K \subset \R^n$ in sich einen Fixpunkt hat.
  Betrachte also die Funktion
  \begin{align*}
    \varphi: \begin{cases}
      [-2,2] &\to [-2,2] \\
      y_0 &\mapsto y_{0,y_0}(2\pi)
    \end{cases}
  \end{align*}
  Aufgrund Punkt b) ist diese Funktion wohldefiniert und mit dem Brouwerschen Fixpunktsatz
  existiert ein $y_1 \in [-2,2]$, sodass
  \begin{align*}
    y_{0,y_1}(0) = y_1 = y_{0,y_1}(2\pi).
  \end{align*}
  $\widetilde{y}(t) = y_{0,y_1}(2\pi + t)$ erfüllt klarerweise das Anfangswertproblem $\widetilde{y}^{\prime}(t) =
  \widetilde{f}(t,\widetilde{y}(t)) = f(2\pi+t,\widetilde{y}(t)),
  \widetilde{y}(0) = y_1$. Es gilt
  \begin{align*}
    f(2\pi + t,y) = -y^3 + \sin(2\pi + t, y) = f(t,y)
  \end{align*}
  Jetzt gilt aufgrund der Eindeutigkeit von Anfangswertproblemen $y_{0,y_1}(2\pi) = \widetilde{y}(t) = y_{0,y_1}(t)$.
\end{enumerate}
\end{solution}
