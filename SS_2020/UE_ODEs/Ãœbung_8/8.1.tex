\begin{exercise}
Sei $A \in \R^{d \times d}$ und $\omega > s(A) := \max \{\Re(\lambda): \lambda \in \sigma(A)\}$.
Zeigen Sie: Es gibt ein $M \geq 1$, sodass
\begin{align*}
  |\exp(tA)| \leq M\exp(\omega t), \qquad t \geq 0.
\end{align*}
Warum gilt diese Aussage nicht, wenn lediglich $\omega \geq s(A)$ gefordert wird?
\end{exercise}
\begin{solution}
Anmerkung: Wir verwenden für das Beispiel die Zeilensummennorm. Die Aussage ist
aufgrund der Äquivalenz aller Normen im endlich-dimensionalen für beliebige Normen gültig. \\
Sei also $A = VJV{-1}$ mit der zugehörigen Jordan-Normalform $J$.
\begin{align*}
  \|\exp(tA)\| = \|V\exp(tJ)V^{-1}\| \leq \underbrace{\|V\|\|V^{-1}\|}_{=:\widetilde{M}}\|\exp(tJ)\|
\end{align*}
Da die Exponentialfunktion einer Block-Diagonalmatrix wieder eine Block-Diagonalmatrix ist,
können wir für jeden Eigenwert $\lambda$ die zugehörigen Jordan-Kästchen betrachten.
Für
\begin{align*}
  \widetilde{J} = \begin{pmatrix}
    \lambda & 1 & & \\
    & \ddots & \ddots & \\
    & & \ddots & 1 \\
    & & & \lambda
  \end{pmatrix}
\end{align*}
ist
\begin{align*}
  \exp(t\widetilde{J}) = \exp(\lambda t)\begin{pmatrix}
    1 & t & \dots & \frac{t^{r-1}}{(r-1)!} \\
    & 1 & \dots & \frac{t^{r-2}}{(r-2)!} \\
    & & \ddots & \vdots \\
    & & & 1
  \end{pmatrix}
\end{align*}
Damit ist $\|\exp(t\widetilde{J})\| \leq \exp(\lambda t)\sum_{i=0}^{d-1} t^d$.
Da $|\exp(\lambda)| = \exp(\Re(\lambda))$ brauchen wir für $\|\exp(tJ)\|$
aufgrund der Block-Diagonalform nur den Eigenwert
mit größtem Realteil betrachten. Das größte Jordan-Kästchen davon hat maximal Dimension $d$
und es folgt
\begin{align*}
  \|\exp(tA)\| \leq \widetilde{M}\|\exp(s(A)t)\|\sum_{i=0}^{d-1} t^i \stackrel{!}{\leq} M \exp(\omega t).
\end{align*}
Wir formen um und erhalten für $M$
\begin{align*}
  M\geq \widetilde{M}\exp(\underbrace{t(s(A) - \omega)}_{< 0})\sum_{i=0}^{d-1} t^d =: f(t)
\end{align*}
Wie man leicht sieht, ist
\begin{align*}
  \lim_{t \to \infty}f(t) = 0
\end{align*}
und somit können wir $M := \max_{t \geq 0} f(t)$ wählen. \\
Als Gegenbeispiel dafür, dass die Aussage für $\omega \geq s(A)$ nicht mehr stimmt,
betrachte
\begin{align*}
  A = \begin{pmatrix}
    1 & 1 \\ 0 & 1
  \end{pmatrix}
\end{align*}
mit dem einzigen Eigenwert $\lambda = 1$. Wähle also $\omega = s(A) = 1$ und berechne
\begin{align*}
  \|\exp(tA)\| = \exp(1)\left\|\begin{pmatrix}
    1 & t \\ 0 & 1
  \end{pmatrix}\right\|
  = \exp(1)(t+1)
\end{align*}
Es gibt also kein $M > 0$, sodass die Gleichung
\begin{align*}
  \exp(1)(t+1) \leq M \exp(1)
\end{align*}
für alle $t \geq 0$ erfüllt wird.
\end{solution}
