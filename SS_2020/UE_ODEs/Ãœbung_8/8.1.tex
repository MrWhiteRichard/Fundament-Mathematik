\begin{exercise}
Sei $A \in \R^{d \times d}$ und $\omega > s(A) := \max \{\Re(\lambda): \lambda \in \sigma(A)\}$.
Zeigen Sie: Es gibt ein $M \geq 1$, sodass
\begin{align*}
  |\exp(tA)| \leq M\exp(\omega t), \qquad t \geq 0.
\end{align*}
Warum gilt diese Aussage nicht, wenn lediglich $\omega \geq s(A)$ gefordert wird?
\end{exercise}
\begin{solution}
Annahme: Mit $|\exp(tA)|$ ist die Zeilensummennorm dieser Matrix gemeint.
Sei also $A = VJV{-1}$ mit der zugehörigen Jordan-Normalform $J$.
Seien also $V = (v_1,v_d)$ eine Basis von (komplexen) Eigenvektoren von $A$.
\begin{align*}
  \|\exp(tA)\| = \|V\exp(tJ)V^{-1}\| \leq \underbrace{\|V\|\|V^{-1}\|}_{=:M}\|\exp(tD)\|
  = M|\exp(t\max \{\lambda: \lambda \in \sigma(A)\})| < M\exp(t\omega).
\end{align*}

\end{solution}
