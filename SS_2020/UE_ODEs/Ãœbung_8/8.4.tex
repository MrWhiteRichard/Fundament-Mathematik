\begin{exercise}[Gradientensysteme]
\phantom{}
\begin{enumerate}[label = \textbf{\alph*)}]
  \item Sei $d = 1, f \in C(\R, \R).$ Zu zeigen: Die autonome ODE $y' = f(y)$ hat eine strikte Ljapunovfunktion.
  \item Sei $d > 1, f \in C(\R^d, \R^d).$ Die ODE $y' = f(y)$ heißt Gradientensystem, falls es ein $F \in C^1(\R^d, \R)$ gibt mit $\nabla F = f$. Zu zeigen: Ein Gradientensystem hat eine strikte Ljapunovfunktion.
\end{enumerate}
\end{exercise}

\begin{solution}
\leavevmode \\
\begin{enumerate}[label = \textbf{\alph*)}]
  \item Nach Satz 5.13 ist $V \in C^1(G, \R)$ genau dann eine Ljapunovfunktion der autonomen ODE $y' = f(y)$, falls $\forall g \in G: \nabla V(y) \cdot f(y) \leq 0$. Wir definieren
  \begin{align}
      V(y) := -\int_0^y f(s)~ \mathrm{d}s.
  \end{align}

  Es folgt $V'(y) = -f(y)$ und somit $V'(y) \cdot f(y) = -f(y)^2 \leq 0.$

  Ein hinreichendes Kriterium für die Striktheit dieser Ljapunovfunktion ist die der letzten Ungleichung, falls $y$ keine Ruhelage, also $f(y) \neq 0$ ist. Das ist offenbar der Fall.
  \item Für $V := -F$ gilt
  \begin{align}
      \nabla V(y) \cdot f(y) = - \nabla F(y) \cdot f(y) = - f(y) \cdot f(y) = - \| f(y) \|^2 \leq 0.
  \end{align}

  Da Gleichheit genau dann gilt, falls $f(y) = 0$, ist $V$ auch hier strikt.
\end{enumerate}
\end{solution}
