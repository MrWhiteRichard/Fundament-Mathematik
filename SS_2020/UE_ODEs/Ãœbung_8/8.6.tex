\begin{exercise}
Das folgende Beispiel zeigt, dass man im Fall nichtautonomer linearer ODEs
\begin{align*}
  y^{\prime} = A(t)y
\end{align*}
von den Eigenwerten der Matrix $A(t)$ nicht auf die Stabilität der Ruhelage $y = 0$
schließen kann. Es sei
\begin{align*}
  A(t) = \begin{pmatrix}
    -1 + \frac{3}{2}\cos^2(t) & 1 -\frac{3}{2}\sin(t)\cos(t) \\
    -1 - \frac{3}{2}\sin(t)\cos(t) & -1 + \frac{3}{2}\sin^2(t)
  \end{pmatrix}.
\end{align*}
Zeigen Sie:
\begin{enumerate}[label = \textbf{\alph*)}]
  \item Die Eigenwerte $\lambda_{1,2}(t)$ von $A(t), t \in \R$ haben negativen Realteil.
  \item
  \begin{align*}
    y(t) = \exp(\nicefrac{t}{2})\begin{pmatrix}
      -\cos(t) \\ \sin(t)
    \end{pmatrix}
  \end{align*}
  ist eine Lösung der ODE.
  \item Die Lösung $y = 0$ ist instabil.
\end{enumerate}
\end{exercise}
\begin{solution}
Beweis.
\end{solution}
