\begin{exercise}
Das folgende Beispiel zeigt, dass man im Fall nichtautonomer linearer ODEs
\begin{align*}
  y^{\prime} = A(t)y
\end{align*}
von den Eigenwerten der Matrix $A(t)$ nicht auf die Stabilität der Ruhelage $y = 0$
schließen kann. Es sei
\begin{align*}
  A(t) = \begin{pmatrix}
    -1 + \frac{3}{2}\cos^2(t) & 1 -\frac{3}{2}\sin(t)\cos(t) \\
    -1 - \frac{3}{2}\sin(t)\cos(t) & -1 + \frac{3}{2}\sin^2(t)
  \end{pmatrix}.
\end{align*}
Zeigen Sie:
\begin{enumerate}[label = \textbf{\alph*)}]
  \item Die Eigenwerte $\lambda_{1,2}(t)$ von $A(t), t \in \R$ haben negativen Realteil.
  \item
  \begin{align*}
    y(t) = \exp(\nicefrac{t}{2})\begin{pmatrix}
      -\cos(t) \\ \sin(t)
    \end{pmatrix}
  \end{align*}
  ist eine Lösung der ODE.
  \item Die Lösung $y = 0$ ist instabil.
\end{enumerate}
\end{exercise}
\begin{solution}
\begin{enumerate}[label = \textbf{\alph*)}]
  \item Das charakteristische Polynom ist gegeben durch
  \begin{align*}
    \chi (\lambda) =& (-1 + \frac{3}{2}\cos^2 t - \lambda)(-1 + \frac{3}{2}\sin^2 t - \lambda) - (\frac{9}{4}\sin^2 t \cos^2 t - 1) \\
    =& \lambda^2 +2\lambda - \frac{3}{2} \lambda + 1 - \frac{3}{2} + \frac{9}{4}\sin^2 t \cos^2 t - \frac{9}{4} \sin^{2} t \cos^2 t + 1 \\
    =& \lambda^2 + \frac{\lambda}{2} + \frac{1}{2}
  \end{align*}
  Die Nullstellen lauten $- \frac{1}{4} \pm i\frac{\sqrt{7}}{4}$, haben also negativen Realteil.
  \item Wir leiten $y$ ab und erhalten
  \begin{align*}
    y^{\prime}(t) =& \begin{pmatrix}
      \sin(t)\exp(\frac{t}{2}) -\frac{1}{2}\cos(t)\exp(\frac{t}{2}) \\ \cos(t)\exp(\frac{t}{2}) + \frac{1}{2}\sin(t)\exp(\frac{t}{2})
    \end{pmatrix} \\
    =& \exp\left(\frac{t}{2}\right) \begin{pmatrix}
      \sin(t) -\frac{1}{2}\cos(t) \\ \cos(t) + \frac{1}{2}\sin(t)
    \end{pmatrix} \\
    =& \exp\left(\frac{t}{2}\right)\begin{pmatrix}
      -1 + \frac{3}{2}\cos^2(t) & 1 -\frac{3}{2}\sin(t)\cos(t) \\
      -1 - \frac{3}{2}\sin(t)\cos(t) & -1 + \frac{3}{2}\sin^2(t)
    \end{pmatrix}\begin{pmatrix}
      -\cos(t) \\ \sin(t)
    \end{pmatrix} = A(t)y.
  \end{align*}.
  \item Gemäß Satz 5.6 ist $y = 0$ genau dann stabil, wenn ${\sup}_{t \geq t_0} \| Y(t)\| < \infty$. Die Lösung aus Aufgabenteil b) ist eine Spalte der Fundamentalmatrix und somit $Y$ in der Norm unbeschränkt. Also ist $y= 0$ nicht stabil.
  \end{enumerate}
\end{solution}
