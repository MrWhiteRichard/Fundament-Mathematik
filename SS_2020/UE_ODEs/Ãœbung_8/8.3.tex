\begin{exercise}
Eine skalare ODE der Form
\begin{align*}
  y^{\prime} = g(t)y + h(t)y^2 + k(t)
\end{align*}
heißt Riccatigleichung. Sei $y_1$ eine Lösung dieser Gleichung.
\begin{enumerate}[label = \textbf{\alph*)}]
  \item Überprüfen Sie, dass jede Lösung $x$ der Bernoullischen ODE
  \begin{align*}
    x^{\prime} = (g(t) + 2y_1(t)h(t))x + h(t)x^2
  \end{align*}
  eine Lösung $y = y_1 + x$ der Riccatischen Gleichung erzeugt.
  \item Geben Sie die allgemeine Lösung der ODE
  \begin{align*}
    y^{\prime} = 3\left(2(t+1)^2 - \frac{1}{t+1}\right)y - 3(t+1)y^2 - 3(t+1)^3 + 4
  \end{align*}
  an. \\
  \textit{Hinweis:} Versuchen Sie ein lineares Polynom als spezielle Lösung.
\end{enumerate}
\end{exercise}

\begin{solution}
\begin{enumerate}[label = \textbf{\alph*)}]
  \item Wir rechnen ganz einfach nach:

  \begin{align*}
    y^\prime = y_1^\prime + x^\prime = g(t)y_1 +h(t)y_1^2 + k(t) +
    (g(t) + 2y_1(t)h(t))x + h(t)x^2 = g(t)(y_1+x) + h(t)(y_1^2 +2y_1x +x^2) + k(t)
  \end{align*}

  Also ist $y$ wirklich eine Lösung der Riccatischen Gleichung.

  \item Wir folgen dem Hinweis und sehen schnell, dass $y_1 = t+1$ eine Lösung der Gleichung ist.
  Um jetzt die allgemeine Lösung zu finden, verwenden wir \textbf{a)}. Wir suchen also
  Lösungen der Bernoullischen ODE

  \begin{align*}
    x^\prime = (6(t+1)^2 -\frac{3}{t+1}-6(t+1)^2)x - 3(t+1)x^2 = -\frac{3}{t+1}x
    -3(t+1)x^2
  \end{align*}

  Um diese zu lösen substituieren wir $\widetilde{x}=x^{-1}$ und erhalten für
  $\widetilde{x}$ das System:

  \begin{align*}
    \widetilde{x}^\prime = \frac{3}{t+1}\widetilde{x} + 3t+3
  \end{align*}

  Wir suchen zuerst eine Lösung für das homogene System, dieses ist Seperierbar und wir erhalten

  \begin{align*}
    \widetilde{x}=(t+1)^3
  \end{align*}

  als Lösung des homogenen Systems. Um nun eine Partikulärlösung zu finden verwenden
  wir Satz 3.17.

  \begin{align*}
    \widetilde{x}_p(t) = (t+1)^3 \int_{t_0}^t \frac{3(s+1)}{(s+1)^3}  ds=
    3(t+1)^3(\frac{1}{t_0 + 1}-\frac{1}{t+1}) = \frac{3(t+1)^3}{(t_0+1)} - 3(t+1)^2
  \end{align*}

  Setzen wir der Einfachheit (und Lesbarkeit) halber $t_0 = 0$, dann hat
  jede Lösung unseres Systems für $\widetilde{x}$ hat nun die Form

  \begin{align*}
    \widetilde{x}_c = \widetilde{x}_p + c(t+1)^3 = (t+1)^3(3+c)-3(t+1)^2
  \end{align*}

  mit $c \in \R$.
  Wenn wir nun Rücksubstituieren erhalten wir für $x$ einfach $x = 1/\widetilde{x}$ und
  wir haben allgemeine Lösungen für unsere ursprüngliche ODE mit

  \begin{align*}
    y_c = y_1 + x_c = (t+1) + \frac{1}{\widetilde{x}_c} = (t+1) + \frac{1}{(t+1)^3(3+c)-3(t+1)^2}
  \end{align*}
\end{enumerate}
\end{solution}
