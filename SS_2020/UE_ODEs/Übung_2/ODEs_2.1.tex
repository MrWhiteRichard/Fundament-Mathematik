\begin{exercise}

Sei $G \subseteq \R^{d+1}$ Gebiet, $f \in C(G; \R^d)$.

a) zeige die Äquivalenz folgender Aussagen:

\begin{enumerate}
  \item* (1) $\forall (t, y) \in G \exists U$ Umgebung $\exists L_U > 0 \forall (\tilde{t}, x), (\tilde{t}, y) \in U
  (\left\lVert f(\tilde{t}, x), f(\tilde{t}, y) \right\rVert \leq L_U \left\lVert x-y \right\rVert).$ \label{one}
  \item* (2) $\forall K \subseteq G kompakt \exists L_K > 0 \forall (\tilde{t}, x), (\tilde{t}, y) \in K
  (\left\lVert f(\tilde{t}, x), f(\tilde{t}, y) \right\rVert \leq L_K \left\lVert x-y \right\rVert).$ \label{two}
\end{enumerate}

\end{exercise}

\begin{solution}

(2) $\Rightarrow$ (1):
Sei $(t, y) \in \R^d$.
% $G$ ist Gebiet $\Rightarrow \exits \epsilon > 0 (U_\epsilon((t, y)) \subseteq G)$.
Insbesondere ist die abgeschlossene $\epsilon /2$-Kugel um $(t,y)$ in $G$ enthalten und kompakt, $(1)$ gilt also insbesondere für die offene $\epsilon /4$-Umgebung um $(t,y)$.

(1) $\Rightarrow$ (2):
Wir führen einen Widerspruchsbeweis, es gelte also \eqref{one} und nicht \eqref{two}. Das heißt es gibt ein kompates $K \subseteq G$, für das 
  \begin{align*}
    \forall L \in \R^+: \exists \pbraces{s_L,x_L}, \pbraces{s_L, z_L} \in K: \Vbraces{f(s_L,x_L) - f(s_L,z_L)}_{\R^d} > L \Vbraces{x_L - z_L}_{\R^d}
  \end{align*}
gilt. Wir erhalten also zwei Netze $\pbraces{s_L,x_L}, \pbraces{s_L, z_L}$ aus $K$, wobei wir auf $L$ die natürliche Ordnung auf $\R^+$ betrachten. Da $K$ kompakt ist hat $\pbraces{s_L,x_L}$ ein gegen ein $(s,x) \in K$ konvergentes Teilnetz $\pbraces{s_M, x_M}$. Da $f$ stetig ist wissen wir, dass $f(K)$ kompakt und damit beschränkt ist und damit ist der Zähler im Bruch
  \begin{align*}
    \Vbraces{x_L - z_L}_{\R^d} < \frac{\Vbraces{f(s_L,x_L) - f(s_L,z_L)}_{\R^d}}{L}
  \end{align*}
  beschränkt, also ist $\Vbraces{x_L - z_L}_{\R^d}$ eine Nullfolge. Daraus erhalten wir unter Benützung der Dreiecksungleichung, dass auch $(s_M,z_M)$ gegen $(s,x)$ konvergiert. Nun ist aber $(s,x) \in G$ ein Punkt der \eqref{one} nicht erfüllt, was ein Widerspruch ist.
\end{solution}
