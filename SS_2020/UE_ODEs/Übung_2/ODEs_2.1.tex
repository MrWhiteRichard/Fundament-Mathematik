\begin{exercise}

Sei $G \subseteq \R^{d+1}$ Gebiet, $f \in C(G; \R^d)$.

a) zeige die Äquivalenz folgender Aussagen:

\begin{enumerate}
  \item* (1) $\forall (t, y) \in G \exists U$ Umgebung $\exists L_U > 0 \forall (\tilde{t}, x), (\tilde{t}, y) \in U
  (\left\lVert f(\tilde{t}, x), f(\tilde{t}, y) \right\rVert \leq L_U \left\lVert x-y \right\rVert).$
  \item* (2) $\forall K \subseteq G kompakt \exists L_K > 0 \forall (\tilde{t}, x), (\tilde{t}, y) \in K
  (\left\lVert f(\tilde{t}, x), f(\tilde{t}, y) \right\rVert \leq L_K \left\lVert x-y \right\rVert).$
\end{enumerate}

\end{exercise}

\begin{solution}

(2) $\Rightarrow$ (1):
Sei $(t, y) \in \R^d$.
% $G$ ist Gebiet $\Rightarrow \exits \epsilon > 0 (U_\epsilon((t, y)) \subseteq G)$.
Insbesondere ist die abgeschlossene $\epsilon /2$-Kugel um $(t,y)$ in $G$ enthalten und kompakt, $(1)$ gilt also insbesondere für die offene $\epsilon /4$-Umgebung um $(t,y)$.

(1) $\Rightarrow$ (2):
Jede Umgebung enthält eine $\epsilon$-Kugel.
Daher gilt $\forall x \in K \exists \epsilon_x > 0 ( (1)$ gilt für $U_{\epsilon_x}(x))$.
Weil $K$ kompakt ist, existiert eine endl. TÜ von $K$ durch solche Kugeln:
$K \subseteq \bigcup_{i = 1}^n U_{\epsilon_{x_i}}(x_i)$.
Wähle $L_K := \max L_{U_{\epsilon_{x_i}}(x_i)}$.

\end{solution}
