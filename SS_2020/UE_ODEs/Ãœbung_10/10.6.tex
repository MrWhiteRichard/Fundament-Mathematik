\begin{exercise}
Betrachten Sie die Schwingung einer einseitig eingespannten Saite, welche die
Differentialgleichung
\begin{align*}
  \frac{1}{c^2}\frac{\partial^2}{\partial t ^2}y(x,t) = \frac{\partial^2}{\partial x^2}
  y(x,t), \qquad x \in (0,1), \qquad t > 0
\end{align*}
mit den Randbedingungen
\begin{align*}
  y(0,t) = 0, \qquad \frac{\partial}{\partial x}y(1,t) = 0, \qquad t > 0,
\end{align*}
erfüllt. Die Anfangsbedingungen seien $y(\cdot,0) = y_0(\cdot)$ und
$\frac{\partial}{\partial t}y(\cdot,0) = y_1(\cdot)$. Formulieren und lösen Sie
das Sturm-Liouville Eigenwertproblem, welches durch den Ansatz der Separation
der Variablen entsteht. Geben Sie eine (formale) Lösung als Reihe an.
\end{exercise}
\begin{solution}
Machen wir den Ansatz mit Separation der Variablen. Das bedeutet wir setzen

\begin{align*}
  y(x,t)
  =
  v(x)w(t)
\end{align*}

Setzen wir nun in unsere Differentialgleichung ein erhalten wir

\begin{align*}
  \frac{1}{c^2}w^\primeprime(t)v(x)
  \stackrel{!}{=}
  v^\primeprime(x)w(t) \\
  \implies
  \frac{1}{c^2}\frac{w^\primeprime(t)}{w(t)}
  =
  \frac{v^\primeprime(x)}{v(x)}
\end{align*}

Dabei hängt die linke Seite nur von $t$ und die rechte Seite nur von $x$ ab. Deswegen
muss es also eine Konstante $\lambda$ geben, die folgendes erfüllt:

\begin{align*}
  \frac{1}{c^2}\frac{w^\primeprime(t)}{w(t)}
  =
  -\lambda
  =
  \frac{v^\primeprime(x)}{v(x)}
  \implies
  \begin{cases}
    -\lambda v - v^\primeprime = 0 & \text{auf } (0,1), \quad v(0)=0=v^\prime(1) \\
    -\lambda c^2 w - w^\primeprime = 0 & \text{auf }(0,\infty)
  \end{cases}
\end{align*}

Aus der Vorlesung kennen wir die allgemeinen Lösungen für $v$ und  $w$:

\begin{align*}
  v(x) = c_1 \sin(\sqrt{\lambda}x) + c_2 \cos(\sqrt{\lambda}x) \\
  w(t) = c_3 \sin(c\sqrt{\lambda}t) + c_4 \cos(c\sqrt{\lambda}t)
\end{align*}

Um nun die Randbedingungen zu erfüllen, setzen wir zuerst $c_2 = 0$ und erhalten
\begin{align*}
  y(0,t) = v(0)w(t) = c_1\sin(0)w(t) = 0.
\end{align*}
Sehen wir uns nun die erste Ableitung
von $v$ an (o.B.d.A $c_1 \neq 0$)
\begin{align*}
  v^\prime(1)
  =
  c_1 \sqrt{\lambda}\cos(\sqrt{\lambda})
  \stackrel{!}{=}
  0 \\
  \implies
  \lambda = 0
  \lor
  \lambda
  =
  \frac{\pi^2}{4}n^2 \quad \text{mit } n\in \Z \backslash\{0\}
\end{align*}

Wir setzen $\lambda = \frac{\pi^2}{4}n^2$ für beliebiges $n \in \N$ und haben nun mit

\begin{align*}
  v(x)w(t)
  =
  c_1\sin\left(n\frac{\pi}{2} x\right)
  \left(c_3 \sin\left(c\frac{\pi}{2}nt\right) +
  c_4 \cos\left(c\frac{\pi}{2}nt\right)\right)
\end{align*}

eine Lösung unserer Differentialgleichung die auch die Randbedingungen erfüllt.
Machen wir nun den Ansatz für die Lösung mit dem Superpositionsprinzip

\begin{align*}
  y(x,t)
  =
  \sum_{n=1}^{\infty} \sin\left(\frac{nx\pi}{2}\right)
  \left(c_{1,n}\sin\left(\frac{nct\pi}{2}\right)+c_{2,n}\cos\left(\frac{nct\pi}{2}\right)\right)
\end{align*}

Um nun die Koeffizienten $c_{1,n},c_{2,n}$ zu erhalten, sehen wir uns die
Anfangsbedingungen an.

\begin{align*}
  y_0(x)
  \stackrel{!}{=}
  \sum_{n=1}^\infty c_{2,n}\sin\left(\frac{nx\pi}{2}\right),       \quad x \in (0,1) \\
  y_1(x)
  \stackrel{!}{=}
  \sum_{n=1}^\infty c_{1,n}\frac{nc\pi}{2}\cos\left(\frac{nx\pi}{2}\right),  \quad x \in (0,1)
\end{align*}

Falls man nun $y_0$ und $y_1$ (anti)-symmetrisch auf $(-1,1)$
fortsetzt und dann die Sinusreihen bildet (falls diese dann auch gegen die entsprechenden Funktionen
konvergieren) erhält man ebenso die Darstellung:

\begin{align*}
  y_0(x)
  =
  \sum_{n=1}^\infty a_n \sin(\pi nx) \\
  y_1(x)
  =
  \sum_{n=1}^\infty b_n \cos(n \pi x), \quad \text{mit} \\
  a_n
  =
  \int_0^2 y_0(x)\sin(n\pi x) dx \\
  b_n
  =
  \int_0^2 y_1(x)\cos(n\pi x) dx \\
\end{align*}

Womit man zumindest $c_{1,n}$ und $c_{2,n}$ für $n \in 2\N$ bestimmen könnte(?)

\end{solution}
