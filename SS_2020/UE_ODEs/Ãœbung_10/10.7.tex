\begin{exercise}
\leavevmode \\
\begin{enumerate}[label = \textbf{\alph*)}]
  \item Zeigen Sie, dass für jedes $f \in C([0,1])$ die Randwertaufgabe
  \begin{align*}
    -y^{\primeprime} + ay = f, \qquad y^{\prime}(0) = y^{\prime}(1) = 0, \qquad a \in (0,\infty)
  \end{align*}
  eine eindeutige Lösung besitzt.
  \item Bestimmen Sie die Greensche Funktion für das Randwertproblem.
  \item Gegeben sei eine stetige Funktion $F: [0,1] \times \R \to \R$, die
  global Lipschitz-stetig bezüglich $u$ ist, also
  \begin{align*}
    |F(t,u) - F(t,\widetilde{u})| \leq L|u - \widetilde{u}|, \qquad
    \forall t \in [0,1], u , \widetilde{u} \in \R
  \end{align*}
  mit Lipschitzkonstante $0 < L < a$. Zeigen Sie, dass unter diesen Voraussetzungen
  das nichtlineare Randwertproblem
  \begin{align*}
    -y^{\primeprime} + ay = F(t,y), \qquad y^{\prime}(0) = y^{\prime}(1) = 0
  \end{align*}
  eine eindeutige Lösung besitzt.
  \item Falls $L \geq a$, dann besitzt dieses Randwertproblem im Allgemeinen keine
  eindeutige Lösung.
\end{enumerate}
\end{exercise}
\begin{solution}
\leavevmode \\
\begin{enumerate}[label = \textbf{\alph*)}]
  \item Die ODE ist bereits ist selbstadjungierter Form mit $p(x) = 1, q(x) = a$.
  Bestimmen wir zunächst ein Fundamentalsystem für $-y^{\primeprime} + ay = 0$.
  Zwei linear unabhängige Lösungen kann man direkt ablesen mit $y_1(x) = \exp(\sqrt{a}x), y_2(x) = \exp(-\sqrt{a}x)$.
  Das Randwertproblem ist genau dann für alle \\
  $f \in C([a,b];\R), \rho_1,\rho_2 \in \R$
  eindeutig lösbar, wenn
  \begin{align*}
    0 \stackrel{!}{\neq} \det\left(\begin{pmatrix}
      R_1y_1 & R_1y_2 \\ R_2y_1 & R_2y_2
    \end{pmatrix}\right) =
    \det\left(\begin{pmatrix}
      \sqrt{a} & -\sqrt{a} \\ \sqrt{a}\exp(\sqrt{a}) & -\sqrt{a}\exp(-\sqrt{a})
    \end{pmatrix}\right)
    = a\exp(\sqrt{a}) - a\exp(-\sqrt{a}).
  \end{align*}
  Dies wird in offensichtlicher Weise von allen $a > 0$ erfüllt.
  \item Additionstherome:
  \begin{align}
    \sinh(x \pm y) = \sinh(x)\cosh(y) \pm \cosh(x)\sinh(y) \label{sinh} \\
    \cosh(x \pm y) = \cosh(x)\cosh(y) \pm \sinh(x)\sinh(y) \label{cosh}
  \end{align}
  Für die Konstruktion der Greenschen Funktion verwenden wir das (hoffentlich)
  bequemere Fundamentalsystem
  \begin{align*}
  y_1(x) &= \frac{\exp(\sqrt{a}x)+\exp(-\sqrt{a}x)}{2} = \cosh(\sqrt{a}x), \\
  y_2(x) &= \frac{\exp(\sqrt{a}x)+\exp(-\sqrt{a}x)}{2} = \sinh(\sqrt{a}x)
  \end{align*}
  oder das noch bequemere
  \begin{align*}
  y_1(x) &= \cosh(\sqrt{a}x), \\
  y_2(x) &= \cosh(-\sqrt{a})\cosh(\sqrt{a}x) + \sinh(-\sqrt{a})\sinh(\sqrt{a}x) =
  \cosh(\sqrt{a}(x-1)).
  \end{align*}
  Wir machen den Ansatz
  \begin{align*}
    G(x,t) = \begin{cases}
      a_1(t)\cosh(\sqrt{a}x) + a_2(t)\cosh(\sqrt{a}(x-1)), & x \leq t \\
      b_1(t)\cosh(\sqrt{a}x) + b_2(t)\cosh(\sqrt{a}(x-1)), & x > t
    \end{cases},
  \end{align*}
  welcher nach Konstruktion sicher die Bedingung
  \begin{itemize}
    \item $LG(\cdot,t) = 0$ auf $(a,t) \cup (t,b)$ erfüllt.
    \item $R_1G(\cdot,t) = 0$:
    \begin{align*}
      R_1G(\cdot,t) = \partial_xG(0,t) = a_1(t)\sqrt{a}\sinh(0) - a_2(t)\sqrt{a}\sinh(-\sqrt{a}) =  - a_2(t)\sqrt{a}\sinh(-\sqrt{a}) \stackrel{!}{=} 0.
    \end{align*}
    Also erhalten wir $a_2(t) = 0$.
    \item $R_2G(\cdot,t) = 0$:
    \begin{align*}
      R_2G(\cdot,t) &= \partial_xG(1,t) = b_1(t)\sqrt{a}\sinh(\sqrt{a}) - b_2(t)\sqrt{a}\sinh(0)
      = b_1(t)\sqrt{a}\sinh(\sqrt{a}) \stackrel{!}{=} 0
    \end{align*}
    Wir erhalten $b_1(t) = 0$.
    \item $G(\cdot,t)$ ist stetig bei $x = t$: \\
    \begin{align*}
      G(t^-,t) = a_1(t)\cosh(\sqrt{a}t) \stackrel{!}{=}
      b_2(t)\cosh(\sqrt{a}(t-1)) = G(t^+,t).
    \end{align*}
    Wir erhalten $a_1(t) = b_2(t)\frac{\cosh(\sqrt{a}(t-1))}{\cosh(\sqrt{a}t)}$.
    \item $\partial_x G(t^+,t)- \partial_x G(t^-,t) = - \frac{1}{p(t)} = -1$:
    \begin{align*}
      \partial_x G(t^+,t)- \partial_x G(t^-,t) =
      b_2(t)\sqrt{a}\sinh(\sqrt{a}(t-1)) - a_1(t)\sqrt{a}\sinh(\sqrt{a}t) \stackrel{!}{=} -1.
    \end{align*}
    Es folgt
    \begin{align*}
      b_2(t) &= -\frac{1}{\sqrt{a}\sinh(\sqrt{a}(t-1)) - \frac{\cosh(\sqrt{a}(t-1))}{\cosh(\sqrt{a}t)}\sqrt{a}\sinh(\sqrt{a}t)} \\
      &= -\frac{\cosh(\sqrt{a}t)}{\sqrt{a}\left(\sinh(\sqrt{a}(t-1))\cosh(\sqrt{a}t) - \cosh(\sqrt{a}(t-1))\sinh(\sqrt{a}t)\right)} \\
      &= -\frac{\cosh(\sqrt{a}t)}{\sqrt{a}\sinh(-\sqrt{a})}
      = \frac{\cosh(\sqrt{a}t)}{\sqrt{a}\sinh(\sqrt{a})}
    \end{align*}
    \item Insgesamt liefert das
    \begin{align*}
      G(x,t) = \begin{cases}
      \frac{\cosh(\sqrt{a}(t-1))}{\sqrt{a}\sinh(\sqrt{a})}\cosh(\sqrt{a}x), & x \leq t \\
      \frac{\cosh(\sqrt{a}t)}{\sqrt{a}\sinh(\sqrt{a})}\cosh(\sqrt{a}(x-1)), & x > t
      \end{cases}.
    \end{align*}
  \end{itemize}
  \item Wir betrachten das äquivalente System erster Ordnung:
  \begin{align*}
    \begin{pmatrix}
      y \\ z
    \end{pmatrix}^{\prime} =
    \begin{pmatrix}
      z \\ ay - F(t,y)
    \end{pmatrix}
  \end{align*}
  Wir definieren die Funktion
  \begin{align*}
    f(t,y,z) = \begin{pmatrix}
      z \\ ay - F(t,y)
    \end{pmatrix}.
  \end{align*}
  Gelte $\|(y,z)^{\top} - (\widetilde{y},\widetilde{z})\|_{\infty} \leq \epsilon$. Dann folgt
  \begin{align*}
    \|f(t,y,z) - f(t,\widetilde{y},\widetilde{z})\|_{\infty} \leq \max\{\epsilon, |a\epsilon| + |L\epsilon|\} \leq \max\{1,2a\}\epsilon
  \end{align*}
  damit die Lipschitz-Stetigkeit von $f$.
  Jetzt können wir das dazugehörige Anfangswertproblem betrachten, wenn wir $y(0) = x$ beliebig festsetzen.
  \begin{align*}
    \begin{pmatrix}
      y \\ z
    \end{pmatrix}^{\prime} =
    \begin{pmatrix}
      z \\ ay - F(t,y)
    \end{pmatrix}, \qquad
    \begin{pmatrix}
      y(0) \\ z(0)
    \end{pmatrix}
    = \begin{pmatrix}
      b \\ 0
    \end{pmatrix}.
  \end{align*}
  Definiere $(y_x,y_x^{\prime})^{\top}$ als die eindeutige Lösung des Anfangswertproblem
  zu dem Startwert $(x,0)^{\top}$ und betrachte die Funktion $f: x \mapsto y_x^{\prime}(1)$.
  Wir behaupten, dass diese Funktion injektiv ist. Wäre dem nicht so, betrachte $x_1,x_2$, sodass $f(x_1) = f(x_2)$.
  Es folgt (o.B.d.A. $x_1 > x_2$)
  \begin{align}\label{ws}
    (y_{x_1} - y_{x_2})^{\prime}(0) = 0 = (y_{x_1} - y_{x_2})^{\prime}(1)
  \end{align}
  und
  \begin{align*}
    (y_{x_1} - y_{x_2})^{\primeprime}(t) = a(y_{x_1}(t) - y_{x_2}(t)) - (F(t,y_{x_1}(t)) - F(t,y_{x_2}(t))
    \geq a(y_{x_1}(t) - y_{x_2}(t)) - L|y_{x_1}(t) - y_{x_2}(t)|.
  \end{align*}
  Wir wissen, dass $y_{x_1}(0) = x_1 > x_2 = y_{x_2}(0)$ und damit auch
  $y_{x_1}(t) > y_{x_2}(t)$ für $t \in [0,\epsilon]$ mit $\epsilon$ hinreichend klein. Für diese $t$ folgt dann
  \begin{align*}
    (y_{x_1} - y_{x_2})^{\primeprime}(t) \geq (a-L)|y_{x_1}(t) - y_{x_2}(t)| > 0
  \end{align*}
  und damit ist $(y_{x_1} - y_{x_2})^{\prime}$ auf $[0,\epsilon]$ streng monoton wachsend
  und für $t > 0$ echt positiv und damit ebenso $(y_{x_1} - y_{x_2})$.
  Daher kann nie der Fall $y_{x_1} - y_{x_2} = 0$ eintreten, weil dafür zuvor
  $(y_{x_1} - y_{x_2})^{\prime}$ negativ sein müsste und damit wiederum
  auch $(y_{x_1} - y_{x_2})^{\primeprime}$. \\
  Es folgt also auf ganz $[0,1]$
  \begin{align*}
    (y_{x_1} - y_{x_2})^{\primeprime}(t) \geq (a-L)|y_{x_1}(t) - y_{x_2}(t)| > 0
  \end{align*}
  im Widerspruch zu \eqref{ws}. Damit haben wir die Eindeutigkeit der Lösung gezeigt
  und uns fehlt nur noch die Existenz. \\
  Dafür betrachte wieder $x_1, x_2$ mit $f(x_1) > f(x_2)$. Es folgt
  \begin{align*}
    f_{x_1 - x_2} &= (y_{x_1 - x_2})^{\prime}(1) = (y_{x_1} - y_{x_2})^{\prime}(1) = f(x_1) - f(x_2) > 0 \\
    f_{x_2 - x_1} &< 0
  \end{align*}
  Aufgrund der stetigen Abhängigkeit von den Daten ist $f$ stetig und mit dem Zwischenwertsatz
  existiert ein $x: f(x) = 0$.
  \item Betrachte das Gegenbeispiel $F(t,y) = ay$ mit Lipschitzkonstante $L = a$. Das Randwertproblem lautet nun
  \begin{align*}
    y^{\primeprime} = 0, \qquad y^{\prime}(0) = 0 = y^{\prime}(1),
  \end{align*}
  welches klarerweise von jeder beliebigen konstanten Funktion gelöst wird.
\end{enumerate}
\end{solution}
