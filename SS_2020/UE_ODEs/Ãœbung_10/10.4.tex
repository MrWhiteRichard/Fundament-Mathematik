\begin{exercise}
Betrachten Sie das RWP
\begin{align*}
  Ly &:= -(py^{\prime})^{\prime} + qy = f \text{ auf } (a,b), \\
  R_1y &:= \alpha_1y(a) + \alpha_2p(a)y^{\prime}(a) = \rho_1, \\
  R_2y &:= \beta_1y(b) + \beta_2p(b)y^{\prime}(b) = \rho_2,
\end{align*}
wobei $p,q$ hinreichend glatt sind, $p > 0$ auf $[a,b]$. Sei $(\alpha_1,\alpha_2) \neq (0,0)$
und $(\beta_1,\beta_2) \neq (0,0)$. Nehmen Sie an, dass das RWP für $f = 0, \rho_1,\rho_2 = 0$
nur trivial lösbar sei. Seien weiters $y_1,y_2$ zwei linear unabhängige Lösungen
von $Ly = 0$ mit $R_1y_1 = 0$ und $R_2y_2 = 0$. Dann gilt:
\begin{align*}
  \kappa(x) &:= p(x)(y_1^{\prime}y_2 - y_1y_2^{\prime}) \equiv \text{const} \neq 0 \\
  G(x,t) &= \frac{1}{\kappa}\Bigg\{\begin{matrix}
    y_1(t)y_2(x), & a \leq t \leq x \leq b \\
    y_1(x)y_2(t), & a \leq x \leq t \leq b
  \end{matrix}
\end{align*}
\end{exercise}

\begin{solution}
Berechne
\begin{align*}
  \kappa^{\prime}(x) &= p^{\prime}(x)(y_1^{\prime}y_2 - y_1y_2^{\prime})
  + p(x)(y_1^{\prime}y_2^{\prime} + y_1^{\primeprime}y_2 - y_1^{\prime}y_2^{\prime} - y_1y_2^{\primeprime})
  = p^{\prime}(x)(y_1^{\prime}y_2 - y_1y_2^{\prime})
  + p(x)(y_1^{\primeprime}y_2  - y_1y_2^{\primeprime}) \\
  &= y_2(p^{\prime}(x)y_1^{\prime} + p(x)y_1^{\primeprime}) -
  y_1(p^{\prime}(x)y_2^{\prime} + p(x)y_2^{\primeprime})
  = y_2(py_1^{\prime})^{\prime} - y_1(py_2^{\prime})^{\prime} \\
  &= y_2qy_1 - y_1qy_2 = 0.
\end{align*}
Oder auch mit der Lagrange-Identität. Sei also angenommen, dass $y_1,y_2 \in C_2([a,b])$. Dann gilt
\begin{align*}
  \kappa^{\prime}(x) = (p(x)(y_1^{\prime}y_2 - y_1y_2^{\prime}))^{\prime} = y_1Ly_2 - y_2Ly_1 = 0.
\end{align*}
Angenommen $\kappa(x) \equiv 0$, dann folgt aufgrund $p(x) > 0$
\begin{align*}
  y_1^{\prime}y_2 = y_1y_2^{\prime} \iff
  \frac{y_1^{\prime}}{y_1} = \frac{y_2^{\prime}}{y_2} \iff
  (\ln(y_1))^{\prime} = (\ln(y_2))^{\prime} \iff
  \ln(y_1) = \ln(y_2) + C \iff
  y_1 = y_2\exp(C)
\end{align*}
im Widerspruch zur linearen Unabhängigkeit von $y_1,y_2$. \\
Sei $H$ nun eine beliebige Funktion, welche die Bedingungen aus Satz 6.7.
erfüllt. Wir zeigen, dass daraus bereits $H = G$ folgt.
\begin{itemize}
  \item $LH(\cdot,t) = 0$ auf $(a,t) \cup (t,b)$: \\
  Es folgt für alle $t \in T:$
  \begin{align*}
    \forall x < t:& \exists \gamma_{1,t}, \gamma_{2,t} \in \R:
    H(x,t) = \gamma_{1,t}y_1(x) + \gamma_{2,t}y_2(x) \\
    \forall x > t:& \exists \delta_{1,t}, \delta_{2,t} \in \R:
    H(x,t) = \gamma_{1,t}y_1(x) + \delta_{2,t}y_2(x)
  \end{align*}
  und
  \begin{align*}
    H(x,t) = \begin{cases}
      \gamma_1(t)y_1(x) + \gamma_2(t)y_2(x), & x < t \\
      \delta_1(t)y_1(x) + \delta_2(t)y_2(x), & x > t.
    \end{cases}
  \end{align*}
  \item $R_1H(\cdot,t) = 0$:
  \begin{align*}
    R_1H(\cdot,t) = R_1(\gamma_1(t)y_1 + \gamma_2(t)y_2)
    = \gamma_1(t)R_1(y_1) + \gamma_2(t)R_1(y_2) = \gamma_2(t)R_1(y_2) \stackrel{!}{=} 0.
  \end{align*}
  Da das Problem für $\rho_1 = \rho_2 = 0$ nur trivial lösbar ist, folgt $R_1(y_2) \neq 0$
  und damit $\gamma_2(t) = 0$.
  \item $R_2H(\cdot,t) = 0$: \\
  Analog sieht man ein, dass $\delta_1(t) = 0$.
  \item $H(\cdot,t)$ ist stetig bei $x = t$:
  \begin{align*}
    \gamma_1(t)y_1(t) = \delta_2(t)y_2(t)
    \iff
    \gamma_1(t) = \frac{\delta_2(t)y_2(t)}{y_1(t)}
  \end{align*}
  \item $\partial_x H(t^+,t)- \partial_x H(t^-,t) = - \frac{1}{p(t)}$:
  \begin{align*}
    \partial_x H(t^+,t) - \partial_x H(t^-,t) &= \delta_2(t)y_2^{\prime}(t)- \gamma_1(t)y_1^{\prime}(t)
    = \delta_2(t)y_2^{\prime}(t)- \frac{\delta_2(t)y_2(t)}{y_1(t)}y_1^{\prime}(t) \\
    &= \delta_2(t)\left(y_2^{\prime}(t)-  \frac{y_2(t)}{y_1(t)}y_1^{\prime}(t)\right)
    \stackrel{!}{=} -\frac{1}{p(t)}.
  \end{align*}
  Wir erhalten
  \begin{align*}
    \delta_2(t) &= -\left(p(t)\left(y_2^{\prime}(t)-  \frac{y_2(t)}{y_1(t)}y_1^{\prime}(t)\right)\right)^{-1}
    = -\left(p(t)\frac{y_2^{\prime}y_1(t) - y_2(t)y_1^{\prime}(t)}{y_1(t)}\right)^{-1} \\
    &= \left(\frac{\kappa}{y_1(t)}\right)^{-1} = \frac{y_1(t)}{\kappa}
  \end{align*}
  und
  \begin{align*}
    \gamma_1(t) = \frac{y_2(t)}{\kappa}
  \end{align*}
\end{itemize}
Insgesamt gilt also
\begin{align*}
  H(x,t) = \begin{Bmatrix}
    \frac{y_2(t)y_1(x)}{\kappa}, & t \leq x \\
    \frac{y_1(t)y_2(x)}{\kappa}, & t \geq x
  \end{Bmatrix} = G(x,t).
\end{align*}
\end{solution}
