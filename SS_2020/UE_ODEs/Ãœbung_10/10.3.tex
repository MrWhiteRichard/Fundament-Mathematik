\begin{exercise}
Bestimmen Sie Lösungen der folgenden Randwertprobleme, falls diese existieren:
\begin{enumerate}[label = \textbf{\alph*)}]
  \item $(yy^{\prime})^{\prime} = 1 - 3t$ mit $y(0) = y(1) = 0$,
  \item $y^{\primeprime} + 4y = 0$, sowie $y^{\primeprime} + 4y = 4t$ mit
  $y(-\pi) = y(\pi), y^{\prime}(\pi) = y^{\prime}(-\pi)$,
  \item $y^{\primeprime} + 2y^{\prime} + 5y = 0$ mit $y(0) = 2, y(\nicefrac{\pi}{4}) = 1$.
\end{enumerate}
\end{exercise}
\begin{solution}
\leavevmode \\
\begin{enumerate}[label = \textbf{\alph*)}]
  \item Sei $y$ eine Lösung des Randwertproblems. Dann erfüllt
  $x = yy^{\prime}$ das Randwertproblem
  \begin{align}\label{i}
    x^{\prime} = 1 - 3t, \qquad x(0) = y(0)y^{\prime}(0) = 0, \qquad x(1) = y(1)y^{\prime}(1) = 0.
  \end{align}
  Betrachten wir davon nur die erste Randbedingung, erhalten wir ein Anfangswertproblem,
  welches eindeutig lösbar ist mit der Lösung
  \begin{align*}
    x(t) = \int_0^t 1 -3\tau d\tau = t - \frac{3}{2}t^2
  \end{align*}
  und $x(1) = -\frac{1}{2} \neq 0$. Damit hat das Randwertproblem \eqref{i}
  keine Lösung und somit kann das originale Problem ebenso nicht lösbar sein.
  \item Um eine Lösung des Systems $Ly = y^{\primeprime} + 4y = 0$
  zu erhalten, berechnen wir die Nullstellen des zugehörigen charakteristischen Polynoms
  \begin{align*}
    \chi(\lambda) = \lambda^2 + 4 \stackrel{!}{=} 0 \iff \lambda = \pm2i
  \end{align*}
  und bekommen die Lösungen $y_1(x) = \exp(2ix), y_2(x) = \exp(-2ix)$.
  Wir schreiben die ODE in ein System erster Ordnung um:
  \begin{align*}
    \begin{pmatrix}
      y \\ z
    \end{pmatrix}^{\prime} =
    \begin{pmatrix}
      0 & 1 \\ -4 & 0
    \end{pmatrix}
    \begin{pmatrix}
      y \\ z
    \end{pmatrix}
  \end{align*}
  Die zugehörige Fundamentalmatrix lautet
  \begin{align*}
    Y(x) = \begin{pmatrix}
      \exp(2ix) & \exp(-2ix) \\ 2i\exp(2ix) & -2i\exp(-2ix)
    \end{pmatrix}
  \end{align*}
  Wir stellen die Matrix $B$ auf:
  \begin{align*}
    B &= R_{-\pi}Y(-\pi) + R_{\pi}Y(\pi) =
    \begin{pmatrix}
      1 & 0 \\ 0 & 1
    \end{pmatrix}
    \begin{pmatrix}
      \exp(-2i\pi) & \exp(2i\pi) \\ 2i\exp(-2i\pi) & -2i\exp(2i\pi)
    \end{pmatrix} +
    \begin{pmatrix}
      -1 & 0 \\ 0 & -1
    \end{pmatrix}
    \begin{pmatrix}
      \exp(2i\pi) & \exp(-2i\pi) \\ 2i\exp(2i\pi) & -2i\exp(-2i\pi)
    \end{pmatrix} \\
    &= \begin{pmatrix}
      \exp(-2i\pi) - \exp(2i\pi) & \exp(2i\pi) - \exp(-2i\pi) \\
      2i\exp(-2i\pi) - 2i\exp(2i\pi) & -2i\exp(2i\pi) + 2i\exp(-2i\pi)
    \end{pmatrix}
    = \begin{pmatrix}
      0 & 0 \\
      0 & 0
    \end{pmatrix}
  \end{align*}
  Wir berechnen
  \begin{align*}
    \ell - R_{\pi}y_p(\pi) = \begin{pmatrix}
      0 \\ 0
    \end{pmatrix} +
    \begin{pmatrix}
      1 & 0 \\ 0 & 1
    \end{pmatrix}
    \begin{pmatrix}
      0 \\ 0
    \end{pmatrix}
    = \begin{pmatrix}
      0 \\ 0
    \end{pmatrix}
  \end{align*}
  Damit ist das System lösbar mit einem zwei-dimensionalen Lösungsraum.
  In der Tat sieht man, dass jede Linearkombination
  \begin{align*}
    y(x) = a\exp(2ix) + b\exp(-2ix)
  \end{align*}
  aufgrund $\exp(\pm2i\pi) = 1$
  \begin{align*}
    y(\pi) - y(-\pi) &= (a\exp(2i\pi) - a\exp(-2i\pi)) + (b\exp(-2i\pi) -b\exp(2i\pi)) = (a-a)+ (b-b) = 0 \\
    y^{\prime}(\pi) - y^{\prime}(-\pi) &= (2ai\exp(2i\pi) - 2ai\exp(-2i\pi)) + (-2bi\exp(-2i\pi) + 2bi\exp(2i\pi)) = 0 \\
  \end{align*}
  die Randwertbedingungen erfüllt. \\
  Berechnen wir nun die Partikulärlösung für das inhomogene System:
  \begin{align*}
    y_p(\pi) &= Y(\pi)\int_{-\pi}^\pi Y^{-1}(s)(4s) ds \\
    &= \begin{pmatrix}
      1 & 1 \\ 2i & -2i
    \end{pmatrix}\int_{-\pi}^\pi
    \frac{1}{4}\begin{pmatrix}
      2(\cos(2s) - i\sin(2s)) & -i\cos(2s) - \sin(2s) \\
      2(\cos(2s) + i\sin(2s)) & i\cos(2s) - \sin(2s)
    \end{pmatrix}
    \begin{pmatrix}
      0 \\ 4s
    \end{pmatrix} ds \\
    &= \begin{pmatrix}
      1 & 1 \\ 2i & -2i
    \end{pmatrix}\int_{-\pi}^x
    \begin{pmatrix}
      -is\cos(2s) - s\sin(2s) \\
      is\cos(2s) - s\sin(2s)
    \end{pmatrix}ds \\
    &= \frac{1}{4}\begin{pmatrix}
      1 & 1 \\ 2i & -2i
    \end{pmatrix}
    \begin{pmatrix}
      \pi \\
      \pi
    \end{pmatrix}
    = \begin{pmatrix}
      \frac{\pi}{2} \\ 0
    \end{pmatrix}
    \neq \begin{pmatrix}
      0 \\ 0
    \end{pmatrix}.
  \end{align*}
  Also ist das inhomogene System nicht lösbar.
  \item Das zugehörige charakteristische Polynom lautet
  \begin{align*}
    \chi(\lambda) = \lambda^2 + 2\lambda + 5 \stackrel{!}{=} 0
    \iff \lambda = -1 \pm 2i
  \end{align*}
  Umschreiben auf lineares System erster Ordnung liefert
  \begin{align*}
    \begin{pmatrix}
      y \\ z
    \end{pmatrix}^{\prime}
    = \begin{pmatrix}
      0 & 1 \\
      -5 & -2
    \end{pmatrix}
    \begin{pmatrix}
      y \\ z
    \end{pmatrix}
  \end{align*}
  Die Matrix $B$ lautet
  \begin{align*}
    B &= \begin{pmatrix}
      \exp((-1 + 2i)0) & \exp((-1 - 2i)0) \\ 0 & 0
    \end{pmatrix}
    +
    \begin{pmatrix}
       0 & 0\\ (-1 + 2i)\exp((-1 + 2i)\frac{\pi}{4}) & (-1 -2i)\exp((-1 - 2i)\frac{\pi}{4})
    \end{pmatrix} \\
    &= \begin{pmatrix}
      1 & 1\\
      (-1 + 2i)i\exp(\frac{\pi}{4}) & (1 + 2i)i\exp(\frac{\pi}{4})
    \end{pmatrix},
  \end{align*}
  welche klarerweise regulär ist und uns damit eindeutige Lösbarkeit liefert.
  Um diese zu bestimmen, lösen wir das Gleichungssystem
  \begin{align*}
    \begin{pmatrix}
      1 & 1\\
      (-1 + 2i)i\exp(\frac{\pi}{4}) & (1 + 2i)i\exp(\frac{\pi}{4})
    \end{pmatrix}
    \begin{pmatrix}
      a \\ b
    \end{pmatrix} =
    \begin{pmatrix}
      2 \\ 1
    \end{pmatrix}
  \end{align*}
  und erhalten $(a,b)^{\top} = (1+2i,1-2i)^{\top}$. Unsere eindeutige Lösung lautet daher
  \begin{align*}
    y(x) = (1+2i)\exp(-1 + 2i) + (1-2i)\exp(-1-2i).
  \end{align*}
\end{enumerate}
\end{solution}
