\begin{exercise}
  Lösen Sie die ODE vom \textit{Bernoullischen Typ}:

  \begin{align*}
    y' = f(t)y + g(t) y^n , \qquad n \neq 0,1
  \end{align*}

  mittels der Transformation $\tilde{y} = y^{1-n}$.
\end{exercise}

\begin{solution}
  Nach der Definition von $\tilde{y}$ gilt
  \begin{align*}
    \tilde{y}'(t) = (1-n)y^{-n}(t)y'(t) =& (1-n)y^{-n}(t)(f(t)y(t) + g(t)y^n(t)) \\
    =& (1-n)f(t)\tilde{y}(t) + (1-n)g(t)
  \end{align*}
  \begin{align}\label{eq1}
    \Rightarrow \tilde{y}' - (1-n)f(t)\tilde{y} = (1-n) g(t)
  \end{align}

  Man definiert nun $u(t) := \exp(-\int_0^t (1-n)f(\tau)d\tau)$, sodass gilt $u'(t) = -(1-n)f(t)u(t)$. Somit erhalten wir
  \begin{align*}
    \big(\tilde{y}(t)u(t)\big)' = \tilde{y}'(t)u(t) - (1-n)f(t)u(t) \tilde{y}(t)
    = u(t)\left(\tilde{y}'(t) - (1-n)f(t)\tilde{y}(t)\right)\stackrel{\eqref{eq1}}{=} (1-n)g(t)u(t)
  \end{align*}
  Integrieren auf beiden Seiten führt zu
  \begin{align*}
    u(t) \tilde{y}(t) = \int (1-n)g(\tau)u(\tau)d\tau \\
    \Rightarrow \tilde{y}(t) = \frac{\int (1-n)g(\tau)u(\tau)d\tau}{u(t)}.
  \end{align*}

  Nun nur noch Rücksubstitution von $\tilde{y} = y^{1-n}$:
  \begin{align*}
    y(t) = \tilde{y}^{\frac{1}{1-n}}(t) = & \left(\frac{\int (1-n)g(\tau)u(\tau)d\tau}{u(t)}\right)^{\frac{1}{1-n}} \\
    =& \left(\frac{\int (1-n)g(\tau) \exp\left(-\int_0^\tau (1-n)f(\xi)d\xi\right)d\tau}{\exp\left(-\int_0^t (1-n)f(\tau)d\tau\right)}\right)^{\frac{1}{1-n}} \\
    =& \exp\left(\int_0^t f(\tau)d\tau\right)\left(\int (1-n)g(\tau) \exp\left(-\int_0^\tau (1-n)f(\xi)d\xi\right)d\tau \right)^{\frac{1}{1-n}}
  \end{align*}
\end{solution}
