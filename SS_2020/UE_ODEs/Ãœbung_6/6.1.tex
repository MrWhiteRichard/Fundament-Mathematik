\begin{exercise}
  Gegeben sei die skalare ODE

  \begin{align*}
    x'' + q(t)x = 0
  \end{align*}

  mit einer stetigen Funktion $q: \R \rightarrow \R$

  Es seien $t \mapsto x(t)$ und $t \mapsto y(t)$ zwei Lösunger der ODE.
  Ihre Wronskideterminante ist durch

  \begin{align*}
    W(t) := x(t)y'(t)-x'(t)y(t)
  \end{align*}

  definiert. Die Lösungen $x(t)$ und $y(t)$ sind l.u. wenn
  $W(t) \neq 0, \forall t \in \R$ gilt.

  \begin{itemize}
    \item[a)] Zeigen sie, dass $W(t)$ konstant ist.
    \item[b)] Zeigen sie unter Verwendung von a), dass für l.u. Lösungen
    $x(t)$ und $y(t)$ gilt:
    \begin{itemize}
      \item[(i)] Aus $x(t_1) = 0$ folgt $x'(t_1) \neq 0$ und $y(t_1) \neq 0$
      \item[(ii)] Falls $x(t_1)=x(t_2)=0$ gilt und $x(t) \neq 0$ für
      $t \in (t_1 , t_2 )$ dann hat $y(t)$ in $(t_1 , t_2 )$ genau eine Nullstelle.
    \end{itemize}
  \end{itemize}

\end{exercise}

\begin{solution}
Beweisen wir diese Aussagen also.
\begin{itemize}
  \item[a)] Um das zu zeigen, berechnen wir einfach die Ableitung:
  \begin{align*}
    W'(t)= x'(t)y'(t) + x(t)y''(t) - x''(t)y(t) - x'(t)y'(t) =
    x'(t)y'(t) - x(t)q(t)y(t) + x(t)q(t)y(t) - x'(t)y'(t) = 0
  \end{align*}
  also ist die Wronskideterminante wirklich konstant.

  \item[b)] Ad (i): Da $x$ und $y$ unabhängige Lösungen sind wissen wir durch ihre
  Wronskideterminante:

  \begin{align*}
    x(t)y'(t)-x'(t)y(t) = c, \qquad c \in \R \backslash \{0\}
  \end{align*}

  Gilt also an einem Punkt $t_1: x(t_1 ) = 0$ sehen wir, dass dort

  \begin{align*}
    -x'(t_1 )y(t_1 ) = c
  \end{align*}

  Da $c \neq 0$ können wir daraus die Behauptung sofort schließen.

  Ad (ii): Nach (i) wissen wir: $y(t_1 ) \neq 0, y(t_2 ) \neq 0$. Wir können auch

  \begin{align} \label{Vorzeichen}
    -x'(t_1 )y(t_1 ) = c = -x'(t_2 )y(t_2 )
    \Rightarrow x'(t_1 )y(t_1 ) = x'(t_2 )y(t_2 )
  \end{align}

  schließen. Nun zeigen wir noch $\sgn(x'(t_1 )) = - \sgn(x'(t_2 ))$. Dazu bemerken wir
  zuerst, dass $x'(t_ i ) \neq 0, i=1,2$ da sonst $W(t_i) = 0$ im Widerspruch zu
  unserer Voraussetzung. Wenn nun $\sgn(x'(t_1 )) = 1$ gilt heißt das auch
  $\forall t \in (t_1, t_2): x(t)>0$ damit muss klarerweise $\sgn(x'(t_2 ))=-1$.
  \begin{align*}
    x^{\prime}(t_1) &= \lim_{h \rightarrow 0} \frac{x(t_1 + h) -x(t_1)}{h} = \frac{x(t_1 + h)}{h} > 0 \\
    x^{\prime}(t_2) &= \lim_{h \rightarrow 0} \frac{x(t_2 - h) -x(t_2)}{-h} = \frac{x(t_2 - h)}{-h} < 0
  \end{align*}
  Gemeinsam mit \eqref{Vorzeichen} sehen wir, dass $\sgn(y(t_1 )) = -\sgn(y(t_2 ))$.
  Dazu muss $y$ also einen Vorzeichenwechsel in $(t_1 , t_2 )$ und folglich auch
  zumindest eine Nullstelle haben.

  Um die Eindeutigkeit jener zeigen wir mit einem Widerspruchsbeweis: Sei $\tilde{t}$
  die kleinste Nullstelle von $y$ und $\hat{t}$ die nächste weitere Nullstelle in
  $(t_1 , t_2 )$. Es gilt also:

  \begin{align*}
    x( \tilde{t} ) y'( \tilde{t} ) = x( \hat{t} ) y'( \hat{t} )
  \end{align*}

  Da nun aber $x$ keine Vorzeichenwechsel in dem Intervall hat folgt daraus
  $\sgn(y^{\prime}(\tilde{t})) = \sgn(y^{\prime}(\hat{t}))$. Wir haben aber auch
  $\forall t \in (\tilde{t},\hat{t}): y(t) \neq 0 $ vorausgesetzt. Mit den selben Überlegungen
  wie bei $x$ weiter oben ist dies jedoch ein Widerspruch.
\end{itemize}
\end{solution}
