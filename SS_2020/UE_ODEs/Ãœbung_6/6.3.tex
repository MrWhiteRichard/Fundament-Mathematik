\begin{exercise}
  In Aufgabe 5.1 haben wir homogene (skalare) ODEs kennengelernt und durch
  Substitution auf eine separierte ODE zurückgeführt. Ein allgemeinerer Typ ist
  von der Form

  \begin{align} \label{allgemeiner Typ}
    y' = f \bigg(\frac{ay + bt + c}{\alpha y + \beta t + \gamma}
    \bigg)
  \end{align}

  für Konstanten $a,b,c,\alpha ,\beta ,\gamma$. Ziel ist es, durch Variablensubstitution
  eine seperierbare ODE zu erhalten. Definieren Sie hierzu die Matrix

  \begin{align*}
    A = \left(
      \begin{array}{cc}
        a & b \\
        \alpha & \beta
      \end{array}
    \right)
  \end{align*}

  \begin{itemize}
    \item[a)] Betrachten Sie den Fall $\det A = 0$. Hinweis: Bemerken Sie, dass
    $ay +bt = \lambda (\alpha y + \beta t)$ für ein $\lambda \in \R$. Überlegen
    Sie sich, dass Sie durch Substitution sogar die Form
    \begin{align*}
      \tilde{y'}=\tilde{f}(\tilde{y})
    \end{align*}
    erhalten können.

    \item[b)] Geben Sie für den Fall $\det A \neq 0$ eine Funktion $\tilde{f}$ an,
    die das Gewünschte leistet. Hinweis: betrachten Sie die Lösung $(t_0 , y_0 )$
    von
    \begin{align*}
      A\left(
      \begin{array}{c}
        y_0 \\
        t_0
      \end{array}
      \right) = -\left(
      \begin{array}{c}
        c \\
        \gamma
      \end{array}
      \right)
    \end{align*}
    Zeigen Sie, dass dann die Funktion $\tilde{y}(t) := y(t + t_0 ) - y_0$ eine
    ODE der Form \eqref{allgemeiner Typ} mit $c = \gamma = 0$ erfüllt. Überlegen
    Sie sich, dass letzere Form eifach auf die Form einer homogenen ODE gebracht
    werden kann.
  \end{itemize}
\end{exercise}

\begin{solution}
  To do!
\end{solution}
