\begin{definition}
    Eine skalare ODE heißt genau dann separabel, wenn sie von der Form
    \begin{align*}
        y^\prime = f(y)g(t) 
    \end{align*}
    ist.
\end{definition}

\begin{exercise}
    Betrachten Sie das AWP $y^\prime = f(y)g(t)$ mit $y(t_0) = y_0$. 
    \begin{enumerate}[label = \alph*)]
        \item Falls $f$ und $g$ stetig (bei $y_0$ und $t_0$) sind und $f(y_0) \neq 0$, dann sit das AWP (eindeutig) lösbar. Die Lösung ist charakterisiert durch
        \begin{align}
            F(y(t)) - G(t) + c = 0, \label{charakter}
        \end{align}
        wobei $c \in \R$ geeignet gewählt ist und $F^\prime(y) = \frac{1}{f(y)}$ und $G^\prime(t) = g(t)$. Was ist $c$? Was passiert im Fall $f(y_0) = 0$?
        \item Lösen Sie das AWP 
        \begin{align*}
            y^\prime = y^2, \qquad y(0) = y_0
        \end{align*}
        \item Lösen Sie das AWP (logistische ODE)
        \begin{align*}
            y^\prime = y(1 - y), \qquad y(0) = y_0.
        \end{align*}
        \item Lösen Sie das AWP
        \begin{align*}
            y^\prime = \cos(t) \pbraces{\cos(y)}^2, \qquad y(0) = 0
        \end{align*}
    \end{enumerate}
\end{exercise}
\begin{solution}
    Hier könnte Ihre Werbung stehen!
    \begin{enumerate}[label = \alph*)]
        \item Der Existenzsatz von Peano garantiert eine Lösung der Differentialgleichung, zumindest in einer kleinen Umgebung von $t_0$. Der Hauptsatz der Differential- und Integralrechnung garantiert Funktionen $F$ und $G$ gemäß Angabe. Durch einmaliges Ableiten von \eqref{charakter} ergibt sich, dass diese Gleichung das AWP charakterisiert, wobei $c = G(t_0) - F(y_0)$ gelten muss. Im Fall $f(y_0) = 0$ ist jedenfalls $y(t) = y_0$ eine Lösung der Differentialgleichung.
        
        Seien nun $y$ und $z$ zwei Lösungen der Differentialgleichung, wobei wieder $f(y_0) \neq 0$ gelten soll. Dann gilt zumindest in einer Umgebung um $t_0$
        \begin{align*}
            \frac{y^\prime(t)}{f(y(t))} = \frac{z^\prime(t)}{f(z(t))}
        \end{align*}
        und daraus folgt wegen $y(t_0) = y_0 = z(t_0)$ auch $F(y(t)) = F(z(t))$. Da $F$ wegen der Stegtigkeit von $f$ in einer kleinen Umgebung um $y_0$ streng monoton wachsend und damit bijektiv ist, muss in dieser Umgebung sicher $y(t) = z(t)$ gelten.
        
        Offene Fragen: Was ist das maximale Existenzintervall der Lösung? Stimmen $z$ und $y$ überall wo es eine Lösung gibt überein? Ist die Lösung im Fall $f(y_0) = 0$ eindeutig?.
        \item fehlt
        \item Wir berechnen
        \begin{align*}
            \int \frac{1}{y(1 - y)} = -\log\pbraces{\frac{1 - y}{y}}
        \end{align*}
        und erhalten so für die Lösung
        \begin{align*}
            -\log\pbraces{\frac{1 - y(t)}{y(t)}} = t + \log(c) \Leftrightarrow \frac{1 - y(t)}{y(t)} = c\rm{e}^{-t} \Leftrightarrow y(t) = \frac{1}{1 + c\rm{e}^{-t}}
        \end{align*}
        und aus der Anfangsbedingung ermitteln wir $c$ zu
        \begin{align*}
            y_0 = y(0) = \frac{1}{1 + c} \Leftrightarrow c = \frac{1}{y_0} - 1,
        \end{align*}
        insgesamt gilt also (vlg. Skriptum S. 2)
        \begin{align*}
            y(t) = \frac{1}{1 + \pbraces{\frac{1}{y_0} - 1} \rm{e}^{-t}}.
        \end{align*}
        \item Auch hier berechnen wir 
        \begin{align*}
            \int \frac{1}{\pbraces{\cos(y)}^2} = \tan(y)
        \end{align*}
        und erhalten so für die Lösung der Differentialgleichung
        \begin{align*}
            \tan(y(t)) = \sin(t) + c \Leftrightarrow y(t) = \arctan(\sin(t) + c)
        \end{align*}
        und die Anfangsbedingung liefert
        \begin{align*}
            0 = y(0) = \arctan(c) \Leftrightarrow c = 0
        \end{align*}
        womit die Lösung insgesamt die Gestalt
        \begin{align*}
            y(t) = \arctan(\sin(t))
        \end{align*}
        hat.
    \end{enumerate}
\end{solution}