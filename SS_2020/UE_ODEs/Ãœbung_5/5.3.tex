\begin{exercise}
Welche der folgenden Funktionen kann eine Lösung einer ODE $y^{\prime}
= Ay$ mit $A \in \mathbb{R}^{2\times 2}$ sein?
\begin{align}
  y_1(t) &= (3\exp(t) + \exp(-t), \exp(2t))^{\top} \label{f1}\\
  y_2(t) &= (3\exp(t) + \exp(-t), \exp(t))^{\top} \label{f2}\\
  y_3(t) &= (3\exp(t) + \exp(-t), t\exp(t))^{\top} \label{f3} \\
  y_4(t) &= (3\exp(t), t^2\exp(t))^{\top} \label{f4} \\
  y_5(t) &= (\exp(t) + 2\exp(-t), \exp(t) + 2\exp(-t))^{\top} \label{f5}
\end{align}
\end{exercise}
\begin{solution}
Betrachten wir zuerst den Fall, dass $A$ zwei Eigenwerte besitzt.
Eine Lösung der ODE $y^{\prime} = Ay$ muss sich als Linearkombination
der Basisfunktionen
\begin{align*}
  y^1(t) := (\exp(\lambda_1t)v_1,\exp(\lambda_1t)v_2)^{\top}, \\
  y^2(t) := (\exp(\lambda_2t)w_1,\exp(\lambda_2t)w_2)^{\top}, \\
\end{align*}
darstellen lassen, wobei $(\lambda_1,v), (\lambda_2,w)$ die beiden Eigenpaare
der Matrix $A$ sind.
Eine allgemeine Lösung der ODE lautet also folgendermaßen:
\begin{align*}
  y(t) = c_1(\exp(\lambda_1t)v_1,\exp(\lambda_1t)v_2)^{\top} + c_2(\exp(\lambda_2t)w_1,\exp(\lambda_2t)w_2)^{\top}
\end{align*}
Die Funktionen $y_3,y_4$ fallen sofort als Kandidaten aus, da sie sich überhaupt
nicht in dieser Form darstellen lassen. Ebenso kann $y_1$ keine Lösung sein,
da dafür drei unterschiedliche Eigenwerte benötigt werden würden.
\begin{align*}
  y_5(t) = \exp(1t)(1,1)^{\top} + \exp(-1t)(2,2)^{\top}
\end{align*}
Folglich müssten die Eigenpaare $(1,(1,1)^{\top},-1,(2,2)^{\top})$ lauten, was klarerweise
unmöglich ist.
Es bleibt also nur noch $y_2$ übrig. Diese Funktion stellt in der Tat eine mögliche
Lösung dar, da
\begin{align*}
  y_2(t) = \exp(1t)(3,1)^{\top} + \exp(-1t)(1,0)
\end{align*}
mit $(1,0), (3,1)$ zwei linear unabhänige Eigenvektoren gegeben sind.
Die zugehörige Matrix $A$ lautet
\begin{align*}
A =
  \begin{pmatrix}
    1 & 3 \\ 0 & 1
  \end{pmatrix}
  \begin{pmatrix}
    -1 & 0 \\ 0 & 1
  \end{pmatrix}
  \begin{pmatrix}
    1 & 3 \\ 0 & 1
  \end{pmatrix}^{-1}
  =
  \begin{pmatrix}
    -1 & 6 \\ 0 & 1
  \end{pmatrix}
\end{align*}
Für den Fall, dass $A$ nur ein Eigenpaar $(\lambda,v)$ und einen Hauptvektor $w$
besitzt, lautet die allgemeine Lösung
\begin{align*}
  y(t) = \exp(\lambda t)(c_1v_1 + c_2(w_1 + tv_1),c_1v_2 + c_2(w_2 + tv_2))^{\top}
\end{align*}
Allerdings lässt sich in diesem Fall keine der obigen Funktionen auf so eine Form
mit $c_2 \neq 0$ bringen. Also bleibt die einzig mögliche Lösung $y_2$.
\end{solution}
