\begin{exercise}
\textit{Homogene ODEs} der Form $y^{\prime}(t) = f(\nicefrac{y}{t})$
können mit der Substitution $\nicefrac{y}{t} = x$ vereinfacht werden.
Geben Sie die allgemeine Lösung an für:
\begin{itemize}
  \item [\textbf{a)}] $y^{\prime}(t) = \left(\nicefrac{y}{t}\right)^2
  + 2\nicefrac{y}{t}$
  \item [\textbf{b)}] $y^{\prime}(t) = \nicefrac{y}{t} - \tan(\nicefrac{y}{t})$
\end{itemize}
\end{exercise}
\begin{solution}
\leavevmode \\
\begin{itemize}
  \item [\textbf{a)}] Wir substituieren $x(t) := \nicefrac{y(t)}{t}$ und erhalten
  \begin{align*}
    (tx)^{\prime} = x^2 + 2x \iff x + tx^\prime = x^2 + 2x \iff t x^\prime = x^2 + x \iff \frac{x^{\prime}}{x^2 + x} = \frac{1}{t}
  \end{align*}
  Jetzt können wir den Ansatz für separable ODEs verwenden.
  \begin{gather*}
    \int_{t_0}^t \frac{x^{\prime}(\tau)}{x(\tau)^2 + x(\tau)}d\tau = \int_{t_0}^t \frac{1}{\tau}d\tau \\
    \iff \int_{x(t_0)}^{x(t)} x^{-1} dx - \int_{x(t_0)}^{x(t)} (x + 1)^{-1} dx = \int_{x(t_0)}^{x(t)} \frac{1 + x - x}{x(x + 1)} dx = \int_{x(t_0)}^{x(t)} \frac{1}{x^2 + x}dx = \ln(t) - \ln(t_0)\\
    \iff \ln(x(t)) - \ln(x(t) + 1) - \ln(x(t_0)) + \ln(x(t_0) + 1) = \ln(t) - \ln(t_0)\\
    \iff \frac{x(t)(x(t_0) + 1)}{x(t_0)(x(t) + 1)} = \frac{t}{t_0} \\
    x(t)t_0(x(t_0) + 1) = x(t_0)t(x(t) + 1) \\
    \iff x(t)(t_0(x(t_0) + 1) - x(t_0)t) = x(t_0)t \\
    \iff x(t) = \frac{x(t_0)t}{t_0(x(t_0) + 1) - x(t_0)t}
  \end{gather*}
  Wir machen die Probe
  \begin{align*}
    x^{\prime} &= \frac{x(t_0)(t_0(x(t_0) + 1) - x(t_0)t) + x(t_0)^2t}{(t_0(x(t_0) + 1) - x(t_0)t)^2}
    = \frac{x(t_0)t_0(x(t_0) + 1)}{(t_0(x(t_0) + 1) - x(t_0)t)^2}\\
    &\stackrel{!}{=} \frac{x^2 + x}{t}
    = \frac{x(t_0)^2t^2}{t(t_0(x(t_0) + 1) - x(t_0)t)^2} + \frac{x(t_0)t}{t(t_0(x(t_0) + 1) - x(t_0)t)} \\
    &= \frac{x(t_0)^2t + x(t_0)(t_0(x(t_0) + 1) - x(t_0)t)}{(x(t_0)t - t_0(x(t_0) - 1))^2}\\
    &= \frac{x(t_0)^2t + x(t_0)^2t_0 - x(t_0)^2t + x(t_0)t_0)}{(x(t_0)t - t_0(x(t_0) - 1))^2}\\
    &= \frac{x(t_0)^2t_0 + x(t_0)t_0}{(x(t_0)t - t_0(x(t_0) - 1))^2}
    = \frac{x(t_0)t_0(x(t_0) + 1)}{(t_0(x(t_0) + 1) - x(t_0)t)^2}\\
  \end{align*}
  und frohlocken.
  Unsere endgültige Lösung lautet dann
  \begin{align*}
    y(t) = tx(t) = \frac{x(t_0)t^2}{t_0(x(t_0) + 1) - x(t_0)t}
  \end{align*}
  \item [\textbf{b)}]
  Mit der selben Methode wie in a) erhalten wir
  \begin{align*}
    (tx)^{\prime} &= x - \tan(x) \iff x + tx^\prime = x - \tan(x) \iff tx^\prime = -\tan(x) \iff \frac{x^{\prime}}{-\tan(x)} = \frac{1}{t}, 
  \end{align*}
  eine separable ODE, die wir lösen können. 
  Wir berechnen zuerst eine Stammfunktion 
  \begin{align*}
    \int \frac{1}{\tan(x)} = \int \frac{\cos(x)}{\sin(x)} \quad \vbraces{\begin{array}{c} u = \sin(x) \\ dx = (\cos(x))^{-1} du \end{array}} = \int u^{-1} = \ln(u) = \ln(\sin(x))
  \end{align*}
  und verwenden das um die Lösung unserer ODE zu finden.
  \begin{align*}
    \int_{t_0}^t \frac{x^{\prime}(\tau)}{-\tan(x(\tau))}d\tau &= \int_{t_0}^t \frac{-1}{\tau}d\tau \\
    \int_{x(t_0)}^{x(t)} \frac{1}{-\tan(x)}dx &= \ln(t) - \ln(t_0) \\
    \ln(\sin(x(t_0))) - \ln(\sin(x(t))) &= \ln(t) - \ln(t_0) \\
    \ln(\sin(x(t))) &= \ln(\sin(x(t_0))) - \ln(t) + \ln(t_0) \\
    \sin(x(t)) &= \frac{\sin(x(t_0))t_0}{t} \\
    x(t) &= \arcsin\left(\frac{\sin(x(t_0))t_0}{t}\right)
  \end{align*}
  Unsere Lösung lautet also
  \begin{align*}
    y(t) = tx(t) = t\arcsin\left(\frac{\sin(x(t_0))t_0}{t}\right).
  \end{align*}
  Wir machen die Probe
  \begin{align*}
    y^\prime(t) &= \arcsin\left(\frac{\sin(x(t_0))t_0}{t}\right) - t \sin(x(t_0))t_0 t^{-2} \pbraces{\sqrt{1 - \frac{(\sin(x(t_0)) t_0)^2}{t^2}}}^{-1} \\
    &= \frac{y(t)}{t} - \sin\pbraces{\arcsin\left(\frac{\sin(x(t_0))t_0}{t}\right)} \pbraces{ \sqrt{1 - \pbraces{\sin\pbraces{\arcsin\left(\frac{\sin(x(t_0))t_0}{t}\right)}}^2} }^{-1} \\
    &= \frac{y(t)}{t} - \sin\pbraces{\frac{y(t)}{t}} \pbraces{\cos\pbraces{\arcsin\left(\frac{\sin(x(t_0))t_0}{t}\right)}}^{-1} \\ 
    &= \frac{y(t)}{t} - \sin\pbraces{\frac{y(t)}{t}} \pbraces{\cos\pbraces{\frac{y(t)}{t}}}^{-1} = \frac{y(t)}{t} - \tan\pbraces{\frac{y(t)}{t}}
  \end{align*}
  und frohlocken.
\end{itemize}
\end{solution}
