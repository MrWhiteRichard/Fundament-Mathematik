\begin{exercise}
Sei $A \in \mathbb{R}^{d\times d}$.
\begin{itemize}
  \item [\textbf{a)}]Zeigen Sie: Ist $(v,\lambda) \in \mathbb{C}^d
  \times \mathbb{C}$ ein Eigenpaar von $A$, so sind die Funktionen
  $y_1(t) = \exp(\lambda t)v$ und $y_2(t) = \exp(\bar{\lambda}t)\bar{v}$
  Lösungen der ODE $y^{\prime} = Ay$ an.
  \item [\textbf{b)}] Nehmen Sie an, dass $A \in \mathbb{R}^
  {d \times d}$ (komplex) diagonalisierbar ist. Geben Sie ein
  \textit{reelles} Fundamentalsystem der ODE $y^{\prime} = Ay$ an.
\end{itemize}
\end{exercise}
\begin{solution}
\leavevmode \\
\begin{itemize}
  \item [\textbf{a)}]
  Mit der Rechenregel $\bar{a}\bar{b} = \bar{ab}$ und der Tatsache, dass $\bar{A} = A$
  sehen wir ein, dass
  \begin{align*}
    y_1^{\prime}(t) &= \exp(\lambda t)\lambda v = \exp(\lambda t)A v = Ay_1(t) \\
    y_2^{\prime}(t) &= \exp(\bar{\lambda}t)\bar{\lambda}\bar{v} = \exp(\bar{\lambda}t) \bar{\lambda v} =
    \exp(\bar{\lambda}t) \bar{Av} =  \exp(\bar{\lambda}t) \bar{A}\bar{v} = A\exp(\bar{\lambda}t)\bar{v} = Ay_2(t)
  \end{align*}
  gilt.
  \item [\textbf{b)}]
  Seien $(\lambda_i,\dots,\lambda_n,\bar{\lambda_1},\dots,\bar{\lambda_n},a_1,\dots,a_m)$
  mit $a_i,\dots,a_m \in \mathbb{R}$ die Eigenwerte, \\
  $(v_i,\dots,v_n,\bar{v_1},\dots,\bar{v_n},x_1,\dots,x_m)$, $x_1,\dots,x_m \in \mathbb{R}$
  die zugehörigen Eigenvektoren von $A$.
  Für jedes komplexe Eigenpaar $(\lambda_i, v_i), i = 1,\dots,n$, sowie jedes
  reelle Eigenpaar $(a_j,x_j), j = 1,\dots,m$ stellen die Funktionen
  \begin{align*}
    y_{i}(t) &= \exp(\lambda t)v_i, \\
    y_{n + i}(t) &= \exp(\bar{\lambda} t)\bar{v_i} \\
    y_{2n + j} &= \exp(a_j)x_j
  \end{align*}
  Lösungen der Differentialgleichung dar. Damit mit $\lambda_i = b_i + ic_i$ sind sowohl
  \begin{align*}
    z_i(t) &:= y_{i1}(t) + y_{i2}(t) = \exp(\lambda_i t)v_i + \exp(\bar{\lambda_i} t)\bar{v_i}
    = \exp(\lambda_i t)v_i + \bar{\exp(\lambda_i t)}\bar{v_i}
    = \exp(\lambda_i t)v_i + \bar{\exp(\lambda_i t)v_i}
    = 2\mathfrak{R}(\exp(\lambda_i t)v_i)
    = 2\mathfrak{R}(\exp(b_i)(\cos(c_i)+i\sin(c_i))v_i) \\
    &= 2\exp(b_i)\mathfrak{R}(\cos(c_i)v_i+i\sin(c_i)v_i)
    = 2\exp(b_i)\left(\cos(c_i)\mathfrak{R}(v_i) - \sin(c_i)\mathfrak{I}(v_i))\right)
  \end{align*}
  als auch
  \begin{align*}
    z_{n+i}(t) &:= -i(y_{i1}(t) - y_{i2}(t)) = 2\mathfrak{I}(\exp(\lambda_i t)v_i)
    = 2\exp(b_i)\mathfrak{I}(\cos(c_i)v_i+i\sin(c_i)v_i) \\
    &= 2\exp(b_i)\left(\cos(c_i)\mathfrak{I}(v_i) - \sin(c_i)\mathfrak{R}(v_i)\right)
  \end{align*}
  reelle Lösungen der Differentialgleichung.
  Für die verbleibenden reellen Eigenvektoren können wir $y_i$ einfach als
  \begin{align*}
    z_i(t) := \exp(a_i t)x_i
  \end{align*}
  definieren. Anders formuliert erhalten wir unser reelles Fundamentalsystem
  durch Anwenden der regulären Matrix
  \begin{align*}
    A = \begin{pmatrix}
      1 & 0 & 0 & -i & 0 & 0 & 0 & 0 & 0\\
      0 & \ddots & 0 & 0 & \ddots & 0 & 0 & 0 & 0\\
      0 & 0 & 1 & 0 & 0 & -i & 0 & 0 & 0\\
      1 & 0 & 0 & i & 0 & 0 & 0 & 0 & 0\\
      0 & \ddots & 0 & 0 & \ddots & 0 & 0 & 0 & 0\\
      0 & 0 & 1 & 0 & 0 & i & 0 & 0 & 0\\
      0 & 0 & 0 & 0 & 0 & 0 & 1 & 0 & 0 \\
      0 & 0 & 0 & 0 & 0 & 0 & 0 & \ddots & 0 \\
      0 & 0 & 0 & 0 & 0 & 0 & 0 & 0 & 1
    \end{pmatrix}
  \end{align*}
  auf die komplexe Fundamentalmatrix $Y(t)$. Daher müssen unsere Funktionen $(z_i)_{i=1}^d$
  ebenso linear unabhängig sein und bilden somit ein reelles Fundamentalsystem.
\end{itemize}
\end{solution}
