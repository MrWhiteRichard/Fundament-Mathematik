\begin{exercise}
Geben Sie reelle Fundamentalsysteme für die Systeme $y^{\prime} = A_iy$
an, wobei
\begin{align*}
  A_1 &= \begin{pmatrix}
    -1 & 1 & -1 \\
    2 & -1 & 2 \\
    2 & 2 & -1
  \end{pmatrix} \\
  A_2 &= \begin{pmatrix}
    3 & -3 & 2 \\
    -1 & 5 & -2 \\
    -1 & 3 & 0
  \end{pmatrix} \\
  A_3 &= \begin{pmatrix}
    6 & -17 \\
    1 & -2
  \end{pmatrix}
\end{align*}
\end{exercise}
\begin{solution}
Aus Satz 3.15 wissen wir, dass $Y(t) = \exp(tA)$ eine allgemeine Fundamentalmatrix für $y^{\prime} = Ay$
darstellt. $\exp(tA)$ können wir dann durch Transformation von $A = VDV^{-1}$ in Jordan-Normalform berechnen.
Wenn $A$ diagonalisierbar ist, können wir die allgemeine Lösung zu
\begin{align*}
  Y(t) = exp(tA) = V\exp(tD)V^{-1}
\end{align*}
vereinfachen. Da $V$ regulär ist, ist auch $v\exp(tD)$ schon eine Fundamentalmatrix.
\begin{itemize}
  \item $A_1$ ist diagonalisierbar mit den Eigenpaaren
  \begin{align*}
  (\lambda_1,v_1) = (-3,(2,-3,1)^{\top})\\
  (\lambda_2,v_2) = (-1,(-1,1,1)^{\top}) \\
  (\lambda_3,v_3) = (1,(0,1,1)^{\top}))
  \end{align*}
  Ein reelles Fundamentalsystem von $A_1$ lautet damit
  \begin{align*}
    Y(t) = \begin{pmatrix}
      2 & -1 & 0 \\ -3 & 1 & 1 \\ 1 & 1 & 1
    \end{pmatrix}
    \begin{pmatrix}
      \exp(-3t) & 0 & 0 \\
      0 & \exp(-t) & 0 \\
      0 & 0 & \exp(t)
    \end{pmatrix} =
    \begin{pmatrix}
      2\exp(-3t) & -1\exp(-t) & 0 \\
      -3\exp(-3t) & 1\exp(-t) & 1\exp(t) \\
      1\exp(-3t) & 1\exp(-t) & 1\exp(t)
    \end{pmatrix}.
  \end{align*}
  \item $A_2$ ist diagonalisierbar mit den Eigenpaaren
  \begin{align*}
  (\lambda_1,v_1) = (4,(-1,1,1)^{\top}) \\
  (\lambda_2,v_2) = (2,(-2,0,1)^{\top}) \\
  (\lambda_3,v_3) = (2,(3,1,0)^{\top})
  \end{align*}
  Ein reelles Fundamentalsystem von $A_2$ lautet damit
  \begin{align*}
    Y(t) = \begin{pmatrix}
    -1\exp(4t) & -2\exp(2t) & 3\exp(2t) \\
    1\exp(4t) & 0\exp(2t) & 1\exp(2t) \\
    1\exp(4t) & 1\exp(2t) & 0\exp(2t)
    \end{pmatrix}.
  \end{align*}
  \item $A_3$ ist komplex diagonalisierbar mit den Eigenpaaren
  \begin{align*}
    (\lambda_1,v_1) = (2 + i,(4+i,1)^{\top}) \\
    (\lambda_2,v_2) = (2-i,(4-i,1)^{\top})
  \end{align*}
  Damit erhalten wir zunächst direkt ein komplexes Fundamentalsystem.
  \begin{align*}
  Y(t) = \begin{pmatrix}
    (4+i)\exp((2+i)t)  & (4-i)\exp((2-i)t)(\cos(t) + 4\sin(t)) \\
    (4+i)\exp((2+i)t)\cos(t) & (4-i)\exp((2-i)t)\sin(t)
  \end{pmatrix}
  \end{align*}
  Durch die Linearkombinationen $\frac{1}{2}(\exp(t\lambda_1)v_1 + \exp(t\lambda_2)v_2)$, sowie
  $\frac{-i}{2}(\exp(t\lambda_1)v_1 + \exp(t\lambda_2)v_2)$ der beiden Spalten und der Identität
  $\exp(x + iy) = \exp(x)(\cos(y) + i\sin(y))$ erhalten wir das gewünschte reelle
  Fundamentalsystem.
  \begin{align*}
    Y(t) = \begin{pmatrix}
      \exp(2t)(4\cos(t) -\sin(t))  & \exp(2t)(\cos(t) + 4\sin(t)) \\
      \exp(2t)\cos(t) & \exp(2t)\sin(t)
    \end{pmatrix}
  \end{align*}
\end{itemize}


\end{solution}
