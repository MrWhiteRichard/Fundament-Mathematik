\begin{exercise}
Geben Sie reelle Fundamentalsysteme für die Systeme $y^{\prime} = A_iy$
an, wobei
\begin{align*}
  A_1 &= \begin{pmatrix}
    -1 & 1 & -1 \\
    2 & -1 & 2 \\
    2 & 2 & -1
  \end{pmatrix} \\
  A_2 &= \begin{pmatrix}
    3 & -3 & 2 \\
    -1 & 5 & -2 \\
    -1 & 3 & 0
  \end{pmatrix} \\
  A_3 &= \begin{pmatrix}
    6 & -17 \\
    1 & -2
  \end{pmatrix}
\end{align*}
\end{exercise}
\begin{solution}
\leavevmode \\
\begin{itemize}
  \item [$A_1$]
  $(-3,(2,-3,1)^{\top}),(-1,(-1,1,1)^{\top},(1,(0,1,1)^{\top}))$ sind alle
  Eigenpaare von $A_1$.
  Ein reelles Fundamentalsystem von $A_1$ lautet damit
  \begin{align*}
    Y(t) = \begin{pmatrix}
      2\exp(-3t) & -1\exp(-t) & 0\exp(t) \\
      -3\exp(-3t) & 1\exp(-t) & 1\exp(t) \\
      1\exp(-3t) & 1\exp(-t) & 1\exp(t)
    \end{pmatrix}.
  \end{align*}
  \item [$A_2$]
  $(4,(-1,1,1)^{\top}),(2,(-2,0,1)^{\top}),(2,(3,1,0)^{\top})$
  sind alle Eigenpaare von $A_2$.
  Ein reelles Fundamentalsystem von $A_2$ lautet damit
  \begin{align*}
    Y(t) = \begin{pmatrix}
    -1\exp(4t) & -2\exp(2t) & 3\exp(2t) \\
    1\exp(4t) & 0\exp(2t) & 1\exp(2t) \\
    1\exp(4t) & 1\exp(2t) & 0\exp(2t)
    \end{pmatrix}.
  \end{align*}
  \item [$A_3$]
  $(2 + i,(4+i,1)^{\top}),(2-i,(4-i,1)^{\top})$ sind alle Eigenpaare von $A_3$.
  Damit erhalten wir zunächst direkt ein komplexes Fundamentalsystem.
  \begin{align*}
  Y(t) = \begin{pmatrix}
    (4+i)\exp((2+i)t)  & (4-i)\exp((2-i)t)(\cos(t) + 4\sin(t)) \\
    (4+i)\exp((2+i)t)\cos(t) & (4-i)\exp((2-i)t)\sin(t)
  \end{pmatrix}
  \end{align*}
  Durch Linearkombinationen der beiden Spalten und der Identität
  $\exp(x + iy) = \exp(x)(\cos(y) + i\sin(y))$ erhalten wir das gewünschte reelle
  Fundamentalsystem.
  \begin{align*}
    Y(t) = \begin{pmatrix}
      \exp(2t)(4\cos(t) -\sin(t))  & \exp(2t)(\cos(t) + 4\sin(t)) \\
      \exp(2t)\cos(t) & \exp(2t)\sin(t)
    \end{pmatrix}
  \end{align*}
\end{itemize}


\end{solution}
