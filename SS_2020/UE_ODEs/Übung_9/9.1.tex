\begin{exercise}
Ziel dieser Aufgabe ist eine Beziehung zwischen der linearisierten Stabilität und
dem Konzept der Ljapunovfunktion. Genauer: Wir zeigen die Äquivalenz der folgenden
beiden Aussagen für eine Matrix $A \in R^{d \times d}$:
\begin{enumerate}
  \item Die Ruhelage $y^* = 0$ der ODE $y^{\prime} = Ay$ ist asymptotisch stabil.
  \item Es gibt eine symmetrisch positiv definite Lösung der Matrixgleichung
  (\glqq Ljapunovgleichung\grqq)
  \begin{align} \label{ljapunov}
    A^{\top}Q + QA = -I.
  \end{align}
\end{enumerate}
\begin{enumerate}[label = \textbf{\alph*)}]
  \item Sei die Ruhelage $y^* = 0$ asymptotisch stabil. Definieren Sie die Matrix
  \begin{align*}
    Q := \int_{t = 0}^{\infty} \exp(tA)^{\top}\exp(tA) dt
  \end{align*}
  Zeigen Sie: $Q$ ist symmetrisch positiv definit (insbesondere also $(Qx,x)_2 > 0$
  für $x \neq 0$) und erfüllt \eqref{ljapunov}. Wie Teilaufgabe b) zeigen wird,
  ist die Funktion $V(x) := (Qx,x)_2$ eine strikte Ljapunovfunktion für die obige ODE.
  \item Sei $Q$ eine symmetrisch positiv definite Lösung von \eqref{ljapunov}.
  Zeigen Sie: $V(x) := (Qx,x)_2$ ist eine strikte Ljapunovfunktion für die obige ODE.
  Zeigen Sie: Die Ruhelage $y^* = 0$ ist asymptotisch stabil.
\end{enumerate}
\end{exercise}
\begin{solution}
Beweis.
\end{solution}
