\begin{exercise}
Betrachten Sie die folgende Verallgemeinerung eines mathematischen Pendels (ohne Reibung):
\begin{align*}
  y^{\primeprime} + g(y) = 0,
\end{align*}
wobei die Funktion $g$ auf $(-a,a)$ definiert ist und $g(0) = 0, g(x) > 0$ für $x > 0$
und $g(x) < 0$ für $x < 0$ erfüllt ($xg(x) > 0$ für $x \neq 0$). Überführen Sie
diese ODE 2.Ordnung in ein System erster Ordnung. Geben Sie eine Ljapunovfunktion
an. Zeigen Sie, dass $(y,y^{\prime}) = (0,0)$ eine stabile Ruhelage ist.
\end{exercise}
\begin{solution}
Um die ODE in ein System erster Ordnung zu Überführen definieren wir $y_1:=y,
y_2:= y^\prime = y_1^\prime$. Dann sieht unser System, mit $y= (y_1,y_2)^T$ wie folgt aus:

\begin{align*}
  y^\prime = \left(
  \begin{array}{c}
    y_1^\prime \\
    y_2^\prime
  \end{array}
  \right) = \left(
  \begin{array}{c}
    y_2 \\
    -g(y_1)
  \end{array}
  \right) =: f(y)
\end{align*}

Um nun eine Ljapunovfunktion zu finden wenden wir uns an Satz 5.13 der besagt, das diese
durch

\begin{align*}
  \nabla V(y) \cdot f(y) \leq 0 \quad \forall y \in G
\end{align*}

gekennzeichnet sind. Wenn wir nun

\begin{align*}
  V(y_1,y_2) = \frac{y_2^2}{2} + \int_0^{y_1}g(\tau)d\tau
\end{align*}

wählen, sehen wir dass diese Bedingung wirklich erfüllt ist, da

\begin{align*}
  \nabla V(y) \cdot f(y) = \left(
  \begin{array}{c}
    g(y_1)\\
    y_2
  \end{array}
  \right)
  \cdot \left(
  \begin{array}{c}
   y_2 \\
  -g(y_1)
  \end{array}
  \right) = g(y_1) y_2 - g(y_1) y_2 = 0
\end{align*}

Dabei existiert $\int_0^{y_1} g(\tau)d\tau$ da $g$ nach unser allgemeinen Annahme
lipschitzstetig ist (weil unsere rechte Seite $f$ lipschitzstetig sein soll).

Klarerweise ist $(y_1,y_2) = (0,0)$ eine Ruhelage. Mit der direkten Methode von Ljapunov
können wir nun auch einsehen, dass diese stabil ist. Denn es gilt $V(0,0) = 0$ und für
beliebige $(y_1,y_2) \in (-a,a) \times \R \backslash \{(0,0)\}$ gilt:

\begin{align*}
  V(y_1,y_2) > 0
\end{align*}

Da $\frac{y_2^2}{2} > 0$ und da $\sgn(g(y_1))=\sgn(y_1)$.
\end{solution}
