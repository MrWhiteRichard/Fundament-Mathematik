\begin{exercise}
Betrachten Sie das System
\begin{align*}
  x^{\prime} &= -y + x(1 - x^2 - y^2) \\
  y^{\prime} &= x + y(1 - x^2 - y^2)
\end{align*}
Zeigen Sie: Die Mengen $M_1 = \{(0,0)\}, M_2 = \{(x,y): x^2 + y^2 = 1\}$
und $M_3 = \{(x,y): x^2 + y^2 < 1\}$ sind invariante Mengen.
\end{exercise}
\begin{solution}
Betrachte
\begin{align*}
  f(0,0) = (0,0)
\end{align*}
Damit ist $(0,0)$ eine Ruhelage und $M_1$ klarerweise eine invariante Menge. \\
Umrechnen in Polarkoordinaten:
\begin{align}\label{eq1}
  x = r\cos(\phi) \implies x^{\prime} = r^\prime\cos(\phi)  - r\sin(\phi)\phi^{\prime}
\end{align}
\begin{align*}
  y = r\sin(\phi) \implies y^{\prime} = r^{\prime}\sin(\phi) + r\cos(\phi)\phi^{\prime}
\end{align*}
\begin{align*}
  r^2 &= x^2 + y^2 \implies
  2rr^{\prime} = 2xx^{\prime} + 2yy^{\prime} = 2x(-y + x(1 - x^2 - y^2)) + 2y(x + y(1 - x^2 - y^2)) \\
  &= 2x^2(1 - x^2 - y^2) + 2y^2(1 - x^2 - y^2)
  = 2r^2(1 - r^2)
\end{align*}
Also erhalten wir
\begin{align*}
  r^{\prime} = r(1 - r^2).
\end{align*}
Weiters gilt aufgrund \eqref{eq1}
\begin{align*}
  r \sin(\phi)\phi^{\prime} &= -x^{\prime} + r^{\prime}\cos(\phi) = y + (r^2 - 1)x - r(r^2 - 1)\cos(\phi) \\
  &= r\sin(\phi) + (r^2 - 1)(x-\cos(\phi)) =
  r\sin(\phi) \implies \phi^{\prime} = 1.
\end{align*}
In dieser Darstellung lassen sich die letzten beiden Aussagen direkt aus der Differentialgleichung ablesen:
Für $x^2 + y^2 = r^2 = 1$ gilt
\begin{align*}
  r^{\prime} = 1(1-1)= 0
\end{align*}
und damit für alle folgt $\forall t > 0: x_{0,x_0}(t)^2 + y_{0,y_0}(t)^2 = 1$.
Da sich aufgrund der Eindeutigkeit von Anfangswertproblemen Lösungen nicht schneiden
können, muss auch $M_3$ eine invariante Menge sein.
\end{solution}
