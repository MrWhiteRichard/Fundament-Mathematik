\begin{exercise}

Die Gleichung

\begin{align*}
  \frac{x^2}{a^2} + \frac{y^2}{b^2} = 1
\end{align*}

beschreibt eine Ellipse in der Ebene. Zeichnen Sie die Ellipse. Bestimmen Sie einen Tangentialvektor, einen Normalvektor und die Gleichung der Tangente an die Ellipse in einem beliebigen Punkt $(x_0, y_0)$ der Ellipse. Machen Sie dies auf zwei Arten:

\begin{itemize}

  \item[\textbf{a)}] durch Auflösen der Gleichung nach $x$ oder $y$.

  \item[\textbf{b)}] unter Verwendung der Parametrisierung
  \begin{align*}
    x(t) = a \cos{t},
    \quad
    y(t) = b \sin{t},
    \quad
    t \in [0, 2 \pi].
  \end{align*}

\end{itemize}

\end{exercise}

\begin{solution}

Wir lösen die Gleichung nach $y$ auf und erhalten
\begin{align*}
  y_{\pm} = \pm b \sqrt{1 - \pbraces{ \frac{x}{a} }^2} 
\end{align*}
und fasst man das als Funktion von $x$ auf so ergibt sich für die Ableitung
\begin{align*}
  y^\prime_{\pm}(x) = \mp \frac{2bx}{a^2 \sqrt{1 - \pbraces{ \frac{x}{a} }^2} }.
\end{align*}
Ein Tangentialvektor im Punkt $(x,y)^T$ der Ellipse ist dann gegeben durch
\begin{align*}
  t(x,y) =
  \begin{cases}
    \pbraces{1, y_+^\prime(x)}^T & ,y > 0 \\
    \pbraces{0, 1}^T & ,y = 0 \\
    \pbraces{1, y_-^\prime(x)}^T & ,y < 0
  \end{cases}
\end{align*}
und ein Normalvektor durch
\begin{align*}
  n(x,y) =
  \begin{cases}
    \pbraces{ -y_+^\prime(x), 1}^T & ,y > 0 \\
    \pbraces{1, 0}^T & ,y = 0 \\
    \pbraces{ -y_-^\prime(x), 1}^T & ,y < 0
  \end{cases} .
\end{align*}

Für einen beliebigen Punkt $(x_0, y_0)^T$ auf der Ellipse ist die Tangentialebene gegeben durch die Gleichung 
\begin{align*}
  \begin{cases}
    y = y_0 + (x - x_0)y_+^\prime(x_0) & ,y_0 > 0 \\
    y = \pm a & ,y_0 = 0 \land x_0 = \pm a \\
    y = y_0 + (x - x_0) y_-^\prime(x_0) & ,y_0 < 0
  \end{cases}.
\end{align*}

Nun parametrisieren wir die Ellipse gemäß Angabe und erhalten für den Tangentialvektor $s$ in Abhängigkeit von $t$
\begin{align*}
  s(t) = 
  \begin{pmatrix}
    x^\prime(t) \\ y^\prime(t)
  \end{pmatrix} =
  \begin{pmatrix}
    -a \sin(t) \\ b \cos(t)
  \end{pmatrix} .
\end{align*}
Für den Normalvektor ergibt sich dann 
\begin{align*}
  n(t) = 
  \begin{pmatrix}
    b \cos(t) \\ a \sin(t)
  \end{pmatrix} .
\end{align*}
Schließlich sind
\begin{align*}
  \begin{cases}
    b \sin(t) - \pbraces{ x - a \cos(t) } \frac{b \cos(t)}{a \sin(t)} = y & ,\sin(t) \neq 0 \\
    x = a & , \sin(t) = 0 \land \cos(t) = 1 \\
    x = -a & , \sin(t) = 0 \land \cos(t) = -1
  \end{cases}
\end{align*}
Gleichungen für die Tangentialebene.
\end{solution}
