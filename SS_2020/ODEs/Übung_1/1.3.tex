\begin{exercise}

Betrachten Sie die \textit{autonome}, explizite Differentialgleichung $y^{(k)}(t) = f(y(t), y^\prime(t), \ldots, y^{(k-1)}(t))$. Sei $t_0 \in \R$. Zeigen Sie: Falls $y$ eine Lösung der ODE ist, dann ist auch $\phi: t \mapsto y(t - t_0)$ eine Lösung.

\end{exercise}

\begin{solution}
Damit die Aufgabe lösbar ist, müssen wir zusätzlich annehmen, dass das nicht angegebene
Intervall $J = \mathbb{R}$. Zusätzlich setze ich $F: G \mapsto \mathbb{R}^n$ mit $G := \mathbb{R}^{kn}$.\\
Sei $y$ eine Lösung der obigen ODE. Dann gilt:
\begin{itemize}
\item $y \in C^k(J;\mathbb{R}^n)$
\item $\text{graph~} y := \{(y(t), y^\prime(t), \ldots, y^{(k-1)}(t)): t \in J\} \subset G$
\item $\forall t \in J: y^{(k)}(t) = f(y(t), y^\prime(t), \ldots, y^{(k-1)}(t))$
\end{itemize}
Da $y \in C^k(\mathbb{R};\mathbb{R}^n)$ ist auch $\phi = y \circ \tau_{-t_0} \in C^k(\mathbb{R};\mathbb{R}^n)$. \\
Nachdem wir annehmen, dass $G = \mathbb{R}^{kn}$ ist auch die zweite Bedingung für $\phi$
trivialerweise erfüllt. \\
Schließlich folgt mit $\phi^{(k)}(t) = y^{(k)}(t-t_0) = f(y(t- t_0), y^\prime(t - t_0), \ldots, y^{(k-1)}(t-t_0))
= f(\phi(t), \phi^\prime(t), \ldots, \phi^{(k-1)}(t))$
auch die dritte Bedingung.
\end{solution}
