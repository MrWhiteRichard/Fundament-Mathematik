\begin{exercise}

Das \Quote{mathematische Pendel} wird beschrieben durch die ODE
\begin{align*}
  \phi^\primeprime(t) + \omega^2 \sin{\phi(t)} = 0,
  \quad
  \omega = \frac{g}{\ell},
\end{align*}

wobei $g$ die Erdbeschleunigung und $\ell$ die Fadenlänge des Pendels ist. Die Funktion $\phi: t \mapsto \phi(t)$ beschreibt die Auslenkung des Pendels.

\begin{itemize}

  \item[\textbf{a)}] Schreiben Sie die ODE in ein System 1. Ordnung um. Hat Ihre neue Variable eine physikalische
  Bedeutung?

  \item[\textbf{b)}] Geben Sie eine Erhaltungsgröße an. Begrunden Sie, warum es sich um solche handelt.

\end{itemize}

\end{exercise}

\begin{solution}

Trivial!

\end{solution}
