\begin{exercise}[34/1]

Sei $E$ ein Spektralmaß.
Zeige, dass der Operator $\Int{\phi}{E}$ genau dann kompakt ist, wenn

\begin{align*}
  \Forall r > 0:
  \dim{\ran{E}}
  \pbraces
  {
    \Bbraces
    {
      w \in \C:
      |\phi(w)| \geq r
    }
  }
  < \infty.
\end{align*}

\end{exercise}

\begin{solution}

%\phantom{}

%\includegraphicsboxed{Definition 7.1.1}

Wir wissen, dass wir Orthogonalprojektionen $T$ auch durch die Operatorgleichung
\begin{align*}
  T^2 = T = T^*
\end{align*}
charakterisieren können. Insbesondere sind daher Orthogonalprojektionen selbstadjungiert. \\
Der Operator $A := \Int{\phi}{E}$ wird durch folgende Gleichung charakterisiert:
\begin{align}
  \left(Ag,h\right) = \int_{\C}\phi dE_{g,h}, \qquad E_{g,h}(\Delta) := (E(\Delta)g,h) = (g,E(\Delta)h).
\end{align}
$\phi: \C \to \C$ ist dabei eine beschränkte, messbare Funktion. \\
Approximiere $A$ mit $A_n := \Int{\phi}{E_n}$, wobei
\begin{align*}
  B_n &:= \left\{\omega \in \C: |\phi(\omega)| \geq \frac{1}{n}\right\}, \\
  E_n(\Delta) &:= E(\Delta \cap B_n).
\end{align*}
Damit wird für $n \in \N$ beliebig klarerweise wieder ein Spektralmaß definiert. \\
Zwei Behauptungen: \\
Erstens: $A_n$ hat endlich-dimensionales Bild. \\
Dazu betrachte $E_{g,h}^n(\Delta) := (E_n(\Delta)g,h)$
\begin{align*}
  \dim (\ran(E_n(\C))) = \dim (\ran(E(B_n))) < \infty.
\end{align*}
Nun folgt für alle $g \in \ker(E(B_n))$
\begin{align*}
  E_{g,h}^n(\Delta) = (E_n(\Delta)g,h) = (0,h) = 0
\end{align*}
und damit für alle $h \in \mathcal{H}$
\begin{align*}
  (A_ng,h) = \int_{\C}\phi dE_{g,h}^n = 0,
\end{align*}
also gilt $\ker(E(B_n)) \subseteq \ker(A_n)$ und damit
$\ker(E(B_n))^{\bot} \supseteq \ker(A_n)^{\top} \supseteq \ran(A_n)$.
Jetzt verwenden wir, dass $E(B_n)$ ein selbstadjungierter
Operator ist und erhalten mit Proposition 6.6.2
\begin{align*}
  \ker(E(B_n))^{\bot} = (\ran(E(B_n)^*)^{\bot})^{\bot} = \overline{\ran(E(B_n))}
  = \ran(E(B_n)).
\end{align*}
Also gilt
\begin{align*}
  \dim(\ran(A_n)) \leq \dim(\ran(E(B_n))) < \infty.
\end{align*}
Zweitens: $A_n \to A$ in der Operatornorm. \\
Dazu berechne zuerst
\begin{align*}
  E_{g,h}^n(\Delta) = (E^n(\Delta)g,h) = (E(\Delta \cap B_n)g,h) = (E(\Delta)E(B_n)g,h)
  = E_{E(B_n)g,h}(\Delta).
\end{align*}
Also folgt mit
\begin{align*}
    (A_ng,h) = \int_{\C}\phi dE_{g,h}^n = \int_{\C}\phi dE_{E(B_n)g,h} = (AE(B_n)g,h)
    = (\Phi_E(\phi)\Phi_E(\1_{B_n}) dE g,h) = (\Phi_E(\1_{B_n}\phi) dE g,h)
\end{align*}
dass $A_n = \Phi_E(\1_{B_n}\phi) = \int_{\C} \1_{B_n}\phi dE$.
Wir definieren $\phi_n := \1_{B_n}\phi$ und bemerken, dass
\begin{align*}
  \|\phi - \phi_n\|_{\infty} = \|(1 - \1_{B_n})\phi\|_{\infty} = \sup\{|\phi(x)|: x \notin B_n\}
  \leq \frac{1}{n} \xrightarrow{n \to \infty} 0.
\end{align*}
Also konvergieren die $\phi_n$ gleichmäßig gegen $\phi$ und mit Satz 7.1.9.(viii)
folgt, dass
\begin{align*}
  \lim_{n \to \infty} A_n = \lim_{n \to \infty}\Phi_E(\1_{B_n}\phi) =
  \lim_{n \to \infty}\Phi_E(\phi_n) = \Phi_E(\phi) = A
\end{align*}
in der Operatornorm und aufrgund der Abgeschlossenheit der kompakten Operatoren
bezüglich der Operatornorm ist $A$ damit kompakt. \\
Sei umgekehrt $\Int{\phi}{E}$ kompakt. \\
Fall 1: $\phi = 0$: \\
Dann ist für alle $r > 0$
\begin{align*}
  \dim{\ran{E}}\pbraces{\Bbraces{w \in \C:|\phi(w)| \geq r}}
  = \dim{\ran{E}}(\emptyset) = 0.
\end{align*}
Fall 2: $\phi \neq 0:$ \\
Angenommen, es existiert ein $r > 0$, sodass
\begin{align*}
  \dim{\ran{E}}\pbraces{\underbrace{\Bbraces{w \in \C:|\phi(w)| \geq r}}_{=: B_r}} = \infty.
\end{align*}
Da $A := \int_{\C} \phi dE$ kompakt ist, ist es auch
\begin{align*}
  AA^* = \int_{\C} \phi dE\int_{\C} \overline{\phi} dE = \int_{\C} |\phi|^2 dE.
\end{align*}
Dieser Operator ist somit kompakt und selbstadjungiert und es gilt
\begin{align*}
  AA^*x = \sum_{n=1}^{\infty}\lambda_nP_nx, \quad x \in \mathcal{H}
\end{align*}
Sei zudem $(u_i)_{i \in I}$ eine ONB von $\mathcal{H}$ aus Eigenvektoren von $AA^*$
und betrachte den Operator $AA^*|_{\overline{\ran(E(B_r))}} \in L_b({\overline{\ran(E(B_r))}})$.
Diese Abbildung ist wohldefiniert, da nach dem ersten Punkt
\begin{align*}
  \ran(AA^*E(B_r)) \subseteq \ran(E(B_r))
\end{align*}
und damit auch aufgrund der Stetigkeit von $AA^*$
\begin{align*}
  \ran(AA^*\overline{E(B_r)}) \subseteq \overline{\ran(AA^*E(B_r))} \subseteq \overline{\ran(E(B_r))}.
\end{align*}
Die Einschränkung ist klarerweise wieder kompakt und selbstadjungiert und wir können
nochmals den Spektralsatz anwenden und erhalten eine ONB $(b_j)_{j \in J}$ von $\overline{E(B_r)}$
aus Eigenvektoren von $AA^*|_{\overline{\ran(E(B_r))}}$, welche sicher
auch Eigenvektoren von $AA^*$ sind.
Laut Proposition 6.5.11 gilt, dass
$F := \{\lambda: |\lambda| \geq r, \lambda \in \sigma_p(AA^*)\}$ eine endliche Menge
ist.
Proposition 6.5.8 liefert uns $\dim(\ker(AA^* -\lambda I)) < \infty$.
Für alle $b_j,j \in J$ gilt nun
\begin{align*}
  \|AA^*b_j\|^2 &= (AA^*b_j,AA^*b_j) = \int_{\C}|\phi|^2 dE_{b_j,AA^*b_j} \\
   &\geq \int_{B_r}r^2 dE_{b_j,AA^*b_j} = r^2(E(B_r)b_j,AA^*x) = r(b_j,AA^*b_j) = r(AA^*b_j,b_j) = r^2\int_{\C}|\phi|^2 dE_{b_j,b_j} \\
    &\geq r^4(E(B_r)b_j,b_j) = r^4(b_j,b_j) = r^4\|b_j\|^2.
\end{align*}
Damit gilt: Der zugehörige Eigenwert zu $b_j$ ist größer gleich $r$ und es gilt
\begin{align*}
  \dim\left(\ran\left(\bigoplus_{k \in \N: |\lambda_k| \geq r}P_k\right)\right) \geq \dim(\ran(E(B_r))) = \infty.
\end{align*}
Da das Bild einer Summe von Orthogonalprojektionen die direkte Summe der
einzelnen Bilder ist, muss es ein $k \in \N$ geben mit $\dim(\ran(P_k)) = \infty$ und $AA^*$ kann folglich nicht
kompakt sein und daher auch $A$ nicht. Widerspruch!
\end{solution}
