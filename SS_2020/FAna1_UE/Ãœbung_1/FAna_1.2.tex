\begin{exercise}

Sei $X$ ein topologischer Vektorraum.
Zeige

\begin{align*}
  \Forall A \subseteq X, \enspace \text{kreisförmig}:
  \pbraces
  {
    A^\circ \enspace \text{kreisförmig}
    \Leftrightarrow
    (A^\circ = \emptyset \lor 0 \in A^\circ)
  }
\end{align*}

Finde ein Beispiel eines topologischen Vektorraumes $X$ und einer kreisförmigen Menge $A \subseteq X$, deren
Inneres nicht kreisförmig ist.

\end{exercise}

\begin{solution}

\phantom{}

\begin{enumerate}[label = (\roman*)]

  \item
  Sei $A \subseteq X$ kreisförmig.

  \begin{enumerate}

    \item
    [\Quote{$\Rightarrow$}]
    Sei $A^\circ \neq \emptyset$ kreisförmig.
    Dann $\Exists x \in A^\circ$.
    Wegen der Kreisförmigkeit (und $|0| \leq 1$) ist auch $0 = 0x \in A^\circ$.

    \item
    [\Quote{$\Leftarrow$}]
    Wir unterscheiden zwei Fälle.

    \begin{enumerate}[label = Fall \arabic*:]

      \item
      Sei $A^\circ = \emptyset$.
      Dann ist natürlich $A^\circ$ kreisförmig.

      \item
      Sei $0 \in A^\circ$.
      Wir wählen ein beliebiges $x \in A^\circ$ und ein $\lambda \in \C$ mit $\vbraces{\lambda} \leq 1$.

      \begin{enumerate}[label = Fall 2.\arabic*:]

        \item
        Sei $\lambda = 0$. Dann ist $\lambda x = 0 \in A^\circ$.

        \item
        Sei $\lambda \neq 0$.
        Laut der Definition von $A^\circ$, $\Exists U \subseteq A, \text{offen}: x \in U$.
        Da $A$ kreisförmig ist und $M_\lambda: X \to X : x \mapsto \lambda x$ ein Homöomorphismus ist, ist $\lambda x \in \lambda U \subseteq A$ und $\lambda U = M_\lambda(U)$ offen.
        Also ist $A$ eine Umgebung von $\lambda x$ und damit $\lambda x \in A^\circ$.

      \end{enumerate}

    \end{enumerate}

    Insgesamt ist also $A^\circ$ kreisförmig.

  \end{enumerate}

  \item
  Wir betrachten den topologischen Vektorraum $(\C^2, \norm[2]{\cdot})$.

  \begin{align*}
    A :=
    \Bbraces
    {
      (v, w)^T \in \C^2:
      \vbraces{v} \leq \vbraces{w}
    }
  \end{align*}

  Wir zeigen, dass $A$ kreisförmig ist.
  Dazu wählen wir $(v, w)^T \in A$ und $\lambda \in \C$ mit $\vbraces{\lambda} \leq 1$.

  \begin{align*}
    |v| \leq |w|
    \Rightarrow
    |\lambda|
    |v| \leq |\lambda| |w|
    \Rightarrow
    |\lambda v| \leq |\lambda w|
    \Rightarrow
    \lambda (v, w)^T \in A
  \end{align*}

  Wir zeigen, dass $A^\circ$ nicht kreisförmig ist.
  Das ist äquivalent dazu, dass ...

  \begin{itemize}[label = {}]

    \item
    [\Quote{$A^\circ \neq \emptyset$}:]
    Das folgt aus $(0, 1)^T \in A^\circ$.

    \item
    [\Quote{$0 \notin A^\circ$}:]
    Sei $U \in \mathfrak{U}(0)$.
    Weil die offenen $\epsilon$-Kugeln um $0$ eine Umgebungsbasis von $\mathfrak{U}(0)$ bilden, $\Exists \epsilon > 0: U_\epsilon(0) \subseteq U$.
    Wir betrachten den Punkt $x := (\epsilon / 2, \epsilon / 4)^T$.

    \begin{align*}
      \norm[2]
      {
        \pbraces
        {
          \frac{\epsilon}{2},
          \frac{\epsilon}{4}
        }^T
      }
      =
      \sqrt
      {
        \vbraces{\frac{\epsilon}{2}}^2 +
        \vbraces{\frac{\epsilon}{4}}^2
      }
      =
      \sqrt
      {
        \frac{\epsilon^2}{4} +
        \frac{\epsilon^2}{16}
      }
      =
      \epsilon
      \sqrt
      {
        \frac{5}{16}
      }
      < \epsilon
    \end{align*}

    Also, ist $x \in U_\epsilon(0)$.
    Allerdings, ist $|\epsilon / 2| \not \leq |\epsilon / 4| $ und daher $x \notin A \not \supseteq U$.
    Deshalb, ist $A \notin \mathfrak{U}(0)$ und damit $0 \notin A^\circ$.

  \end{itemize}

\end{enumerate}

\end{solution}
