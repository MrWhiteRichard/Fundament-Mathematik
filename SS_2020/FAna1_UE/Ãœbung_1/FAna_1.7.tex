\begin{exercise}

Sei $X$ ein topologischer Vektorraum und $B \subseteq X$.
Zeige, dass die folgenden Aussagen äquivalent sind:

\begin{enumerate}[label = (\roman*)]
  \item $B$ ist beschränkt.
  \item Zu jeder Nullumgebung $U$ gibt es eine Zahl $\mu_U > 0$, sodass $B \subseteq \lambda U$ für alle $\lambda > \mu_U$.
  \item Für jede Folge $(x_n)_{n \in \N}$ von Elementen von $B$ und jede Folge $(\alpha_n)_{n \in \N}$ komplexer Zahlen mit $\lim_{n \to \infty} \alpha_n = 0$ gilt $\lim_{n \to \infty} \alpha_n x_n = 0$.
\end{enumerate}

\end{exercise}

\begin{solution}

\enquote{(i) $\Rightarrow$ (ii)}:
Seien $W, U \in \mathfrak{U}(0)$, mit $W \subseteq U$ kreisförmig.
Nachdem $B$ beschränkt ist, $\Exists \mu_U > 0: \Forall \lambda > \mu_U:$

\begin{align*}
  B
  \subseteq
  \mu_U W
  \stackrel{!}{\subseteq}
  \lambda W
  \subseteq
  \lambda U.
\end{align*}

Dabei gilt \enquote{!}, weil $|\frac{\mu_u}{\lambda}| \leq 1$ und somit $\frac{\mu_u}{\lambda} W \subseteq W$, da $W$ kreisförmig ist. \\

\enquote{(ii) $\Rightarrow$ (i)}:
Trivial! \\

\enquote{(i) $\Rightarrow$ (iii)}:
Sei $U \in \mathfrak{U}(0)$ kreisförmig, und $(x_n)_{n \in \N} \in B^\N$.
Weil $B$ beschränkt ist, $\Exists \lambda > 0: (x_n)_{n \in \N} \in B^\N \subseteq \lambda U^\N$.
Wenn nun $(\alpha_n)_{n \in \N} \in \C^\N$, mit $\alpha_n \to 0$, dann gilt für fast alle $n \in \N:$

\begin{align*}
  |\alpha_n| \leq \frac{1}{\lambda}
  \Rightarrow
  |\alpha_n \lambda| \leq 1.
\end{align*}

Weil $U$ kreisförmig ist und $\frac{x_n}{\lambda} \in U$, folgt schließlich $\alpha_n x_n \in U$. \\

\enquote{(iii) $\Rightarrow$ (i)}:
Angenommen, $\Exists U \in \mathfrak{U}(0): \Forall \lambda > 0: B \not \subseteq \lambda U$, d.h. $\Exists x_\lambda \in B: x_\lambda \notin \lambda U$.
Sei nun $(\alpha_n)_{n\in\mathbb{N}} \in \R^+: \alpha_n \to 0$ und definiere $\lambda_n := \frac{1}{\alpha_n}$.
Weil $B$ ja nicht beschränkt ist, muss $\Forall n \in \N: \Exists x_n \in B:$

\begin{align*}
  x_n \notin \lambda_n U
  \Rightarrow
  \alpha_n x_n \notin U.
\end{align*}

Somit $\Exists (x_n)_{n \in \N} \in B^\N, \Exists (\alpha_n)_{n \in \N} \in \C^\N: \alpha_n \to 0, \alpha_n x_n \not \to 0$, d.h. $\Exists U \in \mathfrak{U}(0): \Forall N \in \N: \Exists n \geq N: \alpha_n x_n \notin U$.

\end{solution}
