\begin{exercise}

Sei $X$ ein topologischer Vektorraum und $\mathfrak{W}$ eine Basis des Umgebungsfilters der Null in $X$.
Zeige

\begin{align*}
  \Forall A \subseteq X:
  \overline{A} = \bigcap_{W \in \mathfrak{W}}(A + W)
\end{align*}

\end{exercise}

\begin{solution}

Wir zeigen zuerst die Inklusion von links nach rechts: $\overline{A} \subseteq \bigcap_{W \in \mathfrak{W}}(A + W)$\\
Sei $x \in \overline{A}$ beliebig, also $\forall U$ Umgebung von $x: A \cap U \neq \emptyset$. \\
Da $\mathfrak{W}$ eine Umgebungbasis von 0 ist, ist $x + \mathfrak{W}$ eine Umgebungsbasis von x. \\
Sei $W \in \mathfrak{W}$ beliebig und wähle $W_0 \subset W$ kreisförmige Umgebung der 0. \\
$W_0$ ist somit insbesondere symmetrisch, $x + W_0$ eine Umgebung von x und es gilt:
\begin{align*}
\emptyset \neq (x + W_0) \cap A = (x - W_0) \cap A = x \cap (A + W_0) \subseteq x \cap (A + W)
\end{align*}
Also gilt: $x \in \bigcap_{W \in \mathfrak{W}}(A + W)$ \\

Umgekehrt betrachte $y \in \bigcap_{W \in \mathfrak{W}}(A + W)$, sowie $ U \in \mathfrak{U}(0)$. \\
Dann existiert $W \in \mathfrak{W}: W \subseteq U$ und $W_0$ kreisförmige Umgebung der 0: $W_0 \subseteq W$. \\
Weiters wähle $W_1 \in \mathfrak{W}: W_1 \subseteq W_0$. \\
$W_0$ ist somit insbesondere symmetrisch, $y \in (A + W_1) \subseteq (A + W_0)$ und es gilt:
\begin{align*}
  \emptyset \neq y \cap (A + W_0) = (y - W_0) \cap A = (y + W_0) \cap A \subseteq (y + W) \cap A \subseteq (y + U) \cap A
\end{align*}

Also haben wir: $y \in \overline{A}$, da $y + \mathfrak{U}(0)$ der Umgebungsfilter von y ist.

\end{solution}
