\begin{exercise}[23/1]

Sei $S$ der Shift-Operator am $\ell^2(\N)$.
\begin{align*}
  S:
  \begin{cases}
    \ell^2(\N)              & \to     \ell^2(\N) \\
    (x_1, x_2, x_3, \ldots) & \mapsto (0, x_1, x_2, \ldots)
  \end{cases}
\end{align*}
\begin{align*}
  S^*:
  \begin{cases}
    \ell^2(\N)             & \to     \ell^2(\N) \\
    (x_1, x_2, x_3, \ldots) & \mapsto (x_2, x_3, x_4, \ldots)
  \end{cases}
\end{align*}
\begin{enumerate}[label = (\alph*)]

  \item
  Zeige, dass

  \begin{align*}
    \sigma_p(S^\ast) = \sigma_r(S) = \Bbraces{\lambda \in \C: |\lambda| < 1}, \\
    \sigma_c(S^\ast) = \sigma_c(S) = \Bbraces{\lambda \in \C: |\lambda| = 1}, \\
    \sigma_r(S^\ast) = \sigma_p(S) = \emptyset
  \end{align*}

  \item
  Bestimme $\sigma_{app}(S)$ und finde zu jedem Punkt $\lambda \in \sigma_{app}(S)$ eine Folge wie in der Definnition des approximativen Punktspektrums verlangt.

  \item
  Für $\lambda \in \C$ mit $|\lambda| \neq 1$ bestimme $\dim{(\ell^2(\N) / \ran{S - \lambda})}$.

\end{enumerate}

\end{exercise}

\begin{solution}
\begin{enumerate}[label = (\alph*)]
  \item \begin{align*}
    S^* - \lambda I:
    \begin{cases}
      \ell^2(\N)             & \to     \ell^2(\N) \\
      (x_1, x_2, x_3, \ldots) & \mapsto (x_2 - \lambda x_1, x_3 - \lambda x_2, x_4 - \lambda x_3, \ldots)
    \end{cases}
  \end{align*}
  \begin{align*}
    S - \lambda I:
    \begin{cases}
      \ell^2(\N)             & \to     \ell^2(\N) \\
      (x_1, x_2, x_3, \ldots) & \mapsto (- \lambda x_1, x_1 - \lambda x_2, x_2 - \lambda x_3, \ldots)
    \end{cases}
  \end{align*}
  Aus der letzten Übung ist bekannt, dass $\|S\| = \|S^*\| = 1$. Mit Lemma 6.4.10
  folgt daraus bereits

  \begin{align*}
    \sigma(S), \sigma(S^*) \subseteq K_{1}(0) = \Bbraces{\lambda \in \C: |\lambda| \leq 1}
  \end{align*}
  Wir zeigen: $\sigma_p(S^\ast) = \Bbraces{\lambda \in \C: |\lambda| < 1}$: \\
  Sei $x \in \ker(S^* - \lambda I)$. Dann gilt
  \begin{align*}
    x_2 = \lambda x_1, \quad x_3 = \lambda x_2 = \lambda^2 x_1, \dots, x_n = \lambda x_{n-1} = \lambda^{n-1} x_1, \dots
  \end{align*}
  Sei $x_1 \neq 0$. Dann folgt $x \in \ell^2(\N) \iff |\lambda| < 1$.
  Also haben wir für $\lambda < 1$
  \begin{align*}
    \{0\} \neq \ker(S^* - \lambda I) = \ran(S - \overline{\lambda})^{\bot}.
  \end{align*}
  Wir zeigen: $\sigma_r(S) = \Bbraces{\lambda \in \C: |\lambda| < 1}$: \\
  Sei nun $a \neq 0 \in \ker(S^* - \lambda I)$.
  Wir wissen, dass $\ran(S - \overline{\lambda})^{\bot} = \overline{\ran(S - \overline{\lambda})}^{\bot}$
  und damit auch $a \notin \overline{\ran(S - \overline{\lambda})}$. Also ist
  \begin{align*}
     \overline{\ran(S - \overline{\lambda})} = (\ran(S - \overline{\lambda})^{\bot})^{\bot}
     = \ker(S^* - \lambda I)^{\bot} \neq \ell^2(\N) \iff |\lambda| < 1.
  \end{align*}
  Um zu zeigen, dass $\lambda$ im Residualspektrum liegt, müssen wir jetzt nur noch
  überprüfen, dass $\lambda$ kein Eigenwert von $S$ ist.
  Sei $x \in \ker(S - \lambda I)$. Dann gilt
  \begin{align*}
    -\lambda x_1 = 0, \quad x_1 - \lambda x_2 = -\lambda x_2 = 0, \dots, -\lambda x_n = 0, \dots
  \end{align*}
  und damit $x = 0$. Also ist $\lambda$ kein Eigenwert von $S$ und es gilt
  \begin{align*}
    \sigma_p(S^\ast) = \sigma_r(S) = \Bbraces{\lambda \in \C: |\lambda| < 1}.
  \end{align*}
  Daraus folgt
  \begin{align*}
    \Bbraces{\lambda \in \C: |\lambda| < 1} &= \sigma_p(S^\ast) \subseteq \sigma(S^*) \subseteq \Bbraces{\lambda \in \C: |\lambda| \leq 1} \\
    \Bbraces{\lambda \in \C: |\lambda| < 1} &= \sigma_r(S) \subseteq \sigma(S) \subseteq \Bbraces{\lambda \in \C: |\lambda| \leq 1}.
  \end{align*}
  Da das Spektrum abgeschlossen ist, folgt damit schon
  \begin{align*}
    \sigma(S) = \sigma(S^*) = \Bbraces{\lambda \in \C: |\lambda| \leq 1}.
  \end{align*}
  Sei nun $|\lambda| = 1$: \\
  Wir haben bereits gesehen, dass für alle $\lambda$
  \begin{align*}
    \ker(S - \lambda I) = \{0\}.
  \end{align*}
  und damit also
  \begin{align*}
    \sigma_p(S) = \emptyset.
  \end{align*}
  Gäbe es ein $\lambda \in \sigma_r(S^*)$, dann hätten wir mit
  \begin{align*}
    \ell^2(\N) \neq \overline{\ran(S^* - \lambda)} = (\ran(S^* - \lambda)^{\bot})^{\bot} =
    (\ker(S - \overline{\lambda}))^{\bot} = \{0\}^{\bot} = \ell^2(\N)
  \end{align*}
  einen Widerspruch. Zusammengefasst gilt also:
  \begin{align*}
    \sigma_p(S^\ast) &= \sigma_r(S) = \Bbraces{\lambda \in \C: |\lambda| < 1} \\
    \sigma_p(S) &= \sigma_r(S^*) = \emptyset \\
    \sigma(S) &= \sigma(S^*) = \Bbraces{\lambda \in \C: |\lambda| \leq 1}
  \end{align*}
  und aufgrund $\sigma(S) = \sigma_p(S) ~\dot \cup~ \sigma_c(S) ~\dot \cup~ \sigma_r(S)$ gilt auch
  \begin{align*}
    \sigma_c(S^\ast) &= \sigma_c(S) = \Bbraces{\lambda \in \C: |\lambda| = 1}.
  \end{align*}
  \item Nach Definition ist
  \begin{align*}
    \sigma_{app}(S) := \{\lambda \in \C: \exists (\vv{x_n})_{n \in \N} \subseteq \ell^2(\N): \|x_n\| = 1, (S - \lambda)(x_n) \to 0\}.
  \end{align*}
  Wir kennen bereits die Beziehung
  \begin{align*}
    \{\lambda \in \C: |\lambda| = 1 \} = \sigma_p(S) \cup \sigma_c(S) \subseteq \sigma_{app}(S) \subseteq \sigma(S).
  \end{align*}
  Sei $\lambda \in \sigma(S), (x_n)_{n \in N} \subset \ell^2(\N): \forall n \in \N: \|x_n\| = 1$ beliebig. Dann gilt
  \begin{align*}
    \|(S - \lambda)(x_n)\|^2 &=  (\lambda x_{n_1})^2 + \sum_{k \in \N}(x_{n_k}- \lambda x_{n_{k+1}})^2
    \geq |\lambda|^2x_{n_1}^2 + \sum_{k \in \N}(|x_{n_k}|- |\lambda| |x_{n_{k+1}}|)^2 \\
    &= (1 + |\lambda|)\underbrace{\sum_{k \in \N}|x_{n_k}|^2}_{= 1} - 2|\lambda|\sum_{k \in \N}|x_{n_k}| |x_{n_{k+1}}|
    \geq 1 + \lambda - 2|\lambda|\underbrace{\|x_n\|_2}_{=1}\underbrace{\|S^*x_n\|_2}_{\leq 1} \geq 1 - |\lambda|,
  \end{align*}
  wobei die vorletzte Ungleichung aufgrund Hölder gilt. \\
  Also kann $(S - \lambda)(x_n)$ für $|\lambda| < 1$ nicht gegen $0$ konvergieren und
  \begin{align*}
    \sigma_{app}(S) = \{\lambda \in \C: |\lambda| = 1 \}.
  \end{align*}
  Nun zur konkreten Angabe der Folge für $|\lambda| = 1$. Definiere
  \begin{align*}
    x_{n_k} = \begin{cases}
      \frac{\lambda^{1-k}}{\sqrt{n}}, & k \leq n \\
      0, & \text{sonst}
    \end{cases}.
  \end{align*}
  Es gilt
  \begin{align*}
    \|x_n\| = \sum_{k=1}^n\frac{|\lambda|^{2(1-k)}}{n} = \sum_{k=1}^n\frac{1}{n} = 1
  \end{align*}
  und
  \begin{align*}
    \|(S - \lambda)(x_n)\|^2 = \frac{|\lambda|^2}{n} + \sum_{k = 1}^{n-1} \left(\frac{\lambda^{1-k}}{\sqrt{n}}-
    \frac{\lambda^{1-k}}{\sqrt{n}}\right)^2 +\frac{|\lambda|^{1-n}}{n} = \frac{2}{n} \to 0.
  \end{align*}
  \item Fall 1: $|\lambda| > 1$: \\
  Es gilt $\lambda \in \rho(S)$, also $\ran(S - \lambda) = \ell^2(\N)$ und damit
  $\dim{(\ell^2(\N) / \ran{S - \lambda})} = \dim{(\ell^2(\N) / \ell^2(\N))} = 0$. \\
  Fall 2: $|\lambda| < 1$: \\
  Wir wissen aus der Vorlesung, dass
  \begin{align*}
    \lambda \in \sigma_{app}(S) \iff \lambda \in \sigma_p(S) \lor (\ker(S - \lambda) = \{0\} \land (S - \lambda)^{-1} \text{ unbeschränkt}).
  \end{align*}
  Da $\lambda$ weder im Punktspektrum, noch im approximativen Spektrum von $S$ liegt,
  folgt daher, dass $(S - \lambda)^{-1}$ beschränkt sein muss. Weiters gilt, dass $(S - \lambda)$
  injektiv. Laut Bemerkung 4.3.5 sind diese beiden Bedingungen äquivalent zur Existenz eines $a > 0$, sodass
  \begin{align*}
    \forall x \in \ell^2(\N): a\|x\| \leq \|(S - \lambda)x\|.
  \end{align*}
  Jetzt können wir Lemma 4.3.6 anwenden und erhalten, dass $\ran(S - \lambda)$ abgeschlossen ist. Also gilt
  \begin{align*}
    \ran(S - \lambda) = \overline{\ran(S - \lambda)} = (\ran{S - \lambda}^{\bot})^{\bot}
    = (\ker{S^* - \overline{\lambda}})^{\bot} = \{(\overline{\lambda}^{n-1} z)_{n \in \N}: z \in \C\}^{\bot}.
  \end{align*}
  Schließlich folgt mit Lemma 5.4.9, dass
  \begin{align*}
    \dim{(\ell^2(\N) / \ran{S - \lambda})} &= \dim{(\ran{S - \lambda})^{\bot}}
    = \dim(\{(\overline{\lambda}^{n-1} z)_{n \in \N}: z \in \C\}^{\bot})^{\bot} \\
    &= \dim\overline{\{(\overline{\lambda}^{n-1} z)_{n \in \N}: z \in \C\}}
    = \dim\{(\overline{\lambda}^{n-1} z)_{n \in \N}: z \in \C\} = 1.
  \end{align*}
\end{enumerate}
\end{solution}
