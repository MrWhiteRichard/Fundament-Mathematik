\documentclass{article}
\newcommand{\dmu}{~\mathrm{d} \mu}
% ---------------------------------------------------------------- %
% short package descriptions are copied from
% https://ctan.org/

% ---------------------------------------------------------------- %

% Accept different input encodings
\usepackage[utf8]{inputenc}

% Standard package for selecting font encodings
\usepackage[T1]{fontenc}

% ---------------------------------------------------------------- %

% Multilingual support for Plain TEX or LATEX
\usepackage[ngerman]{babel}

% ---------------------------------------------------------------- %

% Set all page margins to 1.5cm
\usepackage{fullpage}

% Margin adjustment and detection of odd/even pages
\usepackage{changepage}

% Flexible and complete interface to document dimensions
\usepackage{geometry}

% ---------------------------------------------------------------- %
% mathematics

\usepackage{amsmath}  % AMS mathematical facilities for LATEX
\usepackage{amssymb}
\usepackage{amsfonts} % TEX fonts from the American Mathematical Society
\usepackage{amsthm}   % Typesetting theorems (AMS style)

% Mathematical tools to use with amsmath
\usepackage{mathtools}

% Support for using RSFS fonts in maths
\usepackage{mathrsfs}

% Commands to produce dots in math that respect font size
\usepackage{mathdots}

% "Blackboard-style" cm fonts
\usepackage{bbm}

% Typeset in-line fractions in a "nice" way
\usepackage{nicefrac}

% Typeset quotient structures with LATEX
\usepackage{faktor}

% Vector arrows
\usepackage{esvect}

% St Mary Road symbols for theoretical computer science
\usepackage{stmaryrd}

% Three series of mathematical symbols
\usepackage{mathabx}

% ---------------------------------------------------------------- %
% algorithms

% Package for typesetting pseudocode
\usepackage{algpseudocode}

% Typeset source code listings using LATEX
\usepackage{listings}

% Reimplementation of and extensions to LATEX verbatim
\usepackage{verbatim}

% If necessary, please use the following 2 packages locally, but never both.
% This is because the algorithm environment gets defined in both packages, which leads to name conflicts.
% \usepackage{algorithm2e}
% \usepackage{algorithm}

% ---------------------------------------------------------------- %
% utilities

% A generic document command parser
\usepackage{xparse}

% Extended conditional commands
\usepackage{xifthen}

% e-TEX tools for LATEX
\usepackage{etoolbox}

% Define commands with suffixes
\usepackage{suffix}

% Extensive support for hypertext in LATEX
\usepackage{hyperref}

% Driver-independent color extensions for LATEX and pdfLATEX
\usepackage{xcolor}

% ---------------------------------------------------------------- %
% graphics

% -------------------------------- %

\usepackage{tikz}

% MISC
\usetikzlibrary{patterns}
\usetikzlibrary{decorations.markings}
\usetikzlibrary{positioning}
\usetikzlibrary{arrows}
\usetikzlibrary{arrows.meta}
\usetikzlibrary{overlay-beamer-styles}

% finite state machines
\usetikzlibrary{automata}

% turing machines
\usetikzlibrary{calc}
\usetikzlibrary{chains}
\usetikzlibrary{decorations.pathmorphing}

% -------------------------------- %

% Draw tree structures
\usepackage[noeepic]{qtree}

% Enhanced support for graphics
\usepackage{graphicx}

% Figures broken into subfigures
\usepackage{subfig}

% Improved interface for floating objects
\usepackage{float}

% Control float placement
\usepackage{placeins}

% Include PDF documents in LATEX
\usepackage{pdfpages}

% ---------------------------------------------------------------- %

% Control layout of itemize, enumerate, description
\usepackage[inline]{enumitem}

% Intermix single and multiple columns
\usepackage{multicol}
\setlength{\columnsep}{1cm}

% Coloured boxes, for LATEX examples and theorems, etc
\usepackage{tcolorbox}

% ---------------------------------------------------------------- %
% tables

% Tabulars with adjustable-width columns
\usepackage{tabularx}

% Tabular column heads and multilined cells
\usepackage{makecell}

% Publication quality tables in LATEX
\usepackage{booktabs}

% ---------------------------------------------------------------- %
% bibliography and quoting

% Sophisticated Bibliographies in LATEX
\usepackage[backend = biber, style = alphabetic]{biblatex}

% Context sensitive quotation facilities
\usepackage{csquotes}

% ---------------------------------------------------------------- %

% ---------------------------------------------------------------- %
% special letters

\newcommand{\N}{\mathbb N}
\newcommand{\Z}{\mathbb Z}
\newcommand{\Q}{\mathbb Q}
\newcommand{\R}{\mathbb R}
\newcommand{\C}{\mathbb C}
\newcommand{\K}{\mathbb K}
\newcommand{\T}{\mathbb T}
\newcommand{\E}{\mathbb E}
\newcommand{\V}{\mathbb V}
\renewcommand{\S}{\mathbb S}
\renewcommand{\P}{\mathbb P}
\newcommand{\1}{\mathbbm 1}
\newcommand{\G}{\mathbb G}

\newcommand{\iu}{\mathrm i}

% ---------------------------------------------------------------- %
% quantors

\newcommand{\Forall}        {\forall ~}
\newcommand{\Exists}        {\exists ~}
\newcommand{\nExists}       {\nexists ~}
\newcommand{\ExistsOnlyOne} {\exists! ~}
\newcommand{\nExistsOnlyOne}{\nexists! ~}
\newcommand{\ForAlmostAll}  {\forall^\infty ~}

% ---------------------------------------------------------------- %
% graphics boxed

\newcommand
{\includegraphicsboxed}
[2][0.75]
{
    \begin{center}
        \begin{tcolorbox}[standard jigsaw, opacityback = 0]

            \centering
            \includegraphics[width = #1 \textwidth]{#2}

        \end{tcolorbox}
    \end{center}
}

\newcommand
{\includegraphicsunboxed}
[2][0.75]
{
    \begin{center}
        \includegraphics[width = #1 \textwidth]{#2}
    \end{center}
}

\NewDocumentCommand
{\includegraphicsgraphicsboxed}
{ O{0.75} O{0.25} m m}
{
    \begin{center}
        \begin{tcolorbox}[standard jigsaw, opacityback = 0]

            \centering
            \includegraphics[width = #1 \textwidth]{#3} \\
            \vspace{#2 cm}
            \includegraphics[width = #1 \textwidth]{#4}

        \end{tcolorbox}
    \end{center}
}

\NewDocumentCommand
{\includegraphicsgraphicsunboxed}
{ O{0.75} O{0.25} m m}
{
    \begin{center}

        \centering
        \includegraphics[width = #1 \textwidth]{#3} \\
        \vspace{#2 cm}
        \includegraphics[width = #1 \textwidth]{#4}

    \end{center}
}

% ---------------------------------------------------------------- %
% braces

\newcommand{\pbraces}[1]{{\left  ( #1 \right  )}}
\newcommand{\bbraces}[1]{{\left  [ #1 \right  ]}}
\newcommand{\Bbraces}[1]{{\left \{ #1 \right \}}}
\newcommand{\vbraces}[1]{{\left  | #1 \right  |}}
\newcommand{\Vbraces}[1]{{\left \| #1 \right \|}}

\newcommand{\abraces}[1]{{\left \langle #1 \right \rangle}}

\newcommand{\floorbraces}[1]{{\left \lfloor #1 \right \rfloor}}
\newcommand{\ceilbraces} [1]{{\left \lceil  #1 \right \rceil }}

\newcommand{\dbbraces}    [1]{{\llbracket     #1 \rrbracket}}
\newcommand{\dpbraces}    [1]{{\llparenthesis #1 \rrparenthesis}}
\newcommand{\dfloorbraces}[1]{{\llfloor       #1 \rrfloor}}
\newcommand{\dceilbraces} [1]{{\llceil        #1 \rrceil}}

\newcommand{\dabraces}[1]{{\left \langle \left \langle #1 \right \rangle \right \rangle}}

\newcommand{\abs}  [1]{\vbraces{#1}}
\newcommand{\round}[1]{\bbraces{#1}}
\newcommand{\floor}[1]{\floorbraces{#1}}
\newcommand{\ceil} [1]{\ceilbraces{#1}}

% ---------------------------------------------------------------- %

% MISC

% metric spaces
\newcommand{\norm}[2][]{\Vbraces{#2}_{#1}}
\DeclareMathOperator{\metric}{d}
\DeclareMathOperator{\dist}  {dist}
\DeclareMathOperator{\diam}  {diam}

% O-notation
\newcommand{\landau}{{\scriptstyle \mathcal{O}}}
\newcommand{\Landau}{\mathcal{O}}

% ---------------------------------------------------------------- %

% math operators

% hyperbolic trigonometric function inverses
\DeclareMathOperator{\areasinh}{areasinh}
\DeclareMathOperator{\areacosh}{areacosh}
\DeclareMathOperator{\areatanh}{areatanh}

% special functions
\DeclareMathOperator{\id} {id}
\DeclareMathOperator{\sgn}{sgn}
\DeclareMathOperator{\Inv}{Inv}
\DeclareMathOperator{\erf}{erf}
\DeclareMathOperator{\pv} {pv}

% exponential function as power
\WithSuffix \newcommand \exp* [1]{\mathrm{e}^{#1}}

% operations on sets
\DeclareMathOperator{\meas}{meas}
\DeclareMathOperator{\card}{card}
\DeclareMathOperator{\Span}{span}
\DeclareMathOperator{\conv}{conv}
\DeclareMathOperator{\cof}{cof}
\DeclareMathOperator{\mean}{mean}
\DeclareMathOperator{\avg}{avg}
\DeclareMathOperator*{\argmax}{argmax}
\DeclareMathOperator*{\argsmax}{argsmax}

% number theory stuff
\DeclareMathOperator{\ggT}{ggT}
\DeclareMathOperator{\kgV}{kgV}
\DeclareMathOperator{\modulo}{mod}

% polynomial stuff
\DeclareMathOperator{\ord}{ord}
\DeclareMathOperator{\grad}{grad}

% function properties
\DeclareMathOperator{\ran}{ran}
\DeclareMathOperator{\supp}{supp}
\DeclareMathOperator{\graph}{graph}
\DeclareMathOperator{\dom}{dom}
\DeclareMathOperator{\Def}{def}
\DeclareMathOperator{\rg}{rg}

% matrix stuff
\DeclareMathOperator{\GL}{GL}
\DeclareMathOperator{\SL}{SL}
\DeclareMathOperator{\U}{U}
\DeclareMathOperator{\SU}{SU}
\DeclareMathOperator{\PSU}{PSU}
% \DeclareMathOperator{\O}{O}
% \DeclareMathOperator{\PO}{PO}
% \DeclareMathOperator{\PSO}{PSO}
\DeclareMathOperator{\diag}{diag}

% algebra stuff
\DeclareMathOperator{\At}{At}
\DeclareMathOperator{\Ob}{Ob}
\DeclareMathOperator{\Hom}{Hom}
\DeclareMathOperator{\End}{End}
\DeclareMathOperator{\Aut}{Aut}
\DeclareMathOperator{\Lin}{L}

% other function classes
\DeclareMathOperator{\Lip}{Lip}
\DeclareMathOperator{\Mod}{Mod}
\DeclareMathOperator{\Dil}{Dil}

% constants
\DeclareMathOperator{\NIL}{NIL}
\DeclareMathOperator{\eps}{eps}

% ---------------------------------------------------------------- %
% doubble & tripple powers

\newcommand
{\primeprime}
{{\prime \prime}}

\newcommand
{\primeprimeprime}
{{\prime \prime \prime}}

\newcommand
{\astast}
{{\ast \ast}}

\newcommand
{\astastast}
{{\ast \ast \ast}}

% ---------------------------------------------------------------- %
% derivatives

\NewDocumentCommand
{\derivative}
{ O{} O{} m m}
{
    \frac
    {\mathrm d^{#2} {#1}}
    {\mathrm d {#3}^{#2}}
}

\NewDocumentCommand
{\pderivative}
{ O{} O{} m m}
{
    \frac
    {\partial^{#2} {#1}}
    {\partial {#3}^{#2}}
}

\DeclareMathOperator{\Div}{div}
\DeclareMathOperator{\rot}{rot}

% ---------------------------------------------------------------- %
% integrals

\NewDocumentCommand
{\Int}
{ O{} O{} m m}
{\int_{#1}^{#2} #3 ~ \mathrm d #4}

\NewDocumentCommand
{\Iint}
{ O{} O{} m m m}
{\iint_{#1}^{#2} #3 ~ \mathrm d #4 ~ \mathrm d #5}

\NewDocumentCommand
{\Iiint}
{ O{} O{} m m m m}
{\iiint_{#1}^{#2} #3 ~ \mathrm d #4 ~ \mathrm d #5 ~ \mathrm d #6}

\NewDocumentCommand
{\Iiiint}
{ O{} O{} m m m m m}
{\iiiint_{#1}^{#2} #3 ~ \mathrm d #4 ~ \mathrm d #5 ~ \mathrm d #6 ~ \mathrm d #7}

\NewDocumentCommand
{\Idotsint}
{ O{} O{} m m m}
{\idotsint_{#1}^{#2} #3 ~ \mathrm d #4 \dots ~ \mathrm d #5}

\NewDocumentCommand
{\Oint}
{ O{} O{} m m}
{\oint_{#1}^{#2} #3 ~ \mathrm d #4}

% ---------------------------------------------------------------- %

% source:
% https://tex.stackexchange.com/questions/203257/tikz-chains-with-one-side-of-the-leftmost-node-thickbold

% #1 (optional): current state, e.g. $q_0$
% #2: cursor position, e.g. 1
% #3: number of displayed cells, e.g. 5
% #4: contents of cells, e.g. {$\triangleright$, $x_1$, \dots, $x_n$, \textvisiblespace}

\newcommand{\turingtape}[4][]
{
    \begin{tikzpicture}

        \tikzset{tape/.style={minimum size=.7cm, draw}}

        \begin{scope}[start chain=0 going right, node distance=0mm]
            \foreach \x [count=\i] in #4
            {
                \ifnum\i=#3 % if last node reset outer sep to 0pt
                    \node [on chain=0, tape, outer sep=0pt] (n\i) {\x};
                    \draw (n\i.north east) -- ++(.1,0) decorate [decoration={zigzag, segment length=.12cm, amplitude=.02cm}] {-- ($(n\i.south east)+(+.1,0)$)} -- (n\i.south east) -- cycle;
                \else
                    \node [on chain=0, tape] (n\i) {\x};
                \fi

                \ifnum\i=1 % if first node draw a thick line at the left
                    \draw [line width=.1cm] (n\i.north west) -- (n\i.south west);
                \fi
            }
 
            \node [right=.25cm of n#3] {$\cdots$};
            \node [tape, above left=.25cm and 1cm of n1] (q) {#1};
            \draw [>=latex, ->] (q) -| (n#2);

        \end{scope}

    \end{tikzpicture}
}

% ---------------------------------------------------------------- %

% ---------------------------------------------------------------- %
% amsthm-environments:

\theoremstyle{definition}

% numbered theorems
\newtheorem{theorem}             {Satz}[section]
\newtheorem{lemma}      [theorem]{Lemma}
\newtheorem{corollary}  [theorem]{Korollar}
\newtheorem{proposition}[theorem]{Proposition}
\newtheorem{remark}     [theorem]{Bemerkung}
\newtheorem{definition} [theorem]{Definition}
\newtheorem{example}    [theorem]{Beispiel}
\newtheorem{heuristics} [theorem]{Heuristik}

% unnumbered theorems
\newtheorem*{theorem*}    {Satz}
\newtheorem*{lemma*}      {Lemma}
\newtheorem*{corollary*}  {Korollar}
\newtheorem*{proposition*}{Proposition}
\newtheorem*{remark*}     {Bemerkung}
\newtheorem*{definition*} {Definition}
\newtheorem*{example*}    {Beispiel}
\newtheorem*{heuristics*} {Heuristik}

% ---------------------------------------------------------------- %
% exercise- and solution-environments:

% Please define this stuff in project ("main.tex"):
% \def \lastexercisenumber {...}

\newtheorem{exercise}{Aufgabe}
\setcounter{exercise}{\lastexercisenumber}

\newenvironment{solution}
{
  \begin{proof}[Lösung]
}{
  \end{proof}
}

% ---------------------------------------------------------------- %
% MISC translations for environment-names

\renewcommand{\proofname} {Beweis}
\renewcommand{\figurename}{Abbildung}
\renewcommand{\tablename} {Tabelle}

% ---------------------------------------------------------------- %

% ---------------------------------------------------------------- %
% https://www.overleaf.com/learn/latex/Code_listing

\definecolor{codegreen} {rgb}{0, 0.6, 0}
\definecolor{codegray}    {rgb}{0.5, 0.5, 0.5}
\definecolor{codepurple}{rgb}{0.58, 0, 0.82}
\definecolor{backcolour}{rgb}{0.95, 0.95, 0.92}

\lstdefinestyle{overleaf}
{
    backgroundcolor = \color{backcolour},
    commentstyle = \color{codegreen},
    keywordstyle = \color{magenta},
    numberstyle = \tiny\color{codegray},
    stringstyle = \color{codepurple},
    basicstyle = \ttfamily \footnotesize,
    breakatwhitespace = false,
    breaklines = true,
    captionpos = b,
    keepspaces = true,
    numbers = left,
    numbersep = 5pt,
    showspaces = false,
    showstringspaces = false,
    showtabs = false,
    tabsize = 2
}

% ---------------------------------------------------------------- %
% https://en.wikibooks.org/wiki/LaTeX/Source_Code_Listings

\lstdefinestyle{customc}
{
    belowcaptionskip = 1 \baselineskip,
    breaklines = true,
    frame = L,
    xleftmargin = \parindent,
    language = C,
    showstringspaces = false,
    basicstyle = \footnotesize \ttfamily,
    keywordstyle = \bfseries \color{green!40!black},
    commentstyle = \itshape \color{purple!40!black},
    identifierstyle = \color{blue},
    stringstyle = \color{orange},
}

\lstdefinestyle{customasm}
{
    belowcaptionskip = 1 \baselineskip,
    frame = L,
    xleftmargin = \parindent,
    language = [x86masm] Assembler,
    basicstyle = \footnotesize\ttfamily,
    commentstyle = \itshape\color{purple!40!black},
}

% ---------------------------------------------------------------- %
% https://tex.stackexchange.com/questions/235731/listings-syntax-for-literate

\definecolor{maroon}        {cmyk}{0, 0.87, 0.68, 0.32}
\definecolor{halfgray}      {gray}{0.55}
\definecolor{ipython_frame} {RGB}{207, 207, 207}
\definecolor{ipython_bg}    {RGB}{247, 247, 247}
\definecolor{ipython_red}   {RGB}{186, 33, 33}
\definecolor{ipython_green} {RGB}{0, 128, 0}
\definecolor{ipython_cyan}  {RGB}{64, 128, 128}
\definecolor{ipython_purple}{RGB}{170, 34, 255}

\lstdefinestyle{stackexchangePython}
{
    breaklines = true,
    %
    extendedchars = true,
    literate =
    {á}{{\' a}} 1 {é}{{\' e}} 1 {í}{{\' i}} 1 {ó}{{\' o}} 1 {ú}{{\' u}} 1
    {Á}{{\' A}} 1 {É}{{\' E}} 1 {Í}{{\' I}} 1 {Ó}{{\' O}} 1 {Ú}{{\' U}} 1
    {à}{{\` a}} 1 {è}{{\` e}} 1 {ì}{{\` i}} 1 {ò}{{\` o}} 1 {ù}{{\` u}} 1
    {À}{{\` A}} 1 {È}{{\' E}} 1 {Ì}{{\` I}} 1 {Ò}{{\` O}} 1 {Ù}{{\` U}} 1
    {ä}{{\" a}} 1 {ë}{{\" e}} 1 {ï}{{\" i}} 1 {ö}{{\" o}} 1 {ü}{{\" u}} 1
    {Ä}{{\" A}} 1 {Ë}{{\" E}} 1 {Ï}{{\" I}} 1 {Ö}{{\" O}} 1 {Ü}{{\" U}} 1
    {â}{{\^ a}} 1 {ê}{{\^ e}} 1 {î}{{\^ i}} 1 {ô}{{\^ o}} 1 {û}{{\^ u}} 1
    {Â}{{\^ A}} 1 {Ê}{{\^ E}} 1 {Î}{{\^ I}} 1 {Ô}{{\^ O}} 1 {Û}{{\^ U}} 1
    {œ}{{\oe}}  1 {Œ}{{\OE}}  1 {æ}{{\ae}}  1 {Æ}{{\AE}}  1 {ß}{{\ss}}  1
    {ç}{{\c c}} 1 {Ç}{{\c C}} 1 {ø}{{\o}} 1 {å}{{\r a}} 1 {Å}{{\r A}} 1
    {€}{{\EUR}} 1 {£}{{\pounds}} 1
}


% Python definition (c) 1998 Michael Weber
% Additional definitions (2013) Alexis Dimitriadis
% modified by me (should not have empty lines)

\lstdefinelanguage{iPython}{
    morekeywords = {access, and, break, class, continue, def, del, elif, else, except, exec, finally, for, from, global, if, import, in, is, lambda, not, or, pass, print, raise, return, try, while}, %
    %
    % Built-ins
    morekeywords = [2]{abs, all, any, basestring, bin, bool, bytearray, callable, chr, classmethod, cmp, compile, complex, delattr, dict, dir, divmod, enumerate, eval, execfile, file, filter, float, format, frozenset, getattr, globals, hasattr, hash, help, hex, id, input, int, isinstance, issubclass, iter, len, list, locals, long, map, max, memoryview, min, next, object, oct, open, ord, pow, property, range, raw_input, reduce, reload, repr, reversed, round, set, setattr, slice, sorted, staticmethod, str, sum, super, tuple, type, unichr, unicode, vars, xrange, zip, apply, buffer, coerce, intern}, %
    %
    sensitive = true, %
    morecomment = [l] \#, %
    morestring = [b]', %
    morestring = [b]", %
    %
    morestring = [s]{'''}{'''}, % used for documentation text (mulitiline strings)
    morestring = [s]{"""}{"""}, % added by Philipp Matthias Hahn
    %
    morestring = [s]{r'}{'},     % `raw' strings
    morestring = [s]{r"}{"},     %
    morestring = [s]{r'''}{'''}, %
    morestring = [s]{r"""}{"""}, %
    morestring = [s]{u'}{'},     % unicode strings
    morestring = [s]{u"}{"},     %
    morestring = [s]{u'''}{'''}, %
    morestring = [s]{u"""}{"""}, %
    %
    % {replace}{replacement}{lenght of replace}
    % *{-}{-}{1} will not replace in comments and so on
    literate = 
    {á}{{\' a}} 1 {é}{{\' e}} 1 {í}{{\' i}} 1 {ó}{{\' o}} 1 {ú}{{\' u}} 1
    {Á}{{\' A}} 1 {É}{{\' E}} 1 {Í}{{\' I}} 1 {Ó}{{\' O}} 1 {Ú}{{\' U}} 1
    {à}{{\` a}} 1 {è}{{\` e}} 1 {ì}{{\` i}} 1 {ò}{{\` o}} 1 {ù}{{\` u}} 1
    {À}{{\` A}} 1 {È}{{\' E}} 1 {Ì}{{\` I}} 1 {Ò}{{\` O}} 1 {Ù}{{\` U}} 1
    {ä}{{\" a}} 1 {ë}{{\" e}} 1 {ï}{{\" i}} 1 {ö}{{\" o}} 1 {ü}{{\" u}} 1
    {Ä}{{\" A}} 1 {Ë}{{\" E}} 1 {Ï}{{\" I}} 1 {Ö}{{\" O}} 1 {Ü}{{\" U}} 1
    {â}{{\^ a}} 1 {ê}{{\^ e}} 1 {î}{{\^ i}} 1 {ô}{{\^ o}} 1 {û}{{\^ u}} 1
    {Â}{{\^ A}} 1 {Ê}{{\^ E}} 1 {Î}{{\^ I}} 1 {Ô}{{\^ O}} 1 {Û}{{\^ U}} 1
    {œ}{{\oe}}  1 {Œ}{{\OE}}  1 {æ}{{\ae}}  1 {Æ}{{\AE}}  1 {ß}{{\ss}}  1
    {ç}{{\c c}} 1 {Ç}{{\c C}} 1 {ø}{{\o}} 1 {å}{{\r a}} 1 {Å}{{\r A}} 1
    {€}{{\EUR}} 1 {£}{{\pounds}} 1
    %
    {^}{{{\color{ipython_purple}\^ {}}}} 1
    { = }{{{\color{ipython_purple} = }}} 1
    %
    {+}{{{\color{ipython_purple}+}}} 1
    {*}{{{\color{ipython_purple}$^\ast$}}} 1
    {/}{{{\color{ipython_purple}/}}} 1
    %
    {+=}{{{+=}}} 1
    {-=}{{{-=}}} 1
    {*=}{{{$^\ast$ = }}} 1
    {/=}{{{/=}}} 1,
    literate = 
    *{-}{{{\color{ipython_purple} -}}} 1
     {?}{{{\color{ipython_purple} ?}}} 1,
    %
    identifierstyle = \color{black}\ttfamily,
    commentstyle = \color{ipython_cyan}\ttfamily,
    stringstyle = \color{ipython_red}\ttfamily,
    keepspaces = true,
    showspaces = false,
    showstringspaces = false,
    %
    rulecolor = \color{ipython_frame},
    frame = single,
    frameround = {t}{t}{t}{t},
    framexleftmargin = 6mm,
    numbers = left,
    numberstyle = \tiny\color{halfgray},
    %
    %
    backgroundcolor = \color{ipython_bg},
    % extendedchars = true,
    basicstyle = \scriptsize,
    keywordstyle = \color{ipython_green}\ttfamily,
}

% ---------------------------------------------------------------- %
% https://tex.stackexchange.com/questions/417884/colour-r-code-to-match-knitr-theme-using-listings-minted-or-other

\geometry{verbose, tmargin = 2.5cm, bmargin = 2.5cm, lmargin = 2.5cm, rmargin = 2.5cm}

\definecolor{backgroundCol}  {rgb}{.97, .97, .97}
\definecolor{commentstyleCol}{rgb}{0.678, 0.584, 0.686}
\definecolor{keywordstyleCol}{rgb}{0.737, 0.353, 0.396}
\definecolor{stringstyleCol} {rgb}{0.192, 0.494, 0.8}
\definecolor{NumCol}         {rgb}{0.686, 0.059, 0.569}
\definecolor{basicstyleCol}  {rgb}{0.345, 0.345, 0.345}

\lstdefinestyle{stackexchangeR}
{
    language = R,                                        % the language of the code
    basicstyle = \small \ttfamily \color{basicstyleCol}, % the size of the fonts that are used for the code
    % numbers = left,                                      % where to put the line-numbers
    numberstyle = \color{green},                         % the style that is used for the line-numbers
    stepnumber = 1,                                      % the step between two line-numbers. If it is 1, each line will be numbered
    numbersep = 5pt,                                     % how far the line-numbers are from the code
    backgroundcolor = \color{backgroundCol},             % choose the background color. You must add \usepackage{color}
    showspaces = false,                                  % show spaces adding particular underscores
    showstringspaces = false,                            % underline spaces within strings
    showtabs = false,                                    % show tabs within strings adding particular underscores
    % frame = single,                                      % adds a frame around the code
    % rulecolor = \color{white},                           % if not set, the frame-color may be changed on line-breaks within not-black text (e.g. commens (green here))
    tabsize = 2,                                         % sets default tabsize to 2 spaces
    captionpos = b,                                      % sets the caption-position to bottom
    breaklines = true,                                   % sets automatic line breaking
    breakatwhitespace = false,                           % sets if automatic breaks should only happen at whitespace
    keywordstyle = \color{keywordstyleCol},              % keyword style
    commentstyle = \color{commentstyleCol},              % comment style
    stringstyle = \color{stringstyleCol},                % string literal style
    literate = %
    *{0}{{{\color{NumCol} 0}}} 1
     {1}{{{\color{NumCol} 1}}} 1
     {2}{{{\color{NumCol} 2}}} 1
     {3}{{{\color{NumCol} 3}}} 1
     {4}{{{\color{NumCol} 4}}} 1
     {5}{{{\color{NumCol} 5}}} 1
     {6}{{{\color{NumCol} 6}}} 1
     {7}{{{\color{NumCol} 7}}} 1
     {8}{{{\color{NumCol} 8}}} 1
     {9}{{{\color{NumCol} 9}}} 1
}

% ---------------------------------------------------------------- %
% Fundament Mathematik

\lstdefinestyle{fundament}{basicstyle = \ttfamily}

% ---------------------------------------------------------------- %


\parskip 0pt
\parindent 0pt

\title
{
  Funktionalanalysis 1 - Übung 6 \\
  \vspace{4pt}
  \normalsize
  \textit{6. UE am 05.06.2020}
}
\author
{
  Richard Weiss       \and
  Florian Schager     \and
  Christian Sallinger \and
  Fabian Zehetgruber  \and
  Paul Winkler        \and
  Christian Göth
}
\date{}

\begin{document}

\maketitle

\begin{exercise}[23/1]

Sei $S$ der Shift-Operator am $\ell^2(\N)$.
\begin{align*}
  S:
  \begin{cases}
    \ell^2(\N)              & \to     \ell^2(\N) \\
    (x_1, x_2, x_3, \ldots) & \mapsto (0, x_1, x_2, \ldots)
  \end{cases}
\end{align*}
\begin{align*}
  S^*:
  \begin{cases}
    \ell^2(\N)             & \to     \ell^2(\N) \\
    (x_1, x_2, x_3, \ldots) & \mapsto (x_2, x_3, x_4, \ldots)
  \end{cases}
\end{align*}
\begin{enumerate}[label = (\alph*)]

  \item
  Zeige, dass

  \begin{align*}
    \sigma_p(S^\ast) = \sigma_r(S) = \Bbraces{\lambda \in \C: |\lambda| < 1}, \\
    \sigma_c(S^\ast) = \sigma_c(S) = \Bbraces{\lambda \in \C: |\lambda| = 1}, \\
    \sigma_r(S^\ast) = \sigma_p(S) = \emptyset
  \end{align*}

  \item
  Bestimme $\sigma_{app}(S)$ und finde zu jedem Punkt $\lambda \in \sigma_{app}(S)$ eine Folge wie in der Definnition des approximativen Punktspektrums verlangt.

  \item
  Für $\lambda \in \C$ mit $|\lambda| \neq 1$ bestimme $\dim{(\ell^2(\N) / \ran{S - \lambda})}$.

\end{enumerate}

\end{exercise}

\begin{solution}
\begin{enumerate}[label = (\alph*)]
  \item \begin{align*}
    S^* - \lambda I:
    \begin{cases}
      \ell^2(\N)             & \to     \ell^2(\N) \\
      (x_1, x_2, x_3, \ldots) & \mapsto (x_2 - \lambda x_1, x_3 - \lambda x_2, x_4 - \lambda x_3, \ldots)
    \end{cases}
  \end{align*}
  \begin{align*}
    S - \lambda I:
    \begin{cases}
      \ell^2(\N)             & \to     \ell^2(\N) \\
      (x_1, x_2, x_3, \ldots) & \mapsto (- \lambda x_1, x_1 - \lambda x_2, x_2 - \lambda x_3, \ldots)
    \end{cases}
  \end{align*}
  Aus der letzten Übung ist bekannt, dass $\|S\| = \|S^*\| = 1$. Mit Lemma 6.4.10
  folgt daraus bereits

  \begin{align*}
    \sigma(S), \sigma(S^*) \subseteq K_{1}(0) = \Bbraces{\lambda \in \C: |\lambda| \leq 1}
  \end{align*}
  Wir zeigen: $\sigma_p(S^\ast) = \Bbraces{\lambda \in \C: |\lambda| < 1}$: \\
  Sei $x \in \ker(S^* - \lambda I)$. Dann gilt
  \begin{align*}
    x_2 = \lambda x_1, \quad x_3 = \lambda x_2 = \lambda^2 x_1, \dots, x_n = \lambda x_{n-1} = \lambda^{n-1} x_1, \dots
  \end{align*}
  Sei $x_1 \neq 0$. Dann folgt $x \in \ell^2(\N) \iff |\lambda| < 1$.
  Also haben wir für $\lambda < 1$
  \begin{align*}
    \{0\} \neq \ker(S^* - \lambda I) = \ran(S - \overline{\lambda})^{\bot}.
  \end{align*}
  Wir zeigen: $\sigma_r(S) = \Bbraces{\lambda \in \C: |\lambda| < 1}$: \\
  Sei nun $a \neq 0 \in \ker(S^* - \lambda I)$.
  Wir wissen, dass $\ran(S - \overline{\lambda})^{\bot} = \overline{\ran(S - \overline{\lambda})}^{\bot}$
  und damit auch $a \notin \overline{\ran(S - \overline{\lambda})}$. Also ist
  \begin{align*}
     \overline{\ran(S - \overline{\lambda})} \neq \ell^2(\N).
  \end{align*}
  Um zu zeigen, dass $\lambda$ im Residualspektrum liegt, müssen wir jetzt nur noch
  überprüfen, dass $\lambda$ kein Eigenwert von $S$ ist.
  Sei $x \in \ker(S - \lambda I)$. Dann gilt
  \begin{align*}
    -\lambda x_1 = 0, \quad x_1 - \lambda x_2 = -\lambda x_2 = 0, \dots, -\lambda x_n = 0, \dots
  \end{align*}
  und damit $x = 0$. Also ist $\lambda$ kein Eigenwert von $S$ und es gilt
  \begin{align*}
    \sigma_p(S^\ast) = \sigma_r(S) = \Bbraces{\lambda \in \C: |\lambda| < 1}.
  \end{align*}
  Daraus folgt
  \begin{align*}
    \Bbraces{\lambda \in \C: |\lambda| < 1} &= \sigma_p(S^\ast) \subseteq \sigma(S^*) \subseteq \Bbraces{\lambda \in \C: |\lambda| \leq 1} \\
    \Bbraces{\lambda \in \C: |\lambda| < 1} &= \sigma_r(S) \subseteq \sigma(S) \subseteq \Bbraces{\lambda \in \C: |\lambda| \leq 1}.
  \end{align*}
  Da das Spektrum abgeschlossen ist, folgt damit schon
  \begin{align*}
    \sigma(S) = \sigma(S^*) = \Bbraces{\lambda \in \C: |\lambda| \leq 1}.
  \end{align*}
  Sei nun $|\lambda| = 1$: \\
  Wir haben bereits gesehen, dass für alle $\lambda$
  \begin{align*}
    \ker(S - \lambda I) = \{0\}.
  \end{align*}
  und damit also
  \begin{align*}
    \sigma_p(S) = \emptyset.
  \end{align*}
  Gäbe es ein $\lambda \in \sigma_r(S^*)$, dann hätten wir mit
  \begin{align*}
    \ell^2(\N) \neq \overline{\ran(S^* - \lambda)} = (\ran(S^* - \lambda)^{\bot})^{\bot} =
    (\ker(S - \overline{\lambda}))^{\bot} = \{0\}^{\bot} = \ell^2(\N)
  \end{align*}
  einen Widerspruch. Zusammengefasst gilt also:
  \begin{align*}
    \sigma_p(S^\ast) &= \sigma_r(S) = \Bbraces{\lambda \in \C: |\lambda| < 1} \\
    \sigma_p(S) &= \sigma_r(S^*) = \emptyset \\
    \sigma(S) &= \sigma(S^*) = \Bbraces{\lambda \in \C: |\lambda| \leq 1}
  \end{align*}
  und aufgrund $\sigma(S) = \sigma_p(S) ~\dot \cup~ \sigma_c(S) ~\dot \cup~ \sigma_r(S)$ gilt auch
  \begin{align*}
    \sigma_c(S^\ast) &= \sigma_c(S) = \Bbraces{\lambda \in \C: |\lambda| = 1}.
  \end{align*}
  \item Nach Definition ist
  \begin{align*}
    \sigma_{app}(S) := \{\lambda \in \C: \exists (\vv{x_n})_{n \in \N} \subseteq \ell^2(\N): \|x_n\| = 1, (S - \lambda)(x_n) \to 0\}.
  \end{align*}
  Wir kennen bereits die Beziehung
  \begin{align*}
    \{\lambda \in \C: |\lambda| = 1 \} = \sigma_p(S) \cup \sigma_c(S) \subseteq \sigma_{app}(S) \subseteq \sigma(S).
  \end{align*}
  Sei $\lambda \in \sigma(S), (x_n)_{n \in N} \subset \ell^2(\N): \forall n \in \N: \|x_n\| = 1$ beliebig. Dann gilt
  \begin{align*}
    \|(S - \lambda)(x_n)\|^2 &=  (\lambda x_{n_1})^2 + \sum_{k \in \N}(x_{n_k}- \lambda x_{n_{k+1}})^2
    \geq |\lambda|^2x_{n_1}^2 + \sum_{k \in \N}(|x_{n_k}|- |\lambda| |x_{n_{k+1}}|)^2 \\
    &= (1 + |\lambda|)\underbrace{\sum_{k \in \N}|x_{n_k}|^2}_{= 1} - 2|\lambda|\sum_{k \in \N}|x_{n_k}| |x_{n_{k+1}}|
    \geq 1 + \lambda - 2|\lambda|\underbrace{\|x_n\|_2}_{=1}\underbrace{\|S^*x_n\|_2}_{\leq 1} \geq 1 - |\lambda|,
  \end{align*}
  wobei die vorletzte Ungleichung aufgrund Hölder gilt. \\
  Also kann $(S - \lambda)(x_n)$ für $|\lambda| < 1$ nicht gegen $0$ konvergieren und
  \begin{align*}
    \sigma_{app}(S) = \{\lambda \in \C: |\lambda| = 1 \}.
  \end{align*}
  Nun zur konkreten Angabe der Folge für $|\lambda| = 1$. Definiere
  \begin{align*}
    x_{n_k} = \begin{cases}
      \frac{\lambda^{1-k}}{\sqrt{n}}, & k \leq n \\
      0, & \text{sonst}
    \end{cases}.
  \end{align*}
  Es gilt
  \begin{align*}
    \|x_n\| = \sum_{k=1}^n\frac{|\lambda|^{2(1-k)}}{n} = \sum_{k=1}^n\frac{1}{n} = 1
  \end{align*}
  und
  \begin{align*}
    \|(S - \lambda)(x_n)\|^2 = \frac{|\lambda|^2}{n} + \sum_{k = 1}^{n-1} \left(\frac{\lambda^{1-k}}{\sqrt{n}}-
    \frac{\lambda^{1-k}}{\sqrt{n}}\right)^2 +\frac{|\lambda|^{1-n}}{n} = \frac{2}{n} \to 0.
  \end{align*}
  \item Fall 1: $|\lambda| > 1$: \\
  Es gilt $\lambda \in \rho(S)$, also $\ran(S - \lambda) = \ell^2(\N)$ und damit
  $\dim{(\ell^2(\N) / \ran{S - \lambda})} = \dim{(\ell^2(\N) / \ell^2(\N))} = 0$. \\
  Fall 2: $|\lambda| < 1$: \\
  Wir wissen aus der Vorlesung, dass
  \begin{align*}
    \lambda \in \sigma_{app}(S) \iff \lambda \in \sigma_p(S) \lor (\ker(S - \lambda) = \{0\} \land (S - \lambda)^{-1} \text{ unbeschränkt}).
  \end{align*}
  Da $\lambda$ weder im Punktspektrum, noch im approximativen Spektrum von $S$ liegt,
  folgt daher, dass $(S - \lambda)^{-1}$ beschränkt sein muss. Weiters gilt, dass $(S - \lambda)$
  injektiv. Laut Bemerkung 4.3.5 sind diese beiden Bedingungen äquivalent zur Existenz eines $a > 0$, sodass
  \begin{align*}
    \forall x \in \ell^2(\N): a\|x\| \leq \|(S - \lambda)x\|.
  \end{align*}
  Jetzt können wir Lemma 4.3.6 anwenden und erhalten, dass $\ran(S - \lambda)$ abgeschlossen ist. Also gilt
  \begin{align*}
    \ran(S - \lambda) = \overline{\ran(S - \lambda)} = (\ran{S - \lambda}^{\bot})^{\bot}
    = (\ker{S^* - \overline{\lambda}})^{\bot} = \{(\overline{\lambda}^{n-1} z)_{n \in \N}: z \in \C\}^{\bot}.
  \end{align*}
  Schließlich folgt mit Lemma 5.4.9, dass
  \begin{align*}
    \dim{(\ell^2(\N) / \ran{S - \lambda})} &= \dim{(\ran{S - \lambda})^{\bot}}
    = \dim(\{(\overline{\lambda}^{n-1} z)_{n \in \N}: z \in \C\}^{\bot})^{\bot} \\
    &= \dim\overline{\{(\overline{\lambda}^{n-1} z)_{n \in \N}: z \in \C\}}
    = \dim\{(\overline{\lambda}^{n-1} z)_{n \in \N}: z \in \C\} = 1.
  \end{align*}
\end{enumerate}
\end{solution}

\begin{exercise}[24/1$^\ast$]

Sei $S$ der Shift-Operator am $\ell^2(\N)$, und sei $M \in \mathcal{B}(\ell^2(\N))$ der Operator definiert als $(\alpha_n)_{n \in \N} \mapsto (\frac{1}{n} \cdot \alpha)_{n \in \N}$.
Betrachte $T := M S$.

\begin{enumerate}[label = (\alph*)]

  \item
  Für $k \in \N$ bestimme $\norm{T^k}$ und berechne $\lim_{n \to \infty} \norm{T^k}^\frac{1}{k}$.

  \item
  Zeige dass $T$ kompakt ist.

  \item
  Zeige dass $\sigma_p(T) = \emptyset$ und $\sigma(T) = \Bbraces{0}$.

\end{enumerate}

\end{exercise}

\begin{solution}

\phantom{}

\begin{enumerate}[label = (\alph*)]

  \item
  Zuerst wollen wir uns ansehen, wie $T$ tatsächlich aussieht.

  \begin{align*}
    T \vec x
    & =
    \pbraces
    {
      0,
      \frac{1!}{2!} x_1,
      \frac{2!}{3!} x_2,
      \frac{3!}{4!} x_3,
      \ldots
    }, \\
    T^2 \vec x
    & =
    \pbraces
    {
      0, 0,
      \frac{1!}{3!} x_1,
      \frac{2!}{4!} x_2,
      \frac{3!}{5!} x_3,
      \ldots
    }, \\
    T^3 \vec x
    & =
    \pbraces
    {
      0, 0, 0,
      \frac{1!}{4!} x_1,
      \frac{2!}{5!} x_2,
      \frac{3!}{6!} x_3,
      \ldots
    }, \\
    \vdots \\
    T^k \vec x
    & =
    \bigg (
      \underbrace{0, \ldots, 0}_{k \text{-mal}},
      \frac{1!}{(k+1)!} x_1,
      \frac{2!}{(k+2)!} x_2,
      \frac{3!}{(k+3)!} x_3,
      \ldots,
      \frac{n!}{(k+n)!} x_n,
      \ldots
    \bigg )
  \end{align*}

  Wir wollen nun eine geeignete Abschätzung für $\norm{T^k}$ finden.
  $\Forall \vec x \in \ell^2(\N):$

  \begin{align*}
    \norm[2]{T^k \vec x}^2
    & =
    \sum_{n=1}^\infty
    \pbraces
    {
      \frac{n!}{(k+n)!}
      x_n
    }^2
    \leq
    \sum_{n=1}^\infty
    \pbraces
    {
      \frac{1}{(k+1)!}
      x_n
    }^2
    =
    \pbraces{\frac{1}{(k+1)!}}^2
    \sum_{n=1}^\infty x_n^2
    =
    \pbraces{\frac{1}{(k+1)!}}^2
    \norm[2]{\vec x}^2 \\
    \implies
    \norm{T^k}
    & \leq
    \frac{1}{(k+1)!}
  \end{align*}

  Mit $x = (1,0,\dots)$ folgt, dass
  \begin{align*}
    \|T^k(x_n)_{n \in \N}\|_2^2 = \left(\frac{1}{(k+1)!}\right)^2,
  \end{align*}
  und damit $\|T_k\| = \frac{1}{(k+1)!}$. \\

  \underline{Grenzwert abschätzen (mit Integral):} \\

  Mit den Logarithmus-Regeln, können wir Produkte, also insbesondere Fakultäten, zu Summen machen.

  \begin{align*}
    \norm{T^k}^{1/k}
    =
    \pbraces{\frac{1}{(k+1)!}}^{1/k}
    =
    \exp \ln \pbraces{\frac{1}{(k+1)!}}^{1/k}
    =
    \exp \pbraces{-\frac{1}{k} \sum_{i=1}^{k+1} \ln{i}}
  \end{align*}

  Manchmal, kann man Summen mit Integralen abschätzen, die WolframAlpha dann berechnen kann.

  \begin{align*}
    \frac{1}{k}
    \sum_{i=1}^{k+1} \ln{i}
    =
    \frac{1}{k}
    \sum_{i=0}^k \ln{(i+1)}
    \geq
    \frac{1}{k}
    \Int[1][k]{\ln{(t+1)}}{t}
    =
    \frac{1}{k}
    \pbraces
    {
      -k + (k+2) \ln{(k+2)} - \ln{4}
    }
    \xrightarrow{k \to \infty} \infty
  \end{align*}

  Weil $\exp$ stetig ist, rutscht der $\lim$ rein.

  \begin{align*}
    \norm{T^k}^{1/k}
    \xrightarrow{k \to \infty} 0
  \end{align*}

  \underline{Grenzwert abschätzen (mit Induktion):} \\

  Wir zeigen mittels Induktion, dass $k! \geq (\frac{k}{3})^k$.
  Für $k = 1$ stimmt die Ungleichung klarerweise und
  \begin{align*}
    (k + 1)! = (k+1)k! \geq (k+1)(\frac{k}{3})^k
    = \frac{\frac{k^k}{(k+1)^k}(k+1)^{k+1}}{3^k}
    = \frac{(k+1)^{k+1}}{(1 + \frac{1}{k})^k3^k}
    \geq \frac{(k+1)^{k+1}}{e3^k} \geq \frac{(k+1)^{k+1}}{3^{k+1}}.
  \end{align*}
Damit folgt
\begin{align*}
  \|T^k\|^{\nicefrac{1}{k}} \leq \frac{1}{\frac{k}{3}(k+1)^{\nicefrac{1}{k}}}
  \leq \frac{3}{k} \to 0.
\end{align*}

  \item
  Nachdem $T = M S$, können wir auch, laut Proposition 6.5.4 (iv), zeigen, dass $M$ oder $S$ kompakt ist.

  \includegraphicsboxed{Proposition 6.5.4}

  $S$ ist auf jeden Fall mal NICHT kompakt!
  Sonst wäre, laut Proposition 6.5.4 (iv) $S^\ast S = I$ kompakt!
  Man erinnere sich, dass, laut Aufgabe 19/1, $S^\ast$ der umgekehrte Shift-Operator ist. \\

  Wir hoffen also, dass $M$ kompakt ist.
  Das zeigen wir so wie in Aufgabe 20/1, indem wir
  $M$ mit kompakten Operatoren approximieren.

  \begin{align*}
    M_j:
    \vec \alpha
    \mapsto
    \pbraces
    {
      \frac{\alpha_1}{1},
      \ldots,
      \frac{\alpha_j}{j},
      0, 0, 0, \ldots
    }
  \end{align*}

  Diese Operatoren haben endlich-dimensionales Bild, sind linear und beschränkt, daher
  laut Proposition 6.5.4 (i) kompakt.
  Wir benötigen also nur noch die Konvergenz von $(M_j)_{j \in \N}$ gegen $M$ in der Operator-Norm.

  \begin{align*}
    \norm[2]{(M - M_j) \vec \alpha}^2
    & =
    \sum_{n = j+1}^\infty
    \pbraces
    {
      \frac{\alpha_n}{n}
    }^2
    \leq
    \norm[2]{\vec \alpha}^2
    \sum_{n = j+1}^\infty
    \frac{1}{n^2}, \\
    \implies &
    \norm{M - M_j}
    \leq
    \sqrt
    {
      \sum_{n = j+1}^\infty
      \frac{1}{n^2}
    }
    \xrightarrow{j \to \infty} 0
  \end{align*}

  Laut Proposition 6.5.4 (iii), ist $M$ also kompakt.
  Damit, ist auch $T$ kompakt.

  \item
  \phantom{}

  \includegraphicsboxed{Satz 6.4.14}

  Laut Satz 6.4.14, ist also

  \begin{align*}
    0
    =
    \lim_{k \to \infty} \norm{T^k}^{1/k}
    =
    r(T)
    =
    \max_{\lambda \in \sigma(T)} |\lambda|.
  \end{align*}

  Weil $T$ nur links-invertierbar ist, ist $0 \in \sigma(T)$.
  Insgesamt, erhalten wir also $\sigma(T) = \Bbraces{0}$. \\

  Aus der links-Invertierbarkeit von $T$, folgt Injektivität, und damit $\ker{(T - 0 I)} = \ker{T} = \Bbraces{0}$.
  Also ist $0 \notin \sigma_p(T)$ und damit $\sigma_p(T) = \emptyset$.

\end{enumerate}

\end{solution}


\phantom{}

\underline{Vorbemerkung für die nächsten Aufgaben:} \\

Sei $X$ eine Menge und $\nu$ ein Maß auf $X$.
Hat man eine Funktion $k: X \times X \to C$, so kann man die Abbildung $K$ betrachten, die definiert ist als

\begin{align*}
  (K f)(x)
  :=
  \Int[X]{k(x, y) f(y)}{\nu(y)}
\end{align*}

Man bezeichnet $K$ als den \textit{Integraloperator mit Kern $k$ und Maß $\nu$}.
Natürlich ist von vornherein überhaupt nicht klar für welche Funktionen $f$ das Integral in der Definition von $K$ überhaupt existiert, und für welche Räume man $K$ als Operator zwischen ihnen betrachten kann.
Dies muss von Fall zu Fall spezifiert werden, und es ergeben sich verschiedene Ergebnisse, je nachdem welche Voraussetzungen man an $k$ stellt und zwischen welchen Räumen man den Operator $K$ betrachtet.

\begin{exercise}[IO/1]

Sei $X$ eine Menge, $\mu$ ein $\sigma$-endliches Maß auf $X$, sei $k \in L^2(\mu \times \mu)$, und betrachte den Integraloperator $K$ mit Kern $k$ und Maß $\mu$. \\

Zeige, dass $K \in \mathcal{B}(L^2(\mu))$ mit $\norm{K} \leq \norm[L^2(\mu \times \mu)]{k}$.
Zeige, dass seine Hilbertraumadjungierte $K^\ast \in \mathcal{B}(L^2(\mu))$ der Integraloperator mit Kern $k^\ast(x, y) := \overline{k(y, x)}$ ist.

\end{exercise}

\begin{solution}
  Die Linearität von $K$ folgt direkt aus der des Integrals.
  Wir zeigen zuerst $\|K\| \leq \|k\|$; insbesondere ist $K$ ein beschränkter Operator:
  \begin{align*}
    \|Kf\|_{L^2(\mu)} = \int_X |Kf(x)|^2 ~\mathrm{d}\mu(x)
    = \int_X \left|\int_X k(x,y) f(y) ~\mathrm{d}\mu(y)\right|^2 ~\mathrm{d}\mu(x) \\
    \stackrel{\text{CSB}}{\leq} \int_X \left(\int_X |k(x,y)|^2  ~\mathrm{d}\mu(y)\right) \left(\int_X |f(y)|^2 ~\mathrm{d}\mu(y)\right)
    ~\mathrm{d}\mu(x) \leq \|f\|_{L^2(\mu)} \int_X \int_X |k(x,y)|^2 ~\mathrm{d}\mu(y) ~\mathrm{d}\mu(x) \\ \stackrel{\text{Fubini}}{=} \|f\|_{L^2(\mu)} \int_{X^2} |k(x,y)|^2 ~\mathrm{d}(\mu\times\mu)(x,y)
    = \|f\|_{L^2(\mu)} \|k\|_{L^2(\mu\times\mu)}.
\end{align*}

  Zur Berechnung der Hilbertraumadjungierten erinnern wir uns an Folgendes:

  \begin{itemize}
      \item $L^2(\mu)$ ist isomorph zu seinem eigenen Dualraum vermöge folgender Abbildung:

      $\phi: L^2(\mu) \rightarrow L^2(\mu)^\prime: f \mapsto
      (g \mapsto \int_X f\overline{g} ~\mathrm{d}\mu).$
      \item Die Konjugierte von $K$ ist wie folgt definiert:

      $K^\prime: L^2(\mu)^\prime \rightarrow L^2(\mu)^\prime:
      y^\prime \mapsto (x \mapsto y'(K(x))).$
  \end{itemize}

  Damit berechnen wir
  \begin{align*}
      K^*(f) := \phi^{-1}(K^{\prime}(\phi(f))) =
      \phi^{-1}\left(K^\prime\left(g \mapsto \int_X f\overline g  ~\mathrm{d}\mu\right)\right)
      = \phi^{-1}\left(z \mapsto \int_X f(x) ~\overline{K(z)}(x) ~\mathrm{d}\mu(x)\right) \\
      = \phi^{-1}\left(z \mapsto \int_X f(x)
      ~\overline{\int_X k(x,y) z(y) ~\mathrm{d}\mu(y)}
      ~\mathrm{d}\mu(x)\right)
      = \phi^{-1}\left(z \mapsto \int_X
      ~\int_X f(x) \overline{k(x,y)} \overline{z(y)} ~\mathrm{d}\mu(y)
      ~\mathrm{d}\mu(x)\right) \\
      \stackrel{\text{Fubini}}{=} \phi^{-1}\left(z \mapsto \int_X
      ~\int_X f(x) \overline{k(x,y)} \overline{z(y)} ~\mathrm{d}\mu(x)
      ~\mathrm{d}\mu(y)\right)
      = \phi^{-1}\left(z \mapsto \int_X
      \left(\int_X f(x) \overline{k(x,y)} ~\mathrm{d}\mu(x) \right) \overline{z(y)}
      ~\mathrm{d}\mu(y)\right) \\
    = \int_X \overline{k(x,y)} f(x) ~\mathrm{d}\mu(x).
    \end{align*}


\end{solution}

\begin{exercise}[IO/2]

Sei $X$ eine Menge und $\mu$ ein $\sigma$-endliches Maß auf $X$.
Zeige:

\begin{enumerate}[label = (\alph*)]

  \item
  Seien $a_i, b_i \in L^2(\mu), i = 1, \ldots, n$.
  Setze $k(s, t) := \sum_{i=1}^n a_i(s) b_i(t)$ und betrachte den Integraloperator $K \in \mathcal{B}(L^2(\mu))$ mit Kern $k$.
  Dann ist $\dim{\ran{K}} \leq n$.

  \item
  Sei $k \in L^2(\mu \times \mu)$.
  Dann ist der Integraloperator $K \in \mathcal{B}(L^2(\mu))$ mit Kern $k$ und Maß $\mu$ kompakt.

\end{enumerate}

\end{exercise}

\begin{solution}

\phantom{}

\begin{enumerate}[label = (\alph*)]

  \item
  Zuerst wollen wir uns ansehen, wie $K$ tatsächlich aussieht.

  \begin{align*}
    (K f)(s)
    & =
    \Int[X]
    {
      \sum_{i=1}^n
      a_i(s) b_i(t) f(t)
    }{\mu(t)}
    =
    \sum_{i=1}^n
    a_i(s) \Int[X]{b_i(t) f(t)}{\mu(t)}
    =
    \sum_{i=1}^n
    a_i(s) (b_i, f)
  \end{align*}

  Damit, sieht das Bild von $K$ wie folgt aus.

  \begin{align*}
    \ran{K}
    =
    \Bbraces
    {
      K f:
      f \in L^2(\mu)
    }
    =
    \Bbraces
    {
      \sum_{i=1}^n
      a_i (b_i, f):
      f \in L^2(\mu)
    }
    \subseteq
    \Span \Bbraces{a_1, \ldots, a_n}
    =: A_n
  \end{align*}

  Daher, muss $\dim \ran{K} \leq \dim{A_n} \leq n$.

  \item
  Sei $K_n$ ein Integraloperator der Form
\begin{align}
    K_n f(s) = \int_X \left(\sum_{i=1}^n a_i(s) b_i(t)\right) f(t) \dmu(t).
\end{align}

$K_n$ hat nach Aufgabe (a) endlichdimensionales Bild und ist daher nach Proposition 6.5.4 kompakt. Wir wollen $K$ nun durch eine Folge $(K_n)_{n \in \mathbb{N}}$ solcher Operatoren approximieren. Es gilt
\begin{align}
    \|(K-K_n) f\|_2^2 = \int_X \left| \int_X k(s,t) f(t) \dmu(t) - \int_X \left(\sum_{i=1}^n a_i(s) b_i(t)\right) f(t) \dmu(t) \right|^2 \dmu(s) \\
    \leq \int_X \left( \int_X |k(s,t) - \sum_{i=1}^n a_i(s) b_i(t)| |f(t)| \dmu(t)\right)^2 \dmu(s) \\
    \leq \int_X \left(\int_X |k(s,t) - \sum_{i=1}^n a_i(s) b_i(t)|^2 \dmu(t)\right) \left(\int_X |f(t)|^2 \dmu(t)\right) \dmu(s) \\
    = \|f\|_2 \int_{X^2} |k(s,t) - \sum_{i=1}^n a_i(s) b_i(t)|^2 ~\mathrm{d}(\mu\times\mu)(s,t).
\end{align}

Aus der Maßtheorie wissen wir, dass wir $k$ von unten durch Treppenfunktionen der Form $t_n(s,t) = \sum_{i=1}^{m_n} \alpha_{n_i} \1_{C_{n_i}}(s,t)$ approximieren können; seien $(t_n)_{n \in \mathbb{N}}$ also von dieser Gestalt und gelte
\begin{align}
    \forall n \in \mathbb{N}:~ \|t_n\|_{L^2(\mu\times\mu)} \leq \|k\|_{L^2(\mu\times\mu)}, ~~\lim\limits_{n \rightarrow \infty}{\|k-t_n\|_{L^2(\mu\times\mu)}} = 0.
\end{align}

Sei $C \in \mathfrak{S} \times \mathfrak{S}$ beliebig; weil $\mu$ sigmaendlich ist, können wir nach dem Approximationssatz (vgl. Kusolitsch Satz 2.59) ein $D$ aus dem zugehörigen Ring finden, das $C$ beliebig genau approximiert. Genauer erhalten wir für beliebiges $\epsilon > 0$ ein $D$ mit
\begin{align}
    D = \bigcup_{i=1}^{l} A_j \times B_j \text{~(wobei~} A_j, B_j \in \mathfrak{S}), ~~\mu\times\mu(C \triangle D) < \epsilon.
\end{align}

Damit gilt
\begin{align}
    \|\1_C - \1_D\|_2^2 = \int_{X^2} |\1_C(s,t) - \1_D(s,t)|^2 ~\mathrm{d}(\mu\times\mu)(s,t) = \int_{X^2} \1_{C \triangle D}~\mathrm{d}(\mu\times\mu)(s,t) = \mu\times\mu(C \triangle D) < \epsilon.
\end{align}

Wir können also eine Folge $D_n$ wählen mit $\|\1_C - \1_D\|_2 < \frac{1}{n},$ also $D_n \xrightarrow{\|.\|_2} C.$

Zurück zur Aufgabe: Für gegebenes $\epsilon > 0$ wählen wir für jedes $C_{n_i}$ ein $D_{n_i}$ mit $\|\1_{C_{n_i}} - \1_{D_{n_i}}\|_2 < \frac{\epsilon}{m_n ~|\alpha_{n_i}|}.$

Mit dieser Wahl ist
\begin{align}
    \left\| \sum_{i=1}^{m_n} \alpha_{n_i} \1_{C_{n_i}} - \sum_{i=1}^{m_n} \alpha_{n_i} \1_{D_{n_i}}\right\|_2
    \leq \sum_{i=1}^{m_n} |\alpha_{n_i}| \|\1_{C_{n_i}} - \1_{D_{n_i}} \|_2 < \epsilon.
\end{align}

Weiters gilt
\begin{align}
    \sum_{i=1}^{m_n} \alpha_{n_i} \1_{D_{n_i}}(s,t)
    = \sum_{i=1}^{m_n} \alpha_{n_i} \sum_{j=1}^{l_{n_i}}  \1_{A_{n_{i_j}} \times B_{n_{i_j}}}(s,t)
    = \sum_{i=1}^{m_n} \sum_{j=1}^{l_{n_i}}   \alpha_{n_i} \1_{A_{n_{i_j}}}(s) \1_{B_{n_{i_j}}}(t).
\end{align}

Wir finden also Funktionen der Form
\begin{align}
    s_{n_q} (s,t) = \sum_{j=1}^{p_{n_q}} \beta_{n_q} \1_{A_{n_{i_j}}}(s) \1_{B_{n_{i_j}}}(t)
\end{align}
mit $\|s_{n_q} - t_n\| \xrightarrow{q \rightarrow \infty} 0.$
Sei nun $\epsilon > 0$ beliebig. Dann gilt für hinreichend große $n$ und $q$
\begin{align}
    \| k-s_{n_q} \|_2 \leq \|k - t_n\|_2 + \|t_n - s_{n_q}\|_2 < \epsilon.
\end{align}

\end{enumerate}

\end{solution}

\begin{exercise}[IO/3]

Der Volterra-Operator ist der Integraloperator $(V f)(x) := \Int[0][x]{f(t)}{t}$.
Zeige, dass $V \in \mathcal{B}(L^2(0, 1))$ mit $\norm{V} = \frac{2}{\pi}$, dass $V$ kompakt ist, und dass

\begin{align*}
  (V^\ast f)(x)
  =
  \Int[x][1]{f(t)}{t}.
\end{align*}

Zeige, dass $\sigma(V) = \sigma_c(V) = \Bbraces{0}$.

\end{exercise}

\begin{solution}

ToDo!

\end{solution}

\begin{exercise}[IO/4]

Sei $k \in C([0, 1]^2)$, und betrachte den Integraloperator $(K f)(x) := \Int[0][1]{k(x, t)f(t)}{t}$.
Zeige, dass $K \in \mathcal{B}(C([0, 1]))$ mit

\begin{align*}
  \norm{K}
  =
  \sup_{x \in [0, 1]} \Int[0][1]{|k(x, t)|}{t}
  \leq
  \norm[{C([0, 1]^2)}]{k}.
\end{align*}

Zeige, dass $K$ kompakt ist.

\end{exercise}

\begin{solution}

Die Linearität von $K$ folgt aus der Linearität des Integrals.

Für $f \in C([0,1])$ ist auch $Kf$ wieder stetig.

Sei nun $f \in C([0,1])$ beliebig mit $\|f\|_{\infty} = 1$. Dann gilt

\begin{align*}
  \|Kf\|_{\infty} =& \sup_{x \in (0,1)} | \int_{0}^{1} k(x,t)f(t)dt | \leq \sup_{x \in (0,1)} \int_{0}^{1} |k(x,t)||f(t)|dt \\
  \leq& \|f\|_{\infty} \sup_{x \in (0,1)} \int_{0}^{1} |k(x,t)|dt = \sup_{x \in (0,1)} \int_{0}^{1} |k(x,t)|dt
\end{align*}

Somit erhalten wir laut dem Lemma vom iterierten Supremum also $\|K\| \leq \sup_{x \in (0,1)} \int_{0}^{1} |k(x,t)|dt \leq \|k\|_{\infty}$.

Für die andere Richtung der ersten Ungleichung betrachten wir das $x_0 \in [0,1]$, wo $\int_{0}^{1} |k(x,t)|dt$ das Supremum/ Maximum annimmt. Dieses existiert, da das Parameterintegral stetig ist.

Definiere nun
\begin{align*}
  f(t) := \begin{cases}
    \frac{\overline{k(x_0,t)}}{|k(x_0,t)|}, & |k(x_0,t)| \neq 0 \\
    0, & \text{sonst}
  \end{cases} \in L^1
\end{align*}
\begin{align*}
  \|K\| \geq \frac{\|Kf\|}{\|f\|} =
\end{align*}

Es gibt also eine Folge $(f_n)_{n \in \N} \in C_c^{\infty}$ mit
\begin{align*}
  f_n \stackrel{\|\cdot\|_1}{\to} f
\end{align*}

Nach Kusolitsch Satz 13.25 gilt auch Konvergenz im Maß und nach Kusolitsch Satz 7.88 können wir eine Teilfolge $f_{n_l}$ wählen, die $\lambda-$f.ü. konvergiert.

Damit gilt

\begin{align*}
  \|K\| \geq& \lim_{l \to \infty} \|Kf_{n_l}\| \geq \lim_{l \to \infty} |Kf_{n_l}(x_0)| = \lim_{n \to \infty} |\int_{0}^{1} k(x_0,t)f_{n_l}(t) dt| \\
  =& |\int_{0}^{1} k(x_0,t)f(t) dt| = \int_{0}^{1} |k(x_0,t)| dt = \sup_{x \in (0,1)} \int_{0}^{1} |k(x_0,t)| dt.
\end{align*}

wobei die drittletzte Gleichheit gilt mit Konvergenz durch Majorisierung nach Kusolitsch Satz 9.23.

\begin{align*}
  \Forall l \in \N, t \in [0,1]: |k(x_0,t)f_{n_l}(t)| = |k(x_0,t)| \leq \|k(x_0,\cdot)\|_{\infty}
\end{align*}
Erste Gleichheit ist mir noch nicht klar....Fabian fragen

Nun zu der Kompaktheit von $K$.

 Um zu zeigen, dass $K(K_{1}^{\|\cdot\|_{\infty}}(0)) = \{Kf : f \in K_{1}^{\|\cdot\|_{\infty}}(0)\}$ relativ kompakt ist, verwenden wir den Satz von Arzela-Ascoli.

 Es reicht also zu zeigen, dass $K(K_{1}^{\|\cdot\|_{\infty}}(0))$ punktweise beschränkt und gleichgradig stetig ist.

 \begin{enumerate}

 \item Punktweise beschränkt

 Für $x \in [0,1]$ und $f \in K_{1}^{\|\cdot\|_{\infty}}(0)$ beliebig gilt:
\begin{align*}
  |Kf(x)| \leq \|Kf\|_{\infty} \leq \|K\| \|f\|_{\infty} \leq \|K\|
\end{align*}

\item Gleichgradig stetig:

Allgemein gilt für $x_1, x_2 \in [0,1]$ und $f \in K_{1}^{\|\cdot\|_{\infty}}(0)$ beliebig, dass
\begin{align*}
  |Kf(x_2) - Kf(x_1)| =& | \int_{0}^{1} k(x_2,t)f(t) - k(x_1,t)f(t) dt| \leq \int_{0}^{1} |f(t)||k(x_2,t)- k(x_1,t)|dt \\
  \leq& \int_{0}^{1} |k(x_2,t)- k(x_1,t)|dt
\end{align*}

Sei also $\epsilon > 0$ beliebig, dann wähle $\delta$ so, dass
\begin{align*}
  \Forall (x_1,t_1),(x_2,t_2): | (x_1,t_1) - (x_2,t_2)| \leq \delta \Rightarrow | k(x_1,t_1) - k(x_2,t_2)| \leq \epsilon
\end{align*}

Dann gilt für beliebiges $x_1 \in [0,1]$, dass für alle $f \in K_{1}^{\|\cdot\|_{\infty}}(0)$ und $x_2 \in [x_1-\delta,x_1+\delta]$

\begin{align*}
  |Kf(x_2) - Kf(x_1)| \leq \int_{0}^{1} |k(x_2,t)- k(x_1,t)|dt \leq \int_{0}^{1} \epsilon dt = \epsilon
\end{align*}

 \end{enumerate}


\end{solution}

\begin{exercise}[IO/5]

Gibt es eine stetige Funktion $f: [0, 1] \to \R$, die der Gleichung

\begin{align*}
  f(x) + \Int[0][x]{e^{x \cos{t}} f(t)}{t}
  =
  x^2 + 1,
  \quad
  x \in [0, 1],
\end{align*}

genügt?
Falls ja, ist $f$ eindeutig? \\

\textit{Hinweis.}
Ist der Punkt $-1$ im Spektrum des Operators?

\end{exercise}

\begin{solution}

ToDo!

\end{solution}


\end{document}
