\begin{exercise}[20/1]

Sei $(a_n)_{n \in \N} \in \ell^1(\N)$, und betrachte den Operator auf $\ell^2(\N)$ der durch die Matrix

\begin{align*}
  A
  :=
  (a_{i+j-1})_{i,j = 1}^\infty
  =
  \begin{pmatrix}
    a_1    & a_2    & a_3    & \cdots \\
    a_2    & a_3    & \cdots &        \\
    a_3    & \cdots &        &        \\
    \cdots &        &        &
  \end{pmatrix}
\end{align*}

gegeben ist.
Explizite agiert $A$ also als $(x_n)_{n \in \N} \mapsto \pbraces{\sum_{k=1}^\infty a_{k+n-1} x_k}_{n \in \N}$.
Sei (der Einfachheit halber) weiters voausgesetzt, dass $(a_n)_{n \in \N}$ monoton fallend ist.
Zeige, dass $A$ kompakt ist. \\

\textit{Hinweis.}
Approximiere $A$ mit Operatoren die endlichdimensionales Bild haben.

\end{exercise}

\begin{solution}

Zuerst zeigen wir, dass $A: \ell^2(\N) \to \ell^2(\N)$ tatsächlich eine Selbstabbildung ist.
Sei dazu $(x_n)_{n \in \N} \in \ell^2(\N)$.
Mit der Ungleichung von Cauchy-Schwarz-Bunjaycovski, und weil die Folge $(a_n)_{n \in \N}$ monoton fallend ist, folgt

\begin{align*}
  \norm[2]{A((x_n)_{n \in \N})}^2
  & =
  \sum_{n=1}^\infty
  \vbraces
  {
    \sum_{k=1}^\infty
    a_{k+n-1} x_k
  }^2
  \leq
  \sum_{n=1}^\infty
  \norm[1]{(a_{k+n-1} x_k)_{k \in \N}}^2
  \leq
  \sum_{n=1}^\infty
  \norm[2]{(a_{k+n-1})_{k \in \N}}^2
  \norm[2]{(x_k)_{k \in \N}}^2 \\
  & =
  \norm[2]{(x_k)_{k \in \N}}^2
  \sum_{n=1}^\infty
  \sum_{k=1}^\infty
  |a_{k+n-1}|^2
  \leq
  \norm[2]{(x_k)_{k \in \N}}^2
  \sum_{n=1}^\infty
  \sum_{k=1}^\infty
  |a_n| |a_k|
  =
  \norm[2]{(x_k)_{k \in \N}}^2
  \norm[1]{(a_n)_{n \in \N}}^2 < \infty
\end{align*}

Dem Hinweis folgend, definieren wir, für $j \in \N$, Operatoren

\begin{align*}
  A_j:
  \ell^2(\N)
  \to
  \Bbraces
  {
    (y_n)_{n \in \N} \in \ell^2(\N):
    \Forall n > j:
    y_n = 0
  }:
  (x_n)_{n \in \N}
  \mapsto
  \begin{cases}
    \sum_{k=1}^\infty
    a_{k+n-1} x_k
    &
    \text{für } n \leq j, \\
    0
    &
    \text{für } n > j.
  \end{cases}
\end{align*}

Diese Operatoren haben endlichdimensionales Bild.
Wir wollen zeigen, dass diese kompakt sind.

\includegraphicsboxed{Prop. 6.5.4}

Laut Proposition 6.5.4 (i), gilt es zu zeigen, dass die $(A_j)_{j \in \N}$ linear und beschränkt sind.

\begin{itemize}

  \item
  \Quote{Linearität}:

  \begin{align*}
    A_j((x_n)_{n \in \N} + c (y_n)_{n \in \N})
    & =
    \pbraces
    {
      \sum_{k=1}^\infty
      a_{k+n-1}
      (x_k + c y_k)
    }_{n=1}^j
    =
    \pbraces
    {
      \sum_{k=1}^\infty
      a_{k+n-1} x_k +
      c
      \sum_{k=1}^\infty
      a_{k+n-1} y_k
    }_{n=1}^j \\
    & =
    \pbraces
    {
      \sum_{k=1}^\infty
      a_{k+n-1} x_k
    }_{n=1}^j +
    c \pbraces
    {
      \sum_{k=1}^\infty
      a_{k+n-1} y_k
    }_{n=1}^j
    =
    A_j((x_n)_{n \in \N}) +
    c
    A_j((y_n)_{n \in \N}).
  \end{align*}

  \item
  \Quote{Beschränktheit}:

  Weil die Folge $(a_n)_{n \in \N}$ monoton fallend ist, gilt $\Forall n \in \N:$

  \begin{align*}
    \norm[2]{(a_{k+n-1})_{k \in \N}}
    \leq
    \norm[2]{(a_k)_{k \in \N}} < \infty.
  \end{align*}

  Ähnlich wie zuvor das Bild von $A$, kan man das von $A_j$ abschätzen.

  \begin{align*}
    \norm[2]{A_j((x_n)_{n \in \N})}^2
    & =
    \sum_{n=1}^j
    \vbraces
    {
      \sum_{k=1}^\infty
      a_{k+n-1} x_k
    }^2
    \leq
    \sum_{n=1}^j
    \norm[1]{(a_{k+n-1} x_k)_{k \in \N}}^2
    \leq
    \sum_{n=1}^j
    \norm[2]{(a_{k+n-1})_{k \in \N}}^2
    \norm[2]{(x_k)_{k \in \N}}^2 \\
    \implies &
    \norm{A_j}
    \leq
    \sqrt
    {
      \sum_{n=1}^j
      \norm[2]{(a_{k+n-1})_{k \in \N}}^2
    } < \infty
  \end{align*}

\end{itemize}

Jetzt brauchen wir noch die Konvergenz der $(A_j)_{j \in \N}$ gegen $A$ in der Operatornorm.

\begin{align*}
  \norm[2]{(A - A_j)(x_n)_{n \in \N}}^2
  & =
  \norm[2]
  {
    \pbraces
    {
      \sum_{k=1}^\infty
      a_{k+n-1} x_k
    }_{n = j+1}^\infty
  }^2
  =
  \sum_{n = j+1}^\infty
  \vbraces
  {
    \sum_{k=1}^\infty
    a_{k+n-1} x_k
  }^2
  \leq
  \sum_{n = j+1}^\infty
  \norm[1]{(a_{k+n-1} x_k)_{k \in \N}}^2 \\
  & \leq
  \sum_{n = j+1}^\infty
  \norm[2]{(a_{k+n-1})_{k \in \N}}^2
  \norm[2]{(x_n)_{n \in \N}}^2
  =
  \norm[2]{(x_n)_{n \in \N}}^2
  \sum_{n = j+1}^\infty
  \sum_{k=1}^\infty
  |a_{k+n-1}|^2 \\
  & \leq
  \norm[2]{(x_n)_{n \in \N}}^2
  \sum_{n = j+1}^\infty
  \sum_{k=1}^\infty
  |a_n| |a_k|
  =
  \norm[2]{(x_n)_{n \in \N}}^2
  \norm[1]{(a_k)_{k \in \N}}
  \sum_{n = j+1}^\infty |a_n| \\
  \implies &
  \norm{A - A_j}
  \leq
  \sqrt
  {
    \norm[1]{(a_k)_{k \in \N}}
    \sum_{n = j+1}^\infty |a_n|
  }
  \xrightarrow{j \to \infty} 0
\end{align*}

Letztere Konvergenz gilt, weil $(a_n)_{n \in \N} \in \ell^1(\N)$. \\

Laut Proposition 6.5.3 (iii), ist die Menge der kompakten Operatoren bezüglich der Operatornorm in $L_b(X, Y)$ abgeschlossen.
Der Grenzwert $A$, der $(A_j)_{j \in \N}$, ist also ebenfalls kompakt. \\

\underline{Wenn das nicht reicht:} \\

\includegraphicsboxed{Kor. 4.2.3}

$A$ ist, wegen Korollar 4.2.3, tatsächlich linear und beschränkt.
(Wir haben ja gerade gezeigt, dass der Grenzwert existiert).

\end{solution}
