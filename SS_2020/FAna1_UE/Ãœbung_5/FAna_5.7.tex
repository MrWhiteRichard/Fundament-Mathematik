\begin{exercise}[22/2$^\ast$]

Sei $\mathcal{K}(\C)$ die Menge aller kompakten Teilmengen von $\C$, und

\begin{align*}
  d_H(M, N)
  :=
  \max \Bbraces
  {
    \sup_{x \in M} \inf_{y \in N} \vbraces{x - y},
    \sup_{y \in N} \inf_{x \in M} \vbraces{x - y}
  },
  \quad
  M, N \in \mathcal{K}(\C).
\end{align*}

Es gilt dass $d_H$ eine Metrik ist, die \textit{Hausdorff-Metrik}.
Sei nun $X$ ein Banachraum.
Zeige:

\begin{enumerate}[label = (\alph*)]

  \item
  Sind $A, B \in \mathcal{B}(X)$ mit $A B = B A$, dann ist $d_H(\sigma(A), \sigma(B)) \leq r(A - B)$.

  \item
  Ist $\mathcal{C}$ eine kommutative Teilalgebra von $\mathcal{B}(X)$, so ist die Funktion

  \begin{align*}
    \Sigma:
    \begin{cases}
      (\mathcal{C}, \norm[\mathcal{B}(X)]{.}) & \to (\mathcal{K}(\C), d_H) \\
      A & \mapsto \sigma(A)
    \end{cases}
  \end{align*}

  stetig.

\end{enumerate}

\end{exercise}

\begin{solution}

\phantom{}

\begin{enumerate}[label = (\alph*)]

  \item
  Wir zeigen zuerst eine Hilfsaussage.

  \begin{align} \label{Lemmachen}
    \Forall C \in \mathcal{B}(X):
    r(C) = r(-C)
  \end{align}

  \textit{Beweis.}

  \begin{align*}
    (
      C - \lambda I \in \Inv(\mathcal{B}(X))
      & \iff
      (-C) - (-\lambda) I = -(C - \lambda I) \in \Inv(\mathcal{B}(X))
    ) \\
    & \implies \\
    (
      \lambda \in \sigma (C)
      & \iff
      -\lambda \in \sigma (-C)
    ) \\
    & \implies
    r(C) = \max_{\lambda \in \sigma (C)} |\lambda| = \max_{\lambda \in \sigma (C)} |-\lambda| =
    \max_{\lambda \in \sigma (-C)} |\lambda| = r(-C)
  \end{align*}

  \includegraphicsboxed{Lemma 6.4.10}

  Wir wissen nach Lemma 6.4.10, dass die Spektren kompakt sind.
  Somit, ist $d_H(\sigma(A), \sigma(B))$ wohldefiniert. \\

  Sei o.B.d.A. $\sigma(A) \neq \sigma(B)$ (sonst ist die linke Seite $=0$).
  Sei o.B.d.A. $\sigma(A) \supsetneq \sigma(B)$.
  Wir können also ein $\lambda \in \sigma(A) \setminus \sigma(B)$ wählen und wollen Folgendes zeigen (auch für $B$).

  \begin{align} \label{minuslambda}
    \sigma(A-\lambda I) = \sigma(A) - \lambda
  \end{align}

  \textit{Beweis.}

  \begin{enumerate}[label = \arabic*.]

    \item
    Sei $\tau \in \sigma(A-\lambda I)$. Dann gilt nach der Definition von $\sigma$, dass

    \begin{align*}
      A - (\lambda + \tau) I
      =
      (A - \lambda I) - \tau I
      \notin
      \Inv(\mathcal{B}(X)) \\
      \implies
      \lambda + \tau \in \sigma(A).
    \end{align*}

    \item
    Sei $\tau \in \sigma(A) - \lambda$. Dann gilt nach der Definition von $\sigma$, dass

    \begin{align*}
      (A - \lambda I) - \tau I
      =
      A - (\lambda + \tau) I
      \notin
      \Inv(\mathcal{B}(X)) \\
      \implies
      \tau \in \sigma(A - \lambda I).
    \end{align*}

  \end{enumerate}

  Wir setzen nun $T := A - B$. Dann folgt $A- \lambda I = B - \lambda I + T$. \\

  Wir haben $\lambda$ so gewählt, dass $\lambda \in \sigma(A)$.
  Also ist auch $0 \in \sigma(A - \lambda I)$ und $0 \notin \rho(A - \lambda I)$. \\
  Wir haben $\lambda$ so gewählt, dass $\lambda \notin \sigma(B)$.
  Also ist auch $0 \notin \sigma(B - \lambda I)$ und $0 \in \rho(B - \lambda I)$. \\

  Wir betrachten die Kontraposition der vorigen Aufgabe.
  $(B - \lambda I)$ nehme die Rolle von $A$ ein und $(A - \lambda I)$ die von $B$.
  Offensichtlich kommutieren sie.

  \begin{align*}
    r(A - B)
    =
    r(T)
    \geq
    r((B-\lambda I)^{-1})^{-1}.
  \end{align*}

  \includegraphicsboxed{Lemma 6.4.8}

  \begin{align*}
    \implies
    r(A - B)
    & \geq
    r((B - \lambda I)^{-1})^{-1}
    \stackrel{\text{Definition}}{=}
    \pbraces
    {
      \max_{\beta \in \sigma((B - \lambda I)^{-1})} |\beta|
    }^{-1}
    \stackrel{\text{Lemma 6.4.8}}{=}
    \pbraces
    {
      \max_{\beta \in \sigma(B - \lambda I)} |\beta|^{-1}
    }^{-1} \\
    & \stackrel{\eqref{minuslambda}}{=}
    \pbraces
    {
      \max_{\beta \in \sigma(B)}
      |\beta - \lambda|^{-1}
    }^{-1}
    =
    \min_{\beta \in \sigma(B)}
    |\beta - \lambda|
    =
    \inf_{\beta \in \sigma(B)}
    |\beta - \lambda|
  \end{align*}

  Da $\lambda$ beliebig war, gilt also auch

  \begin{align*}
    r(A - B)
    \geq
    \sup_{\lambda \in \sigma(A) \setminus \sigma(B)}
    \inf_{\beta \in \sigma(B)} \vbraces{\beta - \lambda}
    =
    \sup_{\lambda \in \sigma(A)}
    \inf_{\beta \in \sigma(B)}
    \vbraces{\beta - \lambda}.
  \end{align*}

  Die Gleichung gilt, da für $\lambda \in \sigma(B) \cap \sigma(A)$ das Infimum gleich $0$ ist.

  Analog zeigt man

  \begin{align*}
    r(B - A)
    \geq
    \sup_{\beta \in \sigma(B)}
    \inf_{\lambda \in \sigma(A)}
    \vbraces{\beta - \lambda}.
  \end{align*}

  Laut \eqref{Lemmachen}, ist $r(A - B) = r(B - A)$.

  \begin{align*}
    \implies
    r(A-B)
    \geq
    \max \Bbraces
    {
      \sup_{\lambda \in \sigma(A)}
      \inf_{\beta \in \sigma(B)}
      \vbraces{\beta - \lambda},
      \sup_{\beta \in \sigma(B)}
      \inf_{\lambda \in \sigma(A)}
      \vbraces{\beta - \lambda}
    }
    =
    d_H(\sigma(A), \sigma(B))
  \end{align*}

  \item
  $\Sigma$ ist sogar Lipschitz-stetig mit Lipschitz-Konstante $1$.

  \begin{align*}
    d_H(\Sigma(A), \Sigma(B))
    =
    d_H(\sigma(A), \sigma(B))
    \leq
    r(A-B)
    \leq
    \norm{A - B}
  \end{align*}

\end{enumerate}

\end{solution}
