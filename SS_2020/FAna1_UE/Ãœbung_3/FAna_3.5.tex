\begin{exercise}

\phantom{}

\begin{enumerate}[label = (\roman*)]

  \item
  Zeige, dass es keine Funktion $f: \R \rightarrow \R$ gibt, die an allen rationalen Punkten stetig, aber an allen irrationalen Punkten unstetig ist.

  \item
  Finde eine Funktion $f: \R \rightarrow \R$ die an allen irrationalen Punkten stetig, aber an allen rationalen Punkten unstetig ist.

\end{enumerate}

\textit{Hinweis:}
Ist die Teilmenge $\Q$ von $\R$ (welche ja dicht liegt) eine $G_\delta$-Menge?

\end{exercise}

\begin{solution}

\phantom{}

\begin{enumerate}[label = (\roman*)]

  \item
  Angenommen, $\Q$ ist eine $G_\delta$-Menge.
  Wir wissen, dass

  \begin{align*}
    \Q \quad \textrm{und} \quad \R \setminus \Q
    \quad
    \textrm{dicht liegen in}
    \quad
    \R.
  \end{align*}

  Weiters erkennen wir, dass

  \begin{align*}
    \R \setminus \Q
    =
    \bigcap_{q \in \Q}( \R \setminus \Bbraces{q})
    \quad
    \textrm{eine $G_\delta$-Menge ist.}
  \end{align*}

  Laut Aufgabe 3, wäre dann deren Schnitt

  \begin{align*}
    \emptyset
    =
    \R \cap \Q^\complement \cap \Q
    =
    (\R \setminus \Q) \cap \Q \quad \textrm{eine dichte $G_\delta$-Menge.}
  \end{align*}

  Widerspruch!
  Nun wissen wir also, dass $\Q$ keine $G_\delta$-Menge ist.

  Angenommen, es gäbe eine Funktion $f: \R \to \R$ die bei $\Q$ stetig ist und bei $\R \setminus \Q$ unstetig ist.
  Dann wäre, laut Aufgabe 4, die Menge der Stetigkeitspunkte von $f$, also $\Q$, eine $G_\delta$-Menge.
  Widerspruch!
  Also, kann es solch eine Funktion nicht geben.

  \item
  Wir kennen so eine Funktion bereits aus Analysis 1.

  \begin{align*}
    f:
    \R \to \R:
    x \mapsto
    \begin{cases}
      \frac{1}{q},
      & \textrm{wenn} \enspace
      x = \frac{p}{q} \in \Q, \enspace \ggT(p, q) = 1, \\
      0,
      & \textrm{wenn} \enspace
      x \in \R \setminus \Q
    \end{cases}
  \end{align*}

  $\Forall x \in \R:$

  \begin{itemize}

    \item
    [\blockquote{$x \in \Q$}:]

    Wir schreiben $x$ als vollständig gekürzten Bruch $x = \frac{p}{q}$.
    $\Forall \delta > 0: \Exists y \in (\R \setminus \Q) \cap U_\delta(x):$

    \begin{align*}
      \vbraces{f(x) - f(y)}
      =
      \vbraces{f(x)}
      =
      \frac{1}{q}
      =:
      \epsilon.
    \end{align*}

    Also, ist $f$ im Punkt $x$ nicht stetig.

    \item
    [\blockquote{$x \in \R \setminus \Q$}:]

    Es gilt also $f(x) = 0$. Das ist wichtig!
    Seien $\epsilon > 0$ beliebig und

    \begin{align*}
      \delta
      :=
      \min_{q = 1}^{\ceil{1/\epsilon}}
      \pbraces
      {
        \min_{p \in \Z}
        \vbraces{\frac{p}{q} - x}
      }
      =
      \min \Bbraces
      {
        \vbraces{\frac{p}{q} - x}:
        q = 1, \ldots, \ceil{1/\epsilon},
        \enspace
        p \in \Z
      }.
    \end{align*}

    $\Exists q = 1, \ldots, \ceil{1/\epsilon}, \Exists p \in \Z:$

    \begin{align*}
      \delta = \vbraces{\frac{p}{q} - x}.
    \end{align*}

    $z := \frac{p}{q}$ ist die, zu $x$, näheste rationale Zahl, mit Nenner $q = 1, \ldots, \ceil{1/\epsilon}$.

    \begin{align*}
      \implies
      q < 1/\epsilon
      \implies
      \epsilon < 1/q
    \end{align*}

    $\Forall y \in U_\delta(x):$ \\

    \begin{itemize}

      \item
      [\blockquote{$y \in \R \setminus \Q$}:]

      \begin{align*}
        \implies
        f(y) = 0
        \implies
        |f(x) - f(y)| = 0 < \epsilon
      \end{align*}

      \item
      [\blockquote{$y \in \Q$}:]

      Wir schreiben $y$ als vollständig gekürzten Bruch $y = \frac{a}{b}$.
      Weil $|y - x| < \delta = |z - x|$, muss $y \neq z$.
      $y$ ist außerdem näher an $x$ als $z$ (an $x$).
      Was war $z$ nochmal?
      Also, $y$'s Nenner

      \begin{align*}
        b > \ceil{1/\epsilon} \geq 1/\epsilon
        \implies
        \epsilon > 1/b = |f(y) - f(x)|.
      \end{align*}

    \end{itemize}

  \end{itemize}

\end{enumerate}

\end{solution}
