\begin{exercise}

Eine Menge heißt $G_\delta$-Menge, wenn sie der abzählbare Durchschnitt offener Mengen ist.
Zeige, dass der Durchschnitt von abzählbar vielen dichten $G_\delta$-Mengen eines vollständigen metrischen Raumes wieder eine dichte $G_\delta$-Menge ist.

\end{exercise}

\begin{solution}

Sei $(X, d)$ vollständiger metrischer Raum und $(M_n)_{n \in \N}$
eine Folge von dichten $G_\delta$-Mengen.
D.h. $\Forall n \in \N: \Exists (O_{n, k})_{k \in \N} \in \mathcal{T}_{\norm[X]{\cdot}}^\N:$

\begin{align*}
  M_n = \bigcap_{k \in \N} O_{n, k}.
\end{align*}

Insbesondere, ist der Durchschnitt $M$ eine $G_\delta$-Menge.

\begin{align*}
  M := \bigcap_{n \in \N} M_n
     = \bigcap_{n, k \in \N} O_{n, k}
\end{align*}

Für alle $n, k \in \N$, ist $M_n \subseteq O_{n, k}$ und $M_n$ liegt dicht in $X$.
Damit, liegt auch $O_{n, k}$ dicht.
Laut dem Satz von Baire 4.1.1, liegt auch der Durchschnitt $M$ dicht.

\end{solution}
