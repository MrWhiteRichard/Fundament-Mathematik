\begin{exercise}

Sei $(H,(\cdot, \cdot)_H)$ ein Hilbertraum, und $[\cdot, \cdot]$ ein weiteres Skalarproduktauf $H$.
Sei vorausgesetzt, dass $[\cdot,\cdot]$ eine stetige, koerzive Sesquilinearform ist, also dass (hier bezeichnet $\norm[H]{\cdot}$ die von $(\cdot, \cdot)_H$ induzierte Norm)

\begin{align*}
  \Exists C > 0:
  \Forall x, y \in H:
  |[x, y]| \leq C\norm[H]{x} \norm[H]{y},
  \qquad
  \Exists m > 0:
  \Forall x \in H:
  [x, x] \geq m \norm[H]{x}^2.
\end{align*}

Weiters bezeichne $G$ den Gram-Operator der Sesquilinearform $[\cdot, \cdot]$ bezüglich $(\cdot, \cdot)_H$. \\
Zeige, dass es einen eindeutigen Operator $T \in \mathcal{B}(H)$ gibt, sodass $GT = TG = \id_H $ gilt. \\

\textit{Hinweis:}
Zeige, dass $(H, [\cdot, \cdot])$ ein Hilbertraum ist.

\end{exercise}

\begin{solution}

Wir zeigen, dass $(H, [\cdot, \cdot])$ ein Hilbertraum ist.

\begin{itemize}

  \item
  \enquote{positive Definitheit}: \\

  Die positive Definitheit von $[\cdot, \cdot]$ folgt direkt aus der zweiten Voraussetzung. \\

  \item
  \enquote{Vollständigkeit}:

  Sei $(x_n)_{n \in \N}$ eine Cauchyfolge bezüglich der von $[\cdot, \cdot]$ induzierten Norm.
  Es gilt nach Voraussetzung

  \begin{align*}
      \norm[H]{x_n - x_k}
      \leq
      \frac
      {
        \norm[{[\cdot, \cdot]}]{x_n - x_k}
      }{
        \sqrt{m}
      }.
  \end{align*}

  Die Folge ist also auch bezüglich der ursprünglichen Norm eine Cauchyfolge.
  Damit hat sie einen Grenzwert in $H$.

\end{itemize}

Weiters ist das ursprüngliche Skalarprodukt eine, durch $\frac{1}{m}$, beschränkte Sesquilinearform, weil

\begin{align*}
  (x, y)_H
  \leq
  \norm[H]{x} \norm[H]{y}
  \leq
  \frac{1}{m} \norm[{[\cdot, \cdot]}]{x} \norm[{[\cdot, \cdot]}]{y}.
\end{align*}

\includegraphicsboxed{Satz 3.2.6}

Nach dem Satz 3.2.6 (von Lax-Milgram), $\ExistsOnlyOne T \in L_b(H): \Forall x, y \in H:$

\begin{align*}
    (x, y)_H = [T x, y].
\end{align*}

$T$ ist der Gramoperator von $(\cdot, \cdot)_H$ bezüglich $[\cdot, \cdot]$.
$G$ ist jener von $[\cdot, \cdot]$ bezüglich $(\cdot, \cdot)_H$.
Also gilt $\Forall x, y \in H:$

\begin{align*}
    (x, y)_H = [T x, y] = (G T x, y)_H,
    \enspace
    [x, y] = (G x, y)_H = [T G x, y].
\end{align*}

Die erste Gleichung ist äquivalent zu $(y, x)_H = (y, GTx)_H$. \\

Ist $\Phi$ die Abbildung $y \mapsto (x \mapsto (x, y)_H)$, dann gilt also $\Phi(x) = \Phi(GTx)$.
Nach Proposition 3.2.5 (siehe oben), ist $\Phi$ bijektiv.
Damit ist $\Phi$ insbesondere injektiv und für alle $x$ gilt $x = GTx$. \\

Aus der zweiten Gleichung folgt, mit demselben Argument angewandt auf den Hilbertraum $(H, [\cdot, \cdot])$, dass $x = T G x.$

\end{solution}
