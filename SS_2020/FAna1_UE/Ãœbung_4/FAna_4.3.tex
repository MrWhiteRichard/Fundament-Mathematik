\begin{exercise}

Sei $n \in \N$, $\R^n[z]$ die Menge aller reellen Polynome vom Grad $\leq n$ und

\begin{align*}
  K := \Bbraces
  {
    p \in \R^n[z]:
    p^\primeprime(x) \geq 0,
    x \in [-1, 1]
  }.
\end{align*}

Zeige, dass es für jedes $f \in L^2([-1, 1])$ genau ein $p_0 \in K$ gibt, sodass

\begin{align*}
  \Int[-1][1]{|f(t) - p_0(t)|^2}{t}
  \leq
  \Int[-1][1]{|f(t) - p(t)|^2}{t}
\end{align*}

für alle $p \in K$. \\

\textit{Hinweis:}
Die Menge aller komplexen Polynome vom Grad $\leq n$ ist ein endlich dimensionaler Unterraum und $p \mapsto p(t)$, sowie $p \mapsto p^\primeprime(t)$ sind lineare Funktionale auf diesem Unterraum für jedes $t \in [a, b]$.
Schließe, dass $K$ abgeschlossen und konvex ist.

\end{exercise}

\begin{solution}

Wir befinden uns im Hilbertraum $L^2([-1, 1])$ mit

\begin{align*}
  (f, g) := \Int[-1][1]{f \overline{g}}{\lambda}.
\end{align*}

Weiters definieren wir

\begin{align*}
  P_n := \Bbraces
  {
    p: [-1, 1] \to \C: x
    \mapsto \sum_{i=0}^n a_i x^i:
    a_1, \ldots, a_n \in \C
  }
\end{align*}

$P_n$ ist ein $(n+1)$-dimensionaler Unterraum von $L^2([-1, 1])$, weil für $f, g \in P_n$

\begin{align*}
  \grad{(f + g)}
  \leq
  \max \Bbraces{\grad{f}, \grad{g}}
  \leq n,
\end{align*}

und für $\lambda \in \C$, $f + \lambda g \in P_n$ gilt.

\includegraphicsboxed{Satz 2.2.1}

$P_n$ ist als endlich-dimensionaler Unterraum abgeschlossen (Satz 2.2.1).

\begin{align*}
  \alpha_t:
  P_n \to \C:
  p  \mapsto p(t),
  \enspace
  t \in [-1, 1]
\end{align*}

Wir zeigen nun, dass $\alpha_t$ ein lineares Funktional ist.

\begin{itemize}

  \item
  \enquote{Linearität}: \\

  Evaluierungsfunktionen sind linear.

  \item
  \enquote{Stetigkeit}: \\

  Da $\dim(P_n) = n + 1 < \infty$ sind alle Normen äquivalent.

  \begin{align*}
    \norm{\alpha_t}
    =
    \sup \Bbraces
    {
      |\alpha_t(p)|:
      \norm[\infty]{p} = 1
    }
    =
    \sup \Bbraces
    {
      |p(t)|:
      \norm[\infty]{p} = 1
    }
    = 1 < \infty
  \end{align*}

  Also ist $\alpha_t$ beschränkt, und damit stetig.

\end{itemize}

\begin{align*}
  \beta_t :
  P_n \to \C:
  p \mapsto p^\primeprime(t),
  \enspace
  t \in [-1, 1]
\end{align*}

Die Funktionen sind wieder lineare Funktionale.

\begin{itemize}

  \item
  \enquote{Linearität}: \\

  Evaluierungsfunktionen und der Ableitungsoperator sind linear.

  \item
  \enquote{Stetigkeit}: \\

  \begin{align*}
    \Forall t \in [-1, 1]:
    p^\primeprime(t)
    =
    \lim_{h \to 0^+} \frac{p(t+h) - 2 p(t) + p(t-h)}{h^2}
  \end{align*}

  Sei nun $(h_k)_{k \in \N}$ eine Nullfolge aus $\R^+$ und setze für $k \in \N$:

  \begin{align*}
    \gamma_{t, k}
    :=
    \frac
    {
      \alpha_{t + h_k} -
      2 \alpha_t +
      \alpha_{t - h_k}
    }
    {h_k^2}
  \end{align*}

  Das ist ein stetiges, lineares Funktional.
  Für alle $p \in P_n$ existiert der Grenzwert (als Teilfolge eines Netzes) $\lim_{k \to \infty} \gamma_{t, k}(p) = p^\primeprime(t)$.

  \includegraphicsboxed{Korollar 4.2.3}

  $P_n$ ist ein Banachraum, also ist nach Korollar 4.2.3

  \begin{align*}
    \beta_t:
    P_n \to \C:
    p \mapsto \lim_{k \to \infty} \gamma_{t, k}(p) = p^\primeprime(t)
  \end{align*}

  ein stetiges lineares Funktional.

\end{itemize}

$\R, \R^+ \cup \Bbraces{0} \subset \C$ sind abgeschlossen und konvex.
Also sind $\Forall t \in [-1, 1]$:

\begin{align*}
  \alpha_t^{-1}(\R)
  =
  \Bbraces{p \in P_n: p(t) \in \R},
  \enspace
  \beta_t^{-1}(\R^+ \cup \Bbraces{0})
  =
  \Bbraces{p \in P_n: p^\primeprime(t) \geq 0}
\end{align*}

abgeschlossen und konvex.
Also ist auch

\begin{align*}
  L := \bigcap_{t \in [-1, 1]}\alpha_t^{-1}(\R)
\end{align*}

abgeschlossen und konvex.
Sei nun $(p: x \mapsto \sum_{i=1}^n a_i x^i) \in L$ beliebig.
Dann gilt

\begin{align*}
  \Forall t \in [-1, 1]: p(t) \in \R.
\end{align*}

Angenommen, es gäbe einen imaginären Koeffizienten.
Dann kann man definieren ...

\begin{align*}
  k := \min \Bbraces{i \in \N: \Im(a_i) \neq 0},
  \enspace
  b := \max \Bbraces{|\Im(a_i)|: i = k + 1, \dots, n}.
\end{align*}

Wir wählen jetzt ein $t \in ]0, 1]$ mit

\begin{align*}
  t^{k+1} b < \frac{|\Im(a_k)|}{n} t^k.
\end{align*}

Also gilt

\begin{align*}
  t < \frac{|\Im(a_k)|}{nb}
  \implies
  t^k \pbraces{tb - \frac{|\Im(a_k)|}{n}} < 0.
\end{align*}

Dann ist insbesondere $\Forall l \geq k + 1:$

\begin{align*}
  t^lb
  \leq
  t^{k+1}b
  <
  \frac{|\Im(a_k)|}{n}t^k.
\end{align*}

Weiters gilt

\begin{align*}
  \vbraces{\sum_{i = k + 1}^n \Im(a_i)t^i}
  \leq
  \sum_{i = k + 1}^n \vbraces{\Im(a_i)} t^i
  \leq
  \sum_{i = k+1}^n b t^i
  <
  \vbraces{\Im(a_k)} t^k.
\end{align*}

Also folgt

\begin{align*}
  \vbraces{\Im(p(t))}
  =
  \vbraces{\sum_{i = k}^n \Im(a_i)t^i}
  \geq
  \vbraces
  {
    \vbraces{\Im(a_k)} t^k -
    \vbraces{\sum_{i = k+1}^n \Im(a_i)t^i}
  } > 0.
\end{align*}

Dies ist aber ein Widerspruch zu $p(t) \in \R$!
Also ist

\begin{align*}
  L =
  \R^n[z] =
  \Bbraces
  {
    p: [-1, 1] \to \R:
    x \mapsto \sum_{i = 0}^n a_ix^i:
    a_1, \ldots, a_n \in \R
  }.
\end{align*}

Außerdem ist

\begin{align*}
  M =
  \bigcap_{t \in [-1, 1]} \beta_t^{-1}(\R^+ \cup \Bbraces{0})
\end{align*}

konvex und abgeschlossen.
Also auch $L \cap M = K \neq \emptyset$.

\includegraphicsboxed{Satz 3.2.3}

Schließlich folgt die gewünschte Aussage mit Satz 3.2.3 (i).
Der besagt, dass zu jedem $f \in L^2([-1, 1])$ ein eindeutiges Element $p_K(x) \in K$ gibt, sodass

\begin{align*}
  \norm{x - p_K(x)}
  =
  \inf \Bbraces{\norm{y - x}: y \in K}.
\end{align*}

\end{solution}
