\begin{exercise}

Sei $X$ ein Banachraum. Dann sind äquivalent:

\begin{enumerate}[label = \roman*)]

  \item
  $X$ ist reflexiv.

  \item
  $X^\prime$ ist reflexiv.

  \item
  Die abgeschlossene Einheitskugel $B_X$ von $X$ ist $w$-kompakt, also kompakt bezüglich $\sigma(X,X^\prime)$.

\end{enumerate}

\textit{Hinweis:}
Zeige mit Hilfe des Satzes von Banach-Alaoglu und der Aufgabe 06/2, dass aus $X$ reflexiv bereits $B_X, B_{X^\prime}$ $w$-kompakt folgt.
Mit dem Satz von Goldstine zeige, dass aus $B_X$ $w$-kompakt $\iota(B_X) = B_X$ folgt.
Für (ii) $\implies$ (i) erinnere man sich zusätzlich daran, dass der schwache Abschluss bei konvexen Mengen gleich dem Norm-Abschluss ist.

\end{exercise}

\begin{solution}

\leavevmode \\

\includegraphicsboxed{Definition 5.5.4}

\begin{itemize}

  \item
  \Quote{(i) $\implies$ (iii)}: \\

  Sei $X$ reflexiv.
  Nach dem Satz von Banach-Alaoglu angewandt auf den normierten Raum $(X^\prime,\norm{\cdot})$ ist $B_{X^\primeprime}$ kompakt bezüglich $\sigma(X^\primeprime, \iota_1(X^\prime))$.
  Nach Aufgabe 6/2 ist $\iota: (X,\sigma(X,X^\prime)) \to (\iota(X), \sigma(X^\primeprime,\iota_1(X^\prime))|_{\iota(X)})$ ein Homöomorphismus und wegen der Reflexivität gilt $\iota(X) = X^\primeprime$.

  \includegraphicsboxed{Korollar 5.2.4}

  Mit Korollar 5.2.4 gilt

  \begin{align*}
    \iota^{-1}(B_{X^\primeprime})
    & =
    \Bbraces
    {
      x \in X:
      \norm[X^\primeprime]{\iota(x)} \leq 1
    }
    =
    \Bbraces
    {
      x \in X:
      \sup \Bbraces
      {
        |f(x)|:
        f \in X^\prime,
        \norm{f} = 1
      }
      \leq 1
    } \\
    & =
    \Bbraces
    {
      x \in X:
      \norm{x} \leq 1
    }
    = B_X
  \end{align*}

  Damit ist $B_X$ als Bild einer kompakten Menge $B_{X^\primeprime}$ unter einer stetigen Abbildung wieder kompakt.

  \item
  \Quote{(iii) $\implies$ (i)}: \\

  Sei also $B_X$ kompakt bezüglich $\sigma(X, X^\prime)$.

  \includegraphicsboxed{Satz 5.5.5}

  Nach dem Satz von Goldstine ist

  \begin{align*}
    \overline
    {
      \iota(B_X)}^{\sigma(X^\primeprime, \iota_1(X))
    }
    =
    B_{X^\primeprime}
  \end{align*}

  und $\iota$ ist ein Homöomorphismus.
  Also ist $\iota(B_X)$ bezüglich $\sigma(X^\primeprime, \iota_1(X^\prime))$ kompakt.

  \includegraphicsboxed{Kaltenbäck Lemma 12.11.7}

  Da $(X^\primeprime, \sigma(X^\primeprime, \iota_1(X^\prime)))$ Hausdorff ist, ist $\iota(B_X)$ insbesondere auch abgeschlossen (Kaltenbäck Lemma 12.11.7).
  Also ist $\iota(B_X) = \overline{\iota(B_X)} = B_{X^\primeprime}$.
  Außerdem ist $\iota$ nach Lemma 5.5.2 und Bemerkung 5.5.3 injektiv und linear.
  Sei nun $y \in X^\primeprime$ mit $y \neq 0$ und betrachte $z := \frac{y}{\norm[X^\primeprime]{y}} \in X^\primeprime$, $\norm[X^\primeprime]{z} = 1$.
  Es gilt $z \in B_{X^\primeprime}$, also folgt

  \begin{align*}
    \Exists x \in B_X:
    \iota(x) = z
  \end{align*}

  Für $u := \norm[X^\primeprime]{y} x \in X$ gilt dann

  \begin{align*}
    \iota(u)
    =
    \norm[X^\primeprime]{y} \iota(x)
    =
    \norm[X^\primeprime]{y} z
    =
    \norm[X^\primeprime]{y} \frac{y}{\norm[X^\primeprime]{y}}
    = y
  \end{align*}

  und es gilt $\iota(X) = X^\primeprime$, also ist $X$ reflexiv.

  \item
  \Quote{(i) $\implies$ (ii)}: \\

  Sei $X$ reflexiv.
  Nach dem Satz von Banach-Alaoglu ist $B_{X^\prime}$ abgeschlossen bezüglich $\sigma(X^\prime, \iota(X)) = \sigma(X^\prime, X^\primeprime)$.
  Nun erhalten wir mit \Quote{(iii) $\implies$ (i)}, dass $X^\prime$ reflexiv ist.

  \item
  \Quote{(ii) $\implies$ (i)}: \\

  Sei $X^\prime$ reflexiv.
  Dann ist mit dem Satz von Banach-Alaoglu $B_{X^\primeprime}$ bezüglich $\sigma(X^\primeprime, \iota_1(X^\prime)) = \sigma(X^\primeprime, X^\primeprimeprime)$ kompakt.

  \includegraphicsboxed{Satz 5.3.8}

  Mit dem Satz von Goldstine und der Konvexität von $\iota(B_X)$ mit Satz 5.3.8

  \begin{align*}
    \overline{\iota(B_X)}^{\norm[X^\primeprime]{\cdot}}
    =
    \overline{\iota(B_X)}^{\sigma(X^\primeprime, X^\primeprimeprime)}
    =
    B_{X^\primeprime}.
  \end{align*}

  \includegraphicsboxed{Lemma 5.5.2}
  \includegraphicsboxed{Bemerkung 5.5.3}

  Da $B_X$ abgeschlossen bezüglich $\norm{\cdot}$ ist und

  \begin{align*}
    \iota^{-1}:
    (\iota(X), \norm[X^\primeprime]{\cdot}|_{\iota(X)})
    \to
    (X, \norm{\cdot})
  \end{align*}

  laut Lemma 5.5.2 und Bemerkung 5.5.3 stetig und surjektiv ist, ist $(\iota^{-1})^{-1}(B_X)$ abgeschlossen bezüglich $\norm[X^\primeprime]{\cdot}$.
  Also ist

  \begin{align*}
    \iota(B_X)
    =
    \overline{\iota(B_X)}^{\norm[X^\primeprime]{\cdot}}
    =
    B_{X^\primeprime}.
  \end{align*}

  Wie in der Implikation \Quote{(iii) $\implies$ (i)} folgt dann $\iota(X) = X^\primeprime$.

\end{itemize}

\end{solution}
