\begin{exercise}

Ein normierter Raum $Y$ heißt strikt konvex, wenn gilt

\begin{align*}
  x, y \in Y,
  \norm{x} = \norm{y} = 1,
  \norm{\frac{x+y}{2}} = 1
  \implies
  x = y
\end{align*}

Sei nun $X$ ein normierter Raum, $M$ ein linearer Teilraum von $X$, und $f: M \to \C$ ein beschränktes lineares Funktional.
Zeige:
Ist $X^\prime$ strikt konvex, so hat $f$ genau eine normerhaltende Fortsetzung.

\end{exercise}

\begin{solution}

Laut Korollar 5.2.4, $\Exists F \in X^\prime:$

\begin{align*}
  F|_M = f,
  \quad
  \norm[X^\prime]{F} = \norm[M^\prime]{f}.
\end{align*}

Seien $F_1$, $F_2$ normerhaltende Fortsetzungen von $f$, d.h.

\begin{align*}
  F_1|_M = F_2|_M = f,
  \enspace
  \norm[X\prime]{F_1} = \norm[X^\prime]{F_2} = \norm[M^\prime]{f}.
\end{align*}

\begin{enumerate}[label = Fall \arabic*:]

  \item
  Wenn $f = 0$, dann muss auch $\norm[X\prime]{F_1} = \norm[X^\prime]{F_2} = \norm[M^\prime]{f} = 0$ und somit $F_1, F_2 = 0$.

  \item
  Wenn $f \neq 0$, dann gilt $\Forall i = 1, 2:$

  \begin{align*}
    \norm[X^\prime]{\frac{F_i}{\norm[M^\prime]{f}}}
    =
    \frac{\norm[X^\prime]{F_i}}{\norm[M^\prime]{f}} = 1.
  \end{align*}

  Aus der Dreiecksungleichung, folgt

  \begin{align*}
    \norm[X^\prime]
    {
      \Frac{2}
      {
        \frac{F_1}{\norm[M^\prime]{f}} +
        \frac{F_2}{\norm[M^\prime]{f}}
      }
    }
    =
    \frac{1}{2 \norm[M^\prime]{f}} \norm[X^\prime]{F_1 + F_2}
    \leq
    \Frac{2 \norm[M^\prime]{f}}
    {
      \norm[X^\prime]{F_1} +
      \norm[X^\prime]{F_2}
    }
    = 1.
  \end{align*}

  Die andere Richtung, dieser Abschätzung, erhalten wir mit

  \begin{align*}
    \norm[X^\prime]{F_1 + F_2}
    & =
    \sup \Bbraces{(F_1 + F_2)(x): x \in X, \norm{x} = 1} \\
    & \geq
    \sup \Bbraces{(F_1 + F_2)(x): x \in M, \norm{x} = 1} \\
    & =
    \sup \Bbraces{2 f(x): x \in M, \norm{x} = 1}
    =
    2 \norm[M^\prime]{f}.
  \end{align*}

  Aus der strikten Konvexität von $X^\prime$, folgt somit

  \begin{align*}
    \frac{F_1}{\norm[M^\prime]{f}}
    =
    \frac{F_2}{\norm[M^\prime]{f}}
    \implies
    F_1 = F_2.
  \end{align*}

\end{enumerate}

\end{solution}
