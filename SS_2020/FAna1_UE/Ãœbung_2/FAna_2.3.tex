\begin{exercise}

Betrachte den Raum $L^2(-1, 1)$ (der $L^2$-Raum bezüglich dem Lebesgue Maß auf $(-1, 1)$), und die beiden konvexen Teilmengen

\begin{align*}
  A :=
  \Bbraces{
    f \in L^2(-1, 1):
    f \text{ stetig}, f(0) = 0
  },
  \quad
  B :=
  \Bbraces{
    f \in L^2(-1, 1):
    f \text{ stetig}, f(0) = 1
  }.
\end{align*}

Existiert $\phi \in L^2(-1, 1)^\prime$ mit $\Forall a \in A, b \in B. \enspace \text{Re } \phi(a) < \text{Re } \phi(b)$?
Falls ja, finde ein solches Funktional.
Falls nein, zeige dass es keines gibt.

\end{exercise}

\begin{solution}

Zuerst, entschlüsseln wir die Angabe.

\begin{align*}
  A & :=
  \Bbraces
  {
    f \in L^2(-1, 1):
    \Exists f_\Text{rep} \in f:
    f_\Text{rep} \text{ stetig},
    f_\Text{rep}(0) = 0
  }, \\
  \quad
  B & :=
  \Bbraces
  {
    f \in L^2(-1, 1):
    \Exists f_\Text{rep} \in f:
    f_\Text{rep} \text{ stetig},
    f_\Text{rep}(0) = 1
  }
\end{align*}

Jetzt bemerken wir noch, dass $A$ ein linearer (und $B$ ein affiner) Teilraum von $L^2(-1, 1)$ ist.
Weiters, gelten $B = A + 1$ und $A = B - B$. \\

Nun zum \Quote{wahren} Teil der Aufgabe:
Die Antwort lautet \Quote{nein}! \\

Um das einzusehen, bemerken wir zuerst, dass für jedes $\phi \in L^2(-1, 1)^\prime$, ebenso $\phi(A)$ ein linearer Teilraum, jetzt von
$\C$, ist.
Dafür gibt es nur zwei Fälle.

\begin{enumerate}[label = Fall \arabic*:]

  \item
  Wenn $\phi(A) = \C$, dann kann die Bedingung $\Forall a \in A, b \in B: \text{Re } \phi(a) < \text{Re } \phi(b)$
  offensichtlich nicht erfüllt werden.

  \item
  Sei $\phi(A) = \Bbraces{0}$.
  Dann muss aber $\phi|_B$ konstant gleich einem $c \in \C$ sein, weil $\Forall b_1, b_2 \in B:$

  \begin{align*}
    \phi(b_1) - \phi(b_2)
    =
    \phi(\underbrace{b_1 - b_2}_{\in A})
    =
    0.
  \end{align*}

  Betrachte für beliebiges $\epsilon \in (0, 1]$ die Funktion $b_\epsilon \in B$.

  \begin{align*}
    b_\epsilon:
    \begin{cases}
      (-1, 1) \to \C \\
      x \mapsto
      \1_{(-\epsilon, \epsilon)}(x)
      \sqrt{1 - \frac{|x|}{\epsilon}}
    \end{cases}
  \end{align*}

  Wir berechnen deren Norm.

  \begin{align*}
    \norm[L^2(-1, 1)]{b_\epsilon}^2
    =
    \Int
    [-1][1]
    {
      \vbraces
      {
        \1_{(-\epsilon, \epsilon)}(x)
        \sqrt{1 - \frac{|x|}{\epsilon}}
      }^2
    }
    {x}
    =
    \Int
    [-\epsilon][\epsilon]
    {1 - \frac{|x|}{\epsilon}}
    {x}
    =
    \Int
    [-\epsilon][\epsilon]
    {}{x}
    - \frac{2}{\epsilon}
    \Int
    [0][\epsilon]
    {x}
    {x}
    =
    2 \pbraces
    {
      \epsilon -
      \frac{1}{\epsilon} \frac{\epsilon^2}{2}
    }
    =
    \epsilon
  \end{align*}

  Weil $\phi$ linear ist, muss $\phi(0) = 0$.
  Weil $\phi$ stetig ist, betrachten wir folgenden Grenzwert.

  \begin{align*}
    c
    =
    \phi(b_\epsilon)
    \xrightarrow{\epsilon \to 0}
    0
  \end{align*}

  Damit, muss $c = 0$ und schließlich $\Forall a \in A, b \in B: \phi(a) = \phi(b) = 0$,
  also kann die Bedingung $\Forall a \in A, b \in B: \text{Re } \phi(a) < \text{Re } \phi(b)$
  von keinem $\phi \in L^2(-1, 1)^\prime$ erfüllt werden.

\end{enumerate}

\end{solution}
