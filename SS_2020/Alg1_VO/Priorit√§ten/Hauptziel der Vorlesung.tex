\section{Hauptziel der Vorlesung}

\begin{itemize}

  \item
  [3.1.4]
  Quotienten- bzw. Differenzenmonoid

  \item
  [3.2.5]
  Permutationsgruppen

  \item
  [3.3.5]
  Quotientenkörper

  \item
  [3.4.4]
  Zerlegung von Torsionsgruppen in ihre $p$-Anteile

  \item
  [3.4.5]
  Endliche abelsche Gruppen

  \item
  [4.1.2]
  Bekannte Beispiele und Definition einer freien Algebra

  \item
  [4.1.3]
  Die freie Algebra als homomorphes Bild der Termalgebra

  \item
  [5.2.1]
  Faktorielle Ringe

  \item
  [5.2.2]
  Hauptidealringe

  \item
  [5.2.3]
  Euklidische Ringe

  \item
  [5.3.2]
  Polynomringe über faktoriellen Ringen sind faktoriell

  \item
  [6.2.1]
  Adjunktion einer Nullstelle

  \item
  [6.2.2]
  Die Konstruktion von Zerfällungskörper und algebraischem Abschluss

  \item
  [6.2.3]
  Die Eindeutigkeit von Zerfällungskörpern und algebraischem Abschluss

  \item
  [6.3.1]
  Klassifikation endlicher Körper

  \item
  [6.3.4]
  Konstruktion endlicher Köper

\end{itemize}
