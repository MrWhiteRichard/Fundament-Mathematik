\section{Nebenziel der Vorlesung}

\begin{itemize}

  \item
  [3.3.7]
  Der Chinesische Restsatz

  \item
  [3.3.8]
  Beispiele nichtkommutativer Ringe

  \item
  [3.4.1]
  Unter- und Faktormoduln, Homomorphismen und direkte Summen

  \item
  [3.4.2]
  Schwache Produkte - direkte Summen

  \item
  [3.4.3]
  Abelsche Gruppen als Moduln über $\Z$ und $\Z_m$

  \item
  [3.4.6]
  Abelsche Gruppen als Moduln über ihrem Endomorphismenring

  \item
  [3.5.3]
  Angeordnete Körper und nochmals $\R$

  \item
  [3.6.9]
  Der Darstellungssatz von Stone

  \item
  [4.1.4]
  Die freie Gruppe

  \item
  [4.1.6]
  Die freie Algebra als subdirektes Produkt

  \item
  [4.1.7]
  Der Satz von Birkhof

  \item
  [4.2.2]
  Konstruktion des Koproduktes als freie Algebra

  \item
  [4.2.3]
  Polynomalgebren

  \item
  [5.3.1]
  Der Quotientenkörper eines faktoriellen Rings

  \item
  [5.3.5]
  Gebrochen rationale Funktionen und ihre Partialbruchzerlegung

  \item
  [5.3.6]
  Interpolation nach Lagrange und nach Newton

  \item
  [6.1.5]
  Transzendente Körpererweiterungen

  \item
  [6.1.6]
  Anwendung: Konstruierbarkeit mit Zirkel und Lineal

  \item
  [6.2.5]
  Einheitswurzeln und Kreisteilungspolynome

  \item
  [6.3.2]
  Die Unterkörper eines endlichen Körpers

  \item
  [6.3.3]
  Irreduzible Polynome über endlichen Primkörpern

  \item
  [6.3.5]
  Der algebraische Abschluss eines endlichen Körpers

\end{itemize}
